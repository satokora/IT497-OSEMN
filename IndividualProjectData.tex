\documentclass{article}\usepackage[]{graphicx}\usepackage[]{color}
%% maxwidth is the original width if it is less than linewidth
%% otherwise use linewidth (to make sure the graphics do not exceed the margin)
\makeatletter
\def\maxwidth{ %
  \ifdim\Gin@nat@width>\linewidth
    \linewidth
  \else
    \Gin@nat@width
  \fi
}
\makeatother

\definecolor{fgcolor}{rgb}{0.345, 0.345, 0.345}
\newcommand{\hlnum}[1]{\textcolor[rgb]{0.686,0.059,0.569}{#1}}%
\newcommand{\hlstr}[1]{\textcolor[rgb]{0.192,0.494,0.8}{#1}}%
\newcommand{\hlcom}[1]{\textcolor[rgb]{0.678,0.584,0.686}{\textit{#1}}}%
\newcommand{\hlopt}[1]{\textcolor[rgb]{0,0,0}{#1}}%
\newcommand{\hlstd}[1]{\textcolor[rgb]{0.345,0.345,0.345}{#1}}%
\newcommand{\hlkwa}[1]{\textcolor[rgb]{0.161,0.373,0.58}{\textbf{#1}}}%
\newcommand{\hlkwb}[1]{\textcolor[rgb]{0.69,0.353,0.396}{#1}}%
\newcommand{\hlkwc}[1]{\textcolor[rgb]{0.333,0.667,0.333}{#1}}%
\newcommand{\hlkwd}[1]{\textcolor[rgb]{0.737,0.353,0.396}{\textbf{#1}}}%

\usepackage{framed}
\makeatletter
\newenvironment{kframe}{%
 \def\at@end@of@kframe{}%
 \ifinner\ifhmode%
  \def\at@end@of@kframe{\end{minipage}}%
  \begin{minipage}{\columnwidth}%
 \fi\fi%
 \def\FrameCommand##1{\hskip\@totalleftmargin \hskip-\fboxsep
 \colorbox{shadecolor}{##1}\hskip-\fboxsep
     % There is no \\@totalrightmargin, so:
     \hskip-\linewidth \hskip-\@totalleftmargin \hskip\columnwidth}%
 \MakeFramed {\advance\hsize-\width
   \@totalleftmargin\z@ \linewidth\hsize
   \@setminipage}}%
 {\par\unskip\endMakeFramed%
 \at@end@of@kframe}
\makeatother

\definecolor{shadecolor}{rgb}{.97, .97, .97}
\definecolor{messagecolor}{rgb}{0, 0, 0}
\definecolor{warningcolor}{rgb}{1, 0, 1}
\definecolor{errorcolor}{rgb}{1, 0, 0}
\newenvironment{knitrout}{}{} % an empty environment to be redefined in TeX

\usepackage{alltt}
\IfFileExists{upquote.sty}{\usepackage{upquote}}{}
\begin{document}
\section*{What were the most popular topics in IEEE publications in 2014?}

\begin{knitrout}
\definecolor{shadecolor}{rgb}{0.969, 0.969, 0.969}\color{fgcolor}\begin{kframe}


{\ttfamily\noindent\itshape\color{messagecolor}{\#\# Loading required package: XML\\\#\# Loading required package: ggplot2}}\begin{verbatim}
## [[1]]
## <root>
##   <totalfound>1195</totalfound>
##   <totalsearched>3953943</totalsearched>
##   <document>
##     <rank>1</rank>
##     <title><![CDATA[Abductive Analysis of Administrative Policies in Rule-Based Access Control]]></title>
##     <authors><![CDATA[Gupta, P.;  Stoller, S.D.;  Zhongyuan Xu]]></authors>
##     <affiliations><![CDATA[Google, Inc., Mountain View, CA, USA]]></affiliations>
##     <controlledterms>
##       <term><![CDATA[DP management]]></term>
##       <term><![CDATA[authorisation]]></term>
##     </controlledterms>
##     <thesaurusterms>
##       <term><![CDATA[Access control]]></term>
##       <term><![CDATA[Algorithm design and analysis]]></term>
##       <term><![CDATA[Grammar]]></term>
##       <term><![CDATA[Hospitals]]></term>
##       <term><![CDATA[Organizations]]></term>
##       <term><![CDATA[Semantics]]></term>
##     </thesaurusterms>
##     <pubtitle><![CDATA[Dependable and Secure Computing, IEEE Transactions on]]></pubtitle>
##     <punumber><![CDATA[8858]]></punumber>
##     <pubtype><![CDATA[Journals & Magazines]]></pubtype>
##     <publisher><![CDATA[IEEE]]></publisher>
##     <volume><![CDATA[11]]></volume>
##     <issue><![CDATA[5]]></issue>
##     <py><![CDATA[2014]]></py>
##     <spage><![CDATA[412]]></spage>
##     <epage><![CDATA[424]]></epage>
##     <abstract><![CDATA[In large organizations, access control policies are managed by multiple users (administrators). An administrative policy specifies how each user in an enterprise may change the policy. Fully understanding the consequences of an administrative policy in an enterprise system can be difficult, because of the scale and complexity of the access control policy and the administrative policy, and because sequences of changes by different users may interact in unexpected ways. Administrative policy analysis helps by answering questions such as user-permission reachability, which asks whether specified users can together change the policy in a way that achieves a specified goal, namely, granting a specified permission to a specified user. This paper presents a rule-based access control policy language, a rule-based administrative policy model that controls addition and removal of facts and rules, and an abductive analysis algorithm for user-permission reachability. Abductive analysis means that the algorithm can analyze policy rules even if the facts initially in the policy (e.g., information about users) are unavailable. The algorithm does this by computing minimal sets of facts that, if present in the initial policy, imply reachability of the goal.]]></abstract>
##     <issn><![CDATA[1545-5971]]></issn>
##     <arnumber><![CDATA[6616529]]></arnumber>
##     <doi><![CDATA[10.1109/TDSC.2013.42]]></doi>
##     <publicationId><![CDATA[6616529]]></publicationId>
##     <mdurl><![CDATA[http://ieeexplore.ieee.org/xpl/articleDetails.jsp?tp=&arnumber=6616529&contentType=Journals+%26+Magazines]]></mdurl>
##     <pdf><![CDATA[http://ieeexplore.ieee.org/stamp/stamp.jsp?arnumber=6616529]]></pdf>
##   </document>
##   <document>
##     <rank>2</rank>
##     <title><![CDATA[Single-Chip Parametric Frequency Up/Down Converter Using Parallel PPLN Waveguides]]></title>
##     <authors><![CDATA[Kazama, T.;  Umeki, T.;  Asobe, M.;  Takenouchi, H.]]></authors>
##     <affiliations><![CDATA[NTT Photonics Labs., NTT Corp., Atsugi, Japan]]></affiliations>
##     <controlledterms>
##       <term><![CDATA[demultiplexing]]></term>
##       <term><![CDATA[demultiplexing equipment]]></term>
##       <term><![CDATA[integrated optoelectronics]]></term>
##       <term><![CDATA[light interference]]></term>
##       <term><![CDATA[lithium compounds]]></term>
##       <term><![CDATA[mirrors]]></term>
##       <term><![CDATA[monolithic integrated circuits]]></term>
##       <term><![CDATA[multiplexing equipment]]></term>
##       <term><![CDATA[optical communication equipment]]></term>
##       <term><![CDATA[optical fabrication]]></term>
##       <term><![CDATA[optical harmonic generation]]></term>
##       <term><![CDATA[optical pumping]]></term>
##       <term><![CDATA[optical waveguides]]></term>
##       <term><![CDATA[ridge waveguides]]></term>
##       <term><![CDATA[wavelength division multiplexing]]></term>
##     </controlledterms>
##     <thesaurusterms>
##       <term><![CDATA[Mirrors]]></term>
##       <term><![CDATA[Mixers]]></term>
##       <term><![CDATA[Optical harmonic generation]]></term>
##       <term><![CDATA[Optical pumping]]></term>
##       <term><![CDATA[Optical waveguides]]></term>
##       <term><![CDATA[Wavelength division multiplexing]]></term>
##     </thesaurusterms>
##     <pubtitle><![CDATA[Photonics Technology Letters, IEEE]]></pubtitle>
##     <punumber><![CDATA[68]]></punumber>
##     <pubtype><![CDATA[Journals & Magazines]]></pubtype>
##     <publisher><![CDATA[IEEE]]></publisher>
##     <volume><![CDATA[26]]></volume>
##     <issue><![CDATA[22]]></issue>
##     <py><![CDATA[2014]]></py>
##     <spage><![CDATA[2248]]></spage>
##     <epage><![CDATA[2251]]></epage>
##     <abstract><![CDATA[We propose a novel structure for a monolithically integrated parametric frequency up and down converter using two parallel periodically poled lithium niobate (PPLN) waveguides combined with a reflective wavelength-division multiplexer (WDM) to realize second-harmonic-pumped parametric mixing on one chip. The parallel configuration of the PPLN waveguides provides a long interaction length for efficient conversion. The reflective WDM used for (de)multiplexing 0.78- and 1.56-&#x03BC;m light consisted of two multimode interference (MMI) (de)multiplexers and a dichroic mirror. The utilization of the wavelength selectivity of both the MMI (de)multiplexers and the dichroic mirror enabled us to achieve high isolation of the fundamental light for generating a second-harmonic pump. We fabricated the integrated device using a direct-bonded LiNbO3 ridge waveguide. Frequency conversion achieved with two-stage second-harmonic generation and difference frequency generation processes was successfully demonstrated with a high isolation of the fundamental and a parametric gain of the signal in the integrated device.]]></abstract>
##     <issn><![CDATA[1041-1135]]></issn>
##     <htmlFlag><![CDATA[1]]></htmlFlag>
##     <arnumber><![CDATA[6894121]]></arnumber>
##     <doi><![CDATA[10.1109/LPT.2014.2353033]]></doi>
##     <publicationId><![CDATA[6894121]]></publicationId>
##     <mdurl><![CDATA[http://ieeexplore.ieee.org/xpl/articleDetails.jsp?tp=&arnumber=6894121&contentType=Journals+%26+Magazines]]></mdurl>
##     <pdf><![CDATA[http://ieeexplore.ieee.org/stamp/stamp.jsp?arnumber=6894121]]></pdf>
##   </document>
##   <document>
##     <rank>3</rank>
##     <title><![CDATA[Nanosecond Imaging of Shock- and Jet-Like Features]]></title>
##     <authors><![CDATA[Tubman, E.R.;  Crowston, R.;  Alraddadi, R.;  Doyle, H.W.;  Meinecke, J.;  Cross, J.E.;  Bolis, R.;  Lamb, D.;  Tzeferacos, P.;  Doria, D.;  Reville, B.;  Ahmed, H.;  Borghesi, M.;  Gregori, G.;  Woolsey, N.C.]]></authors>
##     <affiliations><![CDATA[Dept. of Phys., Univ. of York, York, UK]]></affiliations>
##     <controlledterms>
##       <term><![CDATA[argon]]></term>
##       <term><![CDATA[plasma diagnostics]]></term>
##       <term><![CDATA[plasma jets]]></term>
##       <term><![CDATA[plasma production by laser]]></term>
##       <term><![CDATA[plasma shock waves]]></term>
##     </controlledterms>
##     <thesaurusterms>
##       <term><![CDATA[Cameras]]></term>
##       <term><![CDATA[Educational institutions]]></term>
##       <term><![CDATA[Laser beams]]></term>
##       <term><![CDATA[Particle beams]]></term>
##       <term><![CDATA[Plasmas]]></term>
##       <term><![CDATA[Probes]]></term>
##       <term><![CDATA[Shock waves]]></term>
##     </thesaurusterms>
##     <pubtitle><![CDATA[Plasma Science, IEEE Transactions on]]></pubtitle>
##     <punumber><![CDATA[27]]></punumber>
##     <pubtype><![CDATA[Journals & Magazines]]></pubtype>
##     <publisher><![CDATA[IEEE]]></publisher>
##     <volume><![CDATA[42]]></volume>
##     <issue><![CDATA[10]]></issue>
##     <part><![CDATA[1]]></part>
##     <py><![CDATA[2014]]></py>
##     <spage><![CDATA[2496]]></spage>
##     <epage><![CDATA[2497]]></epage>
##     <abstract><![CDATA[The production of shock- and collimated jet-like features is recorded from the self-emission of a plasma using a 16- frame camera, which can show the progression of the interaction over short (100s ns) durations. A cluster of laser beams, with intensity 10<sup>15</sup> W/cm<sup>2</sup>, was focused onto a planar aluminum foil to produce a plasma that expanded into 0.7 mbar of argon gas. The acquisition of 16 ultrafast images on a single shot allows prompt spatial and temporal characterization of the plasma and enables the velocity of the jet- and shock-like features to be calculated.]]></abstract>
##     <issn><![CDATA[0093-3813]]></issn>
##     <htmlFlag><![CDATA[1]]></htmlFlag>
##     <arnumber><![CDATA[6850042]]></arnumber>
##     <doi><![CDATA[10.1109/TPS.2014.2331995]]></doi>
##     <publicationId><![CDATA[6850042]]></publicationId>
##     <mdurl><![CDATA[http://ieeexplore.ieee.org/xpl/articleDetails.jsp?tp=&arnumber=6850042&contentType=Journals+%26+Magazines]]></mdurl>
##     <pdf><![CDATA[http://ieeexplore.ieee.org/stamp/stamp.jsp?arnumber=6850042]]></pdf>
##   </document>
##   <document>
##     <rank>4</rank>
##     <title><![CDATA[Implementation of the Natural Mode Analysis for Nanotopologies Using a Volumetric Method of Moments (V-MoM) Algorithm]]></title>
##     <authors><![CDATA[Xuezhi Zheng;  Valev, V.K.;  Verellen, N.;  Volskiy, V.;  Herrmann, L.O.;  Van Dorpe, P.;  Baumberg, J.J.;  Vandenbosch, G.A.E.;  Moschchalkov, V.V.]]></authors>
##     <affiliations><![CDATA[Dept. of Electr. Eng. (ESAT-TELEMIC), KU Leuven, Leuven, Belgium]]></affiliations>
##     <controlledterms>
##       <term><![CDATA[method of moments]]></term>
##       <term><![CDATA[nanophotonics]]></term>
##       <term><![CDATA[nanostructured materials]]></term>
##     </controlledterms>
##     <thesaurusterms>
##       <term><![CDATA[Algorithm design and analysis]]></term>
##       <term><![CDATA[Impedance]]></term>
##       <term><![CDATA[Matrix decomposition]]></term>
##       <term><![CDATA[Method of moments]]></term>
##       <term><![CDATA[Plasmons]]></term>
##       <term><![CDATA[Vectors]]></term>
##     </thesaurusterms>
##     <pubtitle><![CDATA[Photonics Journal, IEEE]]></pubtitle>
##     <punumber><![CDATA[4563994]]></punumber>
##     <pubtype><![CDATA[Journals & Magazines]]></pubtype>
##     <publisher><![CDATA[IEEE]]></publisher>
##     <volume><![CDATA[6]]></volume>
##     <issue><![CDATA[4]]></issue>
##     <py><![CDATA[2014]]></py>
##     <spage><![CDATA[1]]></spage>
##     <epage><![CDATA[13]]></epage>
##     <abstract><![CDATA[Within the framework of a method-of-moments algorithm, the natural frequencies and natural modes of a scatterer/antenna can be found by looking for the complex frequencies where the determinant of the impedance matrix becomes zero. However, for nanotopologies, the value of this determinant can easily go beyond the resolution or representation limits available on present-day computers. In this work, we propose a substitute for the matrix determinant that avoids this problem altogether: the characteristic term. A robust numerical procedure for locating natural frequencies using this new target function is outlined. Then, the natural frequencies of three different nanostructures, namely, a gold and nickel nanopatch and a gold nanodimer, are calculated and validated by numerically and experimentally obtained scattering/extinction spectra.]]></abstract>
##     <issn><![CDATA[1943-0655]]></issn>
##     <htmlFlag><![CDATA[1]]></htmlFlag>
##     <arnumber><![CDATA[6838978]]></arnumber>
##     <doi><![CDATA[10.1109/JPHOT.2014.2331236]]></doi>
##     <publicationId><![CDATA[6838978]]></publicationId>
##     <mdurl><![CDATA[http://ieeexplore.ieee.org/xpl/articleDetails.jsp?tp=&arnumber=6838978&contentType=Journals+%26+Magazines]]></mdurl>
##     <pdf><![CDATA[http://ieeexplore.ieee.org/stamp/stamp.jsp?arnumber=6838978]]></pdf>
##   </document>
##   <document>
##     <rank>5</rank>
##     <title><![CDATA[A Real-Time Motion-Feature-Extraction VLSI Employing Digital-Pixel-Sensor-Based Parallel Architecture]]></title>
##     <authors><![CDATA[Hongbo Zhu;  Shibata, T.]]></authors>
##     <affiliations><![CDATA[VLSI Design & Educ. Center, Univ. of Tokyo, Tokyo, Japan]]></affiliations>
##     <controlledterms>
##       <term><![CDATA[CMOS image sensors]]></term>
##       <term><![CDATA[VLSI]]></term>
##       <term><![CDATA[edge detection]]></term>
##       <term><![CDATA[feature extraction]]></term>
##       <term><![CDATA[image motion analysis]]></term>
##       <term><![CDATA[object detection]]></term>
##     </controlledterms>
##     <thesaurusterms>
##       <term><![CDATA[Feature extraction]]></term>
##       <term><![CDATA[Image edge detection]]></term>
##       <term><![CDATA[Image sensors]]></term>
##       <term><![CDATA[System-on-chip]]></term>
##       <term><![CDATA[Very large scale integration]]></term>
##     </thesaurusterms>
##     <pubtitle><![CDATA[Circuits and Systems for Video Technology, IEEE Transactions on]]></pubtitle>
##     <punumber><![CDATA[76]]></punumber>
##     <pubtype><![CDATA[Journals & Magazines]]></pubtype>
##     <publisher><![CDATA[IEEE]]></publisher>
##     <volume><![CDATA[24]]></volume>
##     <issue><![CDATA[10]]></issue>
##     <py><![CDATA[2014]]></py>
##     <spage><![CDATA[1787]]></spage>
##     <epage><![CDATA[1799]]></epage>
##     <abstract><![CDATA[A very-large-scale integration capable of extracting motion features from moving images in real time has been developed employing row-parallel and pixel-parallel architectures based on the digital pixel sensor technology. Directional edge filtering of input images is carried out in row-parallel processing to minimize the chip real estate. To achieve a real-time response of the system, a fully pixel-parallel architecture has been explored in adaptive binarization of filtered images for essential feature extraction as well as in their temporal integration and derivative operations. As a result, self-speed-adaptive motion feature extraction has been established. The chip was designed and fabricated in a 65-nm CMOS technology and used to build an object detection system. Motion-sensitive target image localization was demonstrated as an illustrative example.]]></abstract>
##     <issn><![CDATA[1051-8215]]></issn>
##     <htmlFlag><![CDATA[1]]></htmlFlag>
##     <arnumber><![CDATA[6778799]]></arnumber>
##     <doi><![CDATA[10.1109/TCSVT.2014.2313899]]></doi>
##     <publicationId><![CDATA[6778799]]></publicationId>
##     <mdurl><![CDATA[http://ieeexplore.ieee.org/xpl/articleDetails.jsp?tp=&arnumber=6778799&contentType=Journals+%26+Magazines]]></mdurl>
##     <pdf><![CDATA[http://ieeexplore.ieee.org/stamp/stamp.jsp?arnumber=6778799]]></pdf>
##   </document>
##   <document>
##     <rank>6</rank>
##     <title><![CDATA[A Low-Cost and High-Wavelength-Precision Fabrication Method for Multiwavelength DFB Semiconductor Laser Array]]></title>
##     <authors><![CDATA[Yuechun Shi;  Rui Liu;  Shengchun Liu;  Xiaojun Zhu]]></authors>
##     <affiliations><![CDATA[Nat. Lab. of Solid State Microstructures, Nanjing Univ., Nanjing, China]]></affiliations>
##     <controlledterms>
##       <term><![CDATA[diffraction gratings]]></term>
##       <term><![CDATA[distributed feedback lasers]]></term>
##       <term><![CDATA[holography]]></term>
##       <term><![CDATA[optical fabrication]]></term>
##       <term><![CDATA[optical waveguides]]></term>
##       <term><![CDATA[semiconductor laser arrays]]></term>
##       <term><![CDATA[waveguide discontinuities]]></term>
##     </controlledterms>
##     <thesaurusterms>
##       <term><![CDATA[Arrayed waveguide gratings]]></term>
##       <term><![CDATA[Arrays]]></term>
##       <term><![CDATA[Fabrication]]></term>
##       <term><![CDATA[Gratings]]></term>
##       <term><![CDATA[Semiconductor laser arrays]]></term>
##       <term><![CDATA[Waveguide lasers]]></term>
##     </thesaurusterms>
##     <pubtitle><![CDATA[Photonics Journal, IEEE]]></pubtitle>
##     <punumber><![CDATA[4563994]]></punumber>
##     <pubtype><![CDATA[Journals & Magazines]]></pubtype>
##     <publisher><![CDATA[IEEE]]></publisher>
##     <volume><![CDATA[6]]></volume>
##     <issue><![CDATA[3]]></issue>
##     <py><![CDATA[2014]]></py>
##     <spage><![CDATA[1]]></spage>
##     <epage><![CDATA[12]]></epage>
##     <abstract><![CDATA[A new method to fabricate the multiwavelength DFB semiconductor laser array (MLA) is proposed for the first time based on sampled grating and s-bent waveguide. According to the detailed precision analysis, the lasing wavelength accuracy of the proposed structure is significantly improved. Because the common holography exposure and micrometer photolithography are used to fabricate the sampled grating and the bent waveguide in this method, the fabrication cost is very low. Therefore, it offers a suitable method for massive fabrication.]]></abstract>
##     <issn><![CDATA[1943-0655]]></issn>
##     <htmlFlag><![CDATA[1]]></htmlFlag>
##     <arnumber><![CDATA[6798654]]></arnumber>
##     <doi><![CDATA[10.1109/JPHOT.2014.2317674]]></doi>
##     <publicationId><![CDATA[6798654]]></publicationId>
##     <mdurl><![CDATA[http://ieeexplore.ieee.org/xpl/articleDetails.jsp?tp=&arnumber=6798654&contentType=Journals+%26+Magazines]]></mdurl>
##     <pdf><![CDATA[http://ieeexplore.ieee.org/stamp/stamp.jsp?arnumber=6798654]]></pdf>
##   </document>
##   <document>
##     <rank>7</rank>
##     <title><![CDATA[Mid-IR Ultrashort Pulsed Fiber-Based Lasers]]></title>
##     <authors><![CDATA[Sorokina, I.T.;  Dvoyrin, V.V.;  Tolstik, N.;  Sorokin, E.]]></authors>
##     <affiliations><![CDATA[Dept. of Phys., Norwegian Univ. of Sci. & Technol., Trondheim, Norway]]></affiliations>
##     <controlledterms>
##       <term><![CDATA[II-VI semiconductors]]></term>
##       <term><![CDATA[chromium]]></term>
##       <term><![CDATA[fibre lasers]]></term>
##       <term><![CDATA[laser mode locking]]></term>
##       <term><![CDATA[laser reliability]]></term>
##       <term><![CDATA[laser tuning]]></term>
##       <term><![CDATA[optical pulse generation]]></term>
##       <term><![CDATA[reviews]]></term>
##       <term><![CDATA[semiconductor lasers]]></term>
##       <term><![CDATA[silicon compounds]]></term>
##       <term><![CDATA[supercontinuum generation]]></term>
##       <term><![CDATA[thulium]]></term>
##       <term><![CDATA[wide band gap semiconductors]]></term>
##       <term><![CDATA[zinc compounds]]></term>
##     </controlledterms>
##     <thesaurusterms>
##       <term><![CDATA[Fiber lasers]]></term>
##       <term><![CDATA[Laser mode locking]]></term>
##       <term><![CDATA[Optical fiber amplifiers]]></term>
##       <term><![CDATA[Power lasers]]></term>
##       <term><![CDATA[Quantum cascade lasers]]></term>
##       <term><![CDATA[Solitons]]></term>
##     </thesaurusterms>
##     <pubtitle><![CDATA[Selected Topics in Quantum Electronics, IEEE Journal of]]></pubtitle>
##     <punumber><![CDATA[2944]]></punumber>
##     <pubtype><![CDATA[Journals & Magazines]]></pubtype>
##     <publisher><![CDATA[IEEE]]></publisher>
##     <volume><![CDATA[20]]></volume>
##     <issue><![CDATA[5]]></issue>
##     <py><![CDATA[2014]]></py>
##     <spage><![CDATA[99]]></spage>
##     <epage><![CDATA[110]]></epage>
##     <abstract><![CDATA[We review the latest breakthroughs in ultrafast fiber laser technology in the mid-IR wavelength range &#x2265;2 &#x03BC;m. In particular, we concentrate on two novel laser systems built around passively mode-locked Tm:fiber lasers and fiber-based Cr:ZnS lasers, generating sub-100 femtosecond pulses and frequency combs with several Watt average output powers, hundreds of kilowatts peak powers and tens of nanojoule pulse energies. The tunability in the broad wavelength range between 2 and 2.5 &#x03BC;m as well as a simple all-silica-fiber design makes the Tm-fiber laser a truly unique broadband light source, particularly relevant for applications in gas sensing, fine material processing of semiconductors, composite materials, glasses and plastics, as well as for brain surgery, breath analysis, remote sensing and stand-off trace gas detection, especially in oil and gas industry. We also review techniques for coherent supercontinuum generation in the mid-IR, including a novel technique of direct generation of the supercontinuum in the fiber. A competing Watt level few-optical cycle Cr:ZnS laser operating at 2.4 &#x03BC;m (Patent pending, ATLA Lasers, Trondheim, Norway) is distinguished by extremely short pulse duration of only 41 fs, reliability and compactness. This unique ultrashort-pulsed laser generates intrinsically coherent frequency combs, which further extends the application range to high-resolution and high-sensitivity spectroscopy and optical clocks.]]></abstract>
##     <issn><![CDATA[1077-260X]]></issn>
##     <htmlFlag><![CDATA[1]]></htmlFlag>
##     <arnumber><![CDATA[6762904]]></arnumber>
##     <doi><![CDATA[10.1109/JSTQE.2014.2310655]]></doi>
##     <publicationId><![CDATA[6762904]]></publicationId>
##     <mdurl><![CDATA[http://ieeexplore.ieee.org/xpl/articleDetails.jsp?tp=&arnumber=6762904&contentType=Journals+%26+Magazines]]></mdurl>
##     <pdf><![CDATA[http://ieeexplore.ieee.org/stamp/stamp.jsp?arnumber=6762904]]></pdf>
##   </document>
##   <document>
##     <rank>8</rank>
##     <title><![CDATA[Precise Simulation of Spectrum for Green Emitting Phosphors Pumped by a Blue LED Die]]></title>
##     <authors><![CDATA[Tsung-Hsun Yang;  Ching-Yi Chen;  Yu-Yu Chang;  Glorieux, B.;  Yi-Neng Peng;  He-Xiang Chen;  Te-Yuan Chung;  Tsung-Xiang Lee;  Ching-Cherng Sun]]></authors>
##     <affiliations><![CDATA[Dept. of Opt. & Photonics, Nat. Central Univ., Chungli, Taiwan]]></affiliations>
##     <controlledterms>
##       <term><![CDATA[light emitting diodes]]></term>
##       <term><![CDATA[optical projectors]]></term>
##       <term><![CDATA[optical pumping]]></term>
##       <term><![CDATA[phosphors]]></term>
##     </controlledterms>
##     <thesaurusterms>
##       <term><![CDATA[Absorption]]></term>
##       <term><![CDATA[Green products]]></term>
##       <term><![CDATA[Light emitting diodes]]></term>
##       <term><![CDATA[Optical device fabrication]]></term>
##       <term><![CDATA[Optical pumping]]></term>
##       <term><![CDATA[Phosphors]]></term>
##       <term><![CDATA[Scattering]]></term>
##     </thesaurusterms>
##     <pubtitle><![CDATA[Photonics Journal, IEEE]]></pubtitle>
##     <punumber><![CDATA[4563994]]></punumber>
##     <pubtype><![CDATA[Journals & Magazines]]></pubtype>
##     <publisher><![CDATA[IEEE]]></publisher>
##     <volume><![CDATA[6]]></volume>
##     <issue><![CDATA[4]]></issue>
##     <py><![CDATA[2014]]></py>
##     <spage><![CDATA[1]]></spage>
##     <epage><![CDATA[10]]></epage>
##     <abstract><![CDATA[This paper proposes an effective approach for spectrum shaping in the optical modeling of phosphor-converted white light-emitting diodes (LEDs) in which overlapping of the absorption and emission spectra of the phosphor are considered. The spectrum shaping method was applied to explore the wavelength-dependent absorption effect and reabsorption by the green phosphor. The experimental results indicated that the adjustment factor for blue light can enable the blue light spectrum to fit well with the measured spectrum. The adjustment factor was linearly related to the transmission ratio of the blue light. The blue light emitted by the blue die and the green light emitted by the phosphor were simulated and predicted in an accurate way as compared with the experimental measurements. The enhanced accuracy in assessing the spectra resulted in enhanced precision of chromaticities. In the experiments, the color differences were one order smaller in CIE 1931 chromaticity (&#x0394;x, &#x0394;y) than in models without spectrum shaping and were almost imperceptible to the human eye. The novel optical modeling of green phosphor pumped by a blue die facilitates the application of these materials in high-color rendering in white LEDs and projection displays.]]></abstract>
##     <issn><![CDATA[1943-0655]]></issn>
##     <htmlFlag><![CDATA[1]]></htmlFlag>
##     <arnumber><![CDATA[6866111]]></arnumber>
##     <doi><![CDATA[10.1109/JPHOT.2014.2341012]]></doi>
##     <publicationId><![CDATA[6866111]]></publicationId>
##     <mdurl><![CDATA[http://ieeexplore.ieee.org/xpl/articleDetails.jsp?tp=&arnumber=6866111&contentType=Journals+%26+Magazines]]></mdurl>
##     <pdf><![CDATA[http://ieeexplore.ieee.org/stamp/stamp.jsp?arnumber=6866111]]></pdf>
##   </document>
##   <document>
##     <rank>9</rank>
##     <title><![CDATA[Aspects of and Insights Into the Rigorous Validation, Verification, and Testing Processes for a Commercial Electromagnetic Field Solver Package]]></title>
##     <authors><![CDATA[Jakobus, U.;  Marchand, R.G.;  Ludick, D.J.]]></authors>
##     <affiliations><![CDATA[EM Software & Syst.-S.A. (Pty) Ltd., Stellenbosch, South Africa]]></affiliations>
##     <controlledterms>
##       <term><![CDATA[computational electromagnetics]]></term>
##       <term><![CDATA[electromagnetic compatibility]]></term>
##       <term><![CDATA[numerical analysis]]></term>
##       <term><![CDATA[program testing]]></term>
##       <term><![CDATA[program verification]]></term>
##     </controlledterms>
##     <thesaurusterms>
##       <term><![CDATA[Benchmark testing]]></term>
##       <term><![CDATA[Computational modeling]]></term>
##       <term><![CDATA[Electromagnetic compatibility]]></term>
##       <term><![CDATA[Electromagnetics]]></term>
##       <term><![CDATA[Method of moments]]></term>
##       <term><![CDATA[Numerical models]]></term>
##     </thesaurusterms>
##     <pubtitle><![CDATA[Electromagnetic Compatibility, IEEE Transactions on]]></pubtitle>
##     <punumber><![CDATA[15]]></punumber>
##     <pubtype><![CDATA[Journals & Magazines]]></pubtype>
##     <publisher><![CDATA[IEEE]]></publisher>
##     <volume><![CDATA[56]]></volume>
##     <issue><![CDATA[4]]></issue>
##     <py><![CDATA[2014]]></py>
##     <spage><![CDATA[759]]></spage>
##     <epage><![CDATA[770]]></epage>
##     <abstract><![CDATA[This paper focuses on rigorous validation, verification, and testing methodologies applied to a commercial electromagnetic software package to ensure that as accurate as possible results are given dependent on the accuracy of the solution method, for instance, whether a full-wave or approximate numerical method is used. In this paper, the general availability of reliable benchmark results such as analytical solutions, measurements, results from other codes and other numerical methods, and general benchmarking activities will be presented. The cross-validation aspects, once the benchmark results are available, will be discussed with respect to amongst other sequential runs compared with parallel multcore/cluster runs or, with and without, GPU acceleration. Internal consistency checks (which are a required but not necessary condition when assessing the accuracy) such as power budget, mesh size convergence, or boundary condition error estimates are also covered. Special emphasis is put on the validation of the actual computational model that is used as input to simulations. This is necessary, for example, because incomplete representation of real geometry might ignore small details that are needed for the specific quantity that is analyzed. Also, uncertainties with regards to material parameters or transition impedances could lead to discrepancies between the computed results and reality that are not to be attributed to the electromagnetic solution as such, but rather the model generation.]]></abstract>
##     <issn><![CDATA[0018-9375]]></issn>
##     <htmlFlag><![CDATA[1]]></htmlFlag>
##     <arnumber><![CDATA[6722884]]></arnumber>
##     <doi><![CDATA[10.1109/TEMC.2014.2299408]]></doi>
##     <publicationId><![CDATA[6722884]]></publicationId>
##     <mdurl><![CDATA[http://ieeexplore.ieee.org/xpl/articleDetails.jsp?tp=&arnumber=6722884&contentType=Journals+%26+Magazines]]></mdurl>
##     <pdf><![CDATA[http://ieeexplore.ieee.org/stamp/stamp.jsp?arnumber=6722884]]></pdf>
##   </document>
##   <document>
##     <rank>10</rank>
##     <title><![CDATA[High-Isolation X-Band Marine Radar Antenna Design]]></title>
##     <authors><![CDATA[Fang-Yao Kuo;  Ruey-Bing Hwang]]></authors>
##     <affiliations><![CDATA[Dept. of Electr. Eng., Nat. Chiao-Tung Univ., Hsinchu, Taiwan]]></affiliations>
##     <controlledterms>
##       <term><![CDATA[marine communication]]></term>
##       <term><![CDATA[marine radar]]></term>
##       <term><![CDATA[microstrip antenna arrays]]></term>
##       <term><![CDATA[microwave antenna arrays]]></term>
##       <term><![CDATA[radar antennas]]></term>
##       <term><![CDATA[radar clutter]]></term>
##     </controlledterms>
##     <thesaurusterms>
##       <term><![CDATA[Antenna arrays]]></term>
##       <term><![CDATA[Antenna measurements]]></term>
##       <term><![CDATA[Arrays]]></term>
##       <term><![CDATA[Microstrip]]></term>
##       <term><![CDATA[Ports (Computers)]]></term>
##       <term><![CDATA[Radar antennas]]></term>
##     </thesaurusterms>
##     <pubtitle><![CDATA[Antennas and Propagation, IEEE Transactions on]]></pubtitle>
##     <punumber><![CDATA[8]]></punumber>
##     <pubtype><![CDATA[Journals & Magazines]]></pubtype>
##     <publisher><![CDATA[IEEE]]></publisher>
##     <volume><![CDATA[62]]></volume>
##     <issue><![CDATA[5]]></issue>
##     <py><![CDATA[2014]]></py>
##     <spage><![CDATA[2331]]></spage>
##     <epage><![CDATA[2337]]></epage>
##     <abstract><![CDATA[This paper presents a high-isolation printed antenna array for marine radar applications. The antenna array is composed of 32 identical square microstrip patches operated at a center frequency of 9.35 GHz and includes a 100-MHz bandwidth (subject to a 1.5:1 voltage standing wave ratio [VSWR]). The patch antennas are arranged in four arms, each of which contains eight elements and is series-fed using Chebyshev tapering (25 dB side-lobe level). To apply the antenna in marine radar applications, an antenna with horizontal polarization was employed because, in comparison with vertical polarization, it can relatively reduce the sea clutter reflectivity. Therefore, a slit was carved on each patch element to change the current path, thereby enabling horizontal polarization. The antenna gain, 3-dB beamwidth, side-lobe level, and front-to-back ratio were 22 dBi, 5.3 <sup>&#x00B0;</sup>, 26.4 dB, and 38.5 dB, respectively. Additionally, metallic baffles were introduced for increasing the isolation between the transmitting and receiving antennas to 60 dB .]]></abstract>
##     <issn><![CDATA[0018-926X]]></issn>
##     <htmlFlag><![CDATA[1]]></htmlFlag>
##     <arnumber><![CDATA[6746051]]></arnumber>
##     <doi><![CDATA[10.1109/TAP.2014.2307296]]></doi>
##     <publicationId><![CDATA[6746051]]></publicationId>
##     <mdurl><![CDATA[http://ieeexplore.ieee.org/xpl/articleDetails.jsp?tp=&arnumber=6746051&contentType=Journals+%26+Magazines]]></mdurl>
##     <pdf><![CDATA[http://ieeexplore.ieee.org/stamp/stamp.jsp?arnumber=6746051]]></pdf>
##   </document>
##   <document>
##     <rank>11</rank>
##     <title><![CDATA[An Efficient Numerical Full-Vectorial Mode Solver Based on Fourier Series Expansion Method]]></title>
##     <authors><![CDATA[Hsiao, C.S.;  Chiang, Y.J.;  Wang, L.;  Ching, T.K.]]></authors>
##     <affiliations><![CDATA[Dept. of Photonics & Commun., Asia Univ., Taichung, Taiwan]]></affiliations>
##     <controlledterms>
##       <term><![CDATA[Fourier series]]></term>
##       <term><![CDATA[Newton-Raphson method]]></term>
##       <term><![CDATA[eigenvalues and eigenfunctions]]></term>
##       <term><![CDATA[matrix algebra]]></term>
##       <term><![CDATA[optical waveguide theory]]></term>
##       <term><![CDATA[refractive index]]></term>
##     </controlledterms>
##     <thesaurusterms>
##       <term><![CDATA[Equations]]></term>
##       <term><![CDATA[Fourier series]]></term>
##       <term><![CDATA[Magnetic fields]]></term>
##       <term><![CDATA[Matrices]]></term>
##       <term><![CDATA[Optical waveguides]]></term>
##       <term><![CDATA[Transmission line matrix methods]]></term>
##     </thesaurusterms>
##     <pubtitle><![CDATA[Photonics Journal, IEEE]]></pubtitle>
##     <punumber><![CDATA[4563994]]></punumber>
##     <pubtype><![CDATA[Journals & Magazines]]></pubtype>
##     <publisher><![CDATA[IEEE]]></publisher>
##     <volume><![CDATA[6]]></volume>
##     <issue><![CDATA[4]]></issue>
##     <py><![CDATA[2014]]></py>
##     <spage><![CDATA[1]]></spage>
##     <epage><![CDATA[17]]></epage>
##     <abstract><![CDATA[A new algorithm of full-vectorial eigenmode solver is presented, which is utilized to determine the modal index of dielectric optical waveguides. The approach is based on the Fourier cosine and sine series expansions of the magnetic field distributions and the refractive index profile. By substituting these series expansions in the wave equation, a pair of second-order differential matrix equations is obtained by collecting all the terms with the same spatial frequency. With boundary conditions used, a matrix equation with a dimension of (N+1) by (N+1), where N is the number of terms for truncated series, is obtained, which can be easily solved by using the Newton-Raphson root-shooting algorithm. The presented scheme requires considerably less computational time and memory storage by only considering the finite terms of the Fourier cosine/sinusoidal series. Calculated results by our proposed method are in good agreement with those obtained by BeamPROP and COMSOL and compare well with other available methods, demonstrating the accuracy and efficiency and also the applicability of our proposed method.]]></abstract>
##     <issn><![CDATA[1943-0655]]></issn>
##     <htmlFlag><![CDATA[1]]></htmlFlag>
##     <arnumber><![CDATA[6849421]]></arnumber>
##     <doi><![CDATA[10.1109/JPHOT.2014.2335714]]></doi>
##     <publicationId><![CDATA[6849421]]></publicationId>
##     <mdurl><![CDATA[http://ieeexplore.ieee.org/xpl/articleDetails.jsp?tp=&arnumber=6849421&contentType=Journals+%26+Magazines]]></mdurl>
##     <pdf><![CDATA[http://ieeexplore.ieee.org/stamp/stamp.jsp?arnumber=6849421]]></pdf>
##   </document>
##   <document>
##     <rank>12</rank>
##     <title><![CDATA[On the Equivalence Between a Minimal Codomain Cardinality Riesz Basis Construction, a System of Hadamard&#x2013;Sylvester Operators, and a Class of Sparse, Binary Optimization Problems]]></title>
##     <authors><![CDATA[Nelson, J.D.B.]]></authors>
##     <affiliations><![CDATA[Dept. of Stat. Sci., Univ. Coll. London, London, UK]]></affiliations>
##     <controlledterms>
##       <term><![CDATA[Fourier series]]></term>
##       <term><![CDATA[optimisation]]></term>
##       <term><![CDATA[polynomials]]></term>
##     </controlledterms>
##     <thesaurusterms>
##       <term><![CDATA[Approximation methods]]></term>
##       <term><![CDATA[Context]]></term>
##       <term><![CDATA[Dictionaries]]></term>
##       <term><![CDATA[Dynamic range]]></term>
##       <term><![CDATA[Optimization]]></term>
##       <term><![CDATA[Polynomials]]></term>
##       <term><![CDATA[Signal processing]]></term>
##     </thesaurusterms>
##     <pubtitle><![CDATA[Signal Processing, IEEE Transactions on]]></pubtitle>
##     <punumber><![CDATA[78]]></punumber>
##     <pubtype><![CDATA[Journals & Magazines]]></pubtype>
##     <publisher><![CDATA[IEEE]]></publisher>
##     <volume><![CDATA[62]]></volume>
##     <issue><![CDATA[20]]></issue>
##     <py><![CDATA[2014]]></py>
##     <spage><![CDATA[5270]]></spage>
##     <epage><![CDATA[5281]]></epage>
##     <abstract><![CDATA[Piecewise, low-order polynomial, Riesz basis families are constructed such that they share the same coefficient functionals of smoother, orthonormal bases in a localized indexing subset. It is shown that a minimal cardinality basis codomain can be realized by inducing sparsity, via l1 regularization, in the distributional derivatives of the basis functions and that the optimal construction can be found numerically by constrained binary optimization over a suitably large dictionary. Furthermore, it is shown that a subset of these solutions are equivalent to a specific, constrained analytical solution, derived via Sylvester-type Hadamard operators.]]></abstract>
##     <issn><![CDATA[1053-587X]]></issn>
##     <htmlFlag><![CDATA[1]]></htmlFlag>
##     <arnumber><![CDATA[6870501]]></arnumber>
##     <doi><![CDATA[10.1109/TSP.2014.2345346]]></doi>
##     <publicationId><![CDATA[6870501]]></publicationId>
##     <mdurl><![CDATA[http://ieeexplore.ieee.org/xpl/articleDetails.jsp?tp=&arnumber=6870501&contentType=Journals+%26+Magazines]]></mdurl>
##     <pdf><![CDATA[http://ieeexplore.ieee.org/stamp/stamp.jsp?arnumber=6870501]]></pdf>
##   </document>
##   <document>
##     <rank>13</rank>
##     <title><![CDATA[Interferometric Scheme for High-Sensitivity Coaxial Focusing in Projection Lithography]]></title>
##     <authors><![CDATA[Chengliang Di;  Song Hu;  Wei Yan;  Yanli Li;  Guang Li;  Junmin Tong]]></authors>
##     <affiliations><![CDATA[State Key Lab. of Opt. Technol. for Microfabrication, Inst. of Opt. & Electron., Chengdu, China]]></affiliations>
##     <controlledterms>
##       <term><![CDATA[CCD image sensors]]></term>
##       <term><![CDATA[Michelson interferometers]]></term>
##       <term><![CDATA[focal planes]]></term>
##       <term><![CDATA[laser beams]]></term>
##       <term><![CDATA[light interference]]></term>
##       <term><![CDATA[light interferometry]]></term>
##       <term><![CDATA[mirrors]]></term>
##       <term><![CDATA[optical focusing]]></term>
##       <term><![CDATA[optical projectors]]></term>
##       <term><![CDATA[photolithography]]></term>
##     </controlledterms>
##     <thesaurusterms>
##       <term><![CDATA[Charge coupled devices]]></term>
##       <term><![CDATA[Focusing]]></term>
##       <term><![CDATA[Interferometric lithography]]></term>
##       <term><![CDATA[Laser beams]]></term>
##       <term><![CDATA[Lenses]]></term>
##       <term><![CDATA[Optical interferometry]]></term>
##     </thesaurusterms>
##     <pubtitle><![CDATA[Photonics Journal, IEEE]]></pubtitle>
##     <punumber><![CDATA[4563994]]></punumber>
##     <pubtype><![CDATA[Journals & Magazines]]></pubtype>
##     <publisher><![CDATA[IEEE]]></publisher>
##     <volume><![CDATA[6]]></volume>
##     <issue><![CDATA[3]]></issue>
##     <py><![CDATA[2014]]></py>
##     <spage><![CDATA[1]]></spage>
##     <epage><![CDATA[10]]></epage>
##     <abstract><![CDATA[Focusing of wafer plane is an essential factor to determine the ultimate feature size of the stepper such as projection lithographic system. Based on Michelson interferometeric system, this paper demonstrates an interferometric focusing scheme for projection lithography to coaxially locate the ideal focal plane of the projective objective. The collimated incident laser beam is divided into the reference arm and object arm. The latter propagates through the objective lens and then interferes with the slightly deflected reference beam that reflected back by a fixed mirror, giving rise to an interferential pattern on the CCD. Any amounts of defocusing can be directly indicated from the demodulated phase of the interferential pattern. In this manner, the focusing sensitivity at nanometer scale is experimentally attainable, which shows great superiority over traditional methods, particularly the limited focal length of current projective objective lens.]]></abstract>
##     <issn><![CDATA[1943-0655]]></issn>
##     <htmlFlag><![CDATA[1]]></htmlFlag>
##     <arnumber><![CDATA[6820728]]></arnumber>
##     <doi><![CDATA[10.1109/JPHOT.2014.2326676]]></doi>
##     <publicationId><![CDATA[6820728]]></publicationId>
##     <mdurl><![CDATA[http://ieeexplore.ieee.org/xpl/articleDetails.jsp?tp=&arnumber=6820728&contentType=Journals+%26+Magazines]]></mdurl>
##     <pdf><![CDATA[http://ieeexplore.ieee.org/stamp/stamp.jsp?arnumber=6820728]]></pdf>
##   </document>
##   <document>
##     <rank>14</rank>
##     <title><![CDATA[Intersublevel Photoabsorption and Photoelectric Processes in ZnO Quantum Dot Embedded in <inline-formula> <img src="/images/tex/594.gif" alt="\hbox {HfO}_{2}"> </inline-formula> and AlN Matrices]]></title>
##     <authors><![CDATA[Maikhuri, D.;  Purohit, S.P.;  Mathur, K.C.]]></authors>
##     <affiliations><![CDATA[Dept. of Phys. & Mater. Sci. & Eng., Jaypee Inst. of Inf. Technol., Noida, India]]></affiliations>
##     <controlledterms>
##       <term><![CDATA[II-VI semiconductors]]></term>
##       <term><![CDATA[alumina]]></term>
##       <term><![CDATA[conduction bands]]></term>
##       <term><![CDATA[hafnium compounds]]></term>
##       <term><![CDATA[nonlinear optics]]></term>
##       <term><![CDATA[photoelectricity]]></term>
##       <term><![CDATA[photoexcitation]]></term>
##       <term><![CDATA[semiconductor quantum dots]]></term>
##       <term><![CDATA[wide band gap semiconductors]]></term>
##       <term><![CDATA[zinc compounds]]></term>
##     </controlledterms>
##     <thesaurusterms>
##       <term><![CDATA[Hafnium compounds]]></term>
##       <term><![CDATA[III-V semiconductor materials]]></term>
##       <term><![CDATA[Nonlinear optics]]></term>
##       <term><![CDATA[Optical polarization]]></term>
##       <term><![CDATA[Transmission line matrix methods]]></term>
##       <term><![CDATA[Zinc oxide]]></term>
##     </thesaurusterms>
##     <pubtitle><![CDATA[Photonics Journal, IEEE]]></pubtitle>
##     <punumber><![CDATA[4563994]]></punumber>
##     <pubtype><![CDATA[Journals & Magazines]]></pubtype>
##     <publisher><![CDATA[IEEE]]></publisher>
##     <volume><![CDATA[6]]></volume>
##     <issue><![CDATA[5]]></issue>
##     <py><![CDATA[2014]]></py>
##     <spage><![CDATA[1]]></spage>
##     <epage><![CDATA[15]]></epage>
##     <abstract><![CDATA[In this paper, we investigate the linear and nonlinear photoabsorption processes in the conduction-band-confined levels of a singly charged ZnO quantum dot (QD) surrounded by HfO<sub>2</sub> and AlN matrices. We also investigate the photoelectric process in which the conduction band electron ejects from the dot to the vacuum. We use the effective mass approximation with a finite barrier height at the dot-matrix interface. We consider the self-energy of the electron in the dot and the local field effect. The electromagnetic interaction of the incident radiation with the electron in the dot is considered in the electric dipole plus quadrupole approximation. Results for the photoabsorption coefficient and the photoelectric process are presented for different dot sizes and different intensities of incident radiation. It is found that the inclusion of the quadrupole effect reveals new photoabsorption peaks in the absorption spectra. Both the photoabsorption and photoelectric processes significantly depend on the dot size and the surrounding matrix. The change in the intensity of the incident radiation significantly influences the nonlinear photoabsorption. The photoabsorption coefficient and the photoelectric cross sections are found to be relatively higher for the ZnO QD embedded in the high-dielectric constant matrix HfO<sub>2</sub> as compared with the lower-dielectric constant AlN matrix.]]></abstract>
##     <issn><![CDATA[1943-0655]]></issn>
##     <htmlFlag><![CDATA[1]]></htmlFlag>
##     <arnumber><![CDATA[6798755]]></arnumber>
##     <doi><![CDATA[10.1109/JPHOT.2014.2317677]]></doi>
##     <publicationId><![CDATA[6798755]]></publicationId>
##     <mdurl><![CDATA[http://ieeexplore.ieee.org/xpl/articleDetails.jsp?tp=&arnumber=6798755&contentType=Journals+%26+Magazines]]></mdurl>
##     <pdf><![CDATA[http://ieeexplore.ieee.org/stamp/stamp.jsp?arnumber=6798755]]></pdf>
##   </document>
##   <document>
##     <rank>15</rank>
##     <title><![CDATA[An Unbiased Risk Estimator for Image Denoising in the Presence of Mixed Poisson&#x2013;Gaussian Noise]]></title>
##     <authors><![CDATA[Le Montagner, Y.;  Angelini, E.D.;  Olivo-Marin, J.-C.]]></authors>
##     <affiliations><![CDATA[Inst. Mines-Telecom, Telecom ParisTech, Paris, France]]></affiliations>
##     <controlledterms>
##       <term><![CDATA[Gaussian noise]]></term>
##       <term><![CDATA[biomedical optical imaging]]></term>
##       <term><![CDATA[image denoising]]></term>
##       <term><![CDATA[mean square error methods]]></term>
##       <term><![CDATA[medical image processing]]></term>
##       <term><![CDATA[minimisation]]></term>
##       <term><![CDATA[risk analysis]]></term>
##     </controlledterms>
##     <thesaurusterms>
##       <term><![CDATA[Biological system modeling]]></term>
##       <term><![CDATA[Gaussian noise]]></term>
##       <term><![CDATA[Noise reduction]]></term>
##       <term><![CDATA[Numerical models]]></term>
##       <term><![CDATA[Random variables]]></term>
##       <term><![CDATA[Vectors]]></term>
##     </thesaurusterms>
##     <pubtitle><![CDATA[Image Processing, IEEE Transactions on]]></pubtitle>
##     <punumber><![CDATA[83]]></punumber>
##     <pubtype><![CDATA[Journals & Magazines]]></pubtype>
##     <publisher><![CDATA[IEEE]]></publisher>
##     <volume><![CDATA[23]]></volume>
##     <issue><![CDATA[3]]></issue>
##     <py><![CDATA[2014]]></py>
##     <spage><![CDATA[1255]]></spage>
##     <epage><![CDATA[1268]]></epage>
##     <abstract><![CDATA[The behavior and performance of denoising algorithms are governed by one or several parameters, whose optimal settings depend on the content of the processed image and the characteristics of the noise, and are generally designed to minimize the mean squared error (MSE) between the denoised image returned by the algorithm and a virtual ground truth. In this paper, we introduce a new Poisson-Gaussian unbiased risk estimator (PG-URE) of the MSE applicable to a mixed Poisson-Gaussian noise model that unifies the widely used Gaussian and Poisson noise models in fluorescence bioimaging applications. We propose a stochastic methodology to evaluate this estimator in the case when little is known about the internal machinery of the considered denoising algorithm, and we analyze both theoretically and empirically the characteristics of the PG-URE estimator. Finally, we evaluate the PG-URE-driven parametrization for three standard denoising algorithms, with and without variance stabilizing transforms, and different characteristics of the Poisson-Gaussian noise mixture.]]></abstract>
##     <issn><![CDATA[1057-7149]]></issn>
##     <htmlFlag><![CDATA[1]]></htmlFlag>
##     <arnumber><![CDATA[6714502]]></arnumber>
##     <doi><![CDATA[10.1109/TIP.2014.2300821]]></doi>
##     <publicationId><![CDATA[6714502]]></publicationId>
##     <mdurl><![CDATA[http://ieeexplore.ieee.org/xpl/articleDetails.jsp?tp=&arnumber=6714502&contentType=Journals+%26+Magazines]]></mdurl>
##     <pdf><![CDATA[http://ieeexplore.ieee.org/stamp/stamp.jsp?arnumber=6714502]]></pdf>
##   </document>
##   <document>
##     <rank>16</rank>
##     <title><![CDATA[Predictive Hot-Carrier Modeling of n-Channel MOSFETs]]></title>
##     <authors><![CDATA[Bina, M.;  Tyaginov, S.;  Franco, J.;  Rupp, K.;  Wimmer, Y.;  Osintsev, D.;  Kaczer, B.;  Grasser, T.]]></authors>
##     <affiliations><![CDATA[Inst. for Microelectron., Tech. Univ. Wien, Vienna, Austria]]></affiliations>
##     <controlledterms>
##       <term><![CDATA[Boltzmann equation]]></term>
##       <term><![CDATA[MOSFET]]></term>
##       <term><![CDATA[dissociation]]></term>
##       <term><![CDATA[electron-electron scattering]]></term>
##       <term><![CDATA[hot carriers]]></term>
##       <term><![CDATA[hydrogen bonds]]></term>
##       <term><![CDATA[ionisation potential]]></term>
##       <term><![CDATA[semiconductor device breakdown]]></term>
##       <term><![CDATA[semiconductor device models]]></term>
##       <term><![CDATA[semiconductor device reliability]]></term>
##       <term><![CDATA[silicon]]></term>
##     </controlledterms>
##     <thesaurusterms>
##       <term><![CDATA[Degradation]]></term>
##       <term><![CDATA[Hot carriers]]></term>
##       <term><![CDATA[MOSFET]]></term>
##       <term><![CDATA[Mathematical model]]></term>
##       <term><![CDATA[Oscillators]]></term>
##       <term><![CDATA[Semiconductor device modeling]]></term>
##       <term><![CDATA[Stress]]></term>
##     </thesaurusterms>
##     <pubtitle><![CDATA[Electron Devices, IEEE Transactions on]]></pubtitle>
##     <punumber><![CDATA[16]]></punumber>
##     <pubtype><![CDATA[Journals & Magazines]]></pubtype>
##     <publisher><![CDATA[IEEE]]></publisher>
##     <volume><![CDATA[61]]></volume>
##     <issue><![CDATA[9]]></issue>
##     <py><![CDATA[2014]]></py>
##     <spage><![CDATA[3103]]></spage>
##     <epage><![CDATA[3110]]></epage>
##     <abstract><![CDATA[We present a physics-based hot-carrier degradation (HCD) model and validate it against measurement data on SiON n-channel MOSFETs of various channel lengths, from ultrascaled to long-channel transistors. The HCD model is capable of representing HCD in all these transistors stressed under different conditions using a unique set of model parameters. The degradation is modeled as a dissociation of Si-H bonds induced by two competing processes. It can be triggered by solitary highly energetical charge carriers or by excitation of multiple vibrational modes of the bond. In addition, we show that the influence of electron-electron scattering (EES), the dipole-field interaction, and the dispersion of the Si-H bond energy are crucial for understanding and modeling HCD. All model ingredients are considered on the basis of a deterministic Boltzmann transport equation solver, which serves as the transport kernel of a physics-based HCD model. Using this model, we analyze the role of each ingredient and show that EES may only be neglected in long-channel transistors, but is essential in ultrascaled devices.]]></abstract>
##     <issn><![CDATA[0018-9383]]></issn>
##     <htmlFlag><![CDATA[1]]></htmlFlag>
##     <arnumber><![CDATA[6871362]]></arnumber>
##     <doi><![CDATA[10.1109/TED.2014.2340575]]></doi>
##     <publicationId><![CDATA[6871362]]></publicationId>
##     <mdurl><![CDATA[http://ieeexplore.ieee.org/xpl/articleDetails.jsp?tp=&arnumber=6871362&contentType=Journals+%26+Magazines]]></mdurl>
##     <pdf><![CDATA[http://ieeexplore.ieee.org/stamp/stamp.jsp?arnumber=6871362]]></pdf>
##   </document>
##   <document>
##     <rank>17</rank>
##     <title><![CDATA[Prediction of Fasting Plasma Glucose Status Using Anthropometric Measures for Diagnosing Type 2 Diabetes]]></title>
##     <authors><![CDATA[Bum Ju Lee;  Boncho Ku;  Jiho Nam;  Duong Duc Pham;  Jong Yeol Kim]]></authors>
##     <affiliations><![CDATA[Korea Inst. of Oriental Med., Daejeon, South Korea]]></affiliations>
##     <controlledterms>
##       <term><![CDATA[Bayes methods]]></term>
##       <term><![CDATA[anthropometry]]></term>
##       <term><![CDATA[biochemistry]]></term>
##       <term><![CDATA[blood]]></term>
##       <term><![CDATA[diseases]]></term>
##       <term><![CDATA[learning (artificial intelligence)]]></term>
##       <term><![CDATA[medical signal processing]]></term>
##       <term><![CDATA[patient diagnosis]]></term>
##       <term><![CDATA[regression analysis]]></term>
##       <term><![CDATA[sensitivity analysis]]></term>
##       <term><![CDATA[signal classification]]></term>
##     </controlledterms>
##     <thesaurusterms>
##       <term><![CDATA[Data models]]></term>
##       <term><![CDATA[Diabetes]]></term>
##       <term><![CDATA[Logistics]]></term>
##       <term><![CDATA[Optical wavelength conversion]]></term>
##       <term><![CDATA[Power measurement]]></term>
##       <term><![CDATA[Predictive models]]></term>
##       <term><![CDATA[Sensitivity]]></term>
##     </thesaurusterms>
##     <pubtitle><![CDATA[Biomedical and Health Informatics, IEEE Journal of]]></pubtitle>
##     <punumber><![CDATA[6221020]]></punumber>
##     <pubtype><![CDATA[Journals & Magazines]]></pubtype>
##     <publisher><![CDATA[IEEE]]></publisher>
##     <volume><![CDATA[18]]></volume>
##     <issue><![CDATA[2]]></issue>
##     <py><![CDATA[2014]]></py>
##     <spage><![CDATA[555]]></spage>
##     <epage><![CDATA[561]]></epage>
##     <abstract><![CDATA[It is well known that body fat distribution and obesity are important risk factors for type 2 diabetes. Prediction of type 2 diabetes using a combination of anthropometric measures remains a controversial issue. This study aims to predict the fasting plasma glucose (FPG) status that is used in the diagnosis of type 2 diabetes by a combination of various measures among Korean adults. A total of 4870 subjects (2955 females and 1915 males) participated in this study. Based on 37 anthropometric measures, we compared predictions of FPG status using individual versus combined measures using two machine-learning algorithms. The values of the area under the receiver operating characteristic curve in the predictions by logistic regression and naive Bayes classifier based on the combination of measures were 0.741 and 0.739 in females, respectively, and were 0.687 and 0.686 in males, respectively. Our results indicate that prediction of FPG status using a combination of anthropometric measures was superior to individual measures alone in both females and males. We show that using balanced data of normal and high FPG groups can improve the prediction and reduce the intrinsic bias of the model toward the majority class.]]></abstract>
##     <issn><![CDATA[2168-2194]]></issn>
##     <htmlFlag><![CDATA[1]]></htmlFlag>
##     <arnumber><![CDATA[6556939]]></arnumber>
##     <doi><![CDATA[10.1109/JBHI.2013.2264509]]></doi>
##     <publicationId><![CDATA[6556939]]></publicationId>
##     <mdurl><![CDATA[http://ieeexplore.ieee.org/xpl/articleDetails.jsp?tp=&arnumber=6556939&contentType=Journals+%26+Magazines]]></mdurl>
##     <pdf><![CDATA[http://ieeexplore.ieee.org/stamp/stamp.jsp?arnumber=6556939]]></pdf>
##   </document>
##   <document>
##     <rank>18</rank>
##     <title><![CDATA[Bayesian Discovery of Threat Networks]]></title>
##     <authors><![CDATA[Smith, S.T.;  Kao, E.K.;  Senne, K.D.;  Bernstein, G.;  Philips, S.]]></authors>
##     <affiliations><![CDATA[MIT Lincoln Lab., Lexington, MA, USA]]></affiliations>
##     <controlledterms>
##       <term><![CDATA[Bayes methods]]></term>
##       <term><![CDATA[network theory (graphs)]]></term>
##       <term><![CDATA[random processes]]></term>
##       <term><![CDATA[spatiotemporal phenomena]]></term>
##     </controlledterms>
##     <thesaurusterms>
##       <term><![CDATA[Bayes methods]]></term>
##       <term><![CDATA[Detection algorithms]]></term>
##       <term><![CDATA[Hidden Markov models]]></term>
##       <term><![CDATA[Laplace equations]]></term>
##       <term><![CDATA[Signal processing algorithms]]></term>
##       <term><![CDATA[Symmetric matrices]]></term>
##       <term><![CDATA[Vectors]]></term>
##     </thesaurusterms>
##     <pubtitle><![CDATA[Signal Processing, IEEE Transactions on]]></pubtitle>
##     <punumber><![CDATA[78]]></punumber>
##     <pubtype><![CDATA[Journals & Magazines]]></pubtype>
##     <publisher><![CDATA[IEEE]]></publisher>
##     <volume><![CDATA[62]]></volume>
##     <issue><![CDATA[20]]></issue>
##     <py><![CDATA[2014]]></py>
##     <spage><![CDATA[5324]]></spage>
##     <epage><![CDATA[5338]]></epage>
##     <abstract><![CDATA[A novel unified Bayesian framework for network detection is developed, under which a detection algorithm is derived based on random walks on graphs. The algorithm detects threat networks using partial observations of their activity, and is proved to be optimum in the Neyman-Pearson sense. The algorithm is defined by a graph, at least one observation, and a diffusion model for threat. A link to well-known spectral detection methods is provided, and the equivalence of the random walk and harmonic solutions to the Bayesian formulation is proven. A general diffusion model is introduced that utilizes spatio-temporal relationships between vertices, and is used for a specific space-time formulation that leads to significant performance improvements on coordinated covert networks. This performance is demonstrated using a new hybrid mixed-membership blockmodel introduced to simulate random covert networks with realistic properties.]]></abstract>
##     <issn><![CDATA[1053-587X]]></issn>
##     <htmlFlag><![CDATA[1]]></htmlFlag>
##     <arnumber><![CDATA[6850046]]></arnumber>
##     <doi><![CDATA[10.1109/TSP.2014.2336613]]></doi>
##     <publicationId><![CDATA[6850046]]></publicationId>
##     <mdurl><![CDATA[http://ieeexplore.ieee.org/xpl/articleDetails.jsp?tp=&arnumber=6850046&contentType=Journals+%26+Magazines]]></mdurl>
##     <pdf><![CDATA[http://ieeexplore.ieee.org/stamp/stamp.jsp?arnumber=6850046]]></pdf>
##   </document>
##   <document>
##     <rank>19</rank>
##     <title><![CDATA[A New FAPAR Analytical Model Based on the Law of Energy Conservation: A Case Study in China]]></title>
##     <authors><![CDATA[Wenjie Fan;  Yuan Liu;  Xiru Xu;  Gaoxing Chen;  Beitong Zhang]]></authors>
##     <affiliations><![CDATA[Inst. of Remote Sensing & Geogr. Inf. Syst., Peking Univ., Beijing, China]]></affiliations>
##     <controlledterms>
##       <term><![CDATA[energy conservation]]></term>
##       <term><![CDATA[geophysical techniques]]></term>
##       <term><![CDATA[remote sensing]]></term>
##       <term><![CDATA[vegetation]]></term>
##     </controlledterms>
##     <thesaurusterms>
##       <term><![CDATA[Absorption]]></term>
##       <term><![CDATA[Biological system modeling]]></term>
##       <term><![CDATA[Indexes]]></term>
##       <term><![CDATA[Photonics]]></term>
##       <term><![CDATA[Scattering]]></term>
##       <term><![CDATA[Soil]]></term>
##       <term><![CDATA[Vegetation mapping]]></term>
##     </thesaurusterms>
##     <pubtitle><![CDATA[Selected Topics in Applied Earth Observations and Remote Sensing, IEEE Journal of]]></pubtitle>
##     <punumber><![CDATA[4609443]]></punumber>
##     <pubtype><![CDATA[Journals & Magazines]]></pubtype>
##     <publisher><![CDATA[IEEE]]></publisher>
##     <volume><![CDATA[7]]></volume>
##     <issue><![CDATA[9]]></issue>
##     <py><![CDATA[2014]]></py>
##     <spage><![CDATA[3945]]></spage>
##     <epage><![CDATA[3955]]></epage>
##     <abstract><![CDATA[The fraction of absorbed photosynthetically active radiation (FAPAR) characterizes the energy-absorption ability of the vegetation canopy. It is a critical input to many land-surface models such as crop growth models, net primary productivity models, and climate models. There is a great need for FAPAR products derived from remote-sensing data. The objective of this research is to develop a new instantaneous quantitative FAPAR model based on the law of energy conservation and the concept of recollision probability (p). Using the ray-tracing method, the FAPAR-P model separates direct energy absorption by the canopy from energy absorption caused by multiple scattering between the soil and the canopy. Direct sunlight and diffuse skylight are also considered. This model has a clear physical meaning and can be applied to continuous and discrete vegetation. The model was validated by Monte Carlo (MC) simulation and field measurements in the Heihe River basin, China, which proved its reliability for FAPAR calculations.]]></abstract>
##     <issn><![CDATA[1939-1404]]></issn>
##     <htmlFlag><![CDATA[1]]></htmlFlag>
##     <arnumber><![CDATA[6838974]]></arnumber>
##     <doi><![CDATA[10.1109/JSTARS.2014.2325673]]></doi>
##     <publicationId><![CDATA[6838974]]></publicationId>
##     <mdurl><![CDATA[http://ieeexplore.ieee.org/xpl/articleDetails.jsp?tp=&arnumber=6838974&contentType=Journals+%26+Magazines]]></mdurl>
##     <pdf><![CDATA[http://ieeexplore.ieee.org/stamp/stamp.jsp?arnumber=6838974]]></pdf>
##   </document>
##   <document>
##     <rank>20</rank>
##     <title><![CDATA[High-Frequency Oscillations Recorded on the Scalp of Patients With Epilepsy Using Tripolar Concentric Ring Electrodes]]></title>
##     <authors><![CDATA[Besio, W.G.;  Martinez-Juarez, I.E.;  Makeyev, O.;  Gaitanis, J.N.;  Blum, A.S.;  Fisher, R.S.;  Medvedev, A.V.]]></authors>
##     <affiliations><![CDATA[Dept. of Electr., Comput., & Biomed. Eng., Univ. of Rhode Island, Kingston, RI, USA]]></affiliations>
##     <controlledterms>
##       <term><![CDATA[biomedical electrodes]]></term>
##       <term><![CDATA[data acquisition]]></term>
##       <term><![CDATA[data analysis]]></term>
##       <term><![CDATA[electroencephalography]]></term>
##       <term><![CDATA[medical disorders]]></term>
##       <term><![CDATA[medical signal detection]]></term>
##       <term><![CDATA[medical signal processing]]></term>
##       <term><![CDATA[neurophysiology]]></term>
##       <term><![CDATA[oscillations]]></term>
##     </controlledterms>
##     <thesaurusterms>
##       <term><![CDATA[Educational institutions]]></term>
##       <term><![CDATA[Electrodes]]></term>
##       <term><![CDATA[Electroencephalography]]></term>
##       <term><![CDATA[Epilepsy]]></term>
##       <term><![CDATA[Hafnium compounds]]></term>
##       <term><![CDATA[Oscillators]]></term>
##       <term><![CDATA[Scalp]]></term>
##     </thesaurusterms>
##     <pubtitle><![CDATA[Translational Engineering in Health and Medicine, IEEE Journal of]]></pubtitle>
##     <punumber><![CDATA[6221039]]></punumber>
##     <pubtype><![CDATA[Journals & Magazines]]></pubtype>
##     <publisher><![CDATA[IEEE]]></publisher>
##     <volume><![CDATA[2]]></volume>
##     <py><![CDATA[2014]]></py>
##     <spage><![CDATA[1]]></spage>
##     <epage><![CDATA[11]]></epage>
##     <abstract><![CDATA[Epilepsy is the second most prevalent neurological disorder (~1% prevalence) affecting ~67 million people worldwide with up to 75% from developing countries. The conventional electroencephalogram is plagued with artifacts from movements, muscles, and other sources. Tripolar concentric ring electrodes automatically attenuate muscle artifacts and provide improved signal quality. We performed basic experiments in healthy humans to show that tripolar concentric ring electrodes can indeed record the physiological alpha waves while eyes are closed. We then conducted concurrent recordings with conventional disc electrodes and tripolar concentric ring electrodes from patients with epilepsy. We found that we could detect high frequency oscillations, a marker for early seizure development and epileptogenic zone, on the scalp surface that appeared to become more narrow-band just prior to seizures. High frequency oscillations preceding seizures were present in an average of 35.5% of tripolar concentric ring electrode data channels for all the patients with epilepsy whose seizures were recorded and absent in the corresponding conventional disc electrode data. An average of 78.2% of channels that contained high frequency oscillations were within the seizure onset or irritative zones determined independently by three epileptologists based on conventional disc electrode data and videos.]]></abstract>
##     <issn><![CDATA[2168-2372]]></issn>
##     <htmlFlag><![CDATA[1]]></htmlFlag>
##     <arnumber><![CDATA[6846315]]></arnumber>
##     <doi><![CDATA[10.1109/JTEHM.2014.2332994]]></doi>
##     <publicationId><![CDATA[6846315]]></publicationId>
##     <mdurl><![CDATA[http://ieeexplore.ieee.org/xpl/articleDetails.jsp?tp=&arnumber=6846315&contentType=Journals+%26+Magazines]]></mdurl>
##     <pdf><![CDATA[http://ieeexplore.ieee.org/stamp/stamp.jsp?arnumber=6846315]]></pdf>
##   </document>
##   <document>
##     <rank>21</rank>
##     <title><![CDATA[A Contract-Based Methodology for Aircraft Electric Power System Design]]></title>
##     <authors><![CDATA[Nuzzo, P.;  Huan Xu;  Ozay, N.;  Finn, J.B.;  Sangiovanni-Vincentelli, A.L.;  Murray, R.M.;  Donze, A.;  Seshia, S.A.]]></authors>
##     <affiliations><![CDATA[Dept. of Electr. Eng. & Comput. Sci., Univ. of California at Berkeley, Berkeley, CA, USA]]></affiliations>
##     <controlledterms>
##       <term><![CDATA[CAD]]></term>
##       <term><![CDATA[aircraft]]></term>
##       <term><![CDATA[aircraft power systems]]></term>
##       <term><![CDATA[integer programming]]></term>
##       <term><![CDATA[linear programming]]></term>
##       <term><![CDATA[power engineering computing]]></term>
##       <term><![CDATA[temporal logic]]></term>
##     </controlledterms>
##     <thesaurusterms>
##       <term><![CDATA[Aircraft manufacture]]></term>
##       <term><![CDATA[Cyberphysical systems]]></term>
##       <term><![CDATA[Design automation]]></term>
##       <term><![CDATA[Design methodology]]></term>
##       <term><![CDATA[Power system stabilty]]></term>
##     </thesaurusterms>
##     <pubtitle><![CDATA[Access, IEEE]]></pubtitle>
##     <punumber><![CDATA[6287639]]></punumber>
##     <pubtype><![CDATA[Journals & Magazines]]></pubtype>
##     <publisher><![CDATA[IEEE]]></publisher>
##     <volume><![CDATA[2]]></volume>
##     <py><![CDATA[2014]]></py>
##     <spage><![CDATA[1]]></spage>
##     <epage><![CDATA[25]]></epage>
##     <abstract><![CDATA[In an aircraft electric power system, one or more supervisory control units actuate a set of electromechanical switches to dynamically distribute power from generators to loads, while satisfying safety, reliability, and real-time performance requirements. To reduce expensive redesign steps, this control problem is generally addressed by minor incremental changes on top of consolidated solutions. A more systematic approach is hindered by a lack of rigorous design methodologies that allow estimating the impact of earlier design decisions on the final implementation. To achieve an optimal implementation that satisfies a set of requirements, we propose a platform-based methodology for electric power system design, which enables independent implementation of system topology (i.e., interconnection among elements) and control protocol by using a compositional approach. In our flow, design space exploration is carried out as a sequence of refinement steps from the initial specification toward a final implementation by mapping higher level behavioral and performance models into a set of either existing or virtual library components at the lower level of abstraction. Specifications are first expressed using the formalisms of linear temporal logic, signal temporal logic, and arithmetic constraints on Boolean variables. To reason about different requirements, we use specialized analysis and synthesis frameworks and formulate assume guarantee contracts at the articulation points in the design flow. We show the effectiveness of our approach on a proof-of-concept electric power system design.]]></abstract>
##     <issn><![CDATA[2169-3536]]></issn>
##     <htmlFlag><![CDATA[1]]></htmlFlag>
##     <arnumber><![CDATA[6690099]]></arnumber>
##     <doi><![CDATA[10.1109/ACCESS.2013.2295764]]></doi>
##     <publicationId><![CDATA[6690099]]></publicationId>
##     <mdurl><![CDATA[http://ieeexplore.ieee.org/xpl/articleDetails.jsp?tp=&arnumber=6690099&contentType=Journals+%26+Magazines]]></mdurl>
##     <pdf><![CDATA[http://ieeexplore.ieee.org/stamp/stamp.jsp?arnumber=6690099]]></pdf>
##   </document>
##   <document>
##     <rank>22</rank>
##     <title><![CDATA[Fast Exact Euclidean Distance (FEED): A New Class of Adaptable Distance Transforms]]></title>
##     <authors><![CDATA[Schouten, T.E.;  Van Den Broek, E.L.]]></authors>
##     <affiliations><![CDATA[Inst. for Comput. & Inf. Sci. (ICIS), Radboud Univ., Nijmegen, Netherlands]]></affiliations>
##     <controlledterms>
##       <term><![CDATA[image processing]]></term>
##       <term><![CDATA[transforms]]></term>
##     </controlledterms>
##     <thesaurusterms>
##       <term><![CDATA[Algorithm design and analysis]]></term>
##       <term><![CDATA[Approximation algorithms]]></term>
##       <term><![CDATA[Euclidean distance]]></term>
##       <term><![CDATA[Feeds]]></term>
##       <term><![CDATA[Search problems]]></term>
##       <term><![CDATA[Transforms]]></term>
##     </thesaurusterms>
##     <pubtitle><![CDATA[Pattern Analysis and Machine Intelligence, IEEE Transactions on]]></pubtitle>
##     <punumber><![CDATA[34]]></punumber>
##     <pubtype><![CDATA[Journals & Magazines]]></pubtype>
##     <publisher><![CDATA[IEEE]]></publisher>
##     <volume><![CDATA[36]]></volume>
##     <issue><![CDATA[11]]></issue>
##     <py><![CDATA[2014]]></py>
##     <spage><![CDATA[2159]]></spage>
##     <epage><![CDATA[2172]]></epage>
##     <abstract><![CDATA[A new unique class of foldable distance transforms of digital images (DT) is introduced, baptized: Fast exact euclidean distance (FEED) transforms. FEED class algorithms calculate the DT starting-directly from the definition or rather its inverse. The principle of FEED class algorithms is introduced, followed by strategies for their efficient implementation. It is shown that FEED class algorithms unite properties of ordered propagation, raster scanning, and independent scanning DT. Moreover, FEED class algorithms shown to have a unique property: they can be tailored to the images under investigation. Benchmarks are conducted on both the Fabbri et al. data set and on a newly developed data set. Three baseline, three approximate, and three state-of-the-art DT algorithms were included, in addition to two implementations of FEED class algorithms. It illustrates that FEED class algorithms i) provide truly exact Euclidean DT; ii) do no suffer from disconnected Voronoi tiles, which is a unique feature for non-parallel but fast DT; iii) outperform any other approximate and exact Euclidean DT with its time complexity O(N), even after their optimization; and iv) are unequaled in that they can be adapted to the characteristics of the image class at hand.]]></abstract>
##     <issn><![CDATA[0162-8828]]></issn>
##     <htmlFlag><![CDATA[1]]></htmlFlag>
##     <arnumber><![CDATA[6717981]]></arnumber>
##     <doi><![CDATA[10.1109/TPAMI.2014.25]]></doi>
##     <publicationId><![CDATA[6717981]]></publicationId>
##     <mdurl><![CDATA[http://ieeexplore.ieee.org/xpl/articleDetails.jsp?tp=&arnumber=6717981&contentType=Journals+%26+Magazines]]></mdurl>
##     <pdf><![CDATA[http://ieeexplore.ieee.org/stamp/stamp.jsp?arnumber=6717981]]></pdf>
##   </document>
##   <document>
##     <rank>23</rank>
##     <title><![CDATA[Numerical Surrogates for Human Observers in Myocardial Motion Evaluation From SPECT Images]]></title>
##     <authors><![CDATA[Marin, T.;  Kalayeh, M.M.;  Parages, F.M.;  Brankov, J.G.]]></authors>
##     <affiliations><![CDATA[Med. imaging Res. Center, Illinois Inst. of Technol., Chicago, IL, USA]]></affiliations>
##     <controlledterms>
##       <term><![CDATA[cardiology]]></term>
##       <term><![CDATA[feature extraction]]></term>
##       <term><![CDATA[haemodynamics]]></term>
##       <term><![CDATA[medical disorders]]></term>
##       <term><![CDATA[medical image processing]]></term>
##       <term><![CDATA[motion estimation]]></term>
##       <term><![CDATA[muscle]]></term>
##       <term><![CDATA[physiological models]]></term>
##       <term><![CDATA[regression analysis]]></term>
##       <term><![CDATA[single photon emission computed tomography]]></term>
##       <term><![CDATA[support vector machines]]></term>
##     </controlledterms>
##     <thesaurusterms>
##       <term><![CDATA[Feature extraction]]></term>
##       <term><![CDATA[Image sequences]]></term>
##       <term><![CDATA[Kernel]]></term>
##       <term><![CDATA[Myocardium]]></term>
##       <term><![CDATA[Observers]]></term>
##       <term><![CDATA[Predictive models]]></term>
##       <term><![CDATA[Single photon emission computed tomography]]></term>
##     </thesaurusterms>
##     <pubtitle><![CDATA[Medical Imaging, IEEE Transactions on]]></pubtitle>
##     <punumber><![CDATA[42]]></punumber>
##     <pubtype><![CDATA[Journals & Magazines]]></pubtype>
##     <publisher><![CDATA[IEEE]]></publisher>
##     <volume><![CDATA[33]]></volume>
##     <issue><![CDATA[1]]></issue>
##     <py><![CDATA[2014]]></py>
##     <spage><![CDATA[38]]></spage>
##     <epage><![CDATA[47]]></epage>
##     <abstract><![CDATA[In medical imaging, the gold standard for image-quality assessment is a task-based approach in which one evaluates human observer performance for a given diagnostic task (e.g., detection of a myocardial perfusion or motion defect). To facilitate practical task-based image-quality assessment, model observers are needed as approximate surrogates for human observers. In cardiac-gated SPECT imaging, diagnosis relies on evaluation of the myocardial motion as well as perfusion. Model observers for the perfusion-defect detection task have been studied previously, but little effort has been devoted toward development of a model observer for cardiac-motion defect detection. In this work, we describe two model observers for predicting human observer performance in detection of cardiac-motion defects. Both proposed methods rely on motion features extracted using previously reported deformable mesh model for myocardium motion estimation. The first method is based on a Hotelling linear discriminant that is similar in concept to that used commonly for perfusion-defect detection. In the second method, based on relevance vector machines (RVM) for regression, we compute average human observer performance by first directly predicting individual human observer scores, and then using multi reader receiver operating characteristic analysis. Our results suggest that the proposed RVM model observer can predict human observer performance accurately, while the new Hotelling motion-defect detector is somewhat less effective.]]></abstract>
##     <issn><![CDATA[0278-0062]]></issn>
##     <htmlFlag><![CDATA[1]]></htmlFlag>
##     <arnumber><![CDATA[6584807]]></arnumber>
##     <doi><![CDATA[10.1109/TMI.2013.2279517]]></doi>
##     <publicationId><![CDATA[6584807]]></publicationId>
##     <mdurl><![CDATA[http://ieeexplore.ieee.org/xpl/articleDetails.jsp?tp=&arnumber=6584807&contentType=Journals+%26+Magazines]]></mdurl>
##     <pdf><![CDATA[http://ieeexplore.ieee.org/stamp/stamp.jsp?arnumber=6584807]]></pdf>
##   </document>
##   <document>
##     <rank>24</rank>
##     <title><![CDATA[Scalable Nearest Neighbor Algorithms for High Dimensional Data]]></title>
##     <authors><![CDATA[Muja, M.;  Lowe, D.G.]]></authors>
##     <affiliations><![CDATA[BitLit Media Inc., Vancouver, BC, Canada]]></affiliations>
##     <controlledterms>
##       <term><![CDATA[computer vision]]></term>
##       <term><![CDATA[image matching]]></term>
##       <term><![CDATA[learning (artificial intelligence)]]></term>
##       <term><![CDATA[search problems]]></term>
##       <term><![CDATA[trees (mathematics)]]></term>
##     </controlledterms>
##     <thesaurusterms>
##       <term><![CDATA[Approximation algorithms]]></term>
##       <term><![CDATA[Approximation methods]]></term>
##       <term><![CDATA[Clustering algorithms]]></term>
##       <term><![CDATA[Computer vision]]></term>
##       <term><![CDATA[Machine learning algorithms]]></term>
##       <term><![CDATA[Partitioning algorithms]]></term>
##       <term><![CDATA[Vegetation]]></term>
##     </thesaurusterms>
##     <pubtitle><![CDATA[Pattern Analysis and Machine Intelligence, IEEE Transactions on]]></pubtitle>
##     <punumber><![CDATA[34]]></punumber>
##     <pubtype><![CDATA[Journals & Magazines]]></pubtype>
##     <publisher><![CDATA[IEEE]]></publisher>
##     <volume><![CDATA[36]]></volume>
##     <issue><![CDATA[11]]></issue>
##     <py><![CDATA[2014]]></py>
##     <spage><![CDATA[2227]]></spage>
##     <epage><![CDATA[2240]]></epage>
##     <abstract><![CDATA[For many computer vision and machine learning problems, large training sets are key for good performance. However, the most computationally expensive part of many computer vision and machine learning algorithms consists of finding nearest neighbor matches to high dimensional vectors that represent the training data. We propose new algorithms for approximate nearest neighbor matching and evaluate and compare them with previous algorithms. For matching high dimensional features, we find two algorithms to be the most efficient: the randomized k-d forest and a new algorithm proposed in this paper, the priority search k-means tree. We also propose a new algorithm for matching binary features by searching multiple hierarchical clustering trees and show it outperforms methods typically used in the literature. We show that the optimal nearest neighbor algorithm and its parameters depend on the data set characteristics and describe an automated configuration procedure for finding the best algorithm to search a particular data set. In order to scale to very large data sets that would otherwise not fit in the memory of a single machine, we propose a distributed nearest neighbor matching framework that can be used with any of the algorithms described in the paper. All this research has been released as an open source library called fast library for approximate nearest neighbors (FLANN), which has been incorporated into OpenCV and is now one of the most popular libraries for nearest neighbor matching.]]></abstract>
##     <issn><![CDATA[0162-8828]]></issn>
##     <htmlFlag><![CDATA[1]]></htmlFlag>
##     <arnumber><![CDATA[6809191]]></arnumber>
##     <doi><![CDATA[10.1109/TPAMI.2014.2321376]]></doi>
##     <publicationId><![CDATA[6809191]]></publicationId>
##     <mdurl><![CDATA[http://ieeexplore.ieee.org/xpl/articleDetails.jsp?tp=&arnumber=6809191&contentType=Journals+%26+Magazines]]></mdurl>
##     <pdf><![CDATA[http://ieeexplore.ieee.org/stamp/stamp.jsp?arnumber=6809191]]></pdf>
##   </document>
##   <document>
##     <rank>25</rank>
##     <title><![CDATA[Cell-Aware Test]]></title>
##     <authors><![CDATA[Hapke, F.;  Redemund, W.;  Glowatz, A.;  Rajski, J.;  Reese, M.;  Hustava, M.;  Keim, M.;  Schloeffel, J.;  Fast, A.]]></authors>
##     <affiliations><![CDATA[Mentor Graphics Dev. (Deutschland) GmbH, Hamburg, Germany]]></affiliations>
##     <controlledterms>
##       <term><![CDATA[CMOS integrated circuits]]></term>
##       <term><![CDATA[MOSFET]]></term>
##       <term><![CDATA[automatic test pattern generation]]></term>
##       <term><![CDATA[failure analysis]]></term>
##       <term><![CDATA[integrated circuit testing]]></term>
##     </controlledterms>
##     <thesaurusterms>
##       <term><![CDATA[Automatic test pattern generation]]></term>
##       <term><![CDATA[Bridge circuits]]></term>
##       <term><![CDATA[Layout]]></term>
##       <term><![CDATA[Libraries]]></term>
##       <term><![CDATA[Logic gates]]></term>
##       <term><![CDATA[Resistors]]></term>
##       <term><![CDATA[Transistors]]></term>
##     </thesaurusterms>
##     <pubtitle><![CDATA[Computer-Aided Design of Integrated Circuits and Systems, IEEE Transactions on]]></pubtitle>
##     <punumber><![CDATA[43]]></punumber>
##     <pubtype><![CDATA[Journals & Magazines]]></pubtype>
##     <publisher><![CDATA[IEEE]]></publisher>
##     <volume><![CDATA[33]]></volume>
##     <issue><![CDATA[9]]></issue>
##     <py><![CDATA[2014]]></py>
##     <spage><![CDATA[1396]]></spage>
##     <epage><![CDATA[1409]]></epage>
##     <abstract><![CDATA[This paper describes the new cell-aware test (CAT) approach, which enables a transistor-level and defect-based ATPG on full CMOS-based designs to significantly reduce the defect rate of manufactured ICs, including FinFET technologies. We present results from a defect-oriented CAT fault model generation for 1,940 standard library cells, as well as the application of CAT to several industrial designs. We present high volume production test results from a 32 nm notebook processor and from a 350 nm automotive design, including the achieved defect rate reduction in defective-parts-per-million. We also present CAT diagnosis and physical failure analysis results from one failing part and give an outlook for using the functionality for quickly ramping up the yield in advanced technology nodes.]]></abstract>
##     <issn><![CDATA[0278-0070]]></issn>
##     <htmlFlag><![CDATA[1]]></htmlFlag>
##     <arnumber><![CDATA[6879635]]></arnumber>
##     <doi><![CDATA[10.1109/TCAD.2014.2323216]]></doi>
##     <publicationId><![CDATA[6879635]]></publicationId>
##     <mdurl><![CDATA[http://ieeexplore.ieee.org/xpl/articleDetails.jsp?tp=&arnumber=6879635&contentType=Journals+%26+Magazines]]></mdurl>
##     <pdf><![CDATA[http://ieeexplore.ieee.org/stamp/stamp.jsp?arnumber=6879635]]></pdf>
##   </document>
##   <document>
##     <rank>26</rank>
##     <title><![CDATA[Smart Driving of a Vehicle Using Model Predictive Control for Improving Traffic Flow]]></title>
##     <authors><![CDATA[Samad Kamal, M.A.;  Imura, J.-I.;  Hayakawa, T.;  Ohata, A.;  Aihara, K.]]></authors>
##     <affiliations><![CDATA[Japan Sci. & Technol. Agency, Kawaguchi, Japan]]></affiliations>
##     <controlledterms>
##       <term><![CDATA[numerical analysis]]></term>
##       <term><![CDATA[predictive control]]></term>
##       <term><![CDATA[road traffic control]]></term>
##     </controlledterms>
##     <thesaurusterms>
##       <term><![CDATA[Acceleration]]></term>
##       <term><![CDATA[Delays]]></term>
##       <term><![CDATA[Jamming]]></term>
##       <term><![CDATA[Roads]]></term>
##       <term><![CDATA[Trajectory]]></term>
##       <term><![CDATA[Vehicle dynamics]]></term>
##       <term><![CDATA[Vehicles]]></term>
##     </thesaurusterms>
##     <pubtitle><![CDATA[Intelligent Transportation Systems, IEEE Transactions on]]></pubtitle>
##     <punumber><![CDATA[6979]]></punumber>
##     <pubtype><![CDATA[Journals & Magazines]]></pubtype>
##     <publisher><![CDATA[IEEE]]></publisher>
##     <volume><![CDATA[15]]></volume>
##     <issue><![CDATA[2]]></issue>
##     <py><![CDATA[2014]]></py>
##     <spage><![CDATA[878]]></spage>
##     <epage><![CDATA[888]]></epage>
##     <abstract><![CDATA[Traffic management on road networks is an emerging research field in control engineering due to the strong demand to alleviate traffic congestion in urban areas. Interaction among vehicles frequently causes congestion as well as bottlenecks in road capacity. In dense traffic, waves of traffic density propagate backward as drivers try to keep safe distances through frequent acceleration and deceleration. This paper presents a vehicle driving system in a model predictive control framework that effectively improves traffic flow. The vehicle driving system regulates safe intervehicle distance under the bounded driving torque condition by predicting the preceding traffic. It also focuses on alleviating the effect of braking on the vehicles that follow, which helps jamming waves attenuate to in the traffic. The proposed vehicle driving system has been evaluated through numerical simulation in dense traffic.]]></abstract>
##     <issn><![CDATA[1524-9050]]></issn>
##     <arnumber><![CDATA[6698313]]></arnumber>
##     <doi><![CDATA[10.1109/TITS.2013.2292500]]></doi>
##     <publicationId><![CDATA[6698313]]></publicationId>
##     <mdurl><![CDATA[http://ieeexplore.ieee.org/xpl/articleDetails.jsp?tp=&arnumber=6698313&contentType=Journals+%26+Magazines]]></mdurl>
##     <pdf><![CDATA[http://ieeexplore.ieee.org/stamp/stamp.jsp?arnumber=6698313]]></pdf>
##   </document>
##   <document>
##     <rank>27</rank>
##     <title><![CDATA[Capability Assessment of Fully Polarimetric ALOS&#x2013;PALSAR Data for Discriminating Wet Snow From Other Scattering Types in Mountainous Regions]]></title>
##     <authors><![CDATA[Singh, G.;  Venkataraman, G.;  Yamaguchi, Y.;  Sang-Eun Park]]></authors>
##     <affiliations><![CDATA[Grad. Sch. of Sci. & Technol., Niigata Univ., Niigata, Japan]]></affiliations>
##     <controlledterms>
##       <term><![CDATA[glaciology]]></term>
##       <term><![CDATA[hydrological techniques]]></term>
##       <term><![CDATA[remote sensing by radar]]></term>
##       <term><![CDATA[snow]]></term>
##       <term><![CDATA[synthetic aperture radar]]></term>
##     </controlledterms>
##     <pubtitle><![CDATA[Geoscience and Remote Sensing, IEEE Transactions on]]></pubtitle>
##     <punumber><![CDATA[36]]></punumber>
##     <pubtype><![CDATA[Journals & Magazines]]></pubtype>
##     <publisher><![CDATA[IEEE]]></publisher>
##     <volume><![CDATA[52]]></volume>
##     <issue><![CDATA[2]]></issue>
##     <py><![CDATA[2014]]></py>
##     <spage><![CDATA[1177]]></spage>
##     <epage><![CDATA[1196]]></epage>
##     <abstract><![CDATA[This paper examines the capability assessment of fully polarimetric L-band data for the snow and nonsnow-area classifications. The data sets used are the fully polarimetric Advanced Land Observation Satellite-Phased Array-Type L-Band Synthetic Aperture Radar data, optical Advanced Land Observing Satellite (ALOS)-advanced visible and near-infrared radiometer-2 data close to the radar acquisition, and environmental satellite-advanced synthetic aperture radar data. Several parameters are used to discriminate the snow-covered areas from nonsnow-covered areas in the Indian Himalayan region, including backscattering coefficients, the ratio of cross/copolarized backscattering power and polarization fraction (PF) value. Supervised classification schemes are employed using polarimetric decomposition methods based on the complex Wishart classifier. The accuracy of the classification was found to be 97.95% for the Wishart-supervised classification. Among various parameters and methods, it was found that the alternative newly proposed PF scheme, based on the implementation of fully polarimetric synthetic aperture radar data, yielded the best classification result in the absence of the training samples. The PF value has been effective for discrimination of the snow-covered areas from nonsnow-covered areas, debris-covered glacier, and vegetation. The results of this investigation show that L-band fully polarimetric SAR data provide considerable improvement but may not possess the optimal capability to discriminate snow from other inherent natural and man-made scatterers in heavy snow-laden mountainous scenarios, which may require fully polarimetric S-band or C-band PolSAR measurements.]]></abstract>
##     <issn><![CDATA[0196-2892]]></issn>
##     <htmlFlag><![CDATA[1]]></htmlFlag>
##     <arnumber><![CDATA[6512585]]></arnumber>
##     <doi><![CDATA[10.1109/TGRS.2013.2248369]]></doi>
##     <publicationId><![CDATA[6512585]]></publicationId>
##     <mdurl><![CDATA[http://ieeexplore.ieee.org/xpl/articleDetails.jsp?tp=&arnumber=6512585&contentType=Journals+%26+Magazines]]></mdurl>
##     <pdf><![CDATA[http://ieeexplore.ieee.org/stamp/stamp.jsp?arnumber=6512585]]></pdf>
##   </document>
##   <document>
##     <rank>28</rank>
##     <title><![CDATA[Adaptive Control for Laser Transmitter Feedforward Linearization System]]></title>
##     <authors><![CDATA[Neo, Y.S.;  Idrus, S.M.;  Rahmat, M.F.;  Alavi, S.E.;  Amiri, I.S.]]></authors>
##     <affiliations><![CDATA[PhotoJohornic Res. Lab., Univ. Teknol. Malaysia, Skudai, Malaysia]]></affiliations>
##     <controlledterms>
##       <term><![CDATA[adaptive control]]></term>
##       <term><![CDATA[linearisation techniques]]></term>
##       <term><![CDATA[optical modulation]]></term>
##       <term><![CDATA[optical transmitters]]></term>
##       <term><![CDATA[optimisation]]></term>
##     </controlledterms>
##     <thesaurusterms>
##       <term><![CDATA[Adaptive control]]></term>
##       <term><![CDATA[Diode lasers]]></term>
##       <term><![CDATA[Equations]]></term>
##       <term><![CDATA[Feedforward neural networks]]></term>
##       <term><![CDATA[Linear programming]]></term>
##       <term><![CDATA[Mathematical model]]></term>
##       <term><![CDATA[Semiconductor lasers]]></term>
##     </thesaurusterms>
##     <pubtitle><![CDATA[Photonics Journal, IEEE]]></pubtitle>
##     <punumber><![CDATA[4563994]]></punumber>
##     <pubtype><![CDATA[Journals & Magazines]]></pubtype>
##     <publisher><![CDATA[IEEE]]></publisher>
##     <volume><![CDATA[6]]></volume>
##     <issue><![CDATA[4]]></issue>
##     <py><![CDATA[2014]]></py>
##     <spage><![CDATA[1]]></spage>
##     <epage><![CDATA[10]]></epage>
##     <abstract><![CDATA[In this paper, an adaptive control system is employed in a novel implementation technique of the feed forward linearization system for optical analog communication systems' laser transmitter. The adaptive control system applies the Newton trust-region dogleg algorithm, which is a numerical optimization algorithm, to automatically tune the adjustment parameters in the feedforward loops to optimize the feed forward system performance and adapt to process variations. At the end of this paper, significant reductions of over 20 dBm in the third-order inter modulation distortion products have been achieved for operating frequencies from 5.0 to 5.8 GHz.]]></abstract>
##     <issn><![CDATA[1943-0655]]></issn>
##     <htmlFlag><![CDATA[1]]></htmlFlag>
##     <arnumber><![CDATA[6848805]]></arnumber>
##     <doi><![CDATA[10.1109/JPHOT.2014.2335711]]></doi>
##     <publicationId><![CDATA[6848805]]></publicationId>
##     <mdurl><![CDATA[http://ieeexplore.ieee.org/xpl/articleDetails.jsp?tp=&arnumber=6848805&contentType=Journals+%26+Magazines]]></mdurl>
##     <pdf><![CDATA[http://ieeexplore.ieee.org/stamp/stamp.jsp?arnumber=6848805]]></pdf>
##   </document>
##   <document>
##     <rank>29</rank>
##     <title><![CDATA[Predicting Outcomes of Nonsmall Cell Lung Cancer Using CT Image Features]]></title>
##     <authors><![CDATA[Hawkins, S.H.;  Korecki, J.N.;  Balagurunathan, Y.;  Yuhua Gu;  Kumar, V.;  Basu, S.;  Hall, L.O.;  Goldgof, D.B.;  Gatenby, R.A.;  Gillies, R.J.]]></authors>
##     <affiliations><![CDATA[Dept. of Comput. Sci. & Eng., Univ. of South Florida, Tampa, FL, USA]]></affiliations>
##     <controlledterms>
##       <term><![CDATA[cancer]]></term>
##       <term><![CDATA[cellular biophysics]]></term>
##       <term><![CDATA[computerised tomography]]></term>
##       <term><![CDATA[diseases]]></term>
##       <term><![CDATA[feature selection]]></term>
##       <term><![CDATA[image classification]]></term>
##       <term><![CDATA[lung]]></term>
##       <term><![CDATA[medical image processing]]></term>
##     </controlledterms>
##     <thesaurusterms>
##       <term><![CDATA[Biomedical image processing]]></term>
##       <term><![CDATA[Cancer]]></term>
##       <term><![CDATA[Cells (biology)]]></term>
##       <term><![CDATA[Computed tomography]]></term>
##       <term><![CDATA[Diseases]]></term>
##       <term><![CDATA[Lung cancer]]></term>
##       <term><![CDATA[Support vector machines]]></term>
##       <term><![CDATA[Three-dimensional displays]]></term>
##       <term><![CDATA[Tumors]]></term>
##     </thesaurusterms>
##     <pubtitle><![CDATA[Access, IEEE]]></pubtitle>
##     <punumber><![CDATA[6287639]]></punumber>
##     <pubtype><![CDATA[Journals & Magazines]]></pubtype>
##     <publisher><![CDATA[IEEE]]></publisher>
##     <volume><![CDATA[2]]></volume>
##     <py><![CDATA[2014]]></py>
##     <spage><![CDATA[1418]]></spage>
##     <epage><![CDATA[1426]]></epage>
##     <abstract><![CDATA[Nonsmall cell lung cancer is a prevalent disease. It is diagnosed and treated with the help of computed tomography (CT) scans. In this paper, we apply radiomics to select 3-D features from CT images of the lung toward providing prognostic information. Focusing on cases of the adenocarcinoma nonsmall cell lung cancer tumor subtype from a larger data set, we show that classifiers can be built to predict survival time. This is the first known result to make such predictions from CT scans of lung cancer. We compare classifiers and feature selection approaches. The best accuracy when predicting survival was 77.5% using a decision tree in a leave-one-out cross validation and was obtained after selecting five features per fold from 219.]]></abstract>
##     <issn><![CDATA[2169-3536]]></issn>
##     <htmlFlag><![CDATA[1]]></htmlFlag>
##     <arnumber><![CDATA[6966732]]></arnumber>
##     <doi><![CDATA[10.1109/ACCESS.2014.2373335]]></doi>
##     <publicationId><![CDATA[6966732]]></publicationId>
##     <mdurl><![CDATA[http://ieeexplore.ieee.org/xpl/articleDetails.jsp?tp=&arnumber=6966732&contentType=Journals+%26+Magazines]]></mdurl>
##     <pdf><![CDATA[http://ieeexplore.ieee.org/stamp/stamp.jsp?arnumber=6966732]]></pdf>
##   </document>
##   <document>
##     <rank>30</rank>
##     <title><![CDATA[Internet of Things for Smart Cities]]></title>
##     <authors><![CDATA[Zanella, A.;  Bui, N.;  Castellani, A.;  Vangelista, L.;  Zorzi, M.]]></authors>
##     <affiliations><![CDATA[Dept. of Inf. Eng., Univ. of Padova, Padua, Italy]]></affiliations>
##     <controlledterms>
##       <term><![CDATA[Internet]]></term>
##       <term><![CDATA[Internet of Things]]></term>
##       <term><![CDATA[protocols]]></term>
##     </controlledterms>
##     <thesaurusterms>
##       <term><![CDATA[Business]]></term>
##       <term><![CDATA[Cities and towns]]></term>
##       <term><![CDATA[IEEE 802.15 Standards]]></term>
##       <term><![CDATA[Monitoring]]></term>
##       <term><![CDATA[Smart buildings]]></term>
##       <term><![CDATA[Smart homes]]></term>
##       <term><![CDATA[Urban areas]]></term>
##     </thesaurusterms>
##     <pubtitle><![CDATA[Internet of Things Journal, IEEE]]></pubtitle>
##     <punumber><![CDATA[6488907]]></punumber>
##     <pubtype><![CDATA[Journals & Magazines]]></pubtype>
##     <publisher><![CDATA[IEEE]]></publisher>
##     <volume><![CDATA[1]]></volume>
##     <issue><![CDATA[1]]></issue>
##     <py><![CDATA[2014]]></py>
##     <spage><![CDATA[22]]></spage>
##     <epage><![CDATA[32]]></epage>
##     <abstract><![CDATA[The Internet of Things (IoT) shall be able to incorporate transparently and seamlessly a large number of different and heterogeneous end systems, while providing open access to selected subsets of data for the development of a plethora of digital services. Building a general architecture for the IoT is hence a very complex task, mainly because of the extremely large variety of devices, link layer technologies, and services that may be involved in such a system. In this paper, we focus specifically to an urban IoT system that, while still being quite a broad category, are characterized by their specific application domain. Urban IoTs, in fact, are designed to support the Smart City vision, which aims at exploiting the most advanced communication technologies to support added-value services for the administration of the city and for the citizens. This paper hence provides a comprehensive survey of the enabling technologies, protocols, and architecture for an urban IoT. Furthermore, the paper will present and discuss the technical solutions and best-practice guidelines adopted in the Padova Smart City project, a proof-of-concept deployment of an IoT island in the city of Padova, Italy, performed in collaboration with the city municipality.]]></abstract>
##     <issn><![CDATA[2327-4662]]></issn>
##     <htmlFlag><![CDATA[1]]></htmlFlag>
##     <arnumber><![CDATA[6740844]]></arnumber>
##     <doi><![CDATA[10.1109/JIOT.2014.2306328]]></doi>
##     <publicationId><![CDATA[6740844]]></publicationId>
##     <mdurl><![CDATA[http://ieeexplore.ieee.org/xpl/articleDetails.jsp?tp=&arnumber=6740844&contentType=Journals+%26+Magazines]]></mdurl>
##     <pdf><![CDATA[http://ieeexplore.ieee.org/stamp/stamp.jsp?arnumber=6740844]]></pdf>
##   </document>
##   <document>
##     <rank>31</rank>
##     <title><![CDATA[A Simplified MIMO Channel Characteristics Evaluation Scheme Based on Ray Tracing and Its Application to Indoor Radio Systems]]></title>
##     <authors><![CDATA[Arikawa, S.;  Karasawa, Y.]]></authors>
##     <affiliations><![CDATA[Wireless Commun. Res. Center (AWCC), Univ. of Electro-Commun., Chofu, Japan]]></affiliations>
##     <controlledterms>
##       <term><![CDATA[MIMO communication]]></term>
##       <term><![CDATA[antenna arrays]]></term>
##       <term><![CDATA[indoor radio]]></term>
##       <term><![CDATA[radiowave propagation]]></term>
##       <term><![CDATA[ray tracing]]></term>
##       <term><![CDATA[receiving antennas]]></term>
##       <term><![CDATA[statistical analysis]]></term>
##       <term><![CDATA[transmitting antennas]]></term>
##     </controlledterms>
##     <thesaurusterms>
##       <term><![CDATA[Approximation methods]]></term>
##       <term><![CDATA[Arrays]]></term>
##       <term><![CDATA[MIMO]]></term>
##       <term><![CDATA[Ray tracing]]></term>
##       <term><![CDATA[Receiving antennas]]></term>
##       <term><![CDATA[Transmitting antennas]]></term>
##     </thesaurusterms>
##     <pubtitle><![CDATA[Antennas and Wireless Propagation Letters, IEEE]]></pubtitle>
##     <punumber><![CDATA[7727]]></punumber>
##     <pubtype><![CDATA[Journals & Magazines]]></pubtype>
##     <publisher><![CDATA[IEEE]]></publisher>
##     <volume><![CDATA[13]]></volume>
##     <py><![CDATA[2014]]></py>
##     <spage><![CDATA[1737]]></spage>
##     <epage><![CDATA[1740]]></epage>
##     <abstract><![CDATA[In this letter, we propose a method of propagation characteristics analysis that enables simplified assessment of the transmission characteristics of multiple-input-multiple-output (MIMO) communication in indoor environments and assess its accuracy. In the proposed method, ray-tracing analysis is combined with a statistical method based on antenna array principles. Ray-tracing analysis is a rigorous method, but it is highly demanding in terms of computing time. Accordingly, in our method, ray tracing is only implemented between a pair of transmitting and receiving antenna points that are used as reference, after which statistical analysis is applied to estimate the properties of the surrounding areas. The results of an environment analysis of an empty room containing no objects produced via our method were compared to a room analyzed by ray tracing alone, from which we found our simplified method to be highly accurate within a range of several wavelengths from the receiving reference points.]]></abstract>
##     <issn><![CDATA[1536-1225]]></issn>
##     <arnumber><![CDATA[6891152]]></arnumber>
##     <doi><![CDATA[10.1109/LAWP.2014.2353663]]></doi>
##     <publicationId><![CDATA[6891152]]></publicationId>
##     <mdurl><![CDATA[http://ieeexplore.ieee.org/xpl/articleDetails.jsp?tp=&arnumber=6891152&contentType=Journals+%26+Magazines]]></mdurl>
##     <pdf><![CDATA[http://ieeexplore.ieee.org/stamp/stamp.jsp?arnumber=6891152]]></pdf>
##   </document>
##   <document>
##     <rank>32</rank>
##     <title><![CDATA[Design and Analysis of a CO-OFDM Transmitter With Limited Modulator Extinction Ratio]]></title>
##     <authors><![CDATA[Lingchen Huang;  Changjian Guo;  Sailing He]]></authors>
##     <affiliations><![CDATA[ZJU-SCNU Joint Res. Center of Photonics, South China Normal Univ., Guangzhou, China]]></affiliations>
##     <controlledterms>
##       <term><![CDATA[OFDM modulation]]></term>
##       <term><![CDATA[Q-factor]]></term>
##       <term><![CDATA[digital-analogue conversion]]></term>
##       <term><![CDATA[light coherence]]></term>
##       <term><![CDATA[optical transmitters]]></term>
##       <term><![CDATA[signal processing]]></term>
##     </controlledterms>
##     <thesaurusterms>
##       <term><![CDATA[Erbium]]></term>
##       <term><![CDATA[Optical distortion]]></term>
##       <term><![CDATA[Optical fibers]]></term>
##       <term><![CDATA[Optical modulation]]></term>
##       <term><![CDATA[Optical transmitters]]></term>
##     </thesaurusterms>
##     <pubtitle><![CDATA[Photonics Journal, IEEE]]></pubtitle>
##     <punumber><![CDATA[4563994]]></punumber>
##     <pubtype><![CDATA[Journals & Magazines]]></pubtype>
##     <publisher><![CDATA[IEEE]]></publisher>
##     <volume><![CDATA[6]]></volume>
##     <issue><![CDATA[3]]></issue>
##     <py><![CDATA[2014]]></py>
##     <spage><![CDATA[1]]></spage>
##     <epage><![CDATA[7]]></epage>
##     <abstract><![CDATA[In this paper, we consider the design and analysis of a coherent optical OFDM (CO-OFDM) transmitter using an optical in-phase/quadrature (I/Q) modulator with limited extinction ratio (ER) under four modulation formats. The transmitter-side digital signal processing (DSP), including hard clipping and predistortion, is introduced to combat the penalty induced by the limited ER of the modulator and the quantization noise of the digital-to-analog converter (DAC). The DAC resolution needed for transmitters using modulators with various ER values are examined. The peak-to-peak driving voltage (Vpp) and the clipping ratio (CR) of the transmitted signals are also optimized with respect to minimizing the Q-factor penalty for the modulator with different ER values. The simulation results show that, unlike previous work using modulators with an infinite ER, the optimum Vpp and CR values vary with different ERs and modulation levels.]]></abstract>
##     <issn><![CDATA[1943-0655]]></issn>
##     <htmlFlag><![CDATA[1]]></htmlFlag>
##     <arnumber><![CDATA[6798749]]></arnumber>
##     <doi><![CDATA[10.1109/JPHOT.2014.2317676]]></doi>
##     <publicationId><![CDATA[6798749]]></publicationId>
##     <mdurl><![CDATA[http://ieeexplore.ieee.org/xpl/articleDetails.jsp?tp=&arnumber=6798749&contentType=Journals+%26+Magazines]]></mdurl>
##     <pdf><![CDATA[http://ieeexplore.ieee.org/stamp/stamp.jsp?arnumber=6798749]]></pdf>
##   </document>
##   <document>
##     <rank>33</rank>
##     <title><![CDATA[Optimization-Based Islanding of Power Networks Using Piecewise Linear AC Power Flow]]></title>
##     <authors><![CDATA[Trodden, P.A.;  Bukhsh, W.A.;  Grothey, A.;  McKinnon, K.I.M.]]></authors>
##     <affiliations><![CDATA[Dept. of Autom. Control & Syst. Eng., Univ. of Sheffield, Sheffield, UK]]></affiliations>
##     <controlledterms>
##       <term><![CDATA[AC generators]]></term>
##       <term><![CDATA[distributed power generation]]></term>
##       <term><![CDATA[distribution networks]]></term>
##       <term><![CDATA[load flow]]></term>
##       <term><![CDATA[load shedding]]></term>
##       <term><![CDATA[piecewise linear techniques]]></term>
##       <term><![CDATA[reactive power]]></term>
##     </controlledterms>
##     <thesaurusterms>
##       <term><![CDATA[Equations]]></term>
##       <term><![CDATA[Generators]]></term>
##       <term><![CDATA[Load flow]]></term>
##       <term><![CDATA[Mathematical model]]></term>
##       <term><![CDATA[Piecewise linear approximation]]></term>
##       <term><![CDATA[Reactive power]]></term>
##       <term><![CDATA[Switches]]></term>
##     </thesaurusterms>
##     <pubtitle><![CDATA[Power Systems, IEEE Transactions on]]></pubtitle>
##     <punumber><![CDATA[59]]></punumber>
##     <pubtype><![CDATA[Journals & Magazines]]></pubtype>
##     <publisher><![CDATA[IEEE]]></publisher>
##     <volume><![CDATA[29]]></volume>
##     <issue><![CDATA[3]]></issue>
##     <py><![CDATA[2014]]></py>
##     <spage><![CDATA[1212]]></spage>
##     <epage><![CDATA[1220]]></epage>
##     <abstract><![CDATA[In this paper, a flexible optimization-based framework for intentional islanding is presented. The decision is made of which transmission lines to switch in order to split the network while minimizing disruption, the amount of load shed, or grouping coherent generators. The approach uses a piecewise linear model of AC power flow, which allows the voltage and reactive power to be considered directly when designing the islands. Demonstrations on standard test networks show that solution of the problem provides islands that are balanced in real and reactive power, satisfy AC power flow laws, and have a healthy voltage profile.]]></abstract>
##     <issn><![CDATA[0885-8950]]></issn>
##     <htmlFlag><![CDATA[1]]></htmlFlag>
##     <arnumber><![CDATA[6687285]]></arnumber>
##     <doi><![CDATA[10.1109/TPWRS.2013.2291660]]></doi>
##     <publicationId><![CDATA[6687285]]></publicationId>
##     <mdurl><![CDATA[http://ieeexplore.ieee.org/xpl/articleDetails.jsp?tp=&arnumber=6687285&contentType=Journals+%26+Magazines]]></mdurl>
##     <pdf><![CDATA[http://ieeexplore.ieee.org/stamp/stamp.jsp?arnumber=6687285]]></pdf>
##   </document>
##   <document>
##     <rank>34</rank>
##     <title><![CDATA[Development of mHealth Applications for Pre-Eclampsia Triage]]></title>
##     <authors><![CDATA[Dunsmuir, D.T.;  Payne, B.A.;  Cloete, G.;  Petersen, C.L.;  Gorges, M.;  Lim, J.;  von Dadelszen, P.;  Dumont, G.A.;  Ansermino, J.M.]]></authors>
##     <affiliations><![CDATA[Dept. of Anesthesiology, Pharmacology & Therapeutics, Univ. of British Columbia, Vancouver, BC, Canada]]></affiliations>
##     <controlledterms>
##       <term><![CDATA[biomedical telemetry]]></term>
##       <term><![CDATA[electronic health records]]></term>
##       <term><![CDATA[health care]]></term>
##       <term><![CDATA[medical computing]]></term>
##       <term><![CDATA[mobile computing]]></term>
##       <term><![CDATA[obstetrics]]></term>
##       <term><![CDATA[oximetry]]></term>
##       <term><![CDATA[patient diagnosis]]></term>
##       <term><![CDATA[patient treatment]]></term>
##       <term><![CDATA[smart phones]]></term>
##     </controlledterms>
##     <thesaurusterms>
##       <term><![CDATA[Data collection]]></term>
##       <term><![CDATA[Decision trees]]></term>
##       <term><![CDATA[Mobile handsets]]></term>
##       <term><![CDATA[Predictive models]]></term>
##       <term><![CDATA[Pregnancy]]></term>
##       <term><![CDATA[Protocols]]></term>
##       <term><![CDATA[Risk management]]></term>
##     </thesaurusterms>
##     <pubtitle><![CDATA[Biomedical and Health Informatics, IEEE Journal of]]></pubtitle>
##     <punumber><![CDATA[6221020]]></punumber>
##     <pubtype><![CDATA[Journals & Magazines]]></pubtype>
##     <publisher><![CDATA[IEEE]]></publisher>
##     <volume><![CDATA[18]]></volume>
##     <issue><![CDATA[6]]></issue>
##     <py><![CDATA[2014]]></py>
##     <spage><![CDATA[1857]]></spage>
##     <epage><![CDATA[1864]]></epage>
##     <abstract><![CDATA[The development of mobile applications for the diagnosis and management of pregnant women with pre-eclampsia is described. These applications are designed for use by community-based health care providers (c-HCPs) in health facilities and during home visits to collect symptoms and perform clinical measurements (including pulse oximeter readings). The clinical data collected in women with pre-eclampsia are used as the inputs to a predictive model providing a risk score for the development of adverse outcomes. Based on this risk, the applications provide recommendations on treatment, referral, and reassessment. c-HCPs can access patient records across multiple visits, using multiple devices that are synchronized using a secure Research Electronic Data Capture server. A unique feature of these applications is the ability to measure oxygen saturation with a pulse oximeter connected to a smartphone (Phone Oximeter). The mobile health application development process, including challenges encountered and solutions are described.]]></abstract>
##     <issn><![CDATA[2168-2194]]></issn>
##     <htmlFlag><![CDATA[1]]></htmlFlag>
##     <arnumber><![CDATA[6716019]]></arnumber>
##     <doi><![CDATA[10.1109/JBHI.2014.2301156]]></doi>
##     <publicationId><![CDATA[6716019]]></publicationId>
##     <mdurl><![CDATA[http://ieeexplore.ieee.org/xpl/articleDetails.jsp?tp=&arnumber=6716019&contentType=Journals+%26+Magazines]]></mdurl>
##     <pdf><![CDATA[http://ieeexplore.ieee.org/stamp/stamp.jsp?arnumber=6716019]]></pdf>
##   </document>
##   <document>
##     <rank>35</rank>
##     <title><![CDATA[Knowledge Acquisition Method Based on Singular Value Decomposition for Human Motion Analysis]]></title>
##     <authors><![CDATA[Yinlai Jiang;  Hayashi, I.;  Shuoyu Wang]]></authors>
##     <affiliations><![CDATA[Res. Inst., Kochi Univ. of Technol., Kami, Japan]]></affiliations>
##     <controlledterms>
##       <term><![CDATA[gait analysis]]></term>
##       <term><![CDATA[gesture recognition]]></term>
##       <term><![CDATA[knowledge acquisition]]></term>
##       <term><![CDATA[motion estimation]]></term>
##       <term><![CDATA[singular value decomposition]]></term>
##       <term><![CDATA[time series]]></term>
##     </controlledterms>
##     <thesaurusterms>
##       <term><![CDATA[Biological system modeling]]></term>
##       <term><![CDATA[Data mining]]></term>
##       <term><![CDATA[Estimation]]></term>
##       <term><![CDATA[Gesture recognition]]></term>
##       <term><![CDATA[Hidden Markov models]]></term>
##       <term><![CDATA[Matrix decomposition]]></term>
##     </thesaurusterms>
##     <pubtitle><![CDATA[Knowledge and Data Engineering, IEEE Transactions on]]></pubtitle>
##     <punumber><![CDATA[69]]></punumber>
##     <pubtype><![CDATA[Journals & Magazines]]></pubtype>
##     <publisher><![CDATA[IEEE]]></publisher>
##     <volume><![CDATA[26]]></volume>
##     <issue><![CDATA[12]]></issue>
##     <py><![CDATA[2014]]></py>
##     <spage><![CDATA[3038]]></spage>
##     <epage><![CDATA[3050]]></epage>
##     <abstract><![CDATA[The knowledge remembered by the human body and reflected by the dexterity of body motion is called embodied knowledge. In this paper, we propose a new method using singular value decomposition for extracting embodied knowledge from the time-series data of the motion. We compose a matrix from the time-series data and use the left singular vectors of the matrix as the patterns of the motion and the singular values as a scalar, by which each corresponding left singular vector affects the matrix. Two experiments were conducted to validate the method. One is a gesture recognition experiment in which we categorize gesture motions by two kinds of models with indexes of similarity and estimation that use left singular vectors. The proposed method obtained a higher correct categorization ratio than principal component analysis (PCA) and correlation efficiency (CE). The other is an ambulation evaluation experiment in which we distinguished the levels of walking disability. The first singular values derived from the walking acceleration were suggested to be a reliable criterion to evaluate walking disability. Finally we discuss the characteristic and significance of the embodied knowledge extraction using the singular value decomposition proposed in this paper.]]></abstract>
##     <issn><![CDATA[1041-4347]]></issn>
##     <htmlFlag><![CDATA[1]]></htmlFlag>
##     <arnumber><![CDATA[6786494]]></arnumber>
##     <doi><![CDATA[10.1109/TKDE.2014.2316521]]></doi>
##     <publicationId><![CDATA[6786494]]></publicationId>
##     <mdurl><![CDATA[http://ieeexplore.ieee.org/xpl/articleDetails.jsp?tp=&arnumber=6786494&contentType=Journals+%26+Magazines]]></mdurl>
##     <pdf><![CDATA[http://ieeexplore.ieee.org/stamp/stamp.jsp?arnumber=6786494]]></pdf>
##   </document>
##   <document>
##     <rank>36</rank>
##     <title><![CDATA[Cloud Mobile Media: Reflections and Outlook]]></title>
##     <authors><![CDATA[Yonggang Wen;  Xiaoqing Zhu;  Rodrigues, J.J.P.C.;  Chang Wen Chen]]></authors>
##     <affiliations><![CDATA[Sch. of Comput. Eng., Nanyang Technol. Univ., Singapore, Singapore]]></affiliations>
##     <controlledterms>
##       <term><![CDATA[cloud computing]]></term>
##       <term><![CDATA[mobile computing]]></term>
##       <term><![CDATA[multimedia computing]]></term>
##     </controlledterms>
##     <thesaurusterms>
##       <term><![CDATA[Cloud computing]]></term>
##       <term><![CDATA[Logic gates]]></term>
##       <term><![CDATA[Media]]></term>
##       <term><![CDATA[Mobile communication]]></term>
##       <term><![CDATA[Mobile computing]]></term>
##       <term><![CDATA[Mobile handsets]]></term>
##       <term><![CDATA[Wireless communication]]></term>
##     </thesaurusterms>
##     <pubtitle><![CDATA[Multimedia, IEEE Transactions on]]></pubtitle>
##     <punumber><![CDATA[6046]]></punumber>
##     <pubtype><![CDATA[Journals & Magazines]]></pubtype>
##     <publisher><![CDATA[IEEE]]></publisher>
##     <volume><![CDATA[16]]></volume>
##     <issue><![CDATA[4]]></issue>
##     <py><![CDATA[2014]]></py>
##     <spage><![CDATA[885]]></spage>
##     <epage><![CDATA[902]]></epage>
##     <abstract><![CDATA[This paper surveys the emerging paradigm of cloud mobile media. We start with two alternative perspectives for cloud mobile media networks: an end-to-end view and a layered view. Summaries of existing research in this area are organized according to the layered service framework: i) cloud resource management and control in infrastructure-as-a-service (IaaS), ii) cloud-based media services in platform-as-a-service (PaaS), and iii) novel cloud-based systems and applications in software-as-a-service (SaaS). We further substantiate our proposed design principles for cloud-based mobile media using a concrete case study: a cloud-centric media platform (CCMP) developed at Nanyang Technological University. Finally, this paper concludes with an outlook of open research problems for realizing the vision of cloud-based mobile media.]]></abstract>
##     <issn><![CDATA[1520-9210]]></issn>
##     <htmlFlag><![CDATA[1]]></htmlFlag>
##     <arnumber><![CDATA[6782722]]></arnumber>
##     <doi><![CDATA[10.1109/TMM.2014.2315596]]></doi>
##     <publicationId><![CDATA[6782722]]></publicationId>
##     <mdurl><![CDATA[http://ieeexplore.ieee.org/xpl/articleDetails.jsp?tp=&arnumber=6782722&contentType=Journals+%26+Magazines]]></mdurl>
##     <pdf><![CDATA[http://ieeexplore.ieee.org/stamp/stamp.jsp?arnumber=6782722]]></pdf>
##   </document>
##   <document>
##     <rank>37</rank>
##     <title><![CDATA[Performance Investigation of IM/DD Compatible SSB-OFDM Systems Based on Optical Multicarrier Sources]]></title>
##     <authors><![CDATA[Vujicic, V.;  Anandarajah, P.M.;  Zhou, R.;  Browning, C.;  Barry, L.P.]]></authors>
##     <affiliations><![CDATA[Sch. of Electron. Eng., Dublin City Univ., Dublin, Ireland]]></affiliations>
##     <controlledterms>
##       <term><![CDATA[OFDM modulation]]></term>
##       <term><![CDATA[optical communication]]></term>
##       <term><![CDATA[wavelength division multiplexing]]></term>
##     </controlledterms>
##     <thesaurusterms>
##       <term><![CDATA[OFDM]]></term>
##       <term><![CDATA[Optical fibers]]></term>
##       <term><![CDATA[Optical filters]]></term>
##       <term><![CDATA[Optical modulation]]></term>
##       <term><![CDATA[Optical polarization]]></term>
##       <term><![CDATA[Optical receivers]]></term>
##     </thesaurusterms>
##     <pubtitle><![CDATA[Photonics Journal, IEEE]]></pubtitle>
##     <punumber><![CDATA[4563994]]></punumber>
##     <pubtype><![CDATA[Journals & Magazines]]></pubtype>
##     <publisher><![CDATA[IEEE]]></publisher>
##     <volume><![CDATA[6]]></volume>
##     <issue><![CDATA[5]]></issue>
##     <py><![CDATA[2014]]></py>
##     <spage><![CDATA[1]]></spage>
##     <epage><![CDATA[10]]></epage>
##     <abstract><![CDATA[The authors investigate and compare the performance of three different optical frequency comb sources in an intensity modulated and directly detected compatible SSB OFDM system. The influence of relative intensity noise and carrier-to-noise ratio of the different multicarrier sources, on the performance of a 12.5 Gb/s compatible SSB OFDM system, were investigated experimentally and by simulation. The results obtained show that, system performance depends significantly on the relative intensity noise and carrier-to-noise ratio levels of the specific multicarrier optical source employed. However, it is shown that optical multicarrier sources with low relative intensity noise and high carrier-to-noise ratio levels can assure very good performance in a WDM-OFDM system.]]></abstract>
##     <issn><![CDATA[1943-0655]]></issn>
##     <htmlFlag><![CDATA[1]]></htmlFlag>
##     <arnumber><![CDATA[6917211]]></arnumber>
##     <doi><![CDATA[10.1109/JPHOT.2014.2361673]]></doi>
##     <publicationId><![CDATA[6917211]]></publicationId>
##     <mdurl><![CDATA[http://ieeexplore.ieee.org/xpl/articleDetails.jsp?tp=&arnumber=6917211&contentType=Journals+%26+Magazines]]></mdurl>
##     <pdf><![CDATA[http://ieeexplore.ieee.org/stamp/stamp.jsp?arnumber=6917211]]></pdf>
##   </document>
##   <document>
##     <rank>38</rank>
##     <title><![CDATA[Kilowatt Peak Power Pulses From a Passively Q-Switched Yb-Doped Fiber Laser With a Smaller-Core Yb-Doped Fiber as a Saturable Absorber]]></title>
##     <authors><![CDATA[Yi Lu;  Xijia Gu]]></authors>
##     <affiliations><![CDATA[Dept. of Electr. & Comput. Eng., Ryerson Univ., Toronto, ON, Canada]]></affiliations>
##     <controlledterms>
##       <term><![CDATA[Q-switching]]></term>
##       <term><![CDATA[fibre lasers]]></term>
##       <term><![CDATA[laser stability]]></term>
##       <term><![CDATA[optical pulse generation]]></term>
##       <term><![CDATA[optical saturable absorption]]></term>
##       <term><![CDATA[stimulated Raman scattering]]></term>
##       <term><![CDATA[ytterbium]]></term>
##     </controlledterms>
##     <thesaurusterms>
##       <term><![CDATA[Fiber lasers]]></term>
##       <term><![CDATA[Gain]]></term>
##       <term><![CDATA[Laser excitation]]></term>
##       <term><![CDATA[Optical fiber amplifiers]]></term>
##       <term><![CDATA[Optical fiber couplers]]></term>
##       <term><![CDATA[Power lasers]]></term>
##       <term><![CDATA[Pump lasers]]></term>
##     </thesaurusterms>
##     <pubtitle><![CDATA[Photonics Journal, IEEE]]></pubtitle>
##     <punumber><![CDATA[4563994]]></punumber>
##     <pubtype><![CDATA[Journals & Magazines]]></pubtype>
##     <publisher><![CDATA[IEEE]]></publisher>
##     <volume><![CDATA[6]]></volume>
##     <issue><![CDATA[3]]></issue>
##     <py><![CDATA[2014]]></py>
##     <spage><![CDATA[1]]></spage>
##     <epage><![CDATA[7]]></epage>
##     <abstract><![CDATA[We demonstrated a passively Q-switched Yb-doped fiber laser in an all-fiber configuration that used a piece of a Yb-doped fiber with a smaller core than the gain fiber as a saturable absorber. The laser generated stable output pulses at 1064 nm with a narrow line width of less than 0.2 nm. The Q-switched pulses have a pulse width of 140 ns and pulse energy of 141 &#x03BC;J at a repetition rate of 100 kHz. The peak power of ~ 1 kW and high slope efficiency of 51% were obtained. The repetition rate of this laser can be varied from a few kHz to 100 kHz with a potential of reaching up to 250 kHz. Stimulated Raman scattering was observed, though at 70 dB below the laser emission. The estimated stimulated Raman threshold is 6.2 kW, which allows this laser to further scale up the power. Because of its high peak power and adequate average power, the laser can be used as a stand-alone module for some applications in material processing. It can be also used as a seed laser for further power amplification.]]></abstract>
##     <issn><![CDATA[1943-0655]]></issn>
##     <htmlFlag><![CDATA[1]]></htmlFlag>
##     <arnumber><![CDATA[6809179]]></arnumber>
##     <doi><![CDATA[10.1109/JPHOT.2014.2320745]]></doi>
##     <publicationId><![CDATA[6809179]]></publicationId>
##     <mdurl><![CDATA[http://ieeexplore.ieee.org/xpl/articleDetails.jsp?tp=&arnumber=6809179&contentType=Journals+%26+Magazines]]></mdurl>
##     <pdf><![CDATA[http://ieeexplore.ieee.org/stamp/stamp.jsp?arnumber=6809179]]></pdf>
##   </document>
##   <document>
##     <rank>39</rank>
##     <title><![CDATA[Experimental Procedure for Determination of the Dielectric Properties of Biological Samples in the 2-50 GHz Range]]></title>
##     <authors><![CDATA[Odelstad, E.;  Raman, S.;  Rydberg, A.;  Augustine, R.]]></authors>
##     <affiliations><![CDATA[Dept. of Phys. & Astron., Uppsala Univ., Uppsala, Sweden]]></affiliations>
##     <controlledterms>
##       <term><![CDATA[Maxwell equations]]></term>
##       <term><![CDATA[bioelectric potentials]]></term>
##       <term><![CDATA[biomedical measurement]]></term>
##       <term><![CDATA[bone]]></term>
##       <term><![CDATA[cancer]]></term>
##       <term><![CDATA[cellular biophysics]]></term>
##       <term><![CDATA[high-frequency effects]]></term>
##       <term><![CDATA[muscle]]></term>
##       <term><![CDATA[permittivity]]></term>
##       <term><![CDATA[statistical analysis]]></term>
##       <term><![CDATA[suspensions]]></term>
##     </controlledterms>
##     <thesaurusterms>
##       <term><![CDATA[Dielectrics]]></term>
##       <term><![CDATA[Mathematical model]]></term>
##       <term><![CDATA[Permittivity]]></term>
##       <term><![CDATA[Permittivity measurement]]></term>
##       <term><![CDATA[Probes]]></term>
##       <term><![CDATA[Solids]]></term>
##       <term><![CDATA[Suspensions]]></term>
##     </thesaurusterms>
##     <pubtitle><![CDATA[Translational Engineering in Health and Medicine, IEEE Journal of]]></pubtitle>
##     <punumber><![CDATA[6221039]]></punumber>
##     <pubtype><![CDATA[Journals & Magazines]]></pubtype>
##     <publisher><![CDATA[IEEE]]></publisher>
##     <volume><![CDATA[2]]></volume>
##     <py><![CDATA[2014]]></py>
##     <spage><![CDATA[1]]></spage>
##     <epage><![CDATA[8]]></epage>
##     <abstract><![CDATA[The objective of this paper was to test and evaluate an experimental procedure for providing data on the complex permittivity of different cell lines in the 2-50-GHz range at room temperature, for the purpose of future dosimetric studies. The complex permittivity measurements were performed on cells suspended in culture medium using an open-ended coaxial probe. Maxwell's mixture equation then allows the calculation of the permittivity profiles of the cells from the difference in permittivity between the cell suspensions and pure culture medium. The open-ended coaxial probe turned out to be very sensitive to disturbances affecting the measurements, resulting in poor precision. Permittivity differences were not large in relation to the spread of the measurements and repeated measurements were performed to improve statistics. The 95% confidence intervals were computed for the arithmetic means of the measured permittivity differences in order to test the statistical significance. The results showed that for bone cells at the lowest tested concentration (33 500/ml), there were significance in the real part of the permittivity at frequencies above 30 GHz, and no significance in the imaginary part. For the second lowest concentration (67 000/ml) there was no significance at all. For a medium concentration of bone cells (135 000/ml) there was no significance in the real part, but there was significance in the imaginary part at frequencies below about 25 GHz. The cell suspension with a concentration of 1 350 000/ml had significance in the real part for both high (above 30 GHz) and low (below 15 GHz) frequencies. The imaginary part showed significance for frequencies below 25 GHz. In the case of an osteosarcoma cell line with a concentration of 2 700 000/ml, only the imaginary part showed significance, and only for frequencies below 15 GHz. For muscle cells at a concentration of 743 450/ml, there was only significance in the imaginary part for frequencies below 5 GHz. The e- perimental data indicated that the complex permittivity of the culture medium may be used for modeling of cell suspensions.]]></abstract>
##     <issn><![CDATA[2168-2372]]></issn>
##     <htmlFlag><![CDATA[1]]></htmlFlag>
##     <arnumber><![CDATA[6862841]]></arnumber>
##     <doi><![CDATA[10.1109/JTEHM.2014.2340412]]></doi>
##     <publicationId><![CDATA[6862841]]></publicationId>
##     <mdurl><![CDATA[http://ieeexplore.ieee.org/xpl/articleDetails.jsp?tp=&arnumber=6862841&contentType=Journals+%26+Magazines]]></mdurl>
##     <pdf><![CDATA[http://ieeexplore.ieee.org/stamp/stamp.jsp?arnumber=6862841]]></pdf>
##   </document>
##   <document>
##     <rank>40</rank>
##     <title><![CDATA[Oversampled Graph Laplacian Matrix for Graph Filter Banks]]></title>
##     <authors><![CDATA[Sakiyama, A.;  Tanaka, Y.]]></authors>
##     <affiliations><![CDATA[Grad. Sch. of BASE, Tokyo Univ. of Agric. & Technol., Koganei, Japan]]></affiliations>
##     <controlledterms>
##       <term><![CDATA[Laplace equations]]></term>
##       <term><![CDATA[channel bank filters]]></term>
##       <term><![CDATA[graph theory]]></term>
##       <term><![CDATA[matrix algebra]]></term>
##       <term><![CDATA[signal processing]]></term>
##     </controlledterms>
##     <thesaurusterms>
##       <term><![CDATA[Bipartite graph]]></term>
##       <term><![CDATA[Eigenvalues and eigenfunctions]]></term>
##       <term><![CDATA[Laplace equations]]></term>
##       <term><![CDATA[Licenses]]></term>
##       <term><![CDATA[Matrix decomposition]]></term>
##       <term><![CDATA[Signal processing]]></term>
##       <term><![CDATA[Transforms]]></term>
##     </thesaurusterms>
##     <pubtitle><![CDATA[Signal Processing, IEEE Transactions on]]></pubtitle>
##     <punumber><![CDATA[78]]></punumber>
##     <pubtype><![CDATA[Journals & Magazines]]></pubtype>
##     <publisher><![CDATA[IEEE]]></publisher>
##     <volume><![CDATA[62]]></volume>
##     <issue><![CDATA[24]]></issue>
##     <py><![CDATA[2014]]></py>
##     <spage><![CDATA[6425]]></spage>
##     <epage><![CDATA[6437]]></epage>
##     <abstract><![CDATA[We describe a method of oversampling signals defined on a weighted graph by using an oversampled graph Laplacian matrix. The conventional method of using critically sampled graph filter banks has to decompose the original graph into bipartite subgraphs, and a transform has to be performed on each subgraph because of the spectral folding phenomenon caused by downsampling of graph signals. Therefore, the conventional method cannot always utilize all edges of the original graph in a single stage transformation. Our method is based on oversampling of the underlying graph itself, and it can append nodes and edges to the graph somewhat arbitrarily. We use this approach to make one oversampled bipartite graph that includes all edges of the original non-bipartite graph. We apply the oversampled graph with the critically sampled graph filter bank or the oversampled one for decomposing graph signals and show the performances on some experiments.]]></abstract>
##     <issn><![CDATA[1053-587X]]></issn>
##     <htmlFlag><![CDATA[1]]></htmlFlag>
##     <arnumber><![CDATA[6937182]]></arnumber>
##     <doi><![CDATA[10.1109/TSP.2014.2365761]]></doi>
##     <publicationId><![CDATA[6937182]]></publicationId>
##     <mdurl><![CDATA[http://ieeexplore.ieee.org/xpl/articleDetails.jsp?tp=&arnumber=6937182&contentType=Journals+%26+Magazines]]></mdurl>
##     <pdf><![CDATA[http://ieeexplore.ieee.org/stamp/stamp.jsp?arnumber=6937182]]></pdf>
##   </document>
##   <document>
##     <rank>41</rank>
##     <title><![CDATA[Generation of Flat Optical Frequency Comb Using a Single Polarization Modulator and a Brillouin-Assisted Power Equalizer]]></title>
##     <authors><![CDATA[Wei Li;  Wen Ting Wang;  Wen Hui Sun;  Li Xian Wang;  Jian Guo Liu;  Ning Hua Zhu]]></authors>
##     <affiliations><![CDATA[State Key Lab. on Integrated Optoelectron., Inst. of Semicond., Beijing, China]]></affiliations>
##     <controlledterms>
##       <term><![CDATA[equalisers]]></term>
##       <term><![CDATA[microwave photonics]]></term>
##       <term><![CDATA[optical fibre polarisation]]></term>
##       <term><![CDATA[optical modulation]]></term>
##       <term><![CDATA[optoelectronic devices]]></term>
##       <term><![CDATA[oscillators]]></term>
##       <term><![CDATA[power apparatus]]></term>
##       <term><![CDATA[stimulated Brillouin scattering]]></term>
##     </controlledterms>
##     <thesaurusterms>
##       <term><![CDATA[Erbium-doped fiber amplifiers]]></term>
##       <term><![CDATA[Gain]]></term>
##       <term><![CDATA[Optical attenuators]]></term>
##       <term><![CDATA[Optical variables measurement]]></term>
##       <term><![CDATA[Radio frequency]]></term>
##       <term><![CDATA[Scattering]]></term>
##     </thesaurusterms>
##     <pubtitle><![CDATA[Photonics Journal, IEEE]]></pubtitle>
##     <punumber><![CDATA[4563994]]></punumber>
##     <pubtype><![CDATA[Journals & Magazines]]></pubtype>
##     <publisher><![CDATA[IEEE]]></publisher>
##     <volume><![CDATA[6]]></volume>
##     <issue><![CDATA[2]]></issue>
##     <py><![CDATA[2014]]></py>
##     <spage><![CDATA[1]]></spage>
##     <epage><![CDATA[8]]></epage>
##     <abstract><![CDATA[We propose a novel approach for the generation of flat optical frequency comb (OFC) using a single polarization modulator (PolM) and a Brillouin-assisted power equalizer (BAPE). The reference radio-frequency (RF) signal applied to the PolM is provided by an optoelectronic oscillator (OEO). The spectra of the optical signals launched to the BAPE are optimized in advance for the generation of OFCs with different number of lines. The BAPE is introduced to flatten the uneven OFC lines by attenuating the optical lines having power beyond the threshold of the stimulated Brillouin scattering (SBS) and amplifying the ones having power below the SBS threshold with the amplifier inside the BAPE. In addition, the OFC generator is optical wavelength independent since there is no optical filter involved in our approach. We experimentally demonstrated that the proposed OFC generator can produce 5, 7, 9, and 11 lines within a spectral flatness of 0.5, 1.5, 2, and 4.1 dB, respectively.]]></abstract>
##     <issn><![CDATA[1943-0655]]></issn>
##     <htmlFlag><![CDATA[1]]></htmlFlag>
##     <arnumber><![CDATA[6766253]]></arnumber>
##     <doi><![CDATA[10.1109/JPHOT.2014.2311455]]></doi>
##     <publicationId><![CDATA[6766253]]></publicationId>
##     <mdurl><![CDATA[http://ieeexplore.ieee.org/xpl/articleDetails.jsp?tp=&arnumber=6766253&contentType=Journals+%26+Magazines]]></mdurl>
##     <pdf><![CDATA[http://ieeexplore.ieee.org/stamp/stamp.jsp?arnumber=6766253]]></pdf>
##   </document>
##   <document>
##     <rank>42</rank>
##     <title><![CDATA[Topology Preserving Maps&#x2014;Extracting Layout Maps of Wireless Sensor Networks From Virtual Coordinates]]></title>
##     <authors><![CDATA[Dhanapala, D.C.;  Jayasumana, A.P.]]></authors>
##     <affiliations><![CDATA[Dept. of Electr. & Comput. Eng., Colorado State Univ., Fort Collins, CO, USA]]></affiliations>
##     <controlledterms>
##       <term><![CDATA[sensor placement]]></term>
##       <term><![CDATA[singular value decomposition]]></term>
##       <term><![CDATA[telecommunication network topology]]></term>
##       <term><![CDATA[wireless sensor networks]]></term>
##     </controlledterms>
##     <pubtitle><![CDATA[Networking, IEEE/ACM Transactions on]]></pubtitle>
##     <punumber><![CDATA[90]]></punumber>
##     <pubtype><![CDATA[Journals & Magazines]]></pubtype>
##     <publisher><![CDATA[IEEE]]></publisher>
##     <volume><![CDATA[22]]></volume>
##     <issue><![CDATA[3]]></issue>
##     <py><![CDATA[2014]]></py>
##     <spage><![CDATA[784]]></spage>
##     <epage><![CDATA[797]]></epage>
##     <abstract><![CDATA[A method for obtaining topology-preserving maps (TPMs) from virtual coordinates (VCs) of wireless sensor networks is presented. In a virtual coordinate system (VCS), a node is identified by a vector containing its distances, in hops, to a small subset of nodes called anchors. Layout information such as physical voids, shape, and even relative physical positions of sensor nodes with respect to x- y directions are absent in a VCS description. The proposed technique uses Singular Value Decomposition to isolate dominant radial information and to extract topological information from the VCS for networks deployed on 2-D/3-D surfaces and in 3-D volumes. The transformation required for TPM extraction can be generated using the coordinates of a subset of nodes, resulting in sensor-network-friendly implementation alternatives. TPMs of networks representing a variety of topologies are extracted. Topology preservation error ( ETP), a metric that accounts for both the number and degree of node flips, is defined and used to evaluate 2-D TPMs. The techniques extract TPMs with ( ETP) less than 2%. Topology coordinates provide an economical alternative to physical coordinates for many sensor networking algorithms.]]></abstract>
##     <issn><![CDATA[1063-6692]]></issn>
##     <htmlFlag><![CDATA[1]]></htmlFlag>
##     <arnumber><![CDATA[6542699]]></arnumber>
##     <doi><![CDATA[10.1109/TNET.2013.2263254]]></doi>
##     <publicationId><![CDATA[6542699]]></publicationId>
##     <mdurl><![CDATA[http://ieeexplore.ieee.org/xpl/articleDetails.jsp?tp=&arnumber=6542699&contentType=Journals+%26+Magazines]]></mdurl>
##     <pdf><![CDATA[http://ieeexplore.ieee.org/stamp/stamp.jsp?arnumber=6542699]]></pdf>
##   </document>
##   <document>
##     <rank>43</rank>
##     <title><![CDATA[Anomaly Detection in Wireless Sensor Networks in a Non-Stationary Environment]]></title>
##     <authors><![CDATA[O'Reilly, C.;  Gluhak, A.;  Imran, M.A.;  Rajasegarar, S.]]></authors>
##     <affiliations><![CDATA[Centre for Commun. Syst. Res., Univ. of Surrey, Guildford, UK]]></affiliations>
##     <controlledterms>
##       <term><![CDATA[data analysis]]></term>
##       <term><![CDATA[wireless sensor networks]]></term>
##     </controlledterms>
##     <thesaurusterms>
##       <term><![CDATA[Correlation]]></term>
##       <term><![CDATA[Data models]]></term>
##       <term><![CDATA[Distributed databases]]></term>
##       <term><![CDATA[Monitoring]]></term>
##       <term><![CDATA[Testing]]></term>
##       <term><![CDATA[Time measurement]]></term>
##       <term><![CDATA[Wireless sensor networks]]></term>
##     </thesaurusterms>
##     <pubtitle><![CDATA[Communications Surveys & Tutorials, IEEE]]></pubtitle>
##     <punumber><![CDATA[9739]]></punumber>
##     <pubtype><![CDATA[Journals & Magazines]]></pubtype>
##     <publisher><![CDATA[IEEE]]></publisher>
##     <volume><![CDATA[16]]></volume>
##     <issue><![CDATA[3]]></issue>
##     <py><![CDATA[2014]]></py>
##     <spage><![CDATA[1413]]></spage>
##     <epage><![CDATA[1432]]></epage>
##     <abstract><![CDATA[Anomaly detection in a WSN is an important aspect of data analysis in order to identify data items that significantly differ from normal data. A characteristic of the data generated by a WSN is that the data distribution may alter over the lifetime of the network due to the changing nature of the phenomenon being observed. Anomaly detection techniques must be able to adapt to a non-stationary data distribution in order to perform optimally. In this survey, we provide a comprehensive overview of approaches to anomaly detection in a WSN and their operation in a non-stationary environment.]]></abstract>
##     <issn><![CDATA[1553-877X]]></issn>
##     <htmlFlag><![CDATA[1]]></htmlFlag>
##     <arnumber><![CDATA[6710227]]></arnumber>
##     <doi><![CDATA[10.1109/SURV.2013.112813.00168]]></doi>
##     <publicationId><![CDATA[6710227]]></publicationId>
##     <mdurl><![CDATA[http://ieeexplore.ieee.org/xpl/articleDetails.jsp?tp=&arnumber=6710227&contentType=Journals+%26+Magazines]]></mdurl>
##     <pdf><![CDATA[http://ieeexplore.ieee.org/stamp/stamp.jsp?arnumber=6710227]]></pdf>
##   </document>
##   <document>
##     <rank>44</rank>
##     <title><![CDATA[Full-Wave Modeling of Near-Field Radar Data for Planar Layered Media Reconstruction]]></title>
##     <authors><![CDATA[Lambot, S.;  Andre, F.]]></authors>
##     <affiliations><![CDATA[Earth & Life Inst., Univ. Catholique de Louvain, Louvain-la-Neuve, Belgium]]></affiliations>
##     <controlledterms>
##       <term><![CDATA[Green's function methods]]></term>
##       <term><![CDATA[electromagnetic wave reflection]]></term>
##       <term><![CDATA[electromagnetic wave transmission]]></term>
##       <term><![CDATA[radar antennas]]></term>
##       <term><![CDATA[radar receivers]]></term>
##       <term><![CDATA[radar transmitters]]></term>
##       <term><![CDATA[ultra wideband antennas]]></term>
##       <term><![CDATA[ultra wideband radar]]></term>
##     </controlledterms>
##     <pubtitle><![CDATA[Geoscience and Remote Sensing, IEEE Transactions on]]></pubtitle>
##     <punumber><![CDATA[36]]></punumber>
##     <pubtype><![CDATA[Journals & Magazines]]></pubtype>
##     <publisher><![CDATA[IEEE]]></publisher>
##     <volume><![CDATA[52]]></volume>
##     <issue><![CDATA[5]]></issue>
##     <py><![CDATA[2014]]></py>
##     <spage><![CDATA[2295]]></spage>
##     <epage><![CDATA[2303]]></epage>
##     <abstract><![CDATA[A new near-field radar modeling approach for wave propagation in planar layered media is presented. The radar antennas are intrinsically modeled using an equivalent set of infinitesimal electric dipoles and characteristic, frequency-dependent, global reflection, and transmission coefficients. These coefficients determine through a plane wave decomposition wave propagation between the radar reference plane, point sources, and field points. The interactions between the antenna and layered medium are thereby inherently accounted for. The fields are calculated using 3-D Green's functions. We validated the model using an ultrawideband frequency-domain radar with a transmitting and receiving Vivaldi antenna operating in the range 0.8-4 GHz. The antenna characteristic coefficients are obtained from near- and far-field measurements over a copper plane. The proposed model provides unprecedented accuracy for describing near-field radar measurements collected over a water layer, the frequency-dependent electrical properties of which were described using the Debye model. Layer thicknesses could be retrieved through full-wave inversion. The proposed approach demonstrated great promise for nondestructive testing of planar materials and digital soil mapping using ground-penetrating radar.]]></abstract>
##     <issn><![CDATA[0196-2892]]></issn>
##     <htmlFlag><![CDATA[1]]></htmlFlag>
##     <arnumber><![CDATA[6520885]]></arnumber>
##     <doi><![CDATA[10.1109/TGRS.2013.2259243]]></doi>
##     <publicationId><![CDATA[6520885]]></publicationId>
##     <mdurl><![CDATA[http://ieeexplore.ieee.org/xpl/articleDetails.jsp?tp=&arnumber=6520885&contentType=Journals+%26+Magazines]]></mdurl>
##     <pdf><![CDATA[http://ieeexplore.ieee.org/stamp/stamp.jsp?arnumber=6520885]]></pdf>
##   </document>
##   <document>
##     <rank>45</rank>
##     <title><![CDATA[An Integrated Analog Readout for Multi-Frequency Bioimpedance Measurements]]></title>
##     <authors><![CDATA[Kassanos, P.;  Constantinou, L.;  Triantis, I.F.;  Demosthenous, A.]]></authors>
##     <affiliations><![CDATA[Dept. of Electron. & Electr. Eng., Univ. Coll. London, London, UK]]></affiliations>
##     <controlledterms>
##       <term><![CDATA[bioelectric phenomena]]></term>
##       <term><![CDATA[electric impedance measurement]]></term>
##       <term><![CDATA[readout electronics]]></term>
##       <term><![CDATA[skin]]></term>
##     </controlledterms>
##     <thesaurusterms>
##       <term><![CDATA[Current measurement]]></term>
##       <term><![CDATA[Demodulation]]></term>
##       <term><![CDATA[Frequency measurement]]></term>
##       <term><![CDATA[Frequency modulation]]></term>
##       <term><![CDATA[Impedance]]></term>
##       <term><![CDATA[Impedance measurement]]></term>
##       <term><![CDATA[Semiconductor device measurement]]></term>
##     </thesaurusterms>
##     <pubtitle><![CDATA[Sensors Journal, IEEE]]></pubtitle>
##     <punumber><![CDATA[7361]]></punumber>
##     <pubtype><![CDATA[Journals & Magazines]]></pubtype>
##     <publisher><![CDATA[IEEE]]></publisher>
##     <volume><![CDATA[14]]></volume>
##     <issue><![CDATA[8]]></issue>
##     <py><![CDATA[2014]]></py>
##     <spage><![CDATA[2792]]></spage>
##     <epage><![CDATA[2800]]></epage>
##     <abstract><![CDATA[Bioimpedance spectroscopy is used in a wide range of biomedical applications. This paper presents an integrated analog readout, which employs synchronous detection to perform galvanostatic multi-channel, multi-frequency bioimpedance measurements. The circuit was fabricated in a 0.35-&#x03BC;m CMOS technology and occupies an area of 1.52 mm<sup>2</sup>. The effect of random dc offsets is investigated, along with the use of chopping to minimize them. Impedance measurements of a known RC load and skin (using commercially available electrodes) demonstrate the operation of the system over a frequency range up to 1 MHz. The circuit operates from a &#x00B1;2.5 V power supply and has a power consumption of 3.4-mW per channel.]]></abstract>
##     <issn><![CDATA[1530-437X]]></issn>
##     <htmlFlag><![CDATA[1]]></htmlFlag>
##     <arnumber><![CDATA[6783990]]></arnumber>
##     <doi><![CDATA[10.1109/JSEN.2014.2315963]]></doi>
##     <publicationId><![CDATA[6783990]]></publicationId>
##     <mdurl><![CDATA[http://ieeexplore.ieee.org/xpl/articleDetails.jsp?tp=&arnumber=6783990&contentType=Journals+%26+Magazines]]></mdurl>
##     <pdf><![CDATA[http://ieeexplore.ieee.org/stamp/stamp.jsp?arnumber=6783990]]></pdf>
##   </document>
##   <document>
##     <rank>46</rank>
##     <title><![CDATA[Efficient Determination of the Uncertainty for the Optimization of SPECT System Design: A Subsampled Fisher Information Matrix]]></title>
##     <authors><![CDATA[Fuin, N.;  Pedemonte, S.;  Arridge, S.;  Ourselin, S.;  Hutton, B.F.]]></authors>
##     <affiliations><![CDATA[Inst. of Nucl. Med., Univ. Coll. London Hosp., London, UK]]></affiliations>
##     <controlledterms>
##       <term><![CDATA[adaptive systems]]></term>
##       <term><![CDATA[collimators]]></term>
##       <term><![CDATA[graphics processing units]]></term>
##       <term><![CDATA[image reconstruction]]></term>
##       <term><![CDATA[image sampling]]></term>
##       <term><![CDATA[medical image processing]]></term>
##       <term><![CDATA[optimisation]]></term>
##       <term><![CDATA[single photon emission computed tomography]]></term>
##     </controlledterms>
##     <thesaurusterms>
##       <term><![CDATA[Approximation methods]]></term>
##       <term><![CDATA[Cameras]]></term>
##       <term><![CDATA[Image reconstruction]]></term>
##       <term><![CDATA[Optimization]]></term>
##       <term><![CDATA[Single photon emission computed tomography]]></term>
##       <term><![CDATA[Uncertainty]]></term>
##     </thesaurusterms>
##     <pubtitle><![CDATA[Medical Imaging, IEEE Transactions on]]></pubtitle>
##     <punumber><![CDATA[42]]></punumber>
##     <pubtype><![CDATA[Journals & Magazines]]></pubtype>
##     <publisher><![CDATA[IEEE]]></publisher>
##     <volume><![CDATA[33]]></volume>
##     <issue><![CDATA[3]]></issue>
##     <py><![CDATA[2014]]></py>
##     <spage><![CDATA[618]]></spage>
##     <epage><![CDATA[635]]></epage>
##     <abstract><![CDATA[System designs in single photon emission tomography (SPECT) can be evaluated based on the fundamental trade-off between bias and variance that can be achieved in the reconstruction of emission tomograms. This trade off can be derived analytically using the Cramer-Rao type bounds, which imply the calculation and the inversion of the Fisher information matrix (FIM). The inverse of the FIM expresses the uncertainty associated to the tomogram, enabling the comparison of system designs. However, computing, storing and inverting the FIM is not practical with 3-D imaging systems. In order to tackle the problem of the computational load in calculating the inverse of the FIM, a method based on the calculation of the local impulse response and the variance, in a single point, from a single row of the FIM, has been previously proposed for system design. However this approximation (circulant approximation) does not capture the global interdependence between the variables in shift-variant systems such as SPECT, and cannot account e.g., for data truncation or missing data. Our new formulation relies on subsampling the FIM. The FIM is calculated over a subset of voxels arranged in a grid that covers the whole volume. Every element of the FIM at the grid points is calculated exactly, accounting for the acquisition geometry and for the object. This new formulation reduces the computational complexity in estimating the uncertainty, but nevertheless accounts for the global interdependence between the variables, enabling the exploration of design spaces hindered by the circulant approximation. The graphics processing unit accelerated implementation of the algorithm reduces further the computation times, making the algorithm a good candidate for real-time optimization of adaptive imaging systems. This paper describes the subsampled FIM formulation and implementation details. The advantages and limitations of the new approximation are explored, in comparison with the circulant approxima- ion, in the context of design optimization of a parallel-hole collimator SPECT system and of an adaptive imaging system (similar to the commercially available D-SPECT).]]></abstract>
##     <issn><![CDATA[0278-0062]]></issn>
##     <htmlFlag><![CDATA[1]]></htmlFlag>
##     <arnumber><![CDATA[6675890]]></arnumber>
##     <doi><![CDATA[10.1109/TMI.2013.2292805]]></doi>
##     <publicationId><![CDATA[6675890]]></publicationId>
##     <mdurl><![CDATA[http://ieeexplore.ieee.org/xpl/articleDetails.jsp?tp=&arnumber=6675890&contentType=Journals+%26+Magazines]]></mdurl>
##     <pdf><![CDATA[http://ieeexplore.ieee.org/stamp/stamp.jsp?arnumber=6675890]]></pdf>
##   </document>
##   <document>
##     <rank>47</rank>
##     <title><![CDATA[Biometric Antispoofing Methods: A Survey in Face Recognition]]></title>
##     <authors><![CDATA[Galbally, J.;  Marcel, S.;  Fierrez, J.]]></authors>
##     <affiliations><![CDATA[Inst. for the Protection & Security of the Citizen, Joint Res. Centre of the Eur. Comm., Ispra, Italy]]></affiliations>
##     <controlledterms>
##       <term><![CDATA[biometrics (access control)]]></term>
##       <term><![CDATA[face recognition]]></term>
##       <term><![CDATA[security of data]]></term>
##     </controlledterms>
##     <thesaurusterms>
##       <term><![CDATA[Access control]]></term>
##       <term><![CDATA[Authentication]]></term>
##       <term><![CDATA[Biomedical monitoring]]></term>
##       <term><![CDATA[Biometrics]]></term>
##       <term><![CDATA[Computer security]]></term>
##       <term><![CDATA[Distance measurement]]></term>
##       <term><![CDATA[Fingerprint recognition]]></term>
##       <term><![CDATA[Immune system]]></term>
##       <term><![CDATA[Iris recognition]]></term>
##       <term><![CDATA[Speech recognition]]></term>
##     </thesaurusterms>
##     <pubtitle><![CDATA[Access, IEEE]]></pubtitle>
##     <punumber><![CDATA[6287639]]></punumber>
##     <pubtype><![CDATA[Journals & Magazines]]></pubtype>
##     <publisher><![CDATA[IEEE]]></publisher>
##     <volume><![CDATA[2]]></volume>
##     <py><![CDATA[2014]]></py>
##     <spage><![CDATA[1530]]></spage>
##     <epage><![CDATA[1552]]></epage>
##     <abstract><![CDATA[In recent decades, we have witnessed the evolution of biometric technology from the first pioneering works in face and voice recognition to the current state of development wherein a wide spectrum of highly accurate systems may be found, ranging from largely deployed modalities, such as fingerprint, face, or iris, to more marginal ones, such as signature or hand. This path of technological evolution has naturally led to a critical issue that has only started to be addressed recently: the resistance of this rapidly emerging technology to external attacks and, in particular, to spoofing. Spoofing, referred to by the term presentation attack in current standards, is a purely biometric vulnerability that is not shared with other IT security solutions. It refers to the ability to fool a biometric system into recognizing an illegitimate user as a genuine one by means of presenting a synthetic forged version of the original biometric trait to the sensor. The entire biometric community, including researchers, developers, standardizing bodies, and vendors, has thrown itself into the challenging task of proposing and developing efficient protection methods against this threat. The goal of this paper is to provide a comprehensive overview on the work that has been carried out over the last decade in the emerging field of antispoofing, with special attention to the mature and largely deployed face modality. The work covers theories, methodologies, state-of-the-art techniques, and evaluation databases and also aims at providing an outlook into the future of this very active field of research.]]></abstract>
##     <issn><![CDATA[2169-3536]]></issn>
##     <htmlFlag><![CDATA[1]]></htmlFlag>
##     <arnumber><![CDATA[6990726]]></arnumber>
##     <doi><![CDATA[10.1109/ACCESS.2014.2381273]]></doi>
##     <publicationId><![CDATA[6990726]]></publicationId>
##     <mdurl><![CDATA[http://ieeexplore.ieee.org/xpl/articleDetails.jsp?tp=&arnumber=6990726&contentType=Journals+%26+Magazines]]></mdurl>
##     <pdf><![CDATA[http://ieeexplore.ieee.org/stamp/stamp.jsp?arnumber=6990726]]></pdf>
##   </document>
##   <document>
##     <rank>48</rank>
##     <title><![CDATA[Simulation and Experiment of a Ku-Band Gyro-TWT]]></title>
##     <authors><![CDATA[Jianxun Wang;  Yong Luo;  Yong Xu;  Ran Yan;  Youlei Pu;  Xue Deng;  Hai Wang]]></authors>
##     <affiliations><![CDATA[Sch. of Phys. Electron., Univ. of Electron. Sci. & Technol. of China, Chengdu, China]]></affiliations>
##     <controlledterms>
##       <term><![CDATA[gyrotrons]]></term>
##       <term><![CDATA[linear codes]]></term>
##       <term><![CDATA[millimetre wave couplers]]></term>
##       <term><![CDATA[travelling wave amplifiers]]></term>
##       <term><![CDATA[travelling wave tubes]]></term>
##     </controlledterms>
##     <thesaurusterms>
##       <term><![CDATA[Bandwidth]]></term>
##       <term><![CDATA[Couplers]]></term>
##       <term><![CDATA[Integrated circuit modeling]]></term>
##       <term><![CDATA[Oscillators]]></term>
##       <term><![CDATA[Power generation]]></term>
##       <term><![CDATA[Solid modeling]]></term>
##       <term><![CDATA[Superconducting magnets]]></term>
##     </thesaurusterms>
##     <pubtitle><![CDATA[Electron Devices, IEEE Transactions on]]></pubtitle>
##     <punumber><![CDATA[16]]></punumber>
##     <pubtype><![CDATA[Journals & Magazines]]></pubtype>
##     <publisher><![CDATA[IEEE]]></publisher>
##     <volume><![CDATA[61]]></volume>
##     <issue><![CDATA[6]]></issue>
##     <py><![CDATA[2014]]></py>
##     <spage><![CDATA[1818]]></spage>
##     <epage><![CDATA[1823]]></epage>
##     <abstract><![CDATA[Design techniques and experimental results are presented on a Ku-band TE<sub>11</sub> mode gyro-traveling wave tube. The hot test of this amplifier gives more than 153-kW output power, 2.3-GHz bandwidth (14%), 41-dB saturated gain, and 20% efficiency driven by a 63 kV, 12-A electron beam with a pitch angle (v<sub>t</sub>v<sub>z</sub>) of 1.2, and velocity spread of 5%. A linear polarized TE<sub>11</sub> mode input coupler is used to introduce the input power. The stability of the amplifier from oscillation, including both the operating TE<sub>11</sub> mode and the backward wave TE<sub>21</sub> mode, has been investigated with linear codes, nonlinear selfconsistent theory, and 3-D PIC CHIPIC simulation. To suppress the potential gyro-backward wave oscillator interactions, the high frequency circuit is loaded with lossy ceramic rings. The lossy structure is optimized by nonlinear theory and 3-D PIC simulation. A low velocity spread magnetron injection gun is designed with a new structure.]]></abstract>
##     <issn><![CDATA[0018-9383]]></issn>
##     <htmlFlag><![CDATA[1]]></htmlFlag>
##     <arnumber><![CDATA[6783728]]></arnumber>
##     <doi><![CDATA[10.1109/TED.2013.2296552]]></doi>
##     <publicationId><![CDATA[6783728]]></publicationId>
##     <mdurl><![CDATA[http://ieeexplore.ieee.org/xpl/articleDetails.jsp?tp=&arnumber=6783728&contentType=Journals+%26+Magazines]]></mdurl>
##     <pdf><![CDATA[http://ieeexplore.ieee.org/stamp/stamp.jsp?arnumber=6783728]]></pdf>
##   </document>
##   <document>
##     <rank>49</rank>
##     <title><![CDATA[Hardware Development and Locomotion Control Strategy for an Over-Ground Gait Trainer: NaTUre-Gaits]]></title>
##     <authors><![CDATA[Trieu Phat Luu;  Kin Huat Low;  Xingda Qu;  Hup Boon Lim;  Kay Hiang Hoon]]></authors>
##     <affiliations><![CDATA[Sch. of Mech. & Aerosp. Eng., Nanyang Technol. Univ., Singapore, Singapore]]></affiliations>
##     <controlledterms>
##       <term><![CDATA[gait analysis]]></term>
##       <term><![CDATA[injuries]]></term>
##       <term><![CDATA[medical robotics]]></term>
##       <term><![CDATA[mobile robots]]></term>
##       <term><![CDATA[orthotics]]></term>
##       <term><![CDATA[patient rehabilitation]]></term>
##     </controlledterms>
##     <thesaurusterms>
##       <term><![CDATA[Gait analysis]]></term>
##       <term><![CDATA[Legged locomotion]]></term>
##       <term><![CDATA[Mobile communication]]></term>
##       <term><![CDATA[Patient rehabilitation]]></term>
##       <term><![CDATA[Predictive models]]></term>
##     </thesaurusterms>
##     <pubtitle><![CDATA[Translational Engineering in Health and Medicine, IEEE Journal of]]></pubtitle>
##     <punumber><![CDATA[6221039]]></punumber>
##     <pubtype><![CDATA[Journals & Magazines]]></pubtype>
##     <publisher><![CDATA[IEEE]]></publisher>
##     <volume><![CDATA[2]]></volume>
##     <py><![CDATA[2014]]></py>
##     <spage><![CDATA[1]]></spage>
##     <epage><![CDATA[9]]></epage>
##     <abstract><![CDATA[Therapist-assisted body weight supported (TABWS) gait rehabilitation was introduced two decades ago. The benefit of TABWS in functional recovery of walking in spinal cord injury and stroke patients has been demonstrated and reported. However, shortage of therapists, labor-intensiveness, and short duration of training are some limitations of this approach. To overcome these deficiencies, robotic-assisted gait rehabilitation systems have been suggested. These systems have gained attentions from researchers and clinical practitioner in recent years. To achieve the same objective, an over-ground gait rehabilitation system, NaTUre-gaits, was developed at the Nanyang Technological University. The design was based on a clinical approach to provide four main features, which are pelvic motion, body weight support, over-ground walking experience, and lower limb assistance. These features can be achieved by three main modules of NaTUre-gaits: 1) pelvic assistance mechanism, mobile platform, and robotic orthosis. Predefined gait patterns are required for a robotic assisted system to follow. In this paper, the gait pattern planning for NaTUre-gaits was accomplished by an individual-specific gait pattern prediction model. The model generates gait patterns that resemble natural gait patterns of the targeted subjects. The features of NaTUre-gaits have been demonstrated by walking trials with several subjects. The trials have been evaluated by therapists and doctors. The results show that 10-m walking trial with a reduction in manpower. The task-specific repetitive training approach and natural walking gait patterns were also successfully achieved.]]></abstract>
##     <issn><![CDATA[2168-2372]]></issn>
##     <htmlFlag><![CDATA[1]]></htmlFlag>
##     <arnumber><![CDATA[6729037]]></arnumber>
##     <doi><![CDATA[10.1109/JTEHM.2014.2303807]]></doi>
##     <publicationId><![CDATA[6729037]]></publicationId>
##     <mdurl><![CDATA[http://ieeexplore.ieee.org/xpl/articleDetails.jsp?tp=&arnumber=6729037&contentType=Journals+%26+Magazines]]></mdurl>
##     <pdf><![CDATA[http://ieeexplore.ieee.org/stamp/stamp.jsp?arnumber=6729037]]></pdf>
##   </document>
##   <document>
##     <rank>50</rank>
##     <title><![CDATA[Test Versus Security: Past and Present]]></title>
##     <authors><![CDATA[Da Rolt, J.;  Das, A.;  Di Natale, G.;  Flottes, M.-L.;  Rouzeyre, B.;  Verbauwhede, I.]]></authors>
##     <affiliations><![CDATA[UFRGS, Porto Alegre, Brazil]]></affiliations>
##     <controlledterms>
##       <term><![CDATA[VLSI]]></term>
##       <term><![CDATA[design for testability]]></term>
##       <term><![CDATA[embedded systems]]></term>
##       <term><![CDATA[integrated circuit testing]]></term>
##       <term><![CDATA[public key cryptography]]></term>
##     </controlledterms>
##     <thesaurusterms>
##       <term><![CDATA[Computer crime]]></term>
##       <term><![CDATA[Computer hacking]]></term>
##       <term><![CDATA[Computer security]]></term>
##       <term><![CDATA[Cryptography]]></term>
##       <term><![CDATA[Decoding]]></term>
##     </thesaurusterms>
##     <pubtitle><![CDATA[Emerging Topics in Computing, IEEE Transactions on]]></pubtitle>
##     <punumber><![CDATA[6245516]]></punumber>
##     <pubtype><![CDATA[Journals & Magazines]]></pubtype>
##     <publisher><![CDATA[IEEE]]></publisher>
##     <volume><![CDATA[2]]></volume>
##     <issue><![CDATA[1]]></issue>
##     <py><![CDATA[2014]]></py>
##     <spage><![CDATA[50]]></spage>
##     <epage><![CDATA[62]]></epage>
##     <abstract><![CDATA[Cryptographic circuits need to be protected against side-channel attacks, which target their physical attributes while the cryptographic algorithm is in execution. There can be various side-channels, such as power, timing, electromagnetic radiation, fault response, and so on. One such important side-channel is the design-for-testability (DfT) infrastructure present for effective and timely testing of VLSI circuits. The attacker can extract secret information stored on the chip by scanning out test responses against some chosen plaintext inputs. The purpose of this paper is to first present a detailed survey on the state-of-the-art in scan-based side-channel attacks on symmetric and public-key cryptographic hardware implementations, both in the absence and presence of advanced DfT structures, such as test compression and X-masking, which may make the attack difficult. Then, the existing scan attack countermeasures are evaluated for determining their security against known scan attacks. In addition, JTAG vulnerability and security countermeasures are also analyzed as part of the external test interface. A comparative area-timing-security analysis of existing countermeasures at various abstraction levels is presented in order to help an embedded security designer make an informed choice for his intended application.]]></abstract>
##     <issn><![CDATA[2168-6750]]></issn>
##     <htmlFlag><![CDATA[1]]></htmlFlag>
##     <arnumber><![CDATA[6733305]]></arnumber>
##     <doi><![CDATA[10.1109/TETC.2014.2304492]]></doi>
##     <publicationId><![CDATA[6733305]]></publicationId>
##     <mdurl><![CDATA[http://ieeexplore.ieee.org/xpl/articleDetails.jsp?tp=&arnumber=6733305&contentType=Journals+%26+Magazines]]></mdurl>
##     <pdf><![CDATA[http://ieeexplore.ieee.org/stamp/stamp.jsp?arnumber=6733305]]></pdf>
##   </document>
##   <document>
##     <rank>51</rank>
##     <title><![CDATA[A Social Compute Cloud: Allocating and Sharing Infrastructure Resources via Social Networks]]></title>
##     <authors><![CDATA[Caton, S.;  Haas, C.;  Chard, K.;  Bubendorfer, K.;  Rana, O.F.]]></authors>
##     <affiliations><![CDATA[Karlsruhe Service Res. Inst., Karlsruhe Inst. of Technol., Karlsruhe, Germany]]></affiliations>
##     <controlledterms>
##       <term><![CDATA[cloud computing]]></term>
##       <term><![CDATA[microcomputers]]></term>
##       <term><![CDATA[resource allocation]]></term>
##       <term><![CDATA[social networking (online)]]></term>
##     </controlledterms>
##     <thesaurusterms>
##       <term><![CDATA[Cloud computing]]></term>
##       <term><![CDATA[Computational modeling]]></term>
##       <term><![CDATA[Educational institutions]]></term>
##       <term><![CDATA[Facebook]]></term>
##       <term><![CDATA[Resource management]]></term>
##       <term><![CDATA[Supply and demand]]></term>
##     </thesaurusterms>
##     <pubtitle><![CDATA[Services Computing, IEEE Transactions on]]></pubtitle>
##     <punumber><![CDATA[4629386]]></punumber>
##     <pubtype><![CDATA[Journals & Magazines]]></pubtype>
##     <publisher><![CDATA[IEEE]]></publisher>
##     <volume><![CDATA[7]]></volume>
##     <issue><![CDATA[3]]></issue>
##     <py><![CDATA[2014]]></py>
##     <spage><![CDATA[359]]></spage>
##     <epage><![CDATA[372]]></epage>
##     <abstract><![CDATA[Social network platforms have rapidly changed the way that people communicate and interact. They have enabled the establishment of, and participation in, digital communities as well as the representation, documentation and exploration of social relationships. We believe that as `apps' become more sophisticated, it will become easier for users to share their own services, resources and data via social networks. To substantiate this, we present a social compute cloud where the provisioning of cloud infrastructure occurs through &#x201C;friend&#x201D; relationships. In a social compute cloud, resource owners offer virtualized containers on their personal computer(s) or smart device(s) to their social network. However, as users may have complex preference structures concerning with whom they do or do not wish to share their resources, we investigate, via simulation, how resources can be effectively allocated within a social community offering resources on a best effort basis. In the assessment of social resource allocation, we consider welfare, allocation fairness, and algorithmic runtime. The key findings of this work illustrate how social networks can be leveraged in the construction of cloud computing infrastructures and how resources can be allocated in the presence of user sharing preferences.]]></abstract>
##     <issn><![CDATA[1939-1374]]></issn>
##     <arnumber><![CDATA[6727497]]></arnumber>
##     <doi><![CDATA[10.1109/TSC.2014.2303091]]></doi>
##     <publicationId><![CDATA[6727497]]></publicationId>
##     <mdurl><![CDATA[http://ieeexplore.ieee.org/xpl/articleDetails.jsp?tp=&arnumber=6727497&contentType=Journals+%26+Magazines]]></mdurl>
##     <pdf><![CDATA[http://ieeexplore.ieee.org/stamp/stamp.jsp?arnumber=6727497]]></pdf>
##   </document>
##   <document>
##     <rank>52</rank>
##     <title><![CDATA[Alamouti-Type STBC for Atmospheric Optical Communication Using Coherent Detection]]></title>
##     <authors><![CDATA[Mingbo Niu;  Cheng, J.;  Holzman, J.F.]]></authors>
##     <affiliations><![CDATA[Dept. of Electr. & Comput. Eng., Queen's Univ., Kingston, ON, Canada]]></affiliations>
##     <controlledterms>
##       <term><![CDATA[atmospheric optics]]></term>
##       <term><![CDATA[atmospheric turbulence]]></term>
##       <term><![CDATA[optical communication]]></term>
##       <term><![CDATA[phase noise]]></term>
##       <term><![CDATA[space-time block codes]]></term>
##     </controlledterms>
##     <thesaurusterms>
##       <term><![CDATA[Coherence]]></term>
##       <term><![CDATA[Optical fiber communication]]></term>
##       <term><![CDATA[Optical noise]]></term>
##       <term><![CDATA[Optical receivers]]></term>
##       <term><![CDATA[Optical transmitters]]></term>
##       <term><![CDATA[Phase noise]]></term>
##     </thesaurusterms>
##     <pubtitle><![CDATA[Photonics Journal, IEEE]]></pubtitle>
##     <punumber><![CDATA[4563994]]></punumber>
##     <pubtype><![CDATA[Journals & Magazines]]></pubtype>
##     <publisher><![CDATA[IEEE]]></publisher>
##     <volume><![CDATA[6]]></volume>
##     <issue><![CDATA[1]]></issue>
##     <py><![CDATA[2014]]></py>
##     <spage><![CDATA[1]]></spage>
##     <epage><![CDATA[17]]></epage>
##     <abstract><![CDATA[Alamouti-type space-time block coding is studied for optical wireless communication systems using coherent detection over atmospheric turbulence channels. Atmospheric turbulence-induced fading and phase noise are known to impair the performance of coherent optical wireless systems. Two new Alamouti-type space-time coded architectures are proposed to overcome turbulence-induced fading and phase noise from transmitter lasers, receiver local oscillators, and turbulence channels. Their error rate performance is analyzed for a wide range of turbulence conditions. With developed analytical error rate expressions, the usefulness of the proposed Alamouti-type space-time block coding is demonstrated for coherent optical wireless communication systems. Numerical results confirm the performance improvements of the proposed systems over that of single-input-single-output systems.]]></abstract>
##     <issn><![CDATA[1943-0655]]></issn>
##     <htmlFlag><![CDATA[1]]></htmlFlag>
##     <arnumber><![CDATA[6725602]]></arnumber>
##     <doi><![CDATA[10.1109/JPHOT.2014.2302807]]></doi>
##     <publicationId><![CDATA[6725602]]></publicationId>
##     <mdurl><![CDATA[http://ieeexplore.ieee.org/xpl/articleDetails.jsp?tp=&arnumber=6725602&contentType=Journals+%26+Magazines]]></mdurl>
##     <pdf><![CDATA[http://ieeexplore.ieee.org/stamp/stamp.jsp?arnumber=6725602]]></pdf>
##   </document>
##   <document>
##     <rank>53</rank>
##     <title><![CDATA[Gallium Nitride as an Electromechanical Material]]></title>
##     <authors><![CDATA[Rais-Zadeh, M.;  Gokhale, V.J.;  Ansari, A.;  Faucher, M.;  The&#x0301; ron, D.;  Cordier, Y.;  Buchaillot, L.]]></authors>
##     <affiliations><![CDATA[Univ. of Michigan, Ann Arbor, MI, USA]]></affiliations>
##     <controlledterms>
##       <term><![CDATA[III-V semiconductors]]></term>
##       <term><![CDATA[acoustics]]></term>
##       <term><![CDATA[gallium compounds]]></term>
##       <term><![CDATA[micromechanical devices]]></term>
##       <term><![CDATA[piezoelectricity]]></term>
##       <term><![CDATA[thermal properties]]></term>
##       <term><![CDATA[wide band gap semiconductors]]></term>
##     </controlledterms>
##     <thesaurusterms>
##       <term><![CDATA[Gallium nitride]]></term>
##       <term><![CDATA[III-V semiconductor materials]]></term>
##       <term><![CDATA[Silicon]]></term>
##       <term><![CDATA[Stress]]></term>
##       <term><![CDATA[Substrates]]></term>
##       <term><![CDATA[Temperature measurement]]></term>
##     </thesaurusterms>
##     <pubtitle><![CDATA[Microelectromechanical Systems, Journal of]]></pubtitle>
##     <punumber><![CDATA[84]]></punumber>
##     <pubtype><![CDATA[Journals & Magazines]]></pubtype>
##     <publisher><![CDATA[IEEE]]></publisher>
##     <volume><![CDATA[23]]></volume>
##     <issue><![CDATA[6]]></issue>
##     <py><![CDATA[2014]]></py>
##     <spage><![CDATA[1252]]></spage>
##     <epage><![CDATA[1271]]></epage>
##     <abstract><![CDATA[Gallium nitride (GaN) is a wide bandgap semiconductor material and is the most popular material after silicon in the semiconductor industry. The prime movers behind this trend are LEDs, microwave, and more recently, power electronics. New areas of research also include spintronics and nanoribbon transistors, which leverage some of the unique properties of GaN. GaN has electron mobility comparable with silicon, but with a bandgap that is three times larger, making it an excellent candidate for high-power applications and high-temperature operation. The ability to form thin-AlGaN/GaN heterostructures, which exhibit the 2-D electron gas phenomenon leads to high-electron mobility transistors, which exhibit high Johnson's figure of merit. Another interesting direction for GaN research, which is largely unexplored, is GaN-based micromechanical devices or GaN microelectromechanical systems (MEMS). To fully unlock the potential of GaN and realize new advanced all-GaN integrated circuits, it is essential to cointegrate passive devices (such as resonators and filters), sensors (such as temperature and gas sensors), and other more than Moore functional devices with GaN active electronics. Therefore, there is a growing interest in the use of GaN as a mechanical material. This paper reviews the electromechanical, thermal, acoustic, and piezoelectric properties of GaN, and describes the working principle of some of the reported high-performance GaN-based microelectromechanical components. It also provides an outlook for possible research directions in GaN MEMS.]]></abstract>
##     <issn><![CDATA[1057-7157]]></issn>
##     <htmlFlag><![CDATA[1]]></htmlFlag>
##     <arnumber><![CDATA[6895237]]></arnumber>
##     <doi><![CDATA[10.1109/JMEMS.2014.2352617]]></doi>
##     <publicationId><![CDATA[6895237]]></publicationId>
##     <mdurl><![CDATA[http://ieeexplore.ieee.org/xpl/articleDetails.jsp?tp=&arnumber=6895237&contentType=Journals+%26+Magazines]]></mdurl>
##     <pdf><![CDATA[http://ieeexplore.ieee.org/stamp/stamp.jsp?arnumber=6895237]]></pdf>
##   </document>
##   <document>
##     <rank>54</rank>
##     <title><![CDATA[Are Children More Exposed to Radio Frequency Energy From Mobile Phones Than Adults?]]></title>
##     <authors><![CDATA[Foster, K.R.;  Chung-Kwang Chou]]></authors>
##     <affiliations><![CDATA[Dept. of Bioeng., Univ. of Pennsylvania, Philadelphia, PA, USA]]></affiliations>
##     <controlledterms>
##       <term><![CDATA[bioelectric phenomena]]></term>
##       <term><![CDATA[bone]]></term>
##       <term><![CDATA[brain]]></term>
##       <term><![CDATA[dielectric properties]]></term>
##       <term><![CDATA[dosimetry]]></term>
##       <term><![CDATA[mobile handsets]]></term>
##       <term><![CDATA[paediatrics]]></term>
##     </controlledterms>
##     <thesaurusterms>
##       <term><![CDATA[Hazards]]></term>
##       <term><![CDATA[Magnetic heads]]></term>
##       <term><![CDATA[Mobile handsets]]></term>
##       <term><![CDATA[Pediatrics]]></term>
##       <term><![CDATA[Radiation effects]]></term>
##       <term><![CDATA[Radio frequency]]></term>
##       <term><![CDATA[Specific absorption rate]]></term>
##     </thesaurusterms>
##     <pubtitle><![CDATA[Access, IEEE]]></pubtitle>
##     <punumber><![CDATA[6287639]]></punumber>
##     <pubtype><![CDATA[Journals & Magazines]]></pubtype>
##     <publisher><![CDATA[IEEE]]></publisher>
##     <volume><![CDATA[2]]></volume>
##     <py><![CDATA[2014]]></py>
##     <spage><![CDATA[1497]]></spage>
##     <epage><![CDATA[1509]]></epage>
##     <abstract><![CDATA[There has been long-standing controversy, both among scientists and in the public, about whether children absorb more radio frequency (RF) energy in their heads than adults when using a mobile telephone. This review summarizes the current understanding of this issue, and some of the complexities in comparing the absorption of RF energy in different individuals from use of mobile phones. The discussion is limited to dosimetric issues, i.e., possible age-related differences in absorption of RF energy in the heads of mobile phone users. For most metrics of exposure, in particular those relevant to assessing the compliance of handsets with regulatory limits, there is no clear evidence for age-related differences in exposure. For two metrics of exposure, there is a clear evidence that age can play a factor: 1) the local specific absorption rate (SAR), in particular anatomically defined locations within the brain, will vary with head size and hence with age and 2) the SAR, in particular tissues, (e.g., bone marrow in the skull) can vary with age due to age-related differences in the dielectric properties of tissue. However, these differences involve SAR levels that are below the 1-g or 10-g peak spatial SAR (psSAR averaged over 1 or 10 g of tissue) and have no significance for compliance assessment. Age-related differences observed in worst case simulations such as presently considered are difficult to generalize to human populations under real-world exposure conditions due to many variables that determine SAR during realistic usages.]]></abstract>
##     <issn><![CDATA[2169-3536]]></issn>
##     <htmlFlag><![CDATA[1]]></htmlFlag>
##     <arnumber><![CDATA[6982034]]></arnumber>
##     <doi><![CDATA[10.1109/ACCESS.2014.2380355]]></doi>
##     <publicationId><![CDATA[6982034]]></publicationId>
##     <mdurl><![CDATA[http://ieeexplore.ieee.org/xpl/articleDetails.jsp?tp=&arnumber=6982034&contentType=Journals+%26+Magazines]]></mdurl>
##     <pdf><![CDATA[http://ieeexplore.ieee.org/stamp/stamp.jsp?arnumber=6982034]]></pdf>
##   </document>
##   <document>
##     <rank>55</rank>
##     <title><![CDATA[Performance Improvement in Analog Photonics Link Incorporating Digital Post-Compensation and Low-Noise Electrical Amplifier]]></title>
##     <authors><![CDATA[Yitang Dai;  Yan Cui;  Xi Liang;  Feifei Yin;  Jianqiang Li;  Kun Xu;  Jintong Lin]]></authors>
##     <affiliations><![CDATA[State Key Lab. of Inf. Photonics & Opt. Commun., Beijing Univ. of Posts & Telecommun., Beijing, China]]></affiliations>
##     <controlledterms>
##       <term><![CDATA[analogue circuits]]></term>
##       <term><![CDATA[low noise amplifiers]]></term>
##       <term><![CDATA[microwave photonics]]></term>
##       <term><![CDATA[signal processing]]></term>
##     </controlledterms>
##     <thesaurusterms>
##       <term><![CDATA[Bandwidth]]></term>
##       <term><![CDATA[Gain]]></term>
##       <term><![CDATA[Modulation]]></term>
##       <term><![CDATA[Noise]]></term>
##       <term><![CDATA[Noise measurement]]></term>
##       <term><![CDATA[Photonics]]></term>
##       <term><![CDATA[Radio frequency]]></term>
##     </thesaurusterms>
##     <pubtitle><![CDATA[Photonics Journal, IEEE]]></pubtitle>
##     <punumber><![CDATA[4563994]]></punumber>
##     <pubtype><![CDATA[Journals & Magazines]]></pubtype>
##     <publisher><![CDATA[IEEE]]></publisher>
##     <volume><![CDATA[6]]></volume>
##     <issue><![CDATA[4]]></issue>
##     <py><![CDATA[2014]]></py>
##     <spage><![CDATA[1]]></spage>
##     <epage><![CDATA[7]]></epage>
##     <abstract><![CDATA[In this paper, we investigate the performance improvement in analog photonics link incorporating digital signal processing technique and low-noise electrical amplifiers. The proposed linearization does not require to know the precise transfer function of the whole nonlinear system, which is achieved by directly acquiring the output third-order intercept point from the system hardware. Through using a high-performance pre- and post-amplifier, improved link noise figure and gain are obtained while the nonlinearity of the electrical amplifiers can be also compensated by the proposed algorithm. Experimentally, spurious-free dynamic range of 128.3 dB in 1-Hz bandwidth is achieved for a digitizer noise limited analog photonics link. The measured noise figure and gain of the photonics link are 8.9 and 27.5 dB, respectively.]]></abstract>
##     <issn><![CDATA[1943-0655]]></issn>
##     <htmlFlag><![CDATA[1]]></htmlFlag>
##     <arnumber><![CDATA[6847160]]></arnumber>
##     <doi><![CDATA[10.1109/JPHOT.2014.2332474]]></doi>
##     <publicationId><![CDATA[6847160]]></publicationId>
##     <mdurl><![CDATA[http://ieeexplore.ieee.org/xpl/articleDetails.jsp?tp=&arnumber=6847160&contentType=Journals+%26+Magazines]]></mdurl>
##     <pdf><![CDATA[http://ieeexplore.ieee.org/stamp/stamp.jsp?arnumber=6847160]]></pdf>
##   </document>
##   <document>
##     <rank>56</rank>
##     <title><![CDATA[Unraveling the Enhanced Electrical Conductivity of PEDOT:PSS Thin Films for ITO-Free Organic Photovoltaics]]></title>
##     <authors><![CDATA[Sheng Hsiung Chang;  Chien-Hung Chiang;  Feng-Sheng Kao;  Chuen-Lin Tien;  Chun-Guey Wu]]></authors>
##     <affiliations><![CDATA[Res. Center for New Generation Photovoltaics, Nat. Central Univ., Jongli, Taiwan]]></affiliations>
##     <controlledterms>
##       <term><![CDATA[Raman spectra]]></term>
##       <term><![CDATA[compressive strength]]></term>
##       <term><![CDATA[conducting polymers]]></term>
##       <term><![CDATA[electrical conductivity]]></term>
##       <term><![CDATA[electron density]]></term>
##       <term><![CDATA[infrared spectra]]></term>
##       <term><![CDATA[polymer blends]]></term>
##       <term><![CDATA[polymer films]]></term>
##       <term><![CDATA[solar cells]]></term>
##       <term><![CDATA[thin film devices]]></term>
##       <term><![CDATA[visible spectra]]></term>
##     </controlledterms>
##     <thesaurusterms>
##       <term><![CDATA[Adhesives]]></term>
##       <term><![CDATA[Annealing]]></term>
##       <term><![CDATA[Conductivity]]></term>
##       <term><![CDATA[Electrodes]]></term>
##       <term><![CDATA[Films]]></term>
##       <term><![CDATA[Photovoltaic systems]]></term>
##     </thesaurusterms>
##     <pubtitle><![CDATA[Photonics Journal, IEEE]]></pubtitle>
##     <punumber><![CDATA[4563994]]></punumber>
##     <pubtype><![CDATA[Journals & Magazines]]></pubtype>
##     <publisher><![CDATA[IEEE]]></publisher>
##     <volume><![CDATA[6]]></volume>
##     <issue><![CDATA[4]]></issue>
##     <py><![CDATA[2014]]></py>
##     <spage><![CDATA[1]]></spage>
##     <epage><![CDATA[7]]></epage>
##     <abstract><![CDATA[We present the structure and the optical and mechanical properties of highly conductive PEDOT:PSS (1:2.5 wt%, PH1000) thin films fabricated with and without an immersion treatment process using a solution containing 67% ethylene glycol and 33% hexafluoro-isopropyl alcohol, by volume. The enhanced electrical conductivity of the PEDOT:PSS thin films originated from the formation of a conducting PEDOT network in combination with an increased electron concentration due to the conformational changes in the PEDOT chains. The modified PEDOT:PSS thin film was used as a transparent anode electrode for P3HT:PCBM blended film-based photovoltaics, resulting in a power conversion efficiency of 3.28% under 1-sun illumination.]]></abstract>
##     <issn><![CDATA[1943-0655]]></issn>
##     <htmlFlag><![CDATA[1]]></htmlFlag>
##     <arnumber><![CDATA[6849426]]></arnumber>
##     <doi><![CDATA[10.1109/JPHOT.2014.2331254]]></doi>
##     <publicationId><![CDATA[6849426]]></publicationId>
##     <mdurl><![CDATA[http://ieeexplore.ieee.org/xpl/articleDetails.jsp?tp=&arnumber=6849426&contentType=Journals+%26+Magazines]]></mdurl>
##     <pdf><![CDATA[http://ieeexplore.ieee.org/stamp/stamp.jsp?arnumber=6849426]]></pdf>
##   </document>
##   <document>
##     <rank>57</rank>
##     <title><![CDATA[Multiwavelength Brillouin-Thulium Fiber Laser]]></title>
##     <authors><![CDATA[Xiong Wang;  Pu Zhou;  Xiaolin Wang;  Hu Xiao;  Lei Si]]></authors>
##     <affiliations><![CDATA[Coll. of Optoelectron. Sci. & Eng., Nat. Univ. of Defense Technol., Changsha, China]]></affiliations>
##     <controlledterms>
##       <term><![CDATA[fibre lasers]]></term>
##       <term><![CDATA[laser beams]]></term>
##       <term><![CDATA[optical fibre polarisation]]></term>
##       <term><![CDATA[optical rotation]]></term>
##       <term><![CDATA[stimulated Brillouin scattering]]></term>
##       <term><![CDATA[thulium]]></term>
##     </controlledterms>
##     <thesaurusterms>
##       <term><![CDATA[Cavity resonators]]></term>
##       <term><![CDATA[Fiber lasers]]></term>
##       <term><![CDATA[Optical fiber amplifiers]]></term>
##       <term><![CDATA[Optical fiber couplers]]></term>
##       <term><![CDATA[Optical fiber networks]]></term>
##       <term><![CDATA[Optical fiber polarization]]></term>
##       <term><![CDATA[Scattering]]></term>
##     </thesaurusterms>
##     <pubtitle><![CDATA[Photonics Journal, IEEE]]></pubtitle>
##     <punumber><![CDATA[4563994]]></punumber>
##     <pubtype><![CDATA[Journals & Magazines]]></pubtype>
##     <publisher><![CDATA[IEEE]]></publisher>
##     <volume><![CDATA[6]]></volume>
##     <issue><![CDATA[1]]></issue>
##     <py><![CDATA[2014]]></py>
##     <spage><![CDATA[1]]></spage>
##     <epage><![CDATA[7]]></epage>
##     <abstract><![CDATA[We demonstrate the first multiwavelength Brillouin-thulium fiber laser (BTFL) with wavelengths near 1.97 &#x03BC;m. The multiwavelength fiber laser was realized based on stimulated Brillouin scattering and nonlinear polarization rotation effects. The laser can generate five Brillouin wavelengths with spacing of about 0.105 nm, which is identical with the wavelength spacing of adjacent Brillouin Stokes lines. The peak power fluctuations of the multiwavelength BTFL were less than 0.5 dB, and the wavelength shifts were less than 0.01 nm in 20 min.]]></abstract>
##     <issn><![CDATA[1943-0655]]></issn>
##     <htmlFlag><![CDATA[1]]></htmlFlag>
##     <arnumber><![CDATA[6690213]]></arnumber>
##     <doi><![CDATA[10.1109/JPHOT.2013.2295471]]></doi>
##     <publicationId><![CDATA[6690213]]></publicationId>
##     <mdurl><![CDATA[http://ieeexplore.ieee.org/xpl/articleDetails.jsp?tp=&arnumber=6690213&contentType=Journals+%26+Magazines]]></mdurl>
##     <pdf><![CDATA[http://ieeexplore.ieee.org/stamp/stamp.jsp?arnumber=6690213]]></pdf>
##   </document>
##   <document>
##     <rank>58</rank>
##     <title><![CDATA[Rectangular Fixed-Gantry CT Prototype: Combining CNT X-Ray Sources and Accelerated Compressed Sensing-Based Reconstruction]]></title>
##     <authors><![CDATA[Gonzales, B.;  Spronk, D.;  Yuan Cheng;  Tucker, A.W.;  Beckman, M.;  Zhou, O.;  Jianping Lu]]></authors>
##     <affiliations><![CDATA[XinRay Syst. Inc., Research Triangle Park, NC, USA]]></affiliations>
##     <controlledterms>
##       <term><![CDATA[X-ray imaging]]></term>
##       <term><![CDATA[X-ray tubes]]></term>
##       <term><![CDATA[carbon nanotubes]]></term>
##       <term><![CDATA[compressed sensing]]></term>
##       <term><![CDATA[computational geometry]]></term>
##       <term><![CDATA[computerised tomography]]></term>
##       <term><![CDATA[focal planes]]></term>
##       <term><![CDATA[image reconstruction]]></term>
##       <term><![CDATA[iterative methods]]></term>
##       <term><![CDATA[minimisation]]></term>
##     </controlledterms>
##     <thesaurusterms>
##       <term><![CDATA[Biomedical image processing]]></term>
##       <term><![CDATA[Carbon nanotubes]]></term>
##       <term><![CDATA[Computed tomography]]></term>
##       <term><![CDATA[Electron tubes]]></term>
##       <term><![CDATA[Geometry]]></term>
##       <term><![CDATA[Image reconstruction]]></term>
##       <term><![CDATA[Reconstruction algorithms]]></term>
##       <term><![CDATA[X-ray imaging]]></term>
##     </thesaurusterms>
##     <pubtitle><![CDATA[Access, IEEE]]></pubtitle>
##     <punumber><![CDATA[6287639]]></punumber>
##     <pubtype><![CDATA[Journals & Magazines]]></pubtype>
##     <publisher><![CDATA[IEEE]]></publisher>
##     <volume><![CDATA[2]]></volume>
##     <py><![CDATA[2014]]></py>
##     <spage><![CDATA[971]]></spage>
##     <epage><![CDATA[981]]></epage>
##     <abstract><![CDATA[Carbon nanotube (CNT)-based multibeam X-ray tubes provide an array of individually controllable X-ray focal spots. The CNT tube allows for flexible placement and distribution of X-ray focal spots in a system. Using a CNT tube, a computed tomography (CT) system with a noncircular geometry and a nonrotating gantry can be created. The noncircular CT geometry can be optimized around a specific imaging problem, utilizing the flexibility of CNT multibeam X-ray tubes to achieve the optimal focal spot distribution for the design constraints of the problem. Iterative reconstruction algorithms provide flexible CT reconstruction to accommodate the noncircular geometry. Compressed sensing-based iterative reconstruction algorithms apply a sparsity constraint to the reconstructed images that can partially account for missing angular coverage due to the noncircular geometry. In this paper, we present a laboratory prototype CT system that uses CNT multibeam X-ray tubes; a rectangular, nonrotating imaging geometry; and an accelerated compressed sensing-based iterative reconstruction algorithm. We apply a total variation minimization as our sparsity constraint. We present the advanced CNT multibeam tubes and show the stability and flexibility of these new tubes. We also present the unique imaging geometry and discuss the design constraints that influenced the specific system design. The reconstruction method is presented along with an overview of the acceleration of the algorithm to near real-time reconstruction. We demonstrate that the prototype reconstructed images have image quality comparable with a conventional CT system. The prototype is optimized for airport checkpoint baggage screening, but the concepts developed may apply to other application-specific CT imaging systems.]]></abstract>
##     <issn><![CDATA[2169-3536]]></issn>
##     <htmlFlag><![CDATA[1]]></htmlFlag>
##     <arnumber><![CDATA[6882769]]></arnumber>
##     <doi><![CDATA[10.1109/ACCESS.2014.2351751]]></doi>
##     <publicationId><![CDATA[6882769]]></publicationId>
##     <mdurl><![CDATA[http://ieeexplore.ieee.org/xpl/articleDetails.jsp?tp=&arnumber=6882769&contentType=Journals+%26+Magazines]]></mdurl>
##     <pdf><![CDATA[http://ieeexplore.ieee.org/stamp/stamp.jsp?arnumber=6882769]]></pdf>
##   </document>
##   <document>
##     <rank>59</rank>
##     <title><![CDATA[Single-Flange 2-Port TRL Calibration for Accurate THz <formula formulatype="inline"> <img src="/images/tex/702.gif" alt="{\rm S}"> </formula>-Parameter Measurements of Waveguide Integrated Circuits]]></title>
##     <authors><![CDATA[Hanning, J.;  Stenarson, J.;  Yhland, K.;  Sobis, P.J.;  Bryllert, T.;  Stake, J.]]></authors>
##     <affiliations><![CDATA[Dept. of Microtechnol. & Nanosci.-MC2, Chalmers Univ. of Technol., Go&#x0308;teborg, Sweden]]></affiliations>
##     <controlledterms>
##       <term><![CDATA[S-parameters]]></term>
##       <term><![CDATA[calibration]]></term>
##       <term><![CDATA[measurement uncertainty]]></term>
##       <term><![CDATA[microwave integrated circuits]]></term>
##       <term><![CDATA[network analysers]]></term>
##       <term><![CDATA[substrate integrated waveguides]]></term>
##     </controlledterms>
##     <thesaurusterms>
##       <term><![CDATA[Calibration]]></term>
##       <term><![CDATA[Educational institutions]]></term>
##       <term><![CDATA[Microwave circuits]]></term>
##       <term><![CDATA[Standards]]></term>
##       <term><![CDATA[Uncertainty]]></term>
##     </thesaurusterms>
##     <pubtitle><![CDATA[Terahertz Science and Technology, IEEE Transactions on]]></pubtitle>
##     <punumber><![CDATA[5503871]]></punumber>
##     <pubtype><![CDATA[Journals & Magazines]]></pubtype>
##     <publisher><![CDATA[IEEE]]></publisher>
##     <volume><![CDATA[4]]></volume>
##     <issue><![CDATA[5]]></issue>
##     <py><![CDATA[2014]]></py>
##     <spage><![CDATA[582]]></spage>
##     <epage><![CDATA[587]]></epage>
##     <abstract><![CDATA[This paper describes a single flange 2-port measurement setup for S-parameter characterization of waveguide integrated devices. The setup greatly reduces calibration and measurement uncertainty by eliminating vector network analyzer (VNA) extender cable movement and minimizing the effect of waveguide manufacturing tolerances. Change time of standards is also improved, reducing the influence of VNA drift on the uncertainty. A TRL calibration kit has been manufactured and measurements are demonstrated in WR-03 (220-325 GHz).]]></abstract>
##     <issn><![CDATA[2156-342X]]></issn>
##     <htmlFlag><![CDATA[1]]></htmlFlag>
##     <arnumber><![CDATA[6877746]]></arnumber>
##     <doi><![CDATA[10.1109/TTHZ.2014.2342497]]></doi>
##     <publicationId><![CDATA[6877746]]></publicationId>
##     <mdurl><![CDATA[http://ieeexplore.ieee.org/xpl/articleDetails.jsp?tp=&arnumber=6877746&contentType=Journals+%26+Magazines]]></mdurl>
##     <pdf><![CDATA[http://ieeexplore.ieee.org/stamp/stamp.jsp?arnumber=6877746]]></pdf>
##   </document>
##   <document>
##     <rank>60</rank>
##     <title><![CDATA[Blind Spectral Weighting for Robust Speaker Identification under Reverberation Mismatch]]></title>
##     <authors><![CDATA[Sadjadi, S.O.;  Hansen, J.H.L.]]></authors>
##     <affiliations><![CDATA[Center for Robust Speech Syst. (CRSS), Univ. of Texas at Dallas, Richardson, TX, USA]]></affiliations>
##     <controlledterms>
##       <term><![CDATA[anechoic chambers (acoustic)]]></term>
##       <term><![CDATA[feature extraction]]></term>
##       <term><![CDATA[reverberation]]></term>
##       <term><![CDATA[speaker recognition]]></term>
##       <term><![CDATA[speech intelligibility]]></term>
##     </controlledterms>
##     <thesaurusterms>
##       <term><![CDATA[Estimation]]></term>
##       <term><![CDATA[Licenses]]></term>
##       <term><![CDATA[Microphones]]></term>
##       <term><![CDATA[Reverberation]]></term>
##       <term><![CDATA[Speech]]></term>
##       <term><![CDATA[Speech enhancement]]></term>
##     </thesaurusterms>
##     <pubtitle><![CDATA[Audio, Speech, and Language Processing, IEEE/ACM Transactions on]]></pubtitle>
##     <punumber><![CDATA[6570655]]></punumber>
##     <pubtype><![CDATA[Journals & Magazines]]></pubtype>
##     <publisher><![CDATA[IEEE]]></publisher>
##     <volume><![CDATA[22]]></volume>
##     <issue><![CDATA[5]]></issue>
##     <py><![CDATA[2014]]></py>
##     <spage><![CDATA[937]]></spage>
##     <epage><![CDATA[945]]></epage>
##     <abstract><![CDATA[Room reverberation poses various deleterious effects on performance of automatic speech systems. Speaker identification (SID) performance, in particular, degrades rapidly as reverberation time increases. Reverberation causes two forms of spectro-temporal distortions on speech signals: i) self-masking which is due to early reflections and ii) overlap-masking which is due to late reverberation. Overlap-masking effect of reverberation has been shown to have a greater adverse impact on performance of speech systems. Motivated by this fact, this study proposes a blind spectral weighting (BSW) technique for suppressing the reverberation overlap-masking effect on SID systems. The technique is blind in the sense that prior knowledge of neither the anechoic signal nor the room impulse response is required. Performance of the proposed technique is evaluated on speaker verification tasks under simulated and actual reverberant mismatched conditions. Evaluations are conducted in the context of the conventional GMM-UBM as well as the state-of-the-art i-vector based systems. The GMM-UBM experiments are performed using speech material from a new data corpus well suited for speaker verification experiments under actual reverberant mismatched conditions, entitled MultiRoom8. The i-vector experiments are carried out with microphone (interview and phonecall) data from the NIST SRE 2010 extended evaluation set which are digitally convolved with three different measured room impulse responses extracted from the Aachen impulse response (AIR) database. Experimental results prove that incorporating the proposed blind technique into the standard MFCC feature extraction framework yields significant improvement in SID performance under reverberation mismatch.]]></abstract>
##     <issn><![CDATA[2329-9290]]></issn>
##     <htmlFlag><![CDATA[1]]></htmlFlag>
##     <arnumber><![CDATA[6762999]]></arnumber>
##     <doi><![CDATA[10.1109/TASLP.2014.2311329]]></doi>
##     <publicationId><![CDATA[6762999]]></publicationId>
##     <mdurl><![CDATA[http://ieeexplore.ieee.org/xpl/articleDetails.jsp?tp=&arnumber=6762999&contentType=Journals+%26+Magazines]]></mdurl>
##     <pdf><![CDATA[http://ieeexplore.ieee.org/stamp/stamp.jsp?arnumber=6762999]]></pdf>
##   </document>
##   <document>
##     <rank>61</rank>
##     <title><![CDATA[3-D Transient Analysis of TSV-Induced Substrate Noise: Improved Noise Reduction in 3-D-ICs With Incorporation of Guarding Structures]]></title>
##     <authors><![CDATA[Lin, L.J.-H.;  Yih-Peng Chiou]]></authors>
##     <affiliations><![CDATA[Dept. of Electr. Eng., Nat. Taiwan Univ., Taipei, Taiwan]]></affiliations>
##     <controlledterms>
##       <term><![CDATA[copper]]></term>
##       <term><![CDATA[integrated circuit modelling]]></term>
##       <term><![CDATA[integrated circuit noise]]></term>
##       <term><![CDATA[three-dimensional integrated circuits]]></term>
##       <term><![CDATA[transient analysis]]></term>
##     </controlledterms>
##     <thesaurusterms>
##       <term><![CDATA[Capacitance]]></term>
##       <term><![CDATA[Couplings]]></term>
##       <term><![CDATA[Noise]]></term>
##       <term><![CDATA[Substrates]]></term>
##       <term><![CDATA[Threshold voltage]]></term>
##       <term><![CDATA[Through-silicon vias]]></term>
##       <term><![CDATA[Transistors]]></term>
##     </thesaurusterms>
##     <pubtitle><![CDATA[Electron Device Letters, IEEE]]></pubtitle>
##     <punumber><![CDATA[55]]></punumber>
##     <pubtype><![CDATA[Journals & Magazines]]></pubtype>
##     <publisher><![CDATA[IEEE]]></publisher>
##     <volume><![CDATA[35]]></volume>
##     <issue><![CDATA[6]]></issue>
##     <py><![CDATA[2014]]></py>
##     <spage><![CDATA[660]]></spage>
##     <epage><![CDATA[662]]></epage>
##     <abstract><![CDATA[Substrate coupling in 3-D-ICs using Cu through silicon vias (TSVs) is a predicament widely documented in recent literature. Yet, discussions remain limited to the electromagnetic framework, such that a complete understanding of noise propagation and absorption is hampered. This letter thoroughly examines these phenomena in the TSVs from the integrated perspectives of semiconductor physics and electromagnetic theory and investigates the noise reduction method using the combination of p+ guard-ring and grounded TSV via 3-D device simulation.]]></abstract>
##     <issn><![CDATA[0741-3106]]></issn>
##     <htmlFlag><![CDATA[1]]></htmlFlag>
##     <arnumber><![CDATA[6814812]]></arnumber>
##     <doi><![CDATA[10.1109/LED.2014.2318301]]></doi>
##     <publicationId><![CDATA[6814812]]></publicationId>
##     <mdurl><![CDATA[http://ieeexplore.ieee.org/xpl/articleDetails.jsp?tp=&arnumber=6814812&contentType=Journals+%26+Magazines]]></mdurl>
##     <pdf><![CDATA[http://ieeexplore.ieee.org/stamp/stamp.jsp?arnumber=6814812]]></pdf>
##   </document>
##   <document>
##     <rank>62</rank>
##     <title><![CDATA[Scheduling in Dense Small Cells With Successive Interference Cancellation]]></title>
##     <authors><![CDATA[Ronghui Hou;  Yarong Xie;  King-Shan Lui;  Jiandong Li]]></authors>
##     <affiliations><![CDATA[State Key Lab. of Integrated Service Networks, Xidian Univ., Xian, China]]></affiliations>
##     <controlledterms>
##       <term><![CDATA[cellular radio]]></term>
##       <term><![CDATA[interference suppression]]></term>
##       <term><![CDATA[scheduling]]></term>
##     </controlledterms>
##     <thesaurusterms>
##       <term><![CDATA[Decoding]]></term>
##       <term><![CDATA[Interference]]></term>
##       <term><![CDATA[Mobile communication]]></term>
##       <term><![CDATA[Optimal scheduling]]></term>
##       <term><![CDATA[Scattering]]></term>
##       <term><![CDATA[Silicon carbide]]></term>
##       <term><![CDATA[Throughput]]></term>
##     </thesaurusterms>
##     <pubtitle><![CDATA[Communications Letters, IEEE]]></pubtitle>
##     <punumber><![CDATA[4234]]></punumber>
##     <pubtype><![CDATA[Journals & Magazines]]></pubtype>
##     <publisher><![CDATA[IEEE]]></publisher>
##     <volume><![CDATA[18]]></volume>
##     <issue><![CDATA[6]]></issue>
##     <py><![CDATA[2014]]></py>
##     <spage><![CDATA[1035]]></spage>
##     <epage><![CDATA[1038]]></epage>
##     <abstract><![CDATA[Smart interference management has been receiving attention to improve network throughput. Successive interference cancellation (SIC) is one of the promising techniques, which allows multiple concurrent transmissions from different transmitters to the same receiver. We study the scheduling issue with SIC in dense small cells. This paper proposes a novel scheduling framework, which facilitates us to develop practical algorithms to find the solution.]]></abstract>
##     <issn><![CDATA[1089-7798]]></issn>
##     <htmlFlag><![CDATA[1]]></htmlFlag>
##     <arnumber><![CDATA[6803899]]></arnumber>
##     <doi><![CDATA[10.1109/LCOMM.2014.2318310]]></doi>
##     <publicationId><![CDATA[6803899]]></publicationId>
##     <mdurl><![CDATA[http://ieeexplore.ieee.org/xpl/articleDetails.jsp?tp=&arnumber=6803899&contentType=Journals+%26+Magazines]]></mdurl>
##     <pdf><![CDATA[http://ieeexplore.ieee.org/stamp/stamp.jsp?arnumber=6803899]]></pdf>
##   </document>
##   <document>
##     <rank>63</rank>
##     <title><![CDATA[Assessment of Laparoscopic Skills Based on Force and Motion Parameters]]></title>
##     <authors><![CDATA[Horeman, T.;  Dankelman, J.;  Jansen, F.W.;  van den Dobbelsteen, J.J.]]></authors>
##     <affiliations><![CDATA[Dept. of Biomech. Eng., Delft Univ. of Technol., Delft, Netherlands]]></affiliations>
##     <controlledterms>
##       <term><![CDATA[biological tissues]]></term>
##       <term><![CDATA[biomechanics]]></term>
##       <term><![CDATA[biomedical equipment]]></term>
##       <term><![CDATA[force control]]></term>
##       <term><![CDATA[surgery]]></term>
##       <term><![CDATA[training]]></term>
##     </controlledterms>
##     <thesaurusterms>
##       <term><![CDATA[Biomedical measurement]]></term>
##       <term><![CDATA[Correlation]]></term>
##       <term><![CDATA[Force]]></term>
##       <term><![CDATA[Instruments]]></term>
##       <term><![CDATA[Laparoscopes]]></term>
##       <term><![CDATA[Surgery]]></term>
##       <term><![CDATA[Training]]></term>
##     </thesaurusterms>
##     <pubtitle><![CDATA[Biomedical Engineering, IEEE Transactions on]]></pubtitle>
##     <punumber><![CDATA[10]]></punumber>
##     <pubtype><![CDATA[Journals & Magazines]]></pubtype>
##     <publisher><![CDATA[IEEE]]></publisher>
##     <volume><![CDATA[61]]></volume>
##     <issue><![CDATA[3]]></issue>
##     <py><![CDATA[2014]]></py>
##     <spage><![CDATA[805]]></spage>
##     <epage><![CDATA[813]]></epage>
##     <abstract><![CDATA[Box trainers equipped with sensors may help in acquiring objective information about a trainee's performance while performing training tasks with real instruments. The main aim of this study is to investigate the added value of force parameters with respect to commonly used motion and time parameters such as path length, motion volume, and task time. Two new dynamic bimanual positioning tasks were developed that not only requiring adequate motion control but also appropriate force control successful completion. Force and motion data for these tasks were studied for three groups of participants with different experience levels in laparoscopy (i.e., 11 novices, 19 intermediates, and 12 experts). In total, 10 of the 13 parameters showed a significant difference between groups. When the data from the significant motion, time, and force parameters are used for classification, it is possible to identify the skills level of the participants with 100% accuracy. Furthermore, the force parameters of many individuals in the intermediate group exceeded the maximum values in the novice and expert group. The relatively high forces used by the intermediates argue for the inclusion of training and assessment of force application during tissue handling in future laparoscopic skills training programs.]]></abstract>
##     <issn><![CDATA[0018-9294]]></issn>
##     <htmlFlag><![CDATA[1]]></htmlFlag>
##     <arnumber><![CDATA[6657797]]></arnumber>
##     <doi><![CDATA[10.1109/TBME.2013.2290052]]></doi>
##     <publicationId><![CDATA[6657797]]></publicationId>
##     <mdurl><![CDATA[http://ieeexplore.ieee.org/xpl/articleDetails.jsp?tp=&arnumber=6657797&contentType=Journals+%26+Magazines]]></mdurl>
##     <pdf><![CDATA[http://ieeexplore.ieee.org/stamp/stamp.jsp?arnumber=6657797]]></pdf>
##   </document>
##   <document>
##     <rank>64</rank>
##     <title><![CDATA[Multichannel Continuously Tunable Microwave Phase Shifter With Capability of Frequency Doubling]]></title>
##     <authors><![CDATA[Zhenhua Feng;  Songnian Fu;  Tang Ming;  Deming Liu]]></authors>
##     <affiliations><![CDATA[Nat. Eng. Lab. for Next Generation Internet Access Syst., Huazhong Univ. of Sci. & Technol., Wuhan, China]]></affiliations>
##     <controlledterms>
##       <term><![CDATA[microwave generation]]></term>
##       <term><![CDATA[microwave phase shifters]]></term>
##       <term><![CDATA[microwave photonics]]></term>
##       <term><![CDATA[optical filters]]></term>
##       <term><![CDATA[optical harmonic generation]]></term>
##       <term><![CDATA[optical modulation]]></term>
##       <term><![CDATA[optical pulse shaping]]></term>
##       <term><![CDATA[phase modulation]]></term>
##       <term><![CDATA[photodetectors]]></term>
##     </controlledterms>
##     <thesaurusterms>
##       <term><![CDATA[Microwave amplifiers]]></term>
##       <term><![CDATA[Microwave filters]]></term>
##       <term><![CDATA[Microwave photonics]]></term>
##       <term><![CDATA[Optical filters]]></term>
##       <term><![CDATA[Optical harmonic generation]]></term>
##       <term><![CDATA[Phase shifters]]></term>
##     </thesaurusterms>
##     <pubtitle><![CDATA[Photonics Journal, IEEE]]></pubtitle>
##     <punumber><![CDATA[4563994]]></punumber>
##     <pubtype><![CDATA[Journals & Magazines]]></pubtype>
##     <publisher><![CDATA[IEEE]]></publisher>
##     <volume><![CDATA[6]]></volume>
##     <issue><![CDATA[1]]></issue>
##     <py><![CDATA[2014]]></py>
##     <spage><![CDATA[1]]></spage>
##     <epage><![CDATA[8]]></epage>
##     <abstract><![CDATA[A photonic generation of tunable microwave phase shifter with capability of frequency doubling is proposed and experimentally demonstrated using optical spectral shaping technique. Our proposed configuration only consists of a phase modulator (PM), a programmable optical filter, and one photodetector (PD). The phase modulated optical signal is fed into the programmable optical filter, where two specific harmonic sidebands of the modulated signal are chosen in order to apply a tunable phase shift to one of the selected sidebands. After optical amplification and beating at the PD, a frequency doubling microwave signal with continuously tunable phase shifter from 0&#x00B0; to 360&#x00B0; is obtained with less than 2.5 dB power fluctuation over a frequency range from 20 GHz to 33 GHz. Moreover, simultaneous four-channel microwave signals with independent phase shift setting are achieved, due to that the proposed configuration is compatible with wavelength division multiplexing (WDM) operation.]]></abstract>
##     <issn><![CDATA[1943-0655]]></issn>
##     <htmlFlag><![CDATA[1]]></htmlFlag>
##     <arnumber><![CDATA[6690170]]></arnumber>
##     <doi><![CDATA[10.1109/JPHOT.2013.2295457]]></doi>
##     <publicationId><![CDATA[6690170]]></publicationId>
##     <mdurl><![CDATA[http://ieeexplore.ieee.org/xpl/articleDetails.jsp?tp=&arnumber=6690170&contentType=Journals+%26+Magazines]]></mdurl>
##     <pdf><![CDATA[http://ieeexplore.ieee.org/stamp/stamp.jsp?arnumber=6690170]]></pdf>
##   </document>
##   <document>
##     <rank>65</rank>
##     <title><![CDATA[Broadband Visible Light Emission From Nominally Undoped and <inline-formula> <img src="/images/tex/21773.gif" alt="\hbox {Cr}^{3+}"> </inline-formula> Doped Garnet Nanopowders]]></title>
##     <authors><![CDATA[Bilir, G.;  Ozen, G.;  Bettinelli, M.;  Piccinelli, F.;  Cesaria, M.;  Di Bartolo, B.]]></authors>
##     <affiliations><![CDATA[Dept. of Phys., Istanbul Tech. Univ., Istanbul, Turkey]]></affiliations>
##     <controlledterms>
##       <term><![CDATA[chromium]]></term>
##       <term><![CDATA[gadolinium compounds]]></term>
##       <term><![CDATA[garnets]]></term>
##       <term><![CDATA[heating]]></term>
##       <term><![CDATA[laser beam effects]]></term>
##       <term><![CDATA[melting]]></term>
##       <term><![CDATA[nanoparticles]]></term>
##       <term><![CDATA[optical pumping]]></term>
##       <term><![CDATA[photoluminescence]]></term>
##       <term><![CDATA[yttrium compounds]]></term>
##     </controlledterms>
##     <thesaurusterms>
##       <term><![CDATA[Broadband communication]]></term>
##       <term><![CDATA[Garnets]]></term>
##       <term><![CDATA[Laser excitation]]></term>
##       <term><![CDATA[Phosphors]]></term>
##       <term><![CDATA[Powders]]></term>
##       <term><![CDATA[Temperature measurement]]></term>
##     </thesaurusterms>
##     <pubtitle><![CDATA[Photonics Journal, IEEE]]></pubtitle>
##     <punumber><![CDATA[4563994]]></punumber>
##     <pubtype><![CDATA[Journals & Magazines]]></pubtype>
##     <publisher><![CDATA[IEEE]]></publisher>
##     <volume><![CDATA[6]]></volume>
##     <issue><![CDATA[4]]></issue>
##     <py><![CDATA[2014]]></py>
##     <spage><![CDATA[1]]></spage>
##     <epage><![CDATA[11]]></epage>
##     <abstract><![CDATA[Synthetic garnet nanopowders of Y<sub>3</sub>Al<sub>5</sub>O<sub>12</sub> (YAG) and Gd<sub>3</sub>Ga<sub>5</sub>O<sub>12</sub> (GGG) were produced, and the occurrence of a broadband bright visible emission by nominally undoped YAG and GGG and Cr<sup>3+</sup> doped GGG, depending on the environment pressure, as well as exciting on the pumping power, was demonstrated. The results indicate that high-intensity infrared laser irradiation in samples not only leads to heating (melting effects) but also produces visible broadband emission. Low pressure of the powders' environment favors the white light emission by lowering the threshold pumping power. A hypothesis on the nature of the emission is presented.]]></abstract>
##     <issn><![CDATA[1943-0655]]></issn>
##     <htmlFlag><![CDATA[1]]></htmlFlag>
##     <arnumber><![CDATA[6861947]]></arnumber>
##     <doi><![CDATA[10.1109/JPHOT.2014.2337873]]></doi>
##     <publicationId><![CDATA[6861947]]></publicationId>
##     <mdurl><![CDATA[http://ieeexplore.ieee.org/xpl/articleDetails.jsp?tp=&arnumber=6861947&contentType=Journals+%26+Magazines]]></mdurl>
##     <pdf><![CDATA[http://ieeexplore.ieee.org/stamp/stamp.jsp?arnumber=6861947]]></pdf>
##   </document>
##   <document>
##     <rank>66</rank>
##     <title><![CDATA[A Vision of IoT: Applications, Challenges, and Opportunities With China Perspective]]></title>
##     <authors><![CDATA[Shanzhi Chen;  Hui Xu;  Dake Liu;  Bo Hu;  Hucheng Wang]]></authors>
##     <affiliations><![CDATA[State Key Lab. of Wireless Mobile Commun., China Acad. of Telecommun. Technol. (CATT), Beijing, China]]></affiliations>
##     <controlledterms>
##       <term><![CDATA[Internet of Things]]></term>
##       <term><![CDATA[research and development]]></term>
##     </controlledterms>
##     <thesaurusterms>
##       <term><![CDATA[Computer architecture]]></term>
##       <term><![CDATA[Industries]]></term>
##       <term><![CDATA[Monitoring]]></term>
##       <term><![CDATA[Sensors]]></term>
##       <term><![CDATA[Standards]]></term>
##       <term><![CDATA[Vehicles]]></term>
##     </thesaurusterms>
##     <pubtitle><![CDATA[Internet of Things Journal, IEEE]]></pubtitle>
##     <punumber><![CDATA[6488907]]></punumber>
##     <pubtype><![CDATA[Journals & Magazines]]></pubtype>
##     <publisher><![CDATA[IEEE]]></publisher>
##     <volume><![CDATA[1]]></volume>
##     <issue><![CDATA[4]]></issue>
##     <py><![CDATA[2014]]></py>
##     <spage><![CDATA[349]]></spage>
##     <epage><![CDATA[359]]></epage>
##     <abstract><![CDATA[Internet of Things (IoT), which will create a huge network of billions or trillions of &#x201C;Things&#x201D; communicating with one another, are facing many technical and application challenges. This paper introduces the status of IoT development in China, including policies, R&amp;D plans, applications, and standardization. With China's perspective, this paper depicts such challenges on technologies, applications, and standardization, and also proposes an open and general IoT architecture consisting of three platforms to meet the architecture challenge. Finally, this paper discusses the opportunity and prospect of IoT.]]></abstract>
##     <issn><![CDATA[2327-4662]]></issn>
##     <htmlFlag><![CDATA[1]]></htmlFlag>
##     <arnumber><![CDATA[6851114]]></arnumber>
##     <doi><![CDATA[10.1109/JIOT.2014.2337336]]></doi>
##     <publicationId><![CDATA[6851114]]></publicationId>
##     <mdurl><![CDATA[http://ieeexplore.ieee.org/xpl/articleDetails.jsp?tp=&arnumber=6851114&contentType=Journals+%26+Magazines]]></mdurl>
##     <pdf><![CDATA[http://ieeexplore.ieee.org/stamp/stamp.jsp?arnumber=6851114]]></pdf>
##   </document>
##   <document>
##     <rank>67</rank>
##     <title><![CDATA[Online Serial Manipulator Calibration Based on Multisensory Process Via Extended Kalman and Particle Filters]]></title>
##     <authors><![CDATA[Guanglong Du;  Ping Zhang]]></authors>
##     <affiliations><![CDATA[Sch. of Mech. & Automotive Eng., South China Univ. of Technol., Guangzhou, China]]></affiliations>
##     <controlledterms>
##       <term><![CDATA[Kalman filters]]></term>
##       <term><![CDATA[calibration]]></term>
##       <term><![CDATA[manipulator kinematics]]></term>
##       <term><![CDATA[nonlinear filters]]></term>
##       <term><![CDATA[parameter estimation]]></term>
##       <term><![CDATA[particle filtering (numerical methods)]]></term>
##       <term><![CDATA[position measurement]]></term>
##     </controlledterms>
##     <thesaurusterms>
##       <term><![CDATA[Kalman filters]]></term>
##       <term><![CDATA[Kinematics]]></term>
##       <term><![CDATA[Particle filters]]></term>
##       <term><![CDATA[Robot kinematics]]></term>
##       <term><![CDATA[Robot sensing systems]]></term>
##     </thesaurusterms>
##     <pubtitle><![CDATA[Industrial Electronics, IEEE Transactions on]]></pubtitle>
##     <punumber><![CDATA[41]]></punumber>
##     <pubtype><![CDATA[Journals & Magazines]]></pubtype>
##     <publisher><![CDATA[IEEE]]></publisher>
##     <volume><![CDATA[61]]></volume>
##     <issue><![CDATA[12]]></issue>
##     <py><![CDATA[2014]]></py>
##     <spage><![CDATA[6852]]></spage>
##     <epage><![CDATA[6859]]></epage>
##     <abstract><![CDATA[An online robot self-calibration method based on an inertial measurement unit (IMU) and a position sensor is presented in this paper. In this method, a position marker and an IMU are required to be rigidly attached to the robot tool to obtain the position of the manipulator from the position sensor and the orientation of the manipulator from the IMU in real time. An efficient approach that incorporates a Kalman filter (KF) and a particle filter to estimate the position and orientation of the manipulator is proposed in this paper. The use of these pose (orientation and position) estimation methods improves the reliability and accuracy of pose measurements. Finally, an extended KF is used to estimate the kinematic parameter errors. The primary advantage of this method over existing automated self-calibration methods is that it does not involve complex steps, such as camera calibration, corner detection, and laser alignment, which makes the proposed robot calibration procedure more autonomous in a dynamic manufacturing environment. Moreover, the reduction of complex steps improves the accuracy of calibration. Experimental studies on a GOOGOL GRB3016 robot show that the proposed method has better accuracy, convenience, and effectiveness.]]></abstract>
##     <issn><![CDATA[0278-0046]]></issn>
##     <htmlFlag><![CDATA[1]]></htmlFlag>
##     <arnumber><![CDATA[6779595]]></arnumber>
##     <doi><![CDATA[10.1109/TIE.2014.2314051]]></doi>
##     <publicationId><![CDATA[6779595]]></publicationId>
##     <mdurl><![CDATA[http://ieeexplore.ieee.org/xpl/articleDetails.jsp?tp=&arnumber=6779595&contentType=Journals+%26+Magazines]]></mdurl>
##     <pdf><![CDATA[http://ieeexplore.ieee.org/stamp/stamp.jsp?arnumber=6779595]]></pdf>
##   </document>
##   <document>
##     <rank>68</rank>
##     <title><![CDATA[Microcontrollers as (In)Security Devices for Pervasive Computing Applications]]></title>
##     <authors><![CDATA[Strobel, D.;  Oswald, D.;  Richter, B.;  Schellenberg, F.;  Paar, C.]]></authors>
##     <affiliations><![CDATA[Horst Gortz Inst. for IT-Security, Ruhr Univ. Bochum, Bochum, Germany]]></affiliations>
##     <controlledterms>
##       <term><![CDATA[Internet of Things]]></term>
##       <term><![CDATA[cryptography]]></term>
##       <term><![CDATA[embedded systems]]></term>
##       <term><![CDATA[microcontrollers]]></term>
##       <term><![CDATA[ubiquitous computing]]></term>
##     </controlledterms>
##     <thesaurusterms>
##       <term><![CDATA[Algorithm design and analysis]]></term>
##       <term><![CDATA[Cryptography]]></term>
##       <term><![CDATA[Embedded systems]]></term>
##       <term><![CDATA[Field programmable gate arrays]]></term>
##       <term><![CDATA[Integrated circuit modeling]]></term>
##       <term><![CDATA[Microcontrollers]]></term>
##       <term><![CDATA[Pervasive computing]]></term>
##       <term><![CDATA[Security]]></term>
##     </thesaurusterms>
##     <pubtitle><![CDATA[Proceedings of the IEEE]]></pubtitle>
##     <punumber><![CDATA[5]]></punumber>
##     <pubtype><![CDATA[Journals & Magazines]]></pubtype>
##     <publisher><![CDATA[IEEE]]></publisher>
##     <volume><![CDATA[102]]></volume>
##     <issue><![CDATA[8]]></issue>
##     <py><![CDATA[2014]]></py>
##     <spage><![CDATA[1157]]></spage>
##     <epage><![CDATA[1173]]></epage>
##     <abstract><![CDATA[Often overlooked, microcontrollers are the central component in embedded systems which drive the evolution toward the Internet of Things (IoT). They are small, easy to handle, low cost, and with myriads of pervasive applications. An increasing number of microcontroller-equipped systems are security and safety critical. In this tutorial, we take a critical look at the security aspects of today's microcontrollers. We demonstrate why the implementation of sensitive applications on a standard microcontroller can lead to severe security problems. To this end, we summarize various threats to microcontroller-based systems, including side-channel analysis and different methods for extracting embedded code. In two case studies, we demonstrate the relevance of these techniques in real-world applications: Both analyzed systems, a widely used digital locking system and the YubiKey 2 onetime password generator, turned out to be susceptible to attacks against the actual implementations, allowing an adversary to extract the cryptographic keys which, in turn, leads to a total collapse of the system security.]]></abstract>
##     <issn><![CDATA[0018-9219]]></issn>
##     <htmlFlag><![CDATA[1]]></htmlFlag>
##     <arnumber><![CDATA[6826474]]></arnumber>
##     <doi><![CDATA[10.1109/JPROC.2014.2325397]]></doi>
##     <publicationId><![CDATA[6826474]]></publicationId>
##     <mdurl><![CDATA[http://ieeexplore.ieee.org/xpl/articleDetails.jsp?tp=&arnumber=6826474&contentType=Journals+%26+Magazines]]></mdurl>
##     <pdf><![CDATA[http://ieeexplore.ieee.org/stamp/stamp.jsp?arnumber=6826474]]></pdf>
##   </document>
##   <document>
##     <rank>69</rank>
##     <title><![CDATA[Estimating Speed Using a Side-Looking Single-Radar Vehicle Detector]]></title>
##     <authors><![CDATA[Shyr-Long Jeng;  Wei-Hua Chieng;  Hsiang-Pin Lu]]></authors>
##     <affiliations><![CDATA[Dept. of Electr. & Electron. Eng., Ta Hwa Univ. of Sci. & Technol., Hsinchu, Taiwan]]></affiliations>
##     <controlledterms>
##       <term><![CDATA[CW radar]]></term>
##       <term><![CDATA[FM radar]]></term>
##       <term><![CDATA[intelligent transportation systems]]></term>
##       <term><![CDATA[road vehicle radar]]></term>
##       <term><![CDATA[synthetic aperture radar]]></term>
##       <term><![CDATA[velocity measurement]]></term>
##     </controlledterms>
##     <thesaurusterms>
##       <term><![CDATA[Detectors]]></term>
##       <term><![CDATA[Doppler effect]]></term>
##       <term><![CDATA[Doppler radar]]></term>
##       <term><![CDATA[Frequency modulation]]></term>
##       <term><![CDATA[Radar detection]]></term>
##       <term><![CDATA[Vehicles]]></term>
##     </thesaurusterms>
##     <pubtitle><![CDATA[Intelligent Transportation Systems, IEEE Transactions on]]></pubtitle>
##     <punumber><![CDATA[6979]]></punumber>
##     <pubtype><![CDATA[Journals & Magazines]]></pubtype>
##     <publisher><![CDATA[IEEE]]></publisher>
##     <volume><![CDATA[15]]></volume>
##     <issue><![CDATA[2]]></issue>
##     <py><![CDATA[2014]]></py>
##     <spage><![CDATA[607]]></spage>
##     <epage><![CDATA[614]]></epage>
##     <abstract><![CDATA[This paper presents a side-looking single-beam microwave vehicle detector (VD) system for estimation of per-vehicle speed and length. The proposed VD system is equipped with a 2-D range Doppler frequency-modulated continuous-wave (FMCW) radar using a squint angle. The associated Fourier processor uses an inverse synthetic aperture radar (ISAR) algorithm to extract range and speed data for each vehicle using a single-beam FMCW radar. The simulation and experimental results show accurate estimations of vehicle speed and length. The measurement errors of speed and length were approximately &#x00B1;4 km/h and &#x00B1;1 m, respectively. The proposed method has excellent detection capability for small moving targets, such as bikes and pedestrians, at speeds down to 5 km/h. A commercial 10.6-GHz radar with signal processing modifications was used in the experiments.]]></abstract>
##     <issn><![CDATA[1524-9050]]></issn>
##     <arnumber><![CDATA[6636147]]></arnumber>
##     <doi><![CDATA[10.1109/TITS.2013.2283528]]></doi>
##     <publicationId><![CDATA[6636147]]></publicationId>
##     <mdurl><![CDATA[http://ieeexplore.ieee.org/xpl/articleDetails.jsp?tp=&arnumber=6636147&contentType=Journals+%26+Magazines]]></mdurl>
##     <pdf><![CDATA[http://ieeexplore.ieee.org/stamp/stamp.jsp?arnumber=6636147]]></pdf>
##   </document>
##   <document>
##     <rank>70</rank>
##     <title><![CDATA[Optical MEMS: From Micromirrors to Complex Systems]]></title>
##     <authors><![CDATA[Solgaard, O.;  Godil, A.A.;  Howe, R.T.;  Lee, L.P.;  Peter, Y.-A.;  Zappe, H.]]></authors>
##     <affiliations><![CDATA[Dept. of Electr. Eng., Stanford Univ., Stanford, CA, USA]]></affiliations>
##     <controlledterms>
##       <term><![CDATA[beam steering]]></term>
##       <term><![CDATA[bioMEMS]]></term>
##       <term><![CDATA[biomedical equipment]]></term>
##       <term><![CDATA[diffractive optical elements]]></term>
##       <term><![CDATA[electrostatic actuators]]></term>
##       <term><![CDATA[integrated optics]]></term>
##       <term><![CDATA[integrated optoelectronics]]></term>
##       <term><![CDATA[lenses]]></term>
##       <term><![CDATA[magnetic actuators]]></term>
##       <term><![CDATA[micro-optomechanical devices]]></term>
##       <term><![CDATA[microcavities]]></term>
##       <term><![CDATA[microfluidics]]></term>
##       <term><![CDATA[micromachining]]></term>
##       <term><![CDATA[micromirrors]]></term>
##       <term><![CDATA[microsensors]]></term>
##       <term><![CDATA[nanophotonics]]></term>
##       <term><![CDATA[optical fabrication]]></term>
##       <term><![CDATA[optical fibre filters]]></term>
##       <term><![CDATA[optical sensors]]></term>
##       <term><![CDATA[optical switches]]></term>
##       <term><![CDATA[photonic crystals]]></term>
##       <term><![CDATA[pneumatic actuators]]></term>
##     </controlledterms>
##     <thesaurusterms>
##       <term><![CDATA[Biomedical optical imaging]]></term>
##       <term><![CDATA[Lenses]]></term>
##       <term><![CDATA[Micromechanical devices]]></term>
##       <term><![CDATA[Optical device fabrication]]></term>
##       <term><![CDATA[Optical sensors]]></term>
##       <term><![CDATA[Silicon]]></term>
##     </thesaurusterms>
##     <pubtitle><![CDATA[Microelectromechanical Systems, Journal of]]></pubtitle>
##     <punumber><![CDATA[84]]></punumber>
##     <pubtype><![CDATA[Journals & Magazines]]></pubtype>
##     <publisher><![CDATA[IEEE]]></publisher>
##     <volume><![CDATA[23]]></volume>
##     <issue><![CDATA[3]]></issue>
##     <py><![CDATA[2014]]></py>
##     <spage><![CDATA[517]]></spage>
##     <epage><![CDATA[538]]></epage>
##     <abstract><![CDATA[Microelectromechanical system (MEMS) technology, and surface micromachining in particular, have led to the development of miniaturized optical devices with a substantial impact in a large number of application areas. The reason is the unique MEMS characteristics that are advantageous in fabrication, systems integration, and operation of micro-optical systems. The precision mechanics of MEMS, microfabrication techniques, and optical functionality all make possible a wide variety of movable and tunable mirrors, lenses, filters, and other optical structures. In these systems, electrostatic, magnetic, thermal, and pneumatic actuators provide mechanical precision and control. The large number of electromagnetic modes that can be accommodated by beam-steering micromirrors and diffractive optical MEMS, combined with the precision of these types of elements, is utilized in fiber-optical switches and filters, including dispersion compensators. The potential to integrate optics with electronics and mechanics is a great advantage in biomedical instrumentation, where the integration of miniaturized optical detection systems with microfluidics enables smaller, faster, more-functional, and cheaper systems. The precise dimensions and alignment of MEMS devices, combined with the mechanical stability that comes with miniaturization, make optical MEMS sensors well suited to a variety of challenging measurements. Micro-optical systems also benefit from the addition of nanostructures to the MEMS toolbox. Photonic crystals and microcavities, which represent the ultimate in miniaturized optical components, enable further scaling of optical MEMS.]]></abstract>
##     <issn><![CDATA[1057-7157]]></issn>
##     <htmlFlag><![CDATA[1]]></htmlFlag>
##     <arnumber><![CDATA[6817527]]></arnumber>
##     <doi><![CDATA[10.1109/JMEMS.2014.2319266]]></doi>
##     <publicationId><![CDATA[6817527]]></publicationId>
##     <mdurl><![CDATA[http://ieeexplore.ieee.org/xpl/articleDetails.jsp?tp=&arnumber=6817527&contentType=Journals+%26+Magazines]]></mdurl>
##     <pdf><![CDATA[http://ieeexplore.ieee.org/stamp/stamp.jsp?arnumber=6817527]]></pdf>
##   </document>
##   <document>
##     <rank>71</rank>
##     <title><![CDATA[An Overview of Background Modeling for Detection of Targets and Anomalies in Hyperspectral Remotely Sensed Imagery]]></title>
##     <authors><![CDATA[Matteoli, S.;  Diani, M.;  Theiler, J.]]></authors>
##     <affiliations><![CDATA[Inf. Eng. Dept., Univ. of Pisa, Pisa, Italy]]></affiliations>
##     <controlledterms>
##       <term><![CDATA[geophysical image processing]]></term>
##       <term><![CDATA[geophysical techniques]]></term>
##       <term><![CDATA[hyperspectral imaging]]></term>
##       <term><![CDATA[remote sensing]]></term>
##     </controlledterms>
##     <thesaurusterms>
##       <term><![CDATA[Adaptation models]]></term>
##       <term><![CDATA[Covariance matrices]]></term>
##       <term><![CDATA[Detection algorithms]]></term>
##       <term><![CDATA[Detectors]]></term>
##       <term><![CDATA[Hyperspectral imaging]]></term>
##       <term><![CDATA[Vectors]]></term>
##     </thesaurusterms>
##     <pubtitle><![CDATA[Selected Topics in Applied Earth Observations and Remote Sensing, IEEE Journal of]]></pubtitle>
##     <punumber><![CDATA[4609443]]></punumber>
##     <pubtype><![CDATA[Journals & Magazines]]></pubtype>
##     <publisher><![CDATA[IEEE]]></publisher>
##     <volume><![CDATA[7]]></volume>
##     <issue><![CDATA[6]]></issue>
##     <py><![CDATA[2014]]></py>
##     <spage><![CDATA[2317]]></spage>
##     <epage><![CDATA[2336]]></epage>
##     <abstract><![CDATA[This paper reviews well-known classic algorithms and more recent experimental approaches for distinguishing the weak signal of a target (either known or anomalous) from the cluttered background of a hyperspectral image. Making this distinction requires characterization of the targets and characterization of the backgrounds, and our emphasis in this review is on the backgrounds. We describe a variety of background modeling strategies-Gaussian and non-Gaussian, global and local, generative and discriminative, parametric and nonparametric, spectral and spatio-spectral-in the context of how they relate to the target and anomaly detection problems. We discuss the major issues addressed by these algorithms, and some of the tradeoffs made in choosing an effective algorithm for a given detection application. We identify connections among these algorithms and point out directions where innovative modeling strategies may be developed into detection algorithms that are more sensitive and reliable.]]></abstract>
##     <issn><![CDATA[1939-1404]]></issn>
##     <htmlFlag><![CDATA[1]]></htmlFlag>
##     <arnumber><![CDATA[6810158]]></arnumber>
##     <doi><![CDATA[10.1109/JSTARS.2014.2315772]]></doi>
##     <publicationId><![CDATA[6810158]]></publicationId>
##     <mdurl><![CDATA[http://ieeexplore.ieee.org/xpl/articleDetails.jsp?tp=&arnumber=6810158&contentType=Journals+%26+Magazines]]></mdurl>
##     <pdf><![CDATA[http://ieeexplore.ieee.org/stamp/stamp.jsp?arnumber=6810158]]></pdf>
##   </document>
##   <document>
##     <rank>72</rank>
##     <title><![CDATA[Iterative Reconstruction for X-Ray Computed Tomography Using Prior-Image Induced Nonlocal Regularization]]></title>
##     <authors><![CDATA[Hua Zhang;  Jing Huang;  Jianhua Ma;  Zhaoying Bian;  Qianjin Feng;  Hongbing Lu;  Zhengrong Liang;  Wufan Chen]]></authors>
##     <affiliations><![CDATA[Sch. of Biomed. Eng., Southern Med. Univ., Guangzhou, China]]></affiliations>
##     <controlledterms>
##       <term><![CDATA[computerised tomography]]></term>
##       <term><![CDATA[dosimetry]]></term>
##       <term><![CDATA[image denoising]]></term>
##       <term><![CDATA[image reconstruction]]></term>
##       <term><![CDATA[iterative methods]]></term>
##       <term><![CDATA[least squares approximations]]></term>
##       <term><![CDATA[medical image processing]]></term>
##       <term><![CDATA[object detection]]></term>
##       <term><![CDATA[phantoms]]></term>
##       <term><![CDATA[statistical analysis]]></term>
##     </controlledterms>
##     <thesaurusterms>
##       <term><![CDATA[Computed tomography]]></term>
##       <term><![CDATA[Educational institutions]]></term>
##       <term><![CDATA[Image edge detection]]></term>
##       <term><![CDATA[Image reconstruction]]></term>
##       <term><![CDATA[Noise]]></term>
##       <term><![CDATA[Phantoms]]></term>
##       <term><![CDATA[X-ray imaging]]></term>
##     </thesaurusterms>
##     <pubtitle><![CDATA[Biomedical Engineering, IEEE Transactions on]]></pubtitle>
##     <punumber><![CDATA[10]]></punumber>
##     <pubtype><![CDATA[Journals & Magazines]]></pubtype>
##     <publisher><![CDATA[IEEE]]></publisher>
##     <volume><![CDATA[61]]></volume>
##     <issue><![CDATA[9]]></issue>
##     <py><![CDATA[2014]]></py>
##     <spage><![CDATA[2367]]></spage>
##     <epage><![CDATA[2378]]></epage>
##     <abstract><![CDATA[Repeated X-ray computed tomography (CT) scans are often required in several specific applications such as perfusion imaging, image-guided biopsy needle, image-guided intervention, and radiotherapy with noticeable benefits. However, the associated cumulative radiation dose significantly increases as comparison with that used in the conventional CT scan, which has raised major concerns in patients. In this study, to realize radiation dose reduction by reducing the X-ray tube current and exposure time (mAs) in repeated CT scans, we propose a prior-image induced nonlocal (PINL) regularization for statistical iterative reconstruction via the penalized weighted least-squares (PWLS) criteria, which we refer to as &#x201C;PWLS-PINL&#x201D;. Specifically, the PINL regularization utilizes the redundant information in the prior image and the weighted least-squares term considers a data-dependent variance estimation, aiming to improve current low-dose image quality. Subsequently, a modified iterative successive overrelaxation algorithm is adopted to optimize the associative objective function. Experimental results on both phantom and patient data show that the present PWLS-PINL method can achieve promising gains over the other existing methods in terms of the noise reduction, low-contrast object detection, and edge detail preservation.]]></abstract>
##     <issn><![CDATA[0018-9294]]></issn>
##     <htmlFlag><![CDATA[1]]></htmlFlag>
##     <arnumber><![CDATA[6646222]]></arnumber>
##     <doi><![CDATA[10.1109/TBME.2013.2287244]]></doi>
##     <publicationId><![CDATA[6646222]]></publicationId>
##     <mdurl><![CDATA[http://ieeexplore.ieee.org/xpl/articleDetails.jsp?tp=&arnumber=6646222&contentType=Journals+%26+Magazines]]></mdurl>
##     <pdf><![CDATA[http://ieeexplore.ieee.org/stamp/stamp.jsp?arnumber=6646222]]></pdf>
##   </document>
##   <document>
##     <rank>73</rank>
##     <title><![CDATA[Can Transient Phenomena Help Improving Time Resolution in Scintillators?]]></title>
##     <authors><![CDATA[Lecoq, P.;  Korzhik, M.;  Vasiliev, A.]]></authors>
##     <affiliations><![CDATA[CERN, Geneva, Switzerland]]></affiliations>
##     <controlledterms>
##       <term><![CDATA[hot carriers]]></term>
##       <term><![CDATA[ionisation]]></term>
##       <term><![CDATA[phonons]]></term>
##       <term><![CDATA[photoluminescence]]></term>
##       <term><![CDATA[refractive index]]></term>
##       <term><![CDATA[shock waves]]></term>
##       <term><![CDATA[solid scintillation detectors]]></term>
##     </controlledterms>
##     <thesaurusterms>
##       <term><![CDATA[Absorption]]></term>
##       <term><![CDATA[Charge carrier processes]]></term>
##       <term><![CDATA[Crystals]]></term>
##       <term><![CDATA[Luminescence]]></term>
##       <term><![CDATA[Phonons]]></term>
##       <term><![CDATA[Photonics]]></term>
##       <term><![CDATA[Transient analysis]]></term>
##     </thesaurusterms>
##     <pubtitle><![CDATA[Nuclear Science, IEEE Transactions on]]></pubtitle>
##     <punumber><![CDATA[23]]></punumber>
##     <pubtype><![CDATA[Journals & Magazines]]></pubtype>
##     <publisher><![CDATA[IEEE]]></publisher>
##     <volume><![CDATA[61]]></volume>
##     <issue><![CDATA[1]]></issue>
##     <part><![CDATA[2]]></part>
##     <py><![CDATA[2014]]></py>
##     <spage><![CDATA[229]]></spage>
##     <epage><![CDATA[234]]></epage>
##     <abstract><![CDATA[The time resolution of a scintillator-based detector is directly driven by the density of photoelectrons generated in the photodetector at the detection threshold. At the scintillator level it is related to the intrinsic light yield, the pulse shape (rise time and decay time) and the light transport from the gamma-ray conversion point to the photodetector. When aiming at 10 ps time resolution, fluctuations in the thermalization and relaxation time of hot electrons and holes generated by the interaction of ionization radiation with the crystal become important. These processes last for up to a few tens of ps and are followed by a complex trapping-detrapping process, Poole-Frenkel effect, Auger ionization of traps and electron-hole recombination, which can last for a few ns with very large fluctuations. This paper will review the different processes at work and evaluate if some of the transient phenomena taking place during the fast thermalization phase can be exploited to extract a time tag with a precision in the few ps range. A very interesting part of the sequence is when the hot electrons and holes pass below the limit of the ionization threshold. The only way to relax their energy is then through collisions with the lattice resulting in the production of optical and acoustic phonons with relatively high energy (up to several tens of meV) near the ionization threshold. As the rate of such electron/phonon exchange is about 100 events/ps/electron or hole and as the number of electrons/holes generated after mutiplication in a high light yield scintillator like LSO can be as high as 100,000 or more, we end up with an energy deposition rate of about 100 KeV/ps. This energy deposition rate contributes to many fast processes with a characteristic time in the ps range such as band-to-band luminescence, hot intraband luminescence, acoustic shock wave generation, fast local variation of index of refraction, etc. We will discuss if the part of the total energy which is rel- ased this way, and which can represent between 50% and 90% of the energy of the incoming ionization radiation, can be efficiently exploited to improve the time resolution of scintillators, presently limited to the 100 ps range.]]></abstract>
##     <issn><![CDATA[0018-9499]]></issn>
##     <htmlFlag><![CDATA[1]]></htmlFlag>
##     <arnumber><![CDATA[6651724]]></arnumber>
##     <doi><![CDATA[10.1109/TNS.2013.2282232]]></doi>
##     <publicationId><![CDATA[6651724]]></publicationId>
##     <mdurl><![CDATA[http://ieeexplore.ieee.org/xpl/articleDetails.jsp?tp=&arnumber=6651724&contentType=Journals+%26+Magazines]]></mdurl>
##     <pdf><![CDATA[http://ieeexplore.ieee.org/stamp/stamp.jsp?arnumber=6651724]]></pdf>
##   </document>
##   <document>
##     <rank>74</rank>
##     <title><![CDATA[Compact Airborne C-Band Radar Sounder]]></title>
##     <authors><![CDATA[Oyan, M.J.;  Hamran, S.-E.;  Damsgard, L.;  Berger, T.]]></authors>
##     <affiliations><![CDATA[Norwegian Defence Res. Establ. (FFI), Kjeller, Norway]]></affiliations>
##     <controlledterms>
##       <term><![CDATA[CW radar]]></term>
##       <term><![CDATA[FM radar]]></term>
##       <term><![CDATA[airborne radar]]></term>
##       <term><![CDATA[antennas]]></term>
##       <term><![CDATA[autonomous aerial vehicles]]></term>
##       <term><![CDATA[data acquisition]]></term>
##       <term><![CDATA[glaciology]]></term>
##       <term><![CDATA[hydrological equipment]]></term>
##       <term><![CDATA[ice]]></term>
##       <term><![CDATA[remote sensing by radar]]></term>
##       <term><![CDATA[snow]]></term>
##     </controlledterms>
##     <thesaurusterms>
##       <term><![CDATA[Airborne radar]]></term>
##       <term><![CDATA[Delays]]></term>
##       <term><![CDATA[Ground penetrating radar]]></term>
##       <term><![CDATA[Radar antennas]]></term>
##       <term><![CDATA[Receivers]]></term>
##       <term><![CDATA[Spaceborne radar]]></term>
##     </thesaurusterms>
##     <pubtitle><![CDATA[Geoscience and Remote Sensing, IEEE Transactions on]]></pubtitle>
##     <punumber><![CDATA[36]]></punumber>
##     <pubtype><![CDATA[Journals & Magazines]]></pubtype>
##     <publisher><![CDATA[IEEE]]></publisher>
##     <volume><![CDATA[52]]></volume>
##     <issue><![CDATA[10]]></issue>
##     <py><![CDATA[2014]]></py>
##     <spage><![CDATA[6326]]></spage>
##     <epage><![CDATA[6332]]></epage>
##     <abstract><![CDATA[We describe a lightweight wideband C-band radar sounder, which is designed for use in a small unmanned aerial vehicle, primarily for measuring snow and ice. The waveform used is a gated frequency modulated continuous wave, which enables transmission at low power with relatively high energy in the compressed pulse. Gating allows use of a single antenna, reducing the influence of the direct wave on the systems' dynamic range. The radar operates at 5.3-GHz center frequency and has a bandwidth of 1 GHz, for a nominal range resolution of 15 cm in air or 12 cm in snow. Laboratory testing and airborne acquisition of data over the glacier Hardangerj&#x00F8;kulen show that the radar appears to work well for the intended purpose of measuring snow layer thickness.]]></abstract>
##     <issn><![CDATA[0196-2892]]></issn>
##     <htmlFlag><![CDATA[1]]></htmlFlag>
##     <arnumber><![CDATA[6729061]]></arnumber>
##     <doi><![CDATA[10.1109/TGRS.2013.2296074]]></doi>
##     <publicationId><![CDATA[6729061]]></publicationId>
##     <mdurl><![CDATA[http://ieeexplore.ieee.org/xpl/articleDetails.jsp?tp=&arnumber=6729061&contentType=Journals+%26+Magazines]]></mdurl>
##     <pdf><![CDATA[http://ieeexplore.ieee.org/stamp/stamp.jsp?arnumber=6729061]]></pdf>
##   </document>
##   <document>
##     <rank>75</rank>
##     <title><![CDATA[Effect of Source/Drain Lateral Straggle on Distortion and Intrinsic Performance of Asymmetric Underlap DG-MOSFETs]]></title>
##     <authors><![CDATA[Koley, K.;  Dutta, A.;  Saha, S.K.;  Sarkar, C.K.]]></authors>
##     <affiliations><![CDATA[Electron. & Telecommun. Eng. Dept., Jadavpur Univ., Kolkata, India]]></affiliations>
##     <controlledterms>
##       <term><![CDATA[MOSFET]]></term>
##       <term><![CDATA[distortion]]></term>
##     </controlledterms>
##     <thesaurusterms>
##       <term><![CDATA[Capacitance]]></term>
##       <term><![CDATA[Inductance]]></term>
##       <term><![CDATA[MOSFET]]></term>
##       <term><![CDATA[Performance evaluation]]></term>
##       <term><![CDATA[Radio frequency]]></term>
##       <term><![CDATA[Resistance]]></term>
##     </thesaurusterms>
##     <pubtitle><![CDATA[Electron Devices Society, IEEE Journal of the]]></pubtitle>
##     <punumber><![CDATA[6245494]]></punumber>
##     <pubtype><![CDATA[Journals & Magazines]]></pubtype>
##     <publisher><![CDATA[IEEE]]></publisher>
##     <volume><![CDATA[2]]></volume>
##     <issue><![CDATA[6]]></issue>
##     <py><![CDATA[2014]]></py>
##     <spage><![CDATA[135]]></spage>
##     <epage><![CDATA[144]]></epage>
##     <abstract><![CDATA[This paper presents a systematic study of the effect of source/drain (S/D) implant lateral straggle on the RF performance of the symmetric and asymmetric underlap double gate (UDG) MOSFET devices. The length of the underlap regions (L<sub>un</sub>) on each side of the gate is a critical technology parameter in determining the performance of UDG-MOSFETs. However, the value of L<sub>un</sub> is susceptible to variation due to S/D implant lateral diffusion. Therefore, it is critical to investigate the impact of S/D implant lateral straggle on the performance of UDG-MOSFETs. This paper shows that the improvement in the RF performance of the UDG-MOSFETs over the conventional DG-MOSFETs can be achieved by optimizing the S/D lateral straggle of the asymmetric UDG-MOSFETs. The RF performance study includes intrinsic capacitances and resistances, transport delay, inductance, and the cut-off frequency.]]></abstract>
##     <issn><![CDATA[2168-6734]]></issn>
##     <htmlFlag><![CDATA[1]]></htmlFlag>
##     <arnumber><![CDATA[6862860]]></arnumber>
##     <doi><![CDATA[10.1109/JEDS.2014.2342613]]></doi>
##     <publicationId><![CDATA[6862860]]></publicationId>
##     <mdurl><![CDATA[http://ieeexplore.ieee.org/xpl/articleDetails.jsp?tp=&arnumber=6862860&contentType=Journals+%26+Magazines]]></mdurl>
##     <pdf><![CDATA[http://ieeexplore.ieee.org/stamp/stamp.jsp?arnumber=6862860]]></pdf>
##   </document>
##   <document>
##     <rank>76</rank>
##     <title><![CDATA[An Event-Based Simulation Framework to Examine the Response of Power Grid to the Charging Demand of Plug-In Hybrid Electric Vehicles]]></title>
##     <authors><![CDATA[Darabi, Z.;  Ferdowsi, M.]]></authors>
##     <affiliations><![CDATA[T&D Dept., SNC-LAVALIN, Binghamton, NY, USA]]></affiliations>
##     <controlledterms>
##       <term><![CDATA[discrete event simulation]]></term>
##       <term><![CDATA[hybrid electric vehicles]]></term>
##       <term><![CDATA[power grids]]></term>
##       <term><![CDATA[probability]]></term>
##       <term><![CDATA[secondary cells]]></term>
##     </controlledterms>
##     <pubtitle><![CDATA[Industrial Informatics, IEEE Transactions on]]></pubtitle>
##     <punumber><![CDATA[9424]]></punumber>
##     <pubtype><![CDATA[Journals & Magazines]]></pubtype>
##     <publisher><![CDATA[IEEE]]></publisher>
##     <volume><![CDATA[10]]></volume>
##     <issue><![CDATA[1]]></issue>
##     <py><![CDATA[2014]]></py>
##     <spage><![CDATA[313]]></spage>
##     <epage><![CDATA[322]]></epage>
##     <abstract><![CDATA[This paper describes the development of a discrete-event simulation framework that emulates the interactions between the power grid and plug-in hybrid electric vehicles (PHEVs) and examines whether the capacity of the existing power system can meet the PHEV load demand. The probability distribution functions for the arrival time and energy demand of each vehicle are extracted from real-world statistical transportation data. The power grid's limited generation and transmission capacities are considered to be the major constraints. Therefore, vehicles may have to wait to receive any charge. The proposed simulation framework is justified and described in some detail in applying it to two real cases in the United States to determine certain regions' grid potential to support PHEVs. Both Level-1 and -2 charging are considered.]]></abstract>
##     <issn><![CDATA[1551-3203]]></issn>
##     <htmlFlag><![CDATA[1]]></htmlFlag>
##     <arnumber><![CDATA[6513311]]></arnumber>
##     <doi><![CDATA[10.1109/TII.2013.2261305]]></doi>
##     <publicationId><![CDATA[6513311]]></publicationId>
##     <mdurl><![CDATA[http://ieeexplore.ieee.org/xpl/articleDetails.jsp?tp=&arnumber=6513311&contentType=Journals+%26+Magazines]]></mdurl>
##     <pdf><![CDATA[http://ieeexplore.ieee.org/stamp/stamp.jsp?arnumber=6513311]]></pdf>
##   </document>
##   <document>
##     <rank>77</rank>
##     <title><![CDATA[Equivalent Stacking of Polarimetric Synthetic Aperture Radar Interferograms Based on Analysis of Persistent and Distributed Scatterers]]></title>
##     <authors><![CDATA[Ishitsuka, K.;  Tamura, M.;  Matsuoka, T.]]></authors>
##     <affiliations><![CDATA[Dept. of Urban Manage., Kyoto Univ., Kyoto, Japan]]></affiliations>
##     <controlledterms>
##       <term><![CDATA[adaptive filters]]></term>
##       <term><![CDATA[adaptive radar]]></term>
##       <term><![CDATA[groundwater]]></term>
##       <term><![CDATA[image resolution]]></term>
##       <term><![CDATA[parameter estimation]]></term>
##       <term><![CDATA[radar imaging]]></term>
##       <term><![CDATA[radar interferometry]]></term>
##       <term><![CDATA[radar polarimetry]]></term>
##       <term><![CDATA[radar resolution]]></term>
##       <term><![CDATA[synthetic aperture radar]]></term>
##     </controlledterms>
##     <thesaurusterms>
##       <term><![CDATA[Coherence]]></term>
##       <term><![CDATA[Decision support systems]]></term>
##       <term><![CDATA[Decorrelation]]></term>
##       <term><![CDATA[Interferometry]]></term>
##       <term><![CDATA[Remote sensing]]></term>
##       <term><![CDATA[Synthetic aperture radar]]></term>
##     </thesaurusterms>
##     <pubtitle><![CDATA[Geoscience and Remote Sensing Letters, IEEE]]></pubtitle>
##     <punumber><![CDATA[8859]]></punumber>
##     <pubtype><![CDATA[Journals & Magazines]]></pubtype>
##     <publisher><![CDATA[IEEE]]></publisher>
##     <volume><![CDATA[11]]></volume>
##     <issue><![CDATA[8]]></issue>
##     <py><![CDATA[2014]]></py>
##     <spage><![CDATA[1360]]></spage>
##     <epage><![CDATA[1364]]></epage>
##     <abstract><![CDATA[We propose a method of incorporating the copolarized observations (HH and VV observations) of multipolarimetric SAR acquisitions into persistent scatterer interferometry to estimate the surface displacement and the scatterer height. Interferograms obtained from HH and VV observations were equally weighted on the basis that the magnitude of the backscattered signals from persistent scatterers represented by odd-bounce scattering and even-bounce scattering are the same in the copolarized observations. Moreover, because of the coarser slant-range resolution of multipolarimetric SAR data, distributed scatterers (DSs) should be taken into account to obtain a spatial density of pixels that is sufficient to estimate the parameters of interest. To this aim, we applied a small baseline criterion to create the analyzed data sets, and we used space adaptive filtering to improve the signal-to-noise ratio of the DSs. We applied this method to estimate the ground subsidence rate in an area of groundwater extraction in Japan using the SAR images that are acquired by Advanced Land Observing Satellite/Phased Array Type L-band SAR (PALSAR) in the full polarimetric observation mode.]]></abstract>
##     <issn><![CDATA[1545-598X]]></issn>
##     <htmlFlag><![CDATA[1]]></htmlFlag>
##     <arnumber><![CDATA[6691927]]></arnumber>
##     <doi><![CDATA[10.1109/LGRS.2013.2293494]]></doi>
##     <publicationId><![CDATA[6691927]]></publicationId>
##     <mdurl><![CDATA[http://ieeexplore.ieee.org/xpl/articleDetails.jsp?tp=&arnumber=6691927&contentType=Journals+%26+Magazines]]></mdurl>
##     <pdf><![CDATA[http://ieeexplore.ieee.org/stamp/stamp.jsp?arnumber=6691927]]></pdf>
##   </document>
##   <document>
##     <rank>78</rank>
##     <title><![CDATA[Multichannel High-Resolution NMF for Modeling Convolutive Mixtures of Non-Stationary Signals in the Time-Frequency Domain]]></title>
##     <authors><![CDATA[Badeau, R.;  Plumbley, M.D.]]></authors>
##     <affiliations><![CDATA[Institut Mines-Telecom, Telecom ParisTech, CNRS LTCI, Paris, France]]></affiliations>
##     <thesaurusterms>
##       <term><![CDATA[Convolution]]></term>
##       <term><![CDATA[Equations]]></term>
##       <term><![CDATA[Hafnium]]></term>
##       <term><![CDATA[Mathematical model]]></term>
##       <term><![CDATA[Speech]]></term>
##       <term><![CDATA[Time-domain analysis]]></term>
##       <term><![CDATA[Time-frequency analysis]]></term>
##     </thesaurusterms>
##     <pubtitle><![CDATA[Audio, Speech, and Language Processing, IEEE/ACM Transactions on]]></pubtitle>
##     <punumber><![CDATA[6570655]]></punumber>
##     <pubtype><![CDATA[Journals & Magazines]]></pubtype>
##     <publisher><![CDATA[IEEE]]></publisher>
##     <volume><![CDATA[22]]></volume>
##     <issue><![CDATA[11]]></issue>
##     <py><![CDATA[2014]]></py>
##     <spage><![CDATA[1670]]></spage>
##     <epage><![CDATA[1680]]></epage>
##     <abstract><![CDATA[Several probabilistic models involving latent components have been proposed for modeling time-frequency (TF) representations of audio signals such as spectrograms, notably in the nonnegative matrix factorization (NMF) literature. Among them, the recent high-resolution NMF (HR-NMF) model is able to take both phases and local correlations in each frequency band into account, and its potential has been illustrated in applications such as source separation and audio inpainting. In this paper, HR-NMF is extended to multichannel signals and to convolutive mixtures. The new model can represent a variety of stationary and non-stationary signals, including autoregressive moving average (ARMA) processes and mixtures of damped sinusoids. A fast variational expectation-maximization (EM) algorithm is proposed to estimate the enhanced model. This algorithm is applied to piano signals, and proves capable of accurately modeling reverberation, restoring missing observations, and separating pure tones with close frequencies.]]></abstract>
##     <issn><![CDATA[2329-9290]]></issn>
##     <htmlFlag><![CDATA[1]]></htmlFlag>
##     <arnumber><![CDATA[6862864]]></arnumber>
##     <doi><![CDATA[10.1109/TASLP.2014.2341920]]></doi>
##     <publicationId><![CDATA[6862864]]></publicationId>
##     <mdurl><![CDATA[http://ieeexplore.ieee.org/xpl/articleDetails.jsp?tp=&arnumber=6862864&contentType=Journals+%26+Magazines]]></mdurl>
##     <pdf><![CDATA[http://ieeexplore.ieee.org/stamp/stamp.jsp?arnumber=6862864]]></pdf>
##   </document>
##   <document>
##     <rank>79</rank>
##     <title><![CDATA[A Localized Adaptive Strategy to Calculate the Backoff Interval in Contention-Based Vehicular Networks]]></title>
##     <authors><![CDATA[Abdelkader, T.;  Naik, K.]]></authors>
##     <affiliations><![CDATA[Dept. of Inf. Syst., Ain Shams Univ., Cairo, Egypt]]></affiliations>
##     <controlledterms>
##       <term><![CDATA[fuzzy logic]]></term>
##       <term><![CDATA[fuzzy reasoning]]></term>
##       <term><![CDATA[packet radio networks]]></term>
##       <term><![CDATA[telecommunication computing]]></term>
##       <term><![CDATA[vehicular ad hoc networks]]></term>
##       <term><![CDATA[wireless channels]]></term>
##     </controlledterms>
##     <thesaurusterms>
##       <term><![CDATA[Delays]]></term>
##       <term><![CDATA[Fuzzy logic]]></term>
##       <term><![CDATA[Media Access Protocol]]></term>
##       <term><![CDATA[Throughput]]></term>
##       <term><![CDATA[Transmitters]]></term>
##       <term><![CDATA[Vehicle dynamics]]></term>
##     </thesaurusterms>
##     <pubtitle><![CDATA[Access, IEEE]]></pubtitle>
##     <punumber><![CDATA[6287639]]></punumber>
##     <pubtype><![CDATA[Journals & Magazines]]></pubtype>
##     <publisher><![CDATA[IEEE]]></publisher>
##     <volume><![CDATA[2]]></volume>
##     <py><![CDATA[2014]]></py>
##     <spage><![CDATA[215]]></spage>
##     <epage><![CDATA[226]]></epage>
##     <abstract><![CDATA[The dynamic nature of vehicular networks with their fast changing topology poses several challenges to setup communication between vehicles. Packet collisions are considered to be the main source of data loss in contention-based vehicular networks. Retransmission of collided packets is done several times until an acknowledgment of successful reception is received or the maximum number of retries is reached. The retransmission delay is drawn randomly from an interval, called the backoff interval. A good choice of the backoff interval reduces the number of collisions and the waiting periods of data packets, which increases the throughput and decreases the energy consumption. An optimal backoff interval could be obtained if global network information spread in the network in a short time. However, this is practically not achievable which motivates the efficient utilization of local information to approach the optimal performance. In this paper, we propose a localized adaptive strategy that calculates the backoff interval for unicast applications in vehicular networks. The new strategy uses fuzzy logic to adapt the backoff interval to the fast changing vehicular environment using only local information. We present four schemes of that strategy that differ in their behavior and the number of inputs. We compare the proposed schemes with other known schemes, binary exponential backoff, backoff algorithm, and an optimal scheme, in terms of throughput, fairness, and energy consumption. Results show that by proper tuning of the fuzzy parameters and rules, one of the proposed schemes outperform the other schemes, and approach the optimal results.]]></abstract>
##     <issn><![CDATA[2169-3536]]></issn>
##     <htmlFlag><![CDATA[1]]></htmlFlag>
##     <arnumber><![CDATA[6762837]]></arnumber>
##     <doi><![CDATA[10.1109/ACCESS.2014.2309856]]></doi>
##     <publicationId><![CDATA[6762837]]></publicationId>
##     <mdurl><![CDATA[http://ieeexplore.ieee.org/xpl/articleDetails.jsp?tp=&arnumber=6762837&contentType=Journals+%26+Magazines]]></mdurl>
##     <pdf><![CDATA[http://ieeexplore.ieee.org/stamp/stamp.jsp?arnumber=6762837]]></pdf>
##   </document>
##   <document>
##     <rank>80</rank>
##     <title><![CDATA[Breakthroughs in Photonics 2013: Research Highlights on Biosensors Based on Plasmonic Nanostructures]]></title>
##     <authors><![CDATA[Yongkang Gao;  Qiaoqiang Gan;  Bartoli, F.J.]]></authors>
##     <affiliations><![CDATA[Dept. of Electr. & Comput. Eng., Lehigh Univ., Bethlehem, PA, USA]]></affiliations>
##     <controlledterms>
##       <term><![CDATA[biomedical equipment]]></term>
##       <term><![CDATA[biosensors]]></term>
##       <term><![CDATA[nanomedicine]]></term>
##       <term><![CDATA[nanosensors]]></term>
##       <term><![CDATA[plasmonics]]></term>
##       <term><![CDATA[reviews]]></term>
##       <term><![CDATA[surface plasmon resonance]]></term>
##     </controlledterms>
##     <thesaurusterms>
##       <term><![CDATA[Arrays]]></term>
##       <term><![CDATA[Biosensors]]></term>
##       <term><![CDATA[Multiplexing]]></term>
##       <term><![CDATA[Nanobioscience]]></term>
##       <term><![CDATA[Nanostructures]]></term>
##       <term><![CDATA[Plasmons]]></term>
##     </thesaurusterms>
##     <pubtitle><![CDATA[Photonics Journal, IEEE]]></pubtitle>
##     <punumber><![CDATA[4563994]]></punumber>
##     <pubtype><![CDATA[Journals & Magazines]]></pubtype>
##     <publisher><![CDATA[IEEE]]></publisher>
##     <volume><![CDATA[6]]></volume>
##     <issue><![CDATA[2]]></issue>
##     <py><![CDATA[2014]]></py>
##     <spage><![CDATA[1]]></spage>
##     <epage><![CDATA[5]]></epage>
##     <abstract><![CDATA[Label-free biomolecular sensing is by far the most common and successful application area in the emerging field of nanoplasmonics. This review paper highlights the latest progress and achievements made in this area. Key aspects of the nanoplasmonic sensor development, including performance enhancement, efforts to increase multiplexing capacity, and the progress in sensor integration and miniaturization, are discussed.]]></abstract>
##     <issn><![CDATA[1943-0655]]></issn>
##     <htmlFlag><![CDATA[1]]></htmlFlag>
##     <arnumber><![CDATA[6766233]]></arnumber>
##     <doi><![CDATA[10.1109/JPHOT.2014.2311440]]></doi>
##     <publicationId><![CDATA[6766233]]></publicationId>
##     <mdurl><![CDATA[http://ieeexplore.ieee.org/xpl/articleDetails.jsp?tp=&arnumber=6766233&contentType=Journals+%26+Magazines]]></mdurl>
##     <pdf><![CDATA[http://ieeexplore.ieee.org/stamp/stamp.jsp?arnumber=6766233]]></pdf>
##   </document>
##   <document>
##     <rank>81</rank>
##     <title><![CDATA[Highly Sensitive Photodetector Using Ultra-High-Density 1.5-&#x03BC;m Quantum Dots for Advanced Optical Fiber Communications]]></title>
##     <authors><![CDATA[Umezawa, T.;  Akahane, K.;  Yamamoto, N.;  Kanno, A.;  Kawanishi, T.]]></authors>
##     <affiliations><![CDATA[Photonics Devices Res. Lab., Nat. Inst. of Inf. & Commun. Technol., Tokyo, Japan]]></affiliations>
##     <controlledterms>
##       <term><![CDATA[III-V semiconductors]]></term>
##       <term><![CDATA[absorption coefficients]]></term>
##       <term><![CDATA[aluminium compounds]]></term>
##       <term><![CDATA[gallium arsenide]]></term>
##       <term><![CDATA[indium compounds]]></term>
##       <term><![CDATA[internal stresses]]></term>
##       <term><![CDATA[optical fibre communication]]></term>
##       <term><![CDATA[photodetectors]]></term>
##       <term><![CDATA[semiconductor quantum dots]]></term>
##     </controlledterms>
##     <thesaurusterms>
##       <term><![CDATA[Absorption]]></term>
##       <term><![CDATA[Dark current]]></term>
##       <term><![CDATA[Electric fields]]></term>
##       <term><![CDATA[Indium gallium arsenide]]></term>
##       <term><![CDATA[Photodetectors]]></term>
##       <term><![CDATA[Quantum dot lasers]]></term>
##       <term><![CDATA[Quantum dots]]></term>
##     </thesaurusterms>
##     <pubtitle><![CDATA[Selected Topics in Quantum Electronics, IEEE Journal of]]></pubtitle>
##     <punumber><![CDATA[2944]]></punumber>
##     <pubtype><![CDATA[Journals & Magazines]]></pubtype>
##     <publisher><![CDATA[IEEE]]></publisher>
##     <volume><![CDATA[20]]></volume>
##     <issue><![CDATA[6]]></issue>
##     <py><![CDATA[2014]]></py>
##     <spage><![CDATA[147]]></spage>
##     <epage><![CDATA[153]]></epage>
##     <abstract><![CDATA[We have fabricated a high-density 1.5-&#x03BC;m quantum dot photodetector for advanced optical fiber communications and have found unique optical properties, including avalanche multiplication. The structure of the absorption layer had stacked InAs/InGaAlAs layers with a high density of 1 &#x00D7; 10<sup>12</sup> cm<sup>-2</sup>, which consisted of strained 1.5-&#x03BC;m InAs quantum dots and a strain compensation layer of InGaAlAs. A three times larger absorption coefficient than the InGaAs layer, an avalanche multiplication effect, and a low dark current are reported with InAs quantum dot conditions.]]></abstract>
##     <issn><![CDATA[1077-260X]]></issn>
##     <htmlFlag><![CDATA[1]]></htmlFlag>
##     <arnumber><![CDATA[6808407]]></arnumber>
##     <doi><![CDATA[10.1109/JSTQE.2014.2321288]]></doi>
##     <publicationId><![CDATA[6808407]]></publicationId>
##     <mdurl><![CDATA[http://ieeexplore.ieee.org/xpl/articleDetails.jsp?tp=&arnumber=6808407&contentType=Journals+%26+Magazines]]></mdurl>
##     <pdf><![CDATA[http://ieeexplore.ieee.org/stamp/stamp.jsp?arnumber=6808407]]></pdf>
##   </document>
##   <document>
##     <rank>82</rank>
##     <title><![CDATA[Distribution Loss Minimization With Guaranteed Error Bound]]></title>
##     <authors><![CDATA[Inoue, T.;  Takano, K.;  Watanabe, T.;  Kawahara, J.;  Yoshinaka, R.;  Kishimoto, A.;  Tsuda, K.;  Minato, S.-I.;  Hayashi, Y.]]></authors>
##     <affiliations><![CDATA[Minato Discrete Struct. Manipulation Syst. Project, Japan Sci. & Technol. Agency, Sapporo, Japan]]></affiliations>
##     <controlledterms>
##       <term><![CDATA[binary decision diagrams]]></term>
##       <term><![CDATA[distributed power generation]]></term>
##       <term><![CDATA[distribution networks]]></term>
##       <term><![CDATA[losses]]></term>
##       <term><![CDATA[minimisation]]></term>
##       <term><![CDATA[power supply quality]]></term>
##       <term><![CDATA[search problems]]></term>
##       <term><![CDATA[switchgear]]></term>
##     </controlledterms>
##     <thesaurusterms>
##       <term><![CDATA[Boolean functions]]></term>
##       <term><![CDATA[Data structures]]></term>
##       <term><![CDATA[Junctions]]></term>
##       <term><![CDATA[Minimization]]></term>
##       <term><![CDATA[Optimization]]></term>
##       <term><![CDATA[Vectors]]></term>
##       <term><![CDATA[Vegetation]]></term>
##     </thesaurusterms>
##     <pubtitle><![CDATA[Smart Grid, IEEE Transactions on]]></pubtitle>
##     <punumber><![CDATA[5165411]]></punumber>
##     <pubtype><![CDATA[Journals & Magazines]]></pubtype>
##     <publisher><![CDATA[IEEE]]></publisher>
##     <volume><![CDATA[5]]></volume>
##     <issue><![CDATA[1]]></issue>
##     <py><![CDATA[2014]]></py>
##     <spage><![CDATA[102]]></spage>
##     <epage><![CDATA[111]]></epage>
##     <abstract><![CDATA[Determining loss minimum configuration in a distribution network is a hard discrete optimization problem involving many variables. Since more and more dispersed generators are installed on the demand side of power systems and they are reconfigured frequently, developing automatic approaches is indispensable for effectively managing a large-scale distribution network. Existing fast methods employ local updates that gradually improve the loss to solve such an optimization problem. However, they eventually get stuck at local minima, resulting in arbitrarily poor results. In contrast, this paper presents a novel optimization method that provides an error bound on the solution quality. Thus, the obtained solution quality can be evaluated in comparison to the global optimal solution. Instead of using local updates, we construct a highly compressed search space using a binary decision diagram and reduce the optimization problem to a shortest path-finding problem. Our method was shown to be not only accurate but also remarkably efficient; optimization of a large-scale model network with 468 switches was solved in three hours with 1.56% relative error bound.]]></abstract>
##     <issn><![CDATA[1949-3053]]></issn>
##     <htmlFlag><![CDATA[1]]></htmlFlag>
##     <arnumber><![CDATA[6693788]]></arnumber>
##     <doi><![CDATA[10.1109/TSG.2013.2288976]]></doi>
##     <publicationId><![CDATA[6693788]]></publicationId>
##     <mdurl><![CDATA[http://ieeexplore.ieee.org/xpl/articleDetails.jsp?tp=&arnumber=6693788&contentType=Journals+%26+Magazines]]></mdurl>
##     <pdf><![CDATA[http://ieeexplore.ieee.org/stamp/stamp.jsp?arnumber=6693788]]></pdf>
##   </document>
##   <document>
##     <rank>83</rank>
##     <title><![CDATA[A Spiking Self-Organizing Map Combining STDP, Oscillations, and Continuous Learning]]></title>
##     <authors><![CDATA[Rumbell, T.;  Denham, S.L.;  Wennekers, T.]]></authors>
##     <affiliations><![CDATA[Cognition Inst., Plymouth Univ., Plymouth, UK]]></affiliations>
##     <controlledterms>
##       <term><![CDATA[data structures]]></term>
##       <term><![CDATA[self-organising feature maps]]></term>
##       <term><![CDATA[unsupervised learning]]></term>
##     </controlledterms>
##     <thesaurusterms>
##       <term><![CDATA[Brain modeling]]></term>
##       <term><![CDATA[Encoding]]></term>
##       <term><![CDATA[Feedforward neural networks]]></term>
##       <term><![CDATA[Mathematical model]]></term>
##       <term><![CDATA[Neurons]]></term>
##       <term><![CDATA[Oscillators]]></term>
##       <term><![CDATA[Training]]></term>
##     </thesaurusterms>
##     <pubtitle><![CDATA[Neural Networks and Learning Systems, IEEE Transactions on]]></pubtitle>
##     <punumber><![CDATA[5962385]]></punumber>
##     <pubtype><![CDATA[Journals & Magazines]]></pubtype>
##     <publisher><![CDATA[IEEE]]></publisher>
##     <volume><![CDATA[25]]></volume>
##     <issue><![CDATA[5]]></issue>
##     <py><![CDATA[2014]]></py>
##     <spage><![CDATA[894]]></spage>
##     <epage><![CDATA[907]]></epage>
##     <abstract><![CDATA[The self-organizing map (SOM) is a neural network algorithm to create topographically ordered spatial representations of an input data set using unsupervised learning. The SOM algorithm is inspired by the feature maps found in mammalian cortices but lacks some important functional properties of its biological equivalents. Neurons have no direct access to global information, transmit information through spikes and may be using phasic coding of spike times within synchronized oscillations, receive continuous input from the environment, do not necessarily alter network properties such as learning rate and lateral connectivity throughout training, and learn through relative timing of action potentials across a synaptic connection. In this paper, a network of integrate-and-fire neurons is presented that incorporates solutions to each of these issues through the neuron model and network structure. Results of the simulated experiments assessing map formation using artificial data as well as the Iris and Wisconsin Breast Cancer datasets show that this novel implementation maintains fundamental properties of the conventional SOM, thereby representing a significant step toward further understanding of the self-organizational properties of the brain while providing an additional method for implementing SOMs that can be utilized for future modeling in software or special purpose spiking neuron hardware.]]></abstract>
##     <issn><![CDATA[2162-237X]]></issn>
##     <htmlFlag><![CDATA[1]]></htmlFlag>
##     <arnumber><![CDATA[6636061]]></arnumber>
##     <doi><![CDATA[10.1109/TNNLS.2013.2283140]]></doi>
##     <publicationId><![CDATA[6636061]]></publicationId>
##     <mdurl><![CDATA[http://ieeexplore.ieee.org/xpl/articleDetails.jsp?tp=&arnumber=6636061&contentType=Journals+%26+Magazines]]></mdurl>
##     <pdf><![CDATA[http://ieeexplore.ieee.org/stamp/stamp.jsp?arnumber=6636061]]></pdf>
##   </document>
##   <document>
##     <rank>84</rank>
##     <title><![CDATA[Development of a Highly Flexible Mobile GIS-Based System for Collecting Arable Land Quality Data]]></title>
##     <authors><![CDATA[Sijing Ye;  Dehai Zhu;  Xiaochuang Yao;  Nan Zhang;  Shuai Fang;  Lin Li]]></authors>
##     <affiliations><![CDATA[Coll. of Inf. & Electr. Eng., China Agric. Univ., Beijing, China]]></affiliations>
##     <controlledterms>
##       <term><![CDATA[electronic data interchange]]></term>
##       <term><![CDATA[geographic information systems]]></term>
##       <term><![CDATA[geophysical techniques]]></term>
##     </controlledterms>
##     <thesaurusterms>
##       <term><![CDATA[Data collection]]></term>
##       <term><![CDATA[Data models]]></term>
##       <term><![CDATA[Geographic information systems]]></term>
##       <term><![CDATA[Indexes]]></term>
##       <term><![CDATA[Mobile communication]]></term>
##       <term><![CDATA[Spatial databases]]></term>
##       <term><![CDATA[Vectors]]></term>
##     </thesaurusterms>
##     <pubtitle><![CDATA[Selected Topics in Applied Earth Observations and Remote Sensing, IEEE Journal of]]></pubtitle>
##     <punumber><![CDATA[4609443]]></punumber>
##     <pubtype><![CDATA[Journals & Magazines]]></pubtype>
##     <publisher><![CDATA[IEEE]]></publisher>
##     <volume><![CDATA[7]]></volume>
##     <issue><![CDATA[11]]></issue>
##     <py><![CDATA[2014]]></py>
##     <spage><![CDATA[4432]]></spage>
##     <epage><![CDATA[4441]]></epage>
##     <abstract><![CDATA[In recent years, well-designed terminal-based methods for collecting index data have gradually replaced traditional pen-and-paper methods and have been extensively used in numerous studies. These new approaches offer users increased accuracy, efficiency, consumption, and data compatibility compared to traditional methods. In general, we find that spatial data content and quality index systems vary widely across different arable land regions. Thus, a system for the investigation of arable land quality indices that has the flexibility to utilize various types of spatial data and quality indices without requiring program modification is needed. This paper presents the framework, the module partition, and the structure of the data exchange interface for a highly flexible mobile GIS-based system, which we call the &#x201C;arable land quality index data collection system&#x201D; (ALQIDCS). This system incorporates a series of self-adaptive methods, a data table-driven model and two types of formulas for flexible data collection and processing. We tested our prototype system by investigating arable land quality in the Da Xing District, Beijing and in the Te Da La Qi District, Inner Mongolia, China. The results indicate that the ALQIDCS can effectively adapt to variations in spatial data and quality index systems and meet different objectives. The limitations of the ALQIDCS and suggestions for future work are also presented.]]></abstract>
##     <issn><![CDATA[1939-1404]]></issn>
##     <htmlFlag><![CDATA[1]]></htmlFlag>
##     <arnumber><![CDATA[6815642]]></arnumber>
##     <doi><![CDATA[10.1109/JSTARS.2014.2320635]]></doi>
##     <publicationId><![CDATA[6815642]]></publicationId>
##     <mdurl><![CDATA[http://ieeexplore.ieee.org/xpl/articleDetails.jsp?tp=&arnumber=6815642&contentType=Journals+%26+Magazines]]></mdurl>
##     <pdf><![CDATA[http://ieeexplore.ieee.org/stamp/stamp.jsp?arnumber=6815642]]></pdf>
##   </document>
##   <document>
##     <rank>85</rank>
##     <title><![CDATA[A Sensitivity-Analysis-Based Approach for the Calibration of Traffic Simulation Models]]></title>
##     <authors><![CDATA[Ciuffo, B.;  Lima Azevedo, C.]]></authors>
##     <affiliations><![CDATA[Inst. for Energy & Transp., Eur. Comm., Ispra, Italy]]></affiliations>
##     <controlledterms>
##       <term><![CDATA[calibration]]></term>
##       <term><![CDATA[road traffic]]></term>
##       <term><![CDATA[sensitivity analysis]]></term>
##       <term><![CDATA[transportation]]></term>
##     </controlledterms>
##     <thesaurusterms>
##       <term><![CDATA[Analytical models]]></term>
##       <term><![CDATA[Calibration]]></term>
##       <term><![CDATA[Computational modeling]]></term>
##       <term><![CDATA[Data models]]></term>
##       <term><![CDATA[Indexes]]></term>
##       <term><![CDATA[Mathematical model]]></term>
##       <term><![CDATA[Sensitivity]]></term>
##     </thesaurusterms>
##     <pubtitle><![CDATA[Intelligent Transportation Systems, IEEE Transactions on]]></pubtitle>
##     <punumber><![CDATA[6979]]></punumber>
##     <pubtype><![CDATA[Journals & Magazines]]></pubtype>
##     <publisher><![CDATA[IEEE]]></publisher>
##     <volume><![CDATA[15]]></volume>
##     <issue><![CDATA[3]]></issue>
##     <py><![CDATA[2014]]></py>
##     <spage><![CDATA[1298]]></spage>
##     <epage><![CDATA[1309]]></epage>
##     <abstract><![CDATA[In this paper, a multistep sensitivity analysis (SA) approach for model calibration is proposed and applied to a complex traffic simulation model with more than 100 parameters. Throughout this paper, it is argued that the application of SA is crucial for true comprehension and the correct use of traffic simulation models, but it is also acknowledged that the main obstacle toward an extensive use of the most sophisticated techniques is the high number of model runs usually required. For this reason, we have tested the possibility of performing a multistep SA, where, at each step, model parameters are grouped on the basis of possible common features, and a final SA on the parameters pertaining to the most influential groups is then performed. The proposed methodology was applied to an urban motorway case study simulated using MITSIMLab, a complex microscopic traffic simulator. The method allowed the analysis of the role played by all parameters and by the model stochasticity itself, with 80% fewer model evaluations than the standard variance-based approach. Ten model parameters accounted for a big share in the output variance for the specific case study. A Kriging metamodel was then estimated and integrated with the multistep SA results for a global calibration framework in the presence of uncertainty. Results confirm the great potential of this approach and open up to a novel view for the calibration of a traffic simulation model.]]></abstract>
##     <issn><![CDATA[1524-9050]]></issn>
##     <htmlFlag><![CDATA[1]]></htmlFlag>
##     <arnumber><![CDATA[6757042]]></arnumber>
##     <doi><![CDATA[10.1109/TITS.2014.2302674]]></doi>
##     <publicationId><![CDATA[6757042]]></publicationId>
##     <mdurl><![CDATA[http://ieeexplore.ieee.org/xpl/articleDetails.jsp?tp=&arnumber=6757042&contentType=Journals+%26+Magazines]]></mdurl>
##     <pdf><![CDATA[http://ieeexplore.ieee.org/stamp/stamp.jsp?arnumber=6757042]]></pdf>
##   </document>
##   <document>
##     <rank>86</rank>
##     <title><![CDATA[Ray-Tracing-Based mm-Wave Beamforming Assessment]]></title>
##     <authors><![CDATA[Degli-Esposti, V.;  Fuschini, F.;  Vitucci, E.M.;  Barbiroli, M.;  Zoli, M.;  Tian, L.;  Yin, X.;  Dupleich, D.A.;  Muller, R.;  Schneider, C.;  Thoma, R.S.]]></authors>
##     <affiliations><![CDATA[Ilmenau Univ. of Technol., Ilmenau, Germany]]></affiliations>
##     <controlledterms>
##       <term><![CDATA[array signal processing]]></term>
##       <term><![CDATA[millimetre wave antenna arrays]]></term>
##       <term><![CDATA[ray tracing]]></term>
##     </controlledterms>
##     <thesaurusterms>
##       <term><![CDATA[Antenna arrays]]></term>
##       <term><![CDATA[Array signal processing]]></term>
##       <term><![CDATA[Beam steering]]></term>
##       <term><![CDATA[Computational modeling]]></term>
##       <term><![CDATA[Millimeter wave technology]]></term>
##       <term><![CDATA[Performance evaluation]]></term>
##       <term><![CDATA[Predictive models]]></term>
##       <term><![CDATA[Ray tracing]]></term>
##     </thesaurusterms>
##     <pubtitle><![CDATA[Access, IEEE]]></pubtitle>
##     <punumber><![CDATA[6287639]]></punumber>
##     <pubtype><![CDATA[Journals & Magazines]]></pubtype>
##     <publisher><![CDATA[IEEE]]></publisher>
##     <volume><![CDATA[2]]></volume>
##     <py><![CDATA[2014]]></py>
##     <spage><![CDATA[1314]]></spage>
##     <epage><![CDATA[1325]]></epage>
##     <abstract><![CDATA[The use of large-size antenna arrays to implement pencil-beam forming techniques is becoming a key asset to cope with the very high throughput density requirements and high path-loss of future millimeter-wave (mm-wave) gigabit-wireless applications. Suboptimal beamforming (BF) strategies based on search over discrete set of beams (steering vectors) are proposed and implemented in present standards and applications. The potential of fully adaptive advanced BF strategies that will become possible in the future, thanks to the availability of accurate localization and powerful distributed computing, is evaluated in this paper through system simulation. After validation and calibration against mm-wave directional indoor channel measurements, a 3-D ray tracing model is used as a propagation-prediction engine to evaluate performance in a number of simple, reference cases. Ray tracing itself, however, is proposed and evaluated as a real-time prediction tool to assist future BF techniques.]]></abstract>
##     <issn><![CDATA[2169-3536]]></issn>
##     <htmlFlag><![CDATA[1]]></htmlFlag>
##     <arnumber><![CDATA[6942178]]></arnumber>
##     <doi><![CDATA[10.1109/ACCESS.2014.2365991]]></doi>
##     <publicationId><![CDATA[6942178]]></publicationId>
##     <mdurl><![CDATA[http://ieeexplore.ieee.org/xpl/articleDetails.jsp?tp=&arnumber=6942178&contentType=Journals+%26+Magazines]]></mdurl>
##     <pdf><![CDATA[http://ieeexplore.ieee.org/stamp/stamp.jsp?arnumber=6942178]]></pdf>
##   </document>
##   <document>
##     <rank>87</rank>
##     <title><![CDATA[Residual Component Analysis of Hyperspectral Images&#x2014;Application to Joint Nonlinear Unmixing and Nonlinearity Detection]]></title>
##     <authors><![CDATA[Altmann, Y.;  Dobigeon, N.;  McLaughlin, S.;  Tourneret, J.-Y.]]></authors>
##     <affiliations><![CDATA[Univ. of Toulouse, Toulouse, France]]></affiliations>
##     <controlledterms>
##       <term><![CDATA[Gaussian noise]]></term>
##       <term><![CDATA[hyperspectral imaging]]></term>
##       <term><![CDATA[image segmentation]]></term>
##     </controlledterms>
##     <thesaurusterms>
##       <term><![CDATA[Additives]]></term>
##       <term><![CDATA[Bayes methods]]></term>
##       <term><![CDATA[Covariance matrices]]></term>
##       <term><![CDATA[Hyperspectral imaging]]></term>
##       <term><![CDATA[Joints]]></term>
##       <term><![CDATA[Noise]]></term>
##       <term><![CDATA[Vectors]]></term>
##     </thesaurusterms>
##     <pubtitle><![CDATA[Image Processing, IEEE Transactions on]]></pubtitle>
##     <punumber><![CDATA[83]]></punumber>
##     <pubtype><![CDATA[Journals & Magazines]]></pubtype>
##     <publisher><![CDATA[IEEE]]></publisher>
##     <volume><![CDATA[23]]></volume>
##     <issue><![CDATA[5]]></issue>
##     <py><![CDATA[2014]]></py>
##     <spage><![CDATA[2148]]></spage>
##     <epage><![CDATA[2158]]></epage>
##     <abstract><![CDATA[This paper presents a nonlinear mixing model for joint hyperspectral image unmixing and nonlinearity detection. The proposed model assumes that the pixel reflectances are linear combinations of known pure spectral components corrupted by an additional nonlinear term, affecting the end members and contaminated by an additive Gaussian noise. A Markov random field is considered for nonlinearity detection based on the spatial structure of the nonlinear terms. The observed image is segmented into regions where nonlinear terms, if present, share similar statistical properties. A Bayesian algorithm is proposed to estimate the parameters involved in the model yielding a joint nonlinear unmixing and nonlinearity detection algorithm. The performance of the proposed strategy is first evaluated on synthetic data. Simulations conducted with real data show the accuracy of the proposed unmixing and nonlinearity detection strategy for the analysis of hyperspectral images.]]></abstract>
##     <issn><![CDATA[1057-7149]]></issn>
##     <htmlFlag><![CDATA[1]]></htmlFlag>
##     <arnumber><![CDATA[6775297]]></arnumber>
##     <doi><![CDATA[10.1109/TIP.2014.2312616]]></doi>
##     <publicationId><![CDATA[6775297]]></publicationId>
##     <mdurl><![CDATA[http://ieeexplore.ieee.org/xpl/articleDetails.jsp?tp=&arnumber=6775297&contentType=Journals+%26+Magazines]]></mdurl>
##     <pdf><![CDATA[http://ieeexplore.ieee.org/stamp/stamp.jsp?arnumber=6775297]]></pdf>
##   </document>
##   <document>
##     <rank>88</rank>
##     <title><![CDATA[On the Performance Analysis of Hybrid ARQ With Incremental Redundancy and With Code Combining Over Free-Space Optical Channels With Pointing Errors]]></title>
##     <authors><![CDATA[Zedini, E.;  Chelli, A.;  Alouini, M.-S.]]></authors>
##     <affiliations><![CDATA[Comput., Electr., & Math. Sci. & Eng. (CEMSE) Div., King Abdullah Univ. of Sci. & Technol. (KAUST), Thuwal, Saudi Arabia]]></affiliations>
##     <controlledterms>
##       <term><![CDATA[automatic repeat request]]></term>
##       <term><![CDATA[error statistics]]></term>
##       <term><![CDATA[optical links]]></term>
##       <term><![CDATA[redundancy]]></term>
##     </controlledterms>
##     <thesaurusterms>
##       <term><![CDATA[Adaptive optics]]></term>
##       <term><![CDATA[Approximation methods]]></term>
##       <term><![CDATA[Atmospheric modeling]]></term>
##       <term><![CDATA[Bit error rate]]></term>
##       <term><![CDATA[Fading]]></term>
##       <term><![CDATA[Receivers]]></term>
##       <term><![CDATA[Signal to noise ratio]]></term>
##     </thesaurusterms>
##     <pubtitle><![CDATA[Photonics Journal, IEEE]]></pubtitle>
##     <punumber><![CDATA[4563994]]></punumber>
##     <pubtype><![CDATA[Journals & Magazines]]></pubtype>
##     <publisher><![CDATA[IEEE]]></publisher>
##     <volume><![CDATA[6]]></volume>
##     <issue><![CDATA[4]]></issue>
##     <py><![CDATA[2014]]></py>
##     <spage><![CDATA[1]]></spage>
##     <epage><![CDATA[18]]></epage>
##     <abstract><![CDATA[In this paper, we investigate the performance of hybrid automatic repeat request (HARQ) with incremental redundancy (IR) and with code combining (CC) from an information-theoretic perspective over a point-to-point free-space optical (FSO) system. First, we introduce new closed-form expressions for the probability density function, the cumulative distribution function, the moment generating function, and the moments of an FSO link modeled by the Gamma fading channel subject to pointing errors and using intensity modulation with direct detection technique at the receiver. Based on these formulas, we derive exact results for the average bit error rate and the capacity in terms of Meijer's G functions. Moreover, we present asymptotic expressions by utilizing the Meijer's G function expansion and using the moments method, too, for the ergodic capacity approximations. Then, we provide novel analytical expressions for the outage probability, the average number of transmissions, and the average transmission rate for HARQ with IR, assuming a maximum number of rounds for the HARQ protocol. Besides, we offer asymptotic expressions for these results in terms of simple elementary functions. Additionally, we compare the performance of HARQ with IR and HARQ with CC. Our analysis demonstrates that HARQ with IR outperforms HARQ with CC.]]></abstract>
##     <issn><![CDATA[1943-0655]]></issn>
##     <htmlFlag><![CDATA[1]]></htmlFlag>
##     <arnumber><![CDATA[6857378]]></arnumber>
##     <doi><![CDATA[10.1109/JPHOT.2014.2339331]]></doi>
##     <publicationId><![CDATA[6857378]]></publicationId>
##     <mdurl><![CDATA[http://ieeexplore.ieee.org/xpl/articleDetails.jsp?tp=&arnumber=6857378&contentType=Journals+%26+Magazines]]></mdurl>
##     <pdf><![CDATA[http://ieeexplore.ieee.org/stamp/stamp.jsp?arnumber=6857378]]></pdf>
##   </document>
##   <document>
##     <rank>89</rank>
##     <title><![CDATA[Circularly Polarized Patch Antenna for Future 5G Mobile Phones]]></title>
##     <authors><![CDATA[Ka Ming Mak;  Hau Wah Lai;  Kwai Man Luk;  Chi Hou Chan]]></authors>
##     <affiliations><![CDATA[State Key Lab. of Millimeter Waves, City Univ. of Hong Kong, Hong Kong, China]]></affiliations>
##     <controlledterms>
##       <term><![CDATA[5G mobile communication]]></term>
##       <term><![CDATA[antenna radiation patterns]]></term>
##       <term><![CDATA[microstrip antennas]]></term>
##     </controlledterms>
##     <thesaurusterms>
##       <term><![CDATA[5G mobile communication]]></term>
##       <term><![CDATA[Bandwidth]]></term>
##       <term><![CDATA[Dielectric substrates]]></term>
##       <term><![CDATA[Mobile communication]]></term>
##       <term><![CDATA[Mobile handsets]]></term>
##       <term><![CDATA[Patch antennas]]></term>
##       <term><![CDATA[Polarization]]></term>
##     </thesaurusterms>
##     <pubtitle><![CDATA[Access, IEEE]]></pubtitle>
##     <punumber><![CDATA[6287639]]></punumber>
##     <pubtype><![CDATA[Journals & Magazines]]></pubtype>
##     <publisher><![CDATA[IEEE]]></publisher>
##     <volume><![CDATA[2]]></volume>
##     <py><![CDATA[2014]]></py>
##     <spage><![CDATA[1521]]></spage>
##     <epage><![CDATA[1529]]></epage>
##     <abstract><![CDATA[A circularly polarized patch antenna for future fifth-generation mobile phones is presented in this paper. Miniaturization and beamwidth enhancement of a patch antenna are the two main areas to be discussed. By folding the edge of the radiating patch with loading slots, the size of the patch antenna is 44.8% smaller than a conventional half wavelength patch, which allows it to be accommodated inside handsets easily. Wide beamwidth is obtained by surrounding the patch with a dielectric substrate and supporting the antenna by a metallic block. A measured half power beamwidth of 124&#x00B0; is achieved. The impedance bandwidth of the antenna is over 10%, and the 3-dB axial ratio bandwidth is 3.05%. The proposed antenna covers a wide elevation angle and complete azimuth range. A parametric study of the effect of the metallic block and the surrounding dielectric substrate on the gain at a low elevation angle and the axial ratio of the proposed antenna are presented.]]></abstract>
##     <issn><![CDATA[2169-3536]]></issn>
##     <htmlFlag><![CDATA[1]]></htmlFlag>
##     <arnumber><![CDATA[6985762]]></arnumber>
##     <doi><![CDATA[10.1109/ACCESS.2014.2382111]]></doi>
##     <publicationId><![CDATA[6985762]]></publicationId>
##     <mdurl><![CDATA[http://ieeexplore.ieee.org/xpl/articleDetails.jsp?tp=&arnumber=6985762&contentType=Journals+%26+Magazines]]></mdurl>
##     <pdf><![CDATA[http://ieeexplore.ieee.org/stamp/stamp.jsp?arnumber=6985762]]></pdf>
##   </document>
##   <document>
##     <rank>90</rank>
##     <title><![CDATA[4H-SiC N-Channel JFET for Operation in High-Temperature Environments]]></title>
##     <authors><![CDATA[Wei-Chen Lien;  Damrongplasit, N.;  Paredes, J.H.;  Senesky, D.G.;  Liu, T.-J.K.;  Pisano, A.P.]]></authors>
##     <affiliations><![CDATA[Dept. of Electr. Eng. & Comput. Sci., Univ. of California, Berkeley, Berkeley, CA, USA]]></affiliations>
##     <controlledterms>
##       <term><![CDATA[contact resistance]]></term>
##       <term><![CDATA[junction gate field effect transistors]]></term>
##       <term><![CDATA[nickel]]></term>
##       <term><![CDATA[ohmic contacts]]></term>
##       <term><![CDATA[silicon compounds]]></term>
##       <term><![CDATA[titanium compounds]]></term>
##       <term><![CDATA[wide band gap semiconductors]]></term>
##     </controlledterms>
##     <thesaurusterms>
##       <term><![CDATA[JFETs]]></term>
##       <term><![CDATA[Silicon carbide]]></term>
##       <term><![CDATA[Temperature measurement]]></term>
##       <term><![CDATA[Threshold voltage]]></term>
##       <term><![CDATA[Transconductance]]></term>
##     </thesaurusterms>
##     <pubtitle><![CDATA[Electron Devices Society, IEEE Journal of the]]></pubtitle>
##     <punumber><![CDATA[6245494]]></punumber>
##     <pubtype><![CDATA[Journals & Magazines]]></pubtype>
##     <publisher><![CDATA[IEEE]]></publisher>
##     <volume><![CDATA[2]]></volume>
##     <issue><![CDATA[6]]></issue>
##     <py><![CDATA[2014]]></py>
##     <spage><![CDATA[164]]></spage>
##     <epage><![CDATA[167]]></epage>
##     <abstract><![CDATA[Lateral depletion-mode 4H-SiC n-channel junction field-effect transistors (JFETs) are demonstrated to operate with well-behaved electrical characteristics at temperatures up to 600&#x00B0;C in air. Ti/Ni/TiW metal stacks are used to form ohmic contacts to n-type 4H-SiC with specific contact resistance of 1.14 &#x00D7; 10<sup>-3</sup> &#x03A9; cm<sup>2</sup> at 600&#x00B0;C. The on/off drain saturation current ratio and intrinsic gain at 600&#x00B0;C are 1.53 &#x00D7; 10<sup>3</sup> and 57.2, respectively. These results indicate that 4H-SiC JFETs can be used for extremely-high-temperature electronics applications.]]></abstract>
##     <issn><![CDATA[2168-6734]]></issn>
##     <htmlFlag><![CDATA[1]]></htmlFlag>
##     <arnumber><![CDATA[6892931]]></arnumber>
##     <doi><![CDATA[10.1109/JEDS.2014.2355132]]></doi>
##     <publicationId><![CDATA[6892931]]></publicationId>
##     <mdurl><![CDATA[http://ieeexplore.ieee.org/xpl/articleDetails.jsp?tp=&arnumber=6892931&contentType=Journals+%26+Magazines]]></mdurl>
##     <pdf><![CDATA[http://ieeexplore.ieee.org/stamp/stamp.jsp?arnumber=6892931]]></pdf>
##   </document>
##   <document>
##     <rank>91</rank>
##     <title><![CDATA[Maximizing Energy Efficiency in the Vector Precoded MU-MISO Downlink by Selective Perturbation]]></title>
##     <authors><![CDATA[Masouros, C.;  Sellathurai, M.;  Ratnarajah, T.]]></authors>
##     <affiliations><![CDATA[Dept. of Electron. & Electr. Eng., Univ. Coll. London, London, UK]]></affiliations>
##     <controlledterms>
##       <term><![CDATA[perturbation techniques]]></term>
##       <term><![CDATA[precoding]]></term>
##       <term><![CDATA[signal processing]]></term>
##       <term><![CDATA[telecommunication power management]]></term>
##     </controlledterms>
##     <thesaurusterms>
##       <term><![CDATA[Complexity theory]]></term>
##       <term><![CDATA[Educational institutions]]></term>
##       <term><![CDATA[Lattices]]></term>
##       <term><![CDATA[Optimization]]></term>
##       <term><![CDATA[Transmitters]]></term>
##       <term><![CDATA[Vectors]]></term>
##       <term><![CDATA[Wireless communication]]></term>
##     </thesaurusterms>
##     <pubtitle><![CDATA[Wireless Communications, IEEE Transactions on]]></pubtitle>
##     <punumber><![CDATA[7693]]></punumber>
##     <pubtype><![CDATA[Journals & Magazines]]></pubtype>
##     <publisher><![CDATA[IEEE]]></publisher>
##     <volume><![CDATA[13]]></volume>
##     <issue><![CDATA[9]]></issue>
##     <py><![CDATA[2014]]></py>
##     <spage><![CDATA[4974]]></spage>
##     <epage><![CDATA[4984]]></epage>
##     <abstract><![CDATA[We propose an energy-efficient vector perturbation (VP) technique for the downlink of multiuser multiple-input-single-output (MU-MISO) systems. In contrast to conventional VP where the search for perturbation vectors involves all users' symbols, here, the perturbation is applied to a subset of the transmitted symbols. This, therefore, introduces a performance-complexity tradeoff, where the complexity is greatly reduced compared to VP by limiting the dimensions of the sphere search, at the expense of a performance penalty compared to VP. By changing the size of the subset of perturbed users, the aforementioned tradeoff can be controlled to maximize energy efficiency. We further propose three distinct criteria for selecting which users' symbols to perturb, each of which yields a different performance-complexity tradeoff. The presented analytical and simulation results show that partially perturbing the data provides a favorable tradeoff, particularly at low-power transmission where the power consumption associated with the signal processing becomes dominant. In fact, it is shown that diversity close to the one for conventional VP can be achieved at energy efficiency levels improved by up to 300% compared to VP.]]></abstract>
##     <issn><![CDATA[1536-1276]]></issn>
##     <htmlFlag><![CDATA[1]]></htmlFlag>
##     <arnumber><![CDATA[6827257]]></arnumber>
##     <doi><![CDATA[10.1109/TWC.2014.2329480]]></doi>
##     <publicationId><![CDATA[6827257]]></publicationId>
##     <mdurl><![CDATA[http://ieeexplore.ieee.org/xpl/articleDetails.jsp?tp=&arnumber=6827257&contentType=Journals+%26+Magazines]]></mdurl>
##     <pdf><![CDATA[http://ieeexplore.ieee.org/stamp/stamp.jsp?arnumber=6827257]]></pdf>
##   </document>
##   <document>
##     <rank>92</rank>
##     <title><![CDATA[On Generalized Auto-Spectral Coherence Function and Its Applications to Signal Detection]]></title>
##     <authors><![CDATA[Chengshi Zheng;  Hefei Yang;  Xiaodong Li]]></authors>
##     <affiliations><![CDATA[Commun. Acoust. Lab., Inst. of Acoust., Beijing, China]]></affiliations>
##     <controlledterms>
##       <term><![CDATA[signal detection]]></term>
##     </controlledterms>
##     <thesaurusterms>
##       <term><![CDATA[Chirp]]></term>
##       <term><![CDATA[Coherence]]></term>
##       <term><![CDATA[Estimation]]></term>
##       <term><![CDATA[Noise]]></term>
##       <term><![CDATA[Random processes]]></term>
##       <term><![CDATA[Speech]]></term>
##       <term><![CDATA[Transient analysis]]></term>
##     </thesaurusterms>
##     <pubtitle><![CDATA[Signal Processing Letters, IEEE]]></pubtitle>
##     <punumber><![CDATA[97]]></punumber>
##     <pubtype><![CDATA[Journals & Magazines]]></pubtype>
##     <publisher><![CDATA[IEEE]]></publisher>
##     <volume><![CDATA[21]]></volume>
##     <issue><![CDATA[5]]></issue>
##     <py><![CDATA[2014]]></py>
##     <spage><![CDATA[559]]></spage>
##     <epage><![CDATA[563]]></epage>
##     <abstract><![CDATA[Considering that spectral components of one random process are not necessarily independent for all types of signals, this paper defines a generalized auto-spectral coherence function (GAS-CF) to measure this spectral correlation. The GAS-CF is a generalization of the temporal coherence function and the spectral coherence function, where they have already been successfully applied to detect howling components and transient noise components, respectively. After defining the GAS-CF, this paper studies its statistical properties in detail. Simulation results show that the proposed GAS-CF can be applied to detect different types of signals, including transient noise, howling frequency and chirp signal, in a simple way.]]></abstract>
##     <issn><![CDATA[1070-9908]]></issn>
##     <htmlFlag><![CDATA[1]]></htmlFlag>
##     <arnumber><![CDATA[6763081]]></arnumber>
##     <doi><![CDATA[10.1109/LSP.2014.2310772]]></doi>
##     <publicationId><![CDATA[6763081]]></publicationId>
##     <mdurl><![CDATA[http://ieeexplore.ieee.org/xpl/articleDetails.jsp?tp=&arnumber=6763081&contentType=Journals+%26+Magazines]]></mdurl>
##     <pdf><![CDATA[http://ieeexplore.ieee.org/stamp/stamp.jsp?arnumber=6763081]]></pdf>
##   </document>
##   <document>
##     <rank>93</rank>
##     <title><![CDATA[Which Spring is the Best? Comparison of Methods for Virtual Stenting]]></title>
##     <authors><![CDATA[Spranger, K.;  Ventikos, Y.]]></authors>
##     <affiliations><![CDATA[Dept. of Eng. Sci., Univ. of Oxford, Oxford, UK]]></affiliations>
##     <controlledterms>
##       <term><![CDATA[blood vessels]]></term>
##       <term><![CDATA[design]]></term>
##       <term><![CDATA[geometry]]></term>
##       <term><![CDATA[haemodynamics]]></term>
##       <term><![CDATA[matrix algebra]]></term>
##       <term><![CDATA[medical computing]]></term>
##       <term><![CDATA[physiological models]]></term>
##       <term><![CDATA[real-time systems]]></term>
##       <term><![CDATA[stents]]></term>
##       <term><![CDATA[torsion]]></term>
##       <term><![CDATA[virtual reality]]></term>
##     </controlledterms>
##     <thesaurusterms>
##       <term><![CDATA[Computational modeling]]></term>
##       <term><![CDATA[Crimping]]></term>
##       <term><![CDATA[Equations]]></term>
##       <term><![CDATA[Geometry]]></term>
##       <term><![CDATA[Implants]]></term>
##       <term><![CDATA[Mathematical model]]></term>
##       <term><![CDATA[Springs]]></term>
##     </thesaurusterms>
##     <pubtitle><![CDATA[Biomedical Engineering, IEEE Transactions on]]></pubtitle>
##     <punumber><![CDATA[10]]></punumber>
##     <pubtype><![CDATA[Journals & Magazines]]></pubtype>
##     <publisher><![CDATA[IEEE]]></publisher>
##     <volume><![CDATA[61]]></volume>
##     <issue><![CDATA[7]]></issue>
##     <py><![CDATA[2014]]></py>
##     <spage><![CDATA[1998]]></spage>
##     <epage><![CDATA[2010]]></epage>
##     <abstract><![CDATA[This paper presents a methodology for modeling the deployment of implantable devices used in minimally invasive vascular interventions. Motivated by the clinical need to perform preinterventional rehearsals of a stent deployment, we have developed methods enabling virtual device placement inside arteries, under the constraint of real-time application. This requirement of rapid execution narrowed down the search for a suitable method to the concept of a dynamic mesh. Inspired by the idea of a mesh of springs, we have found a novel way to apply it to stent modeling. The experiments conducted in this paper investigate properties of the stent models based on three different spring types: lineal, semitorsional, and torsional springs. Furthermore, this paper compares the results of various deployment scenarios for two different classes of devices: a stent graft and a flow diverter. The presented results can be of a high-potential clinical value, enabling the predictive evaluation of the outcome of a stent deployment treatment.]]></abstract>
##     <issn><![CDATA[0018-9294]]></issn>
##     <htmlFlag><![CDATA[1]]></htmlFlag>
##     <arnumber><![CDATA[6767090]]></arnumber>
##     <doi><![CDATA[10.1109/TBME.2014.2311856]]></doi>
##     <publicationId><![CDATA[6767090]]></publicationId>
##     <mdurl><![CDATA[http://ieeexplore.ieee.org/xpl/articleDetails.jsp?tp=&arnumber=6767090&contentType=Journals+%26+Magazines]]></mdurl>
##     <pdf><![CDATA[http://ieeexplore.ieee.org/stamp/stamp.jsp?arnumber=6767090]]></pdf>
##   </document>
##   <document>
##     <rank>94</rank>
##     <title><![CDATA[Automatic Detection and Classification of Unsafe Events During Power Wheelchair Use]]></title>
##     <authors><![CDATA[Pineau, J.;  Moghaddam, A.K.;  Hiu Kim Yuen;  Archambault, P.S.;  Routhier, F.;  Michaud, F.;  Boissy, P.]]></authors>
##     <affiliations><![CDATA[Sch. of Comput. Sci., McGill Univ., Montreal, QC, Canada]]></affiliations>
##     <controlledterms>
##       <term><![CDATA[data loggers]]></term>
##       <term><![CDATA[handicapped aids]]></term>
##       <term><![CDATA[medical control systems]]></term>
##       <term><![CDATA[time series]]></term>
##       <term><![CDATA[time-frequency analysis]]></term>
##       <term><![CDATA[wavelet transforms]]></term>
##       <term><![CDATA[wheelchairs]]></term>
##     </controlledterms>
##     <thesaurusterms>
##       <term><![CDATA[Accelerometers]]></term>
##       <term><![CDATA[Educational institutions]]></term>
##       <term><![CDATA[Electronic mail]]></term>
##       <term><![CDATA[Feature extraction]]></term>
##       <term><![CDATA[Sensors]]></term>
##       <term><![CDATA[Support vector machines]]></term>
##       <term><![CDATA[Wheelchairs]]></term>
##     </thesaurusterms>
##     <pubtitle><![CDATA[Translational Engineering in Health and Medicine, IEEE Journal of]]></pubtitle>
##     <punumber><![CDATA[6221039]]></punumber>
##     <pubtype><![CDATA[Journals & Magazines]]></pubtype>
##     <publisher><![CDATA[IEEE]]></publisher>
##     <volume><![CDATA[2]]></volume>
##     <py><![CDATA[2014]]></py>
##     <spage><![CDATA[1]]></spage>
##     <epage><![CDATA[9]]></epage>
##     <abstract><![CDATA[Using a powered wheelchair (PW) is a complex task requiring advanced perceptual and motor control skills. Unfortunately, PW incidents and accidents are not uncommon and their consequences can be serious. The objective of this paper is to develop technological tools that can be used to characterize a wheelchair user's driving behavior under various settings. In the experiments conducted, PWs are outfitted with a datalogging platform that records, in real-time, the 3-D acceleration of the PW. Data collection was conducted over 35 different activities, designed to capture a spectrum of PW driving events performed at different speeds (collisions with fixed or moving objects, rolling on incline plane, and rolling across multiple types obstacles). The data was processed using time-series analysis and data mining techniques, to automatically detect and identify the different events. We compared the classification accuracy using four different types of time-series features: 1) time-delay embeddings; 2) time-domain characterization; 3) frequency-domain features; and 4) wavelet transforms. In the analysis, we compared the classification accuracy obtained when distinguishing between safe and unsafe events during each of the 35 different activities. For the purposes of this study, unsafe events were defined as activities containing collisions against objects at different speed, and the remainder were defined as safe events. We were able to accurately detect 98% of unsafe events, with a low (12%) false positive rate, using only five examples of each activity. This proof-of-concept study shows that the proposed approach has the potential of capturing, based on limited input from embedded sensors, contextual information on PW use, and of automatically characterizing a user's PW driving behavior.]]></abstract>
##     <issn><![CDATA[2168-2372]]></issn>
##     <htmlFlag><![CDATA[1]]></htmlFlag>
##     <arnumber><![CDATA[6940270]]></arnumber>
##     <doi><![CDATA[10.1109/JTEHM.2014.2365773]]></doi>
##     <publicationId><![CDATA[6940270]]></publicationId>
##     <mdurl><![CDATA[http://ieeexplore.ieee.org/xpl/articleDetails.jsp?tp=&arnumber=6940270&contentType=Journals+%26+Magazines]]></mdurl>
##     <pdf><![CDATA[http://ieeexplore.ieee.org/stamp/stamp.jsp?arnumber=6940270]]></pdf>
##   </document>
##   <document>
##     <rank>95</rank>
##     <title><![CDATA[Backward Fuzzy Rule Interpolation]]></title>
##     <authors><![CDATA[Shangzhu Jin;  Ren Diao;  Chai Quek;  Qiang Shen]]></authors>
##     <affiliations><![CDATA[Dept. of Comput. Sci., Aberystwyth Univ., Aberystwyth, UK]]></affiliations>
##     <controlledterms>
##       <term><![CDATA[fuzzy reasoning]]></term>
##       <term><![CDATA[interpolation]]></term>
##       <term><![CDATA[knowledge based systems]]></term>
##     </controlledterms>
##     <thesaurusterms>
##       <term><![CDATA[Bismuth]]></term>
##       <term><![CDATA[Cognition]]></term>
##       <term><![CDATA[Educational institutions]]></term>
##       <term><![CDATA[Extrapolation]]></term>
##       <term><![CDATA[Fuzzy sets]]></term>
##       <term><![CDATA[Interpolation]]></term>
##       <term><![CDATA[Safety]]></term>
##     </thesaurusterms>
##     <pubtitle><![CDATA[Fuzzy Systems, IEEE Transactions on]]></pubtitle>
##     <punumber><![CDATA[91]]></punumber>
##     <pubtype><![CDATA[Journals & Magazines]]></pubtype>
##     <publisher><![CDATA[IEEE]]></publisher>
##     <volume><![CDATA[22]]></volume>
##     <issue><![CDATA[6]]></issue>
##     <py><![CDATA[2014]]></py>
##     <spage><![CDATA[1682]]></spage>
##     <epage><![CDATA[1698]]></epage>
##     <abstract><![CDATA[Fuzzy rule interpolation offers a useful means to enhancing the robustness of fuzzy models by making inference possible in sparse rule-based systems. However, in real-world applications of interconnected rule bases, situations may arise when certain crucial antecedents are absent from given observations. If such missing antecedents were involved in the subsequent interpolation process, the final conclusion would not be deducible using conventional means. To address this important issue, a new approach named backward fuzzy rule interpolation and extrapolation (BFRIE) is proposed in this paper, allowing the observations, which directly relate to the conclusion to be inferred or interpolated from the known antecedents and conclusion. This approach supports both backward interpolation and extrapolation which involve multiple fuzzy rules, with each having multiple antecedents. As such, it significantly extends the existing fuzzy rule interpolation techniques. In particular, considering that there may be more than one antecedent value missing in an application problem, two methods are proposed in an attempt to perform backward interpolation with multiple missing antecedent values. Algorithms are given to implement the approaches via the use of the scale and move transformation-based fuzzy interpolation. Experimental studies that are based on a real-world scenario are provided to demonstrate the potential and efficacy of the proposed work.]]></abstract>
##     <issn><![CDATA[1063-6706]]></issn>
##     <htmlFlag><![CDATA[1]]></htmlFlag>
##     <arnumber><![CDATA[6728621]]></arnumber>
##     <doi><![CDATA[10.1109/TFUZZ.2014.2303474]]></doi>
##     <publicationId><![CDATA[6728621]]></publicationId>
##     <mdurl><![CDATA[http://ieeexplore.ieee.org/xpl/articleDetails.jsp?tp=&arnumber=6728621&contentType=Journals+%26+Magazines]]></mdurl>
##     <pdf><![CDATA[http://ieeexplore.ieee.org/stamp/stamp.jsp?arnumber=6728621]]></pdf>
##   </document>
##   <document>
##     <rank>96</rank>
##     <title><![CDATA[Effects of Antenna Switching on Band-Limited Spatial Modulation]]></title>
##     <authors><![CDATA[Ishibashi, K.;  Sugiura, S.]]></authors>
##     <affiliations><![CDATA[Adv. Wireless Commun. Res. Center (AWCC), Univ. of Electro-Commun., Chofu, Japan]]></affiliations>
##     <controlledterms>
##       <term><![CDATA[MIMO communication]]></term>
##       <term><![CDATA[filtering theory]]></term>
##       <term><![CDATA[modulation]]></term>
##       <term><![CDATA[pulse shaping]]></term>
##       <term><![CDATA[transmitting antennas]]></term>
##     </controlledterms>
##     <thesaurusterms>
##       <term><![CDATA[Bandwidth]]></term>
##       <term><![CDATA[MIMO]]></term>
##       <term><![CDATA[Pulse shaping methods]]></term>
##       <term><![CDATA[Radio frequency]]></term>
##       <term><![CDATA[Transmitting antennas]]></term>
##     </thesaurusterms>
##     <pubtitle><![CDATA[Wireless Communications Letters, IEEE]]></pubtitle>
##     <punumber><![CDATA[5962382]]></punumber>
##     <pubtype><![CDATA[Journals & Magazines]]></pubtype>
##     <publisher><![CDATA[IEEE]]></publisher>
##     <volume><![CDATA[3]]></volume>
##     <issue><![CDATA[4]]></issue>
##     <py><![CDATA[2014]]></py>
##     <spage><![CDATA[345]]></spage>
##     <epage><![CDATA[348]]></epage>
##     <abstract><![CDATA[In this letter, we investigate the effects of antenna switching on band-limited spatial modulation (SM), where the employment of an SM-specific practical time-limited shaping filter is taken into account. More specifically, we evaluate the SM scheme's performance penalty imposed on a bandwidth efficiency, which arises due to the time-limited pulse transmissions. Furthermore, to combat this limitation while maintaining lower RF chains than transmit antenna elements, we propose a multiple-RF antenna switching operation that allows us to exploit a longer pulse shape than that enabled by a single-RF SM scheme. It is demonstrated in our simulations that upon increasing the number of transmit antenna elements, the SM scheme is capable of outperforming the classic MIMO systems, under the assumptions of a low number of RF chains and realistic pulse shaping.]]></abstract>
##     <issn><![CDATA[2162-2337]]></issn>
##     <htmlFlag><![CDATA[1]]></htmlFlag>
##     <arnumber><![CDATA[6784012]]></arnumber>
##     <doi><![CDATA[10.1109/LWC.2014.2315819]]></doi>
##     <publicationId><![CDATA[6784012]]></publicationId>
##     <mdurl><![CDATA[http://ieeexplore.ieee.org/xpl/articleDetails.jsp?tp=&arnumber=6784012&contentType=Journals+%26+Magazines]]></mdurl>
##     <pdf><![CDATA[http://ieeexplore.ieee.org/stamp/stamp.jsp?arnumber=6784012]]></pdf>
##   </document>
##   <document>
##     <rank>97</rank>
##     <title><![CDATA[Including Anatomical and Functional Information in MC Simulation of PET and SPECT Brain Studies. Brain-VISET: A Voxel-Based Iterative Method]]></title>
##     <authors><![CDATA[Marti-Fuster, B.;  Esteban, O.;  Thielemans, K.;  Setoain, X.;  Santos, A.;  Ros, D.;  Pavia, J.]]></authors>
##     <affiliations><![CDATA[Physiol. Sci. Dept. I, Univ. of Barcelona-IDIBAPS, Barcelona, Spain]]></affiliations>
##     <controlledterms>
##       <term><![CDATA[Monte Carlo methods]]></term>
##       <term><![CDATA[biomedical MRI]]></term>
##       <term><![CDATA[brain]]></term>
##       <term><![CDATA[image reconstruction]]></term>
##       <term><![CDATA[iterative methods]]></term>
##       <term><![CDATA[medical image processing]]></term>
##       <term><![CDATA[positron emission tomography]]></term>
##       <term><![CDATA[single photon emission computed tomography]]></term>
##     </controlledterms>
##     <thesaurusterms>
##       <term><![CDATA[Attenuation]]></term>
##       <term><![CDATA[Brain modeling]]></term>
##       <term><![CDATA[Computed tomography]]></term>
##       <term><![CDATA[Histograms]]></term>
##       <term><![CDATA[Magnetic resonance imaging]]></term>
##       <term><![CDATA[Positron emission tomography]]></term>
##       <term><![CDATA[Single photon emission computed tomography]]></term>
##     </thesaurusterms>
##     <pubtitle><![CDATA[Medical Imaging, IEEE Transactions on]]></pubtitle>
##     <punumber><![CDATA[42]]></punumber>
##     <pubtype><![CDATA[Journals & Magazines]]></pubtype>
##     <publisher><![CDATA[IEEE]]></publisher>
##     <volume><![CDATA[33]]></volume>
##     <issue><![CDATA[10]]></issue>
##     <py><![CDATA[2014]]></py>
##     <spage><![CDATA[1931]]></spage>
##     <epage><![CDATA[1938]]></epage>
##     <abstract><![CDATA[Monte Carlo (MC) simulation provides a flexible and robust framework to efficiently evaluate and optimize image processing methods in emission tomography. In this work we present Brain-VISET (Voxel-based Iterative Simulation for Emission Tomography), a method that aims to simulate realistic [ <sup>99m</sup>Tc]-SPECT and [ <sup>18</sup>F]-PET brain databases by including anatomical and functional information. To this end, activity and attenuation maps generated using high-resolution anatomical images from patients were used as input maps in a MC projector to simulate SPECT or PET sinograms. The reconstructed images were compared with the corresponding real SPECT or PET studies in an iterative process where the activity inputs maps were being modified at each iteration. Datasets of 30 refractory epileptic patients were used to assess the new method. Each set consisted of structural images (MRI and CT) and functional studies (SPECT and PET), thereby allowing the inclusion of anatomical and functional variability in the simulation input models. SPECT and PET sinograms were obtained using the SimSET package and were reconstructed with the same protocols as those employed for the clinical studies. The convergence of Brain-VISET was evaluated by studying the behavior throughout iterations of the correlation coefficient, the quotient image histogram and a ROI analysis comparing simulated with real studies. The realism of generated maps was also evaluated. Our findings show that Brain-VISET is able to generate realistic SPECT and PET studies and that four iterations is a suitable number of iterations to guarantee a good agreement between simulated and real studies.]]></abstract>
##     <issn><![CDATA[0278-0062]]></issn>
##     <htmlFlag><![CDATA[1]]></htmlFlag>
##     <arnumber><![CDATA[6819831]]></arnumber>
##     <doi><![CDATA[10.1109/TMI.2014.2326041]]></doi>
##     <publicationId><![CDATA[6819831]]></publicationId>
##     <mdurl><![CDATA[http://ieeexplore.ieee.org/xpl/articleDetails.jsp?tp=&arnumber=6819831&contentType=Journals+%26+Magazines]]></mdurl>
##     <pdf><![CDATA[http://ieeexplore.ieee.org/stamp/stamp.jsp?arnumber=6819831]]></pdf>
##   </document>
##   <document>
##     <rank>98</rank>
##     <title><![CDATA[Mapping Biophysical Variables From Solar and Thermal Infrared Remote Sensing: Focus on Agricultural Landscapes With Spatial Heterogeneity]]></title>
##     <authors><![CDATA[Jacob, F.;  Weiss, M.]]></authors>
##     <affiliations><![CDATA[IRD-UMR LISAH, Montpellier, France]]></affiliations>
##     <controlledterms>
##       <term><![CDATA[geophysical techniques]]></term>
##       <term><![CDATA[remote sensing]]></term>
##       <term><![CDATA[vegetation]]></term>
##     </controlledterms>
##     <thesaurusterms>
##       <term><![CDATA[Biological system modeling]]></term>
##       <term><![CDATA[Land surface]]></term>
##       <term><![CDATA[Land surface temperature]]></term>
##       <term><![CDATA[Remote sensing]]></term>
##       <term><![CDATA[Spatial resolution]]></term>
##       <term><![CDATA[Temperature sensors]]></term>
##     </thesaurusterms>
##     <pubtitle><![CDATA[Geoscience and Remote Sensing Letters, IEEE]]></pubtitle>
##     <punumber><![CDATA[8859]]></punumber>
##     <pubtype><![CDATA[Journals & Magazines]]></pubtype>
##     <publisher><![CDATA[IEEE]]></publisher>
##     <volume><![CDATA[11]]></volume>
##     <issue><![CDATA[10]]></issue>
##     <py><![CDATA[2014]]></py>
##     <spage><![CDATA[1844]]></spage>
##     <epage><![CDATA[1848]]></epage>
##     <abstract><![CDATA[This letter closes a Special Stream that focuses on spatial heterogeneity when mapping biophysical variables over agricultural landscape from solar and thermal infrared remote sensing. We propose an overview of the highlights from prior research, we report the main results of the Special Stream, and we discuss future directions. The main outcomes of the Special Stream are related to: 1) the impact on the remotely sensed signal of canopy vertical distribution, shadowing effects, and multiple scattering; 2) the notion of spatial resolution limit in relation to spatial heterogeneity; and 3) the definition of an optimal sampling strategy to spatialize ground measurements.]]></abstract>
##     <issn><![CDATA[1545-598X]]></issn>
##     <htmlFlag><![CDATA[1]]></htmlFlag>
##     <arnumber><![CDATA[6799204]]></arnumber>
##     <doi><![CDATA[10.1109/LGRS.2014.2313592]]></doi>
##     <publicationId><![CDATA[6799204]]></publicationId>
##     <mdurl><![CDATA[http://ieeexplore.ieee.org/xpl/articleDetails.jsp?tp=&arnumber=6799204&contentType=Journals+%26+Magazines]]></mdurl>
##     <pdf><![CDATA[http://ieeexplore.ieee.org/stamp/stamp.jsp?arnumber=6799204]]></pdf>
##   </document>
##   <document>
##     <rank>99</rank>
##     <title><![CDATA[Radio-Frequency Rectifier for Electromagnetic Energy Harvesting: Development Path and Future Outlook]]></title>
##     <authors><![CDATA[Hemour, S.;  Ke Wu]]></authors>
##     <affiliations><![CDATA[Poly-Grames Res. Center, Ecole Polytech. de Montreal, Montreal, QC, Canada]]></affiliations>
##     <controlledterms>
##       <term><![CDATA[diodes]]></term>
##       <term><![CDATA[energy harvesting]]></term>
##       <term><![CDATA[rectification]]></term>
##       <term><![CDATA[rectifiers]]></term>
##       <term><![CDATA[transistors]]></term>
##     </controlledterms>
##     <thesaurusterms>
##       <term><![CDATA[Ambient networks]]></term>
##       <term><![CDATA[Electromagnetic devices]]></term>
##       <term><![CDATA[Energy conversion]]></term>
##       <term><![CDATA[Energy harvesting]]></term>
##       <term><![CDATA[Insulators]]></term>
##       <term><![CDATA[Radio frequency]]></term>
##       <term><![CDATA[Rectifiers]]></term>
##       <term><![CDATA[Renewable energy sources]]></term>
##       <term><![CDATA[Semiconductor diodes]]></term>
##       <term><![CDATA[Wireless communication]]></term>
##     </thesaurusterms>
##     <pubtitle><![CDATA[Proceedings of the IEEE]]></pubtitle>
##     <punumber><![CDATA[5]]></punumber>
##     <pubtype><![CDATA[Journals & Magazines]]></pubtype>
##     <publisher><![CDATA[IEEE]]></publisher>
##     <volume><![CDATA[102]]></volume>
##     <issue><![CDATA[11]]></issue>
##     <py><![CDATA[2014]]></py>
##     <spage><![CDATA[1667]]></spage>
##     <epage><![CDATA[1691]]></epage>
##     <abstract><![CDATA[The roadmap evolution and historical milestones of electromagnetic energy conversion techniques and related breakthroughs over the years are reviewed and presented with particular emphasis on low-density energy-harvest technologies. Electromagnetic sources responsible for the presence of ambient radio-frequency (RF) energy are examined and discussed. The effective use and recycling of such an ambient electromagnetic energy are the most relevant and critical issue for the current and future practicability of wireless energyharvesting devices and systems. In this paper, a set of performance criteria and development considerations, required to meet the need of applications of ambient electromagnetic energy harvesting, are also derived from the radiating source analysis. The criteria can be calculated from a simple measurement of the I-V nonlinear behavior of RF rectification devices such as diodes and transistors, as well as linear frequency behavior (S-parameters). The existing rectifying devices are then reviewed in light of the defined performance criteria. Finally, a technological outlook of the performances that can be expected from different device technologies is assessed and discussed. Since the proposed spindiode technology would present the most promising device platform in the development of the most useful ambient energy harvesters, a special highlight of this disruptive scheme is provided in the presentation of this work.]]></abstract>
##     <issn><![CDATA[0018-9219]]></issn>
##     <htmlFlag><![CDATA[1]]></htmlFlag>
##     <arnumber><![CDATA[6922611]]></arnumber>
##     <doi><![CDATA[10.1109/JPROC.2014.2358691]]></doi>
##     <publicationId><![CDATA[6922611]]></publicationId>
##     <mdurl><![CDATA[http://ieeexplore.ieee.org/xpl/articleDetails.jsp?tp=&arnumber=6922611&contentType=Journals+%26+Magazines]]></mdurl>
##     <pdf><![CDATA[http://ieeexplore.ieee.org/stamp/stamp.jsp?arnumber=6922611]]></pdf>
##   </document>
##   <document>
##     <rank>100</rank>
##     <title><![CDATA[Methods for Nuclei Detection, Segmentation, and Classification in Digital Histopathology: A Review&#x2014;Current Status and Future Potential]]></title>
##     <authors><![CDATA[Irshad, H.;  Veillard, A.;  Roux, L.;  Racoceanu, D.]]></authors>
##     <affiliations><![CDATA[Univ. Joseph Fourier, Grenoble, France]]></affiliations>
##     <controlledterms>
##       <term><![CDATA[brain]]></term>
##       <term><![CDATA[cancer]]></term>
##       <term><![CDATA[cellular biophysics]]></term>
##       <term><![CDATA[feature extraction]]></term>
##       <term><![CDATA[image classification]]></term>
##       <term><![CDATA[image segmentation]]></term>
##       <term><![CDATA[lung]]></term>
##       <term><![CDATA[medical image processing]]></term>
##       <term><![CDATA[reviews]]></term>
##     </controlledterms>
##     <thesaurusterms>
##       <term><![CDATA[Biomedical image processing]]></term>
##       <term><![CDATA[Cancer]]></term>
##       <term><![CDATA[Classification]]></term>
##       <term><![CDATA[Digital systems]]></term>
##       <term><![CDATA[Image segmentation]]></term>
##       <term><![CDATA[Microscopy]]></term>
##       <term><![CDATA[Pathology]]></term>
##     </thesaurusterms>
##     <pubtitle><![CDATA[Biomedical Engineering, IEEE Reviews in]]></pubtitle>
##     <punumber><![CDATA[4664312]]></punumber>
##     <pubtype><![CDATA[Journals & Magazines]]></pubtype>
##     <publisher><![CDATA[IEEE]]></publisher>
##     <volume><![CDATA[7]]></volume>
##     <py><![CDATA[2014]]></py>
##     <spage><![CDATA[97]]></spage>
##     <epage><![CDATA[114]]></epage>
##     <abstract><![CDATA[Digital pathology represents one of the major evolutions in modern medicine. Pathological examinations constitute the gold standard in many medical protocols, and also play a critical and legal role in the diagnosis process. In the conventional cancer diagnosis, pathologists analyze biopsies to make diagnostic and prognostic assessments, mainly based on the cell morphology and architecture distribution. Recently, computerized methods have been rapidly evolving in the area of digital pathology, with growing applications related to nuclei detection, segmentation, and classification. In cancer research, these approaches have played, and will continue to play a key (often bottleneck) role in minimizing human intervention, consolidating pertinent second opinions, and providing traceable clinical information. Pathological studies have been conducted for numerous cancer detection and grading applications, including brain, breast, cervix, lung, and prostate cancer grading. Our study presents, discusses, and extracts the major trends from an exhaustive overview of various nuclei detection, segmentation, feature computation, and classification techniques used in histopathology imagery, specifically in hematoxylin-eosin and immunohistochemical staining protocols. This study also enables us to measure the challenges that remain, in order to reach robust analysis of whole slide images, essential high content imaging with diagnostic biomarkers and prognosis support in digital pathology.]]></abstract>
##     <issn><![CDATA[1937-3333]]></issn>
##     <htmlFlag><![CDATA[1]]></htmlFlag>
##     <arnumber><![CDATA[6690201]]></arnumber>
##     <doi><![CDATA[10.1109/RBME.2013.2295804]]></doi>
##     <publicationId><![CDATA[6690201]]></publicationId>
##     <mdurl><![CDATA[http://ieeexplore.ieee.org/xpl/articleDetails.jsp?tp=&arnumber=6690201&contentType=Journals+%26+Magazines]]></mdurl>
##     <pdf><![CDATA[http://ieeexplore.ieee.org/stamp/stamp.jsp?arnumber=6690201]]></pdf>
##   </document>
##   <document>
##     <rank>101</rank>
##     <title><![CDATA[A Novel Piecewise Linear Recursive Convolution Approach for Dispersive Media Using the Finite-Difference Time-Domain Method]]></title>
##     <authors><![CDATA[Giannakis, I.;  Giannopoulos, A.]]></authors>
##     <affiliations><![CDATA[Inst. of Infrastruct. & Environ., Univ. of Edinburgh, Edinburgh, UK]]></affiliations>
##     <controlledterms>
##       <term><![CDATA[convolution]]></term>
##       <term><![CDATA[dispersive media]]></term>
##       <term><![CDATA[electric fields]]></term>
##       <term><![CDATA[electromagnetic wave propagation]]></term>
##       <term><![CDATA[finite difference time-domain analysis]]></term>
##       <term><![CDATA[piecewise linear techniques]]></term>
##       <term><![CDATA[recursive estimation]]></term>
##     </controlledterms>
##     <thesaurusterms>
##       <term><![CDATA[Convolution]]></term>
##       <term><![CDATA[Dispersion]]></term>
##       <term><![CDATA[Finite difference methods]]></term>
##       <term><![CDATA[Media]]></term>
##       <term><![CDATA[Permittivity]]></term>
##       <term><![CDATA[Time-domain analysis]]></term>
##     </thesaurusterms>
##     <pubtitle><![CDATA[Antennas and Propagation, IEEE Transactions on]]></pubtitle>
##     <punumber><![CDATA[8]]></punumber>
##     <pubtype><![CDATA[Journals & Magazines]]></pubtype>
##     <publisher><![CDATA[IEEE]]></publisher>
##     <volume><![CDATA[62]]></volume>
##     <issue><![CDATA[5]]></issue>
##     <py><![CDATA[2014]]></py>
##     <spage><![CDATA[2669]]></spage>
##     <epage><![CDATA[2678]]></epage>
##     <abstract><![CDATA[Two novel methods for implementing recursively the convolution between the electric field and a time dependent electric susceptibility function in the finite-difference time domain (FDTD) method are presented. Both resulting algorithms are straightforward to implement and employ an inclusive susceptibility function which holds as special cases the Lorentz, Debye, and Drude media relaxations. The accuracy of the new proposed algorithms is found to be systematically improved when compared to existing standard piecewise linear recursive convolution (PLRC) approaches, it is conjectured that the reason for this improvement is that the new proposed algorithms do not make any assumptions about the time variation of the polarization density in each time interval; no finite difference or semi-implicit schemes are used for the calculation of the polarization density. The only assumption that these two new methods make is that the first time derivative of the electric field is constant within each FDTD time interval.]]></abstract>
##     <issn><![CDATA[0018-926X]]></issn>
##     <htmlFlag><![CDATA[1]]></htmlFlag>
##     <arnumber><![CDATA[6748914]]></arnumber>
##     <doi><![CDATA[10.1109/TAP.2014.2308549]]></doi>
##     <publicationId><![CDATA[6748914]]></publicationId>
##     <mdurl><![CDATA[http://ieeexplore.ieee.org/xpl/articleDetails.jsp?tp=&arnumber=6748914&contentType=Journals+%26+Magazines]]></mdurl>
##     <pdf><![CDATA[http://ieeexplore.ieee.org/stamp/stamp.jsp?arnumber=6748914]]></pdf>
##   </document>
##   <document>
##     <rank>102</rank>
##     <title><![CDATA[MESCOS&#x2014;A Multienergy System Cosimulator for City District Energy Systems]]></title>
##     <authors><![CDATA[Molitor, C.;  Gross, S.;  Zeitz, J.;  Monti, A.]]></authors>
##     <affiliations><![CDATA[E.ON Energy Res. Center, Inst. for Autom. of Complex Power Syst., E.ON ERC, RWTH Aachen Univ., Aachen, Germany]]></affiliations>
##     <controlledterms>
##       <term><![CDATA[buildings (structures)]]></term>
##       <term><![CDATA[energy management systems]]></term>
##       <term><![CDATA[power engineering computing]]></term>
##       <term><![CDATA[power grids]]></term>
##       <term><![CDATA[power system control]]></term>
##       <term><![CDATA[power system simulation]]></term>
##       <term><![CDATA[software packages]]></term>
##     </controlledterms>
##     <thesaurusterms>
##       <term><![CDATA[Buildings]]></term>
##       <term><![CDATA[Energy management]]></term>
##       <term><![CDATA[Power system simulation]]></term>
##       <term><![CDATA[Simulation]]></term>
##       <term><![CDATA[Smart grids]]></term>
##       <term><![CDATA[Urban areas]]></term>
##     </thesaurusterms>
##     <pubtitle><![CDATA[Industrial Informatics, IEEE Transactions on]]></pubtitle>
##     <punumber><![CDATA[9424]]></punumber>
##     <pubtype><![CDATA[Journals & Magazines]]></pubtype>
##     <publisher><![CDATA[IEEE]]></publisher>
##     <volume><![CDATA[10]]></volume>
##     <issue><![CDATA[4]]></issue>
##     <py><![CDATA[2014]]></py>
##     <spage><![CDATA[2247]]></spage>
##     <epage><![CDATA[2256]]></epage>
##     <abstract><![CDATA[This work introduces a multidomain simulation platform that enables a holistic analysis of city district scale energy systems. The objective for the development of the simulation platform is to provide a tool that supports the design of control and energy management algorithms for those systems. The platform allows long-term simulations of a large number of buildings, including internal energy supply or energy conversion systems, in combination with external energy supply systems like the electrical grid. The simulation of those physical systems represents the environment for sophisticated control and energy management algorithms that can be tested on the platform. The concept of this work is to combine commercial-off-the-shelf software packages, here simulators and runtime infrastructure (RTI), to a high performance multidomain cosimulation platform. The high performance of the platform regarding computation time has been achieved by exploiting the parallel computing capabilities of modern simulation servers. Especially the computation time of large numbers of instances of Modelica-based models has been reduced significantly by the development of the parallel execution framework (PEF). The implementation of the PEF, including the interface to the individual models and to the RTI, is described in detail. The partitioning of the simulated system among different simulators does not influence the simulation results, as shown on the basis of a small-scale simulation scenario. The performance regarding the computation time is demonstrated on several example simulation scenarios showing the scalability of the platform.]]></abstract>
##     <issn><![CDATA[1551-3203]]></issn>
##     <htmlFlag><![CDATA[1]]></htmlFlag>
##     <arnumber><![CDATA[6846298]]></arnumber>
##     <doi><![CDATA[10.1109/TII.2014.2334058]]></doi>
##     <publicationId><![CDATA[6846298]]></publicationId>
##     <mdurl><![CDATA[http://ieeexplore.ieee.org/xpl/articleDetails.jsp?tp=&arnumber=6846298&contentType=Journals+%26+Magazines]]></mdurl>
##     <pdf><![CDATA[http://ieeexplore.ieee.org/stamp/stamp.jsp?arnumber=6846298]]></pdf>
##   </document>
##   <document>
##     <rank>103</rank>
##     <title><![CDATA[Improved Building Extraction With Integrated Decomposition of Time-Frequency and Entropy-Alpha Using Polarimetric SAR Data]]></title>
##     <authors><![CDATA[Lei Deng;  Cuizhen Wang]]></authors>
##     <affiliations><![CDATA[Coll. of Resource Environ. & Tourism, Capital Normal Univ., Beijing, China]]></affiliations>
##     <controlledterms>
##       <term><![CDATA[buildings (structures)]]></term>
##       <term><![CDATA[electromagnetic wave scattering]]></term>
##       <term><![CDATA[feature extraction]]></term>
##       <term><![CDATA[image classification]]></term>
##       <term><![CDATA[object detection]]></term>
##       <term><![CDATA[radar imaging]]></term>
##       <term><![CDATA[radar polarimetry]]></term>
##       <term><![CDATA[remote sensing by radar]]></term>
##       <term><![CDATA[synthetic aperture radar]]></term>
##       <term><![CDATA[time-frequency analysis]]></term>
##     </controlledterms>
##     <thesaurusterms>
##       <term><![CDATA[Azimuth]]></term>
##       <term><![CDATA[Buildings]]></term>
##       <term><![CDATA[Entropy]]></term>
##       <term><![CDATA[Remote sensing]]></term>
##       <term><![CDATA[Scattering]]></term>
##       <term><![CDATA[Synthetic aperture radar]]></term>
##       <term><![CDATA[Time-frequency analysis]]></term>
##     </thesaurusterms>
##     <pubtitle><![CDATA[Selected Topics in Applied Earth Observations and Remote Sensing, IEEE Journal of]]></pubtitle>
##     <punumber><![CDATA[4609443]]></punumber>
##     <pubtype><![CDATA[Journals & Magazines]]></pubtype>
##     <publisher><![CDATA[IEEE]]></publisher>
##     <volume><![CDATA[7]]></volume>
##     <issue><![CDATA[10]]></issue>
##     <py><![CDATA[2014]]></py>
##     <spage><![CDATA[4058]]></spage>
##     <epage><![CDATA[4068]]></epage>
##     <abstract><![CDATA[Building extraction is one of the primary applications of urban remote sensing. Polarimetric synthetic aperture radar (POLSAR), with its all-weather day and night imaging, canopy penetration and full polarimetric information, provides a unique way to detecting and characterizing urban areas. In this study, the time-frequency decomposition technique and the entropy/alpha-Wishart classifier were integrated to improve building extraction. The entropy/alpha-Wishart classifier was able to extract ortho-oriented buildings. After time-frequency transformation, the variation of entropy, alpha, anisotropy differs for objects with different scattering mechanisms, and the alpha angle of subaperture images was optimal in delineating slant-oriented buildings. A comparison between the integrated approach and the conventional entropy/alpha-Wishart classifier was performed on both C- and L-band NASA/JPL AIRSAR datasets. The overall accuracy and kappa value of the integrated approach was about 20% higher than that of the entropy/alpha-Wishart classifier. The C-band output tends to show more detailed scattering properties whereas the extracted buildings from the L-band image reveal better overall visual results.]]></abstract>
##     <issn><![CDATA[1939-1404]]></issn>
##     <htmlFlag><![CDATA[1]]></htmlFlag>
##     <arnumber><![CDATA[6616632]]></arnumber>
##     <doi><![CDATA[10.1109/JSTARS.2013.2281594]]></doi>
##     <publicationId><![CDATA[6616632]]></publicationId>
##     <mdurl><![CDATA[http://ieeexplore.ieee.org/xpl/articleDetails.jsp?tp=&arnumber=6616632&contentType=Journals+%26+Magazines]]></mdurl>
##     <pdf><![CDATA[http://ieeexplore.ieee.org/stamp/stamp.jsp?arnumber=6616632]]></pdf>
##   </document>
##   <document>
##     <rank>104</rank>
##     <title><![CDATA[Ultra-Broadband Silicon-Wire Polarization Beam Combiner/Splitter Based on a Wavelength Insensitive Coupler With a Point-Symmetrical Configuration]]></title>
##     <authors><![CDATA[Uematsu, T.;  Kitayama, T.;  Ishizaka, Y.;  Saitoh, K.]]></authors>
##     <affiliations><![CDATA[Grad. Sch. of Inf. Sci. & Technol., Hokkaido Univ., Sapporo, Japan]]></affiliations>
##     <controlledterms>
##       <term><![CDATA[nanophotonics]]></term>
##       <term><![CDATA[optical beam splitters]]></term>
##       <term><![CDATA[optical couplers]]></term>
##       <term><![CDATA[optical polarisers]]></term>
##       <term><![CDATA[optical waveguides]]></term>
##       <term><![CDATA[silicon]]></term>
##       <term><![CDATA[ultra wideband technology]]></term>
##     </controlledterms>
##     <thesaurusterms>
##       <term><![CDATA[Couplers]]></term>
##       <term><![CDATA[Couplings]]></term>
##       <term><![CDATA[Delay lines]]></term>
##       <term><![CDATA[Extinction ratio]]></term>
##       <term><![CDATA[Optical waveguides]]></term>
##       <term><![CDATA[Power generation]]></term>
##       <term><![CDATA[Silicon]]></term>
##     </thesaurusterms>
##     <pubtitle><![CDATA[Photonics Journal, IEEE]]></pubtitle>
##     <punumber><![CDATA[4563994]]></punumber>
##     <pubtype><![CDATA[Journals & Magazines]]></pubtype>
##     <publisher><![CDATA[IEEE]]></publisher>
##     <volume><![CDATA[6]]></volume>
##     <issue><![CDATA[1]]></issue>
##     <py><![CDATA[2014]]></py>
##     <spage><![CDATA[1]]></spage>
##     <epage><![CDATA[8]]></epage>
##     <abstract><![CDATA[An ultrabroadband silicon wire polarization beam combiner/splitter (PBCS) based on a wavelength-insensitive coupler is proposed. The proposed PBCS consists of three identical directional couplers and two identical delay lines. We design the PBCS using the 3-D finite element method. Numerical simulations show that the proposed PBCS can achieve the transmittance of more than 90% over a wide wavelength range from 1450 to 1650 nm for both TE and TM polarized modes.]]></abstract>
##     <issn><![CDATA[1943-0655]]></issn>
##     <htmlFlag><![CDATA[1]]></htmlFlag>
##     <arnumber><![CDATA[6725628]]></arnumber>
##     <doi><![CDATA[10.1109/JPHOT.2014.2302808]]></doi>
##     <publicationId><![CDATA[6725628]]></publicationId>
##     <mdurl><![CDATA[http://ieeexplore.ieee.org/xpl/articleDetails.jsp?tp=&arnumber=6725628&contentType=Journals+%26+Magazines]]></mdurl>
##     <pdf><![CDATA[http://ieeexplore.ieee.org/stamp/stamp.jsp?arnumber=6725628]]></pdf>
##   </document>
##   <document>
##     <rank>105</rank>
##     <title><![CDATA[Magnetic Field Sensor Based on U-Bent Single-Mode Fiber and Magnetic Fluid]]></title>
##     <authors><![CDATA[Tiegen Liu;  Yaofei Chen;  Qun Han;  Xiaoying Lu]]></authors>
##     <affiliations><![CDATA[Coll. of Precision Instrum. & Opto-Electron. Eng., Tianjin Univ., Tianjin, China]]></affiliations>
##     <controlledterms>
##       <term><![CDATA[absorption coefficients]]></term>
##       <term><![CDATA[fibre optic sensors]]></term>
##       <term><![CDATA[intensity modulation]]></term>
##       <term><![CDATA[magnetic field measurement]]></term>
##       <term><![CDATA[magnetic fluids]]></term>
##       <term><![CDATA[magnetic sensors]]></term>
##       <term><![CDATA[optical tuning]]></term>
##       <term><![CDATA[refractive index]]></term>
##     </controlledterms>
##     <thesaurusterms>
##       <term><![CDATA[Magnetic fields]]></term>
##       <term><![CDATA[Magnetic liquids]]></term>
##       <term><![CDATA[Magnetostriction]]></term>
##       <term><![CDATA[Optical fiber sensors]]></term>
##       <term><![CDATA[Optical fibers]]></term>
##       <term><![CDATA[Sensitivity]]></term>
##       <term><![CDATA[Temperature sensors]]></term>
##     </thesaurusterms>
##     <pubtitle><![CDATA[Photonics Journal, IEEE]]></pubtitle>
##     <punumber><![CDATA[4563994]]></punumber>
##     <pubtype><![CDATA[Journals & Magazines]]></pubtype>
##     <publisher><![CDATA[IEEE]]></publisher>
##     <volume><![CDATA[6]]></volume>
##     <issue><![CDATA[6]]></issue>
##     <py><![CDATA[2014]]></py>
##     <spage><![CDATA[1]]></spage>
##     <epage><![CDATA[7]]></epage>
##     <abstract><![CDATA[In this paper, an all-fiber magnetic field sensor based on a U-bent single-mode fiber and magnetic fluid (MF) is proposed and investigated. Because of the tunable refractive index and absorption coefficient of MF, the transmission spectrum will change with the magnetic field strength (H), which can be used to demodulate H through the wavelength shift or the intensity change. The influence of the diameter of the U shape to the performance of the sensor is investigated and discussed. In the experiments, the highest sensitivities achieved with wavelength and intensity demodulation are 0.374 nm/Oe and -0.4821 dB/Oe, respectively. The reproducibility of the sensor is studied as well.]]></abstract>
##     <issn><![CDATA[1943-0655]]></issn>
##     <htmlFlag><![CDATA[1]]></htmlFlag>
##     <arnumber><![CDATA[6951405]]></arnumber>
##     <doi><![CDATA[10.1109/JPHOT.2014.2368781]]></doi>
##     <publicationId><![CDATA[6951405]]></publicationId>
##     <mdurl><![CDATA[http://ieeexplore.ieee.org/xpl/articleDetails.jsp?tp=&arnumber=6951405&contentType=Journals+%26+Magazines]]></mdurl>
##     <pdf><![CDATA[http://ieeexplore.ieee.org/stamp/stamp.jsp?arnumber=6951405]]></pdf>
##   </document>
##   <document>
##     <rank>106</rank>
##     <title><![CDATA[Modeling and Analysis of Harmonic Stability in an AC Power-Electronics-Based Power System]]></title>
##     <authors><![CDATA[Xiongfei Wang;  Blaabjerg, F.;  Weimin Wu]]></authors>
##     <affiliations><![CDATA[Dept. of Energy Technol., Aalborg Univ., Aalborg, Denmark]]></affiliations>
##     <controlledterms>
##       <term><![CDATA[Nyquist diagrams]]></term>
##       <term><![CDATA[electric current control]]></term>
##       <term><![CDATA[invertors]]></term>
##       <term><![CDATA[power electronics]]></term>
##       <term><![CDATA[power harmonic filters]]></term>
##       <term><![CDATA[power system control]]></term>
##       <term><![CDATA[time-domain analysis]]></term>
##       <term><![CDATA[voltage control]]></term>
##     </controlledterms>
##     <thesaurusterms>
##       <term><![CDATA[Harmonic analysis]]></term>
##       <term><![CDATA[Impedance]]></term>
##       <term><![CDATA[Inverters]]></term>
##       <term><![CDATA[Power system harmonics]]></term>
##       <term><![CDATA[Power system stability]]></term>
##       <term><![CDATA[Stability analysis]]></term>
##       <term><![CDATA[Voltage control]]></term>
##     </thesaurusterms>
##     <pubtitle><![CDATA[Power Electronics, IEEE Transactions on]]></pubtitle>
##     <punumber><![CDATA[63]]></punumber>
##     <pubtype><![CDATA[Journals & Magazines]]></pubtype>
##     <publisher><![CDATA[IEEE]]></publisher>
##     <volume><![CDATA[29]]></volume>
##     <issue><![CDATA[12]]></issue>
##     <py><![CDATA[2014]]></py>
##     <spage><![CDATA[6421]]></spage>
##     <epage><![CDATA[6432]]></epage>
##     <abstract><![CDATA[This paper addresses the harmonic stability caused by the interactions among the wideband control of power converters and passive components in an ac power-electronics-based power system. The impedance-based analytical approach is employed and expanded to a meshed and balanced three-phase network which is dominated by multiple current- and voltage-controlled inverters with LCL- and LC-filters. A method of deriving the impedance ratios for the different inverters is proposed by means of the nodal admittance matrix. Thus, the contribution of each inverter to the harmonic stability of the power system can be readily predicted through Nyquist diagrams. Time-domain simulations and experimental tests on a three-inverter-based power system are presented. The results validate the effectiveness of the theoretical approach.]]></abstract>
##     <issn><![CDATA[0885-8993]]></issn>
##     <htmlFlag><![CDATA[1]]></htmlFlag>
##     <arnumber><![CDATA[6740802]]></arnumber>
##     <doi><![CDATA[10.1109/TPEL.2014.2306432]]></doi>
##     <publicationId><![CDATA[6740802]]></publicationId>
##     <mdurl><![CDATA[http://ieeexplore.ieee.org/xpl/articleDetails.jsp?tp=&arnumber=6740802&contentType=Journals+%26+Magazines]]></mdurl>
##     <pdf><![CDATA[http://ieeexplore.ieee.org/stamp/stamp.jsp?arnumber=6740802]]></pdf>
##   </document>
##   <document>
##     <rank>107</rank>
##     <title><![CDATA[Catheter-Induced Errors in Pressure Measurements in Vessels: An In-Vitro and Numerical Study]]></title>
##     <authors><![CDATA[de Vecchi, A.;  Clough, R.E.;  Gaddum, N.R.;  Rutten, M.C.M.;  Lamata, P.;  Schaeffter, T.;  Nordsletten, D.A.;  Smith, N.P.]]></authors>
##     <affiliations><![CDATA[Dept. of Biomed. Eng. & Imaging Sci. Div., King's Coll. London, London, UK]]></affiliations>
##     <controlledterms>
##       <term><![CDATA[blood pressure measurement]]></term>
##       <term><![CDATA[blood vessels]]></term>
##       <term><![CDATA[cardiovascular system]]></term>
##       <term><![CDATA[catheters]]></term>
##       <term><![CDATA[diseases]]></term>
##     </controlledterms>
##     <thesaurusterms>
##       <term><![CDATA[Biomedical measurement]]></term>
##       <term><![CDATA[Catheters]]></term>
##       <term><![CDATA[Frequency measurement]]></term>
##       <term><![CDATA[In vitro]]></term>
##       <term><![CDATA[Numerical models]]></term>
##       <term><![CDATA[Pressure measurement]]></term>
##       <term><![CDATA[Wires]]></term>
##     </thesaurusterms>
##     <pubtitle><![CDATA[Biomedical Engineering, IEEE Transactions on]]></pubtitle>
##     <punumber><![CDATA[10]]></punumber>
##     <pubtype><![CDATA[Journals & Magazines]]></pubtype>
##     <publisher><![CDATA[IEEE]]></publisher>
##     <volume><![CDATA[61]]></volume>
##     <issue><![CDATA[6]]></issue>
##     <py><![CDATA[2014]]></py>
##     <spage><![CDATA[1844]]></spage>
##     <epage><![CDATA[1850]]></epage>
##     <abstract><![CDATA[Accurate measurement of blood pressure is important because it is a biomarker for cardiovascular disease. Diagnostic catheterization is routinely used for pressure acquisition in vessels despite being subject to significant measurement errors. To investigate these errors, this study compares pressure measurement using two different techniques in vitro and numerical simulations. Pressure was acquired in a pulsatile flow phantom using a 6F fluid-filled catheter and a 0.014'' pressure wire, which is considered the current gold standard. Numerical simulations of the experimental set-up with and without a catheter were also performed. Despite the low catheter-to-vessel radius ratio, the catheter traces showed a 24% peak systolic pressure overestimation compared to the wire. The numerical models replicated this difference and indicated the cause for overestimation was the increased flow resistance due to the presence of the catheter. Further, the higher frequency pressure oscillations observed in the wire and numerical data were absent in the catheter, resulting in an overestimation of the pulse wave velocity with the latter modality. These results show that catheter geometry produces significant measurement bias in both the peak pressure and the waveform shape even with radius ratios considered acceptable in clinical practice. The wire allows for more accurate pressure quantification, in agreement with the numerical model without a catheter.]]></abstract>
##     <issn><![CDATA[0018-9294]]></issn>
##     <htmlFlag><![CDATA[1]]></htmlFlag>
##     <arnumber><![CDATA[6748872]]></arnumber>
##     <doi><![CDATA[10.1109/TBME.2014.2308594]]></doi>
##     <publicationId><![CDATA[6748872]]></publicationId>
##     <mdurl><![CDATA[http://ieeexplore.ieee.org/xpl/articleDetails.jsp?tp=&arnumber=6748872&contentType=Journals+%26+Magazines]]></mdurl>
##     <pdf><![CDATA[http://ieeexplore.ieee.org/stamp/stamp.jsp?arnumber=6748872]]></pdf>
##   </document>
##   <document>
##     <rank>108</rank>
##     <title><![CDATA[An adaptive home-use robotic rehabilitation system for the upper body]]></title>
##     <authors><![CDATA[Dowling, A.V.;  Barzilay, O.;  Lombrozo, Y.;  Wolf, A.]]></authors>
##     <affiliations><![CDATA[Fac. of Mech. Eng., Technion - Israel Inst. of Technol., Haifa, Israel]]></affiliations>
##     <controlledterms>
##       <term><![CDATA[adaptive systems]]></term>
##       <term><![CDATA[biomechanics]]></term>
##       <term><![CDATA[electromyography]]></term>
##       <term><![CDATA[manipulators]]></term>
##       <term><![CDATA[mean square error methods]]></term>
##       <term><![CDATA[medical disorders]]></term>
##       <term><![CDATA[medical robotics]]></term>
##       <term><![CDATA[neural nets]]></term>
##       <term><![CDATA[patient rehabilitation]]></term>
##       <term><![CDATA[pressure control]]></term>
##       <term><![CDATA[sensors]]></term>
##     </controlledterms>
##     <thesaurusterms>
##       <term><![CDATA[Electromyography]]></term>
##       <term><![CDATA[Musculoskeletal sysgem]]></term>
##       <term><![CDATA[Patient rehabilitation]]></term>
##       <term><![CDATA[Robot sensing systems]]></term>
##       <term><![CDATA[Three-dimensional displays]]></term>
##       <term><![CDATA[Visualization]]></term>
##     </thesaurusterms>
##     <pubtitle><![CDATA[Translational Engineering in Health and Medicine, IEEE Journal of]]></pubtitle>
##     <punumber><![CDATA[6221039]]></punumber>
##     <pubtype><![CDATA[Journals & Magazines]]></pubtype>
##     <publisher><![CDATA[IEEE]]></publisher>
##     <volume><![CDATA[2]]></volume>
##     <py><![CDATA[2014]]></py>
##     <spage><![CDATA[1]]></spage>
##     <epage><![CDATA[10]]></epage>
##     <abstract><![CDATA[Robotic rehabilitation systems have been developed to treat musculoskeletal conditions, but limited availability prevents most patients from using them. The objective of this paper was to create a home-use robotic rehabilitation system. Data were obtained in real time from a Microsoft Kinect<sup>TM</sup> and a wireless surface electromyograph system. Results from the Kinect<sup>TM</sup> sensor were compared to a standard motion capture system. A subject completed visual follow exercise tasks in a 3-D visual environment. Data from two training exercises were used to generate a neural network, which was then used to simulate the subject's individual performance. The subjects completed both the exercise task output from the neural network (custom), and the unmodified task (standard). In addition, a wearable arm robot prototype was built. Basic system identification was completed, and a control algorithm for the robot based on pressure control was designed and tested. The subjects had greater root-mean-square error for position and velocity variables during the custom exercise tasks. These results suggest that the custom task was difficult to complete, possibly because the neural network was unconstrained. Finally, the robot prototype was able to mimic changes in a subject's elbow angle in real time, demonstrating the feasibility of the robotic rehabilitation system.]]></abstract>
##     <issn><![CDATA[2168-2372]]></issn>
##     <htmlFlag><![CDATA[1]]></htmlFlag>
##     <arnumber><![CDATA[6779647]]></arnumber>
##     <doi><![CDATA[10.1109/JTEHM.2014.2314097]]></doi>
##     <publicationId><![CDATA[6779647]]></publicationId>
##     <mdurl><![CDATA[http://ieeexplore.ieee.org/xpl/articleDetails.jsp?tp=&arnumber=6779647&contentType=Journals+%26+Magazines]]></mdurl>
##     <pdf><![CDATA[http://ieeexplore.ieee.org/stamp/stamp.jsp?arnumber=6779647]]></pdf>
##   </document>
##   <document>
##     <rank>109</rank>
##     <title><![CDATA[The Unified-FFT Algorithm for Fast Electromagnetic Analysis of Planar Integrated Circuits Printed on Layered Media Inside a Rectangular Enclosure]]></title>
##     <authors><![CDATA[Rautio, B.J.;  Okhmatovski, V.I.;  Cangellaris, A.C.;  Rautio, J.C.;  Lee, J.K.]]></authors>
##     <affiliations><![CDATA[Dept. of Electr. Eng. & Comput. Sci., Syracuse Univ., Syracuse, NY, USA]]></affiliations>
##     <controlledterms>
##       <term><![CDATA[Green's function methods]]></term>
##       <term><![CDATA[computational complexity]]></term>
##       <term><![CDATA[discrete Fourier transforms]]></term>
##       <term><![CDATA[electromagnetic shielding]]></term>
##       <term><![CDATA[integrated circuits]]></term>
##       <term><![CDATA[method of moments]]></term>
##     </controlledterms>
##     <thesaurusterms>
##       <term><![CDATA[Electromagnetic waveguides]]></term>
##       <term><![CDATA[Fast Fourier transforms]]></term>
##       <term><![CDATA[Iterative methods]]></term>
##       <term><![CDATA[Method of moments]]></term>
##       <term><![CDATA[Nonhomogeneous media]]></term>
##       <term><![CDATA[Testing]]></term>
##       <term><![CDATA[Transmission line matrix methods]]></term>
##     </thesaurusterms>
##     <pubtitle><![CDATA[Microwave Theory and Techniques, IEEE Transactions on]]></pubtitle>
##     <punumber><![CDATA[22]]></punumber>
##     <pubtype><![CDATA[Journals & Magazines]]></pubtype>
##     <publisher><![CDATA[IEEE]]></publisher>
##     <volume><![CDATA[62]]></volume>
##     <issue><![CDATA[5]]></issue>
##     <py><![CDATA[2014]]></py>
##     <spage><![CDATA[1112]]></spage>
##     <epage><![CDATA[1121]]></epage>
##     <abstract><![CDATA[The unified fast Fourier transform (UFFT) methodology is proposed for fast method of moments analysis of dense integrated circuits embedded in layered media inside perfectly electric conducting or perfectly magnetic conducting enclosures of rectangular cross section. The pre-corrected fast Fourier transform (FFT) method is modified to handle the dyadic Green's function (DGF) of shielded layered media through factorization of the DGF into four convolution/correlation terms enabling fast matrix solve operations (MSOs). Calculation of the impedance matrix elements in the form of an infinite series of waveguide modes is cast into the form of a 2-D discrete Fourier transform allowing for fast FFT-accelerated matrix fill operations (MFOs). Fast FFT-enhanced MSOs and MFOs used in conjunction form the UFFT method. The computational complexity and memory requirements for the proposed UFFT solver scale as O(NlogN) and O(N), respectively, where N is the number of unknowns in the discrete approximation of the governing integral equation. New criteria specific to shielded circuits for the projection of the current expansion functions on a uniform FFT grid are developed. The accuracy and efficiency of the solver is demonstrated through its application to multiple examples of full-wave analysis of large planar circuits.]]></abstract>
##     <issn><![CDATA[0018-9480]]></issn>
##     <htmlFlag><![CDATA[1]]></htmlFlag>
##     <arnumber><![CDATA[6797982]]></arnumber>
##     <doi><![CDATA[10.1109/TMTT.2014.2315594]]></doi>
##     <publicationId><![CDATA[6797982]]></publicationId>
##     <mdurl><![CDATA[http://ieeexplore.ieee.org/xpl/articleDetails.jsp?tp=&arnumber=6797982&contentType=Journals+%26+Magazines]]></mdurl>
##     <pdf><![CDATA[http://ieeexplore.ieee.org/stamp/stamp.jsp?arnumber=6797982]]></pdf>
##   </document>
##   <document>
##     <rank>110</rank>
##     <title><![CDATA[The Alternate Arm Converter: A New Hybrid Multilevel Converter With DC-Fault Blocking Capability]]></title>
##     <authors><![CDATA[Merlin, M.M.C.;  Green, T.C.;  Mitcheson, P.D.;  Trainer, D.R.;  Critchley, R.;  Crookes, W.;  Hassan, F.]]></authors>
##     <affiliations><![CDATA[Dept. of EEE, Imperial Coll. London, London, UK]]></affiliations>
##     <controlledterms>
##       <term><![CDATA[HVDC power convertors]]></term>
##       <term><![CDATA[power grids]]></term>
##       <term><![CDATA[power system faults]]></term>
##       <term><![CDATA[static VAr compensators]]></term>
##     </controlledterms>
##     <thesaurusterms>
##       <term><![CDATA[Automatic voltage control]]></term>
##       <term><![CDATA[Insulated gate bipolar transistors]]></term>
##       <term><![CDATA[Mathematical model]]></term>
##       <term><![CDATA[Power conversion]]></term>
##       <term><![CDATA[Topology]]></term>
##     </thesaurusterms>
##     <pubtitle><![CDATA[Power Delivery, IEEE Transactions on]]></pubtitle>
##     <punumber><![CDATA[61]]></punumber>
##     <pubtype><![CDATA[Journals & Magazines]]></pubtype>
##     <publisher><![CDATA[IEEE]]></publisher>
##     <volume><![CDATA[29]]></volume>
##     <issue><![CDATA[1]]></issue>
##     <py><![CDATA[2014]]></py>
##     <spage><![CDATA[310]]></spage>
##     <epage><![CDATA[317]]></epage>
##     <abstract><![CDATA[This paper explains the working principles, supported by simulation results, of a new converter topology intended for HVDC applications, called the alternate arm converter (AAC). It is a hybrid between the modular multilevel converter, because of the presence of H-bridge cells, and the two-level converter, in the form of director switches in each arm. This converter is able to generate a multilevel ac voltage and since its stacks of cells consist of H-bridge cells instead of half-bridge cells, they are able to generate higher ac voltage than the dc terminal voltage. This allows the AAC to operate at an optimal point, called the &#x201C;sweet spot,&#x201D; where the ac and dc energy flows equal. The director switches in the AAC are responsible for alternating the conduction period of each arm, leading to a significant reduction in the number of cells in the stacks. Furthermore, the AAC can keep control of the current in the phase reactor even in case of a dc-side fault and support the ac grid, through a STATCOM mode. Simulation results and loss calculations are presented in this paper in order to support the claimed features of the AAC.]]></abstract>
##     <issn><![CDATA[0885-8977]]></issn>
##     <htmlFlag><![CDATA[1]]></htmlFlag>
##     <arnumber><![CDATA[6623197]]></arnumber>
##     <doi><![CDATA[10.1109/TPWRD.2013.2282171]]></doi>
##     <publicationId><![CDATA[6623197]]></publicationId>
##     <mdurl><![CDATA[http://ieeexplore.ieee.org/xpl/articleDetails.jsp?tp=&arnumber=6623197&contentType=Journals+%26+Magazines]]></mdurl>
##     <pdf><![CDATA[http://ieeexplore.ieee.org/stamp/stamp.jsp?arnumber=6623197]]></pdf>
##   </document>
##   <document>
##     <rank>111</rank>
##     <title><![CDATA[Measurement of Electrical Coupling Between Cardiac Ablation Catheters and Tissue]]></title>
##     <authors><![CDATA[Deno, D.C.;  Sih, H.J.;  Miller, S.P.;  Teplitsky, L.R.;  Kuenzi, R.]]></authors>
##     <affiliations><![CDATA[Cardiovascular & Ablation Technol. Div., St. Jude Med., St. Paul, MN, USA]]></affiliations>
##     <controlledterms>
##       <term><![CDATA[bioelectric phenomena]]></term>
##       <term><![CDATA[biomedical electrodes]]></term>
##       <term><![CDATA[cardiology]]></term>
##       <term><![CDATA[catheters]]></term>
##       <term><![CDATA[medical disorders]]></term>
##       <term><![CDATA[muscle]]></term>
##     </controlledterms>
##     <thesaurusterms>
##       <term><![CDATA[Biomedical measurement]]></term>
##       <term><![CDATA[Catheters]]></term>
##       <term><![CDATA[Electrodes]]></term>
##       <term><![CDATA[Impedance]]></term>
##       <term><![CDATA[Impedance measurement]]></term>
##       <term><![CDATA[Radio frequency]]></term>
##       <term><![CDATA[Surface impedance]]></term>
##     </thesaurusterms>
##     <pubtitle><![CDATA[Biomedical Engineering, IEEE Transactions on]]></pubtitle>
##     <punumber><![CDATA[10]]></punumber>
##     <pubtype><![CDATA[Journals & Magazines]]></pubtype>
##     <publisher><![CDATA[IEEE]]></publisher>
##     <volume><![CDATA[61]]></volume>
##     <issue><![CDATA[3]]></issue>
##     <py><![CDATA[2014]]></py>
##     <spage><![CDATA[765]]></spage>
##     <epage><![CDATA[774]]></epage>
##     <abstract><![CDATA[Managing cardiac arrhythmias with catheter ablation requires positioning electrodes in contact with myocardial tissue. Objective measures to assess contact and effective coupling of ablation energy are sought. An electrical coupling index (ECI) was devised using complex impedance at 20 kHz to perform in the presence of RF ablation and deliver information about electrical interactions between the tip electrode and its adjacent environment. ECI was derived and compared with clinical judgment, pacing threshold, electrogram amplitude, and ablation lesion depth and transmurality in a porcine model. ECI was also compared with force and displacement using ex vivo bovine myocardial muscle. Mean noncontact ECI was 97.2 &#x00B1; 14.3 and increased to 145.2 &#x00B1; 33.6 (p &lt;; 0.001) in clinician assessed (CLIN) moderate contact. ECI significantly improved CLIN's prediction of the variance in pacing threshold from 48.7% to 56.8% ( ). ECI was indicative of contact force under conditions of smooth myocardium. Transmural lesions were associated with higher pre-RF (109 &#x00B1; 17 versus 149 &#x00B1; 25, ) and during-RF (82 &#x00B1; 9 versus 101 &#x00B1; 17, ) ECI levels. ECI is a tip specific, robust, correlate with contact and ablation efficacy, and can potentially add to clinical interpretation of electrical coupling during electrophysiology procedures.]]></abstract>
##     <issn><![CDATA[0018-9294]]></issn>
##     <htmlFlag><![CDATA[1]]></htmlFlag>
##     <arnumber><![CDATA[6656831]]></arnumber>
##     <doi><![CDATA[10.1109/TBME.2013.2289328]]></doi>
##     <publicationId><![CDATA[6656831]]></publicationId>
##     <mdurl><![CDATA[http://ieeexplore.ieee.org/xpl/articleDetails.jsp?tp=&arnumber=6656831&contentType=Journals+%26+Magazines]]></mdurl>
##     <pdf><![CDATA[http://ieeexplore.ieee.org/stamp/stamp.jsp?arnumber=6656831]]></pdf>
##   </document>
##   <document>
##     <rank>112</rank>
##     <title><![CDATA[Chirped frequency transfer: a tool for synchronization and time transfer]]></title>
##     <authors><![CDATA[Raupach, S.M.F.;  Grosche, G.]]></authors>
##     <affiliations><![CDATA[Phys.-Tech. Bundesanstalt (PTB), Braunschweig, Germany]]></affiliations>
##     <controlledterms>
##       <term><![CDATA[delay circuits]]></term>
##       <term><![CDATA[frequency measurement]]></term>
##       <term><![CDATA[geometry]]></term>
##       <term><![CDATA[logic gates]]></term>
##       <term><![CDATA[oscillators]]></term>
##       <term><![CDATA[synchronisation]]></term>
##       <term><![CDATA[time measurement]]></term>
##     </controlledterms>
##     <thesaurusterms>
##       <term><![CDATA[Chirp]]></term>
##       <term><![CDATA[Delays]]></term>
##       <term><![CDATA[Frequency synchronization]]></term>
##       <term><![CDATA[Radiation detectors]]></term>
##       <term><![CDATA[Synchronization]]></term>
##       <term><![CDATA[Time-frequency analysis]]></term>
##     </thesaurusterms>
##     <pubtitle><![CDATA[Ultrasonics, Ferroelectrics, and Frequency Control, IEEE Transactions on]]></pubtitle>
##     <punumber><![CDATA[58]]></punumber>
##     <pubtype><![CDATA[Journals & Magazines]]></pubtype>
##     <publisher><![CDATA[IEEE]]></publisher>
##     <volume><![CDATA[61]]></volume>
##     <issue><![CDATA[6]]></issue>
##     <py><![CDATA[2014]]></py>
##     <spage><![CDATA[920]]></spage>
##     <epage><![CDATA[929]]></epage>
##     <abstract><![CDATA[We propose and demonstrate the phase-stabilized transfer of a chirped frequency as a tool for synchronization and time transfer. Technically, this is done by evaluating remote measurements of the transferred, chirped frequency. The gates of the frequency counters, here driven by a 10-MHz oscillation derived from a hydrogen maser, play a role analogous to the 1-pulse per second (PPS) signals usually employed for time transfer. In general, for time transfer, the gates consequently must be related to the external clock. Synchronizing observations based on frequency measurements, on the other hand, only requires a stable oscillator driving the frequency counters. In a proof of principle, we demonstrate the suppression of symmetrical delays, such as the geometrical path delay. We transfer an optical frequency chirped by around 240 kHz/s over a fiber link of around 149 km. We observe an accuracy and simultaneity, as well as a precision (Allan deviation, 18 000 s averaging interval) of the transferred frequency of around 2 &#x00D7; 10<sup>-19</sup>. We apply chirped frequency transfer to remote measurements of the synchronization between two counters' gate intervals. Here, we find a precision of around 200 ps at an estimated overall uncertainty of around 500 ps. The measurement results agree with those obtained from reference measurements, being well within the uncertainty. In the present setup, timing offsets up to 4 min can be measured unambiguously. We indicate how this range can be extended further.]]></abstract>
##     <issn><![CDATA[0885-3010]]></issn>
##     <htmlFlag><![CDATA[1]]></htmlFlag>
##     <arnumber><![CDATA[6819208]]></arnumber>
##     <doi><![CDATA[10.1109/TUFFC.2014.2988]]></doi>
##     <publicationId><![CDATA[6819208]]></publicationId>
##     <mdurl><![CDATA[http://ieeexplore.ieee.org/xpl/articleDetails.jsp?tp=&arnumber=6819208&contentType=Journals+%26+Magazines]]></mdurl>
##     <pdf><![CDATA[http://ieeexplore.ieee.org/stamp/stamp.jsp?arnumber=6819208]]></pdf>
##   </document>
##   <document>
##     <rank>113</rank>
##     <title><![CDATA[Applications of Fiber Lasers for the Development of Compact Photonic Devices]]></title>
##     <authors><![CDATA[Mary, R.;  Choudhury, D.;  Kar, A.K.]]></authors>
##     <affiliations><![CDATA[Nonlinear Opt. Group, Heriot-Watt Univ., Edinburgh, UK]]></affiliations>
##     <controlledterms>
##       <term><![CDATA[fibre lasers]]></term>
##       <term><![CDATA[high-speed optical techniques]]></term>
##       <term><![CDATA[laser beam applications]]></term>
##       <term><![CDATA[ytterbium]]></term>
##     </controlledterms>
##     <thesaurusterms>
##       <term><![CDATA[Fiber lasers]]></term>
##       <term><![CDATA[Laser beams]]></term>
##       <term><![CDATA[Materials]]></term>
##       <term><![CDATA[Optical fibers]]></term>
##       <term><![CDATA[Refractive index]]></term>
##       <term><![CDATA[Waveguide lasers]]></term>
##     </thesaurusterms>
##     <pubtitle><![CDATA[Selected Topics in Quantum Electronics, IEEE Journal of]]></pubtitle>
##     <punumber><![CDATA[2944]]></punumber>
##     <pubtype><![CDATA[Journals & Magazines]]></pubtype>
##     <publisher><![CDATA[IEEE]]></publisher>
##     <volume><![CDATA[20]]></volume>
##     <issue><![CDATA[5]]></issue>
##     <py><![CDATA[2014]]></py>
##     <spage><![CDATA[72]]></spage>
##     <epage><![CDATA[84]]></epage>
##     <abstract><![CDATA[Ultrafast fiber lasers, with their distinct features of high stability, superior beam quality, compactness and power scalability have revolutionized a variety of applications, ranging from micromachining and medical diagnostics to basic research. One of the applications include Ultrafast Laser Inscription, a technology that has considerably improved and diversified with advances in stable, high power Ytterbium-doped fiber lasers. This paper explores the highly interdisciplinary application realm of Ultrafast Laser Inscription for the development of novel photonic and optofluidic devices.]]></abstract>
##     <issn><![CDATA[1077-260X]]></issn>
##     <htmlFlag><![CDATA[1]]></htmlFlag>
##     <arnumber><![CDATA[6716015]]></arnumber>
##     <doi><![CDATA[10.1109/JSTQE.2014.2301136]]></doi>
##     <publicationId><![CDATA[6716015]]></publicationId>
##     <mdurl><![CDATA[http://ieeexplore.ieee.org/xpl/articleDetails.jsp?tp=&arnumber=6716015&contentType=Journals+%26+Magazines]]></mdurl>
##     <pdf><![CDATA[http://ieeexplore.ieee.org/stamp/stamp.jsp?arnumber=6716015]]></pdf>
##   </document>
##   <document>
##     <rank>114</rank>
##     <title><![CDATA[ATD: A Multiplatform for Semiautomatic 3-D Detection of Kidneys and Their Pathology in Real Time]]></title>
##     <authors><![CDATA[Skounakis, E.;  Banitsas, K.;  Badii, A.;  Tzoulakis, S.;  Maravelakis, E.;  Konstantaras, A.]]></authors>
##     <affiliations><![CDATA[Dept. of Electron. & Comput. Eng., Brunel Univ., Uxbridge, UK]]></affiliations>
##     <controlledterms>
##       <term><![CDATA[biomedical MRI]]></term>
##       <term><![CDATA[diseases]]></term>
##       <term><![CDATA[image recognition]]></term>
##       <term><![CDATA[kidney]]></term>
##       <term><![CDATA[medical disorders]]></term>
##       <term><![CDATA[medical image processing]]></term>
##       <term><![CDATA[tumours]]></term>
##     </controlledterms>
##     <thesaurusterms>
##       <term><![CDATA[Biomedical imaging]]></term>
##       <term><![CDATA[Histograms]]></term>
##       <term><![CDATA[Image segmentation]]></term>
##       <term><![CDATA[Kidney]]></term>
##       <term><![CDATA[Magnetic resonance imaging]]></term>
##       <term><![CDATA[Solid modeling]]></term>
##       <term><![CDATA[Tumors]]></term>
##     </thesaurusterms>
##     <pubtitle><![CDATA[Human-Machine Systems, IEEE Transactions on]]></pubtitle>
##     <punumber><![CDATA[6221037]]></punumber>
##     <pubtype><![CDATA[Journals & Magazines]]></pubtype>
##     <publisher><![CDATA[IEEE]]></publisher>
##     <volume><![CDATA[44]]></volume>
##     <issue><![CDATA[1]]></issue>
##     <py><![CDATA[2014]]></py>
##     <spage><![CDATA[146]]></spage>
##     <epage><![CDATA[153]]></epage>
##     <abstract><![CDATA[This research presents a novel multifunctional platform focusing on the clinical diagnosis of kidneys and their pathology (tumors, stones and cysts), using a &#x201C;templates&#x201D;-based technique. As a first step, specialist clinicians train the system by accurately annotating the kidneys and their abnormalities creating &#x201C;3-D golden standard models.&#x201D; Then, medical technicians experimentally adjust rules and parameters (stored as &#x201C;templates&#x201D;) for the integrated &#x201C;automatic recognition framework&#x201D; to achieve results which are closest to those of the clinicians. These parameters can later be used by nonexperts to achieve increased automation in the identification process. The system's functionality was tested on 20 MRI datasets (552 images), while the &#x201C;automatic 3-D models&#x201D; created were validated against the &#x201C;3-D golden standard models.&#x201D; Results are promising as they yield an average accuracy of 97.2% in successfully identifying kidneys and 96.1% of their abnormalities thus outperforming existing methods both in accuracy and in processing time needed.]]></abstract>
##     <issn><![CDATA[2168-2291]]></issn>
##     <htmlFlag><![CDATA[1]]></htmlFlag>
##     <arnumber><![CDATA[6675075]]></arnumber>
##     <doi><![CDATA[10.1109/THMS.2013.2290011]]></doi>
##     <publicationId><![CDATA[6675075]]></publicationId>
##     <mdurl><![CDATA[http://ieeexplore.ieee.org/xpl/articleDetails.jsp?tp=&arnumber=6675075&contentType=Journals+%26+Magazines]]></mdurl>
##     <pdf><![CDATA[http://ieeexplore.ieee.org/stamp/stamp.jsp?arnumber=6675075]]></pdf>
##   </document>
##   <document>
##     <rank>115</rank>
##     <title><![CDATA[Breakthroughs in Photonics 2013: Toward Feedback-Controlled Integrated Photonics]]></title>
##     <authors><![CDATA[Morichetti, F.;  Grillanda, S.;  Melloni, A.]]></authors>
##     <affiliations><![CDATA[Dipt. di Elettron., Inf. e Bioingegneria, Politec. di Milano, Milan, Italy]]></affiliations>
##     <controlledterms>
##       <term><![CDATA[integrated optics]]></term>
##       <term><![CDATA[optical feedback]]></term>
##     </controlledterms>
##     <thesaurusterms>
##       <term><![CDATA[Monitoring]]></term>
##       <term><![CDATA[Optical feedback]]></term>
##       <term><![CDATA[Optical interferometry]]></term>
##       <term><![CDATA[Optical resonators]]></term>
##       <term><![CDATA[Optical waveguides]]></term>
##       <term><![CDATA[Photonics]]></term>
##       <term><![CDATA[Silicon]]></term>
##     </thesaurusterms>
##     <pubtitle><![CDATA[Photonics Journal, IEEE]]></pubtitle>
##     <punumber><![CDATA[4563994]]></punumber>
##     <pubtype><![CDATA[Journals & Magazines]]></pubtype>
##     <publisher><![CDATA[IEEE]]></publisher>
##     <volume><![CDATA[6]]></volume>
##     <issue><![CDATA[2]]></issue>
##     <py><![CDATA[2014]]></py>
##     <spage><![CDATA[1]]></spage>
##     <epage><![CDATA[6]]></epage>
##     <abstract><![CDATA[We present an overview of the main achievements obtained in 2013 on the monitoring, stabilization, and feedback loop control of passive and active photonic integrated circuits. Key advances contributed to the evolution of photonic technologies from the current device level toward complex, adaptive, and reconfigurable integrated circuits.]]></abstract>
##     <issn><![CDATA[1943-0655]]></issn>
##     <htmlFlag><![CDATA[1]]></htmlFlag>
##     <arnumber><![CDATA[6758361]]></arnumber>
##     <doi><![CDATA[10.1109/JPHOT.2014.2310203]]></doi>
##     <publicationId><![CDATA[6758361]]></publicationId>
##     <mdurl><![CDATA[http://ieeexplore.ieee.org/xpl/articleDetails.jsp?tp=&arnumber=6758361&contentType=Journals+%26+Magazines]]></mdurl>
##     <pdf><![CDATA[http://ieeexplore.ieee.org/stamp/stamp.jsp?arnumber=6758361]]></pdf>
##   </document>
##   <document>
##     <rank>116</rank>
##     <title><![CDATA[fgCAPTCHA: Genetically Optimized Face Image CAPTCHA 5]]></title>
##     <authors><![CDATA[Powell, B.M.;  Goswami, G.;  Vatsa, M.;  Singh, R.;  Noore, A.]]></authors>
##     <affiliations><![CDATA[Lane Dept. of Comput. Sci. & Electr. Eng., West Virginia Univ., Morgantown, WV, USA]]></affiliations>
##     <controlledterms>
##       <term><![CDATA[face recognition]]></term>
##       <term><![CDATA[mobile computing]]></term>
##       <term><![CDATA[security of data]]></term>
##     </controlledterms>
##     <thesaurusterms>
##       <term><![CDATA[CAPTCHAs]]></term>
##       <term><![CDATA[Face detection]]></term>
##       <term><![CDATA[Face recognition]]></term>
##       <term><![CDATA[Mobile communication]]></term>
##       <term><![CDATA[Mobile handsets]]></term>
##       <term><![CDATA[Noise measurement]]></term>
##       <term><![CDATA[Security]]></term>
##     </thesaurusterms>
##     <pubtitle><![CDATA[Access, IEEE]]></pubtitle>
##     <punumber><![CDATA[6287639]]></punumber>
##     <pubtype><![CDATA[Journals & Magazines]]></pubtype>
##     <publisher><![CDATA[IEEE]]></publisher>
##     <volume><![CDATA[2]]></volume>
##     <py><![CDATA[2014]]></py>
##     <spage><![CDATA[473]]></spage>
##     <epage><![CDATA[484]]></epage>
##     <abstract><![CDATA[The increasing use of smartphones, tablets, and other mobile devices poses a significant challenge in providing effective online security. CAPTCHAs, tests for distinguishing human and computer users, have traditionally been popular; however, they face particular difficulties in a modern mobile environment because most of them rely on keyboard input and have language dependencies. This paper proposes a novel image-based CAPTCHA that combines the touch-based input methods favored by mobile devices with genetically optimized face detection tests to provide a solution that is simple for humans to solve, ready for worldwide use, and provides a high level of security by being resilient to automated computer attacks. In extensive testing involving over 2600 users and 40000 CAPTCHA tests, fgCAPTCHA demonstrates a very high human success rate while ensuring a 0% attack rate using three well-known face detection algorithms.]]></abstract>
##     <issn><![CDATA[2169-3536]]></issn>
##     <htmlFlag><![CDATA[1]]></htmlFlag>
##     <arnumber><![CDATA[6807630]]></arnumber>
##     <doi><![CDATA[10.1109/ACCESS.2014.2321001]]></doi>
##     <publicationId><![CDATA[6807630]]></publicationId>
##     <mdurl><![CDATA[http://ieeexplore.ieee.org/xpl/articleDetails.jsp?tp=&arnumber=6807630&contentType=Journals+%26+Magazines]]></mdurl>
##     <pdf><![CDATA[http://ieeexplore.ieee.org/stamp/stamp.jsp?arnumber=6807630]]></pdf>
##   </document>
##   <document>
##     <rank>117</rank>
##     <title><![CDATA[Task Versus Vehicle-Based Control Paradigms in Multiple Unmanned Vehicle Supervision by a Single Operator]]></title>
##     <authors><![CDATA[Cummings, M.L.;  Bertucelli, L.F.;  Macbeth, J.;  Surana, A.]]></authors>
##     <affiliations><![CDATA[Dept. of Mech. Eng., Duke Univ., Durham, NC, USA]]></affiliations>
##     <controlledterms>
##       <term><![CDATA[autonomous aerial vehicles]]></term>
##       <term><![CDATA[level control]]></term>
##     </controlledterms>
##     <thesaurusterms>
##       <term><![CDATA[Algorithm design and analysis]]></term>
##       <term><![CDATA[Automation]]></term>
##       <term><![CDATA[Cameras]]></term>
##       <term><![CDATA[Computer architecture]]></term>
##       <term><![CDATA[Hazards]]></term>
##       <term><![CDATA[Path planning]]></term>
##       <term><![CDATA[Vehicles]]></term>
##     </thesaurusterms>
##     <pubtitle><![CDATA[Human-Machine Systems, IEEE Transactions on]]></pubtitle>
##     <punumber><![CDATA[6221037]]></punumber>
##     <pubtype><![CDATA[Journals & Magazines]]></pubtype>
##     <publisher><![CDATA[IEEE]]></publisher>
##     <volume><![CDATA[44]]></volume>
##     <issue><![CDATA[3]]></issue>
##     <py><![CDATA[2014]]></py>
##     <spage><![CDATA[353]]></spage>
##     <epage><![CDATA[361]]></epage>
##     <abstract><![CDATA[There has recently been a significant amount of activity in developing supervisory control algorithms for multiple unmanned aerial vehicle operation by a single operator. While previous work has demonstrated the favorable impacts that arise in the introduction of increasingly sophisticated autonomy algorithms, little work has performed an explicit comparison of different types of multiple unmanned vehicle control architectures on operator performance and workload. This paper compares a vehicle-based paradigm (where a single operator individually assigns tasks to unmanned assets) to a task-based paradigm (where the operator generates a task list, which is then given to the group of vehicles that determine how to best divide the tasks among themselves.) The results demonstrate significant advantages in using a task-based paradigm for both overall performance and robustness to increased workload. This effort also demonstrated that while previous video gaming experience mattered for performance, the degree of experience that demonstrated benefit was minimal. Further work should focus on designing a flexible automated system that allows operators to focus on a primary goal, but also facilitate lower level control when needed without degradation in performance.]]></abstract>
##     <issn><![CDATA[2168-2291]]></issn>
##     <htmlFlag><![CDATA[1]]></htmlFlag>
##     <arnumber><![CDATA[6750700]]></arnumber>
##     <doi><![CDATA[10.1109/THMS.2014.2304962]]></doi>
##     <publicationId><![CDATA[6750700]]></publicationId>
##     <mdurl><![CDATA[http://ieeexplore.ieee.org/xpl/articleDetails.jsp?tp=&arnumber=6750700&contentType=Journals+%26+Magazines]]></mdurl>
##     <pdf><![CDATA[http://ieeexplore.ieee.org/stamp/stamp.jsp?arnumber=6750700]]></pdf>
##   </document>
##   <document>
##     <rank>118</rank>
##     <title><![CDATA[A Novel Local Pattern Descriptor&#x2014;Local Vector Pattern in High-Order Derivative Space for Face Recognition]]></title>
##     <authors><![CDATA[Kuo-Chin Fan;  Tsung-Yung Hung]]></authors>
##     <affiliations><![CDATA[Nat. Central Univ., Taoyuan, Taiwan]]></affiliations>
##     <controlledterms>
##       <term><![CDATA[face recognition]]></term>
##       <term><![CDATA[image representation]]></term>
##       <term><![CDATA[visual databases]]></term>
##     </controlledterms>
##     <thesaurusterms>
##       <term><![CDATA[Encoding]]></term>
##       <term><![CDATA[Face]]></term>
##       <term><![CDATA[Face recognition]]></term>
##       <term><![CDATA[Feature extraction]]></term>
##       <term><![CDATA[Image coding]]></term>
##       <term><![CDATA[Transforms]]></term>
##       <term><![CDATA[Vectors]]></term>
##     </thesaurusterms>
##     <pubtitle><![CDATA[Image Processing, IEEE Transactions on]]></pubtitle>
##     <punumber><![CDATA[83]]></punumber>
##     <pubtype><![CDATA[Journals & Magazines]]></pubtype>
##     <publisher><![CDATA[IEEE]]></publisher>
##     <volume><![CDATA[23]]></volume>
##     <issue><![CDATA[7]]></issue>
##     <py><![CDATA[2014]]></py>
##     <spage><![CDATA[2877]]></spage>
##     <epage><![CDATA[2891]]></epage>
##     <abstract><![CDATA[In this paper, a novel local pattern descriptor generated by the proposed local vector pattern (LVP) in high-order derivative space is presented for use in face recognition. Based on the vector of each pixel constructed by computing the values between the referenced pixel and the adjacent pixels with diverse distances from different directions, the vector representation of the referenced pixel is generated to provide the 1D structure of micropatterns. With the devise of pairwise direction of vector for each pixel, the LVP reduces the feature length via comparative space transform to encode various spatial surrounding relationships between the referenced pixel and its neighborhood pixels. Besides, the concatenation of LVPs is compacted to produce more distinctive features. To effectively extract more detailed discriminative information in a given subregion, the vector of LVP is refined by varying local derivative directions from the (n) th-order LVP in ((n-1)) th-order derivative space, which is a much more resilient structure of micropatterns than standard local pattern descriptors. The proposed LVP is compared with the existing local pattern descriptors including local binary pattern (LBP), local derivative pattern (LDP), and local tetra pattern (LTrP) to evaluate the performances from input grayscale face images. In addition, extensive experiments conducting on benchmark face image databases, FERET, CAS-PEAL, CMU-PIE, Extended Yale B, and LFW, demonstrate that the proposed LVP in high-order derivative space indeed performs much better than LBP, LDP, and LTrP in face recognition.]]></abstract>
##     <issn><![CDATA[1057-7149]]></issn>
##     <htmlFlag><![CDATA[1]]></htmlFlag>
##     <arnumber><![CDATA[6809981]]></arnumber>
##     <doi><![CDATA[10.1109/TIP.2014.2321495]]></doi>
##     <publicationId><![CDATA[6809981]]></publicationId>
##     <mdurl><![CDATA[http://ieeexplore.ieee.org/xpl/articleDetails.jsp?tp=&arnumber=6809981&contentType=Journals+%26+Magazines]]></mdurl>
##     <pdf><![CDATA[http://ieeexplore.ieee.org/stamp/stamp.jsp?arnumber=6809981]]></pdf>
##   </document>
##   <document>
##     <rank>119</rank>
##     <title><![CDATA[Capacity-Approaching Superposition Coding for Optical Fiber Links]]></title>
##     <authors><![CDATA[Estaran, J.;  Zibar, D.;  Monroy, I.T.]]></authors>
##     <affiliations><![CDATA[Dept. of Photonics Eng., Tech. Univ. of Denmark, Lyngby, Denmark]]></affiliations>
##     <controlledterms>
##       <term><![CDATA[Gaussian distribution]]></term>
##       <term><![CDATA[decoding]]></term>
##       <term><![CDATA[encoding]]></term>
##       <term><![CDATA[optical fibre communication]]></term>
##       <term><![CDATA[optical modulation]]></term>
##     </controlledterms>
##     <thesaurusterms>
##       <term><![CDATA[Constellation diagram]]></term>
##       <term><![CDATA[Encoding]]></term>
##       <term><![CDATA[Optical fiber communication]]></term>
##       <term><![CDATA[Quadrature amplitude modulation]]></term>
##       <term><![CDATA[Receivers]]></term>
##       <term><![CDATA[Signal to noise ratio]]></term>
##     </thesaurusterms>
##     <pubtitle><![CDATA[Lightwave Technology, Journal of]]></pubtitle>
##     <punumber><![CDATA[50]]></punumber>
##     <pubtype><![CDATA[Journals & Magazines]]></pubtype>
##     <publisher><![CDATA[IEEE]]></publisher>
##     <volume><![CDATA[32]]></volume>
##     <issue><![CDATA[17]]></issue>
##     <py><![CDATA[2014]]></py>
##     <spage><![CDATA[2960]]></spage>
##     <epage><![CDATA[2972]]></epage>
##     <abstract><![CDATA[We report on the first experimental demonstration of superposition coded modulation (SCM) for polarization-multiplexed coherent-detection optical fiber links. The proposed coded modulation scheme is combined with phase-shifted bit-to-symbol mapping (PSM) in order to achieve geometric and passive shaping of the signal's waveform. The output constellations in SCM-PSM exhibit nonbijective quasi-Gaussian statistical distributions that asymptotically reach the Shannon capacity limit, showing up to 0.7 dB sensitivity improvement for 256-ary SCM-PSM with respect to 256-ary quadrature amplitude modulation (QAM). The characteristic wave formation based on superposition of antipodal symbols and the lack of need for additional encoders for signal shaping, greatly reduces the transmitter and receiver processing complexity in comparison to conventional alternatives. Single-level coding strategy (SL-SCM) is employed in the framework of bit-interleaved coded modulation with iterative decoding (BICM-ID) for forward error correction. The fiber transmission system is characterized in terms of signal-to-noise ratio for back-to-back case and correlated with simulated results for ideal transmission over additive white Gaussian noise channel. Thereafter, successful demodulation and decoding after dispersion-unmanaged transmission over 240-km standard single mode fiber of dual-polarization 6-Gbaud 16-, 32- and 64-ary SCM-PSM is experimentally demonstrated.]]></abstract>
##     <issn><![CDATA[0733-8724]]></issn>
##     <htmlFlag><![CDATA[1]]></htmlFlag>
##     <arnumber><![CDATA[6843879]]></arnumber>
##     <doi><![CDATA[10.1109/JLT.2014.2333235]]></doi>
##     <publicationId><![CDATA[6843879]]></publicationId>
##     <mdurl><![CDATA[http://ieeexplore.ieee.org/xpl/articleDetails.jsp?tp=&arnumber=6843879&contentType=Journals+%26+Magazines]]></mdurl>
##     <pdf><![CDATA[http://ieeexplore.ieee.org/stamp/stamp.jsp?arnumber=6843879]]></pdf>
##   </document>
##   <document>
##     <rank>120</rank>
##     <title><![CDATA[Visual Tracking: An Experimental Survey]]></title>
##     <authors><![CDATA[Smeulders, A.W.M.;  Chu, D.M.;  Cucchiara, R.;  Calderara, S.;  Dehghan, A.;  Shah, M.]]></authors>
##     <affiliations><![CDATA[Inf. Inst., Univ. of Amsterdam, Amsterdam, Netherlands]]></affiliations>
##     <controlledterms>
##       <term><![CDATA[computer vision]]></term>
##       <term><![CDATA[image sensors]]></term>
##       <term><![CDATA[object tracking]]></term>
##       <term><![CDATA[statistical analysis]]></term>
##       <term><![CDATA[video signal processing]]></term>
##     </controlledterms>
##     <thesaurusterms>
##       <term><![CDATA[Educational institutions]]></term>
##       <term><![CDATA[Object tracking]]></term>
##       <term><![CDATA[Radar tracking]]></term>
##       <term><![CDATA[Robustness]]></term>
##       <term><![CDATA[Target tracking]]></term>
##       <term><![CDATA[Videos]]></term>
##     </thesaurusterms>
##     <pubtitle><![CDATA[Pattern Analysis and Machine Intelligence, IEEE Transactions on]]></pubtitle>
##     <punumber><![CDATA[34]]></punumber>
##     <pubtype><![CDATA[Journals & Magazines]]></pubtype>
##     <publisher><![CDATA[IEEE]]></publisher>
##     <volume><![CDATA[36]]></volume>
##     <issue><![CDATA[7]]></issue>
##     <py><![CDATA[2014]]></py>
##     <spage><![CDATA[1442]]></spage>
##     <epage><![CDATA[1468]]></epage>
##     <abstract><![CDATA[There is a large variety of trackers, which have been proposed in the literature during the last two decades with some mixed success. Object tracking in realistic scenarios is a difficult problem, therefore, it remains a most active area of research in computer vision. A good tracker should perform well in a large number of videos involving illumination changes, occlusion, clutter, camera motion, low contrast, specularities, and at least six more aspects. However, the performance of proposed trackers have been evaluated typically on less than ten videos, or on the special purpose datasets. In this paper, we aim to evaluate trackers systematically and experimentally on 315 video fragments covering above aspects. We selected a set of nineteen trackers to include a wide variety of algorithms often cited in literature, supplemented with trackers appearing in 2010 and 2011 for which the code was publicly available. We demonstrate that trackers can be evaluated objectively by survival curves, Kaplan Meier statistics, and Grubs testing. We find that in the evaluation practice the F-score is as effective as the object tracking accuracy (OTA) score. The analysis under a large variety of circumstances provides objective insight into the strengths and weaknesses of trackers.]]></abstract>
##     <issn><![CDATA[0162-8828]]></issn>
##     <htmlFlag><![CDATA[1]]></htmlFlag>
##     <arnumber><![CDATA[6671560]]></arnumber>
##     <doi><![CDATA[10.1109/TPAMI.2013.230]]></doi>
##     <publicationId><![CDATA[6671560]]></publicationId>
##     <mdurl><![CDATA[http://ieeexplore.ieee.org/xpl/articleDetails.jsp?tp=&arnumber=6671560&contentType=Journals+%26+Magazines]]></mdurl>
##     <pdf><![CDATA[http://ieeexplore.ieee.org/stamp/stamp.jsp?arnumber=6671560]]></pdf>
##   </document>
##   <document>
##     <rank>121</rank>
##     <title><![CDATA[Analysis of a Silicon Reconfigurable Feed-Forward Optical Delay Line]]></title>
##     <authors><![CDATA[Zhilong Chen;  Linjie Zhou;  Jianping Chen]]></authors>
##     <affiliations><![CDATA[Dept. of Electron. Eng., Shanghai Jiao Tong Univ., Shanghai, China]]></affiliations>
##     <controlledterms>
##       <term><![CDATA[Mach-Zehnder interferometers]]></term>
##       <term><![CDATA[elemental semiconductors]]></term>
##       <term><![CDATA[extinction coefficients]]></term>
##       <term><![CDATA[feedforward]]></term>
##       <term><![CDATA[optical attenuators]]></term>
##       <term><![CDATA[optical communication equipment]]></term>
##       <term><![CDATA[optical crosstalk]]></term>
##       <term><![CDATA[optical delay lines]]></term>
##       <term><![CDATA[optical design techniques]]></term>
##       <term><![CDATA[optical losses]]></term>
##       <term><![CDATA[optical switches]]></term>
##       <term><![CDATA[optical tuning]]></term>
##       <term><![CDATA[optical waveguides]]></term>
##       <term><![CDATA[random sequences]]></term>
##       <term><![CDATA[silicon]]></term>
##     </controlledterms>
##     <thesaurusterms>
##       <term><![CDATA[Optical attenuators]]></term>
##       <term><![CDATA[Optical buffering]]></term>
##       <term><![CDATA[Optical crosstalk]]></term>
##       <term><![CDATA[Optical interferometry]]></term>
##       <term><![CDATA[Optical resonators]]></term>
##       <term><![CDATA[Optical switches]]></term>
##       <term><![CDATA[Optical waveguides]]></term>
##     </thesaurusterms>
##     <pubtitle><![CDATA[Photonics Journal, IEEE]]></pubtitle>
##     <punumber><![CDATA[4563994]]></punumber>
##     <pubtype><![CDATA[Journals & Magazines]]></pubtype>
##     <publisher><![CDATA[IEEE]]></publisher>
##     <volume><![CDATA[6]]></volume>
##     <issue><![CDATA[1]]></issue>
##     <py><![CDATA[2014]]></py>
##     <spage><![CDATA[1]]></spage>
##     <epage><![CDATA[11]]></epage>
##     <abstract><![CDATA[We analyze a silicon reconfigurable feed-forward optical delay line (ODL) composed of cascaded Mach-Zehnder interferometer (MZI) switches and waveguide delay pairs. Optical delay is step-tuned by switching among optical routes with an incremental length difference. The crosstalk caused by limited extinction ratios (ERs) of MZI switches and the waveguide loss restricts the ODL buffering capacity. A method to suppress the crosstalk and hence expand the buffering capacity is proposed by inserting variable optical attenuators (VOAs) between successive MZI switches. A design example shows that a seven-stage ODL without VOAs can delay 8-98 bits with a tuning step of 1 bit for a 40 Gbit/s non-return-to-zero (NRZ) 2<sup>13</sup> - 1 pseudo-random bit sequence (PRBS) optical signal. When VOAs with a 10 dB attenuation ratio are included, the signal-to-crosstalk ratio (SCR) of the output signal is increased by 17 dB on average, making the ODL capable of delaying 8-135 bits at its maximum.]]></abstract>
##     <issn><![CDATA[1943-0655]]></issn>
##     <htmlFlag><![CDATA[1]]></htmlFlag>
##     <arnumber><![CDATA[6675010]]></arnumber>
##     <doi><![CDATA[10.1109/JPHOT.2013.2292697]]></doi>
##     <publicationId><![CDATA[6675010]]></publicationId>
##     <mdurl><![CDATA[http://ieeexplore.ieee.org/xpl/articleDetails.jsp?tp=&arnumber=6675010&contentType=Journals+%26+Magazines]]></mdurl>
##     <pdf><![CDATA[http://ieeexplore.ieee.org/stamp/stamp.jsp?arnumber=6675010]]></pdf>
##   </document>
##   <document>
##     <rank>122</rank>
##     <title><![CDATA[Nonuniformity Characterization of CdTe Solar Cells Using LBIC]]></title>
##     <authors><![CDATA[Geisthardt, R.M.;  Sites, J.R.]]></authors>
##     <affiliations><![CDATA[Dept. of Phys., Colorado State Univ., Fort Collins, CO, USA]]></affiliations>
##     <controlledterms>
##       <term><![CDATA[II-VI semiconductors]]></term>
##       <term><![CDATA[OBIC]]></term>
##       <term><![CDATA[cadmium compounds]]></term>
##       <term><![CDATA[electric current measurement]]></term>
##       <term><![CDATA[energy gap]]></term>
##       <term><![CDATA[semiconductor device measurement]]></term>
##       <term><![CDATA[solar absorber-convertors]]></term>
##       <term><![CDATA[solar cells]]></term>
##     </controlledterms>
##     <thesaurusterms>
##       <term><![CDATA[Current measurement]]></term>
##       <term><![CDATA[Measurement by laser beam]]></term>
##       <term><![CDATA[Photovoltaic cells]]></term>
##       <term><![CDATA[Photovoltaic systems]]></term>
##       <term><![CDATA[Time measurement]]></term>
##       <term><![CDATA[Wavelength measurement]]></term>
##     </thesaurusterms>
##     <pubtitle><![CDATA[Photovoltaics, IEEE Journal of]]></pubtitle>
##     <punumber><![CDATA[5503869]]></punumber>
##     <pubtype><![CDATA[Journals & Magazines]]></pubtype>
##     <publisher><![CDATA[IEEE]]></publisher>
##     <volume><![CDATA[4]]></volume>
##     <issue><![CDATA[4]]></issue>
##     <py><![CDATA[2014]]></py>
##     <spage><![CDATA[1114]]></spage>
##     <epage><![CDATA[1118]]></epage>
##     <abstract><![CDATA[Light-beam-induced current measurements have been used for characterization of nonuniformities in cadmium telluride (CdTe) solar cells. Spectral dependence, voltage-bias dependence, and resolution dependence are used for detailed characterization of nonuniformities in junction quality, window thickness, and absorber band gap. These tools were applied to CdTe cells and used to identify thin regions in the CdS layer, regions of modified band gap, and weak diode regions. In addition, an improved procedure has allowed for shorter measurement times without discernable loss of accuracy.]]></abstract>
##     <issn><![CDATA[2156-3381]]></issn>
##     <htmlFlag><![CDATA[1]]></htmlFlag>
##     <arnumber><![CDATA[6797857]]></arnumber>
##     <doi><![CDATA[10.1109/JPHOTOV.2014.2314575]]></doi>
##     <publicationId><![CDATA[6797857]]></publicationId>
##     <mdurl><![CDATA[http://ieeexplore.ieee.org/xpl/articleDetails.jsp?tp=&arnumber=6797857&contentType=Journals+%26+Magazines]]></mdurl>
##     <pdf><![CDATA[http://ieeexplore.ieee.org/stamp/stamp.jsp?arnumber=6797857]]></pdf>
##   </document>
##   <document>
##     <rank>123</rank>
##     <title><![CDATA[Generation of Terahertz Vortices Using Metasurface With Circular Slits]]></title>
##     <authors><![CDATA[Hailong Zhou;  Jianji Dong;  Siqi Yan;  Yifeng Zhou;  Xinliang Zhang]]></authors>
##     <affiliations><![CDATA[Wuhan Nat. Lab. for Optoelectron., Huazhong Univ. of Sci. & Technol., Wuhan, China]]></affiliations>
##     <controlledterms>
##       <term><![CDATA[angular momentum]]></term>
##       <term><![CDATA[laser beams]]></term>
##       <term><![CDATA[micro-optics]]></term>
##       <term><![CDATA[microwave photonics]]></term>
##       <term><![CDATA[optical design techniques]]></term>
##       <term><![CDATA[optical metamaterials]]></term>
##       <term><![CDATA[optical noise]]></term>
##       <term><![CDATA[optical vortices]]></term>
##     </controlledterms>
##     <thesaurusterms>
##       <term><![CDATA[Bandwidth]]></term>
##       <term><![CDATA[Electric fields]]></term>
##       <term><![CDATA[Films]]></term>
##       <term><![CDATA[Graphical models]]></term>
##       <term><![CDATA[Metals]]></term>
##       <term><![CDATA[Noise]]></term>
##       <term><![CDATA[Vectors]]></term>
##     </thesaurusterms>
##     <pubtitle><![CDATA[Photonics Journal, IEEE]]></pubtitle>
##     <punumber><![CDATA[4563994]]></punumber>
##     <pubtype><![CDATA[Journals & Magazines]]></pubtype>
##     <publisher><![CDATA[IEEE]]></publisher>
##     <volume><![CDATA[6]]></volume>
##     <issue><![CDATA[6]]></issue>
##     <py><![CDATA[2014]]></py>
##     <spage><![CDATA[1]]></spage>
##     <epage><![CDATA[7]]></epage>
##     <abstract><![CDATA[We propose a metal device containing circular slits to generate a terahertz (THz) orbital angular momentum beam with numerical simulations. The estimation of the polarization extinction ratio is above 20 dB over the bandwidth ranging from 0.3 to 3 THz, and a mode purity of TC = 1 or -1 is close to 1 over a wide bandwidth range, except for the area near the deteriorated valley. We analyze the OAM spectrum and find that the main noise comes from an OAM mode of TC = -3 or 3. When multiple concentric circular slits are employed, a larger transmittance is obtained without the sacrifice of mode purity. The design of such a device is simple with a size of micrometer order, revealing an option to generate a broadband THz vortex beam.]]></abstract>
##     <issn><![CDATA[1943-0655]]></issn>
##     <htmlFlag><![CDATA[1]]></htmlFlag>
##     <arnumber><![CDATA[6926759]]></arnumber>
##     <doi><![CDATA[10.1109/JPHOT.2014.2363424]]></doi>
##     <publicationId><![CDATA[6926759]]></publicationId>
##     <mdurl><![CDATA[http://ieeexplore.ieee.org/xpl/articleDetails.jsp?tp=&arnumber=6926759&contentType=Journals+%26+Magazines]]></mdurl>
##     <pdf><![CDATA[http://ieeexplore.ieee.org/stamp/stamp.jsp?arnumber=6926759]]></pdf>
##   </document>
##   <document>
##     <rank>124</rank>
##     <title><![CDATA[Influence of Minority Carrier Gas Donors on Low-Frequency Noise in Silicon Nanowires]]></title>
##     <authors><![CDATA[Fobelets, K.;  Meghani, M.;  Li, C.]]></authors>
##     <affiliations><![CDATA[Dept. of Electr. & Electron. Eng., Imperial Coll. London, London, UK]]></affiliations>
##     <controlledterms>
##       <term><![CDATA[1/f noise]]></term>
##       <term><![CDATA[adsorption]]></term>
##       <term><![CDATA[ammonia]]></term>
##       <term><![CDATA[elemental semiconductors]]></term>
##       <term><![CDATA[minority carriers]]></term>
##       <term><![CDATA[nanowires]]></term>
##       <term><![CDATA[nitrogen compounds]]></term>
##       <term><![CDATA[pH]]></term>
##       <term><![CDATA[silicon]]></term>
##       <term><![CDATA[silicon compounds]]></term>
##       <term><![CDATA[surface charging]]></term>
##       <term><![CDATA[surface conductivity]]></term>
##       <term><![CDATA[vacancies (crystal)]]></term>
##     </controlledterms>
##     <thesaurusterms>
##       <term><![CDATA[1f noise]]></term>
##       <term><![CDATA[Gases]]></term>
##       <term><![CDATA[Low-frequency noise]]></term>
##       <term><![CDATA[Nanowires]]></term>
##       <term><![CDATA[Sensors]]></term>
##       <term><![CDATA[Silicon]]></term>
##     </thesaurusterms>
##     <pubtitle><![CDATA[Nanotechnology, IEEE Transactions on]]></pubtitle>
##     <punumber><![CDATA[7729]]></punumber>
##     <pubtype><![CDATA[Journals & Magazines]]></pubtype>
##     <publisher><![CDATA[IEEE]]></publisher>
##     <volume><![CDATA[13]]></volume>
##     <issue><![CDATA[6]]></issue>
##     <py><![CDATA[2014]]></py>
##     <spage><![CDATA[1176]]></spage>
##     <epage><![CDATA[1180]]></epage>
##     <abstract><![CDATA[The interaction of gases such as NH<sub>3</sub> and NO<sub>2</sub> with the surface of core/shell Si/SiO<sub>2</sub> nanowires has been shown to influence their electrical conductivity because NH<sub>3</sub> and NO2 are electron and hole donors, respectively. Using arrays of n- and p-type Si nanowires, we demonstrate that their influence on the low-frequency noise characteristics of the nanowires is largest when the donors are minority carriers. The impact of NO<sub>2</sub> and NH<sub>3</sub> on 1/f noise of p- and n-type nanowires, respectively, is limited. However, 1/f noise increases in n-Si nanowires under influence of NO<sub>2</sub> while it decreases in p-Si nanowires for NH<sub>3</sub>. This effect is attributed to oxygen vacancies in the SiO<sub>2</sub> and the presence or absence of holes, h<sup>+</sup> in the humid gas environment. In addition, gas molecule adsorption in a humid atmosphere influences the pH and thus the surface charge density on the SiO<sub>2</sub> shell, causing changes in the low-frequency noise level via electrostatic interactions.]]></abstract>
##     <issn><![CDATA[1536-125X]]></issn>
##     <htmlFlag><![CDATA[1]]></htmlFlag>
##     <arnumber><![CDATA[6880397]]></arnumber>
##     <doi><![CDATA[10.1109/TNANO.2014.2349738]]></doi>
##     <publicationId><![CDATA[6880397]]></publicationId>
##     <mdurl><![CDATA[http://ieeexplore.ieee.org/xpl/articleDetails.jsp?tp=&arnumber=6880397&contentType=Journals+%26+Magazines]]></mdurl>
##     <pdf><![CDATA[http://ieeexplore.ieee.org/stamp/stamp.jsp?arnumber=6880397]]></pdf>
##   </document>
##   <document>
##     <rank>125</rank>
##     <title><![CDATA[Decomposition of the Kennaugh Matrix Based on a New Norm]]></title>
##     <authors><![CDATA[Biao You;  Jian Yang;  Junjun Yin;  Bin Xu]]></authors>
##     <affiliations><![CDATA[Dept. of Electron. Eng., Tsinghua Univ., Beijing, China]]></affiliations>
##     <controlledterms>
##       <term><![CDATA[remote sensing by radar]]></term>
##       <term><![CDATA[synthetic aperture radar]]></term>
##     </controlledterms>
##     <thesaurusterms>
##       <term><![CDATA[Feature extraction]]></term>
##       <term><![CDATA[Matrix decomposition]]></term>
##       <term><![CDATA[Radar imaging]]></term>
##       <term><![CDATA[Radar polarimetry]]></term>
##       <term><![CDATA[Scattering]]></term>
##       <term><![CDATA[Synchronous digital hierarchy]]></term>
##     </thesaurusterms>
##     <pubtitle><![CDATA[Geoscience and Remote Sensing Letters, IEEE]]></pubtitle>
##     <punumber><![CDATA[8859]]></punumber>
##     <pubtype><![CDATA[Journals & Magazines]]></pubtype>
##     <publisher><![CDATA[IEEE]]></publisher>
##     <volume><![CDATA[11]]></volume>
##     <issue><![CDATA[5]]></issue>
##     <py><![CDATA[2014]]></py>
##     <spage><![CDATA[1000]]></spage>
##     <epage><![CDATA[1004]]></epage>
##     <abstract><![CDATA[In this letter, a new method for Kennaugh matrix decomposition is proposed, and a new norm for the Kennaugh matrix is defined. The Kennaugh matrix is decomposed into two parts: The first is a coherent target matrix, and the second is a residual matrix with minimum norm. The properties of the extracted coherent target are discussed, and an application of the extracted coherent target is implemented. In this application, an incoherent image is converted into a coherent image. Single-look sphere-diplane-helix decomposition is then performed. An experiment on Airborne SAR polarimetric data over San Francisco has been carried out, thus demonstrating the effectiveness of the application.]]></abstract>
##     <issn><![CDATA[1545-598X]]></issn>
##     <htmlFlag><![CDATA[1]]></htmlFlag>
##     <arnumber><![CDATA[6646223]]></arnumber>
##     <doi><![CDATA[10.1109/LGRS.2013.2284336]]></doi>
##     <publicationId><![CDATA[6646223]]></publicationId>
##     <mdurl><![CDATA[http://ieeexplore.ieee.org/xpl/articleDetails.jsp?tp=&arnumber=6646223&contentType=Journals+%26+Magazines]]></mdurl>
##     <pdf><![CDATA[http://ieeexplore.ieee.org/stamp/stamp.jsp?arnumber=6646223]]></pdf>
##   </document>
##   <document>
##     <rank>126</rank>
##     <title><![CDATA[Perceptually Uniform Motion Space]]></title>
##     <authors><![CDATA[Birkeland, A.;  Turkay, C.;  Viola, I.]]></authors>
##     <affiliations><![CDATA[Dept. of Inf., Univ. of Bergen, Bergen, Norway]]></affiliations>
##     <controlledterms>
##       <term><![CDATA[computational fluid dynamics]]></term>
##       <term><![CDATA[computer animation]]></term>
##       <term><![CDATA[flow visualisation]]></term>
##     </controlledterms>
##     <thesaurusterms>
##       <term><![CDATA[Brain modeling]]></term>
##       <term><![CDATA[Data visualization]]></term>
##       <term><![CDATA[Image color analysis]]></term>
##       <term><![CDATA[Motion detection]]></term>
##       <term><![CDATA[Particle measurements]]></term>
##     </thesaurusterms>
##     <pubtitle><![CDATA[Visualization and Computer Graphics, IEEE Transactions on]]></pubtitle>
##     <punumber><![CDATA[2945]]></punumber>
##     <pubtype><![CDATA[Journals & Magazines]]></pubtype>
##     <publisher><![CDATA[IEEE]]></publisher>
##     <volume><![CDATA[20]]></volume>
##     <issue><![CDATA[11]]></issue>
##     <py><![CDATA[2014]]></py>
##     <spage><![CDATA[1542]]></spage>
##     <epage><![CDATA[1554]]></epage>
##     <abstract><![CDATA[Flow data is often visualized by animated particles inserted into a flow field. The velocity of a particle on the screen is typically linearly scaled by the velocities in the data. However, the perception of velocity magnitude in animated particles is not necessarily linear. We present a study on how different parameters affect relative motion perception. We have investigated the impact of four parameters. The parameters consist of speed multiplier, direction, contrast type and the global velocity scale. In addition, we investigated if multiple motion cues, and point distribution, affect the speed estimation. Several studies were executed to investigate the impact of each parameter. In the initial results, we noticed trends in scale and multiplier. Using the trends for the significant parameters, we designed a compensation model, which adjusts the particle speed to compensate for the effect of the parameters. We then performed a second study to investigate the performance of the compensation model. From the second study we detected a constant estimation error, which we adjusted for in the last study. In addition, we connect our work to established theories in psychophysics by comparing our model to a model based on Stevens' Power Law.]]></abstract>
##     <issn><![CDATA[1077-2626]]></issn>
##     <htmlFlag><![CDATA[1]]></htmlFlag>
##     <arnumber><![CDATA[6811168]]></arnumber>
##     <doi><![CDATA[10.1109/TVCG.2014.2322363]]></doi>
##     <publicationId><![CDATA[6811168]]></publicationId>
##     <mdurl><![CDATA[http://ieeexplore.ieee.org/xpl/articleDetails.jsp?tp=&arnumber=6811168&contentType=Journals+%26+Magazines]]></mdurl>
##     <pdf><![CDATA[http://ieeexplore.ieee.org/stamp/stamp.jsp?arnumber=6811168]]></pdf>
##   </document>
##   <document>
##     <rank>127</rank>
##     <title><![CDATA[Ultra-Compact Silicon Photonic 512 &#x00D7; 512 25 GHz Arrayed Waveguide Grating Router]]></title>
##     <authors><![CDATA[Cheung, S.;  Tiehui Su;  Okamoto, K.;  Yoo, S.J.B.]]></authors>
##     <affiliations><![CDATA[Dept. of Electr. & Comput. Eng., Univ. of California, Davis, Davis, CA, USA]]></affiliations>
##     <controlledterms>
##       <term><![CDATA[arrayed waveguide gratings]]></term>
##       <term><![CDATA[integrated optics]]></term>
##       <term><![CDATA[optical communication equipment]]></term>
##       <term><![CDATA[optical crosstalk]]></term>
##       <term><![CDATA[optical design techniques]]></term>
##       <term><![CDATA[optical fabrication]]></term>
##       <term><![CDATA[silicon-on-insulator]]></term>
##       <term><![CDATA[wavelength division multiplexing]]></term>
##     </controlledterms>
##     <thesaurusterms>
##       <term><![CDATA[Arrayed waveguide gratings]]></term>
##       <term><![CDATA[Indexes]]></term>
##       <term><![CDATA[Optical device fabrication]]></term>
##       <term><![CDATA[Silicon]]></term>
##       <term><![CDATA[Slabs]]></term>
##     </thesaurusterms>
##     <pubtitle><![CDATA[Selected Topics in Quantum Electronics, IEEE Journal of]]></pubtitle>
##     <punumber><![CDATA[2944]]></punumber>
##     <pubtype><![CDATA[Journals & Magazines]]></pubtype>
##     <publisher><![CDATA[IEEE]]></publisher>
##     <volume><![CDATA[20]]></volume>
##     <issue><![CDATA[4]]></issue>
##     <py><![CDATA[2014]]></py>
##     <spage><![CDATA[310]]></spage>
##     <epage><![CDATA[316]]></epage>
##     <abstract><![CDATA[This paper discusses design, fabrication, and characterization of a 512 &#x00D7; 512 arrayed waveguide grating router (AWGR) with a channel spacing of 25 GHz. The dimensions of the AWGR is 16 mm &#x00D7; 11 mm and is fabricated on a 250 nm silicon-on-insulator platform. The measured channel crosstalk is approximately -4 dB without any compensation for the phase errors in the arrayed waveguides. The AWGR spectrum in the arrayed waveguide grating arms were characterized by using an optical vector network analyzer. Fabrication details of obtaining low loss silicon ridge waveguides are also discussed.]]></abstract>
##     <issn><![CDATA[1077-260X]]></issn>
##     <htmlFlag><![CDATA[1]]></htmlFlag>
##     <arnumber><![CDATA[6691912]]></arnumber>
##     <doi><![CDATA[10.1109/JSTQE.2013.2295879]]></doi>
##     <publicationId><![CDATA[6691912]]></publicationId>
##     <mdurl><![CDATA[http://ieeexplore.ieee.org/xpl/articleDetails.jsp?tp=&arnumber=6691912&contentType=Journals+%26+Magazines]]></mdurl>
##     <pdf><![CDATA[http://ieeexplore.ieee.org/stamp/stamp.jsp?arnumber=6691912]]></pdf>
##   </document>
##   <document>
##     <rank>128</rank>
##     <title><![CDATA[Electrical Hole Transport Properties of an Ambipolar Organic Compound With Zn-Atoms on a Crystalline Silicon Heterostructure]]></title>
##     <authors><![CDATA[Landi, G.;  Fahrner, W.R.;  Concilio, S.;  Sessa, L.;  Neitzert, H.C.]]></authors>
##     <affiliations><![CDATA[Fac. of Math. & Comput. Sci., Fernuniv. Hagen, Hagen, Germany]]></affiliations>
##     <controlledterms>
##       <term><![CDATA[capacitance]]></term>
##       <term><![CDATA[carrier mobility]]></term>
##       <term><![CDATA[elemental semiconductors]]></term>
##       <term><![CDATA[organic semiconductors]]></term>
##       <term><![CDATA[permittivity]]></term>
##       <term><![CDATA[semiconductor heterojunctions]]></term>
##       <term><![CDATA[semiconductor thin films]]></term>
##       <term><![CDATA[silicon]]></term>
##       <term><![CDATA[zinc]]></term>
##     </controlledterms>
##     <thesaurusterms>
##       <term><![CDATA[Capacitance-voltage characteristics]]></term>
##       <term><![CDATA[Charge carrier mobility]]></term>
##       <term><![CDATA[Crystalline materials]]></term>
##       <term><![CDATA[Dielectric constant]]></term>
##       <term><![CDATA[Electron traps]]></term>
##       <term><![CDATA[Silicon]]></term>
##     </thesaurusterms>
##     <pubtitle><![CDATA[Electron Devices Society, IEEE Journal of the]]></pubtitle>
##     <punumber><![CDATA[6245494]]></punumber>
##     <pubtype><![CDATA[Journals & Magazines]]></pubtype>
##     <publisher><![CDATA[IEEE]]></publisher>
##     <volume><![CDATA[2]]></volume>
##     <issue><![CDATA[6]]></issue>
##     <py><![CDATA[2014]]></py>
##     <spage><![CDATA[179]]></spage>
##     <epage><![CDATA[181]]></epage>
##     <abstract><![CDATA[In this paper, we investigate the electrical hole transport properties of an organic/inorganic heterostructure consisting of a thin organic film, that combines hole and electron conducting molecules around a bridging Zn-atom, deposited on top of an n-type crystalline silicon substrate. Current-voltage characteristics and capacitance voltage measurements have been used for the determination of the organic layer dielectric and hole conduction parameters.]]></abstract>
##     <issn><![CDATA[2168-6734]]></issn>
##     <htmlFlag><![CDATA[1]]></htmlFlag>
##     <arnumber><![CDATA[6874478]]></arnumber>
##     <doi><![CDATA[10.1109/JEDS.2014.2346584]]></doi>
##     <publicationId><![CDATA[6874478]]></publicationId>
##     <mdurl><![CDATA[http://ieeexplore.ieee.org/xpl/articleDetails.jsp?tp=&arnumber=6874478&contentType=Journals+%26+Magazines]]></mdurl>
##     <pdf><![CDATA[http://ieeexplore.ieee.org/stamp/stamp.jsp?arnumber=6874478]]></pdf>
##   </document>
##   <document>
##     <rank>129</rank>
##     <title><![CDATA[Non-Destructive Three-Dimensional Optical Imaging of a Fiber Bragg Grating]]></title>
##     <authors><![CDATA[Xiao Ming Goh;  Shan Shan Kou;  Kouskousis, B.P.;  Dragomir, N.M.;  Collins, S.F.;  Baxter, G.W.;  Roberts, A.]]></authors>
##     <affiliations><![CDATA[Sch. of Phys., Univ. of Melbourne, Melbourne, VIC, Australia]]></affiliations>
##     <controlledterms>
##       <term><![CDATA[Bragg gratings]]></term>
##       <term><![CDATA[image reconstruction]]></term>
##       <term><![CDATA[optical tomography]]></term>
##       <term><![CDATA[refractive index]]></term>
##     </controlledterms>
##     <thesaurusterms>
##       <term><![CDATA[Fiber gratings]]></term>
##       <term><![CDATA[Gratings]]></term>
##       <term><![CDATA[Image reconstruction]]></term>
##       <term><![CDATA[Optical fiber devices]]></term>
##       <term><![CDATA[Optical fibers]]></term>
##       <term><![CDATA[Three-dimensional displays]]></term>
##     </thesaurusterms>
##     <pubtitle><![CDATA[Photonics Journal, IEEE]]></pubtitle>
##     <punumber><![CDATA[4563994]]></punumber>
##     <pubtype><![CDATA[Journals & Magazines]]></pubtype>
##     <publisher><![CDATA[IEEE]]></publisher>
##     <volume><![CDATA[6]]></volume>
##     <issue><![CDATA[5]]></issue>
##     <py><![CDATA[2014]]></py>
##     <spage><![CDATA[1]]></spage>
##     <epage><![CDATA[7]]></epage>
##     <abstract><![CDATA[Here, we demonstrate a non-destructive quantitative phase imaging method, in conjunction with tomographic reconstruction techniques, to reconstruct the three-dimensional refractive index distribution of a fiber Bragg grating. In addition to being able to extract the fundamental period of the grating, we are able to reconstruct the refractive index profile of the fiber core and index variations associated with the grating.]]></abstract>
##     <issn><![CDATA[1943-0655]]></issn>
##     <htmlFlag><![CDATA[1]]></htmlFlag>
##     <arnumber><![CDATA[6910210]]></arnumber>
##     <doi><![CDATA[10.1109/JPHOT.2014.2360282]]></doi>
##     <publicationId><![CDATA[6910210]]></publicationId>
##     <mdurl><![CDATA[http://ieeexplore.ieee.org/xpl/articleDetails.jsp?tp=&arnumber=6910210&contentType=Journals+%26+Magazines]]></mdurl>
##     <pdf><![CDATA[http://ieeexplore.ieee.org/stamp/stamp.jsp?arnumber=6910210]]></pdf>
##   </document>
##   <document>
##     <rank>130</rank>
##     <title><![CDATA[An Analysis of Failure-Related Energy Waste in a Large-Scale Cloud Environment]]></title>
##     <authors><![CDATA[Garraghan, P.;  Moreno, I.S.;  Townend, P.;  Jie Xu]]></authors>
##     <affiliations><![CDATA[Sch. of Comput., Univ. of Leeds, Leeds, UK]]></affiliations>
##     <controlledterms>
##       <term><![CDATA[cloud computing]]></term>
##       <term><![CDATA[energy conservation]]></term>
##       <term><![CDATA[environmental economics]]></term>
##     </controlledterms>
##     <thesaurusterms>
##       <term><![CDATA[Cloud computing]]></term>
##       <term><![CDATA[Computer crashes]]></term>
##       <term><![CDATA[Energy consumption]]></term>
##       <term><![CDATA[Failure analysis]]></term>
##       <term><![CDATA[Hardware]]></term>
##       <term><![CDATA[Waste management]]></term>
##     </thesaurusterms>
##     <pubtitle><![CDATA[Emerging Topics in Computing, IEEE Transactions on]]></pubtitle>
##     <punumber><![CDATA[6245516]]></punumber>
##     <pubtype><![CDATA[Journals & Magazines]]></pubtype>
##     <publisher><![CDATA[IEEE]]></publisher>
##     <volume><![CDATA[2]]></volume>
##     <issue><![CDATA[2]]></issue>
##     <py><![CDATA[2014]]></py>
##     <spage><![CDATA[166]]></spage>
##     <epage><![CDATA[180]]></epage>
##     <abstract><![CDATA[Cloud computing providers are under great pressure to reduce operational costs through improved energy utilization while provisioning dependable service to customers; it is therefore extremely important to understand and quantify the explicit impact of failures within a system in terms of energy costs. This paper presents the first comprehensive analysis of the impact of failures on energy consumption in a real-world large-scale cloud system (comprising over 12 500 servers), including the study of failure and energy trends of the spatial and temporal environmental characteristics. Our results show that 88% of task failure events occur in lower priority tasks producing 13% of total energy waste, and 1% of failure events occur in higher priority tasks due to server failures producing 8% of total energy waste. These results highlight an unintuitive but significant impact on energy consumption due to failures, providing a strong foundation for research into dependable energy-aware cloud computing.]]></abstract>
##     <issn><![CDATA[2168-6750]]></issn>
##     <htmlFlag><![CDATA[1]]></htmlFlag>
##     <arnumber><![CDATA[6731524]]></arnumber>
##     <doi><![CDATA[10.1109/TETC.2014.2304500]]></doi>
##     <publicationId><![CDATA[6731524]]></publicationId>
##     <mdurl><![CDATA[http://ieeexplore.ieee.org/xpl/articleDetails.jsp?tp=&arnumber=6731524&contentType=Journals+%26+Magazines]]></mdurl>
##     <pdf><![CDATA[http://ieeexplore.ieee.org/stamp/stamp.jsp?arnumber=6731524]]></pdf>
##   </document>
##   <document>
##     <rank>131</rank>
##     <title><![CDATA[Adaptive Subpixel Mapping Based on a Multiagent System for Remote-Sensing Imagery]]></title>
##     <authors><![CDATA[Xiong Xu;  Yanfei Zhong;  Liangpei Zhang]]></authors>
##     <affiliations><![CDATA[State Key Lab. of Inf. Eng. in Surveying, Mapping & Remote Sensing, Wuhan Univ., Wuhan, China]]></affiliations>
##     <controlledterms>
##       <term><![CDATA[geophysical image processing]]></term>
##       <term><![CDATA[image classification]]></term>
##       <term><![CDATA[multi-agent systems]]></term>
##       <term><![CDATA[neural nets]]></term>
##       <term><![CDATA[remote sensing]]></term>
##     </controlledterms>
##     <thesaurusterms>
##       <term><![CDATA[Algorithm design and analysis]]></term>
##       <term><![CDATA[Feature extraction]]></term>
##       <term><![CDATA[Image reconstruction]]></term>
##       <term><![CDATA[Multi-agent systems]]></term>
##       <term><![CDATA[Optimization]]></term>
##       <term><![CDATA[Remote sensing]]></term>
##     </thesaurusterms>
##     <pubtitle><![CDATA[Geoscience and Remote Sensing, IEEE Transactions on]]></pubtitle>
##     <punumber><![CDATA[36]]></punumber>
##     <pubtype><![CDATA[Journals & Magazines]]></pubtype>
##     <publisher><![CDATA[IEEE]]></publisher>
##     <volume><![CDATA[52]]></volume>
##     <issue><![CDATA[2]]></issue>
##     <py><![CDATA[2014]]></py>
##     <spage><![CDATA[787]]></spage>
##     <epage><![CDATA[804]]></epage>
##     <abstract><![CDATA[The existence of mixed pixels is a major problem in remote-sensing image classification. Although the soft classification and spectral unmixing techniques can obtain an abundance of different classes in a pixel to solve the mixed pixel problem, the subpixel spatial attribution of the pixel will still be unknown. The subpixel mapping technique can effectively solve this problem by providing a fine-resolution map of class labels from coarser spectrally unmixed fraction images. However, most traditional subpixel mapping algorithms treat all mixed pixels as an identical type, either boundary-mixed pixel or linear subpixel, leading to incomplete and inaccurate results. To improve the subpixel mapping accuracy, this paper proposes an adaptive subpixel mapping framework based on a multiagent system for remote-sensing imagery. In the proposed multiagent subpixel mapping framework, three kinds of agents, namely, feature detection agents, subpixel mapping agents and decision agents, are designed to solve the subpixel mapping problem. Experiments with artificial images and synthetic remote-sensing images were performed to evaluate the performance of the proposed subpixel mapping algorithm in comparison with the hard classification method and other subpixel mapping algorithms: subpixel mapping based on a back-propagation neural network and the spatial attraction model. The experimental results indicate that the proposed algorithm outperforms the other two subpixel mapping algorithms in reconstructing the different structures in mixed pixels.]]></abstract>
##     <issn><![CDATA[0196-2892]]></issn>
##     <htmlFlag><![CDATA[1]]></htmlFlag>
##     <arnumber><![CDATA[6479297]]></arnumber>
##     <doi><![CDATA[10.1109/TGRS.2013.2244095]]></doi>
##     <publicationId><![CDATA[6479297]]></publicationId>
##     <mdurl><![CDATA[http://ieeexplore.ieee.org/xpl/articleDetails.jsp?tp=&arnumber=6479297&contentType=Journals+%26+Magazines]]></mdurl>
##     <pdf><![CDATA[http://ieeexplore.ieee.org/stamp/stamp.jsp?arnumber=6479297]]></pdf>
##   </document>
##   <document>
##     <rank>132</rank>
##     <title><![CDATA[Advanced Lead&#x2013;Acid Batteries and the Development of Grid-Scale Energy Storage Systems]]></title>
##     <authors><![CDATA[McKeon, B.B.;  Furukawa, J.;  Fenstermacher, S.]]></authors>
##     <affiliations><![CDATA[Ecoult, Sydney, NSW, Australia]]></affiliations>
##     <controlledterms>
##       <term><![CDATA[battery management systems]]></term>
##       <term><![CDATA[lead acid batteries]]></term>
##     </controlledterms>
##     <thesaurusterms>
##       <term><![CDATA[Batteries]]></term>
##       <term><![CDATA[Battery management systems]]></term>
##       <term><![CDATA[Discharges (electric)]]></term>
##       <term><![CDATA[Electrodes]]></term>
##       <term><![CDATA[Energy conversion]]></term>
##       <term><![CDATA[Energy storage]]></term>
##       <term><![CDATA[Lead acid batteries]]></term>
##       <term><![CDATA[Power grids]]></term>
##       <term><![CDATA[Smart grids]]></term>
##     </thesaurusterms>
##     <pubtitle><![CDATA[Proceedings of the IEEE]]></pubtitle>
##     <punumber><![CDATA[5]]></punumber>
##     <pubtype><![CDATA[Journals & Magazines]]></pubtype>
##     <publisher><![CDATA[IEEE]]></publisher>
##     <volume><![CDATA[102]]></volume>
##     <issue><![CDATA[6]]></issue>
##     <py><![CDATA[2014]]></py>
##     <spage><![CDATA[951]]></spage>
##     <epage><![CDATA[963]]></epage>
##     <abstract><![CDATA[This paper discusses new developments in lead-acid battery chemistry and the importance of the system approach for implementation of battery energy storage for renewable energy and grid applications. The described solution includes thermal management of an UltraBattery bank, an inverter/charger, and smart grid management, which can monitor the state of charge (SoC) and the state of health (SoH) of the battery during system operation. With such features, it can allow the battery to operate within an optimum SoC window and thus can further maximize the longevity of the UltraBattery. Importantly, the smart battery management can trend the SoH of the battery and allow cell replacement at a convenient time without affecting the system operation. Furthermore, the advanced system package allows remote monitoring and control of operation, thus reducing the running cost of the system. It is clear that the widespread use of renewable-energy systems, in turn, would lead to a reduction in global consumption of the limited supplies of the fossil fuels and in the associated production of greenhouse-gas emissions.]]></abstract>
##     <issn><![CDATA[0018-9219]]></issn>
##     <htmlFlag><![CDATA[1]]></htmlFlag>
##     <arnumber><![CDATA[6809148]]></arnumber>
##     <doi><![CDATA[10.1109/JPROC.2014.2316823]]></doi>
##     <publicationId><![CDATA[6809148]]></publicationId>
##     <mdurl><![CDATA[http://ieeexplore.ieee.org/xpl/articleDetails.jsp?tp=&arnumber=6809148&contentType=Journals+%26+Magazines]]></mdurl>
##     <pdf><![CDATA[http://ieeexplore.ieee.org/stamp/stamp.jsp?arnumber=6809148]]></pdf>
##   </document>
##   <document>
##     <rank>133</rank>
##     <title><![CDATA[Reverse Engineering Digital Circuits Using Structural and Functional Analyses]]></title>
##     <authors><![CDATA[Subramanyan, P.;  Tsiskaridze, N.;  Wenchao Li;  Gascon, A.;  Wei Yang Tan;  Tiwari, A.;  Shankar, N.;  Seshia, S.A.;  Malik, S.]]></authors>
##     <affiliations><![CDATA[Dept. of Electr. Eng., Princeton Univ., Princeton, NJ, USA]]></affiliations>
##     <controlledterms>
##       <term><![CDATA[industrial property]]></term>
##       <term><![CDATA[integrated circuit design]]></term>
##       <term><![CDATA[invasive software]]></term>
##       <term><![CDATA[reverse engineering]]></term>
##       <term><![CDATA[system-on-chip]]></term>
##     </controlledterms>
##     <thesaurusterms>
##       <term><![CDATA[Algorithm design and analysis]]></term>
##       <term><![CDATA[Globalization]]></term>
##       <term><![CDATA[Hardware]]></term>
##       <term><![CDATA[Inference algorithms]]></term>
##       <term><![CDATA[Integrated circuits]]></term>
##       <term><![CDATA[Logic gates]]></term>
##       <term><![CDATA[Reverse engineering]]></term>
##       <term><![CDATA[Trojan horses]]></term>
##     </thesaurusterms>
##     <pubtitle><![CDATA[Emerging Topics in Computing, IEEE Transactions on]]></pubtitle>
##     <punumber><![CDATA[6245516]]></punumber>
##     <pubtype><![CDATA[Journals & Magazines]]></pubtype>
##     <publisher><![CDATA[IEEE]]></publisher>
##     <volume><![CDATA[2]]></volume>
##     <issue><![CDATA[1]]></issue>
##     <py><![CDATA[2014]]></py>
##     <spage><![CDATA[63]]></spage>
##     <epage><![CDATA[80]]></epage>
##     <abstract><![CDATA[Integrated circuits (ICs) are now designed and fabricated in a globalized multivendor environment making them vulnerable to malicious design changes, the insertion of hardware Trojans/malware, and intellectual property (IP) theft. Algorithmic reverse engineering of digital circuits can mitigate these concerns by enabling analysts to detect malicious hardware, verify the integrity of ICs, and detect IP violations. In this paper, we present a set of algorithms for the reverse engineering of digital circuits starting from an unstructured netlist and resulting in a high-level netlist with components such as register files, counters, adders, and subtractors. Our techniques require no manual intervention and experiments show that they determine the functionality of &gt;45% and up to 93% of the gates in each of the test circuits that we examine. We also demonstrate that our algorithms are scalable to real designs by experimenting with a very large, highly-optimized system-on-chip (SOC) design with over 375000 combinational elements. Our inference algorithms cover 68% of the gates in this SOC. We also demonstrate that our algorithms are effective in aiding a human analyst to detect hardware Trojans in an unstructured netlist.]]></abstract>
##     <issn><![CDATA[2168-6750]]></issn>
##     <htmlFlag><![CDATA[1]]></htmlFlag>
##     <arnumber><![CDATA[6683016]]></arnumber>
##     <doi><![CDATA[10.1109/TETC.2013.2294918]]></doi>
##     <publicationId><![CDATA[6683016]]></publicationId>
##     <mdurl><![CDATA[http://ieeexplore.ieee.org/xpl/articleDetails.jsp?tp=&arnumber=6683016&contentType=Journals+%26+Magazines]]></mdurl>
##     <pdf><![CDATA[http://ieeexplore.ieee.org/stamp/stamp.jsp?arnumber=6683016]]></pdf>
##   </document>
##   <document>
##     <rank>134</rank>
##     <title><![CDATA[A Statistical Model for Shadowed Body-Centric Communications Channels: Theory and Validation]]></title>
##     <authors><![CDATA[Cotton, S.L.]]></authors>
##     <affiliations><![CDATA[Inst. of Electron., Commun. & Inf. Technol., Queen's Univ. Belfast, Belfast, UK]]></affiliations>
##     <controlledterms>
##       <term><![CDATA[body area networks]]></term>
##       <term><![CDATA[fading channels]]></term>
##       <term><![CDATA[log normal distribution]]></term>
##       <term><![CDATA[multipath channels]]></term>
##       <term><![CDATA[statistical analysis]]></term>
##     </controlledterms>
##     <thesaurusterms>
##       <term><![CDATA[Antenna measurements]]></term>
##       <term><![CDATA[Communication channels]]></term>
##       <term><![CDATA[Equations]]></term>
##       <term><![CDATA[Fading]]></term>
##       <term><![CDATA[Mathematical model]]></term>
##       <term><![CDATA[Shadow mapping]]></term>
##       <term><![CDATA[Wireless communication]]></term>
##     </thesaurusterms>
##     <pubtitle><![CDATA[Antennas and Propagation, IEEE Transactions on]]></pubtitle>
##     <punumber><![CDATA[8]]></punumber>
##     <pubtype><![CDATA[Journals & Magazines]]></pubtype>
##     <publisher><![CDATA[IEEE]]></publisher>
##     <volume><![CDATA[62]]></volume>
##     <issue><![CDATA[3]]></issue>
##     <py><![CDATA[2014]]></py>
##     <spage><![CDATA[1416]]></spage>
##     <epage><![CDATA[1424]]></epage>
##     <abstract><![CDATA[This paper presents a new statistical signal reception model for shadowed body-centric communications channels. In this model, the potential clustering of multipath components is considered alongside the presence of elective dominant signal components. As typically occurs in body-centric communications channels, the dominant or line-of-sight (LOS) components are shadowed by body matter situated in the path trajectory. This situation may be further exacerbated due to physiological and biomechanical movements of the body. In the proposed model, the resultant dominant component which is formed by the phasor addition of these leading contributions is assumed to follow a lognormal distribution. A wide range of measured and simulated shadowed body-centric channels considering on-body, off-body and body-to-body communications are used to validate the model. During the course of the validation experiments, it was found that, even for environments devoid of multipath or specular reflections generated by the local surroundings, a noticeable resultant dominant component can still exist in body-centric channels where the user's body shadows the direct LOS signal path between the transmitter and the receiver.]]></abstract>
##     <issn><![CDATA[0018-926X]]></issn>
##     <htmlFlag><![CDATA[1]]></htmlFlag>
##     <arnumber><![CDATA[6687207]]></arnumber>
##     <doi><![CDATA[10.1109/TAP.2013.2295211]]></doi>
##     <publicationId><![CDATA[6687207]]></publicationId>
##     <mdurl><![CDATA[http://ieeexplore.ieee.org/xpl/articleDetails.jsp?tp=&arnumber=6687207&contentType=Journals+%26+Magazines]]></mdurl>
##     <pdf><![CDATA[http://ieeexplore.ieee.org/stamp/stamp.jsp?arnumber=6687207]]></pdf>
##   </document>
##   <document>
##     <rank>135</rank>
##     <title><![CDATA[Low-Cost Interfacing of Fibers to Nanophotonic Waveguides: Design for Fabrication and Assembly Tolerances]]></title>
##     <authors><![CDATA[Barwicz, T.;  Taira, Y.]]></authors>
##     <affiliations><![CDATA[IBM T.J. Watson Res. Center, Yorktown Heights, NY, USA]]></affiliations>
##     <controlledterms>
##       <term><![CDATA[elemental semiconductors]]></term>
##       <term><![CDATA[integrated optics]]></term>
##       <term><![CDATA[nanophotonics]]></term>
##       <term><![CDATA[optical design techniques]]></term>
##       <term><![CDATA[optical fibre couplers]]></term>
##       <term><![CDATA[optical fibre fabrication]]></term>
##       <term><![CDATA[optical polymers]]></term>
##       <term><![CDATA[optimisation]]></term>
##       <term><![CDATA[silicon]]></term>
##       <term><![CDATA[tolerance analysis]]></term>
##     </controlledterms>
##     <thesaurusterms>
##       <term><![CDATA[Couplers]]></term>
##       <term><![CDATA[Fabrication]]></term>
##       <term><![CDATA[Indexes]]></term>
##       <term><![CDATA[Optical fiber couplers]]></term>
##       <term><![CDATA[Polymers]]></term>
##     </thesaurusterms>
##     <pubtitle><![CDATA[Photonics Journal, IEEE]]></pubtitle>
##     <punumber><![CDATA[4563994]]></punumber>
##     <pubtype><![CDATA[Journals & Magazines]]></pubtype>
##     <publisher><![CDATA[IEEE]]></publisher>
##     <volume><![CDATA[6]]></volume>
##     <issue><![CDATA[4]]></issue>
##     <py><![CDATA[2014]]></py>
##     <spage><![CDATA[1]]></spage>
##     <epage><![CDATA[18]]></epage>
##     <abstract><![CDATA[The high cost and low scalability of interfacing standard optical fibers to nanophotonic waveguides hinder the deployment of silicon photonics. We propose a mechanically compliant low-cost interface with integrated polymer waveguides. Our concept promises better mechanical reliability than a direct fiber-to-chip coupling and a dramatically larger bandwidth than diffractive couplers. Our computations show a 0.1-dB penalty over a 200-nm bandwidth, whereas typical two-polarization vertical couplers show a ~1-dB penalty over a 30-nm bandwidth. In this paper, we present a comprehensive analysis of the design space using optimization routines to achieve a fabrication- and assembly-tolerant design. We demonstrate the concept feasibility through extensive tolerance analysis with parameter control assumptions derived from low-cost manufacturing.]]></abstract>
##     <issn><![CDATA[1943-0655]]></issn>
##     <htmlFlag><![CDATA[1]]></htmlFlag>
##     <arnumber><![CDATA[6838964]]></arnumber>
##     <doi><![CDATA[10.1109/JPHOT.2014.2331251]]></doi>
##     <publicationId><![CDATA[6838964]]></publicationId>
##     <mdurl><![CDATA[http://ieeexplore.ieee.org/xpl/articleDetails.jsp?tp=&arnumber=6838964&contentType=Journals+%26+Magazines]]></mdurl>
##     <pdf><![CDATA[http://ieeexplore.ieee.org/stamp/stamp.jsp?arnumber=6838964]]></pdf>
##   </document>
##   <document>
##     <rank>136</rank>
##     <title><![CDATA[Nondegenerate Four-Wave Mixing in a Dual-Mode Injection-Locked InAs/InP(100) Nanostructure Laser]]></title>
##     <authors><![CDATA[Cheng Wang;  Grillot, F.;  Fan-Yi Lin;  Aldaya, I.;  Batte, T.;  Gosset, C.;  Decerle, E.;  Even, J.]]></authors>
##     <affiliations><![CDATA[LTCI, Telecom ParisTech, Paris, France]]></affiliations>
##     <controlledterms>
##       <term><![CDATA[III-V semiconductors]]></term>
##       <term><![CDATA[indium compounds]]></term>
##       <term><![CDATA[laser mode locking]]></term>
##       <term><![CDATA[laser tuning]]></term>
##       <term><![CDATA[multiwave mixing]]></term>
##       <term><![CDATA[nanophotonics]]></term>
##       <term><![CDATA[quantum dot lasers]]></term>
##     </controlledterms>
##     <thesaurusterms>
##       <term><![CDATA[Fiber lasers]]></term>
##       <term><![CDATA[Laser mode locking]]></term>
##       <term><![CDATA[Optical fibers]]></term>
##       <term><![CDATA[Optical pumping]]></term>
##       <term><![CDATA[Pump lasers]]></term>
##       <term><![CDATA[Semiconductor lasers]]></term>
##     </thesaurusterms>
##     <pubtitle><![CDATA[Photonics Journal, IEEE]]></pubtitle>
##     <punumber><![CDATA[4563994]]></punumber>
##     <pubtype><![CDATA[Journals & Magazines]]></pubtype>
##     <publisher><![CDATA[IEEE]]></publisher>
##     <volume><![CDATA[6]]></volume>
##     <issue><![CDATA[1]]></issue>
##     <py><![CDATA[2014]]></py>
##     <spage><![CDATA[1]]></spage>
##     <epage><![CDATA[8]]></epage>
##     <abstract><![CDATA[The nondegenerate four-wave mixing (NDFWM) characteristics in a quantum-dot Fabry-Perot laser are investigated, employing the dual-mode injection-locking technique. The solitary laser features two lasing peaks, which provides the possibility for efficient FWM generation. Under optical injection, the NDFWM is operated up to a detuning range of 1.7 THz with a low injection ratio of 0.42. The normalized conversion efficiency (NCE) and the side-mode suppression ratio (SMSR) with respect to the converted signal are analyzed. The highest NCE of -17 dB associated with a SMSR of 20.3 dB is achieved at detuning of 110 GHz.]]></abstract>
##     <issn><![CDATA[1943-0655]]></issn>
##     <htmlFlag><![CDATA[1]]></htmlFlag>
##     <arnumber><![CDATA[6690207]]></arnumber>
##     <doi><![CDATA[10.1109/JPHOT.2013.2295473]]></doi>
##     <publicationId><![CDATA[6690207]]></publicationId>
##     <mdurl><![CDATA[http://ieeexplore.ieee.org/xpl/articleDetails.jsp?tp=&arnumber=6690207&contentType=Journals+%26+Magazines]]></mdurl>
##     <pdf><![CDATA[http://ieeexplore.ieee.org/stamp/stamp.jsp?arnumber=6690207]]></pdf>
##   </document>
##   <document>
##     <rank>137</rank>
##     <title><![CDATA[Tactile Sensing Chips With POSFET Array and Integrated Interface Electronics]]></title>
##     <authors><![CDATA[Dahiya, R.S.;  Adami, A.;  Pinna, L.;  Collini, C.;  Valle, M.;  Lorenzelli, L.]]></authors>
##     <affiliations><![CDATA[Dept. of Electron. & Nanoscale Eng., Univ. of Glasgow, Glasgow, UK]]></affiliations>
##     <controlledterms>
##       <term><![CDATA[CMOS integrated circuits]]></term>
##       <term><![CDATA[field effect transistors]]></term>
##       <term><![CDATA[piezoelectric semiconductors]]></term>
##       <term><![CDATA[tactile sensors]]></term>
##       <term><![CDATA[temperature measurement]]></term>
##     </controlledterms>
##     <thesaurusterms>
##       <term><![CDATA[Arrays]]></term>
##       <term><![CDATA[Capacitance]]></term>
##       <term><![CDATA[Logic gates]]></term>
##       <term><![CDATA[Robot sensing systems]]></term>
##       <term><![CDATA[System-on-chip]]></term>
##       <term><![CDATA[Temperature measurement]]></term>
##     </thesaurusterms>
##     <pubtitle><![CDATA[Sensors Journal, IEEE]]></pubtitle>
##     <punumber><![CDATA[7361]]></punumber>
##     <pubtype><![CDATA[Journals & Magazines]]></pubtype>
##     <publisher><![CDATA[IEEE]]></publisher>
##     <volume><![CDATA[14]]></volume>
##     <issue><![CDATA[10]]></issue>
##     <py><![CDATA[2014]]></py>
##     <spage><![CDATA[3448]]></spage>
##     <epage><![CDATA[3457]]></epage>
##     <abstract><![CDATA[PThis paper presents the advanced version of novel piezoelectric oxide semiconductor field effect transistor (POSFET) devices-based tactile sensing chip. The new version of the tactile sensing chip presented here comprises of a 4 &#x00D7; 4 array of POSFET touch sensing devices and integrated interface electronics (i.e., multiplexers, high compliance current sinks, and voltage output buffers). The chip also includes four temperature diodes for the measurement of contact temperature. Various components on the chip have been characterized systematically and the overall operation of the tactile sensing system has been evaluated. With new design, the POSFET devices have improved performance [i.e., linear response in the dynamic contact forces range of 0.01-3 N and sensitivity (without amplification) of 102.4 mV/N], which is more than twice the performance of their previous implementations. The integrated interface electronics result in reduced interconnections which otherwise would be needed to connect the POSFET array with off-chip interface electronic circuitry. This paper paves the way for CMOS implementation of full on-chip tactile sensing systems based on POSFETs.]]></abstract>
##     <issn><![CDATA[1530-437X]]></issn>
##     <htmlFlag><![CDATA[1]]></htmlFlag>
##     <arnumber><![CDATA[6877629]]></arnumber>
##     <doi><![CDATA[10.1109/JSEN.2014.2346742]]></doi>
##     <publicationId><![CDATA[6877629]]></publicationId>
##     <mdurl><![CDATA[http://ieeexplore.ieee.org/xpl/articleDetails.jsp?tp=&arnumber=6877629&contentType=Journals+%26+Magazines]]></mdurl>
##     <pdf><![CDATA[http://ieeexplore.ieee.org/stamp/stamp.jsp?arnumber=6877629]]></pdf>
##   </document>
##   <document>
##     <rank>138</rank>
##     <title><![CDATA[Growth Parameters of Fully Crystallized YIG, Bi:YIG, and Ce:YIG Films With High Faraday Rotations]]></title>
##     <authors><![CDATA[Block, A.D.;  Dulal, P.;  Stadler, B.J.H.;  Seaton, N.C.A.]]></authors>
##     <affiliations><![CDATA[Dept. of Electr. Eng., Univ. of Minnesota, Minneapolis, MN, USA]]></affiliations>
##     <controlledterms>
##       <term><![CDATA[Faraday effect]]></term>
##       <term><![CDATA[bismuth]]></term>
##       <term><![CDATA[cerium]]></term>
##       <term><![CDATA[crystallisation]]></term>
##       <term><![CDATA[sputter deposition]]></term>
##       <term><![CDATA[thin films]]></term>
##       <term><![CDATA[yttrium compounds]]></term>
##     </controlledterms>
##     <thesaurusterms>
##       <term><![CDATA[Annealing]]></term>
##       <term><![CDATA[Crystallization]]></term>
##       <term><![CDATA[Faraday effect]]></term>
##       <term><![CDATA[Films]]></term>
##       <term><![CDATA[Garnets]]></term>
##       <term><![CDATA[X-ray scattering]]></term>
##     </thesaurusterms>
##     <pubtitle><![CDATA[Photonics Journal, IEEE]]></pubtitle>
##     <punumber><![CDATA[4563994]]></punumber>
##     <pubtype><![CDATA[Journals & Magazines]]></pubtype>
##     <publisher><![CDATA[IEEE]]></publisher>
##     <volume><![CDATA[6]]></volume>
##     <issue><![CDATA[1]]></issue>
##     <py><![CDATA[2014]]></py>
##     <spage><![CDATA[1]]></spage>
##     <epage><![CDATA[8]]></epage>
##     <abstract><![CDATA[We report on the growth of thin films of yttrium iron garnet (YIG) on dielectric substrates. Such films have historically been challenging to grow due to either cracking or incomplete crystallization of the films. We have established the proper growth parameters by tuning seed layer thickness to an optimum of 45 nm. These films were then used as seed layers for growth of films of Bi:YIG and Ce:YIG. Bi:YIG films show a Faraday rotation of 1700 &#x00B0;/cm, and Ce:YIG films show a Faraday rotation of 3700 &#x00B0;/cm.]]></abstract>
##     <issn><![CDATA[1943-0655]]></issn>
##     <htmlFlag><![CDATA[1]]></htmlFlag>
##     <arnumber><![CDATA[6678201]]></arnumber>
##     <doi><![CDATA[10.1109/JPHOT.2013.2293610]]></doi>
##     <publicationId><![CDATA[6678201]]></publicationId>
##     <mdurl><![CDATA[http://ieeexplore.ieee.org/xpl/articleDetails.jsp?tp=&arnumber=6678201&contentType=Journals+%26+Magazines]]></mdurl>
##     <pdf><![CDATA[http://ieeexplore.ieee.org/stamp/stamp.jsp?arnumber=6678201]]></pdf>
##   </document>
##   <document>
##     <rank>139</rank>
##     <title><![CDATA[3-D Printing of Elements in Frequency Selective Arrays]]></title>
##     <authors><![CDATA[Sanz-Izquierdo, B.;  Parker, E.A.]]></authors>
##     <affiliations><![CDATA[Sch. of Eng. & Digital Arts, Univ. of Kent, Canterbury, UK]]></affiliations>
##     <controlledterms>
##       <term><![CDATA[cements (building materials)]]></term>
##       <term><![CDATA[electromagnetic wave propagation]]></term>
##       <term><![CDATA[frequency selective surfaces]]></term>
##       <term><![CDATA[three-dimensional printing]]></term>
##     </controlledterms>
##     <thesaurusterms>
##       <term><![CDATA[Bandwidth]]></term>
##       <term><![CDATA[Fabrication]]></term>
##       <term><![CDATA[Frequency selective surfaces]]></term>
##       <term><![CDATA[Printing]]></term>
##       <term><![CDATA[Resonant frequency]]></term>
##       <term><![CDATA[Three-dimensional displays]]></term>
##       <term><![CDATA[Vectors]]></term>
##     </thesaurusterms>
##     <pubtitle><![CDATA[Antennas and Propagation, IEEE Transactions on]]></pubtitle>
##     <punumber><![CDATA[8]]></punumber>
##     <pubtype><![CDATA[Journals & Magazines]]></pubtype>
##     <publisher><![CDATA[IEEE]]></publisher>
##     <volume><![CDATA[62]]></volume>
##     <issue><![CDATA[12]]></issue>
##     <py><![CDATA[2014]]></py>
##     <spage><![CDATA[6060]]></spage>
##     <epage><![CDATA[6066]]></epage>
##     <abstract><![CDATA[3-D printing is a technology that enables the fabrication of complex objects directly from a digital model. Folding the elements of Frequency Selective arrays in three dimensions gives a significant reduction in the resonant frequency for a given cell dimension, and such structures are candidates for additive manufacture. The aim in this paper is to demonstrate by example the development of novel electromagnetic structures that could be fabricated in parallel and integral with the additive manufacture of buildings, for electromagnetic architecture control. The principle is illustrated with two new geometries based on dipole and loop elements. The cores of these structures were fabricated with a 3-D printer that uses a plaster-based material. Theoretical and experimental results confirm the operation of the surfaces within the UHF frequency band.]]></abstract>
##     <issn><![CDATA[0018-926X]]></issn>
##     <htmlFlag><![CDATA[1]]></htmlFlag>
##     <arnumber><![CDATA[6905785]]></arnumber>
##     <doi><![CDATA[10.1109/TAP.2014.2359470]]></doi>
##     <publicationId><![CDATA[6905785]]></publicationId>
##     <mdurl><![CDATA[http://ieeexplore.ieee.org/xpl/articleDetails.jsp?tp=&arnumber=6905785&contentType=Journals+%26+Magazines]]></mdurl>
##     <pdf><![CDATA[http://ieeexplore.ieee.org/stamp/stamp.jsp?arnumber=6905785]]></pdf>
##   </document>
##   <document>
##     <rank>140</rank>
##     <title><![CDATA[Label-Free <inline-formula> <img src="/images/tex/968.gif" alt="\hbox {Si}_{3}\hbox {N}_{4}"> </inline-formula> Photonic Crystal Based Immunosensors for Diagnostic Applications]]></title>
##     <authors><![CDATA[Zecca, D.;  Qualtieri, A.;  Magno, G.;  Grande, M.;  Petruzzelli, V.;  Prieto-Simon, B.;  D'Orazio, A.;  De Vittorio, M.;  Voelcker, N.H.;  Stomeo, T.]]></authors>
##     <affiliations><![CDATA[Center for Bio-Mol. Nanotechnol., Ist. Italiano di Tecnol. (IIT), Lecce, Italy]]></affiliations>
##     <controlledterms>
##       <term><![CDATA[biochemistry]]></term>
##       <term><![CDATA[biomedical equipment]]></term>
##       <term><![CDATA[cancer]]></term>
##       <term><![CDATA[chemical sensors]]></term>
##       <term><![CDATA[membranes]]></term>
##       <term><![CDATA[molecular biophysics]]></term>
##       <term><![CDATA[nanofabrication]]></term>
##       <term><![CDATA[nanomedicine]]></term>
##       <term><![CDATA[nanosensors]]></term>
##       <term><![CDATA[patient diagnosis]]></term>
##       <term><![CDATA[photonic crystals]]></term>
##       <term><![CDATA[proteins]]></term>
##       <term><![CDATA[silicon compounds]]></term>
##     </controlledterms>
##     <thesaurusterms>
##       <term><![CDATA[Biosensors]]></term>
##       <term><![CDATA[Medical diagnosis]]></term>
##       <term><![CDATA[Optical sensors]]></term>
##       <term><![CDATA[Photonic crystals]]></term>
##       <term><![CDATA[Silicon]]></term>
##     </thesaurusterms>
##     <pubtitle><![CDATA[Photonics Journal, IEEE]]></pubtitle>
##     <punumber><![CDATA[4563994]]></punumber>
##     <pubtype><![CDATA[Journals & Magazines]]></pubtype>
##     <publisher><![CDATA[IEEE]]></publisher>
##     <volume><![CDATA[6]]></volume>
##     <issue><![CDATA[6]]></issue>
##     <py><![CDATA[2014]]></py>
##     <spage><![CDATA[1]]></spage>
##     <epage><![CDATA[7]]></epage>
##     <abstract><![CDATA[Immunosensors are devices that exploit immobilized antibodies to promote the binding of specific analytes related to diseases of medical importance, such as cancer or cardiac dysfunctions. Label-free immunosensors have an important role, due to their simplicity and fast read-out. Here, the proof of concept for an immunosensor based on a 2-D photonic crystal silicon nitride membrane is presented. The device has been fabricated by means of a well-tuned nanofabrication protocol, achieving a high-quality photonic pattern on a large-area membrane (1 mm &#x00D7; 1 mm), and it has been tested for the detection of interleukin-6, getting protein detection at pg/mL concentrations.]]></abstract>
##     <issn><![CDATA[1943-0655]]></issn>
##     <htmlFlag><![CDATA[1]]></htmlFlag>
##     <arnumber><![CDATA[6892930]]></arnumber>
##     <doi><![CDATA[10.1109/JPHOT.2014.2352625]]></doi>
##     <publicationId><![CDATA[6892930]]></publicationId>
##     <mdurl><![CDATA[http://ieeexplore.ieee.org/xpl/articleDetails.jsp?tp=&arnumber=6892930&contentType=Journals+%26+Magazines]]></mdurl>
##     <pdf><![CDATA[http://ieeexplore.ieee.org/stamp/stamp.jsp?arnumber=6892930]]></pdf>
##   </document>
##   <document>
##     <rank>141</rank>
##     <title><![CDATA[Segmentation of the Blood Vessels and Optic Disk in Retinal Images]]></title>
##     <authors><![CDATA[Salazar-Gonzalez, A.;  Kaba, D.;  Yongmin Li;  Xiaohui Liu]]></authors>
##     <affiliations><![CDATA[Dept. of Comput. Sci., Brunel Univ., Uxbridge, UK]]></affiliations>
##     <controlledterms>
##       <term><![CDATA[Markov processes]]></term>
##       <term><![CDATA[biomedical optical imaging]]></term>
##       <term><![CDATA[blood vessels]]></term>
##       <term><![CDATA[diseases]]></term>
##       <term><![CDATA[eye]]></term>
##       <term><![CDATA[feature extraction]]></term>
##       <term><![CDATA[image segmentation]]></term>
##       <term><![CDATA[medical image processing]]></term>
##       <term><![CDATA[random processes]]></term>
##     </controlledterms>
##     <thesaurusterms>
##       <term><![CDATA[Biomedical imaging]]></term>
##       <term><![CDATA[Blood vessels]]></term>
##       <term><![CDATA[Image reconstruction]]></term>
##       <term><![CDATA[Image segmentation]]></term>
##       <term><![CDATA[Markov random fields]]></term>
##       <term><![CDATA[Optical imaging]]></term>
##       <term><![CDATA[Performance evaluation]]></term>
##       <term><![CDATA[Retina]]></term>
##     </thesaurusterms>
##     <pubtitle><![CDATA[Biomedical and Health Informatics, IEEE Journal of]]></pubtitle>
##     <punumber><![CDATA[6221020]]></punumber>
##     <pubtype><![CDATA[Journals & Magazines]]></pubtype>
##     <publisher><![CDATA[IEEE]]></publisher>
##     <volume><![CDATA[18]]></volume>
##     <issue><![CDATA[6]]></issue>
##     <py><![CDATA[2014]]></py>
##     <spage><![CDATA[1874]]></spage>
##     <epage><![CDATA[1886]]></epage>
##     <abstract><![CDATA[Retinal image analysis is increasingly prominent as a nonintrusive diagnosis method in modern ophthalmology. In this paper, we present a novel method to segment blood vessels and optic disk in the fundus retinal images. The method could be used to support nonintrusive diagnosis in modern ophthalmology since the morphology of the blood vessel and the optic disk is an important indicator for diseases like diabetic retinopathy, glaucoma, and hypertension. Our method takes as first step the extraction of the retina vascular tree using the graph cut technique. The blood vessel information is then used to estimate the location of the optic disk. The optic disk segmentation is performed using two alternative methods. The Markov random field (MRF) image reconstruction method segments the optic disk by removing vessels from the optic disk region, and the compensation factor method segments the optic disk using the prior local intensity knowledge of the vessels. The proposed method is tested on three public datasets, DIARETDB1, DRIVE, and STARE. The results and comparison with alternative methods show that our method achieved exceptional performance in segmenting the blood vessel and optic disk.]]></abstract>
##     <issn><![CDATA[2168-2194]]></issn>
##     <htmlFlag><![CDATA[1]]></htmlFlag>
##     <arnumber><![CDATA[6725621]]></arnumber>
##     <doi><![CDATA[10.1109/JBHI.2014.2302749]]></doi>
##     <publicationId><![CDATA[6725621]]></publicationId>
##     <mdurl><![CDATA[http://ieeexplore.ieee.org/xpl/articleDetails.jsp?tp=&arnumber=6725621&contentType=Journals+%26+Magazines]]></mdurl>
##     <pdf><![CDATA[http://ieeexplore.ieee.org/stamp/stamp.jsp?arnumber=6725621]]></pdf>
##   </document>
##   <document>
##     <rank>142</rank>
##     <title><![CDATA[Organic Digital Logic and Analog Circuits Fabricated in a Roll-to-Roll Compatible Vacuum-Evaporation Process]]></title>
##     <authors><![CDATA[Taylor, D.M.;  Patchett, E.R.;  Williams, A.;  Neto, N.J.;  Ding, Z.;  Assender, H.E.;  Morrison, J.J.;  Yeates, S.G.]]></authors>
##     <affiliations><![CDATA[Sch. of Electron. Eng., Bangor Univ., Bangor, UK]]></affiliations>
##     <controlledterms>
##       <term><![CDATA[analogue circuits]]></term>
##       <term><![CDATA[flip-flops]]></term>
##       <term><![CDATA[logic circuits]]></term>
##       <term><![CDATA[logic gates]]></term>
##       <term><![CDATA[thin film transistors]]></term>
##       <term><![CDATA[transistor circuits]]></term>
##       <term><![CDATA[vacuum deposition]]></term>
##     </controlledterms>
##     <thesaurusterms>
##       <term><![CDATA[Fabrication]]></term>
##       <term><![CDATA[Integrated circuit modeling]]></term>
##       <term><![CDATA[Inverters]]></term>
##       <term><![CDATA[Latches]]></term>
##       <term><![CDATA[Logic gates]]></term>
##       <term><![CDATA[Thin film transistors]]></term>
##     </thesaurusterms>
##     <pubtitle><![CDATA[Electron Devices, IEEE Transactions on]]></pubtitle>
##     <punumber><![CDATA[16]]></punumber>
##     <pubtype><![CDATA[Journals & Magazines]]></pubtype>
##     <publisher><![CDATA[IEEE]]></publisher>
##     <volume><![CDATA[61]]></volume>
##     <issue><![CDATA[8]]></issue>
##     <py><![CDATA[2014]]></py>
##     <spage><![CDATA[2950]]></spage>
##     <epage><![CDATA[2956]]></epage>
##     <abstract><![CDATA[We report the fabrication of a range of organic circuits produced by a high-yielding, vacuum-based process compatible with roll-to-roll production. The circuits include inverters, NAND and NOR logic gates, a simple memory element (set-reset latch), and a modified Wilson current mirror circuit. The measured circuit responses are presented together with simulated responses based on a previously reported transistor model of organic transistors produced using our fabrication process. Circuit simulations replicated all the key features of the experimentally observed circuit performance. The logic gates were capable of operating at frequencies in excess of 1 kHz while the current mirror circuit produced currents up to 18 &#x03BC;A.]]></abstract>
##     <issn><![CDATA[0018-9383]]></issn>
##     <htmlFlag><![CDATA[1]]></htmlFlag>
##     <arnumber><![CDATA[6841624]]></arnumber>
##     <doi><![CDATA[10.1109/TED.2014.2329329]]></doi>
##     <publicationId><![CDATA[6841624]]></publicationId>
##     <mdurl><![CDATA[http://ieeexplore.ieee.org/xpl/articleDetails.jsp?tp=&arnumber=6841624&contentType=Journals+%26+Magazines]]></mdurl>
##     <pdf><![CDATA[http://ieeexplore.ieee.org/stamp/stamp.jsp?arnumber=6841624]]></pdf>
##   </document>
##   <document>
##     <rank>143</rank>
##     <title><![CDATA[Deflection Monitoring Method Using Fiber Bragg Gratings Applied to Tracking Particle Detectors]]></title>
##     <authors><![CDATA[Iadicicco, A.;  Della Pietra, M.;  Alviggi, M.;  Canale, V.;  Campopiano, S.]]></authors>
##     <affiliations><![CDATA[Eng. Dept., Univ. of Naples Parthenope, Naples, Italy]]></affiliations>
##     <controlledterms>
##       <term><![CDATA[Bragg gratings]]></term>
##       <term><![CDATA[electrodes]]></term>
##       <term><![CDATA[fibre optic sensors]]></term>
##       <term><![CDATA[microsensors]]></term>
##       <term><![CDATA[optical variables measurement]]></term>
##       <term><![CDATA[particle detectors]]></term>
##       <term><![CDATA[strain measurement]]></term>
##       <term><![CDATA[strain sensors]]></term>
##     </controlledterms>
##     <thesaurusterms>
##       <term><![CDATA[Fiber gratings]]></term>
##       <term><![CDATA[Monitoring]]></term>
##       <term><![CDATA[Radiation detectors]]></term>
##       <term><![CDATA[Strain]]></term>
##     </thesaurusterms>
##     <pubtitle><![CDATA[Photonics Journal, IEEE]]></pubtitle>
##     <punumber><![CDATA[4563994]]></punumber>
##     <pubtype><![CDATA[Journals & Magazines]]></pubtype>
##     <publisher><![CDATA[IEEE]]></publisher>
##     <volume><![CDATA[6]]></volume>
##     <issue><![CDATA[6]]></issue>
##     <py><![CDATA[2014]]></py>
##     <spage><![CDATA[1]]></spage>
##     <epage><![CDATA[10]]></epage>
##     <abstract><![CDATA[This paper proposes the use of fiber Bragg gratings (FBGs) for the deflection monitoring of a micromegas (MM) tracking particle detector to be installed at the European Organization for Nuclear Research during a major upgrade of the experiment ATLAS within 2018. MM detectors are designed to reach high spatial and time resolution, even if the design is not yet finalized. One mandatory issue for the MM detector is a precise monitoring of the deflection of the drift and read-out electrodes and/or of the panel hosting the electrodes. To this aim, FBG strain sensors are proposed and experimentally investigated as a sensing solution to monitor the strain state of the detector support panel hosting the drift and read-out electrodes. Finally, simple postprocessing analysis based on classical beam theory considering a rigid body permits calculating the panel deflection. Preliminary experimental results on first prototypes of small and large detector panels are presented and discussed.]]></abstract>
##     <issn><![CDATA[1943-0655]]></issn>
##     <htmlFlag><![CDATA[1]]></htmlFlag>
##     <arnumber><![CDATA[6887336]]></arnumber>
##     <doi><![CDATA[10.1109/JPHOT.2014.2352639]]></doi>
##     <publicationId><![CDATA[6887336]]></publicationId>
##     <mdurl><![CDATA[http://ieeexplore.ieee.org/xpl/articleDetails.jsp?tp=&arnumber=6887336&contentType=Journals+%26+Magazines]]></mdurl>
##     <pdf><![CDATA[http://ieeexplore.ieee.org/stamp/stamp.jsp?arnumber=6887336]]></pdf>
##   </document>
##   <document>
##     <rank>144</rank>
##     <title><![CDATA[An Efficient Finite Element Approach for Modeling Fibrotic Clefts in the Heart]]></title>
##     <authors><![CDATA[Mendonca Costa, C.;  Campos, F.O.;  Prassl, A.J.;  Weber dos Santos, R.;  Sanchez-Quintana, D.;  Ahammer, H.;  Hofer, E.;  Plank, G.]]></authors>
##     <affiliations><![CDATA[Inst. of Biophys., Med. Univ. of Graz, Graz, Austria]]></affiliations>
##     <controlledterms>
##       <term><![CDATA[biomedical electrodes]]></term>
##       <term><![CDATA[cellular biophysics]]></term>
##       <term><![CDATA[diseases]]></term>
##       <term><![CDATA[electrocardiography]]></term>
##       <term><![CDATA[finite element analysis]]></term>
##       <term><![CDATA[image resolution]]></term>
##       <term><![CDATA[image segmentation]]></term>
##       <term><![CDATA[medical image processing]]></term>
##     </controlledterms>
##     <thesaurusterms>
##       <term><![CDATA[Computational modeling]]></term>
##       <term><![CDATA[Extracellular]]></term>
##       <term><![CDATA[Finite element analysis]]></term>
##       <term><![CDATA[Frequency modulation]]></term>
##       <term><![CDATA[Image resolution]]></term>
##       <term><![CDATA[Iron]]></term>
##       <term><![CDATA[Mathematical model]]></term>
##     </thesaurusterms>
##     <pubtitle><![CDATA[Biomedical Engineering, IEEE Transactions on]]></pubtitle>
##     <punumber><![CDATA[10]]></punumber>
##     <pubtype><![CDATA[Journals & Magazines]]></pubtype>
##     <publisher><![CDATA[IEEE]]></publisher>
##     <volume><![CDATA[61]]></volume>
##     <issue><![CDATA[3]]></issue>
##     <py><![CDATA[2014]]></py>
##     <spage><![CDATA[900]]></spage>
##     <epage><![CDATA[910]]></epage>
##     <abstract><![CDATA[Advanced medical imaging technologies provide a wealth of information on cardiac anatomy and structure at a paracellular resolution, allowing to identify microstructural discontinuities which disrupt the intracellular matrix. Current state-of-the-art computer models built upon such datasets account for increasingly finer anatomical details, however, structural discontinuities at the paracellular level are typically discarded in the model generation process, owing to the significant costs which incur when using high resolutions for explicit representation. In this study, a novel discontinuous finite element (dFE) approach for discretizing the bidomain equations is presented, which accounts for fine-scale structures in a computer model without the need to increase spatial resolution. In the dFE method, this is achieved by imposing infinitely thin lines of electrical insulation along edges of finite elements which approximate the geometry of discontinuities in the intracellular matrix. Simulation results demonstrate that the dFE approach accounts for effects induced by microscopic size scale discontinuities, such as the formation of microscopic virtual electrodes, with vast computational savings as compared to high resolution continuous finite element models. Moreover, the method can be implemented in any standard continuous finite element code with minor effort.]]></abstract>
##     <issn><![CDATA[0018-9294]]></issn>
##     <htmlFlag><![CDATA[1]]></htmlFlag>
##     <arnumber><![CDATA[6671927]]></arnumber>
##     <doi><![CDATA[10.1109/TBME.2013.2292320]]></doi>
##     <publicationId><![CDATA[6671927]]></publicationId>
##     <mdurl><![CDATA[http://ieeexplore.ieee.org/xpl/articleDetails.jsp?tp=&arnumber=6671927&contentType=Journals+%26+Magazines]]></mdurl>
##     <pdf><![CDATA[http://ieeexplore.ieee.org/stamp/stamp.jsp?arnumber=6671927]]></pdf>
##   </document>
##   <document>
##     <rank>145</rank>
##     <title><![CDATA[Coherent Filterless Wideband Microwave/Millimeter-Wave Channelizer Based on Broadband Parametric Mixers]]></title>
##     <authors><![CDATA[Wiberg, A.O.J.;  Esman, D.J.;  Liu, L.;  Adleman, J.R.;  Zlatanovic, S.;  Ataie, V.;  Myslivets, E.;  Kuo, B.P.-P.;  Alic, N.;  Jacobs, E.W.;  Radic, S.]]></authors>
##     <affiliations><![CDATA[California Inst. for Telecommun. & Inf. Technol., Univ. of California San Diego, La Jolla, CA, USA]]></affiliations>
##     <controlledterms>
##       <term><![CDATA[mixers (circuits)]]></term>
##       <term><![CDATA[multicast communication]]></term>
##       <term><![CDATA[optical communication]]></term>
##       <term><![CDATA[telecommunication channels]]></term>
##     </controlledterms>
##     <thesaurusterms>
##       <term><![CDATA[Bandwidth]]></term>
##       <term><![CDATA[Frequency modulation]]></term>
##       <term><![CDATA[Mixers]]></term>
##       <term><![CDATA[Multicast communication]]></term>
##       <term><![CDATA[Nonlinear optics]]></term>
##       <term><![CDATA[Optical filters]]></term>
##       <term><![CDATA[Optical receivers]]></term>
##     </thesaurusterms>
##     <pubtitle><![CDATA[Lightwave Technology, Journal of]]></pubtitle>
##     <punumber><![CDATA[50]]></punumber>
##     <pubtype><![CDATA[Journals & Magazines]]></pubtype>
##     <publisher><![CDATA[IEEE]]></publisher>
##     <volume><![CDATA[32]]></volume>
##     <issue><![CDATA[20]]></issue>
##     <py><![CDATA[2014]]></py>
##     <spage><![CDATA[3609]]></spage>
##     <epage><![CDATA[3617]]></epage>
##     <abstract><![CDATA[An essential capability in many applications, ranging from commercial, surveillance and defense, is to analyze the spectral content of intercepted microwave and millimeter-wave signals over a very wide bandwidth in real-time and with high resolution. A range of photonic schemes have been introduced for the real-time processing of wideband signals to overcome limitations of current conventional electronic frequency measurement approaches. Here, a novel microwave/millimeter-wave channelizer is presented based on a RF photonic front-end employing parametric wavelength multicasting and comb generation. This new technology enables a contiguous bank of channelized coherent I/Q IF signals covering extremely wide RF instantaneous bandwidth. High channel counts and wide RF instantaneous bandwidth are enabled by use of parametrically generated frequency-locked optical combs spanning &gt;4 THz. Full field analysis capabilities of the coherent detection system are demonstrated by frequency domain analysis of 18 contiguous 1.2 GHz IF channels covering 15.5 GHz to 37.1 GHz input frequency range, and time and spectral domain analysis of a 75 GHz harmonically generated input signal. Sensitivity and dynamic range of the system are analyzed and discussed.]]></abstract>
##     <issn><![CDATA[0733-8724]]></issn>
##     <htmlFlag><![CDATA[1]]></htmlFlag>
##     <arnumber><![CDATA[6805609]]></arnumber>
##     <doi><![CDATA[10.1109/JLT.2014.2320445]]></doi>
##     <publicationId><![CDATA[6805609]]></publicationId>
##     <mdurl><![CDATA[http://ieeexplore.ieee.org/xpl/articleDetails.jsp?tp=&arnumber=6805609&contentType=Journals+%26+Magazines]]></mdurl>
##     <pdf><![CDATA[http://ieeexplore.ieee.org/stamp/stamp.jsp?arnumber=6805609]]></pdf>
##   </document>
##   <document>
##     <rank>146</rank>
##     <title><![CDATA[Sensorless Predictive Current Controlled DC&#x2013;DC Converter With a Self-Correction Differential Current Observer]]></title>
##     <authors><![CDATA[Qiao Zhang;  Run Min;  Qiaoling Tong;  Xuecheng Zou;  Zhenglin Liu;  Anwen Shen]]></authors>
##     <affiliations><![CDATA[UK Res. Centre, IMRA Eur., Brighton, UK]]></affiliations>
##     <controlledterms>
##       <term><![CDATA[DC-DC power convertors]]></term>
##       <term><![CDATA[PI control]]></term>
##       <term><![CDATA[electric current control]]></term>
##       <term><![CDATA[observers]]></term>
##       <term><![CDATA[predictive control]]></term>
##       <term><![CDATA[voltage control]]></term>
##     </controlledterms>
##     <thesaurusterms>
##       <term><![CDATA[Current control]]></term>
##       <term><![CDATA[DC-DC power converters]]></term>
##       <term><![CDATA[Observers]]></term>
##       <term><![CDATA[Sensorless control]]></term>
##       <term><![CDATA[Steady-state]]></term>
##       <term><![CDATA[Voltage control]]></term>
##     </thesaurusterms>
##     <pubtitle><![CDATA[Industrial Electronics, IEEE Transactions on]]></pubtitle>
##     <punumber><![CDATA[41]]></punumber>
##     <pubtype><![CDATA[Journals & Magazines]]></pubtype>
##     <publisher><![CDATA[IEEE]]></publisher>
##     <volume><![CDATA[61]]></volume>
##     <issue><![CDATA[12]]></issue>
##     <py><![CDATA[2014]]></py>
##     <spage><![CDATA[6747]]></spage>
##     <epage><![CDATA[6757]]></epage>
##     <abstract><![CDATA[For a sensorless predictive current controlled boost dc-dc converter, its small-signal model that contains a number of parasitic parameters, is derived in this paper. This model indicates that the type of system becomes type 0 even with the correction of voltage loop proportional-integral controller, leading to the existence of output voltage steady-state error. Then a self-correction differential current observer (SDCO) is proposed to eliminate this steady-state error and gain high transient response speed. The self-correction part of the SDCO makes the system become type 1 to achieve no steady-state error for output voltage, whereas the differential part can guarantee that the intermediate calculation results do not overflow. By carrying out a series of simulation verifications, further investigation proves that the proposed algorithm has good robustness. Finally, the effectiveness of the proposed algorithm is verified by experimental results.]]></abstract>
##     <issn><![CDATA[0278-0046]]></issn>
##     <htmlFlag><![CDATA[1]]></htmlFlag>
##     <arnumber><![CDATA[6779639]]></arnumber>
##     <doi><![CDATA[10.1109/TIE.2014.2314062]]></doi>
##     <publicationId><![CDATA[6779639]]></publicationId>
##     <mdurl><![CDATA[http://ieeexplore.ieee.org/xpl/articleDetails.jsp?tp=&arnumber=6779639&contentType=Journals+%26+Magazines]]></mdurl>
##     <pdf><![CDATA[http://ieeexplore.ieee.org/stamp/stamp.jsp?arnumber=6779639]]></pdf>
##   </document>
##   <document>
##     <rank>147</rank>
##     <title><![CDATA[Obtaining Statistically Random Information From Silicon Physical Unclonable Functions]]></title>
##     <authors><![CDATA[Chi-En Yin;  Gang Qu]]></authors>
##     <affiliations><![CDATA[Dept. of Electr. & Comput. Eng., Univ. of Maryland, College Park, MD, USA]]></affiliations>
##     <controlledterms>
##       <term><![CDATA[oscillators]]></term>
##       <term><![CDATA[polynomials]]></term>
##       <term><![CDATA[random processes]]></term>
##       <term><![CDATA[regression analysis]]></term>
##       <term><![CDATA[silicon]]></term>
##     </controlledterms>
##     <thesaurusterms>
##       <term><![CDATA[Encoding]]></term>
##       <term><![CDATA[Fabrication]]></term>
##       <term><![CDATA[NIST]]></term>
##       <term><![CDATA[Polynomials]]></term>
##       <term><![CDATA[Random sequences]]></term>
##       <term><![CDATA[Semiconductor process modeling]]></term>
##       <term><![CDATA[Statistical analysis]]></term>
##       <term><![CDATA[Systematics]]></term>
##     </thesaurusterms>
##     <pubtitle><![CDATA[Emerging Topics in Computing, IEEE Transactions on]]></pubtitle>
##     <punumber><![CDATA[6245516]]></punumber>
##     <pubtype><![CDATA[Journals & Magazines]]></pubtype>
##     <publisher><![CDATA[IEEE]]></publisher>
##     <volume><![CDATA[2]]></volume>
##     <issue><![CDATA[2]]></issue>
##     <py><![CDATA[2014]]></py>
##     <spage><![CDATA[96]]></spage>
##     <epage><![CDATA[106]]></epage>
##     <abstract><![CDATA[Silicon physical unclonable functions (PUFs) utilize the variation during silicon fabrication process to extract information that will be unique for each chip. There have been many recent approaches to how PUF can be used to improve security-related applications. However, it is well known that the fabrication variation has very strong spatial correlation&lt;;xref rid="fn1" ref-type="fn"&gt;&lt;;sup&gt;1&lt;;/sup&gt;&lt;;/xref&gt; and this has been pointed out as a security threat to silicon PUF. In fact, when we apply NIST's statistical test suite for randomness against the random sequences generated from a population of 125 ring oscillator PUFs using classic 1-out-of-8 coding and neighbor coding, none of them can pass all the tests. In this paper, we propose to decouple the unwanted systematic variation from the desired random variation through a regression-based distiller, where the basic idea is to build a model for the systematic variation so we can generate the random sequences only from the true random variation. Applying neighbor coding to the same benchmark data, our experiment shows that second- and third-order polynomials distill random sequences that pass all the NIST randomness tests. So does fourth-order polynomial in the case of 1-out-of-8 coding. Finally, we introduce two generic random sequence generation methods. The sequences they generate fail all the randomness tests, but with the help of our proposed polynomial distiller, all but one tests are passed. These results demonstrate that our method can provide statistically random PUF information and thus bolster the security characteristics of existing PUF schemes.&lt;;fn id="fn1"&gt;&lt;;label&gt;&lt;;sup&gt;1&lt;;/sup&gt;&lt;;/label&gt;&lt;;p&gt;Spatial correlations and systematic fabrication variations referred hereafter are different for each PUF.]]></abstract>
##     <issn><![CDATA[2168-6750]]></issn>
##     <htmlFlag><![CDATA[1]]></htmlFlag>
##     <arnumber><![CDATA[6786309]]></arnumber>
##     <doi><![CDATA[10.1109/TETC.2014.2316497]]></doi>
##     <publicationId><![CDATA[6786309]]></publicationId>
##     <mdurl><![CDATA[http://ieeexplore.ieee.org/xpl/articleDetails.jsp?tp=&arnumber=6786309&contentType=Journals+%26+Magazines]]></mdurl>
##     <pdf><![CDATA[http://ieeexplore.ieee.org/stamp/stamp.jsp?arnumber=6786309]]></pdf>
##   </document>
##   <document>
##     <rank>148</rank>
##     <title><![CDATA[Full ALD Al<sub>2</sub>O<sub>3</sub>/ZrO<sub>2</sub>/SiO<sub>2</sub>/ZrO<sub>2</sub>/Al<sub>2</sub>O<sub>3</sub> Stacks for High-Performance MIM Capacitors]]></title>
##     <authors><![CDATA[Qiu-Xiang Zhang;  Bao Zhu;  Shi-Jin Ding;  Hong-Liang Lu;  Qing-Qing Sun;  Peng Zhou;  Wei Zhang]]></authors>
##     <affiliations><![CDATA[Sch. of Microelectron., Fudan Univ., Shanghai, China]]></affiliations>
##     <controlledterms>
##       <term><![CDATA[MIM devices]]></term>
##       <term><![CDATA[aluminium compounds]]></term>
##       <term><![CDATA[atomic layer deposition]]></term>
##       <term><![CDATA[current density]]></term>
##       <term><![CDATA[leakage currents]]></term>
##       <term><![CDATA[silicon compounds]]></term>
##       <term><![CDATA[thin film capacitors]]></term>
##       <term><![CDATA[zirconium compounds]]></term>
##     </controlledterms>
##     <thesaurusterms>
##       <term><![CDATA[Aluminum oxide]]></term>
##       <term><![CDATA[Capacitance]]></term>
##       <term><![CDATA[Capacitors]]></term>
##       <term><![CDATA[Electric breakdown]]></term>
##       <term><![CDATA[Electron devices]]></term>
##       <term><![CDATA[MIM capacitors]]></term>
##     </thesaurusterms>
##     <pubtitle><![CDATA[Electron Device Letters, IEEE]]></pubtitle>
##     <punumber><![CDATA[55]]></punumber>
##     <pubtype><![CDATA[Journals & Magazines]]></pubtype>
##     <publisher><![CDATA[IEEE]]></publisher>
##     <volume><![CDATA[35]]></volume>
##     <issue><![CDATA[11]]></issue>
##     <py><![CDATA[2014]]></py>
##     <spage><![CDATA[1121]]></spage>
##     <epage><![CDATA[1123]]></epage>
##     <abstract><![CDATA[Metal-insulator-metal (MIM) capacitors with full atomic-layer-deposition Al<sub>2</sub>O<sub>3</sub>/ZrO<sub>2</sub>/SiO<sub>2</sub>/ZrO<sub>2</sub>/Al<sub>2</sub>O<sub>3</sub> stacks were explored for the first time. As the incorporated SiO<sub>2</sub> film thickness increased from 0 to 3 nm, the quadratic and linear voltage coefficients of capacitance (&#x03B1; and &#x03B2;) of the MIM capacitors reduced significantly from positive values to negative ones. For the stack with 3-nm SiO<sub>2</sub> film, a capacitance density of 7.40 fF/&#x03BC;m<sup>2</sup>, &#x03B1; of -121 ppm/V<sup>2</sup>, and &#x03B2; of -116 ppm/V were achieved, together with very low leakage current densities of 3.08 &#x00D7; 10<sup>-8</sup> A/cm<sup>2</sup> at 5 V at room temperature (RT) and 5.89 &#x00D7; 10<sup>-8</sup> A/cm<sup>2</sup> at 3.3 V at 125 &#x00B0;C, high breakdown field of 6.05 MV/cm, and high operating voltage of 6.3 V for a 10-year lifetime at RT. Thus, this type of stacks is a very promising candidate for next generation radio frequency and analog/mixed-signal integrated circuits.]]></abstract>
##     <issn><![CDATA[0741-3106]]></issn>
##     <htmlFlag><![CDATA[1]]></htmlFlag>
##     <arnumber><![CDATA[6916992]]></arnumber>
##     <doi><![CDATA[10.1109/LED.2014.2359195]]></doi>
##     <publicationId><![CDATA[6916992]]></publicationId>
##     <mdurl><![CDATA[http://ieeexplore.ieee.org/xpl/articleDetails.jsp?tp=&arnumber=6916992&contentType=Journals+%26+Magazines]]></mdurl>
##     <pdf><![CDATA[http://ieeexplore.ieee.org/stamp/stamp.jsp?arnumber=6916992]]></pdf>
##   </document>
##   <document>
##     <rank>149</rank>
##     <title><![CDATA[Online Bayesian Learning With Natural Sequential Prior Distribution]]></title>
##     <authors><![CDATA[Nakada, Y.;  Wakahara, M.;  Matsumoto, T.]]></authors>
##     <affiliations><![CDATA[Coll. of Sci. & Eng., Aoyama Gakuin Univ., Sagamihara, Japan]]></affiliations>
##     <controlledterms>
##       <term><![CDATA[belief networks]]></term>
##       <term><![CDATA[learning (artificial intelligence)]]></term>
##       <term><![CDATA[matrix algebra]]></term>
##       <term><![CDATA[multilayer perceptrons]]></term>
##       <term><![CDATA[radial basis function networks]]></term>
##     </controlledterms>
##     <pubtitle><![CDATA[Neural Networks and Learning Systems, IEEE Transactions on]]></pubtitle>
##     <punumber><![CDATA[5962385]]></punumber>
##     <pubtype><![CDATA[Journals & Magazines]]></pubtype>
##     <publisher><![CDATA[IEEE]]></publisher>
##     <volume><![CDATA[25]]></volume>
##     <issue><![CDATA[1]]></issue>
##     <py><![CDATA[2014]]></py>
##     <spage><![CDATA[40]]></spage>
##     <epage><![CDATA[54]]></epage>
##     <abstract><![CDATA[Online Bayesian learning has been successfully applied to online learning for multilayer perceptrons and radial basis functions. In online Bayesian learning, typically, the conventional transition model has been used. Although the conventional transition model is based on the squared norm of the difference between the current parameter vector and the previous parameter vector, the transition model does not adequately consider the difference between the current observation model and the previous observation model. To adequately consider this difference between the observation models, we propose a natural sequential prior. The proposed transition model uses a Fisher information matrix to consider the difference between the observation models more naturally. For validation, the proposed transition model is applied to an online learning problem for a three-layer perceptron.]]></abstract>
##     <issn><![CDATA[2162-237X]]></issn>
##     <htmlFlag><![CDATA[1]]></htmlFlag>
##     <arnumber><![CDATA[6490411]]></arnumber>
##     <doi><![CDATA[10.1109/TNNLS.2013.2250999]]></doi>
##     <publicationId><![CDATA[6490411]]></publicationId>
##     <mdurl><![CDATA[http://ieeexplore.ieee.org/xpl/articleDetails.jsp?tp=&arnumber=6490411&contentType=Journals+%26+Magazines]]></mdurl>
##     <pdf><![CDATA[http://ieeexplore.ieee.org/stamp/stamp.jsp?arnumber=6490411]]></pdf>
##   </document>
##   <document>
##     <rank>150</rank>
##     <title><![CDATA[Robust Broadband Optical Transmission Realized in a Dual-Metallic-Grating Structure]]></title>
##     <authors><![CDATA[Jing Nie;  Hu-Quan Li;  Wen Liu]]></authors>
##     <affiliations><![CDATA[Wuhan Nat. Lab. for Optoelectron., Huazhong Univ. of Sci. & Technol., Wuhan, China]]></affiliations>
##     <controlledterms>
##       <term><![CDATA[diffraction gratings]]></term>
##       <term><![CDATA[infrared spectra]]></term>
##       <term><![CDATA[light transmission]]></term>
##       <term><![CDATA[optical polarisers]]></term>
##       <term><![CDATA[silicon compounds]]></term>
##       <term><![CDATA[silver]]></term>
##     </controlledterms>
##     <thesaurusterms>
##       <term><![CDATA[Broadband communication]]></term>
##       <term><![CDATA[Dielectrics]]></term>
##       <term><![CDATA[Gratings]]></term>
##       <term><![CDATA[Integrated circuit modeling]]></term>
##       <term><![CDATA[Magnetic fields]]></term>
##       <term><![CDATA[Optical polarization]]></term>
##       <term><![CDATA[Silver]]></term>
##     </thesaurusterms>
##     <pubtitle><![CDATA[Photonics Journal, IEEE]]></pubtitle>
##     <punumber><![CDATA[4563994]]></punumber>
##     <pubtype><![CDATA[Journals & Magazines]]></pubtype>
##     <publisher><![CDATA[IEEE]]></publisher>
##     <volume><![CDATA[6]]></volume>
##     <issue><![CDATA[4]]></issue>
##     <py><![CDATA[2014]]></py>
##     <spage><![CDATA[1]]></spage>
##     <epage><![CDATA[8]]></epage>
##     <abstract><![CDATA[We present a new mechanism to realize broadband transmission (BT) at near infrared in a dual-layer metallic-silver-grating structure with a silicon nitride dielectric spacer. The physical mechanism is ascribed to the partial degeneracy between the first-order magnetic polariton (MP<sub>1</sub>) mode supported in the thin dielectric spacer and the Fabry-Pe&#x0301;rot-like resonant modes supported in the gratings' slits. With parameters optimized, high transmittance up to 85% with full-width-at-half-maximum bandwidth of 64% near 1.9 &#x03BC;m can be realized for TM-polarized light. Moreover, the BT can be maintained in a large range of oblique incident angles from 0&#x00B0; to 30&#x00B0;. In another word, the BT is capable of not only collimated beams but also noncollimated incident lights. Moreover, the high-transmission window can be tuned in the IR spectrum region by modulating the metallic gratings' thicknesses.]]></abstract>
##     <issn><![CDATA[1943-0655]]></issn>
##     <htmlFlag><![CDATA[1]]></htmlFlag>
##     <arnumber><![CDATA[6842609]]></arnumber>
##     <doi><![CDATA[10.1109/JPHOT.2014.2332462]]></doi>
##     <publicationId><![CDATA[6842609]]></publicationId>
##     <mdurl><![CDATA[http://ieeexplore.ieee.org/xpl/articleDetails.jsp?tp=&arnumber=6842609&contentType=Journals+%26+Magazines]]></mdurl>
##     <pdf><![CDATA[http://ieeexplore.ieee.org/stamp/stamp.jsp?arnumber=6842609]]></pdf>
##   </document>
##   <document>
##     <rank>151</rank>
##     <title><![CDATA[Letters: optically transparent piezoelectric transducer for ultrasonic particle manipulation]]></title>
##     <authors><![CDATA[Brodie, G.;  Yongqiang Qiu;  Cochran, S.;  Spalding, G.;  Macdonald, M.]]></authors>
##     <affiliations><![CDATA[Inst. for Med. Sci. & Technol., Univ. of Dundee, Dundee, UK]]></affiliations>
##     <controlledterms>
##       <term><![CDATA[coatings]]></term>
##       <term><![CDATA[indium compounds]]></term>
##       <term><![CDATA[lithium compounds]]></term>
##       <term><![CDATA[piezoelectric transducers]]></term>
##       <term><![CDATA[ultrasonic transducers]]></term>
##     </controlledterms>
##     <thesaurusterms>
##       <term><![CDATA[Acoustics]]></term>
##       <term><![CDATA[Biomedical optical imaging]]></term>
##       <term><![CDATA[Indium tin oxide]]></term>
##       <term><![CDATA[Lithium niobate]]></term>
##       <term><![CDATA[Optical imaging]]></term>
##       <term><![CDATA[Optical receivers]]></term>
##       <term><![CDATA[Optical variables control]]></term>
##     </thesaurusterms>
##     <pubtitle><![CDATA[Ultrasonics, Ferroelectrics, and Frequency Control, IEEE Transactions on]]></pubtitle>
##     <punumber><![CDATA[58]]></punumber>
##     <pubtype><![CDATA[Journals & Magazines]]></pubtype>
##     <publisher><![CDATA[IEEE]]></publisher>
##     <volume><![CDATA[61]]></volume>
##     <issue><![CDATA[3]]></issue>
##     <py><![CDATA[2014]]></py>
##     <spage><![CDATA[389]]></spage>
##     <epage><![CDATA[391]]></epage>
##     <abstract><![CDATA[We report an optically transparent ultrasonic device, consisting of indium-tin-oxide-coated lithium niobate (LNO), for use in particle manipulation. This device shows good transparency in the visible and near-infrared wavelengths and, acoustically, compares favorably with conventional prototype devices with silver electrodes.]]></abstract>
##     <issn><![CDATA[0885-3010]]></issn>
##     <htmlFlag><![CDATA[1]]></htmlFlag>
##     <arnumber><![CDATA[6746316]]></arnumber>
##     <doi><![CDATA[10.1109/TUFFC.2014.2923]]></doi>
##     <publicationId><![CDATA[6746316]]></publicationId>
##     <mdurl><![CDATA[http://ieeexplore.ieee.org/xpl/articleDetails.jsp?tp=&arnumber=6746316&contentType=Journals+%26+Magazines]]></mdurl>
##     <pdf><![CDATA[http://ieeexplore.ieee.org/stamp/stamp.jsp?arnumber=6746316]]></pdf>
##   </document>
##   <document>
##     <rank>152</rank>
##     <title><![CDATA[Linewidth Characterization of Integrable Slotted Single-Mode Lasers]]></title>
##     <authors><![CDATA[Abdullaev, A.;  Qiaoyin Lu;  Wei-Hua Guo;  Nawrocka, M.;  Bello, F.;  O'Callaghan, J.;  Donegan, J.F.]]></authors>
##     <affiliations><![CDATA[Photonics Group, Trinity Coll., Dublin, Ireland]]></affiliations>
##     <controlledterms>
##       <term><![CDATA[laser cavity resonators]]></term>
##       <term><![CDATA[laser modes]]></term>
##       <term><![CDATA[optical modulation]]></term>
##       <term><![CDATA[reflectivity]]></term>
##       <term><![CDATA[semiconductor optical amplifiers]]></term>
##     </controlledterms>
##     <thesaurusterms>
##       <term><![CDATA[Cavity resonators]]></term>
##       <term><![CDATA[Distributed feedback devices]]></term>
##       <term><![CDATA[Laser modes]]></term>
##       <term><![CDATA[Measurement by laser beam]]></term>
##       <term><![CDATA[Semiconductor lasers]]></term>
##       <term><![CDATA[Temperature measurement]]></term>
##     </thesaurusterms>
##     <pubtitle><![CDATA[Photonics Technology Letters, IEEE]]></pubtitle>
##     <punumber><![CDATA[68]]></punumber>
##     <pubtype><![CDATA[Journals & Magazines]]></pubtype>
##     <publisher><![CDATA[IEEE]]></publisher>
##     <volume><![CDATA[26]]></volume>
##     <issue><![CDATA[22]]></issue>
##     <py><![CDATA[2014]]></py>
##     <spage><![CDATA[2225]]></spage>
##     <epage><![CDATA[2228]]></epage>
##     <abstract><![CDATA[A single-mode laser platform has been developed with a simple and high-yield fabrication based on etched slots. We present linewidth measurements on such lasers, which are fabricated without any regrowth steps. The group of slots placed on one side of laser cavity provides sufficient reflectivity for lasing, which allows the laser to be integrated with other devices such as semiconductor optical amplifiers and modulators. The laser with an effective cavity length of 450 &#x03BC;m exhibits a threshold current of 31 mA and a side-mode suppression ratio of 48 dB. The laser shows a minimum Lorentzian linewidth of ~720 kHz at room temperature. The linewidth remains ~1 MHz within the temperature range of 10 &#x00B0;C-60 &#x00B0;C.]]></abstract>
##     <issn><![CDATA[1041-1135]]></issn>
##     <htmlFlag><![CDATA[1]]></htmlFlag>
##     <arnumber><![CDATA[6882129]]></arnumber>
##     <doi><![CDATA[10.1109/LPT.2014.2350772]]></doi>
##     <publicationId><![CDATA[6882129]]></publicationId>
##     <mdurl><![CDATA[http://ieeexplore.ieee.org/xpl/articleDetails.jsp?tp=&arnumber=6882129&contentType=Journals+%26+Magazines]]></mdurl>
##     <pdf><![CDATA[http://ieeexplore.ieee.org/stamp/stamp.jsp?arnumber=6882129]]></pdf>
##   </document>
##   <document>
##     <rank>153</rank>
##     <title><![CDATA[Temporal Workload-Aware Replicated Partitioning for Social Networks]]></title>
##     <authors><![CDATA[Turk, A.;  Oguz Selvitopi, R.;  Ferhatosmanoglu, H.;  Aykanat, C.]]></authors>
##     <affiliations><![CDATA[Yahoo Labs., Barcelona, Spain]]></affiliations>
##     <controlledterms>
##       <term><![CDATA[cloud computing]]></term>
##       <term><![CDATA[data handling]]></term>
##       <term><![CDATA[data structures]]></term>
##       <term><![CDATA[graph theory]]></term>
##       <term><![CDATA[query processing]]></term>
##       <term><![CDATA[resource allocation]]></term>
##       <term><![CDATA[social networking (online)]]></term>
##     </controlledterms>
##     <thesaurusterms>
##       <term><![CDATA[Cloning]]></term>
##       <term><![CDATA[Licenses]]></term>
##       <term><![CDATA[Load modeling]]></term>
##       <term><![CDATA[Measurement]]></term>
##       <term><![CDATA[Servers]]></term>
##       <term><![CDATA[Twitter]]></term>
##     </thesaurusterms>
##     <pubtitle><![CDATA[Knowledge and Data Engineering, IEEE Transactions on]]></pubtitle>
##     <punumber><![CDATA[69]]></punumber>
##     <pubtype><![CDATA[Journals & Magazines]]></pubtype>
##     <publisher><![CDATA[IEEE]]></publisher>
##     <volume><![CDATA[26]]></volume>
##     <issue><![CDATA[11]]></issue>
##     <py><![CDATA[2014]]></py>
##     <spage><![CDATA[2832]]></spage>
##     <epage><![CDATA[2845]]></epage>
##     <abstract><![CDATA[Most frequent and expensive queries in social networks involve multi-user operations such as requesting the latest tweets or news-feeds of friends. The performance of such queries are heavily dependent on the data partitioning and replication methodologies adopted by the underlying systems. Existing solutions for data distribution in these systems involve hashor graph-based approaches that ignore the multi-way relations among data. In this work, we propose a novel data partitioning and selective replication method that utilizes the temporal information in prior workloads to predict future query patterns. Our method utilizes the social network structure and the temporality of the interactions among its users to construct a hypergraph that correctly models multi-user operations. It then performs simultaneous partitioning and replication of this hypergraph to reduce the query span while respecting load balance and I/O load constraints under replication. To test our model, we enhance the Cassandra NoSQL system to support selective replication and we implement a social network application (a Twitter clone) utilizing our enhanced Cassandra. We conduct experiments on a cloud computing environment (Amazon EC2) to test the developed systems. Comparison of the proposed method with hash- and enhanced graph-based schemes indicate that it significantly improves latency and throughput.]]></abstract>
##     <issn><![CDATA[1041-4347]]></issn>
##     <arnumber><![CDATA[6720183]]></arnumber>
##     <doi><![CDATA[10.1109/TKDE.2014.2302291]]></doi>
##     <publicationId><![CDATA[6720183]]></publicationId>
##     <mdurl><![CDATA[http://ieeexplore.ieee.org/xpl/articleDetails.jsp?tp=&arnumber=6720183&contentType=Journals+%26+Magazines]]></mdurl>
##     <pdf><![CDATA[http://ieeexplore.ieee.org/stamp/stamp.jsp?arnumber=6720183]]></pdf>
##   </document>
##   <document>
##     <rank>154</rank>
##     <title><![CDATA[Influence of Temperature on Reflectivity of Symmetrical Metal-Cladding Optical Waveguide]]></title>
##     <authors><![CDATA[Hu, J.;  Chen, J.;  Liang, B.;  Jiang, Q.;  Wang, Y.;  Zhuang, S.]]></authors>
##     <affiliations><![CDATA[Shanghai Key Laboratory of Modern Optical System, University of Shanghai for Science and Technology, Shanghai, China]]></affiliations>
##     <thesaurusterms>
##       <term><![CDATA[Optical films]]></term>
##       <term><![CDATA[Optical reflection]]></term>
##       <term><![CDATA[Optical sensors]]></term>
##       <term><![CDATA[Optical waveguides]]></term>
##       <term><![CDATA[Reflectivity]]></term>
##       <term><![CDATA[Sensors]]></term>
##       <term><![CDATA[Temperature]]></term>
##     </thesaurusterms>
##     <pubtitle><![CDATA[Photonics Technology Letters, IEEE]]></pubtitle>
##     <punumber><![CDATA[68]]></punumber>
##     <pubtype><![CDATA[Journals & Magazines]]></pubtype>
##     <publisher><![CDATA[IEEE]]></publisher>
##     <volume><![CDATA[26]]></volume>
##     <issue><![CDATA[21]]></issue>
##     <py><![CDATA[2014]]></py>
##     <spage><![CDATA[2166]]></spage>
##     <epage><![CDATA[2169]]></epage>
##     <abstract><![CDATA[The influence of temperature on the reflectivity of symmetrical metal-cladding optical waveguide (SMCOW) is studied theoretically by means of single-factor investigation under spectral and angular interrogation mode of operation. The theoretical model for temperature dependence of reflectivity includes the temperature dependence of refractive index and thickness of guiding layer, the temperature dependence of the metal film thickness and metal-dielectric function. It is found that the effect of temperature on the reflectivity of SMCOW is mainly attributed to the temperature dependence of refractive index and thickness of guiding layer; on the contrary, the temperature properties of metal film hardly contribute to the influence of temperature on the reflectivity. Based on the analysis, the sensitivities of SMCOW with guiding layer made up of different optical glasses are computed under both spectral and angular interrogation. This letter is supposed to provide direction in designing SMCOW sensors against the temperature variation.]]></abstract>
##     <issn><![CDATA[1041-1135]]></issn>
##     <htmlFlag><![CDATA[1]]></htmlFlag>
##     <arnumber><![CDATA[6880795]]></arnumber>
##     <doi><![CDATA[10.1109/LPT.2014.2349944]]></doi>
##     <publicationId><![CDATA[6880795]]></publicationId>
##     <mdurl><![CDATA[http://ieeexplore.ieee.org/xpl/articleDetails.jsp?tp=&arnumber=6880795&contentType=Journals+%26+Magazines]]></mdurl>
##     <pdf><![CDATA[http://ieeexplore.ieee.org/stamp/stamp.jsp?arnumber=6880795]]></pdf>
##   </document>
##   <document>
##     <rank>155</rank>
##     <title><![CDATA[Improved Rear Surface Passivation of Cu(In,Ga)Se<formula formulatype="inline"> <img src="/images/tex/996.gif" alt="_{\bf 2}"> </formula> Solar Cells: A Combination of an Al<formula formulatype="inline"> <img src="/images/tex/996.gif" alt="_{\bf 2}"> </formula>O <formula formulatype="inline"> <img src="/images/tex/20074.gif" alt="_{\bf 3}"> </formula> Rear Surface Passivation Layer and Nanosized Local Rear Point Contacts]]></title>
##     <authors><![CDATA[Vermang, B.;  Fja&#x0308; llstro&#x0308; m, V.;  Xindong Gao;  Edoff, M.]]></authors>
##     <affiliations><![CDATA[Dept. of Eng. Sci., Uppsala Univ., Uppsala, Sweden]]></affiliations>
##     <controlledterms>
##       <term><![CDATA[alumina]]></term>
##       <term><![CDATA[atomic layer deposition]]></term>
##       <term><![CDATA[copper compounds]]></term>
##       <term><![CDATA[gallium compounds]]></term>
##       <term><![CDATA[indium compounds]]></term>
##       <term><![CDATA[liquid phase deposition]]></term>
##       <term><![CDATA[passivation]]></term>
##       <term><![CDATA[point contacts]]></term>
##       <term><![CDATA[semiconductor thin films]]></term>
##       <term><![CDATA[solar cells]]></term>
##       <term><![CDATA[ternary semiconductors]]></term>
##       <term><![CDATA[thin film devices]]></term>
##     </controlledterms>
##     <thesaurusterms>
##       <term><![CDATA[Aluminum oxide]]></term>
##       <term><![CDATA[Passivation]]></term>
##       <term><![CDATA[Photovoltaic cells]]></term>
##       <term><![CDATA[Silicon]]></term>
##       <term><![CDATA[Standards]]></term>
##       <term><![CDATA[Substrates]]></term>
##     </thesaurusterms>
##     <pubtitle><![CDATA[Photovoltaics, IEEE Journal of]]></pubtitle>
##     <punumber><![CDATA[5503869]]></punumber>
##     <pubtype><![CDATA[Journals & Magazines]]></pubtype>
##     <publisher><![CDATA[IEEE]]></publisher>
##     <volume><![CDATA[4]]></volume>
##     <issue><![CDATA[1]]></issue>
##     <py><![CDATA[2014]]></py>
##     <spage><![CDATA[486]]></spage>
##     <epage><![CDATA[492]]></epage>
##     <abstract><![CDATA[An innovative rear contacting structure for copper indium gallium (di) selenide (CIGS) thin-film solar cells is developed in an industrially viable way and demonstrated in tangible devices. The idea stems from the silicon (Si) industry, where rear surface passivation layers are combined with micron-sized local point contacts to boost the open-circuit voltage (V<sub>OC</sub>) and, hence, cell efficiency. However, compared with Si solar cells, CIGS solar cell minority carrier diffusion lengths are several orders lower in magnitude. Therefore, the proposed CIGS cell design reduces rear surface recombination by combining a rear surface passivation layer and nanosized local point contacts. Atomic layer deposition of Al<sub>2</sub>O<sub>3</sub> is used to passivate the CIGS surface and the formation of nanosphere-shaped precipitates in chemical bath deposition of CdS to generate nanosized point contact openings. The manufactured Al<sub>2</sub>O<sub>3</sub> rear surface passivated CIGS solar cells with nanosized local rear point contacts show a significant improvement in V<sub>OC</sub> compared with unpassivated reference cells.]]></abstract>
##     <issn><![CDATA[2156-3381]]></issn>
##     <htmlFlag><![CDATA[1]]></htmlFlag>
##     <arnumber><![CDATA[6662413]]></arnumber>
##     <doi><![CDATA[10.1109/JPHOTOV.2013.2287769]]></doi>
##     <publicationId><![CDATA[6662413]]></publicationId>
##     <mdurl><![CDATA[http://ieeexplore.ieee.org/xpl/articleDetails.jsp?tp=&arnumber=6662413&contentType=Journals+%26+Magazines]]></mdurl>
##     <pdf><![CDATA[http://ieeexplore.ieee.org/stamp/stamp.jsp?arnumber=6662413]]></pdf>
##   </document>
##   <document>
##     <rank>156</rank>
##     <title><![CDATA[2 <formula formulatype="inline"> <img src="/images/tex/326.gif" alt="\times"> </formula> 2 MIMO OFDM-RoF System Employing LMS-Based Equalizer With I/Q Imbalance Compensation at 60 GHz]]></title>
##     <authors><![CDATA[Hou-Tzu Huang;  Po-Tsung Shih;  Chun-Ting Lin;  Yu-Hsuan Cheng;  Wan-Ling Liang;  Chun-Hung Ho;  Chia-Chien Wei;  Ng'oma, A.]]></authors>
##     <affiliations><![CDATA[Inst. of Photonic Syst., Nat. Chiao Tung Univ., Tainan, Taiwan]]></affiliations>
##     <controlledterms>
##       <term><![CDATA[MIMO systems]]></term>
##       <term><![CDATA[OFDM modulation]]></term>
##       <term><![CDATA[least squares approximations]]></term>
##       <term><![CDATA[quadrature amplitude modulation]]></term>
##       <term><![CDATA[radio-over-fibre]]></term>
##     </controlledterms>
##     <thesaurusterms>
##       <term><![CDATA[Adaptive optics]]></term>
##       <term><![CDATA[MIMO]]></term>
##       <term><![CDATA[OFDM]]></term>
##       <term><![CDATA[Optical fibers]]></term>
##       <term><![CDATA[Optical receivers]]></term>
##       <term><![CDATA[Wireless communication]]></term>
##     </thesaurusterms>
##     <pubtitle><![CDATA[Photonics Journal, IEEE]]></pubtitle>
##     <punumber><![CDATA[4563994]]></punumber>
##     <pubtype><![CDATA[Journals & Magazines]]></pubtype>
##     <publisher><![CDATA[IEEE]]></publisher>
##     <volume><![CDATA[6]]></volume>
##     <issue><![CDATA[3]]></issue>
##     <py><![CDATA[2014]]></py>
##     <spage><![CDATA[1]]></spage>
##     <epage><![CDATA[7]]></epage>
##     <abstract><![CDATA[Multiple-input-multiple-output (MIMO) technology is a promising method to increase spectral efficiency in wireless communications. In this paper, a 60-GHz orthogonal frequency-division multiplexing radio over fiber (OFDM-RoF) system employing 2 &#x00D7; 2 MIMO wireless technology is demonstrated. With the proposed equalizer employing least mean squares (LMS) algorithm, MIMO channel mixing and I/Q-mismatch can be compensated simultaneously. 16-QAM and 32-QAM OFDM signal transmissions under forward error correction (FEC) threshold ( 1&#x00D7;10<sup>-3</sup>) are demonstrated. A data rate of 76.3 Gb/s can be achieved with a bit-loading algorithm over 25-km fiber transmission and 3.5-m wireless transmission.]]></abstract>
##     <issn><![CDATA[1943-0655]]></issn>
##     <htmlFlag><![CDATA[1]]></htmlFlag>
##     <arnumber><![CDATA[6762907]]></arnumber>
##     <doi><![CDATA[10.1109/JPHOT.2014.2310222]]></doi>
##     <publicationId><![CDATA[6762907]]></publicationId>
##     <mdurl><![CDATA[http://ieeexplore.ieee.org/xpl/articleDetails.jsp?tp=&arnumber=6762907&contentType=Journals+%26+Magazines]]></mdurl>
##     <pdf><![CDATA[http://ieeexplore.ieee.org/stamp/stamp.jsp?arnumber=6762907]]></pdf>
##   </document>
##   <document>
##     <rank>157</rank>
##     <title><![CDATA[Force Optimization of a Double-Sided Tubular Linear Induction Motor]]></title>
##     <authors><![CDATA[Musolino, A.;  Raugi, M.;  Rizzo, R.;  Tucci, M.]]></authors>
##     <affiliations><![CDATA[Dept. of Energy, Syst., Territory & Constr. Eng., Univ. of Pisa, Pisa, Italy]]></affiliations>
##     <controlledterms>
##       <term><![CDATA[current distribution]]></term>
##       <term><![CDATA[electric potential]]></term>
##       <term><![CDATA[evolutionary computation]]></term>
##       <term><![CDATA[finite element analysis]]></term>
##       <term><![CDATA[linear induction motors]]></term>
##       <term><![CDATA[magnetic flux]]></term>
##     </controlledterms>
##     <thesaurusterms>
##       <term><![CDATA[Force]]></term>
##       <term><![CDATA[Induction motors]]></term>
##       <term><![CDATA[Licenses]]></term>
##       <term><![CDATA[Optimization]]></term>
##       <term><![CDATA[Stator windings]]></term>
##       <term><![CDATA[Vectors]]></term>
##     </thesaurusterms>
##     <pubtitle><![CDATA[Magnetics, IEEE Transactions on]]></pubtitle>
##     <punumber><![CDATA[20]]></punumber>
##     <pubtype><![CDATA[Journals & Magazines]]></pubtype>
##     <publisher><![CDATA[IEEE]]></publisher>
##     <volume><![CDATA[50]]></volume>
##     <issue><![CDATA[12]]></issue>
##     <py><![CDATA[2014]]></py>
##     <spage><![CDATA[1]]></spage>
##     <epage><![CDATA[11]]></epage>
##     <abstract><![CDATA[An innovative double-sided tubular linear induction motor is presented, and its optimal design in terms of thrust force is discussed. A dedicated semi-analytical model of the device is developed allowing for fast and accurate evaluation of all the electromechanic quantities in the device, including the thrust force, back-electromotive force, distribution of the induced current, and average magnetic flux density in the teeth. This provides a basis for the design optimization, which has been performed by a novel evolutionary algorithm based on the self-organizing maps. Using the semi-analytical formulation, the characterization of the machine is greatly facilitated, thus allowing a fast evaluation of the cost function and design constraints. Finally, the obtained optimal design is validated by comparison with finite elements method analysis.]]></abstract>
##     <issn><![CDATA[0018-9464]]></issn>
##     <htmlFlag><![CDATA[1]]></htmlFlag>
##     <arnumber><![CDATA[6866164]]></arnumber>
##     <doi><![CDATA[10.1109/TMAG.2014.2343591]]></doi>
##     <publicationId><![CDATA[6866164]]></publicationId>
##     <mdurl><![CDATA[http://ieeexplore.ieee.org/xpl/articleDetails.jsp?tp=&arnumber=6866164&contentType=Journals+%26+Magazines]]></mdurl>
##     <pdf><![CDATA[http://ieeexplore.ieee.org/stamp/stamp.jsp?arnumber=6866164]]></pdf>
##   </document>
##   <document>
##     <rank>158</rank>
##     <title><![CDATA[High-Throughput Ionic Liquid Ion Sources Using Arrays of Microfabricated Electrospray Emitters With Integrated Extractor Grid and Carbon Nanotube Flow Control Structures]]></title>
##     <authors><![CDATA[Hill, F.A.;  Heubel, E.V.;  de Leon, P.P.;  Velasquez-Garcia, L.F.]]></authors>
##     <affiliations><![CDATA[Microsyst. Technol. Labs., Massachusetts Inst. of Technol., Cambridge, MA, USA]]></affiliations>
##     <controlledterms>
##       <term><![CDATA[carbon nanotubes]]></term>
##       <term><![CDATA[mass spectroscopy]]></term>
##       <term><![CDATA[microfabrication]]></term>
##       <term><![CDATA[sprays]]></term>
##     </controlledterms>
##     <thesaurusterms>
##       <term><![CDATA[Apertures]]></term>
##       <term><![CDATA[Liquids]]></term>
##       <term><![CDATA[Rough surfaces]]></term>
##       <term><![CDATA[Silicon]]></term>
##       <term><![CDATA[Surface impedance]]></term>
##       <term><![CDATA[Surface roughness]]></term>
##       <term><![CDATA[Surface treatment]]></term>
##     </thesaurusterms>
##     <pubtitle><![CDATA[Microelectromechanical Systems, Journal of]]></pubtitle>
##     <punumber><![CDATA[84]]></punumber>
##     <pubtype><![CDATA[Journals & Magazines]]></pubtype>
##     <publisher><![CDATA[IEEE]]></publisher>
##     <volume><![CDATA[23]]></volume>
##     <issue><![CDATA[5]]></issue>
##     <py><![CDATA[2014]]></py>
##     <spage><![CDATA[1237]]></spage>
##     <epage><![CDATA[1248]]></epage>
##     <abstract><![CDATA[We report the design, fabrication, and experimental characterization of dense, monolithic, and planar arrays of externally-fed electrospray emitters with an integrated extractor grid and carbon nanotube flow control structures for low-voltage and high-throughput electrospray of the ionic liquid EMI-BF4 in vacuum. Microfabricated arrays with as many as 1900 emitters in 1 cm<sup>2</sup> were fabricated and tested. Per-emitter currents as high as 5 &#x03BC;A in both polarities were measured, with start-up bias voltages as low as 470 V and extractor grid transmission as high as 80%. Maximum array emission currents of 1.35 mA (1.35 mA/cm<sup>2</sup>) were measured using arrays of 1900 emitters in 1 cm<sup>2</sup>. A conformal carbon nanotube forest grown on the surface of the emitters acts as a wicking structure that transports liquid to the emitter tips, providing hydraulic impedance to regulate and uniformize the emission across the array. Mass spectrometry of the electrospray beam confirms that emission in both polarities is composed of solvated ions, and etching of the silicon collector electrode is observed. Collector imprints and per-emitter current-voltage characteristics for different emitter array sizes spanning three orders of magnitude show excellent emission uniformity across the array. Performance estimates of the devices as nanosatellite thrusters are provided.]]></abstract>
##     <issn><![CDATA[1057-7157]]></issn>
##     <htmlFlag><![CDATA[1]]></htmlFlag>
##     <arnumber><![CDATA[6816025]]></arnumber>
##     <doi><![CDATA[10.1109/JMEMS.2014.2320509]]></doi>
##     <publicationId><![CDATA[6816025]]></publicationId>
##     <mdurl><![CDATA[http://ieeexplore.ieee.org/xpl/articleDetails.jsp?tp=&arnumber=6816025&contentType=Journals+%26+Magazines]]></mdurl>
##     <pdf><![CDATA[http://ieeexplore.ieee.org/stamp/stamp.jsp?arnumber=6816025]]></pdf>
##   </document>
##   <document>
##     <rank>159</rank>
##     <title><![CDATA[FPGA Security: Motivations, Features, and Applications]]></title>
##     <authors><![CDATA[Trimberger, S.M.;  Moore, J.J.]]></authors>
##     <affiliations><![CDATA[Xilinx, San Jose, CA, USA]]></affiliations>
##     <controlledterms>
##       <term><![CDATA[field programmable gate arrays]]></term>
##       <term><![CDATA[logic design]]></term>
##       <term><![CDATA[security of data]]></term>
##     </controlledterms>
##     <thesaurusterms>
##       <term><![CDATA[Computer security]]></term>
##       <term><![CDATA[Field programmable gate arrays]]></term>
##       <term><![CDATA[Integrated circuit modeling]]></term>
##       <term><![CDATA[Nonvolatile memory]]></term>
##       <term><![CDATA[Programming]]></term>
##       <term><![CDATA[Random access memory]]></term>
##       <term><![CDATA[Reverse engineering]]></term>
##     </thesaurusterms>
##     <pubtitle><![CDATA[Proceedings of the IEEE]]></pubtitle>
##     <punumber><![CDATA[5]]></punumber>
##     <pubtype><![CDATA[Journals & Magazines]]></pubtype>
##     <publisher><![CDATA[IEEE]]></publisher>
##     <volume><![CDATA[102]]></volume>
##     <issue><![CDATA[8]]></issue>
##     <py><![CDATA[2014]]></py>
##     <spage><![CDATA[1248]]></spage>
##     <epage><![CDATA[1265]]></epage>
##     <abstract><![CDATA[Since their inception, field-programmable gate arrays (FPGAs) have grown in capacity and complexity so that now FPGAs include millions of gates of logic, megabytes of memory, high-speed transceivers, analog interfaces, and whole multicore processors. Applications running in the FPGA include communications infrastructure, digital cinema, sensitive database access, critical industrial control, and high-performance signal processing. As the value of the applications and the data they handle have grown, so has the need to protect those applications and data. Motivated by specific threats, this paper describes FPGA security primitives from multiple FPGA vendors and gives examples of those primitives in use in applications.]]></abstract>
##     <issn><![CDATA[0018-9219]]></issn>
##     <htmlFlag><![CDATA[1]]></htmlFlag>
##     <arnumber><![CDATA[6849432]]></arnumber>
##     <doi><![CDATA[10.1109/JPROC.2014.2331672]]></doi>
##     <publicationId><![CDATA[6849432]]></publicationId>
##     <mdurl><![CDATA[http://ieeexplore.ieee.org/xpl/articleDetails.jsp?tp=&arnumber=6849432&contentType=Journals+%26+Magazines]]></mdurl>
##     <pdf><![CDATA[http://ieeexplore.ieee.org/stamp/stamp.jsp?arnumber=6849432]]></pdf>
##   </document>
##   <document>
##     <rank>160</rank>
##     <title><![CDATA[Resonant Absorption of TE-Polarized Light at the Surface of a Dielectric-Coated Metal Grating]]></title>
##     <authors><![CDATA[Wei Chen;  Ling Guo;  Zhijun Sun]]></authors>
##     <affiliations><![CDATA[Dept. of Phys., Xiamen Univ., Xiamen, China]]></affiliations>
##     <controlledterms>
##       <term><![CDATA[diffraction gratings]]></term>
##       <term><![CDATA[light polarisation]]></term>
##       <term><![CDATA[optical phase matching]]></term>
##       <term><![CDATA[optical waveguides]]></term>
##       <term><![CDATA[reflectivity]]></term>
##       <term><![CDATA[surface plasmon resonance]]></term>
##     </controlledterms>
##     <thesaurusterms>
##       <term><![CDATA[Absorption]]></term>
##       <term><![CDATA[Dielectrics]]></term>
##       <term><![CDATA[Gratings]]></term>
##       <term><![CDATA[Metals]]></term>
##       <term><![CDATA[Optical surface waves]]></term>
##       <term><![CDATA[Reflection]]></term>
##       <term><![CDATA[Surface waves]]></term>
##     </thesaurusterms>
##     <pubtitle><![CDATA[Photonics Journal, IEEE]]></pubtitle>
##     <punumber><![CDATA[4563994]]></punumber>
##     <pubtype><![CDATA[Journals & Magazines]]></pubtype>
##     <publisher><![CDATA[IEEE]]></publisher>
##     <volume><![CDATA[6]]></volume>
##     <issue><![CDATA[4]]></issue>
##     <py><![CDATA[2014]]></py>
##     <spage><![CDATA[1]]></spage>
##     <epage><![CDATA[6]]></epage>
##     <abstract><![CDATA[Mesoscopic interaction of transverse electric (TE)-polarized light with metal gratings requires a dielectric coating on the metal surfaces to support a waveguide mode for optical resonances simulating those of surface plasmon resonances for transverse magnetic (TM)-polarized light. Here, we show a resonance-induced absorption of TE-polarized light at a dielectric-coated metal grating. The resonance is identified to have a Bloch wave nature, existing only for its even-order modes, restricted by the coupling phase matching conditions in a reflection configuration mode. It is also shown that the resonance properties are largely subjected to the size of the grating grooves, where localized cavity mode can be excited for large-size grooves, leading to broad absorption peaks due to nonresonant losses and bringing in new absorption peaks resulting from cavity resonances in the grooves.]]></abstract>
##     <issn><![CDATA[1943-0655]]></issn>
##     <htmlFlag><![CDATA[1]]></htmlFlag>
##     <arnumber><![CDATA[6851868]]></arnumber>
##     <doi><![CDATA[10.1109/JPHOT.2014.2337893]]></doi>
##     <publicationId><![CDATA[6851868]]></publicationId>
##     <mdurl><![CDATA[http://ieeexplore.ieee.org/xpl/articleDetails.jsp?tp=&arnumber=6851868&contentType=Journals+%26+Magazines]]></mdurl>
##     <pdf><![CDATA[http://ieeexplore.ieee.org/stamp/stamp.jsp?arnumber=6851868]]></pdf>
##   </document>
##   <document>
##     <rank>161</rank>
##     <title><![CDATA[Toward a Dancing Robot With Listening Capability: Keypose-Based Integration of Lower-, Middle-, and Upper-Body Motions for Varying Music Tempos]]></title>
##     <authors><![CDATA[Okamoto, T.;  Shiratori, T.;  Kudoh, S.;  Nakaoka, S.;  Ikeuchi, K.]]></authors>
##     <affiliations><![CDATA[Univ. of Tokyo, Tokyo, Japan]]></affiliations>
##     <controlledterms>
##       <term><![CDATA[humanoid robots]]></term>
##       <term><![CDATA[mobile robots]]></term>
##       <term><![CDATA[music]]></term>
##     </controlledterms>
##     <thesaurusterms>
##       <term><![CDATA[Foot]]></term>
##       <term><![CDATA[Humanoid robots]]></term>
##       <term><![CDATA[Joints]]></term>
##       <term><![CDATA[Splines (mathematics)]]></term>
##       <term><![CDATA[Timing]]></term>
##       <term><![CDATA[Trajectory]]></term>
##     </thesaurusterms>
##     <pubtitle><![CDATA[Robotics, IEEE Transactions on]]></pubtitle>
##     <punumber><![CDATA[8860]]></punumber>
##     <pubtype><![CDATA[Journals & Magazines]]></pubtype>
##     <publisher><![CDATA[IEEE]]></publisher>
##     <volume><![CDATA[30]]></volume>
##     <issue><![CDATA[3]]></issue>
##     <py><![CDATA[2014]]></py>
##     <spage><![CDATA[771]]></spage>
##     <epage><![CDATA[778]]></epage>
##     <abstract><![CDATA[This paper presents the development toward a dancing robot that can listen to and dance along with musical performances. One of the key components of this robot is the ability to modify its dance motions with varying tempos, without exceeding motor limitations, in the same way that human dancers modify their motions. In this paper, we first observe human performances with varying musical tempos of the same musical piece, and then analyze human modification strategies. The analysis is conducted in terms of three body components: lower, middle, and upper bodies. We assume that these body components have different purposes and different modification strategies, respectively, for the performance of a dance. For all of the motions of these three components, we have found that certain fixed postures, which we call keyposes, tend to be preserved. Thus, this paper presents a method to create motions for robots at a certain music tempo, from human motion at an original music tempo, by using these keyposes. We have implemented these algorithms as an automatic process and validated their effectiveness by using a physical humanoid robot HRP-2. This robot succeeded in performing the Aizu-bandaisan dance, one of the Japanese traditional folk dances, 1.2 and 1.5 times faster than the tempo originally learned, while maintaining its physical constraints. Although we are not achieving a dancing robot which autonomously interacts with varying music tempos, we think that our method has a vital role in the dancing-to-music capability.]]></abstract>
##     <issn><![CDATA[1552-3098]]></issn>
##     <htmlFlag><![CDATA[1]]></htmlFlag>
##     <arnumber><![CDATA[6730703]]></arnumber>
##     <doi><![CDATA[10.1109/TRO.2014.2300212]]></doi>
##     <publicationId><![CDATA[6730703]]></publicationId>
##     <mdurl><![CDATA[http://ieeexplore.ieee.org/xpl/articleDetails.jsp?tp=&arnumber=6730703&contentType=Journals+%26+Magazines]]></mdurl>
##     <pdf><![CDATA[http://ieeexplore.ieee.org/stamp/stamp.jsp?arnumber=6730703]]></pdf>
##   </document>
##   <document>
##     <rank>162</rank>
##     <title><![CDATA[Dynamic Control of the Airy Plasmons in a Graphene Platform]]></title>
##     <authors><![CDATA[Yang, Y.;  Dai, H.T.;  Zhu, B.F.;  Sun, X.W.]]></authors>
##     <affiliations><![CDATA[Tianjin Key Lab. of Low Dimensional Mater. Phys. & Preparing Technol., Tianjin Univ., Tianjin, China]]></affiliations>
##     <controlledterms>
##       <term><![CDATA[graphene]]></term>
##       <term><![CDATA[surface conductivity]]></term>
##       <term><![CDATA[surface plasmons]]></term>
##     </controlledterms>
##     <thesaurusterms>
##       <term><![CDATA[Conductivity]]></term>
##       <term><![CDATA[Educational institutions]]></term>
##       <term><![CDATA[Graphene]]></term>
##       <term><![CDATA[Plasmons]]></term>
##       <term><![CDATA[Surface waves]]></term>
##       <term><![CDATA[Trajectory]]></term>
##       <term><![CDATA[Tuning]]></term>
##     </thesaurusterms>
##     <pubtitle><![CDATA[Photonics Journal, IEEE]]></pubtitle>
##     <punumber><![CDATA[4563994]]></punumber>
##     <pubtype><![CDATA[Journals & Magazines]]></pubtype>
##     <publisher><![CDATA[IEEE]]></publisher>
##     <volume><![CDATA[6]]></volume>
##     <issue><![CDATA[4]]></issue>
##     <py><![CDATA[2014]]></py>
##     <spage><![CDATA[1]]></spage>
##     <epage><![CDATA[7]]></epage>
##     <abstract><![CDATA[We propose to use a 1-atom-thick structure, i.e., a single graphene sheet for the dynamic control of an Airy plasmon (AiP). The graphene layer serves not only as the guiding medium but also as the modulator for the AiP. By altering the external biased voltage, the effective mode index for surface plasmon waves can be modified. Consequently, the deflection and the propagation distance of the AiP are controlled dynamically. Due to the advantages of graphene plasmons, graphene AiPs may lead to compact and flexible AiP devices. This paper could be also beneficial to relevant applications such as tunable plasmonic optical routing and on-chip signal processing.]]></abstract>
##     <issn><![CDATA[1943-0655]]></issn>
##     <htmlFlag><![CDATA[1]]></htmlFlag>
##     <arnumber><![CDATA[6837423]]></arnumber>
##     <doi><![CDATA[10.1109/JPHOT.2014.2331248]]></doi>
##     <publicationId><![CDATA[6837423]]></publicationId>
##     <mdurl><![CDATA[http://ieeexplore.ieee.org/xpl/articleDetails.jsp?tp=&arnumber=6837423&contentType=Journals+%26+Magazines]]></mdurl>
##     <pdf><![CDATA[http://ieeexplore.ieee.org/stamp/stamp.jsp?arnumber=6837423]]></pdf>
##   </document>
##   <document>
##     <rank>163</rank>
##     <title><![CDATA[Investigation of Bending Sensitivity in Partially Doped Core Fiber for Sensing Applications]]></title>
##     <authors><![CDATA[Emami, S.D.;  Abdul-Rashid, H.A.;  Zahedi, F.Z.;  Paul, M.C.;  Das, S.;  Pal, M.;  Harun, S.W.]]></authors>
##     <affiliations><![CDATA[Dept. of Electr. Eng., Univ. of Malaya, Kuala Lumpur, Malaysia]]></affiliations>
##     <controlledterms>
##       <term><![CDATA[bending]]></term>
##       <term><![CDATA[fibre lasers]]></term>
##       <term><![CDATA[fibre optic sensors]]></term>
##       <term><![CDATA[optical design techniques]]></term>
##       <term><![CDATA[optical fibre fabrication]]></term>
##       <term><![CDATA[optical fibre filters]]></term>
##       <term><![CDATA[optical fibre losses]]></term>
##       <term><![CDATA[superradiance]]></term>
##     </controlledterms>
##     <thesaurusterms>
##       <term><![CDATA[Doping]]></term>
##       <term><![CDATA[Erbium]]></term>
##       <term><![CDATA[Optical fiber amplifiers]]></term>
##       <term><![CDATA[Optical fiber sensors]]></term>
##       <term><![CDATA[Refractive index]]></term>
##     </thesaurusterms>
##     <pubtitle><![CDATA[Sensors Journal, IEEE]]></pubtitle>
##     <punumber><![CDATA[7361]]></punumber>
##     <pubtype><![CDATA[Journals & Magazines]]></pubtype>
##     <publisher><![CDATA[IEEE]]></publisher>
##     <volume><![CDATA[14]]></volume>
##     <issue><![CDATA[4]]></issue>
##     <py><![CDATA[2014]]></py>
##     <spage><![CDATA[1295]]></spage>
##     <epage><![CDATA[1303]]></epage>
##     <abstract><![CDATA[A fiber based high sensitive bend sensor is proposed and demonstrated using a uniquely designed partially doped core fiber (PDCF). The fabrication method of PDCF with two core regions, namely an undoped outer region with a diameter of ~ 9.5 &#x03BC;m encompassing a doped, inner core region with a diameter of 4.00 &#x03BC;m is explained. The mechanism of bending effect in proposed PDCF and the experimental setup for amplified spontaneous emission (ASE) based sensor and fiber laser based sensor is illustrated. For ASE sensor, the higher ASE power level loss as the spooling radius is reduced from 20 to 3 cm is measured. The gain peak shift to shorter wavelength with respect to the decrease of the spooling radius from 20 to 3 cm due to higher bending loss at smaller bending radius is observed. The results are in agreement with overlap factor variation of PDCF. As expected from ASE peaks variation, the fiber laser sensor spectral operation is changed from 1539 to 1530 nm range. This phenomenon is due to higher mode field diameter of longer wavelength and result of optical filtering at longer wavelengths. The experimental results showed the output of the ASE is also highly stable, with no observable variation in the power output over a measurement period of 1 h. The PDCF is also temperature insensitive.]]></abstract>
##     <issn><![CDATA[1530-437X]]></issn>
##     <htmlFlag><![CDATA[1]]></htmlFlag>
##     <arnumber><![CDATA[6678756]]></arnumber>
##     <doi><![CDATA[10.1109/JSEN.2013.2294244]]></doi>
##     <publicationId><![CDATA[6678756]]></publicationId>
##     <mdurl><![CDATA[http://ieeexplore.ieee.org/xpl/articleDetails.jsp?tp=&arnumber=6678756&contentType=Journals+%26+Magazines]]></mdurl>
##     <pdf><![CDATA[http://ieeexplore.ieee.org/stamp/stamp.jsp?arnumber=6678756]]></pdf>
##   </document>
##   <document>
##     <rank>164</rank>
##     <title><![CDATA[An Experimental Study of Scale-up, Oxidant, and Response Characteristics in PEM Fuel Cells]]></title>
##     <authors><![CDATA[Kap-Seung Choi;  Jiwoong Ahn;  Jungkoo Lee;  Nguyen Duy Vinh;  Hyung-Man Kim;  Kiwon Park;  Gunyong Hwang]]></authors>
##     <affiliations><![CDATA[Dept. of Automobile Eng., Tongmyong Univ., Busan, South Korea]]></affiliations>
##     <controlledterms>
##       <term><![CDATA[oxidation]]></term>
##       <term><![CDATA[proton exchange membrane fuel cells]]></term>
##     </controlledterms>
##     <thesaurusterms>
##       <term><![CDATA[Anodes]]></term>
##       <term><![CDATA[Cathodes]]></term>
##       <term><![CDATA[Density measurement]]></term>
##       <term><![CDATA[Fuel cells]]></term>
##       <term><![CDATA[Power system measurements]]></term>
##       <term><![CDATA[Resistance]]></term>
##     </thesaurusterms>
##     <pubtitle><![CDATA[Energy Conversion, IEEE Transactions on]]></pubtitle>
##     <punumber><![CDATA[60]]></punumber>
##     <pubtype><![CDATA[Journals & Magazines]]></pubtype>
##     <publisher><![CDATA[IEEE]]></publisher>
##     <volume><![CDATA[29]]></volume>
##     <issue><![CDATA[3]]></issue>
##     <py><![CDATA[2014]]></py>
##     <spage><![CDATA[727]]></spage>
##     <epage><![CDATA[734]]></epage>
##     <abstract><![CDATA[It is important for fuel cell to become flexible in order to accommodate the situations of application areas for reducing costs and expanding coverage. Scale ups in series and parallel to 1-kW class proton exchange membrane (PEM) fuel cell stack are characterized experimentally with oxidant as air and oxygen. Through a serial scaled-up 4-cell stack with an active area of 25 cm<sup>2</sup> and a parallel scaled-up single cell with an active area of 100 cm<sup>2</sup>, respectively, the 1-kW class 20-cell stack with an active area of 150 cm<sup>2</sup> per cell is scaled up from the basic unit cell with maintaining their performances. The polarization and power curves of the 1-kW class PEM fuel cell stack with the reactants H<sub>2</sub>/air and H<sub>2</sub>/O<sub>2</sub> are evaluated. Sufficient power of the PEM fuel cell can operate at flexible nominal current of 40-80 A and nominal voltage of 12-15 V for providing durability and balance between cells. The 1-kW class PEM fuel cell stack is characterized through the comparison of cell voltage differences utilizing H<sub>2</sub>/air and H<sub>2</sub>/O<sub>2</sub>. Both cell voltages with H<sub>2</sub>/air and H<sub>2</sub>/O<sub>2</sub> follow well the one-twentieth of the responding stack voltages with the small difference.]]></abstract>
##     <issn><![CDATA[0885-8969]]></issn>
##     <htmlFlag><![CDATA[1]]></htmlFlag>
##     <arnumber><![CDATA[6823705]]></arnumber>
##     <doi><![CDATA[10.1109/TEC.2014.2322877]]></doi>
##     <publicationId><![CDATA[6823705]]></publicationId>
##     <mdurl><![CDATA[http://ieeexplore.ieee.org/xpl/articleDetails.jsp?tp=&arnumber=6823705&contentType=Journals+%26+Magazines]]></mdurl>
##     <pdf><![CDATA[http://ieeexplore.ieee.org/stamp/stamp.jsp?arnumber=6823705]]></pdf>
##   </document>
##   <document>
##     <rank>165</rank>
##     <title><![CDATA[Optical Tolerance Analysis of the Multi-Beam Limb Viewing Instrument STEAMR]]></title>
##     <authors><![CDATA[Hammar, A.;  Whale, M.;  Forsberg, P.;  Murk, A.;  Emrich, A.;  Stake, J.]]></authors>
##     <affiliations><![CDATA[Dept. of Microtechnol. & Nanosci.-MC2, Chalmers Univ. of Technol., Goteborg, Sweden]]></affiliations>
##     <controlledterms>
##       <term><![CDATA[Monte Carlo methods]]></term>
##       <term><![CDATA[deformation]]></term>
##       <term><![CDATA[focal planes]]></term>
##       <term><![CDATA[optical elements]]></term>
##       <term><![CDATA[optical receivers]]></term>
##       <term><![CDATA[ray tracing]]></term>
##       <term><![CDATA[tolerance analysis]]></term>
##     </controlledterms>
##     <thesaurusterms>
##       <term><![CDATA[Adaptive optics]]></term>
##       <term><![CDATA[Integrated optics]]></term>
##       <term><![CDATA[Monte Carlo methods]]></term>
##       <term><![CDATA[Optical distortion]]></term>
##       <term><![CDATA[Optical receivers]]></term>
##       <term><![CDATA[Submillimeter wave devices]]></term>
##       <term><![CDATA[Tolerance analysis]]></term>
##     </thesaurusterms>
##     <pubtitle><![CDATA[Terahertz Science and Technology, IEEE Transactions on]]></pubtitle>
##     <punumber><![CDATA[5503871]]></punumber>
##     <pubtype><![CDATA[Journals & Magazines]]></pubtype>
##     <publisher><![CDATA[IEEE]]></publisher>
##     <volume><![CDATA[4]]></volume>
##     <issue><![CDATA[6]]></issue>
##     <py><![CDATA[2014]]></py>
##     <spage><![CDATA[714]]></spage>
##     <epage><![CDATA[721]]></epage>
##     <abstract><![CDATA[In this paper, we report on an optical tolerance analysis of the submillimeter atmospheric multi-beam limb sounder, STEAMR. Physical optics and ray-tracing methods were used to quantify and separate errors in beam pointing and distortion due to reflector misalignment and primary reflector surface deformations. Simulations were performed concurrently with the manufacturing of a multi-beam demonstrator of the relay optical system which shapes and images the beams to their corresponding receiver feed horns. Results from Monte Carlo simulations show that the inserts used for reflector mounting should be positioned with an overall accuracy better than 100 &#x03BC;m (~ 1/10 wavelength). Analyses of primary reflector surface deformations show that a deviation of magnitude 100 &#x03BC;m can be tolerable before deployment, whereas the corresponding variations should be less than 30 &#x03BC;m during operation. The most sensitive optical elements in terms of misalignments are found near the focal plane. This localized sensitivity is attributed to the off-axis nature of the beams at this location. Post-assembly mechanical measurements of the reflectors in the demonstrator show that alignment better than 50 &#x03BC;m could be obtained.]]></abstract>
##     <issn><![CDATA[2156-342X]]></issn>
##     <htmlFlag><![CDATA[1]]></htmlFlag>
##     <arnumber><![CDATA[6923496]]></arnumber>
##     <doi><![CDATA[10.1109/TTHZ.2014.2361616]]></doi>
##     <publicationId><![CDATA[6923496]]></publicationId>
##     <mdurl><![CDATA[http://ieeexplore.ieee.org/xpl/articleDetails.jsp?tp=&arnumber=6923496&contentType=Journals+%26+Magazines]]></mdurl>
##     <pdf><![CDATA[http://ieeexplore.ieee.org/stamp/stamp.jsp?arnumber=6923496]]></pdf>
##   </document>
##   <document>
##     <rank>166</rank>
##     <title><![CDATA[Coherent Versus Non-Coherent Reconfigurable Antenna Aided Virtual MIMO Systems]]></title>
##     <authors><![CDATA[Sugiura, S.]]></authors>
##     <affiliations><![CDATA[Dept. of Comput. & Inf. Sci., Tokyo Univ. of Agric. & Technol., Koganei, Japan]]></affiliations>
##     <controlledterms>
##       <term><![CDATA[MIMO communication]]></term>
##       <term><![CDATA[antenna arrays]]></term>
##       <term><![CDATA[channel estimation]]></term>
##       <term><![CDATA[radio transmitters]]></term>
##       <term><![CDATA[signal detection]]></term>
##     </controlledterms>
##     <thesaurusterms>
##       <term><![CDATA[Antenna arrays]]></term>
##       <term><![CDATA[Dispersion]]></term>
##       <term><![CDATA[MIMO]]></term>
##       <term><![CDATA[Receivers]]></term>
##       <term><![CDATA[Transmitting antennas]]></term>
##     </thesaurusterms>
##     <pubtitle><![CDATA[Signal Processing Letters, IEEE]]></pubtitle>
##     <punumber><![CDATA[97]]></punumber>
##     <pubtype><![CDATA[Journals & Magazines]]></pubtype>
##     <publisher><![CDATA[IEEE]]></publisher>
##     <volume><![CDATA[21]]></volume>
##     <issue><![CDATA[4]]></issue>
##     <py><![CDATA[2014]]></py>
##     <spage><![CDATA[390]]></spage>
##     <epage><![CDATA[394]]></epage>
##     <abstract><![CDATA[In this letter, motivated by the recent universal encoding concept of space-time shift keying (STSK), we propose a novel single radio-frequency (RF) based virtual multiple-input multiple-output (VMIMO) architecture, employing a pattern reconfigurable antenna at a transmitter. More specifically, the proposed scheme is based on the concept of pattern-time dispersion matrix activation, which is capable of striking a flexible balance between the transmission rate and the diversity gain, while enabling symbol-based low-complexity detection. Furthermore, we conceive its non-coherently detected counterpart in order to dispense with any channel estimation as well as additional pilot overhead, hence having the potential of outperforming the conventional coherently-detected VMIMO schemes in many practical scenarios.]]></abstract>
##     <issn><![CDATA[1070-9908]]></issn>
##     <htmlFlag><![CDATA[1]]></htmlFlag>
##     <arnumber><![CDATA[6728666]]></arnumber>
##     <doi><![CDATA[10.1109/LSP.2014.2303471]]></doi>
##     <publicationId><![CDATA[6728666]]></publicationId>
##     <mdurl><![CDATA[http://ieeexplore.ieee.org/xpl/articleDetails.jsp?tp=&arnumber=6728666&contentType=Journals+%26+Magazines]]></mdurl>
##     <pdf><![CDATA[http://ieeexplore.ieee.org/stamp/stamp.jsp?arnumber=6728666]]></pdf>
##   </document>
##   <document>
##     <rank>167</rank>
##     <title><![CDATA[A Dynamic QoS-Aware Logistics Service Composition Algorithm Based on Social Network]]></title>
##     <authors><![CDATA[Yang Yu;  Jian Chen;  Shangquan Lin;  Ying Wang]]></authors>
##     <affiliations><![CDATA[Sch. of Software, Sun Yat-sen Univ., Guangzhou, China]]></affiliations>
##     <controlledterms>
##       <term><![CDATA[Web services]]></term>
##       <term><![CDATA[logistics data processing]]></term>
##       <term><![CDATA[quality of service]]></term>
##       <term><![CDATA[social networking (online)]]></term>
##     </controlledterms>
##     <thesaurusterms>
##       <term><![CDATA[Algorithm design and analysis]]></term>
##       <term><![CDATA[Cities and towns]]></term>
##       <term><![CDATA[Heuristic algorithms]]></term>
##       <term><![CDATA[Logistics]]></term>
##       <term><![CDATA[Optimization]]></term>
##       <term><![CDATA[Quality of service]]></term>
##       <term><![CDATA[Social network services]]></term>
##     </thesaurusterms>
##     <pubtitle><![CDATA[Emerging Topics in Computing, IEEE Transactions on]]></pubtitle>
##     <punumber><![CDATA[6245516]]></punumber>
##     <pubtype><![CDATA[Journals & Magazines]]></pubtype>
##     <publisher><![CDATA[IEEE]]></publisher>
##     <volume><![CDATA[2]]></volume>
##     <issue><![CDATA[4]]></issue>
##     <py><![CDATA[2014]]></py>
##     <spage><![CDATA[399]]></spage>
##     <epage><![CDATA[410]]></epage>
##     <abstract><![CDATA[The public logistics platform aims to provide customers with end-to-end logistics services by finding and composing a huge quantity of web services from logistics service providers. But, traditional service composition required predefined business process so that its flexibility is far from satisfactory in the problem. Path planning can be a solution of finding a suitable business path during service composition, but the search space will increase dramatically with the growth of service quantity and is hard to get a result within a tolerable interaction time. In the context of big data, to quickly build a service path with the optimal global QoS has become a problem demanding prompt solution. Sociologists point out that companies prefer familiar partners in the commercial environment. Using this principle, a concept of partner circle is defined, which can significantly reduce the search space in path planning. Combining path planning with service composition, a PartnerFirst algorithm is presented based on the social network, which is the cooperation network of service providers here. Simulation experiment shows that the PartnerFirst algorithm outperforms current approaches over 10 times in efficiency, with just about 10% loss in QoS. The relationship between efficiency and service quantity of the PartnerFirst algorithm is nearly linear. It proves that using social network in dynamic service composition is efficient and effective.]]></abstract>
##     <issn><![CDATA[2168-6750]]></issn>
##     <htmlFlag><![CDATA[1]]></htmlFlag>
##     <arnumber><![CDATA[6797876]]></arnumber>
##     <doi><![CDATA[10.1109/TETC.2014.2316524]]></doi>
##     <publicationId><![CDATA[6797876]]></publicationId>
##     <mdurl><![CDATA[http://ieeexplore.ieee.org/xpl/articleDetails.jsp?tp=&arnumber=6797876&contentType=Journals+%26+Magazines]]></mdurl>
##     <pdf><![CDATA[http://ieeexplore.ieee.org/stamp/stamp.jsp?arnumber=6797876]]></pdf>
##   </document>
##   <document>
##     <rank>168</rank>
##     <title><![CDATA[A Combined Design-Time/Test-Time Study of the Vulnerability of Sub-Threshold Devices to Low Voltage Fault Attacks]]></title>
##     <authors><![CDATA[Barenghi, A.;  Hocquet, C.;  Bol, D.;  Standaert, F.-X.;  Regazzoni, F.;  Koren, I.]]></authors>
##     <affiliations><![CDATA[Dept. of Electron., Inf. & Biotechnol., Politec. di Milano, Milan, Italy]]></affiliations>
##     <controlledterms>
##       <term><![CDATA[VLSI]]></term>
##       <term><![CDATA[cryptography]]></term>
##       <term><![CDATA[electronic design automation]]></term>
##       <term><![CDATA[integrated circuit design]]></term>
##       <term><![CDATA[low-power electronics]]></term>
##     </controlledterms>
##     <thesaurusterms>
##       <term><![CDATA[CMOS integrated circuits]]></term>
##       <term><![CDATA[Ciphers]]></term>
##       <term><![CDATA[Circuit faults]]></term>
##       <term><![CDATA[Encryption]]></term>
##       <term><![CDATA[Radiofrequency identification]]></term>
##       <term><![CDATA[Standards]]></term>
##       <term><![CDATA[Very large scale integration]]></term>
##     </thesaurusterms>
##     <pubtitle><![CDATA[Emerging Topics in Computing, IEEE Transactions on]]></pubtitle>
##     <punumber><![CDATA[6245516]]></punumber>
##     <pubtype><![CDATA[Journals & Magazines]]></pubtype>
##     <publisher><![CDATA[IEEE]]></publisher>
##     <volume><![CDATA[2]]></volume>
##     <issue><![CDATA[2]]></issue>
##     <py><![CDATA[2014]]></py>
##     <spage><![CDATA[107]]></spage>
##     <epage><![CDATA[118]]></epage>
##     <abstract><![CDATA[The continuous scaling of VLSI technology and the possibility to run circuits in subthreshold voltage range make it possible to implement standard cryptographic primitives within the very limited circuit and power budget of radio frequency identification (RFID) devices. However, such cryptographic implementations raise concerns regarding their vulnerability to both active and passive side-channel attacks. In particular, when focusing on RFID targeted designs, it is important to evaluate their resistance against low-cost physical attacks. A low-cost fault injection attack can be mounted, for example, by lowering the supply voltage of the chip with the goal of causing setup time violations. In this paper, we provide an in-depth characterization of a chip implementation of the AES cipher. The chip has been designed using a 65-nm low-power standard cell library and operates in a subthreshold voltage range. We first show that it is possible to inject faults (through lowering the supply voltage) compliant with the fault models required to perform attacks against the AES cipher. We then investigate the possibility of predicting, at design time, which parts of the chip are more likely to be sensitive to such fault injection attacks and produce the desirable (from the point of view of the attacker) faulty behavior. Identifying such sensitive logic signals allows us to suggest to the designer a tailored countermeasure strategy for thwarting these attacks, with a minimal impact on the circuit's performance.]]></abstract>
##     <issn><![CDATA[2168-6750]]></issn>
##     <htmlFlag><![CDATA[1]]></htmlFlag>
##     <arnumber><![CDATA[6797908]]></arnumber>
##     <doi><![CDATA[10.1109/TETC.2014.2316509]]></doi>
##     <publicationId><![CDATA[6797908]]></publicationId>
##     <mdurl><![CDATA[http://ieeexplore.ieee.org/xpl/articleDetails.jsp?tp=&arnumber=6797908&contentType=Journals+%26+Magazines]]></mdurl>
##     <pdf><![CDATA[http://ieeexplore.ieee.org/stamp/stamp.jsp?arnumber=6797908]]></pdf>
##   </document>
##   <document>
##     <rank>169</rank>
##     <title><![CDATA[GridStore: A Puzzle-Based Storage System With Decentralized Control]]></title>
##     <authors><![CDATA[Gue, K.R.;  Furmans, K.;  Seibold, Z.;  Uludag, O.]]></authors>
##     <affiliations><![CDATA[Ind. & Syst. Eng., Auburn Univ., Auburn, AL, USA]]></affiliations>
##     <controlledterms>
##       <term><![CDATA[containers]]></term>
##       <term><![CDATA[conveyors]]></term>
##       <term><![CDATA[decentralised control]]></term>
##       <term><![CDATA[storage automation]]></term>
##     </controlledterms>
##     <thesaurusterms>
##       <term><![CDATA[Centralized control]]></term>
##       <term><![CDATA[Decentralized control]]></term>
##       <term><![CDATA[Materials]]></term>
##       <term><![CDATA[Materials handling]]></term>
##       <term><![CDATA[Message passing]]></term>
##       <term><![CDATA[Protocols]]></term>
##       <term><![CDATA[System recovery]]></term>
##     </thesaurusterms>
##     <pubtitle><![CDATA[Automation Science and Engineering, IEEE Transactions on]]></pubtitle>
##     <punumber><![CDATA[8856]]></punumber>
##     <pubtype><![CDATA[Journals & Magazines]]></pubtype>
##     <publisher><![CDATA[IEEE]]></publisher>
##     <volume><![CDATA[11]]></volume>
##     <issue><![CDATA[2]]></issue>
##     <py><![CDATA[2014]]></py>
##     <spage><![CDATA[429]]></spage>
##     <epage><![CDATA[438]]></epage>
##     <abstract><![CDATA[We describe a high-density storage system for physical goods in which identical conveyor modules can be plugged together to store and retrieve unit-loads or small containers. Material movement conforms to the &#x201C;puzzle architecture&#x201D; found in popular board games such as the 15-puzzle and Rush Hour. Control of the system is decentralized, meaning that each module contains identical operating logic that directs its behavior based on local conditions and message passing. We prove the system deadlock-free and show its performance under a wide variety of operating configurations.]]></abstract>
##     <issn><![CDATA[1545-5955]]></issn>
##     <htmlFlag><![CDATA[1]]></htmlFlag>
##     <arnumber><![CDATA[6601725]]></arnumber>
##     <doi><![CDATA[10.1109/TASE.2013.2278252]]></doi>
##     <publicationId><![CDATA[6601725]]></publicationId>
##     <mdurl><![CDATA[http://ieeexplore.ieee.org/xpl/articleDetails.jsp?tp=&arnumber=6601725&contentType=Journals+%26+Magazines]]></mdurl>
##     <pdf><![CDATA[http://ieeexplore.ieee.org/stamp/stamp.jsp?arnumber=6601725]]></pdf>
##   </document>
##   <document>
##     <rank>170</rank>
##     <title><![CDATA[Optimal Energy-Efficient Relay Deployment for the Bidirectional Relay Transmission Schemes]]></title>
##     <authors><![CDATA[Qimei Cui;  Xianjun Yang;  Ha&#x0308; ma&#x0308; la&#x0308; inen, J.;  Xiaofeng Tao;  Ping Zhang]]></authors>
##     <affiliations><![CDATA[Key Lab. of Universal Wireless Commun., Beijing Univ. of Posts & Telecommun., Beijing, China]]></affiliations>
##     <controlledterms>
##       <term><![CDATA[fading channels]]></term>
##       <term><![CDATA[power amplifiers]]></term>
##       <term><![CDATA[radio networks]]></term>
##       <term><![CDATA[relay networks (telecommunication)]]></term>
##       <term><![CDATA[telecommunication power management]]></term>
##       <term><![CDATA[telecommunication traffic]]></term>
##     </controlledterms>
##     <thesaurusterms>
##       <term><![CDATA[Downlink]]></term>
##       <term><![CDATA[Energy consumption]]></term>
##       <term><![CDATA[Fading]]></term>
##       <term><![CDATA[Licenses]]></term>
##       <term><![CDATA[Power demand]]></term>
##       <term><![CDATA[Relays]]></term>
##       <term><![CDATA[Uplink]]></term>
##     </thesaurusterms>
##     <pubtitle><![CDATA[Vehicular Technology, IEEE Transactions on]]></pubtitle>
##     <punumber><![CDATA[25]]></punumber>
##     <pubtype><![CDATA[Journals & Magazines]]></pubtype>
##     <publisher><![CDATA[IEEE]]></publisher>
##     <volume><![CDATA[63]]></volume>
##     <issue><![CDATA[6]]></issue>
##     <py><![CDATA[2014]]></py>
##     <spage><![CDATA[2625]]></spage>
##     <epage><![CDATA[2641]]></epage>
##     <abstract><![CDATA[Recently, the energy efficiency of a relay network has become a hot research topic in the wireless communication society. In this paper, we investigate the energy efficiency of three basic bidirectional relay transmission schemes [i.e., the four time-slot (4TS), three time-slot (3TS), and two time-slot (2TS) schemes] from the angle of relay deployment. Since a realistic power consumption model is very important in analyzing energy efficiency, and a power amplifier (PA) consumes up to 70% of the total power, we consider a realistic nonideal PA model. The derived closed-form expressions for the optimal relay deployment and the simulation results reveal the following important conclusions. First, it is possible to achieve the optimal energy efficiency and enlarge the cell coverage simultaneously in bad channel conditions, but it may be very challenging in good channel conditions. Second, under asymmetric traffic conditions, particularly when the downlink rate is larger than the uplink rate, all the aforementioned three schemes have almost the same optimal relay deployment, but the 2TS scheme has the highest energy efficiency when the spectral efficiency is large. Third, the relay node should be deployed closer to the base station with the nonideal PA than that with the ideal PA, and the optimal energy efficiency with the nonideal PA is much higher than that with the ideal PA. Moreover, the impact of small-scale fading depends on the value of path loss. To overcome the small-scale fading, the relay network needs to consume more energy.]]></abstract>
##     <issn><![CDATA[0018-9545]]></issn>
##     <htmlFlag><![CDATA[1]]></htmlFlag>
##     <arnumber><![CDATA[6689323]]></arnumber>
##     <doi><![CDATA[10.1109/TVT.2013.2295413]]></doi>
##     <publicationId><![CDATA[6689323]]></publicationId>
##     <mdurl><![CDATA[http://ieeexplore.ieee.org/xpl/articleDetails.jsp?tp=&arnumber=6689323&contentType=Journals+%26+Magazines]]></mdurl>
##     <pdf><![CDATA[http://ieeexplore.ieee.org/stamp/stamp.jsp?arnumber=6689323]]></pdf>
##   </document>
##   <document>
##     <rank>171</rank>
##     <title><![CDATA[Reliability and Secrecy Functions of the Wiretap Channel Under Cost Constraint]]></title>
##     <authors><![CDATA[Te Sun Han;  Endo, H.;  Sasaki, M.]]></authors>
##     <affiliations><![CDATA[Quantum ICT Lab., Nat. Inst. of Inf. & Commun. Technol., Tokyo, Japan]]></affiliations>
##     <controlledterms>
##       <term><![CDATA[Gaussian processes]]></term>
##       <term><![CDATA[telecommunication channels]]></term>
##       <term><![CDATA[telecommunication network reliability]]></term>
##       <term><![CDATA[telecommunication security]]></term>
##     </controlledterms>
##     <thesaurusterms>
##       <term><![CDATA[Channel models]]></term>
##       <term><![CDATA[Decoding]]></term>
##       <term><![CDATA[Manganese]]></term>
##       <term><![CDATA[Measurement uncertainty]]></term>
##       <term><![CDATA[Probability distribution]]></term>
##       <term><![CDATA[Reliability theory]]></term>
##     </thesaurusterms>
##     <pubtitle><![CDATA[Information Theory, IEEE Transactions on]]></pubtitle>
##     <punumber><![CDATA[18]]></punumber>
##     <pubtype><![CDATA[Journals & Magazines]]></pubtype>
##     <publisher><![CDATA[IEEE]]></publisher>
##     <volume><![CDATA[60]]></volume>
##     <issue><![CDATA[11]]></issue>
##     <py><![CDATA[2014]]></py>
##     <spage><![CDATA[6819]]></spage>
##     <epage><![CDATA[6843]]></epage>
##     <abstract><![CDATA[The wiretap channel has been devised and studied first by Wyner, and subsequently extended to the case with nondegraded general wiretap channels by Csisza&#x0301;r and Ko&#x0308;rner. Focusing mainly on the stationary memoryless channel with cost constraint, we newly introduce the notion of reliability and secrecy functions as a fundamental tool to analyze and/or design the performance of an efficient wiretap channel system, including binary symmetric wiretap channels, Poisson wiretap channels, and Gaussian wiretap channels. Compact formulas for those functions are explicitly given for stationary memoryless wiretap channels. It is also demonstrated that, based on such a pair of reliability and secrecy functions, we can control the tradeoff between reliability and secrecy (usually conflicting), both with exponentially decreasing rates as block length (n) becomes large. Four ways to do so are given on the basis of rate shifting, rate exchange, concatenation, and change of cost constraint. In addition, the notion of the (delta ) secrecy capacity is defined and shown to attain the strongest secrecy standard among others. The maximized versus averaged secrecy measures is also discussed.]]></abstract>
##     <issn><![CDATA[0018-9448]]></issn>
##     <htmlFlag><![CDATA[1]]></htmlFlag>
##     <arnumber><![CDATA[6894161]]></arnumber>
##     <doi><![CDATA[10.1109/TIT.2014.2355811]]></doi>
##     <publicationId><![CDATA[6894161]]></publicationId>
##     <mdurl><![CDATA[http://ieeexplore.ieee.org/xpl/articleDetails.jsp?tp=&arnumber=6894161&contentType=Journals+%26+Magazines]]></mdurl>
##     <pdf><![CDATA[http://ieeexplore.ieee.org/stamp/stamp.jsp?arnumber=6894161]]></pdf>
##   </document>
##   <document>
##     <rank>172</rank>
##     <title><![CDATA[FEM Optimization of Energy Density in Tumor Hyperthermia Using Time-Dependent Magnetic Nanoparticle Power Dissipation]]></title>
##     <authors><![CDATA[Koch, C.M.;  Winfrey, A.L.]]></authors>
##     <affiliations><![CDATA[Dept. of Eng. Sci. & Mech., Virginia Polytech. Inst. & State Univ., Blacksburg, VA, USA]]></affiliations>
##     <controlledterms>
##       <term><![CDATA[finite element analysis]]></term>
##       <term><![CDATA[hyperthermia]]></term>
##       <term><![CDATA[magnetic particles]]></term>
##       <term><![CDATA[nanomagnetics]]></term>
##       <term><![CDATA[nanomedicine]]></term>
##       <term><![CDATA[nanoparticles]]></term>
##       <term><![CDATA[patient treatment]]></term>
##       <term><![CDATA[tumours]]></term>
##     </controlledterms>
##     <thesaurusterms>
##       <term><![CDATA[Finite element analysis]]></term>
##       <term><![CDATA[Heating]]></term>
##       <term><![CDATA[Hyperthermia]]></term>
##       <term><![CDATA[Mathematical model]]></term>
##       <term><![CDATA[Nanoparticles]]></term>
##       <term><![CDATA[Power dissipation]]></term>
##       <term><![CDATA[Tumors]]></term>
##     </thesaurusterms>
##     <pubtitle><![CDATA[Magnetics, IEEE Transactions on]]></pubtitle>
##     <punumber><![CDATA[20]]></punumber>
##     <pubtype><![CDATA[Journals & Magazines]]></pubtype>
##     <publisher><![CDATA[IEEE]]></publisher>
##     <volume><![CDATA[50]]></volume>
##     <issue><![CDATA[10]]></issue>
##     <py><![CDATA[2014]]></py>
##     <spage><![CDATA[1]]></spage>
##     <epage><![CDATA[7]]></epage>
##     <abstract><![CDATA[General principles are developed using a finite element model regarding how time-dependent power dissipation of magnetic nanoparticles can be used to optimize hyperthermia selectivity. To make the simulation more realistic, the finite size and spatial location of each individual nanoparticle is taken into consideration. When energy input into the system and duration of treatment is held constant, increasing the maximum power dissipation of nanoparticles increases concentrations of energy in the tumor. Furthermore, when the power dissipation of magnetic nanoparticles rises linearly, the temperature gradient on the edge of the tumor increases exponentially. With energy input held constant, the location and duration of maximum power dissipation in the treatment time scheme will affect the final energy concentration inside the tumor. Finally, connections are made between the simulation results and optimization of the design of nanoparticle power dissipation time-schemes for hyperthermia.]]></abstract>
##     <issn><![CDATA[0018-9464]]></issn>
##     <htmlFlag><![CDATA[1]]></htmlFlag>
##     <arnumber><![CDATA[6835180]]></arnumber>
##     <doi><![CDATA[10.1109/TMAG.2014.2331031]]></doi>
##     <publicationId><![CDATA[6835180]]></publicationId>
##     <mdurl><![CDATA[http://ieeexplore.ieee.org/xpl/articleDetails.jsp?tp=&arnumber=6835180&contentType=Journals+%26+Magazines]]></mdurl>
##     <pdf><![CDATA[http://ieeexplore.ieee.org/stamp/stamp.jsp?arnumber=6835180]]></pdf>
##   </document>
##   <document>
##     <rank>173</rank>
##     <title><![CDATA[Spectral Dependence of Scattered Light in Step-Index Polymer Optical Fibers by Side-Illumination Technique]]></title>
##     <authors><![CDATA[Bikandi, I.;  Illarramendi, M.A.;  Durana, G.;  Aldabaldetreku, G.;  Zubia, J.]]></authors>
##     <affiliations><![CDATA[Dept. of Commun. Eng., Univ. of the Basque Country (UPV/EHU), Bilbao, Spain]]></affiliations>
##     <controlledterms>
##       <term><![CDATA[light scattering]]></term>
##       <term><![CDATA[optical fibre testing]]></term>
##       <term><![CDATA[optical polymers]]></term>
##     </controlledterms>
##     <thesaurusterms>
##       <term><![CDATA[Nonhomogeneous media]]></term>
##       <term><![CDATA[Optical fiber sensors]]></term>
##       <term><![CDATA[Optical fiber theory]]></term>
##       <term><![CDATA[Polymers]]></term>
##       <term><![CDATA[Power capacitors]]></term>
##       <term><![CDATA[Scattering]]></term>
##     </thesaurusterms>
##     <pubtitle><![CDATA[Lightwave Technology, Journal of]]></pubtitle>
##     <punumber><![CDATA[50]]></punumber>
##     <pubtype><![CDATA[Journals & Magazines]]></pubtype>
##     <publisher><![CDATA[IEEE]]></publisher>
##     <volume><![CDATA[32]]></volume>
##     <issue><![CDATA[23]]></issue>
##     <py><![CDATA[2014]]></py>
##     <spage><![CDATA[4539]]></spage>
##     <epage><![CDATA[4543]]></epage>
##     <abstract><![CDATA[In this study, we investigate the spectral distribution of the light scattered in step-index polymer optical fibers by illuminating the fibers transversely to their symmetry axis. For that purpose, we have performed wavelength-dependent scattering measurements over the spectral range 400-750 nm in three different commercial step-index polymer optical fibers. We have estimated the mean size of the most influential scattering centers in these polymer optical fibers, analyzing theoretically the experimental results obtained.]]></abstract>
##     <issn><![CDATA[0733-8724]]></issn>
##     <htmlFlag><![CDATA[1]]></htmlFlag>
##     <arnumber><![CDATA[6913501]]></arnumber>
##     <doi><![CDATA[10.1109/JLT.2014.2360854]]></doi>
##     <publicationId><![CDATA[6913501]]></publicationId>
##     <mdurl><![CDATA[http://ieeexplore.ieee.org/xpl/articleDetails.jsp?tp=&arnumber=6913501&contentType=Journals+%26+Magazines]]></mdurl>
##     <pdf><![CDATA[http://ieeexplore.ieee.org/stamp/stamp.jsp?arnumber=6913501]]></pdf>
##   </document>
##   <document>
##     <rank>174</rank>
##     <title><![CDATA[FINNIM: Iterative Imputation of Missing Values in&#x00A0;Dissolved Gas Analysis Dataset]]></title>
##     <authors><![CDATA[Sahri, Z.;  Yusof, R.;  Watada, J.]]></authors>
##     <affiliations><![CDATA[Univ. Teknikal Malaysia Melaka, Durian Tunggal, Malaysia]]></affiliations>
##     <controlledterms>
##       <term><![CDATA[decision making]]></term>
##       <term><![CDATA[iterative methods]]></term>
##       <term><![CDATA[pattern classification]]></term>
##       <term><![CDATA[power engineering computing]]></term>
##       <term><![CDATA[power transformers]]></term>
##     </controlledterms>
##     <thesaurusterms>
##       <term><![CDATA[Accuracy]]></term>
##       <term><![CDATA[Convergence]]></term>
##       <term><![CDATA[Fault detection]]></term>
##       <term><![CDATA[Gases]]></term>
##       <term><![CDATA[Iterative methods]]></term>
##       <term><![CDATA[Power transformers]]></term>
##     </thesaurusterms>
##     <pubtitle><![CDATA[Industrial Informatics, IEEE Transactions on]]></pubtitle>
##     <punumber><![CDATA[9424]]></punumber>
##     <pubtype><![CDATA[Journals & Magazines]]></pubtype>
##     <publisher><![CDATA[IEEE]]></publisher>
##     <volume><![CDATA[10]]></volume>
##     <issue><![CDATA[4]]></issue>
##     <py><![CDATA[2014]]></py>
##     <spage><![CDATA[2093]]></spage>
##     <epage><![CDATA[2102]]></epage>
##     <abstract><![CDATA[Missing values are a common occurrence in a number of real world databases, and statistical methods have been developed to deal with this problem, referred to as missing data imputation. In the detection and prediction of incipient faults in power transformers using dissolved gas analysis (DGA), the problem of missing values is significant and has resulted in inconclusive decision-making. This study proposes an efficient nonparametric iterative imputation method named FINNIM, which comprises of three components: 1) the imputation ordering; 2) the imputation estimator; and 3) the iterative imputation. The relationship between gases and faults, and the percentage of missing values in an instance are used as a basis for the imputation ordering; whereas the plausible values for the missing values are estimated from k-nearest neighbor instances in the imputation estimator, and the iterative imputation allows complete and incomplete instances in a DGA dataset to be utilized iteratively for imputing all the missing values. Experimental results on both artificially inserted and actual missing values found in a few DGA datasets demonstrate that the proposed method outperforms the existing methods in imputation accuracy, classification performance, and convergence criteria at different missing percentages.]]></abstract>
##     <issn><![CDATA[1551-3203]]></issn>
##     <htmlFlag><![CDATA[1]]></htmlFlag>
##     <arnumber><![CDATA[6882199]]></arnumber>
##     <doi><![CDATA[10.1109/TII.2014.2350837]]></doi>
##     <publicationId><![CDATA[6882199]]></publicationId>
##     <mdurl><![CDATA[http://ieeexplore.ieee.org/xpl/articleDetails.jsp?tp=&arnumber=6882199&contentType=Journals+%26+Magazines]]></mdurl>
##     <pdf><![CDATA[http://ieeexplore.ieee.org/stamp/stamp.jsp?arnumber=6882199]]></pdf>
##   </document>
##   <document>
##     <rank>175</rank>
##     <title><![CDATA[Highly Comparative Feature-Based Time-Series Classification]]></title>
##     <authors><![CDATA[Fulcher, B.D.;  Jones, N.S.]]></authors>
##     <affiliations><![CDATA[Dept. of Phys., Univ. of Oxford, Oxford, UK]]></affiliations>
##     <controlledterms>
##       <term><![CDATA[data mining]]></term>
##       <term><![CDATA[feature extraction]]></term>
##       <term><![CDATA[feature selection]]></term>
##       <term><![CDATA[greedy algorithms]]></term>
##       <term><![CDATA[learning (artificial intelligence)]]></term>
##       <term><![CDATA[pattern classification]]></term>
##       <term><![CDATA[time series]]></term>
##     </controlledterms>
##     <thesaurusterms>
##       <term><![CDATA[Data mining]]></term>
##       <term><![CDATA[Databases]]></term>
##       <term><![CDATA[Feature extraction]]></term>
##       <term><![CDATA[Market research]]></term>
##       <term><![CDATA[Time measurement]]></term>
##       <term><![CDATA[Time series analysis]]></term>
##     </thesaurusterms>
##     <pubtitle><![CDATA[Knowledge and Data Engineering, IEEE Transactions on]]></pubtitle>
##     <punumber><![CDATA[69]]></punumber>
##     <pubtype><![CDATA[Journals & Magazines]]></pubtype>
##     <publisher><![CDATA[IEEE]]></publisher>
##     <volume><![CDATA[26]]></volume>
##     <issue><![CDATA[12]]></issue>
##     <py><![CDATA[2014]]></py>
##     <spage><![CDATA[3026]]></spage>
##     <epage><![CDATA[3037]]></epage>
##     <abstract><![CDATA[A highly comparative, feature-based approach to time series classification is introduced that uses an extensive database of algorithms to extract thousands of interpretable features from time series. These features are derived from across the scientific time-series analysis literature, and include summaries of time series in terms of their correlation structure, distribution, entropy, stationarity, scaling properties, and fits to a range of time-series models. After computing thousands of features for each time series in a training set, those that are most informative of the class structure are selected using greedy forward feature selection with a linear classifier. The resulting feature-based classifiers automatically learn the differences between classes using a reduced number of time-series properties, and circumvent the need to calculate distances between time series. Representing time series in this way results in orders of magnitude of dimensionality reduction, allowing the method to perform well on very large data sets containing long time series or time series of different lengths. For many of the data sets studied, classification performance exceeded that of conventional instance-based classifiers, including one nearest neighbor classifiers using euclidean distances and dynamic time warping and, most importantly, the features selected provide an understanding of the properties of the data set, insight that can guide further scientific investigation.]]></abstract>
##     <issn><![CDATA[1041-4347]]></issn>
##     <htmlFlag><![CDATA[1]]></htmlFlag>
##     <arnumber><![CDATA[6786425]]></arnumber>
##     <doi><![CDATA[10.1109/TKDE.2014.2316504]]></doi>
##     <publicationId><![CDATA[6786425]]></publicationId>
##     <mdurl><![CDATA[http://ieeexplore.ieee.org/xpl/articleDetails.jsp?tp=&arnumber=6786425&contentType=Journals+%26+Magazines]]></mdurl>
##     <pdf><![CDATA[http://ieeexplore.ieee.org/stamp/stamp.jsp?arnumber=6786425]]></pdf>
##   </document>
##   <document>
##     <rank>176</rank>
##     <title><![CDATA[Direct-Current and Alternating-Current Driving Si Quantum Dots-Based Light Emitting Device]]></title>
##     <authors><![CDATA[Weiwei Mu;  Pei Zhang;  Jun Xu;  Shenghua Sun;  Jie Xu;  Wei Li;  Kunji Chen]]></authors>
##     <affiliations><![CDATA[Nat. Lab. of Solid State Microstructures, Nanjing Univ., Nanjing, China]]></affiliations>
##     <controlledterms>
##       <term><![CDATA[driver circuits]]></term>
##       <term><![CDATA[electroluminescence]]></term>
##       <term><![CDATA[elemental semiconductors]]></term>
##       <term><![CDATA[light emitting devices]]></term>
##       <term><![CDATA[quantum dots]]></term>
##       <term><![CDATA[silicon]]></term>
##       <term><![CDATA[silicon compounds]]></term>
##     </controlledterms>
##     <thesaurusterms>
##       <term><![CDATA[DC motors]]></term>
##       <term><![CDATA[Electroluminescence]]></term>
##       <term><![CDATA[Quantum dots]]></term>
##     </thesaurusterms>
##     <pubtitle><![CDATA[Selected Topics in Quantum Electronics, IEEE Journal of]]></pubtitle>
##     <punumber><![CDATA[2944]]></punumber>
##     <pubtype><![CDATA[Journals & Magazines]]></pubtype>
##     <publisher><![CDATA[IEEE]]></publisher>
##     <volume><![CDATA[20]]></volume>
##     <issue><![CDATA[4]]></issue>
##     <py><![CDATA[2014]]></py>
##     <spage><![CDATA[206]]></spage>
##     <epage><![CDATA[211]]></epage>
##     <abstract><![CDATA[Light emitting devices based on Si quantum dots/SiO<sub>2</sub> multilayers with dot size of 2.5 nm have been prepared. Bright white light emission is achieved under the dc driving conditions and the turn-on voltage of the device is as low as 5 V. The frequency-dependent electroluminescence intensity was observed under ac conditions of square and sinusoidal wave. It was found that the emission wavelength changes with frequency when sinusoidal ac is applied. The degradation of emission intensity is less than 12% after 3 h for ac driving condition, exhibiting the better device stability compared to the dc driving one.]]></abstract>
##     <issn><![CDATA[1077-260X]]></issn>
##     <htmlFlag><![CDATA[1]]></htmlFlag>
##     <arnumber><![CDATA[6509918]]></arnumber>
##     <doi><![CDATA[10.1109/JSTQE.2013.2255587]]></doi>
##     <publicationId><![CDATA[6509918]]></publicationId>
##     <mdurl><![CDATA[http://ieeexplore.ieee.org/xpl/articleDetails.jsp?tp=&arnumber=6509918&contentType=Journals+%26+Magazines]]></mdurl>
##     <pdf><![CDATA[http://ieeexplore.ieee.org/stamp/stamp.jsp?arnumber=6509918]]></pdf>
##   </document>
##   <document>
##     <rank>177</rank>
##     <title><![CDATA[A 28 Gb/s 560 mW Multi-Standard SerDes With Single-Stage Analog Front-End and 14-Tap Decision Feedback Equalizer in 28 nm CMOS]]></title>
##     <authors><![CDATA[Kimura, H.;  Aziz, P.M.;  Tai Jing;  Sinha, A.;  Kotagiri, S.P.;  Narayan, R.;  Hairong Gao;  Ping Jing;  Hom, G.;  Anshi Liang;  Zhang, E.;  Kadkol, A.;  Kothari, R.;  Chan, G.;  Yehui Sun;  Ge, B.;  Zeng, J.;  Ling, K.;  Wang, M.C.;  Malipatil, A.;  Lijun Li;  Abel, C.;  Zhong, F.]]></authors>
##     <affiliations><![CDATA[LSI Corp., San Jose, CA, USA]]></affiliations>
##     <controlledterms>
##       <term><![CDATA[CMOS integrated circuits]]></term>
##       <term><![CDATA[decision feedback equalisers]]></term>
##       <term><![CDATA[flip-flops]]></term>
##       <term><![CDATA[operational amplifiers]]></term>
##       <term><![CDATA[radio transceivers]]></term>
##     </controlledterms>
##     <thesaurusterms>
##       <term><![CDATA[Bandwidth]]></term>
##       <term><![CDATA[Clocks]]></term>
##       <term><![CDATA[Couplings]]></term>
##       <term><![CDATA[Decision feedback equalizers]]></term>
##       <term><![CDATA[Inductors]]></term>
##       <term><![CDATA[Phase locked loops]]></term>
##       <term><![CDATA[System-on-chip]]></term>
##     </thesaurusterms>
##     <pubtitle><![CDATA[Solid-State Circuits, IEEE Journal of]]></pubtitle>
##     <punumber><![CDATA[4]]></punumber>
##     <pubtype><![CDATA[Journals & Magazines]]></pubtype>
##     <publisher><![CDATA[IEEE]]></publisher>
##     <volume><![CDATA[49]]></volume>
##     <issue><![CDATA[12]]></issue>
##     <py><![CDATA[2014]]></py>
##     <spage><![CDATA[3091]]></spage>
##     <epage><![CDATA[3103]]></epage>
##     <abstract><![CDATA[This paper presents a 28 Gb/s multistandard SerDes macro which is fabricated in TSMC 28 nm CMOS process. The transimpedance amplifier (TIA) base analog front-end achieved 15 dB high-frequency boost with an on-chip compact passive inductor. The adaptation loop for the boost is decoupled from the decision feedback equalizer (DFE) adaptation by the use of a group delay algorithm. The DFE is a half-rate 1-tap unrolled design with only two total error latches for power and area reduction. A two-stage sense amplifier-based latch achieved sensitivity of 15 mV. The high-speed clock buffer uses a PMOS active inductor circuit with common-mode feedback to optimize the circuit performance. The transceiver achieves error-free operation at 28 Gbps with 34 dB channel loss, consumes the worst case power of 560 mW/lane, and fully complies with multiple standards and applications.]]></abstract>
##     <issn><![CDATA[0018-9200]]></issn>
##     <htmlFlag><![CDATA[1]]></htmlFlag>
##     <arnumber><![CDATA[6894632]]></arnumber>
##     <doi><![CDATA[10.1109/JSSC.2014.2349974]]></doi>
##     <publicationId><![CDATA[6894632]]></publicationId>
##     <mdurl><![CDATA[http://ieeexplore.ieee.org/xpl/articleDetails.jsp?tp=&arnumber=6894632&contentType=Journals+%26+Magazines]]></mdurl>
##     <pdf><![CDATA[http://ieeexplore.ieee.org/stamp/stamp.jsp?arnumber=6894632]]></pdf>
##   </document>
##   <document>
##     <rank>178</rank>
##     <title><![CDATA[Evolutionary Design of Decision-Tree Algorithms Tailored to Microarray Gene Expression Data Sets]]></title>
##     <authors><![CDATA[Barros, R.C.;  Basgalupp, M.P.;  Freitas, A.A.;  de Carvalho, A.C.P.L.F.]]></authors>
##     <affiliations><![CDATA[Fac. de Inf., Pontificia Univ. Catolica do Rio Grande do Sul, Porto Alegre, Brazil]]></affiliations>
##     <controlledterms>
##       <term><![CDATA[biology computing]]></term>
##       <term><![CDATA[decision trees]]></term>
##       <term><![CDATA[evolutionary computation]]></term>
##       <term><![CDATA[genetics]]></term>
##       <term><![CDATA[learning (artificial intelligence)]]></term>
##       <term><![CDATA[pattern classification]]></term>
##     </controlledterms>
##     <thesaurusterms>
##       <term><![CDATA[Accuracy]]></term>
##       <term><![CDATA[Algorithm design and analysis]]></term>
##       <term><![CDATA[Decision trees]]></term>
##       <term><![CDATA[Evolutionary computation]]></term>
##       <term><![CDATA[Machine learning algorithms]]></term>
##       <term><![CDATA[Prediction algorithms]]></term>
##       <term><![CDATA[Training]]></term>
##     </thesaurusterms>
##     <pubtitle><![CDATA[Evolutionary Computation, IEEE Transactions on]]></pubtitle>
##     <punumber><![CDATA[4235]]></punumber>
##     <pubtype><![CDATA[Journals & Magazines]]></pubtype>
##     <publisher><![CDATA[IEEE]]></publisher>
##     <volume><![CDATA[18]]></volume>
##     <issue><![CDATA[6]]></issue>
##     <py><![CDATA[2014]]></py>
##     <spage><![CDATA[873]]></spage>
##     <epage><![CDATA[892]]></epage>
##     <abstract><![CDATA[Decision-tree induction algorithms are widely used in machine learning applications in which the goal is to extract knowledge from data and present it in a graphically intuitive way. The most successful strategy for inducing decision trees is the greedy top-down recursive approach, which has been continuously improved by researchers over the past 40 years. In this paper, we propose a paradigm shift in the research of decision trees: instead of proposing a new manually designed method for inducing decision trees, we propose automatically designing decision-tree induction algorithms tailored to a specific type of classification data set (or application domain). Following recent breakthroughs in the automatic design of machine learning algorithms, we propose a hyper-heuristic evolutionary algorithm called hyper-heuristic evolutionary algorithm for designing decision-tree algorithms (HEAD-DT) that evolves design components of top-down decision-tree induction algorithms. By the end of the evolution, we expect HEAD-DT to generate a new and possibly better decision-tree algorithm for a given application domain. We perform extensive experiments in 35 real-world microarray gene expression data sets to assess the performance of HEAD-DT, and compare it with very well known decision-tree algorithms such as C4.5, CART, and REPTree. Results show that HEAD-DT is capable of generating algorithms that significantly outperform the baseline manually designed decision-tree algorithms regarding predictive accuracy and F-measure.]]></abstract>
##     <issn><![CDATA[1089-778X]]></issn>
##     <htmlFlag><![CDATA[1]]></htmlFlag>
##     <arnumber><![CDATA[6670778]]></arnumber>
##     <doi><![CDATA[10.1109/TEVC.2013.2291813]]></doi>
##     <publicationId><![CDATA[6670778]]></publicationId>
##     <mdurl><![CDATA[http://ieeexplore.ieee.org/xpl/articleDetails.jsp?tp=&arnumber=6670778&contentType=Journals+%26+Magazines]]></mdurl>
##     <pdf><![CDATA[http://ieeexplore.ieee.org/stamp/stamp.jsp?arnumber=6670778]]></pdf>
##   </document>
##   <document>
##     <rank>179</rank>
##     <title><![CDATA[Breakthroughs in Photonics 2013: Advances in Nanoantennas]]></title>
##     <authors><![CDATA[Malheiros-Silveira, G.N.;  Gabrielli, L.H.;  Chang-Hasnain, C.J.;  Hernandez-Figueroa, H.E.]]></authors>
##     <affiliations><![CDATA[Dept. of Commun., Univ. of Campinas, Campinas, Brazil]]></affiliations>
##     <controlledterms>
##       <term><![CDATA[metamaterial antennas]]></term>
##       <term><![CDATA[nanophotonics]]></term>
##     </controlledterms>
##     <thesaurusterms>
##       <term><![CDATA[Optical device fabrication]]></term>
##       <term><![CDATA[Optical resonators]]></term>
##       <term><![CDATA[Optical scattering]]></term>
##       <term><![CDATA[Phased arrays]]></term>
##       <term><![CDATA[Plasmons]]></term>
##       <term><![CDATA[Silicon]]></term>
##     </thesaurusterms>
##     <pubtitle><![CDATA[Photonics Journal, IEEE]]></pubtitle>
##     <punumber><![CDATA[4563994]]></punumber>
##     <pubtype><![CDATA[Journals & Magazines]]></pubtype>
##     <publisher><![CDATA[IEEE]]></publisher>
##     <volume><![CDATA[6]]></volume>
##     <issue><![CDATA[2]]></issue>
##     <py><![CDATA[2014]]></py>
##     <spage><![CDATA[1]]></spage>
##     <epage><![CDATA[6]]></epage>
##     <abstract><![CDATA[The field of nanoantennas overlaps several areas of scientific interest, from fundamental physics to commercial technologies. Advances in the understanding of such complex devices hold the promises to novel applications in sensing, telecommunications, optical processing, and security, among others. Here, we review the main advances in the field of nanoantennas in 2013, from single metallic and dielectric devices to nanostructured metasurfaces and phased arrays.]]></abstract>
##     <issn><![CDATA[1943-0655]]></issn>
##     <htmlFlag><![CDATA[1]]></htmlFlag>
##     <arnumber><![CDATA[6766176]]></arnumber>
##     <doi><![CDATA[10.1109/JPHOT.2014.2311438]]></doi>
##     <publicationId><![CDATA[6766176]]></publicationId>
##     <mdurl><![CDATA[http://ieeexplore.ieee.org/xpl/articleDetails.jsp?tp=&arnumber=6766176&contentType=Journals+%26+Magazines]]></mdurl>
##     <pdf><![CDATA[http://ieeexplore.ieee.org/stamp/stamp.jsp?arnumber=6766176]]></pdf>
##   </document>
##   <document>
##     <rank>180</rank>
##     <title><![CDATA[Optimal ATMS Remapping Algorithm for Climate Research]]></title>
##     <authors><![CDATA[Hu Yang;  Xiaolei Zou]]></authors>
##     <affiliations><![CDATA[Earth Syst. Sci. Interdiscipl. Center, Univ. of Maryland, College Park, MD, USA]]></affiliations>
##     <controlledterms>
##       <term><![CDATA[atmospheric techniques]]></term>
##       <term><![CDATA[microwave antennas]]></term>
##       <term><![CDATA[remote sensing]]></term>
##     </controlledterms>
##     <thesaurusterms>
##       <term><![CDATA[Antennas]]></term>
##       <term><![CDATA[Brightness temperature]]></term>
##       <term><![CDATA[Instruments]]></term>
##       <term><![CDATA[Meteorology]]></term>
##       <term><![CDATA[Microwave radiometry]]></term>
##       <term><![CDATA[Microwave theory and techniques]]></term>
##       <term><![CDATA[Noise]]></term>
##     </thesaurusterms>
##     <pubtitle><![CDATA[Geoscience and Remote Sensing, IEEE Transactions on]]></pubtitle>
##     <punumber><![CDATA[36]]></punumber>
##     <pubtype><![CDATA[Journals & Magazines]]></pubtype>
##     <publisher><![CDATA[IEEE]]></publisher>
##     <volume><![CDATA[52]]></volume>
##     <issue><![CDATA[11]]></issue>
##     <py><![CDATA[2014]]></py>
##     <spage><![CDATA[7290]]></spage>
##     <epage><![CDATA[7296]]></epage>
##     <abstract><![CDATA[In this paper, the Backus-Gilbert (B-G) method was used for the conversion from Advanced Technology Microwave Sounder (ATMS) FOVs to AMSU-A FOVs. This method provides not only an optimal combination of measurements within a specified region but also a quantitative measure of the tradeoff between resolution and noise. Based on a subpixel microwave antenna temperature simulation technique, ATMS observations at a specified FOV size with 1.1&#x00B0; sampling interval are simulated. Errors of remapping results were quantified by using simulated data sets and real AMSU observations. It is shown that the biases and/or standard deviations of brightness temperatures are significantly reduced by using the B-G generated remapping coefficients. For K/Ka bands, a resolution enhancement by the remap of ATMS observations introduces about 0.6 K increase in noise. For other bands, the channel sensitivity was improved for the remapped data.]]></abstract>
##     <issn><![CDATA[0196-2892]]></issn>
##     <htmlFlag><![CDATA[1]]></htmlFlag>
##     <arnumber><![CDATA[6782443]]></arnumber>
##     <doi><![CDATA[10.1109/TGRS.2014.2310702]]></doi>
##     <publicationId><![CDATA[6782443]]></publicationId>
##     <mdurl><![CDATA[http://ieeexplore.ieee.org/xpl/articleDetails.jsp?tp=&arnumber=6782443&contentType=Journals+%26+Magazines]]></mdurl>
##     <pdf><![CDATA[http://ieeexplore.ieee.org/stamp/stamp.jsp?arnumber=6782443]]></pdf>
##   </document>
##   <document>
##     <rank>181</rank>
##     <title><![CDATA[Piecemeal Development of Intelligent Applications for Smart Spaces]]></title>
##     <authors><![CDATA[Ovaska, E.;  Kuusijarvi, J.]]></authors>
##     <affiliations><![CDATA[VTT Tech. Res. Centre of Finland, Oulu, Finland]]></affiliations>
##     <controlledterms>
##       <term><![CDATA[software architecture]]></term>
##       <term><![CDATA[software quality]]></term>
##       <term><![CDATA[software reusability]]></term>
##     </controlledterms>
##     <thesaurusterms>
##       <term><![CDATA[Adaptation models]]></term>
##       <term><![CDATA[Context modeling]]></term>
##       <term><![CDATA[Knowledge based systems]]></term>
##       <term><![CDATA[Ontologies]]></term>
##       <term><![CDATA[Smart design]]></term>
##       <term><![CDATA[Software development]]></term>
##       <term><![CDATA[Unified modeling language]]></term>
##     </thesaurusterms>
##     <pubtitle><![CDATA[Access, IEEE]]></pubtitle>
##     <punumber><![CDATA[6287639]]></punumber>
##     <pubtype><![CDATA[Journals & Magazines]]></pubtype>
##     <publisher><![CDATA[IEEE]]></publisher>
##     <volume><![CDATA[2]]></volume>
##     <py><![CDATA[2014]]></py>
##     <spage><![CDATA[199]]></spage>
##     <epage><![CDATA[214]]></epage>
##     <abstract><![CDATA[Software development is facing new challenges as a result of evolution toward integration and collaboration-based service engineering, which embody high degrees of dynamism both at design time and run-time. Short times-to-market require cost reduction by maximizing software reuse. Openness for new innovations presumes a flexible development platform and fast software engineering practices. User satisfaction assumes situation-based applications of high quality. The main contribution of this paper is the piecemeal service engineering (PSE) approach developed for and tested in application development for smart spaces. The intent of PSE is to maximize the reuse of existing knowledge of business and design practices and existing technical assets in the development of new smart-space applications. Business knowledge is mostly informal and domain-dependent, but architectural knowledge is generic, at least semiformal, and represented in principles, ontologies, patterns, and rules that together form a reusable architectural knowledge base for fast smart-space application development. The PSE facilitates the incremental development of intelligent applications by supporting abstraction, aggregation, and adaptability in smart-space development.]]></abstract>
##     <issn><![CDATA[2169-3536]]></issn>
##     <htmlFlag><![CDATA[1]]></htmlFlag>
##     <arnumber><![CDATA[6754122]]></arnumber>
##     <doi><![CDATA[10.1109/ACCESS.2014.2309396]]></doi>
##     <publicationId><![CDATA[6754122]]></publicationId>
##     <mdurl><![CDATA[http://ieeexplore.ieee.org/xpl/articleDetails.jsp?tp=&arnumber=6754122&contentType=Journals+%26+Magazines]]></mdurl>
##     <pdf><![CDATA[http://ieeexplore.ieee.org/stamp/stamp.jsp?arnumber=6754122]]></pdf>
##   </document>
##   <document>
##     <rank>182</rank>
##     <title><![CDATA[Broadband Extraordinary Optical Transmission Through Gold Diamond-Shaped Nanohole Arrays]]></title>
##     <authors><![CDATA[Yong-Kai Wang;  Yan Qin;  Zhong-Yue Zhang]]></authors>
##     <affiliations><![CDATA[Sch. of Phys. & Inf. Technol., Shaanxi Normal Univ., Xi'an, China]]></affiliations>
##     <controlledterms>
##       <term><![CDATA[finite element analysis]]></term>
##       <term><![CDATA[gold]]></term>
##       <term><![CDATA[infrared spectra]]></term>
##       <term><![CDATA[metallic thin films]]></term>
##       <term><![CDATA[nanostructured materials]]></term>
##       <term><![CDATA[ultraviolet spectra]]></term>
##       <term><![CDATA[visible spectra]]></term>
##     </controlledterms>
##     <thesaurusterms>
##       <term><![CDATA[Broadband communication]]></term>
##       <term><![CDATA[Electric fields]]></term>
##       <term><![CDATA[Gold]]></term>
##       <term><![CDATA[Optical films]]></term>
##       <term><![CDATA[Optical polarization]]></term>
##       <term><![CDATA[Surface morphology]]></term>
##     </thesaurusterms>
##     <pubtitle><![CDATA[Photonics Journal, IEEE]]></pubtitle>
##     <punumber><![CDATA[4563994]]></punumber>
##     <pubtype><![CDATA[Journals & Magazines]]></pubtype>
##     <publisher><![CDATA[IEEE]]></publisher>
##     <volume><![CDATA[6]]></volume>
##     <issue><![CDATA[4]]></issue>
##     <py><![CDATA[2014]]></py>
##     <spage><![CDATA[1]]></spage>
##     <epage><![CDATA[8]]></epage>
##     <abstract><![CDATA[Extraordinary optical transmission (EOT) is widely accepted as a resonant phenomenon. However, the nonresonant EOT phenomenon of subwavelength metallic nanohole arrays on a metallic thin film is significant for harvesting of broadband light, confining optical power in small area, and enhancing local electric field. To achieve nonresonant EOT, a novel paradigm structure comprising periodic diamond-shaped nanohole arrays engraved on a thin gold film is proposed. The transmission properties of the diamond-shaped nanohole arrays are calculated using the finite-element method. Results show that this paradigm structure facilitates a broadband and enhanced transmission in the infrared region. In addition, the effects of incident polarization and structural parameters on the transmission property are also studied.]]></abstract>
##     <issn><![CDATA[1943-0655]]></issn>
##     <htmlFlag><![CDATA[1]]></htmlFlag>
##     <arnumber><![CDATA[6876119]]></arnumber>
##     <doi><![CDATA[10.1109/JPHOT.2014.2343153]]></doi>
##     <publicationId><![CDATA[6876119]]></publicationId>
##     <mdurl><![CDATA[http://ieeexplore.ieee.org/xpl/articleDetails.jsp?tp=&arnumber=6876119&contentType=Journals+%26+Magazines]]></mdurl>
##     <pdf><![CDATA[http://ieeexplore.ieee.org/stamp/stamp.jsp?arnumber=6876119]]></pdf>
##   </document>
##   <document>
##     <rank>183</rank>
##     <title><![CDATA[The SpiNNaker Project]]></title>
##     <authors><![CDATA[Furber, S.B.;  Galluppi, F.;  Temple, S.;  Plana, L.A.]]></authors>
##     <affiliations><![CDATA[Sch. of Comput. Sci., Univ. of Manchester, Manchester, UK]]></affiliations>
##     <controlledterms>
##       <term><![CDATA[multiprocessing systems]]></term>
##       <term><![CDATA[neural net architecture]]></term>
##       <term><![CDATA[parallel processing]]></term>
##       <term><![CDATA[real-time systems]]></term>
##     </controlledterms>
##     <thesaurusterms>
##       <term><![CDATA[Brain modeling]]></term>
##       <term><![CDATA[Computational modeling]]></term>
##       <term><![CDATA[Computer architecture]]></term>
##       <term><![CDATA[Multitasking]]></term>
##       <term><![CDATA[Neural networks]]></term>
##       <term><![CDATA[Neuroscience]]></term>
##       <term><![CDATA[Parallel programming]]></term>
##       <term><![CDATA[Program processors]]></term>
##     </thesaurusterms>
##     <pubtitle><![CDATA[Proceedings of the IEEE]]></pubtitle>
##     <punumber><![CDATA[5]]></punumber>
##     <pubtype><![CDATA[Journals & Magazines]]></pubtype>
##     <publisher><![CDATA[IEEE]]></publisher>
##     <volume><![CDATA[102]]></volume>
##     <issue><![CDATA[5]]></issue>
##     <py><![CDATA[2014]]></py>
##     <spage><![CDATA[652]]></spage>
##     <epage><![CDATA[665]]></epage>
##     <abstract><![CDATA[The spiking neural network architecture (SpiNNaker) project aims to deliver a massively parallel million-core computer whose interconnect architecture is inspired by the connectivity characteristics of the mammalian brain, and which is suited to the modeling of large-scale spiking neural networks in biological real time. Specifically, the interconnect allows the transmission of a very large number of very small data packets, each conveying explicitly the source, and implicitly the time, of a single neural action potential or &#x201C;spike.&#x201D; In this paper, we review the current state of the project, which has already delivered systems with up to 2500 processors, and present the real-time event-driven programming model that supports flexible access to the resources of the machine and has enabled its use by a wide range of collaborators around the world.]]></abstract>
##     <issn><![CDATA[0018-9219]]></issn>
##     <arnumber><![CDATA[6750072]]></arnumber>
##     <doi><![CDATA[10.1109/JPROC.2014.2304638]]></doi>
##     <publicationId><![CDATA[6750072]]></publicationId>
##     <mdurl><![CDATA[http://ieeexplore.ieee.org/xpl/articleDetails.jsp?tp=&arnumber=6750072&contentType=Journals+%26+Magazines]]></mdurl>
##     <pdf><![CDATA[http://ieeexplore.ieee.org/stamp/stamp.jsp?arnumber=6750072]]></pdf>
##   </document>
##   <document>
##     <rank>184</rank>
##     <title><![CDATA[Retinal Laser Lesion Visibility in Simultaneous Ultra-High Axial Resolution Optical Coherence Tomography]]></title>
##     <authors><![CDATA[Steiner, P.;  Enzmann, V.;  Meier, C.;  Povazay, B.;  Kowal, J.H.]]></authors>
##     <affiliations><![CDATA[ARTORG Center, Univ. of Bern, Bern, Switzerland]]></affiliations>
##     <controlledterms>
##       <term><![CDATA[biomedical optical imaging]]></term>
##       <term><![CDATA[eye]]></term>
##       <term><![CDATA[image resolution]]></term>
##       <term><![CDATA[laser applications in medicine]]></term>
##       <term><![CDATA[medical image processing]]></term>
##       <term><![CDATA[optical tomography]]></term>
##     </controlledterms>
##     <thesaurusterms>
##       <term><![CDATA[Coagulation]]></term>
##       <term><![CDATA[Laser applications]]></term>
##       <term><![CDATA[Lesions]]></term>
##       <term><![CDATA[Retina]]></term>
##       <term><![CDATA[Tomography]]></term>
##     </thesaurusterms>
##     <pubtitle><![CDATA[Photonics Journal, IEEE]]></pubtitle>
##     <punumber><![CDATA[4563994]]></punumber>
##     <pubtype><![CDATA[Journals & Magazines]]></pubtype>
##     <publisher><![CDATA[IEEE]]></publisher>
##     <volume><![CDATA[6]]></volume>
##     <issue><![CDATA[6]]></issue>
##     <py><![CDATA[2014]]></py>
##     <spage><![CDATA[1]]></spage>
##     <epage><![CDATA[11]]></epage>
##     <abstract><![CDATA[Ex vivo porcine retina laser lesions applied with varying laser power (20 mW-2 W, 10 ms pulse, 196 lesions) are manually evaluated by microscopic and optical coherence tomography (OCT) visibility, as well as in histological sections immediately after the deposition of the laser energy. An optical coherence tomography system with 1.78 &#x03BC;m axial resolution specifically developed to image thin retinal layers simultaneously to laser therapy is presented, and visibility thresholds of the laser lesions in OCT data and fundus imaging are compared. Optical coherence tomography scans are compared with histological sections to estimate the resolving power for small optical changes in the retinal layers, and real-time time-lapse scans during laser application are shown and analyzed quantitatively. Ultrahigh-resolution OCT inspection features a lesion visibility threshold 40-50 mW (17% reduction) lower than for visual inspection. With the new measurement system, 42% of the lesions that were invisible using state-of-the-art ophthalmoscopic methods could be detected.]]></abstract>
##     <issn><![CDATA[1943-0655]]></issn>
##     <htmlFlag><![CDATA[1]]></htmlFlag>
##     <arnumber><![CDATA[6966735]]></arnumber>
##     <doi><![CDATA[10.1109/JPHOT.2014.2374594]]></doi>
##     <publicationId><![CDATA[6966735]]></publicationId>
##     <mdurl><![CDATA[http://ieeexplore.ieee.org/xpl/articleDetails.jsp?tp=&arnumber=6966735&contentType=Journals+%26+Magazines]]></mdurl>
##     <pdf><![CDATA[http://ieeexplore.ieee.org/stamp/stamp.jsp?arnumber=6966735]]></pdf>
##   </document>
##   <document>
##     <rank>185</rank>
##     <title><![CDATA[Thermal Facial Analysis for Deception Detection]]></title>
##     <authors><![CDATA[Rajoub, B.A.;  Zwiggelaar, R.]]></authors>
##     <affiliations><![CDATA[Centre for Appl. Digital Signal & Image Process., Univ. of Central Lancashire, Preston, UK]]></affiliations>
##     <controlledterms>
##       <term><![CDATA[decision making]]></term>
##       <term><![CDATA[face recognition]]></term>
##       <term><![CDATA[image classification]]></term>
##       <term><![CDATA[infrared imaging]]></term>
##       <term><![CDATA[learning (artificial intelligence)]]></term>
##       <term><![CDATA[object detection]]></term>
##     </controlledterms>
##     <thesaurusterms>
##       <term><![CDATA[Accuracy]]></term>
##       <term><![CDATA[Feature extraction]]></term>
##       <term><![CDATA[Imaging]]></term>
##       <term><![CDATA[Interviews]]></term>
##       <term><![CDATA[Robustness]]></term>
##       <term><![CDATA[Stress]]></term>
##       <term><![CDATA[Thermal analysis]]></term>
##     </thesaurusterms>
##     <pubtitle><![CDATA[Information Forensics and Security, IEEE Transactions on]]></pubtitle>
##     <punumber><![CDATA[10206]]></punumber>
##     <pubtype><![CDATA[Journals & Magazines]]></pubtype>
##     <publisher><![CDATA[IEEE]]></publisher>
##     <volume><![CDATA[9]]></volume>
##     <issue><![CDATA[6]]></issue>
##     <py><![CDATA[2014]]></py>
##     <spage><![CDATA[1015]]></spage>
##     <epage><![CDATA[1023]]></epage>
##     <abstract><![CDATA[Thermal imaging technology can be used to detect stress levels in humans based on the radiated heat from their face. In this paper, we use thermal imaging to monitor the periorbital region's thermal variations and test whether it can offer a discriminative signature for detecting deception. We start by presenting an overview on automated deception detection and propose a novel methodology, which we validate experimentally on 492 thermal responses (249 lies and 243 truths) extracted from 25 participants. The novelty of this paper lies in scoring a larger number of questions per subject, emphasizing a within-person approach for learning from data, proposing a framework for validating the decision making process, and correct evaluation of the generalization performance. A $k$ -nearest neighbor classifier was used to classify the thermal responses using different strategies for data representation. We report an 87% ability to predict the lie/truth responses based on a within-person methodology and fivefold cross validation. Our results also show that the between-person approach for modeling deception does not generalize very well across the training data.]]></abstract>
##     <issn><![CDATA[1556-6013]]></issn>
##     <arnumber><![CDATA[6797879]]></arnumber>
##     <doi><![CDATA[10.1109/TIFS.2014.2317309]]></doi>
##     <publicationId><![CDATA[6797879]]></publicationId>
##     <mdurl><![CDATA[http://ieeexplore.ieee.org/xpl/articleDetails.jsp?tp=&arnumber=6797879&contentType=Journals+%26+Magazines]]></mdurl>
##     <pdf><![CDATA[http://ieeexplore.ieee.org/stamp/stamp.jsp?arnumber=6797879]]></pdf>
##   </document>
##   <document>
##     <rank>186</rank>
##     <title><![CDATA[Comparison of Complex Principal and Independent Components for Quasi-Gaussian Radiated Emissions From Printed Circuit Boards]]></title>
##     <authors><![CDATA[Arnaut, L.R.;  Obiekezie, C.S.]]></authors>
##     <affiliations><![CDATA[George Green Inst. of Electromagn. Res., Univ. of Nottingham, Nottingham, UK]]></affiliations>
##     <controlledterms>
##       <term><![CDATA[Gaussian distribution]]></term>
##       <term><![CDATA[electromagnetic interference]]></term>
##       <term><![CDATA[electromagnetic wave propagation]]></term>
##       <term><![CDATA[independent component analysis]]></term>
##       <term><![CDATA[principal component analysis]]></term>
##       <term><![CDATA[printed circuits]]></term>
##     </controlledterms>
##     <thesaurusterms>
##       <term><![CDATA[Decorrelation]]></term>
##       <term><![CDATA[Eigenvalues and eigenfunctions]]></term>
##       <term><![CDATA[Integrated circuits]]></term>
##       <term><![CDATA[Principal component analysis]]></term>
##       <term><![CDATA[Printed circuits]]></term>
##       <term><![CDATA[Transmission line matrix methods]]></term>
##     </thesaurusterms>
##     <pubtitle><![CDATA[Electromagnetic Compatibility, IEEE Transactions on]]></pubtitle>
##     <punumber><![CDATA[15]]></punumber>
##     <pubtype><![CDATA[Journals & Magazines]]></pubtype>
##     <publisher><![CDATA[IEEE]]></publisher>
##     <volume><![CDATA[56]]></volume>
##     <issue><![CDATA[6]]></issue>
##     <py><![CDATA[2014]]></py>
##     <spage><![CDATA[1598]]></spage>
##     <epage><![CDATA[1603]]></epage>
##     <abstract><![CDATA[Principal component analysis (PCA) and independent component analysis (ICA) for radiated emissions from printed circuits are critically intercompared, revealing similarities and differences of the extracted components between both methods. The input data in this analysis are measured wideband complex-valued magnetic radiated and evanescent fields with quasi-Gaussian spatial distributions. PCA and ICA lead to similar maps of their components when considered as spatial eigenmodes, but independent components exhibit simpler field structure than principal components.]]></abstract>
##     <issn><![CDATA[0018-9375]]></issn>
##     <htmlFlag><![CDATA[1]]></htmlFlag>
##     <arnumber><![CDATA[6891284]]></arnumber>
##     <doi><![CDATA[10.1109/TEMC.2014.2343912]]></doi>
##     <publicationId><![CDATA[6891284]]></publicationId>
##     <mdurl><![CDATA[http://ieeexplore.ieee.org/xpl/articleDetails.jsp?tp=&arnumber=6891284&contentType=Journals+%26+Magazines]]></mdurl>
##     <pdf><![CDATA[http://ieeexplore.ieee.org/stamp/stamp.jsp?arnumber=6891284]]></pdf>
##   </document>
##   <document>
##     <rank>187</rank>
##     <title><![CDATA[An Efficient Hybrid Decoder for Block Turbo Codes]]></title>
##     <authors><![CDATA[Pen-Yao Lu;  Erl-Huei Lu;  Tso-Cho Chen]]></authors>
##     <affiliations><![CDATA[Dept. of Electr. Eng., Chang Gung Univ., Taoyuan, Taiwan]]></affiliations>
##     <controlledterms>
##       <term><![CDATA[block codes]]></term>
##       <term><![CDATA[computational complexity]]></term>
##       <term><![CDATA[decoding]]></term>
##       <term><![CDATA[error statistics]]></term>
##       <term><![CDATA[turbo codes]]></term>
##     </controlledterms>
##     <thesaurusterms>
##       <term><![CDATA[Binary phase shift keying]]></term>
##       <term><![CDATA[Bit error rate]]></term>
##       <term><![CDATA[Complexity theory]]></term>
##       <term><![CDATA[Decoding]]></term>
##       <term><![CDATA[Product codes]]></term>
##       <term><![CDATA[Reliability]]></term>
##       <term><![CDATA[Turbo codes]]></term>
##     </thesaurusterms>
##     <pubtitle><![CDATA[Communications Letters, IEEE]]></pubtitle>
##     <punumber><![CDATA[4234]]></punumber>
##     <pubtype><![CDATA[Journals & Magazines]]></pubtype>
##     <publisher><![CDATA[IEEE]]></publisher>
##     <volume><![CDATA[18]]></volume>
##     <issue><![CDATA[12]]></issue>
##     <py><![CDATA[2014]]></py>
##     <spage><![CDATA[2077]]></spage>
##     <epage><![CDATA[2080]]></epage>
##     <abstract><![CDATA[In this letter, an efficient hybrid decoder for block turbo codes (BTCs) is proposed to improve the hybrid BTC decoder developed by Al-Dweik et al. A simple formula for estimating the extrinsic information is first derived. Then the proposed decoder is constructed by modifying the decoder of Al-Dweik et al. using the formula. Simulation results show that the proposed decoder can substantially reduce the complexity of the decoder of Al-Dweik et al., especially for moderate and high signal-to-noise ratios, with nearly the same bit error rate performance.]]></abstract>
##     <issn><![CDATA[1089-7798]]></issn>
##     <htmlFlag><![CDATA[1]]></htmlFlag>
##     <arnumber><![CDATA[6932429]]></arnumber>
##     <doi><![CDATA[10.1109/LCOMM.2014.2364229]]></doi>
##     <publicationId><![CDATA[6932429]]></publicationId>
##     <mdurl><![CDATA[http://ieeexplore.ieee.org/xpl/articleDetails.jsp?tp=&arnumber=6932429&contentType=Journals+%26+Magazines]]></mdurl>
##     <pdf><![CDATA[http://ieeexplore.ieee.org/stamp/stamp.jsp?arnumber=6932429]]></pdf>
##   </document>
##   <document>
##     <rank>188</rank>
##     <title><![CDATA[Development of a Remote Sensing-Based Method to Map Likelihood of Common Ragweed (Ambrosia artemisiifolia) Presence in Urban Areas]]></title>
##     <authors><![CDATA[Ngom, R.;  Gosselin, P.]]></authors>
##     <affiliations><![CDATA[Centre-Eau Terre Environ., Inst. Nat. de la Rech. Sci., Quebec City, QC, Canada]]></affiliations>
##     <controlledterms>
##       <term><![CDATA[fuzzy logic]]></term>
##       <term><![CDATA[geophysical image processing]]></term>
##       <term><![CDATA[statistical analysis]]></term>
##       <term><![CDATA[vegetation mapping]]></term>
##     </controlledterms>
##     <thesaurusterms>
##       <term><![CDATA[Global Positioning System]]></term>
##       <term><![CDATA[Principal component analysis]]></term>
##       <term><![CDATA[Radiometry]]></term>
##       <term><![CDATA[Remote sensing]]></term>
##       <term><![CDATA[Sensors]]></term>
##       <term><![CDATA[Spatial resolution]]></term>
##       <term><![CDATA[Vegetation mapping]]></term>
##     </thesaurusterms>
##     <pubtitle><![CDATA[Selected Topics in Applied Earth Observations and Remote Sensing, IEEE Journal of]]></pubtitle>
##     <punumber><![CDATA[4609443]]></punumber>
##     <pubtype><![CDATA[Journals & Magazines]]></pubtype>
##     <publisher><![CDATA[IEEE]]></publisher>
##     <volume><![CDATA[7]]></volume>
##     <issue><![CDATA[1]]></issue>
##     <py><![CDATA[2014]]></py>
##     <spage><![CDATA[126]]></spage>
##     <epage><![CDATA[139]]></epage>
##     <abstract><![CDATA[Common Ragweed (Ambrosia artemisiifolia) is a plant that constitutes an important and growing public health concern worldwide as it is probably expanding with climate change, which brings forward the need for improved mapping tools. Our final purpose is to operationalize the use of optical remote sensing for the automated mapping and surveillance of Ambrosia artemisiifolia. Analyses considering the probable spectral instability originating from the variability of the urban landscape and from that of sensors characteristics were developed. Worldview 2, Rapid Eye and SPOT 4 HRVIR sensors were used together with geolocalized surveys of Common Ragweed in Montre&#x0301;al and Valleyfield (Quebec, Canada). Images were standardized and various derivatives variables such as multiple vegetation indexes were created. Spectral confusion, statistical analyses, object-oriented technology and Fuzzy-logic functions were used to develop predictive risks maps of Common Ragweed potential presence. The results showed that the green bands (510-590 nm) of higher spatial resolutions sensors had a higher potential to cope with spectral confusions and changing landscape characteristics and to predict the likelihood of Ambrosia artemisiifolia presence with a recurrent stability. The good agreement between observed and predicted ragweed revealed an important potential for the operationalization of this method.]]></abstract>
##     <issn><![CDATA[1939-1404]]></issn>
##     <htmlFlag><![CDATA[1]]></htmlFlag>
##     <arnumber><![CDATA[6510508]]></arnumber>
##     <doi><![CDATA[10.1109/JSTARS.2013.2254469]]></doi>
##     <publicationId><![CDATA[6510508]]></publicationId>
##     <mdurl><![CDATA[http://ieeexplore.ieee.org/xpl/articleDetails.jsp?tp=&arnumber=6510508&contentType=Journals+%26+Magazines]]></mdurl>
##     <pdf><![CDATA[http://ieeexplore.ieee.org/stamp/stamp.jsp?arnumber=6510508]]></pdf>
##   </document>
##   <document>
##     <rank>189</rank>
##     <title><![CDATA[Multiphysics NVH Modeling: Simulation of a Switched Reluctance Motor for an Electric Vehicle]]></title>
##     <authors><![CDATA[dos Santos, F.L.M.;  Anthonis, J.;  Naclerio, F.;  Gyselinck, J.J.C.;  Van der Auweraer, H.;  Goes, L.C.S.]]></authors>
##     <affiliations><![CDATA[LMS Int., Leuven, Belgium]]></affiliations>
##     <controlledterms>
##       <term><![CDATA[electric vehicles]]></term>
##       <term><![CDATA[finite element analysis]]></term>
##       <term><![CDATA[magnetic forces]]></term>
##       <term><![CDATA[mechatronics]]></term>
##       <term><![CDATA[reluctance motors]]></term>
##       <term><![CDATA[stators]]></term>
##     </controlledterms>
##     <thesaurusterms>
##       <term><![CDATA[Permanent magnet motors]]></term>
##       <term><![CDATA[Reluctance motors]]></term>
##       <term><![CDATA[Rotors]]></term>
##       <term><![CDATA[Stator windings]]></term>
##       <term><![CDATA[Traction motors]]></term>
##     </thesaurusterms>
##     <pubtitle><![CDATA[Industrial Electronics, IEEE Transactions on]]></pubtitle>
##     <punumber><![CDATA[41]]></punumber>
##     <pubtype><![CDATA[Journals & Magazines]]></pubtype>
##     <publisher><![CDATA[IEEE]]></publisher>
##     <volume><![CDATA[61]]></volume>
##     <issue><![CDATA[1]]></issue>
##     <py><![CDATA[2014]]></py>
##     <spage><![CDATA[469]]></spage>
##     <epage><![CDATA[476]]></epage>
##     <abstract><![CDATA[This paper presents a multiphysics modeling of a switched reluctance motor (SRM) to simulate the acoustic radiation of the electrical machine. The proposed method uses a 2-D finite-element model of the motor to simulate its magnetic properties and a multiphysics mechatronic model of the motor and controls to simulate operating conditions. Magnetic forces on the stator are calculated using finite-element analysis and are used as the excitation on a forced response analysis that contains a finite-element model of the motor stator structure. Finally, sound power levels are calculated using the boundary element method. Simulation results of the model are shown and compared with experimental measurements for a four-phase 8/6 SRM.]]></abstract>
##     <issn><![CDATA[0278-0046]]></issn>
##     <htmlFlag><![CDATA[1]]></htmlFlag>
##     <arnumber><![CDATA[6461408]]></arnumber>
##     <doi><![CDATA[10.1109/TIE.2013.2247012]]></doi>
##     <publicationId><![CDATA[6461408]]></publicationId>
##     <mdurl><![CDATA[http://ieeexplore.ieee.org/xpl/articleDetails.jsp?tp=&arnumber=6461408&contentType=Journals+%26+Magazines]]></mdurl>
##     <pdf><![CDATA[http://ieeexplore.ieee.org/stamp/stamp.jsp?arnumber=6461408]]></pdf>
##   </document>
##   <document>
##     <rank>190</rank>
##     <title><![CDATA[A Fully Digital 8<formula formulatype="inline"> <img src="/images/tex/353.gif" alt=",\times,"> </formula>16 SiPM Array for PET Applications With Per-Pixel TDCs and Real-Time Energy Output]]></title>
##     <authors><![CDATA[Braga, L.H.C.;  Gasparini, L.;  Grant, L.;  Henderson, R.K.;  Massari, N.;  Perenzoni, M.;  Stoppa, D.;  Walker, R.]]></authors>
##     <affiliations><![CDATA[Univ. of Trento, Trento, Italy]]></affiliations>
##     <controlledterms>
##       <term><![CDATA[CMOS image sensors]]></term>
##       <term><![CDATA[adders]]></term>
##       <term><![CDATA[avalanche photodiodes]]></term>
##       <term><![CDATA[biomedical electronics]]></term>
##       <term><![CDATA[biosensors]]></term>
##       <term><![CDATA[photomultipliers]]></term>
##       <term><![CDATA[positron emission tomography]]></term>
##     </controlledterms>
##     <thesaurusterms>
##       <term><![CDATA[Arrays]]></term>
##       <term><![CDATA[CMOS integrated circuits]]></term>
##       <term><![CDATA[Detectors]]></term>
##       <term><![CDATA[Photonics]]></term>
##       <term><![CDATA[Positron emission tomography]]></term>
##       <term><![CDATA[Radiation detectors]]></term>
##       <term><![CDATA[Real-time systems]]></term>
##     </thesaurusterms>
##     <pubtitle><![CDATA[Solid-State Circuits, IEEE Journal of]]></pubtitle>
##     <punumber><![CDATA[4]]></punumber>
##     <pubtype><![CDATA[Journals & Magazines]]></pubtype>
##     <publisher><![CDATA[IEEE]]></publisher>
##     <volume><![CDATA[49]]></volume>
##     <issue><![CDATA[1]]></issue>
##     <py><![CDATA[2014]]></py>
##     <spage><![CDATA[301]]></spage>
##     <epage><![CDATA[314]]></epage>
##     <abstract><![CDATA[An 8 &#x00D7; 16 pixel array based on CMOS small-area silicon photomultipliers (mini-SiPMs) detectors for PET applications is reported. Each pixel is 570 &#x00D7; 610 &#x03BC;m<sup>2</sup> in size and contains four digital mini-SiPMs, for a total of 720 SPADs, resulting in a full chip fill-factor of 35.7%. For each gamma detection, the pixel provides the total detected energy and a timestamp, obtained through two 7-b counters and two 12-b 64-ps TDCs. An adder tree overlaid on top of the pixel array sums the sensor total counts at up to 100 Msamples/s, which are then used for detecting the asynchronous gamma events on-chip, while also being output in real-time. Characterization of gamma detection performance with an 3 &#x00D7; 3 &#x00D7; 5 mm<sup>3</sup> LYSO scintillator at 20&#x00B0;C is reported, showing a 511-keV gamma energy resolution of 10.9% and a coincidence timing resolution of 399 ps.]]></abstract>
##     <issn><![CDATA[0018-9200]]></issn>
##     <htmlFlag><![CDATA[1]]></htmlFlag>
##     <arnumber><![CDATA[6642135]]></arnumber>
##     <doi><![CDATA[10.1109/JSSC.2013.2284351]]></doi>
##     <publicationId><![CDATA[6642135]]></publicationId>
##     <mdurl><![CDATA[http://ieeexplore.ieee.org/xpl/articleDetails.jsp?tp=&arnumber=6642135&contentType=Journals+%26+Magazines]]></mdurl>
##     <pdf><![CDATA[http://ieeexplore.ieee.org/stamp/stamp.jsp?arnumber=6642135]]></pdf>
##   </document>
##   <document>
##     <rank>191</rank>
##     <title><![CDATA[Semantic Interoperability Architecture for Pervasive Computing and Internet of Things]]></title>
##     <authors><![CDATA[Kiljander, J.;  D'Elia, A.;  Morandi, F.;  Hyttinen, P.;  Takalo-Mattila, J.;  Ylisaukko-Oja, A.;  Soininen, J.-P.;  Cinotti, T.S.]]></authors>
##     <affiliations><![CDATA[VTT Tech. Res. Centre of Finland, Oulu, Finland]]></affiliations>
##     <controlledterms>
##       <term><![CDATA[Internet of Things]]></term>
##       <term><![CDATA[knowledge representation]]></term>
##       <term><![CDATA[open systems]]></term>
##       <term><![CDATA[semantic Web]]></term>
##     </controlledterms>
##     <thesaurusterms>
##       <term><![CDATA[Computer architecture]]></term>
##       <term><![CDATA[Context awareness]]></term>
##       <term><![CDATA[Interoperability]]></term>
##       <term><![CDATA[Pervasive computing]]></term>
##       <term><![CDATA[Resource description framework]]></term>
##       <term><![CDATA[Semantics]]></term>
##       <term><![CDATA[Sensors]]></term>
##     </thesaurusterms>
##     <pubtitle><![CDATA[Access, IEEE]]></pubtitle>
##     <punumber><![CDATA[6287639]]></punumber>
##     <pubtype><![CDATA[Journals & Magazines]]></pubtype>
##     <publisher><![CDATA[IEEE]]></publisher>
##     <volume><![CDATA[2]]></volume>
##     <py><![CDATA[2014]]></py>
##     <spage><![CDATA[856]]></spage>
##     <epage><![CDATA[873]]></epage>
##     <abstract><![CDATA[Pervasive computing and Internet of Things (IoTs) paradigms have created a huge potential for new business. To fully realize this potential, there is a need for a common way to abstract the heterogeneity of devices so that their functionality can be represented as a virtual computing platform. To this end, we present novel semantic level interoperability architecture for pervasive computing and IoTs. There are two main principles in the proposed architecture. First, information and capabilities of devices are represented with semantic web knowledge representation technologies and interaction with devices and the physical world is achieved by accessing and modifying their virtual representations. Second, global IoT is divided into numerous local smart spaces managed by a semantic information broker (SIB) that provides a means to monitor and update the virtual representation of the physical world. An integral part of the architecture is a resolution infrastructure that provides a means to resolve the network address of a SIB either using a physical object identifier as a pointer to information or by searching SIBs matching a specification represented with SPARQL. We present several reference implementations and applications that we have developed to evaluate the architecture in practice. The evaluation also includes performance studies that, together with the applications, demonstrate the suitability of the architecture to real-life IoT scenarios. In addition, to validate that the proposed architecture conforms to the common IoT-A architecture reference model (ARM), we map the central components of the architecture to the IoT-ARM.]]></abstract>
##     <issn><![CDATA[2169-3536]]></issn>
##     <htmlFlag><![CDATA[1]]></htmlFlag>
##     <arnumber><![CDATA[6879461]]></arnumber>
##     <doi><![CDATA[10.1109/ACCESS.2014.2347992]]></doi>
##     <publicationId><![CDATA[6879461]]></publicationId>
##     <mdurl><![CDATA[http://ieeexplore.ieee.org/xpl/articleDetails.jsp?tp=&arnumber=6879461&contentType=Journals+%26+Magazines]]></mdurl>
##     <pdf><![CDATA[http://ieeexplore.ieee.org/stamp/stamp.jsp?arnumber=6879461]]></pdf>
##   </document>
##   <document>
##     <rank>192</rank>
##     <title><![CDATA[Parsimonious Network Traffic Modeling By Transformed ARMA Models]]></title>
##     <authors><![CDATA[Laner, M.;  Svoboda, P.;  Rupp, M.]]></authors>
##     <affiliations><![CDATA[Inst. of Telecommun., Vienna Univ. of Technol., Vienna, Austria]]></affiliations>
##     <controlledterms>
##       <term><![CDATA[Gaussian processes]]></term>
##       <term><![CDATA[autoregressive moving average processes]]></term>
##       <term><![CDATA[computer networks]]></term>
##       <term><![CDATA[queueing theory]]></term>
##       <term><![CDATA[telecommunication traffic]]></term>
##     </controlledterms>
##     <thesaurusterms>
##       <term><![CDATA[Data models]]></term>
##       <term><![CDATA[Data processing]]></term>
##       <term><![CDATA[Hidden Markov models]]></term>
##       <term><![CDATA[Modeling]]></term>
##       <term><![CDATA[Random processes]]></term>
##       <term><![CDATA[Streaming media]]></term>
##       <term><![CDATA[Telecommunication traffic]]></term>
##     </thesaurusterms>
##     <pubtitle><![CDATA[Access, IEEE]]></pubtitle>
##     <punumber><![CDATA[6287639]]></punumber>
##     <pubtype><![CDATA[Journals & Magazines]]></pubtype>
##     <publisher><![CDATA[IEEE]]></publisher>
##     <volume><![CDATA[2]]></volume>
##     <py><![CDATA[2014]]></py>
##     <spage><![CDATA[40]]></spage>
##     <epage><![CDATA[55]]></epage>
##     <abstract><![CDATA[Generating synthetic data traffic, which statistically resembles its recorded counterpart is one of the main goals of network traffic modeling. Equivalently, one or several random processes shall be created, exhibiting multiple prescribed statistical measures. In this paper, we present a framework enabling the joint representation of distributions, autocorrelations and cross-correlations of multiple processes. This is achieved by so called transformed Gaussian autoregressive moving-average models. They constitute an analytically tractable framework, which allows for the separation of the fitting problems into subproblems for individual measures. Accordingly, known fitting techniques and algorithms can be deployed for the respective solution. The proposed framework exhibits promising properties: 1) relevant statistical properties such as heavy tails and long-range dependences are manageable; 2) the resulting models are parsimonious; 3) the fitting procedure is fully automatic; and 4) the complexity of generating synthetic traffic is very low. We evaluate the framework with traced traffic, i.e., aggregated traffic, online gaming, and video streaming. The queueing responses of synthetic and recorded traffic exhibit identical statistics. This paper provides guidance for high-quality modeling of network traffic. It proposes a unifying framework, validates several fitting algorithms, and suggests combinations of algorithms suited best for specific traffic types.]]></abstract>
##     <issn><![CDATA[2169-3536]]></issn>
##     <htmlFlag><![CDATA[1]]></htmlFlag>
##     <arnumber><![CDATA[6710106]]></arnumber>
##     <doi><![CDATA[10.1109/ACCESS.2013.2297736]]></doi>
##     <publicationId><![CDATA[6710106]]></publicationId>
##     <mdurl><![CDATA[http://ieeexplore.ieee.org/xpl/articleDetails.jsp?tp=&arnumber=6710106&contentType=Journals+%26+Magazines]]></mdurl>
##     <pdf><![CDATA[http://ieeexplore.ieee.org/stamp/stamp.jsp?arnumber=6710106]]></pdf>
##   </document>
##   <document>
##     <rank>193</rank>
##     <title><![CDATA[Joint Contour Nets]]></title>
##     <authors><![CDATA[Carr, H.;  Duke, D.]]></authors>
##     <affiliations><![CDATA[Sch. of Comput., Univ. of Leeds, Leeds, UK]]></affiliations>
##     <controlledterms>
##       <term><![CDATA[data structures]]></term>
##       <term><![CDATA[data visualisation]]></term>
##       <term><![CDATA[trees (mathematics)]]></term>
##     </controlledterms>
##     <thesaurusterms>
##       <term><![CDATA[Algorithm design and analysis]]></term>
##       <term><![CDATA[Isosurfaces]]></term>
##       <term><![CDATA[Jacobian matrices]]></term>
##       <term><![CDATA[Joints]]></term>
##       <term><![CDATA[Level set]]></term>
##       <term><![CDATA[Slabs]]></term>
##     </thesaurusterms>
##     <pubtitle><![CDATA[Visualization and Computer Graphics, IEEE Transactions on]]></pubtitle>
##     <punumber><![CDATA[2945]]></punumber>
##     <pubtype><![CDATA[Journals & Magazines]]></pubtype>
##     <publisher><![CDATA[IEEE]]></publisher>
##     <volume><![CDATA[20]]></volume>
##     <issue><![CDATA[8]]></issue>
##     <py><![CDATA[2014]]></py>
##     <spage><![CDATA[1100]]></spage>
##     <epage><![CDATA[1113]]></epage>
##     <abstract><![CDATA[Contour Trees and Reeb Graphs are firmly embedded in scientific visualization for analysing univariate (scalar) fields. We generalize this analysis to multivariate fields with a data structure called the Joint Contour Net that quantizes the variation of multiple variables simultaneously. We report the first algorithm for constructing the Joint Contour Net, and demonstrate some of the properties that make it practically useful for visualisation, including accelerating computation by exploiting a relationship with rasterisation in the range of the function.]]></abstract>
##     <issn><![CDATA[1077-2626]]></issn>
##     <htmlFlag><![CDATA[1]]></htmlFlag>
##     <arnumber><![CDATA[6684144]]></arnumber>
##     <doi><![CDATA[10.1109/TVCG.2013.269]]></doi>
##     <publicationId><![CDATA[6684144]]></publicationId>
##     <mdurl><![CDATA[http://ieeexplore.ieee.org/xpl/articleDetails.jsp?tp=&arnumber=6684144&contentType=Journals+%26+Magazines]]></mdurl>
##     <pdf><![CDATA[http://ieeexplore.ieee.org/stamp/stamp.jsp?arnumber=6684144]]></pdf>
##   </document>
##   <document>
##     <rank>194</rank>
##     <title><![CDATA[Overview of Dual-Active-Bridge Isolated Bidirectional DC&#x2013;DC Converter for High-Frequency-Link Power-Conversion System]]></title>
##     <authors><![CDATA[Biao Zhao;  Qiang Song;  Wenhua Liu;  Yandong Sun]]></authors>
##     <affiliations><![CDATA[Dept. of Electr. Eng., Tsinghua Univ., Beijing, China]]></affiliations>
##     <controlledterms>
##       <term><![CDATA[DC-DC power convertors]]></term>
##       <term><![CDATA[zero current switching]]></term>
##       <term><![CDATA[zero voltage switching]]></term>
##     </controlledterms>
##     <thesaurusterms>
##       <term><![CDATA[Bridge circuits]]></term>
##       <term><![CDATA[Circuit faults]]></term>
##       <term><![CDATA[Inductors]]></term>
##       <term><![CDATA[Integrated circuit modeling]]></term>
##       <term><![CDATA[Switches]]></term>
##       <term><![CDATA[Topology]]></term>
##     </thesaurusterms>
##     <pubtitle><![CDATA[Power Electronics, IEEE Transactions on]]></pubtitle>
##     <punumber><![CDATA[63]]></punumber>
##     <pubtype><![CDATA[Journals & Magazines]]></pubtype>
##     <publisher><![CDATA[IEEE]]></publisher>
##     <volume><![CDATA[29]]></volume>
##     <issue><![CDATA[8]]></issue>
##     <py><![CDATA[2014]]></py>
##     <spage><![CDATA[4091]]></spage>
##     <epage><![CDATA[4106]]></epage>
##     <abstract><![CDATA[High-frequency-link (HFL) power conversion systems (PCSs) are attracting more and more attentions in academia and industry for high power density, reduced weight, and low noise without compromising efficiency, cost, and reliability. In HFL PCSs, dual-active-bridge (DAB) isolated bidirectional dc-dc converter (IBDC) serves as the core circuit. This paper gives an overview of DAB-IBDC for HFL PCSs. First, the research necessity and development history are introduced. Second, the research subjects about basic characterization, control strategy, soft-switching solution and variant, as well as hardware design and optimization are reviewed and analyzed. On this basis, several typical application schemes of DAB-IBDC for HPL PCSs are presented in a worldwide scope. Finally, design recommendations and future trends are presented. As the core circuit of HFL PCSs, DAB-IBDC has wide prospects. The large-scale practical application of DAB-IBDC for HFL PCSs is expected with the recent advances in solid-state semiconductors, magnetic and capacitive materials, and microelectronic technologies.]]></abstract>
##     <issn><![CDATA[0885-8993]]></issn>
##     <htmlFlag><![CDATA[1]]></htmlFlag>
##     <arnumber><![CDATA[6658916]]></arnumber>
##     <doi><![CDATA[10.1109/TPEL.2013.2289913]]></doi>
##     <publicationId><![CDATA[6658916]]></publicationId>
##     <mdurl><![CDATA[http://ieeexplore.ieee.org/xpl/articleDetails.jsp?tp=&arnumber=6658916&contentType=Journals+%26+Magazines]]></mdurl>
##     <pdf><![CDATA[http://ieeexplore.ieee.org/stamp/stamp.jsp?arnumber=6658916]]></pdf>
##   </document>
##   <document>
##     <rank>195</rank>
##     <title><![CDATA[Semiautomatic Segmentation of Brain Subcortical Structures From High-Field MRI]]></title>
##     <authors><![CDATA[Jinyoung Kim;  Lenglet, C.;  Duchin, Y.;  Sapiro, G.;  Harel, N.]]></authors>
##     <affiliations><![CDATA[Dept. of Electr. & Comput. Eng., Duke Univ., Durham, NC, USA]]></affiliations>
##     <controlledterms>
##       <term><![CDATA[biomedical MRI]]></term>
##       <term><![CDATA[brain]]></term>
##       <term><![CDATA[image segmentation]]></term>
##       <term><![CDATA[medical image processing]]></term>
##       <term><![CDATA[surgery]]></term>
##     </controlledterms>
##     <thesaurusterms>
##       <term><![CDATA[Basal ganglia]]></term>
##       <term><![CDATA[Computational modeling]]></term>
##       <term><![CDATA[Image edge detection]]></term>
##       <term><![CDATA[Image segmentation]]></term>
##       <term><![CDATA[Laplace equations]]></term>
##       <term><![CDATA[Magnetic resonance imaging]]></term>
##       <term><![CDATA[Shape]]></term>
##     </thesaurusterms>
##     <pubtitle><![CDATA[Biomedical and Health Informatics, IEEE Journal of]]></pubtitle>
##     <punumber><![CDATA[6221020]]></punumber>
##     <pubtype><![CDATA[Journals & Magazines]]></pubtype>
##     <publisher><![CDATA[IEEE]]></publisher>
##     <volume><![CDATA[18]]></volume>
##     <issue><![CDATA[5]]></issue>
##     <py><![CDATA[2014]]></py>
##     <spage><![CDATA[1678]]></spage>
##     <epage><![CDATA[1695]]></epage>
##     <abstract><![CDATA[Volumetric segmentation of subcortical structures, such as the basal ganglia and thalamus, is necessary for noninvasive diagnosis and neurosurgery planning. This is a challenging problem due in part to limited boundary information between structures, similar intensity profiles across the different structures, and low contrast data. This paper presents a semiautomatic segmentation system exploiting the superior image quality of ultrahigh field (7 T) MRI. The proposed approach utilizes the complementary edge information in the multiple structural MRI modalities. It combines optimally selected two modalities from susceptibility-weighted, T<sub>2</sub>-weighted, and diffusion MRI, and introduces a tailored new edge indicator function. In addition to this, we employ prior shape and configuration knowledge of the subcortical structures in order to guide the evolution of geometric active surfaces. Neighboring structures are segmented iteratively, constraining oversegmentation at their borders with a nonoverlapping penalty. Several experiments with data acquired on a 7 T MRI scanner demonstrate the feasibility and power of the approach for the segmentation of basal ganglia components critical for neurosurgery applications such as deep brain stimulation surgery.]]></abstract>
##     <issn><![CDATA[2168-2194]]></issn>
##     <htmlFlag><![CDATA[1]]></htmlFlag>
##     <arnumber><![CDATA[6676825]]></arnumber>
##     <doi><![CDATA[10.1109/JBHI.2013.2292858]]></doi>
##     <publicationId><![CDATA[6676825]]></publicationId>
##     <mdurl><![CDATA[http://ieeexplore.ieee.org/xpl/articleDetails.jsp?tp=&arnumber=6676825&contentType=Journals+%26+Magazines]]></mdurl>
##     <pdf><![CDATA[http://ieeexplore.ieee.org/stamp/stamp.jsp?arnumber=6676825]]></pdf>
##   </document>
##   <document>
##     <rank>196</rank>
##     <title><![CDATA[Histograms of Oriented Gradients for Landmine Detection in Ground-Penetrating Radar Data]]></title>
##     <authors><![CDATA[Torrione, P.A.;  Morton, K.D.;  Sakaguchi, R.;  Collins, L.M.]]></authors>
##     <affiliations><![CDATA[Dept. of Electr. & Comput. Eng., Duke Univ., Durham, NC, USA]]></affiliations>
##     <controlledterms>
##       <term><![CDATA[feature extraction]]></term>
##       <term><![CDATA[geophysical techniques]]></term>
##       <term><![CDATA[ground penetrating radar]]></term>
##       <term><![CDATA[landmine detection]]></term>
##       <term><![CDATA[remote sensing by radar]]></term>
##     </controlledterms>
##     <pubtitle><![CDATA[Geoscience and Remote Sensing, IEEE Transactions on]]></pubtitle>
##     <punumber><![CDATA[36]]></punumber>
##     <pubtype><![CDATA[Journals & Magazines]]></pubtype>
##     <publisher><![CDATA[IEEE]]></publisher>
##     <volume><![CDATA[52]]></volume>
##     <issue><![CDATA[3]]></issue>
##     <py><![CDATA[2014]]></py>
##     <spage><![CDATA[1539]]></spage>
##     <epage><![CDATA[1550]]></epage>
##     <abstract><![CDATA[Ground-penetrating radar (GPR) is a powerful and rapidly maturing technology for subsurface threat identification. However, sophisticated processing of GPR data is necessary to reduce false alarms due to naturally occurring subsurface clutter and soil distortions. Most currently fielded GPR-based landmine detection algorithms utilize feature extraction and statistical learning to develop robust classifiers capable of discriminating buried threats from inert subsurface structures. Analysis of these techniques indicates strong underlying similarities between efficient landmine detection algorithms and modern techniques for feature extraction in the computer vision literature. This paper explores the relationship between and application of one modern computer vision feature extraction technique, namely histogram of oriented gradients (HOG), to landmine detection in GPR data. The results presented indicate that HOG features provide a robust tool for target identification for both classification and prescreening and suggest that other techniques from computer vision might also be successfully applied to target detection in GPR data.]]></abstract>
##     <issn><![CDATA[0196-2892]]></issn>
##     <htmlFlag><![CDATA[1]]></htmlFlag>
##     <arnumber><![CDATA[6517972]]></arnumber>
##     <doi><![CDATA[10.1109/TGRS.2013.2252016]]></doi>
##     <publicationId><![CDATA[6517972]]></publicationId>
##     <mdurl><![CDATA[http://ieeexplore.ieee.org/xpl/articleDetails.jsp?tp=&arnumber=6517972&contentType=Journals+%26+Magazines]]></mdurl>
##     <pdf><![CDATA[http://ieeexplore.ieee.org/stamp/stamp.jsp?arnumber=6517972]]></pdf>
##   </document>
##   <document>
##     <rank>197</rank>
##     <title><![CDATA[DIA2: Web-based Cyberinfrastructure for Visual Analysis of Funding Portfolios]]></title>
##     <authors><![CDATA[Madhavan, K.;  Elmqvist, N.;  Vorvoreanu, M.;  Xin Chen;  Yuetling Wong;  Hanjun Xian;  Zhihua Dong;  Johri, A.]]></authors>
##     <affiliations><![CDATA[Purdue Univ., West Lafayette, IN, USA]]></affiliations>
##     <controlledterms>
##       <term><![CDATA[Internet]]></term>
##       <term><![CDATA[data analysis]]></term>
##       <term><![CDATA[data visualisation]]></term>
##       <term><![CDATA[financial data processing]]></term>
##       <term><![CDATA[research and development]]></term>
##     </controlledterms>
##     <thesaurusterms>
##       <term><![CDATA[Data visualization]]></term>
##       <term><![CDATA[Government]]></term>
##       <term><![CDATA[Research and development]]></term>
##       <term><![CDATA[Science - general]]></term>
##       <term><![CDATA[Visual analytics]]></term>
##       <term><![CDATA[Websites]]></term>
##     </thesaurusterms>
##     <pubtitle><![CDATA[Visualization and Computer Graphics, IEEE Transactions on]]></pubtitle>
##     <punumber><![CDATA[2945]]></punumber>
##     <pubtype><![CDATA[Journals & Magazines]]></pubtype>
##     <publisher><![CDATA[IEEE]]></publisher>
##     <volume><![CDATA[20]]></volume>
##     <issue><![CDATA[12]]></issue>
##     <py><![CDATA[2014]]></py>
##     <spage><![CDATA[1823]]></spage>
##     <epage><![CDATA[1832]]></epage>
##     <abstract><![CDATA[We present a design study of the Deep Insights Anywhere, Anytime (DIA2) platform, a web-based visual analytics system that allows program managers and academic staff at the U.S. National Science Foundation to search, view, and analyze their research funding portfolio. The goal of this system is to facilitate users' understanding of both past and currently active research awards in order to make more informed decisions of their future funding. This user group is characterized by high domain expertise yet not necessarily high literacy in visualization and visual analytics-they are essentially casual experts-and thus require careful visual and information design, including adhering to user experience standards, providing a self-instructive interface, and progressively refining visualizations to minimize complexity. We discuss the challenges of designing a system for casual experts and highlight how we addressed this issue by modeling the organizational structure and workflows of the NSF within our system. We discuss each stage of the design process, starting with formative interviews, prototypes, and finally live deployments and evaluation with stakeholders.]]></abstract>
##     <issn><![CDATA[1077-2626]]></issn>
##     <htmlFlag><![CDATA[1]]></htmlFlag>
##     <arnumber><![CDATA[6876046]]></arnumber>
##     <doi><![CDATA[10.1109/TVCG.2014.2346747]]></doi>
##     <publicationId><![CDATA[6876046]]></publicationId>
##     <mdurl><![CDATA[http://ieeexplore.ieee.org/xpl/articleDetails.jsp?tp=&arnumber=6876046&contentType=Journals+%26+Magazines]]></mdurl>
##     <pdf><![CDATA[http://ieeexplore.ieee.org/stamp/stamp.jsp?arnumber=6876046]]></pdf>
##   </document>
##   <document>
##     <rank>198</rank>
##     <title><![CDATA[Automated Lung Segmentation and Image Quality Assessment for Clinical 3-D/4-D-Computed Tomography]]></title>
##     <authors><![CDATA[Jie Wei;  Guang Li]]></authors>
##     <affiliations><![CDATA[Dept. of Comput. Sci., City Coll. of New York, New York, NY, USA]]></affiliations>
##     <controlledterms>
##       <term><![CDATA[computerised tomography]]></term>
##       <term><![CDATA[image segmentation]]></term>
##       <term><![CDATA[learning (artificial intelligence)]]></term>
##       <term><![CDATA[lung]]></term>
##       <term><![CDATA[medical image processing]]></term>
##     </controlledterms>
##     <thesaurusterms>
##       <term><![CDATA[Computed tomography]]></term>
##       <term><![CDATA[Discrete cosine transforms]]></term>
##       <term><![CDATA[Image segmentation]]></term>
##       <term><![CDATA[Lungs]]></term>
##       <term><![CDATA[Noise]]></term>
##       <term><![CDATA[Three-dimensional displays]]></term>
##       <term><![CDATA[Tumors]]></term>
##     </thesaurusterms>
##     <pubtitle><![CDATA[Translational Engineering in Health and Medicine, IEEE Journal of]]></pubtitle>
##     <punumber><![CDATA[6221039]]></punumber>
##     <pubtype><![CDATA[Journals & Magazines]]></pubtype>
##     <publisher><![CDATA[IEEE]]></publisher>
##     <volume><![CDATA[2]]></volume>
##     <py><![CDATA[2014]]></py>
##     <spage><![CDATA[1]]></spage>
##     <epage><![CDATA[10]]></epage>
##     <abstract><![CDATA[4-D-computed tomography (4DCT) provides not only a new dimension of patient-specific information for radiation therapy planning and treatment, but also a challenging scale of data volume to process and analyze. Manual analysis using existing 3-D tools is unable to keep up with vastly increased 4-D data volume, automated processing and analysis are thus needed to process 4DCT data effectively and efficiently. In this paper, we applied ideas and algorithms from image/signal processing, computer vision, and machine learning to 4DCT lung data so that lungs can be reliably segmented in a fully automated manner, lung features can be visualized and measured on the fly via user interactions, and data quality classifications can be computed in a robust manner. Comparisons of our results with an established treatment planning system and calculation by experts demonstrated negligible discrepancies (within &#x00B1;2%) for volume assessment but one to two orders of magnitude performance enhancement. An empirical Fourier-analysis-based quality measure-delivered performances closely emulating human experts. Three machine learners are inspected to justify the viability of machine learning techniques used to robustly identify data quality of 4DCT images in the scalable manner. The resultant system provides a toolkit that speeds up 4-D tasks in the clinic and facilitates clinical research to improve current clinical practice.]]></abstract>
##     <issn><![CDATA[2168-2372]]></issn>
##     <htmlFlag><![CDATA[1]]></htmlFlag>
##     <arnumber><![CDATA[6994263]]></arnumber>
##     <doi><![CDATA[10.1109/JTEHM.2014.2381213]]></doi>
##     <publicationId><![CDATA[6994263]]></publicationId>
##     <mdurl><![CDATA[http://ieeexplore.ieee.org/xpl/articleDetails.jsp?tp=&arnumber=6994263&contentType=Journals+%26+Magazines]]></mdurl>
##     <pdf><![CDATA[http://ieeexplore.ieee.org/stamp/stamp.jsp?arnumber=6994263]]></pdf>
##   </document>
##   <document>
##     <rank>199</rank>
##     <title><![CDATA[HNLF-Based Photonic Pattern Recognition Using Remote Transmitter]]></title>
##     <authors><![CDATA[Kibria, R.;  Lam Bui;  Mitchell, A.;  Austin, M.W.]]></authors>
##     <affiliations><![CDATA[Sch. of Electr. & Comput. Eng., RMIT Univ., Melbourne, VIC, Australia]]></affiliations>
##     <controlledterms>
##       <term><![CDATA[matched filters]]></term>
##       <term><![CDATA[microwave photonics]]></term>
##       <term><![CDATA[multiwave mixing]]></term>
##       <term><![CDATA[optical correlation]]></term>
##       <term><![CDATA[optical fibre communication]]></term>
##       <term><![CDATA[optical pumping]]></term>
##       <term><![CDATA[optical transmitters]]></term>
##       <term><![CDATA[pattern recognition]]></term>
##       <term><![CDATA[photodetectors]]></term>
##     </controlledterms>
##     <thesaurusterms>
##       <term><![CDATA[Correlation]]></term>
##       <term><![CDATA[Delays]]></term>
##       <term><![CDATA[Optical fibers]]></term>
##       <term><![CDATA[Optical transmitters]]></term>
##       <term><![CDATA[Photodetectors]]></term>
##       <term><![CDATA[Photonics]]></term>
##     </thesaurusterms>
##     <pubtitle><![CDATA[Photonics Technology Letters, IEEE]]></pubtitle>
##     <punumber><![CDATA[68]]></punumber>
##     <pubtype><![CDATA[Journals & Magazines]]></pubtype>
##     <publisher><![CDATA[IEEE]]></publisher>
##     <volume><![CDATA[26]]></volume>
##     <issue><![CDATA[5]]></issue>
##     <py><![CDATA[2014]]></py>
##     <spage><![CDATA[457]]></spage>
##     <epage><![CDATA[460]]></epage>
##     <abstract><![CDATA[In this letter, a novel correlation technique based on matched filtering using four wave mixing with possible remote transmission is proposed. A pump wavelength, which is modulated by an input bit pattern is mixed with wavelength channels in a highly nonlinear fiber to generate idler wavelengths, which also carry the bit stream. Specific idler wavelengths are selected using software control of an optical processor to generate the reference template. These idlers are then differentially delayed and summed at a photodetector to produce the required correlation function. This scheme has been experimentally demonstrated and the measured results verify the proposed concept.]]></abstract>
##     <issn><![CDATA[1041-1135]]></issn>
##     <htmlFlag><![CDATA[1]]></htmlFlag>
##     <arnumber><![CDATA[6701195]]></arnumber>
##     <doi><![CDATA[10.1109/LPT.2013.2296506]]></doi>
##     <publicationId><![CDATA[6701195]]></publicationId>
##     <mdurl><![CDATA[http://ieeexplore.ieee.org/xpl/articleDetails.jsp?tp=&arnumber=6701195&contentType=Journals+%26+Magazines]]></mdurl>
##     <pdf><![CDATA[http://ieeexplore.ieee.org/stamp/stamp.jsp?arnumber=6701195]]></pdf>
##   </document>
##   <document>
##     <rank>200</rank>
##     <title><![CDATA[Breakthroughs in Photonics 2013: Passive Mode-Locking of Quantum-Dot Lasers]]></title>
##     <authors><![CDATA[Arsenijevic&#x0301; , D.;  Kleinert, M.;  Bimberg, D.]]></authors>
##     <affiliations><![CDATA[Dept. of Solid-State Phys., Tech. Univ. Berlin, Berlin, Germany]]></affiliations>
##     <controlledterms>
##       <term><![CDATA[jitter]]></term>
##       <term><![CDATA[laser feedback]]></term>
##       <term><![CDATA[laser mode locking]]></term>
##       <term><![CDATA[laser tuning]]></term>
##       <term><![CDATA[quantum dot lasers]]></term>
##     </controlledterms>
##     <thesaurusterms>
##       <term><![CDATA[Jitter]]></term>
##       <term><![CDATA[Laser mode locking]]></term>
##       <term><![CDATA[Optical attenuators]]></term>
##       <term><![CDATA[Optical feedback]]></term>
##       <term><![CDATA[Quantum dot lasers]]></term>
##       <term><![CDATA[Tuning]]></term>
##     </thesaurusterms>
##     <pubtitle><![CDATA[Photonics Journal, IEEE]]></pubtitle>
##     <punumber><![CDATA[4563994]]></punumber>
##     <pubtype><![CDATA[Journals & Magazines]]></pubtype>
##     <publisher><![CDATA[IEEE]]></publisher>
##     <volume><![CDATA[6]]></volume>
##     <issue><![CDATA[2]]></issue>
##     <py><![CDATA[2014]]></py>
##     <spage><![CDATA[1]]></spage>
##     <epage><![CDATA[6]]></epage>
##     <abstract><![CDATA[Most recent achievements in passive mode-locking of quantum-dot lasers, with the main focus on jitter reduction and frequency tuning, are described. Different techniques, leading to record values for integrated jitter of 121 fs and a locking range of 342 MHz, are presented for a 40-GHz laser. Optical feedback is observed to be the method of choice in this field. For the first time, five different optical-feedback regimes are discovered, including the resonant one yielding a radio-frequency linewidth reduction by 99%.]]></abstract>
##     <issn><![CDATA[1943-0655]]></issn>
##     <htmlFlag><![CDATA[1]]></htmlFlag>
##     <arnumber><![CDATA[6747957]]></arnumber>
##     <doi><![CDATA[10.1109/JPHOT.2014.2308195]]></doi>
##     <publicationId><![CDATA[6747957]]></publicationId>
##     <mdurl><![CDATA[http://ieeexplore.ieee.org/xpl/articleDetails.jsp?tp=&arnumber=6747957&contentType=Journals+%26+Magazines]]></mdurl>
##     <pdf><![CDATA[http://ieeexplore.ieee.org/stamp/stamp.jsp?arnumber=6747957]]></pdf>
##   </document>
##   <document>
##     <rank>201</rank>
##     <title><![CDATA[A Reconfigurable High-Order UWB Signal Generation Scheme Using RSOA-MZI Structure]]></title>
##     <authors><![CDATA[Hanlin Feng;  Fok, M.P.;  Shilin Xiao;  Jia Ge;  Qi Zhou;  Locke, M.;  Toole, R.;  Weisheng Hu]]></authors>
##     <affiliations><![CDATA[State Key Lab. of Adv. Opt. Commun. Syst. & Networks, Shanghai Jiao Tong Univ., Shanghai, China]]></affiliations>
##     <controlledterms>
##       <term><![CDATA[Mach-Zehnder interferometers]]></term>
##       <term><![CDATA[microwave photonics]]></term>
##       <term><![CDATA[optical fibres]]></term>
##       <term><![CDATA[semiconductor optical amplifiers]]></term>
##       <term><![CDATA[ultra wideband technology]]></term>
##     </controlledterms>
##     <thesaurusterms>
##       <term><![CDATA[Microwave filters]]></term>
##       <term><![CDATA[Optical attenuators]]></term>
##       <term><![CDATA[Optical fiber amplifiers]]></term>
##       <term><![CDATA[Optical fiber filters]]></term>
##       <term><![CDATA[Optical pulses]]></term>
##       <term><![CDATA[Photonics]]></term>
##     </thesaurusterms>
##     <pubtitle><![CDATA[Photonics Journal, IEEE]]></pubtitle>
##     <punumber><![CDATA[4563994]]></punumber>
##     <pubtype><![CDATA[Journals & Magazines]]></pubtype>
##     <publisher><![CDATA[IEEE]]></publisher>
##     <volume><![CDATA[6]]></volume>
##     <issue><![CDATA[2]]></issue>
##     <py><![CDATA[2014]]></py>
##     <spage><![CDATA[1]]></spage>
##     <epage><![CDATA[7]]></epage>
##     <abstract><![CDATA[We propose and experimentally demonstrate a reflective semiconductor optical amplifier (RSOA)-based Mach-Zehnder interferometer structure for generating high-order ultrawideband (UWB) signals. First, low-order UWB signals are generated by controlling the bias current of the RSOA. To convert the generated low-order UWB signals to high-order UWB signals, the polarization overlaying technique is used in an interferometer structure. The generated high-order UWB signals well satisfy Federal Communications Commission spectral regulations, and the interference with GPS band at 1.57542 GHz is suppressed.]]></abstract>
##     <issn><![CDATA[1943-0655]]></issn>
##     <htmlFlag><![CDATA[1]]></htmlFlag>
##     <arnumber><![CDATA[6746648]]></arnumber>
##     <doi><![CDATA[10.1109/JPHOT.2014.2306832]]></doi>
##     <publicationId><![CDATA[6746648]]></publicationId>
##     <mdurl><![CDATA[http://ieeexplore.ieee.org/xpl/articleDetails.jsp?tp=&arnumber=6746648&contentType=Journals+%26+Magazines]]></mdurl>
##     <pdf><![CDATA[http://ieeexplore.ieee.org/stamp/stamp.jsp?arnumber=6746648]]></pdf>
##   </document>
##   <document>
##     <rank>202</rank>
##     <title><![CDATA[Unconventional Production of Bright White Light Emission by Nd-Doped and Nominally Un-Doped <inline-formula> <img src="/images/tex/951.gif" alt="\hbox {Y}_{2}\hbox {O}_{3}"> </inline-formula> Nano-Powders]]></title>
##     <authors><![CDATA[Bilir, G.;  Ozen, G.;  Collins, J.;  Cesaria, M.;  Di Bartolo, B.]]></authors>
##     <affiliations><![CDATA[Dept. of Phys., Istanbul Tech. Univ., Istanbul, Turkey]]></affiliations>
##     <controlledterms>
##       <term><![CDATA[nanoparticles]]></term>
##       <term><![CDATA[nanophotonics]]></term>
##       <term><![CDATA[neodymium]]></term>
##       <term><![CDATA[optical materials]]></term>
##       <term><![CDATA[optical pumping]]></term>
##       <term><![CDATA[photoluminescence]]></term>
##       <term><![CDATA[yttrium compounds]]></term>
##     </controlledterms>
##     <thesaurusterms>
##       <term><![CDATA[Crystals]]></term>
##       <term><![CDATA[Diffraction]]></term>
##       <term><![CDATA[Ions]]></term>
##       <term><![CDATA[Neodymium]]></term>
##       <term><![CDATA[Temperature measurement]]></term>
##       <term><![CDATA[X-ray diffraction]]></term>
##       <term><![CDATA[X-ray scattering]]></term>
##     </thesaurusterms>
##     <pubtitle><![CDATA[Photonics Journal, IEEE]]></pubtitle>
##     <punumber><![CDATA[4563994]]></punumber>
##     <pubtype><![CDATA[Journals & Magazines]]></pubtype>
##     <publisher><![CDATA[IEEE]]></publisher>
##     <volume><![CDATA[6]]></volume>
##     <issue><![CDATA[4]]></issue>
##     <py><![CDATA[2014]]></py>
##     <spage><![CDATA[1]]></spage>
##     <epage><![CDATA[17]]></epage>
##     <abstract><![CDATA[We report the production of a broad band (ranging from 400 to 900 nm) white light following the monochromatic infrared light (803.5 and 975 nm) excitation of both nominally un-doped and Nd<sup>3+</sup>-doped Y<sub>2</sub>O<sub>3</sub> nano-powders, even up to 20% of Nd<sup>3+</sup> content. Experimental results indicate that such emission feature is a nano-scale phenomenon, cannot be ascribed to an overlap of sharp emission bands in the un-doped case and, even if assisted by the Nd<sup>34</sup> presence, is a host matrix-related process. The measured white light emission is strongly dependent on either environment pressure (a pressure threshold occurs) or pumping power. The rising patterns of the white light emission were found to increase faster for either increasing Nd<sup>3+</sup> content and pumping power or decreasing particle size. Notably, high correlated color temperature (2756 K), color rendering index (99), and efficiency (864 lx/W) values were measured for the un-doped sample under 803.5 nm exciting wavelength.]]></abstract>
##     <issn><![CDATA[1943-0655]]></issn>
##     <htmlFlag><![CDATA[1]]></htmlFlag>
##     <arnumber><![CDATA[6857366]]></arnumber>
##     <doi><![CDATA[10.1109/JPHOT.2014.2339312]]></doi>
##     <publicationId><![CDATA[6857366]]></publicationId>
##     <mdurl><![CDATA[http://ieeexplore.ieee.org/xpl/articleDetails.jsp?tp=&arnumber=6857366&contentType=Journals+%26+Magazines]]></mdurl>
##     <pdf><![CDATA[http://ieeexplore.ieee.org/stamp/stamp.jsp?arnumber=6857366]]></pdf>
##   </document>
##   <document>
##     <rank>203</rank>
##     <title><![CDATA[Autonomous Robots and the SP Theory of Intelligence]]></title>
##     <authors><![CDATA[Wolff, J.G.]]></authors>
##     <affiliations><![CDATA[CognitionResearch.org, Menai Bridge, UK]]></affiliations>
##     <controlledterms>
##       <term><![CDATA[energy conservation]]></term>
##       <term><![CDATA[learning by example]]></term>
##       <term><![CDATA[mobile robots]]></term>
##       <term><![CDATA[unsupervised learning]]></term>
##     </controlledterms>
##     <thesaurusterms>
##       <term><![CDATA[Artificial intelligence]]></term>
##       <term><![CDATA[Cognitive science]]></term>
##       <term><![CDATA[Computational efficiency]]></term>
##       <term><![CDATA[Data compression]]></term>
##       <term><![CDATA[Learning systems]]></term>
##       <term><![CDATA[Man machine interfaces]]></term>
##       <term><![CDATA[Mobile communication]]></term>
##       <term><![CDATA[Mobile robots]]></term>
##       <term><![CDATA[Pattern recognition]]></term>
##       <term><![CDATA[Power supplies]]></term>
##       <term><![CDATA[Robots]]></term>
##       <term><![CDATA[Unsupervised learning]]></term>
##     </thesaurusterms>
##     <pubtitle><![CDATA[Access, IEEE]]></pubtitle>
##     <punumber><![CDATA[6287639]]></punumber>
##     <pubtype><![CDATA[Journals & Magazines]]></pubtype>
##     <publisher><![CDATA[IEEE]]></publisher>
##     <volume><![CDATA[2]]></volume>
##     <py><![CDATA[2014]]></py>
##     <spage><![CDATA[1629]]></spage>
##     <epage><![CDATA[1651]]></epage>
##     <abstract><![CDATA[This paper is about how the SP theory of intelligence and its realization in the SP machine (both outlined in this paper) may help in the design of the brains of autonomous robots, meaning robots that do not depend on external intelligence or power supplies, are mobile, and have human-like versatility and adaptability in intelligence. This paper addresses three main problems: 1) how to increase the computational and energy efficiency of computers and to reduce their size and weight; 2) how to achieve human-like versatility in intelligence; and 3) likewise for human-like adaptability in intelligence. Regarding the first problem, the SP system has the potential for substantial gains in computational efficiency, with corresponding cuts in energy consumption and the bulkiness of computers: 1) by reducing the size of data to be processed; 2) by exploiting statistical information that the system gathers as an integral part of how it works; and 3) via a new version of Donald Hebb's concept of a cell assembly. Toward human-like versatility in intelligence, the SP system has strengths in unsupervised learning, natural language processing, pattern recognition, information retrieval, several kinds of reasoning, planning, problem solving, and more, with seamless integration among structures and functions. The SP system's strengths in unsupervised learning and other aspects of intelligence may help in achieving human-like adaptability in intelligence via: 1) one-trial learning; 2) learning of natural language; 3) learning to see; 4) building 3-D models of objects and of a robot's surroundings; 5) learning regularities in the workings of a robot and in the robot's environment; 6) exploration and play; 7) learning major skills; and 8) learning via demonstration. Also discussed are how the SP system may process parallel streams of information, generalization of knowledge, correction of over-generalizations, learning from dirty data, how to cut the cost of learning, and reinfor- ements and motivations.]]></abstract>
##     <issn><![CDATA[2169-3536]]></issn>
##     <htmlFlag><![CDATA[1]]></htmlFlag>
##     <arnumber><![CDATA[6990727]]></arnumber>
##     <doi><![CDATA[10.1109/ACCESS.2014.2382753]]></doi>
##     <publicationId><![CDATA[6990727]]></publicationId>
##     <mdurl><![CDATA[http://ieeexplore.ieee.org/xpl/articleDetails.jsp?tp=&arnumber=6990727&contentType=Journals+%26+Magazines]]></mdurl>
##     <pdf><![CDATA[http://ieeexplore.ieee.org/stamp/stamp.jsp?arnumber=6990727]]></pdf>
##   </document>
##   <document>
##     <rank>204</rank>
##     <title><![CDATA[Unexpected Strong Optical Absorption in a Free-Standing Optical Thick Transmission Metallic Grating]]></title>
##     <authors><![CDATA[Hu-Quan Li;  Jin-Song Liu]]></authors>
##     <affiliations><![CDATA[Wuhan Nat. Lab. for Optoelectron., Huazhong Univ. of Sci. & Technol., Wuhan, China]]></affiliations>
##     <controlledterms>
##       <term><![CDATA[Fabry-Perot resonators]]></term>
##       <term><![CDATA[diffraction gratings]]></term>
##       <term><![CDATA[light transmission]]></term>
##       <term><![CDATA[metallic thin films]]></term>
##       <term><![CDATA[plasmonics]]></term>
##     </controlledterms>
##     <thesaurusterms>
##       <term><![CDATA[Absorption]]></term>
##       <term><![CDATA[Gratings]]></term>
##       <term><![CDATA[Optical coupling]]></term>
##       <term><![CDATA[Optical polarization]]></term>
##       <term><![CDATA[Optical reflection]]></term>
##       <term><![CDATA[Structural engineering]]></term>
##     </thesaurusterms>
##     <pubtitle><![CDATA[Photonics Journal, IEEE]]></pubtitle>
##     <punumber><![CDATA[4563994]]></punumber>
##     <pubtype><![CDATA[Journals & Magazines]]></pubtype>
##     <publisher><![CDATA[IEEE]]></publisher>
##     <volume><![CDATA[6]]></volume>
##     <issue><![CDATA[2]]></issue>
##     <py><![CDATA[2014]]></py>
##     <spage><![CDATA[1]]></spage>
##     <epage><![CDATA[8]]></epage>
##     <abstract><![CDATA[In this paper, we predict unexpected optical absorption in an optical thick free-standing transmission metallic grating by introducing periodical back grooves into it. The absorption peak could be very high up to 90%. In the structure, the introduced periodical back grooves bring about different orders of transmission minima into the transmission spectrum, which are named as transmission minima introduced by back grooves (TMIBBGs) here. When TMIBBGs approach the Fabry-Pe&#x0301;rot (FP)-like resonant transmission peaks supported in the slits, the latter will be strongly suppressed. However, the reflection dips are almost unaffected. As a result, absorption is greatly enhanced. More interesting, the positions of TMIBBGs can be tuned by modulating the structural parameters of back grooves to selectively inhibit different orders of FP-like resonant transmission peaks and realize corresponding enhanced absorption, meanwhile, without influence to other orders of FP-like resonant transmissions. Moreover, due to the asymmetry of the structure, the enhanced absorption is unidirectional. The physical mechanisms are qualitatively and quantitatively analyzed in this paper.]]></abstract>
##     <issn><![CDATA[1943-0655]]></issn>
##     <htmlFlag><![CDATA[1]]></htmlFlag>
##     <arnumber><![CDATA[6746033]]></arnumber>
##     <doi><![CDATA[10.1109/JPHOT.2014.2306833]]></doi>
##     <publicationId><![CDATA[6746033]]></publicationId>
##     <mdurl><![CDATA[http://ieeexplore.ieee.org/xpl/articleDetails.jsp?tp=&arnumber=6746033&contentType=Journals+%26+Magazines]]></mdurl>
##     <pdf><![CDATA[http://ieeexplore.ieee.org/stamp/stamp.jsp?arnumber=6746033]]></pdf>
##   </document>
##   <document>
##     <rank>205</rank>
##     <title><![CDATA[Evolved Machines Shed Light on Robustness and Resilience]]></title>
##     <authors><![CDATA[Bongard, J.;  Lipson, H.]]></authors>
##     <affiliations><![CDATA[Dept. of Comput. Sci., Univ. of Vermont, Burlington, VT, USA]]></affiliations>
##     <controlledterms>
##       <term><![CDATA[adaptive systems]]></term>
##       <term><![CDATA[biomimetics]]></term>
##       <term><![CDATA[evolutionary computation]]></term>
##       <term><![CDATA[neurophysiology]]></term>
##       <term><![CDATA[robots]]></term>
##     </controlledterms>
##     <thesaurusterms>
##       <term><![CDATA[Cognition]]></term>
##       <term><![CDATA[Computer architecture]]></term>
##       <term><![CDATA[Evolutionary computation]]></term>
##       <term><![CDATA[Legged locomotion]]></term>
##       <term><![CDATA[Morphology]]></term>
##       <term><![CDATA[Neuroscience]]></term>
##       <term><![CDATA[Robot kinematics]]></term>
##       <term><![CDATA[Robot sensing systems]]></term>
##     </thesaurusterms>
##     <pubtitle><![CDATA[Proceedings of the IEEE]]></pubtitle>
##     <punumber><![CDATA[5]]></punumber>
##     <pubtype><![CDATA[Journals & Magazines]]></pubtype>
##     <publisher><![CDATA[IEEE]]></publisher>
##     <volume><![CDATA[102]]></volume>
##     <issue><![CDATA[5]]></issue>
##     <py><![CDATA[2014]]></py>
##     <spage><![CDATA[899]]></spage>
##     <epage><![CDATA[914]]></epage>
##     <abstract><![CDATA[In biomimetic engineering, we may take inspiration from the products of biological evolution: we may instantiate biologically realistic neural architectures and algorithms in robots, or we may construct robots with morphologies that are found in nature. Alternatively, we may take inspiration from the process of evolution: we may evolve populations of robots in simulation and then manufacture physical versions of the most interesting or more capable robots that evolve. If we follow this latter approach and evolve both the neural and morphological subsystems of machines, we can perform controlled experiments that provide unique insight into how bodies and brains can work together to produce adaptive behavior, regardless of whether such bodies and brains are instantiated in a biological or technological substrate. In this paper, we review selected projects that use such methods to investigate the synergies and tradeoffs between neural architecture, morphology, action, and adaptive behavior.]]></abstract>
##     <issn><![CDATA[0018-9219]]></issn>
##     <arnumber><![CDATA[6783985]]></arnumber>
##     <doi><![CDATA[10.1109/JPROC.2014.2312844]]></doi>
##     <publicationId><![CDATA[6783985]]></publicationId>
##     <mdurl><![CDATA[http://ieeexplore.ieee.org/xpl/articleDetails.jsp?tp=&arnumber=6783985&contentType=Journals+%26+Magazines]]></mdurl>
##     <pdf><![CDATA[http://ieeexplore.ieee.org/stamp/stamp.jsp?arnumber=6783985]]></pdf>
##   </document>
##   <document>
##     <rank>206</rank>
##     <title><![CDATA[Imaging Properties of Kinoform Fibonacci Lenses]]></title>
##     <authors><![CDATA[Ferrando, V.;  Calatayud, A.;  Andres, P.;  Torroba, R.;  Furlan, W.D.;  Monsoriu, J.A.]]></authors>
##     <affiliations><![CDATA[Centro de Tecnol. Fisicas, Univ. Politec. de Vale`ncia, Valencia, Spain]]></affiliations>
##     <controlledterms>
##       <term><![CDATA[Fibonacci sequences]]></term>
##       <term><![CDATA[computer-generated holography]]></term>
##       <term><![CDATA[diffractive optical elements]]></term>
##       <term><![CDATA[lenses]]></term>
##       <term><![CDATA[optical images]]></term>
##       <term><![CDATA[spatial light modulators]]></term>
##     </controlledterms>
##     <thesaurusterms>
##       <term><![CDATA[Diffraction]]></term>
##       <term><![CDATA[Focusing]]></term>
##       <term><![CDATA[Lenses]]></term>
##       <term><![CDATA[Optical diffraction]]></term>
##       <term><![CDATA[Optical imaging]]></term>
##       <term><![CDATA[Quantum cascade lasers]]></term>
##     </thesaurusterms>
##     <pubtitle><![CDATA[Photonics Journal, IEEE]]></pubtitle>
##     <punumber><![CDATA[4563994]]></punumber>
##     <pubtype><![CDATA[Journals & Magazines]]></pubtype>
##     <publisher><![CDATA[IEEE]]></publisher>
##     <volume><![CDATA[6]]></volume>
##     <issue><![CDATA[1]]></issue>
##     <py><![CDATA[2014]]></py>
##     <spage><![CDATA[1]]></spage>
##     <epage><![CDATA[6]]></epage>
##     <abstract><![CDATA[In this paper, we present a new kind of bifocal kinoform lenses in which the phase distribution is based on the Fibonacci sequence. The focusing properties of these DOEs coined Kinoform Fibonacci lenses (KFLs) are analytically studied and compared with binary-phase Fibonacci lenses (FLs). It is shown that, under monochromatic illumination, a KFL drives most of the incoming light into two single foci, improving in this way the efficiency of the FLs. We have also implemented these lenses with a spatial light modulator. The first images obtained with this type of lenses are presented and evaluated.]]></abstract>
##     <issn><![CDATA[1943-0655]]></issn>
##     <htmlFlag><![CDATA[1]]></htmlFlag>
##     <arnumber><![CDATA[6731530]]></arnumber>
##     <doi><![CDATA[10.1109/JPHOT.2014.2304560]]></doi>
##     <publicationId><![CDATA[6731530]]></publicationId>
##     <mdurl><![CDATA[http://ieeexplore.ieee.org/xpl/articleDetails.jsp?tp=&arnumber=6731530&contentType=Journals+%26+Magazines]]></mdurl>
##     <pdf><![CDATA[http://ieeexplore.ieee.org/stamp/stamp.jsp?arnumber=6731530]]></pdf>
##   </document>
##   <document>
##     <rank>207</rank>
##     <title><![CDATA[Magnetic Field Sensing Based on Magnetic-Fluid-Clad Fiber-Optic Structure With Up-Tapered Joints]]></title>
##     <authors><![CDATA[Shengli Pu;  Shaohua Dong]]></authors>
##     <affiliations><![CDATA[Coll. of Sci., Univ. of Shanghai for Sci. & Technol., Shanghai, China]]></affiliations>
##     <controlledterms>
##       <term><![CDATA[fibre optic sensors]]></term>
##       <term><![CDATA[light interference]]></term>
##       <term><![CDATA[magnetic fluids]]></term>
##       <term><![CDATA[magneto-optical sensors]]></term>
##       <term><![CDATA[optical fibre cladding]]></term>
##       <term><![CDATA[optical fibre losses]]></term>
##     </controlledterms>
##     <thesaurusterms>
##       <term><![CDATA[Interference]]></term>
##       <term><![CDATA[Joints]]></term>
##       <term><![CDATA[Optical sensors]]></term>
##       <term><![CDATA[Perpendicular magnetic anisotropy]]></term>
##       <term><![CDATA[Sensitivity]]></term>
##     </thesaurusterms>
##     <pubtitle><![CDATA[Photonics Journal, IEEE]]></pubtitle>
##     <punumber><![CDATA[4563994]]></punumber>
##     <pubtype><![CDATA[Journals & Magazines]]></pubtype>
##     <publisher><![CDATA[IEEE]]></publisher>
##     <volume><![CDATA[6]]></volume>
##     <issue><![CDATA[4]]></issue>
##     <py><![CDATA[2014]]></py>
##     <spage><![CDATA[1]]></spage>
##     <epage><![CDATA[6]]></epage>
##     <abstract><![CDATA[A kind of magnetic field sensor based on mode interference effect is proposed. The sensing arm consists of two special up-tapered joints formed on the traditional single-mode fiber by fusion tapering technique. Magnetic fluid is used as the cladding of the structure. The interference valley wavelength or the transmission loss of the sensing structure is sensitive to the external magnetic field, which is utilized for magnetic field sensing. The sensitivity of shift of interference valley wavelength with magnetic field can be up to 325.3 pm/mT in the range of 0-16 mT. The variation of transmission loss at valley wavelength with magnetic field has a sensitivity of -0.2121 dB/mT in the range of 2-30 mT.]]></abstract>
##     <issn><![CDATA[1943-0655]]></issn>
##     <htmlFlag><![CDATA[1]]></htmlFlag>
##     <arnumber><![CDATA[6842631]]></arnumber>
##     <doi><![CDATA[10.1109/JPHOT.2014.2332476]]></doi>
##     <publicationId><![CDATA[6842631]]></publicationId>
##     <mdurl><![CDATA[http://ieeexplore.ieee.org/xpl/articleDetails.jsp?tp=&arnumber=6842631&contentType=Journals+%26+Magazines]]></mdurl>
##     <pdf><![CDATA[http://ieeexplore.ieee.org/stamp/stamp.jsp?arnumber=6842631]]></pdf>
##   </document>
##   <document>
##     <rank>208</rank>
##     <title><![CDATA[Learning Layouts for Single-PageGraphic Designs]]></title>
##     <authors><![CDATA[O'Donovan, P.;  Agarwala, A.;  Hertzmann, A.]]></authors>
##     <affiliations><![CDATA[Dept. of Comput. Sci., Univ. of Toronto, Toronto, ON, Canada]]></affiliations>
##     <controlledterms>
##       <term><![CDATA[computer graphics]]></term>
##       <term><![CDATA[optimisation]]></term>
##     </controlledterms>
##     <thesaurusterms>
##       <term><![CDATA[Algorithm design and analysis]]></term>
##       <term><![CDATA[Computational modeling]]></term>
##       <term><![CDATA[Face]]></term>
##       <term><![CDATA[Layout]]></term>
##       <term><![CDATA[Optimization]]></term>
##       <term><![CDATA[Predictive models]]></term>
##     </thesaurusterms>
##     <pubtitle><![CDATA[Visualization and Computer Graphics, IEEE Transactions on]]></pubtitle>
##     <punumber><![CDATA[2945]]></punumber>
##     <pubtype><![CDATA[Journals & Magazines]]></pubtype>
##     <publisher><![CDATA[IEEE]]></publisher>
##     <volume><![CDATA[20]]></volume>
##     <issue><![CDATA[8]]></issue>
##     <py><![CDATA[2014]]></py>
##     <spage><![CDATA[1200]]></spage>
##     <epage><![CDATA[1213]]></epage>
##     <abstract><![CDATA[This paper presents an approach for automatically creating graphic design layouts using a new energy-based model derived from design principles. The model includes several new algorithms for analyzing graphic designs, including the prediction of perceived importance, alignment detection, and hierarchical segmentation. Given the model, we use optimization to synthesize new layouts for a variety of single-page graphic designs. Model parameters are learned with Nonlinear Inverse Optimization (NIO) from a small number of example layouts. To demonstrate our approach, we show results for applications including generating design layouts in various styles, retargeting designs to new sizes, and improving existing designs. We also compare our automatic results with designs created using crowdsourcing and show that our approach performs slightly better than novice designers.]]></abstract>
##     <issn><![CDATA[1077-2626]]></issn>
##     <htmlFlag><![CDATA[1]]></htmlFlag>
##     <arnumber><![CDATA[6777138]]></arnumber>
##     <doi><![CDATA[10.1109/TVCG.2014.48]]></doi>
##     <publicationId><![CDATA[6777138]]></publicationId>
##     <mdurl><![CDATA[http://ieeexplore.ieee.org/xpl/articleDetails.jsp?tp=&arnumber=6777138&contentType=Journals+%26+Magazines]]></mdurl>
##     <pdf><![CDATA[http://ieeexplore.ieee.org/stamp/stamp.jsp?arnumber=6777138]]></pdf>
##   </document>
##   <document>
##     <rank>209</rank>
##     <title><![CDATA[Propulsion Mechanism of Catalytic Microjet Engines]]></title>
##     <authors><![CDATA[Fomin, V.M.;  Hippler, M.;  Magdanz, V.;  Soler, L.;  Sanchez, S.;  Schmidt, O.G.]]></authors>
##     <affiliations><![CDATA[Inst. for Integrative Nanosci., Leibniz Inst. for Solid State & Mater. Res. Dresden, Dresden, Germany]]></affiliations>
##     <controlledterms>
##       <term><![CDATA[capillarity]]></term>
##       <term><![CDATA[drag]]></term>
##       <term><![CDATA[jet engines]]></term>
##       <term><![CDATA[micromotors]]></term>
##       <term><![CDATA[microrobots]]></term>
##       <term><![CDATA[nanotechnology]]></term>
##       <term><![CDATA[propulsion]]></term>
##     </controlledterms>
##     <thesaurusterms>
##       <term><![CDATA[Electron tubes]]></term>
##       <term><![CDATA[Engines]]></term>
##       <term><![CDATA[Force]]></term>
##       <term><![CDATA[Iron]]></term>
##       <term><![CDATA[Propulsion]]></term>
##       <term><![CDATA[Shape]]></term>
##       <term><![CDATA[Surface tension]]></term>
##     </thesaurusterms>
##     <pubtitle><![CDATA[Robotics, IEEE Transactions on]]></pubtitle>
##     <punumber><![CDATA[8860]]></punumber>
##     <pubtype><![CDATA[Journals & Magazines]]></pubtype>
##     <publisher><![CDATA[IEEE]]></publisher>
##     <volume><![CDATA[30]]></volume>
##     <issue><![CDATA[1]]></issue>
##     <py><![CDATA[2014]]></py>
##     <spage><![CDATA[40]]></spage>
##     <epage><![CDATA[48]]></epage>
##     <abstract><![CDATA[We describe the propulsion mechanism of the catalytic microjet engines that are fabricated using rolled-up nanotech. Microjets have recently shown numerous potential applications in nanorobotics but currently there is a lack of an accurate theoretical model that describes the origin of the motion as well as the mechanism of self-propulsion. The geometric asymmetry of a tubular microjet leads to the development of a capillary force, which tends to propel a bubble toward the larger opening of the tube. Because of this motion in an asymmetric tube, there emerges a momentum transfer to the fluid. In order to compensate this momentum transfer, a jet force acting on the tube occurs. This force, which is counterbalanced by the linear drag force, enables tube velocities of the order of 100 &#x03BC;m/s. This mechanism provides a fundamental explanation for the development of driving forces that are acting on bubbles in tubular microjets.]]></abstract>
##     <issn><![CDATA[1552-3098]]></issn>
##     <htmlFlag><![CDATA[1]]></htmlFlag>
##     <arnumber><![CDATA[6646273]]></arnumber>
##     <doi><![CDATA[10.1109/TRO.2013.2283929]]></doi>
##     <publicationId><![CDATA[6646273]]></publicationId>
##     <mdurl><![CDATA[http://ieeexplore.ieee.org/xpl/articleDetails.jsp?tp=&arnumber=6646273&contentType=Journals+%26+Magazines]]></mdurl>
##     <pdf><![CDATA[http://ieeexplore.ieee.org/stamp/stamp.jsp?arnumber=6646273]]></pdf>
##   </document>
##   <document>
##     <rank>210</rank>
##     <title><![CDATA[Trustworthiness of Medical Devices and Body Area Networks]]></title>
##     <authors><![CDATA[Meng Zhang;  Raghunathan, A.;  Jha, N.K.]]></authors>
##     <affiliations><![CDATA[Dept. of Electr. Eng., Princeton Univ., Princeton, NJ, USA]]></affiliations>
##     <controlledterms>
##       <term><![CDATA[biomedical telemetry]]></term>
##       <term><![CDATA[body area networks]]></term>
##       <term><![CDATA[health care]]></term>
##       <term><![CDATA[telecommunication security]]></term>
##     </controlledterms>
##     <thesaurusterms>
##       <term><![CDATA[Biomedical monitoring]]></term>
##       <term><![CDATA[Body area networks]]></term>
##       <term><![CDATA[Integrated circuit modeling]]></term>
##       <term><![CDATA[Medical diagnostic imaging]]></term>
##       <term><![CDATA[Medical services]]></term>
##       <term><![CDATA[Security]]></term>
##       <term><![CDATA[Software engineering]]></term>
##     </thesaurusterms>
##     <pubtitle><![CDATA[Proceedings of the IEEE]]></pubtitle>
##     <punumber><![CDATA[5]]></punumber>
##     <pubtype><![CDATA[Journals & Magazines]]></pubtype>
##     <publisher><![CDATA[IEEE]]></publisher>
##     <volume><![CDATA[102]]></volume>
##     <issue><![CDATA[8]]></issue>
##     <py><![CDATA[2014]]></py>
##     <spage><![CDATA[1174]]></spage>
##     <epage><![CDATA[1188]]></epage>
##     <abstract><![CDATA[Implantable and wearable medical devices (IWMDs) are commonly used for diagnosing, monitoring, and treating various medical conditions. A general trend in these medical devices is toward increased functional complexity, software programmability, and connectivity to body area networks (BANs). However, as IWMDs become more &#x201C;intelligent,&#x201D; they also become less trustworthy-less reliable and more prone to attacks. Various shortcomings-hardware failures, software errors, wireless attacks, malware and software exploits, and side-channel attacks-could undermine the trustworthiness of IWMDs and BANs. While these concerns have been recognized for some time, recent demonstrations of security attacks on commercial products, e.g., pacemakers and insulin pumps, have elevated medical device security from the realm of theoretical possibility to an immediate concern. The trustworthiness of IWMDs must be addressed aggressively and proactively due to the potential for catastrophic consequences. Conventional fault tolerance and information security solutions, e.g., redundancy and cryptography, that have been employed in general-purpose and embedded computing systems cannot be applied to many IWMDs due to their extreme size and power constraints and unique usage models. While several recent efforts address defense of IWMDs against specific security attacks, a holistic strategy that considers all concerns and types of threats is required. This paper discusses trustworthiness concerns in IWMDs and BANs through a comprehensive identification and analysis of potential threats and, for each threat, provides a discussion of the merits and inadequacies of current solutions.]]></abstract>
##     <issn><![CDATA[0018-9219]]></issn>
##     <htmlFlag><![CDATA[1]]></htmlFlag>
##     <arnumber><![CDATA[6827202]]></arnumber>
##     <doi><![CDATA[10.1109/JPROC.2014.2322103]]></doi>
##     <publicationId><![CDATA[6827202]]></publicationId>
##     <mdurl><![CDATA[http://ieeexplore.ieee.org/xpl/articleDetails.jsp?tp=&arnumber=6827202&contentType=Journals+%26+Magazines]]></mdurl>
##     <pdf><![CDATA[http://ieeexplore.ieee.org/stamp/stamp.jsp?arnumber=6827202]]></pdf>
##   </document>
##   <document>
##     <rank>211</rank>
##     <title><![CDATA[Monitoring and Analysis of Respiratory Patterns Using Microwave Doppler Radar]]></title>
##     <authors><![CDATA[Yee Siong Lee;  Pathirana, P.N.;  Steinfort, C.L.;  Caelli, T.]]></authors>
##     <affiliations><![CDATA[Fac. of Sci. & Technol., Deakin Univ., Waurn Ponds, VIC, Australia]]></affiliations>
##     <controlledterms>
##       <term><![CDATA[Doppler radar]]></term>
##       <term><![CDATA[electromagnetic interference]]></term>
##       <term><![CDATA[health care]]></term>
##       <term><![CDATA[medical disorders]]></term>
##       <term><![CDATA[medical signal detection]]></term>
##       <term><![CDATA[microwave measurement]]></term>
##       <term><![CDATA[patient diagnosis]]></term>
##       <term><![CDATA[patient monitoring]]></term>
##       <term><![CDATA[pneumodynamics]]></term>
##       <term><![CDATA[sleep]]></term>
##     </controlledterms>
##     <thesaurusterms>
##       <term><![CDATA[Belts]]></term>
##       <term><![CDATA[Continuous wavelet transforms]]></term>
##       <term><![CDATA[Doppler radar]]></term>
##       <term><![CDATA[Sleep apnea]]></term>
##       <term><![CDATA[Time-frequency analysis]]></term>
##     </thesaurusterms>
##     <pubtitle><![CDATA[Translational Engineering in Health and Medicine, IEEE Journal of]]></pubtitle>
##     <punumber><![CDATA[6221039]]></punumber>
##     <pubtype><![CDATA[Journals & Magazines]]></pubtype>
##     <publisher><![CDATA[IEEE]]></publisher>
##     <volume><![CDATA[2]]></volume>
##     <py><![CDATA[2014]]></py>
##     <spage><![CDATA[1]]></spage>
##     <epage><![CDATA[12]]></epage>
##     <abstract><![CDATA[Noncontact detection characteristic of Doppler radar provides an unobtrusive means of respiration detection and monitoring. This avoids additional preparations, such as physical sensor attachment or special clothing, which can be useful for certain healthcare applications. Furthermore, robustness of Doppler radar against environmental factors, such as light, ambient temperature, interference from other signals occupying the same bandwidth, fading effects, reduce environmental constraints and strengthens the possibility of employing Doppler radar in long-term respiration detection, and monitoring applications such as sleep studies. This paper presents an evaluation in the of use of microwave Doppler radar for capturing different dynamics of breathing patterns in addition to the respiration rate. Although finding the respiration rate is essential, identifying abnormal breathing patterns in real-time could be used to gain further insights into respiratory disorders and refine diagnostic procedures. Several known breathing disorders were professionally role played and captured in a real-time laboratory environment using a noncontact Doppler radar to evaluate the feasibility of this noncontact form of measurement in capturing breathing patterns under different conditions associated with certain breathing disorders. In addition to that, inhalation and exhalation flow patterns under different breathing scenarios were investigated to further support the feasibility of Doppler radar to accurately estimate the tidal volume. The results obtained for both experiments were compared with the gold standard measurement schemes, such as respiration belt and spirometry readings, yielding significant correlations with the Doppler radar-based information. In summary, Doppler radar is highlighted as an alternative approach not only for determining respiration rates, but also for identifying breathing patterns and tidal volumes as a preferred nonwearable alternative to the conventional - ontact sensing methods.]]></abstract>
##     <issn><![CDATA[2168-2372]]></issn>
##     <htmlFlag><![CDATA[1]]></htmlFlag>
##     <arnumber><![CDATA[6942141]]></arnumber>
##     <doi><![CDATA[10.1109/JTEHM.2014.2365776]]></doi>
##     <publicationId><![CDATA[6942141]]></publicationId>
##     <mdurl><![CDATA[http://ieeexplore.ieee.org/xpl/articleDetails.jsp?tp=&arnumber=6942141&contentType=Journals+%26+Magazines]]></mdurl>
##     <pdf><![CDATA[http://ieeexplore.ieee.org/stamp/stamp.jsp?arnumber=6942141]]></pdf>
##   </document>
##   <document>
##     <rank>212</rank>
##     <title><![CDATA[Tuning the Scattering Response of the Optical Nano Antennas Using Graphene]]></title>
##     <authors><![CDATA[Mehta, B.;  Zaghloul, M.E.]]></authors>
##     <affiliations><![CDATA[Dept. of Electr. & Comput. Eng., George Washington Univ., Washington, DC, USA]]></affiliations>
##     <controlledterms>
##       <term><![CDATA[dipole antennas]]></term>
##       <term><![CDATA[finite difference time-domain analysis]]></term>
##       <term><![CDATA[graphene]]></term>
##       <term><![CDATA[light scattering]]></term>
##       <term><![CDATA[metamaterial antennas]]></term>
##       <term><![CDATA[nanophotonics]]></term>
##       <term><![CDATA[optical tuning]]></term>
##     </controlledterms>
##     <thesaurusterms>
##       <term><![CDATA[Dipole antennas]]></term>
##       <term><![CDATA[Finite difference methods]]></term>
##       <term><![CDATA[Graphene]]></term>
##       <term><![CDATA[Mathematical model]]></term>
##       <term><![CDATA[Optical device fabrication]]></term>
##       <term><![CDATA[Optical imaging]]></term>
##       <term><![CDATA[Optical scattering]]></term>
##     </thesaurusterms>
##     <pubtitle><![CDATA[Photonics Journal, IEEE]]></pubtitle>
##     <punumber><![CDATA[4563994]]></punumber>
##     <pubtype><![CDATA[Journals & Magazines]]></pubtype>
##     <publisher><![CDATA[IEEE]]></publisher>
##     <volume><![CDATA[6]]></volume>
##     <issue><![CDATA[1]]></issue>
##     <py><![CDATA[2014]]></py>
##     <spage><![CDATA[1]]></spage>
##     <epage><![CDATA[8]]></epage>
##     <abstract><![CDATA[In this paper, we propose the tuning of the optical nano antenna in the visible spectrum using graphene. A dipole structure for the nano antenna is considered with the resonating wavelength at the VIS-NIR spectrum range. Resonance is determined using the scattering characteristics of the dipole antenna. Placing monolayer and bilayers of graphene sheets on top of the optical nano antenna will result in a change in the resonating wavelength. The input impedance of the resultant structures is calculated. The FDTD simulation results for resonance characteristics were verified with the experiment results. Such architecture of graphene on top of the dipole nano antennas can be used for different biological and chemical gas sensors.]]></abstract>
##     <issn><![CDATA[1943-0655]]></issn>
##     <htmlFlag><![CDATA[1]]></htmlFlag>
##     <arnumber><![CDATA[6716956]]></arnumber>
##     <doi><![CDATA[10.1109/JPHOT.2014.2300501]]></doi>
##     <publicationId><![CDATA[6716956]]></publicationId>
##     <mdurl><![CDATA[http://ieeexplore.ieee.org/xpl/articleDetails.jsp?tp=&arnumber=6716956&contentType=Journals+%26+Magazines]]></mdurl>
##     <pdf><![CDATA[http://ieeexplore.ieee.org/stamp/stamp.jsp?arnumber=6716956]]></pdf>
##   </document>
##   <document>
##     <rank>213</rank>
##     <title><![CDATA[Processor-Based Strong Physical Unclonable Functions With Aging-Based Response Tuning]]></title>
##     <authors><![CDATA[Joonho Kong;  Koushanfar, F.]]></authors>
##     <affiliations><![CDATA[Dept. of Electr. & Comput. Eng., Rice Univ., Houston, TX, USA]]></affiliations>
##     <controlledterms>
##       <term><![CDATA[computer architecture]]></term>
##       <term><![CDATA[cryptographic protocols]]></term>
##       <term><![CDATA[digital signatures]]></term>
##       <term><![CDATA[microprocessor chips]]></term>
##     </controlledterms>
##     <thesaurusterms>
##       <term><![CDATA[Aging]]></term>
##       <term><![CDATA[Circuit optimization]]></term>
##       <term><![CDATA[Delays]]></term>
##       <term><![CDATA[Logic gates]]></term>
##       <term><![CDATA[Microprocessors]]></term>
##       <term><![CDATA[Multicore processing]]></term>
##       <term><![CDATA[Network security]]></term>
##       <term><![CDATA[Silicon]]></term>
##       <term><![CDATA[Temperature measurement]]></term>
##     </thesaurusterms>
##     <pubtitle><![CDATA[Emerging Topics in Computing, IEEE Transactions on]]></pubtitle>
##     <punumber><![CDATA[6245516]]></punumber>
##     <pubtype><![CDATA[Journals & Magazines]]></pubtype>
##     <publisher><![CDATA[IEEE]]></publisher>
##     <volume><![CDATA[2]]></volume>
##     <issue><![CDATA[1]]></issue>
##     <py><![CDATA[2014]]></py>
##     <spage><![CDATA[16]]></spage>
##     <epage><![CDATA[29]]></epage>
##     <abstract><![CDATA[A strong physically unclonable function (PUF) is a circuit structure that extracts an exponential number of unique chip signatures from a bounded number of circuit components. The strong PUF unique signatures can enable a variety of low-overhead security and intellectual property protection protocols applicable to several computing platforms. This paper proposes a novel lightweight (low overhead) strong PUF based on the timings of a classic processor architecture. A small amount of circuitry is added to the processor for on-the-fly extraction of the unique timing signatures. To achieve desirable strong PUF properties, we develop an algorithm that leverages intentional post-silicon aging to tune the inter- and intra-chip signatures variation. Our evaluation results show that the new PUF meets the desirable inter- and intra-chip strong PUF characteristics, whereas its overhead is much lower than the existing strong PUFs. For the processors implemented in 45 nm technology, the average inter-chip Hamming distance for 32-bit responses is increased by 16.1% after applying our post-silicon tuning method; the aging algorithm also decreases the average intra-chip Hamming distance by 98.1% (for 32-bit responses).]]></abstract>
##     <issn><![CDATA[2168-6750]]></issn>
##     <htmlFlag><![CDATA[1]]></htmlFlag>
##     <arnumber><![CDATA[6656920]]></arnumber>
##     <doi><![CDATA[10.1109/TETC.2013.2289385]]></doi>
##     <publicationId><![CDATA[6656920]]></publicationId>
##     <mdurl><![CDATA[http://ieeexplore.ieee.org/xpl/articleDetails.jsp?tp=&arnumber=6656920&contentType=Journals+%26+Magazines]]></mdurl>
##     <pdf><![CDATA[http://ieeexplore.ieee.org/stamp/stamp.jsp?arnumber=6656920]]></pdf>
##   </document>
##   <document>
##     <rank>214</rank>
##     <title><![CDATA[Power-Constrained Contrast Enhancement Algorithm Using Multiscale Retinex for OLED Display]]></title>
##     <authors><![CDATA[Yeon-Oh Nam;  Dong-Yoon Choi;  Byung Cheol Song]]></authors>
##     <affiliations><![CDATA[Dept. of Electron. Eng., Inha Univ., Incheon, South Korea]]></affiliations>
##     <controlledterms>
##       <term><![CDATA[LED displays]]></term>
##       <term><![CDATA[gain control]]></term>
##       <term><![CDATA[image enhancement]]></term>
##       <term><![CDATA[image sequences]]></term>
##       <term><![CDATA[optimisation]]></term>
##       <term><![CDATA[organic light emitting diodes]]></term>
##       <term><![CDATA[power control]]></term>
##       <term><![CDATA[real-time systems]]></term>
##     </controlledterms>
##     <thesaurusterms>
##       <term><![CDATA[Dynamic range]]></term>
##       <term><![CDATA[Electronic mail]]></term>
##       <term><![CDATA[Materials]]></term>
##       <term><![CDATA[Organic light emitting diodes]]></term>
##       <term><![CDATA[Power control]]></term>
##       <term><![CDATA[Power demand]]></term>
##       <term><![CDATA[Video sequences]]></term>
##     </thesaurusterms>
##     <pubtitle><![CDATA[Image Processing, IEEE Transactions on]]></pubtitle>
##     <punumber><![CDATA[83]]></punumber>
##     <pubtype><![CDATA[Journals & Magazines]]></pubtype>
##     <publisher><![CDATA[IEEE]]></publisher>
##     <volume><![CDATA[23]]></volume>
##     <issue><![CDATA[8]]></issue>
##     <py><![CDATA[2014]]></py>
##     <spage><![CDATA[3308]]></spage>
##     <epage><![CDATA[3320]]></epage>
##     <abstract><![CDATA[This paper presents a power-constrained contrast enhancement algorithm for organic light-emitting diode display based on multiscale retinex (MSR). In general, MSR, which is the key component of the proposed algorithm, consists of power controllable log operation and subbandwise gain control. First, we decompose an input image to MSRs of different sub-bands, and compute a proper gain for each MSR. Second, we apply a coarse-to-fine power control mechanism, which recomputes the MSRs and gains. This step iterates until the target power saving is accurately accomplished. With video sequences, the contrast levels of adjacent images are determined consistently using temporal coherence in order to avoid flickering artifacts. Finally, we present several optimization skills for real-time processing. Experimental results show that the proposed algorithm provides better visual quality than previous methods, and a consistent power-saving ratio without flickering artifacts, even for video sequences.]]></abstract>
##     <issn><![CDATA[1057-7149]]></issn>
##     <htmlFlag><![CDATA[1]]></htmlFlag>
##     <arnumber><![CDATA[6815989]]></arnumber>
##     <doi><![CDATA[10.1109/TIP.2014.2324288]]></doi>
##     <publicationId><![CDATA[6815989]]></publicationId>
##     <mdurl><![CDATA[http://ieeexplore.ieee.org/xpl/articleDetails.jsp?tp=&arnumber=6815989&contentType=Journals+%26+Magazines]]></mdurl>
##     <pdf><![CDATA[http://ieeexplore.ieee.org/stamp/stamp.jsp?arnumber=6815989]]></pdf>
##   </document>
##   <document>
##     <rank>215</rank>
##     <title><![CDATA[Confronting the Variability Issues Affecting the Performance of Next-Generation SRAM Design to Optimize and Predict the Speed and Yield]]></title>
##     <authors><![CDATA[Samandari-Rad, J.;  Guthaus, M.;  Hughey, R.]]></authors>
##     <affiliations><![CDATA[Dept. of Electr. Eng., Univ. of California, Santa Cruz, Santa Cruz, CA, USA]]></affiliations>
##     <controlledterms>
##       <term><![CDATA[SRAM chips]]></term>
##       <term><![CDATA[ageing]]></term>
##       <term><![CDATA[integrated circuit modelling]]></term>
##       <term><![CDATA[integrated circuit reliability]]></term>
##       <term><![CDATA[integrated circuit yield]]></term>
##       <term><![CDATA[negative bias temperature instability]]></term>
##     </controlledterms>
##     <thesaurusterms>
##       <term><![CDATA[Energy efficiency]]></term>
##       <term><![CDATA[Integrated circuit reliability]]></term>
##       <term><![CDATA[Next generation networking]]></term>
##       <term><![CDATA[Performance evaluation]]></term>
##       <term><![CDATA[Reliability engineering]]></term>
##       <term><![CDATA[SRAM chips]]></term>
##       <term><![CDATA[Threshold voltage]]></term>
##       <term><![CDATA[Voltage measurement]]></term>
##     </thesaurusterms>
##     <pubtitle><![CDATA[Access, IEEE]]></pubtitle>
##     <punumber><![CDATA[6287639]]></punumber>
##     <pubtype><![CDATA[Journals & Magazines]]></pubtype>
##     <publisher><![CDATA[IEEE]]></publisher>
##     <volume><![CDATA[2]]></volume>
##     <py><![CDATA[2014]]></py>
##     <spage><![CDATA[577]]></spage>
##     <epage><![CDATA[601]]></epage>
##     <abstract><![CDATA[Effectively confronting device and circuit parameter variations to maintain or improve the design of high performance and energy efficient systems while satisfying historical standards for reliability and lower costs is increasingly challenging with the scaling of technology. In this paper, we develop methods for robust and resilient six-transistor-cell static random access memory (6T-SRAM) designs that mitigate the effects of device and circuit parameter variations. Our interdisciplinary effort involves: 1) using our own recently developed VAR-TX model [1] to illustrate the impact of interdie (also known as die-to-die, D2D) and intradie (also know as within-die, WID) process and operation variations - namely threshold voltage (Vth), gate length (L), and supply voltage (Vdd) - on future different 16-nm architectures and 2) using modified versions of other well-received models to illustrate the impact of variability due to temperature, negative bias temperature instability, aging, and so forth, on existing and next-generation technology nodes. Our goal in combining modeling techniques is to help minimize all major types of variability and to consequently predict and optimize speed and yield for the next generation 6T-SRAMs.]]></abstract>
##     <issn><![CDATA[2169-3536]]></issn>
##     <htmlFlag><![CDATA[1]]></htmlFlag>
##     <arnumber><![CDATA[6815646]]></arnumber>
##     <doi><![CDATA[10.1109/ACCESS.2014.2323233]]></doi>
##     <publicationId><![CDATA[6815646]]></publicationId>
##     <mdurl><![CDATA[http://ieeexplore.ieee.org/xpl/articleDetails.jsp?tp=&arnumber=6815646&contentType=Journals+%26+Magazines]]></mdurl>
##     <pdf><![CDATA[http://ieeexplore.ieee.org/stamp/stamp.jsp?arnumber=6815646]]></pdf>
##   </document>
##   <document>
##     <rank>216</rank>
##     <title><![CDATA[Near-Optimal Sensor Placement for Linear Inverse Problems]]></title>
##     <authors><![CDATA[Ranieri, J.;  Chebira, A.;  Vetterli, M.]]></authors>
##     <affiliations><![CDATA[Sch. of Comput. & Commun. Sci., Ecole Polytech. Fed. de Lausanne (EPFL), Lausanne, Switzerland]]></affiliations>
##     <controlledterms>
##       <term><![CDATA[greedy algorithms]]></term>
##       <term><![CDATA[inverse problems]]></term>
##       <term><![CDATA[mean square error methods]]></term>
##       <term><![CDATA[parameter estimation]]></term>
##       <term><![CDATA[sensor placement]]></term>
##       <term><![CDATA[wireless sensor networks]]></term>
##     </controlledterms>
##     <thesaurusterms>
##       <term><![CDATA[Approximation algorithms]]></term>
##       <term><![CDATA[Approximation methods]]></term>
##       <term><![CDATA[Cost function]]></term>
##       <term><![CDATA[Force]]></term>
##       <term><![CDATA[Inverse problems]]></term>
##       <term><![CDATA[Licenses]]></term>
##       <term><![CDATA[Signal processing algorithms]]></term>
##     </thesaurusterms>
##     <pubtitle><![CDATA[Signal Processing, IEEE Transactions on]]></pubtitle>
##     <punumber><![CDATA[78]]></punumber>
##     <pubtype><![CDATA[Journals & Magazines]]></pubtype>
##     <publisher><![CDATA[IEEE]]></publisher>
##     <volume><![CDATA[62]]></volume>
##     <issue><![CDATA[5]]></issue>
##     <py><![CDATA[2014]]></py>
##     <spage><![CDATA[1135]]></spage>
##     <epage><![CDATA[1146]]></epage>
##     <abstract><![CDATA[A classic problem is the estimation of a set of parameters from measurements collected by only a few sensors. The number of sensors is often limited by physical or economical constraints and their placement is of fundamental importance to obtain accurate estimates. Unfortunately, the selection of the optimal sensor locations is intrinsically combinatorial and the available approximation algorithms are not guaranteed to generate good solutions in all cases of interest. We propose FrameSense, a greedy algorithm for the selection of optimal sensor locations. The core cost function of the algorithm is the frame potential, a scalar property of matrices that measures the orthogonality of its rows. Notably, FrameSense is the first algorithm that is near-optimal in terms of mean square error, meaning that its solution is always guaranteed to be close to the optimal one. Moreover, we show with an extensive set of numerical experiments that FrameSense achieves state-of-the-art performance while having the lowest computational cost, when compared to other greedy methods.]]></abstract>
##     <issn><![CDATA[1053-587X]]></issn>
##     <htmlFlag><![CDATA[1]]></htmlFlag>
##     <arnumber><![CDATA[6709823]]></arnumber>
##     <doi><![CDATA[10.1109/TSP.2014.2299518]]></doi>
##     <publicationId><![CDATA[6709823]]></publicationId>
##     <mdurl><![CDATA[http://ieeexplore.ieee.org/xpl/articleDetails.jsp?tp=&arnumber=6709823&contentType=Journals+%26+Magazines]]></mdurl>
##     <pdf><![CDATA[http://ieeexplore.ieee.org/stamp/stamp.jsp?arnumber=6709823]]></pdf>
##   </document>
##   <document>
##     <rank>217</rank>
##     <title><![CDATA[A Primer on Hardware Security: Models, Methods, and Metrics]]></title>
##     <authors><![CDATA[Rostami, M.;  Koushanfar, F.;  Karri, R.]]></authors>
##     <affiliations><![CDATA[Dept. of Electr. & Comput. Eng., Rice Univ., Houston, TX, USA]]></affiliations>
##     <controlledterms>
##       <term><![CDATA[pattern classification]]></term>
##       <term><![CDATA[security of data]]></term>
##     </controlledterms>
##     <thesaurusterms>
##       <term><![CDATA[Computer security]]></term>
##       <term><![CDATA[Hardware]]></term>
##       <term><![CDATA[Integrated circuit modeling]]></term>
##       <term><![CDATA[Security]]></term>
##       <term><![CDATA[Supply chain management]]></term>
##       <term><![CDATA[Trojan horses]]></term>
##       <term><![CDATA[Watermarking]]></term>
##     </thesaurusterms>
##     <pubtitle><![CDATA[Proceedings of the IEEE]]></pubtitle>
##     <punumber><![CDATA[5]]></punumber>
##     <pubtype><![CDATA[Journals & Magazines]]></pubtype>
##     <publisher><![CDATA[IEEE]]></publisher>
##     <volume><![CDATA[102]]></volume>
##     <issue><![CDATA[8]]></issue>
##     <py><![CDATA[2014]]></py>
##     <spage><![CDATA[1283]]></spage>
##     <epage><![CDATA[1295]]></epage>
##     <abstract><![CDATA[The multinational, distributed, and multistep nature of integrated circuit (IC) production supply chain has introduced hardware-based vulnerabilities. Existing literature in hardware security assumes ad hoc threat models, defenses, and metrics for evaluation, making it difficult to analyze and compare alternate solutions. This paper systematizes the current knowledge in this emerging field, including a classification of threat models, state-of-the-art defenses, and evaluation metrics for important hardware-based attacks.]]></abstract>
##     <issn><![CDATA[0018-9219]]></issn>
##     <htmlFlag><![CDATA[1]]></htmlFlag>
##     <arnumber><![CDATA[6860363]]></arnumber>
##     <doi><![CDATA[10.1109/JPROC.2014.2335155]]></doi>
##     <publicationId><![CDATA[6860363]]></publicationId>
##     <mdurl><![CDATA[http://ieeexplore.ieee.org/xpl/articleDetails.jsp?tp=&arnumber=6860363&contentType=Journals+%26+Magazines]]></mdurl>
##     <pdf><![CDATA[http://ieeexplore.ieee.org/stamp/stamp.jsp?arnumber=6860363]]></pdf>
##   </document>
##   <document>
##     <rank>218</rank>
##     <title><![CDATA[A Compact and Low-Power Fractionally Injection-Locked Quadrature Frequency Synthesizer Using a Self-Synchronized Gating Injection Technique for Software-Defined Radios]]></title>
##     <authors><![CDATA[Wei Deng;  Hara, S.;  Musa, A.;  Okada, K.;  Matsuzawa, A.]]></authors>
##     <affiliations><![CDATA[Dept. of Phys. Electron., Tokyo Inst. of Technol., Tokyo, Japan]]></affiliations>
##     <controlledterms>
##       <term><![CDATA[calibration]]></term>
##       <term><![CDATA[frequency dividers]]></term>
##       <term><![CDATA[frequency synthesizers]]></term>
##       <term><![CDATA[phase locked loops]]></term>
##       <term><![CDATA[phase noise]]></term>
##       <term><![CDATA[software radio]]></term>
##     </controlledterms>
##     <thesaurusterms>
##       <term><![CDATA[Frequency conversion]]></term>
##       <term><![CDATA[Frequency synthesizers]]></term>
##       <term><![CDATA[Phase locked loops]]></term>
##       <term><![CDATA[Power demand]]></term>
##       <term><![CDATA[Resonant frequency]]></term>
##       <term><![CDATA[Synthesizers]]></term>
##       <term><![CDATA[Voltage-controlled oscillators]]></term>
##     </thesaurusterms>
##     <pubtitle><![CDATA[Solid-State Circuits, IEEE Journal of]]></pubtitle>
##     <punumber><![CDATA[4]]></punumber>
##     <pubtype><![CDATA[Journals & Magazines]]></pubtype>
##     <publisher><![CDATA[IEEE]]></publisher>
##     <volume><![CDATA[49]]></volume>
##     <issue><![CDATA[9]]></issue>
##     <py><![CDATA[2014]]></py>
##     <spage><![CDATA[1984]]></spage>
##     <epage><![CDATA[1994]]></epage>
##     <abstract><![CDATA[This paper describes a compact and low-power frequency synthesizer with quadrature phase output for software-defined radios (SDRs). The proposed synthesizer is constructed using a core phase-locked loop (PLL), which is coupled with a fractional-N injection-locked frequency divider (ILFD). The fractional-N injection-locking operation is achieved by the proposed self-synchronized gating injection technique. The principle of a fractional-N injection locking operation and the concept of the proposed circuits are described in detail. Analysis for predicting the locking range of the proposed fractional-N ILFD is investigated. A digital calibration scheme is adopted in order to compensate for process, voltage, and temperature (PVT) variations. Implemented in a 65 nm CMOS process, this work demonstrates continuous frequency coverage from 10 MHz to 6.6 GHz with quadrature phase output while occupying a small area of 0.38 mm<sup>2</sup> and consuming 16 to 26 mW, depending on the output frequency. The normalized phase noise achieves -135.3 dBc/Hz at an offset of 3 MHz and -95.1 dBc/Hz at an offset of 10 kHz, both from a carrier frequency of 1.7 GHz.]]></abstract>
##     <issn><![CDATA[0018-9200]]></issn>
##     <htmlFlag><![CDATA[1]]></htmlFlag>
##     <arnumber><![CDATA[6862065]]></arnumber>
##     <doi><![CDATA[10.1109/JSSC.2014.2334392]]></doi>
##     <publicationId><![CDATA[6862065]]></publicationId>
##     <mdurl><![CDATA[http://ieeexplore.ieee.org/xpl/articleDetails.jsp?tp=&arnumber=6862065&contentType=Journals+%26+Magazines]]></mdurl>
##     <pdf><![CDATA[http://ieeexplore.ieee.org/stamp/stamp.jsp?arnumber=6862065]]></pdf>
##   </document>
##   <document>
##     <rank>219</rank>
##     <title><![CDATA[Self-Mixing Fiber Ring Laser Velocimeter With Orthogonal-Beam Incident System]]></title>
##     <authors><![CDATA[Yunhe Zhao;  Shuang Wu;  Rong Xiang;  Zhigang Cao;  Yu Liu;  Huaqiao Gul;  Jianguo Liu;  Liang Lu;  Benli Yu]]></authors>
##     <affiliations><![CDATA[Key Lab. of Opto-Electron. Inf. Acquisition, Anhui Univ., Hefei, China]]></affiliations>
##     <controlledterms>
##       <term><![CDATA[fibre lasers]]></term>
##       <term><![CDATA[laser velocimeters]]></term>
##       <term><![CDATA[multiwave mixing]]></term>
##       <term><![CDATA[velocity measurement]]></term>
##     </controlledterms>
##     <thesaurusterms>
##       <term><![CDATA[Erbium-doped fiber lasers]]></term>
##       <term><![CDATA[Measurement by laser beam]]></term>
##       <term><![CDATA[Optical fiber communication]]></term>
##       <term><![CDATA[Optical fiber couplers]]></term>
##       <term><![CDATA[Velocity measurement]]></term>
##     </thesaurusterms>
##     <pubtitle><![CDATA[Photonics Journal, IEEE]]></pubtitle>
##     <punumber><![CDATA[4563994]]></punumber>
##     <pubtype><![CDATA[Journals & Magazines]]></pubtype>
##     <publisher><![CDATA[IEEE]]></publisher>
##     <volume><![CDATA[6]]></volume>
##     <issue><![CDATA[2]]></issue>
##     <py><![CDATA[2014]]></py>
##     <spage><![CDATA[1]]></spage>
##     <epage><![CDATA[11]]></epage>
##     <abstract><![CDATA[In this paper, we present a method of velocity measurement using of a novel self-mixing velocimeter with orthogonal-beam incident system (OSMV), which enables the velocity measurement without knowing the incident angle information. The fiber ring laser used could provide a stable and narrow-width laser light source, which could enhance the stability and the signal-to-noise ratio of the novel self-mixing fiber ring laser velocimeter. The result indicates a much better linearity of our novel velocimeter than the traditional one in the turntable experiment due to the use of the orthogonal-beam incident system and fiber ring laser source. The relative error rates of our novel velocimeter system are up to 1.258% in contrast with the 13.720% result of the traditional one.]]></abstract>
##     <issn><![CDATA[1943-0655]]></issn>
##     <htmlFlag><![CDATA[1]]></htmlFlag>
##     <arnumber><![CDATA[6776550]]></arnumber>
##     <doi><![CDATA[10.1109/JPHOT.2014.2311452]]></doi>
##     <publicationId><![CDATA[6776550]]></publicationId>
##     <mdurl><![CDATA[http://ieeexplore.ieee.org/xpl/articleDetails.jsp?tp=&arnumber=6776550&contentType=Journals+%26+Magazines]]></mdurl>
##     <pdf><![CDATA[http://ieeexplore.ieee.org/stamp/stamp.jsp?arnumber=6776550]]></pdf>
##   </document>
##   <document>
##     <rank>220</rank>
##     <title><![CDATA[On-Chip Multi 4-Port Optical Circulators]]></title>
##     <authors><![CDATA[El-Ganainy, R.;  Levy, M.]]></authors>
##     <affiliations><![CDATA[Dept. of Phys., Michigan Technol. Univ., Houghton, MI, USA]]></affiliations>
##     <controlledterms>
##       <term><![CDATA[magneto-optical effects]]></term>
##       <term><![CDATA[optical circulators]]></term>
##       <term><![CDATA[optical waveguides]]></term>
##     </controlledterms>
##     <thesaurusterms>
##       <term><![CDATA[Biomedical optical imaging]]></term>
##       <term><![CDATA[Circulators]]></term>
##       <term><![CDATA[Nonlinear optics]]></term>
##       <term><![CDATA[Optical films]]></term>
##       <term><![CDATA[Optical resonators]]></term>
##       <term><![CDATA[Optical waveguides]]></term>
##     </thesaurusterms>
##     <pubtitle><![CDATA[Photonics Journal, IEEE]]></pubtitle>
##     <punumber><![CDATA[4563994]]></punumber>
##     <pubtype><![CDATA[Journals & Magazines]]></pubtype>
##     <publisher><![CDATA[IEEE]]></publisher>
##     <volume><![CDATA[6]]></volume>
##     <issue><![CDATA[1]]></issue>
##     <py><![CDATA[2014]]></py>
##     <spage><![CDATA[1]]></spage>
##     <epage><![CDATA[8]]></epage>
##     <abstract><![CDATA[We present a new geometry for on-chip optical circulators based on waveguide arrays. The optical array is engineered to mimic the Fock space representation of a noninteracting two-site Bose-Hubbard Hamiltonian. By introducing a carefully tailored magneto-optic nonreciprocity to these structures, the array operates in the perfect transfer and surface Bloch oscillation modes in the forward and backward propagation directions, respectively. We show that an array made of (2N + 1) waveguide channels can function as N 4-port optical circulators with very large isolation ratios and low forward losses. Numerical analysis using beam propagation method indicates a large bandwidth of operation.]]></abstract>
##     <issn><![CDATA[1943-0655]]></issn>
##     <htmlFlag><![CDATA[1]]></htmlFlag>
##     <arnumber><![CDATA[6682990]]></arnumber>
##     <doi><![CDATA[10.1109/JPHOT.2013.2294693]]></doi>
##     <publicationId><![CDATA[6682990]]></publicationId>
##     <mdurl><![CDATA[http://ieeexplore.ieee.org/xpl/articleDetails.jsp?tp=&arnumber=6682990&contentType=Journals+%26+Magazines]]></mdurl>
##     <pdf><![CDATA[http://ieeexplore.ieee.org/stamp/stamp.jsp?arnumber=6682990]]></pdf>
##   </document>
##   <document>
##     <rank>221</rank>
##     <title><![CDATA[Light Trapping in Thin Crystalline Si Solar Cells Using Surface Mie Scatterers]]></title>
##     <authors><![CDATA[Spinelli, P.;  Polman, A.]]></authors>
##     <affiliations><![CDATA[Center for Nanophotonics, FOM Inst., Amsterdam, Netherlands]]></affiliations>
##     <controlledterms>
##       <term><![CDATA[Mie scattering]]></term>
##       <term><![CDATA[dielectric materials]]></term>
##       <term><![CDATA[elemental semiconductors]]></term>
##       <term><![CDATA[light scattering]]></term>
##       <term><![CDATA[nanoparticles]]></term>
##       <term><![CDATA[nanophotonics]]></term>
##       <term><![CDATA[radiation pressure]]></term>
##       <term><![CDATA[semiconductor thin films]]></term>
##       <term><![CDATA[silicon]]></term>
##       <term><![CDATA[solar cells]]></term>
##       <term><![CDATA[surface recombination]]></term>
##       <term><![CDATA[thin film devices]]></term>
##     </controlledterms>
##     <thesaurusterms>
##       <term><![CDATA[Absorption]]></term>
##       <term><![CDATA[Coatings]]></term>
##       <term><![CDATA[Photovoltaic cells]]></term>
##       <term><![CDATA[Silicon]]></term>
##       <term><![CDATA[Slabs]]></term>
##       <term><![CDATA[Standards]]></term>
##       <term><![CDATA[Surface treatment]]></term>
##     </thesaurusterms>
##     <pubtitle><![CDATA[Photovoltaics, IEEE Journal of]]></pubtitle>
##     <punumber><![CDATA[5503869]]></punumber>
##     <pubtype><![CDATA[Journals & Magazines]]></pubtype>
##     <publisher><![CDATA[IEEE]]></publisher>
##     <volume><![CDATA[4]]></volume>
##     <issue><![CDATA[2]]></issue>
##     <py><![CDATA[2014]]></py>
##     <spage><![CDATA[554]]></spage>
##     <epage><![CDATA[559]]></epage>
##     <abstract><![CDATA[Dielectric nanoparticles placed on top of a thin-film solar cell strongly enhance light absorption in the cell over a broad spectral range due to the preferential forward scattering of light from leaky Mie resonances in the particle. In this study, we systematically study with numerical simulations the absorption of light into thin (1-100 &#x03BC;m) crystalline Si solar cells patterned with Si nanocylinder arrays on top of the cell. We then use an analytical model to calculate the solar cell efficiency, based on the simulated absorption spectra. Using realistic values for bulk and surface recombination rates, we find that a 20-&#x03BC;m-thick Si solar cell with 21.5% efficiency can be made by using the Si nanocylinder Mie coating.]]></abstract>
##     <issn><![CDATA[2156-3381]]></issn>
##     <htmlFlag><![CDATA[1]]></htmlFlag>
##     <arnumber><![CDATA[6710163]]></arnumber>
##     <doi><![CDATA[10.1109/JPHOTOV.2013.2292744]]></doi>
##     <publicationId><![CDATA[6710163]]></publicationId>
##     <mdurl><![CDATA[http://ieeexplore.ieee.org/xpl/articleDetails.jsp?tp=&arnumber=6710163&contentType=Journals+%26+Magazines]]></mdurl>
##     <pdf><![CDATA[http://ieeexplore.ieee.org/stamp/stamp.jsp?arnumber=6710163]]></pdf>
##   </document>
##   <document>
##     <rank>222</rank>
##     <title><![CDATA[Optical Vector Network Analyzer Based on Single-Sideband Modulation and Segmental Measurement]]></title>
##     <authors><![CDATA[Wei Li;  Wen Ting Wang;  Li Xian Wang;  Ning Hua Zhu]]></authors>
##     <affiliations><![CDATA[State Key Lab. on Integrated Optoelectron., Inst. of Semicond., Beijing, China]]></affiliations>
##     <controlledterms>
##       <term><![CDATA[band-pass filters]]></term>
##       <term><![CDATA[network analysers]]></term>
##       <term><![CDATA[optical filters]]></term>
##       <term><![CDATA[optical modulation]]></term>
##     </controlledterms>
##     <thesaurusterms>
##       <term><![CDATA[Amplitude modulation]]></term>
##       <term><![CDATA[Frequency modulation]]></term>
##       <term><![CDATA[Optical fibers]]></term>
##       <term><![CDATA[Optical filters]]></term>
##       <term><![CDATA[Optical modulation]]></term>
##       <term><![CDATA[Optical variables measurement]]></term>
##     </thesaurusterms>
##     <pubtitle><![CDATA[Photonics Journal, IEEE]]></pubtitle>
##     <punumber><![CDATA[4563994]]></punumber>
##     <pubtype><![CDATA[Journals & Magazines]]></pubtype>
##     <publisher><![CDATA[IEEE]]></publisher>
##     <volume><![CDATA[6]]></volume>
##     <issue><![CDATA[2]]></issue>
##     <py><![CDATA[2014]]></py>
##     <spage><![CDATA[1]]></spage>
##     <epage><![CDATA[8]]></epage>
##     <abstract><![CDATA[We propose an optical vector network analyzer (OVNA) based on single-sideband (SSB) modulation and segmental measurement. The motivation of this work lies in the fact that it is hard for the previous SSB-based OVNA to measure a device-under-test (DUT) with bandpass response. In our scheme, the transmission response of the DUT is divided into several segments, which are measured one by one using the SSB modulation method. The frequency interval between adjacent segments is precisely controlled by a microwave source. The full transmission response of the DUT is obtained by combining the measured results of different segments. The adjacent segments are connected using the overlapped transmission responses. A proof-of-concept experiment was carried out to measure the transmission response of an optical bandpass filter. It was successfully obtained over a frequency range of 80.05 GHz with a resolution of 25 MHz.]]></abstract>
##     <issn><![CDATA[1943-0655]]></issn>
##     <htmlFlag><![CDATA[1]]></htmlFlag>
##     <arnumber><![CDATA[6766237]]></arnumber>
##     <doi><![CDATA[10.1109/JPHOT.2014.2311443]]></doi>
##     <publicationId><![CDATA[6766237]]></publicationId>
##     <mdurl><![CDATA[http://ieeexplore.ieee.org/xpl/articleDetails.jsp?tp=&arnumber=6766237&contentType=Journals+%26+Magazines]]></mdurl>
##     <pdf><![CDATA[http://ieeexplore.ieee.org/stamp/stamp.jsp?arnumber=6766237]]></pdf>
##   </document>
##   <document>
##     <rank>223</rank>
##     <title><![CDATA[Extending Worst Case Response-Time Analysis for Mixed Messages in Controller Area Network With Priority and FIFO Queues]]></title>
##     <authors><![CDATA[Mubeen, S.;  Maki-Turja, J.;  Sjodin, M.]]></authors>
##     <affiliations><![CDATA[Malardalen Univ., Va&#x0308;ster&#x00E5;s, Sweden]]></affiliations>
##     <controlledterms>
##       <term><![CDATA[controller area networks]]></term>
##       <term><![CDATA[protocols]]></term>
##       <term><![CDATA[queueing theory]]></term>
##       <term><![CDATA[real-time systems]]></term>
##       <term><![CDATA[scheduling]]></term>
##     </controlledterms>
##     <thesaurusterms>
##       <term><![CDATA[Automotive engineering]]></term>
##       <term><![CDATA[Controller area networks]]></term>
##       <term><![CDATA[Protocols]]></term>
##       <term><![CDATA[Queueing analysis]]></term>
##       <term><![CDATA[Timing]]></term>
##     </thesaurusterms>
##     <pubtitle><![CDATA[Access, IEEE]]></pubtitle>
##     <punumber><![CDATA[6287639]]></punumber>
##     <pubtype><![CDATA[Journals & Magazines]]></pubtype>
##     <publisher><![CDATA[IEEE]]></publisher>
##     <volume><![CDATA[2]]></volume>
##     <py><![CDATA[2014]]></py>
##     <spage><![CDATA[365]]></spage>
##     <epage><![CDATA[380]]></epage>
##     <abstract><![CDATA[The existing worst case response-time analysis for controller area network (CAN) with nodes implementing priority and First In First Out (FIFO) queues does not support mixed messages. It assumes that a message is queued for transmission either periodically or sporadically. However, a message can also be queued both periodically and sporadically using mixed transmission mode implemented by several higher level protocols for CAN that are used in the automotive industry. We extend the existing analysis for CAN to support any higher level protocol for CAN that uses periodic, sporadic, and mixed transmission of messages in the systems where some nodes implement priority queues, whereas others implement FIFO queues. In order to provide a proof of concept, we implement the extended analysis in a free tool, conduct an automotive-application case study, and perform comparative evaluation of the extended analysis with the existing analysis.]]></abstract>
##     <issn><![CDATA[2169-3536]]></issn>
##     <htmlFlag><![CDATA[1]]></htmlFlag>
##     <arnumber><![CDATA[6803860]]></arnumber>
##     <doi><![CDATA[10.1109/ACCESS.2014.2319255]]></doi>
##     <publicationId><![CDATA[6803860]]></publicationId>
##     <mdurl><![CDATA[http://ieeexplore.ieee.org/xpl/articleDetails.jsp?tp=&arnumber=6803860&contentType=Journals+%26+Magazines]]></mdurl>
##     <pdf><![CDATA[http://ieeexplore.ieee.org/stamp/stamp.jsp?arnumber=6803860]]></pdf>
##   </document>
##   <document>
##     <rank>224</rank>
##     <title><![CDATA[Development of Low-Voltage Load Models for the Residential Load Sector]]></title>
##     <authors><![CDATA[Collin, A.J.;  Tsagarakis, G.;  Kiprakis, A.E.;  McLaughlin, S.]]></authors>
##     <affiliations><![CDATA[Sch. of Eng., Univ. of Edinburgh, Edinburgh, UK]]></affiliations>
##     <controlledterms>
##       <term><![CDATA[Markov processes]]></term>
##       <term><![CDATA[Monte Carlo methods]]></term>
##       <term><![CDATA[demand side management]]></term>
##       <term><![CDATA[smart power grids]]></term>
##     </controlledterms>
##     <thesaurusterms>
##       <term><![CDATA[Aggregates]]></term>
##       <term><![CDATA[Data models]]></term>
##       <term><![CDATA[Electrical products]]></term>
##       <term><![CDATA[Home appliances]]></term>
##       <term><![CDATA[Load modeling]]></term>
##       <term><![CDATA[Power demand]]></term>
##       <term><![CDATA[Reactive power]]></term>
##     </thesaurusterms>
##     <pubtitle><![CDATA[Power Systems, IEEE Transactions on]]></pubtitle>
##     <punumber><![CDATA[59]]></punumber>
##     <pubtype><![CDATA[Journals & Magazines]]></pubtype>
##     <publisher><![CDATA[IEEE]]></publisher>
##     <volume><![CDATA[29]]></volume>
##     <issue><![CDATA[5]]></issue>
##     <py><![CDATA[2014]]></py>
##     <spage><![CDATA[2180]]></spage>
##     <epage><![CDATA[2188]]></epage>
##     <abstract><![CDATA[A bottom-up modeling approach is presented that uses a Markov chain Monte Carlo (MCMC) method to develop demand profiles. The demand profiles are combined with the electrical characteristics of the appliance to create detailed time-varying models of residential loads suitable for the analysis of smart grid applications and low-voltage (LV) demand-side management. The results obtained demonstrate significant temporal variations in the electrical characteristics of LV customers that are not captured by existing load profile or load model development approaches. The software developed within this work is made freely available for use by the community.]]></abstract>
##     <issn><![CDATA[0885-8950]]></issn>
##     <htmlFlag><![CDATA[1]]></htmlFlag>
##     <arnumber><![CDATA[6733330]]></arnumber>
##     <doi><![CDATA[10.1109/TPWRS.2014.2301949]]></doi>
##     <publicationId><![CDATA[6733330]]></publicationId>
##     <mdurl><![CDATA[http://ieeexplore.ieee.org/xpl/articleDetails.jsp?tp=&arnumber=6733330&contentType=Journals+%26+Magazines]]></mdurl>
##     <pdf><![CDATA[http://ieeexplore.ieee.org/stamp/stamp.jsp?arnumber=6733330]]></pdf>
##   </document>
##   <document>
##     <rank>225</rank>
##     <title><![CDATA[Synchrophasor Measurement Technology in Power Systems: Panorama and State-of-the-Art]]></title>
##     <authors><![CDATA[Aminifar, F.;  Fotuhi-Firuzabad, M.;  Safdarian, A.;  Davoudi, A.;  Shahidehpour, M.]]></authors>
##     <affiliations><![CDATA[Sch. of Electr. & Comput. Eng., Univ. of Tehran, Tehran, Iran]]></affiliations>
##     <controlledterms>
##       <term><![CDATA[phasor measurement]]></term>
##       <term><![CDATA[power engineering computing]]></term>
##     </controlledterms>
##     <thesaurusterms>
##       <term><![CDATA[Bibliographies]]></term>
##       <term><![CDATA[Measurement]]></term>
##       <term><![CDATA[Phasor measurement]]></term>
##       <term><![CDATA[Synchronization]]></term>
##       <term><![CDATA[Wide area measurement]]></term>
##     </thesaurusterms>
##     <pubtitle><![CDATA[Access, IEEE]]></pubtitle>
##     <punumber><![CDATA[6287639]]></punumber>
##     <pubtype><![CDATA[Journals & Magazines]]></pubtype>
##     <publisher><![CDATA[IEEE]]></publisher>
##     <volume><![CDATA[2]]></volume>
##     <py><![CDATA[2014]]></py>
##     <spage><![CDATA[1607]]></spage>
##     <epage><![CDATA[1628]]></epage>
##     <abstract><![CDATA[Phasor measurement units (PMUs) are rapidly being deployed in electric power networks across the globe. Wide-area measurement system (WAMS), which builds upon PMUs and fast communication links, is consequently emerging as an advanced monitoring and control infrastructure. Rapid adaptation of such devices and technologies has led the researchers to investigate multitude of challenges and pursue opportunities in synchrophasor measurement technology, PMU structural design, PMU placement, miscellaneous applications of PMU from local perspectives, and various WAMS functionalities from the system perspective. Relevant research articles appeared in the IEEE and IET publications from 1983 through 2014 are rigorously surveyed in this paper to represent a panorama of research progress lines. This bibliography will aid academic researchers and practicing engineers in adopting appropriate topics and will stimulate utilities toward development and implementation of software packages.]]></abstract>
##     <issn><![CDATA[2169-3536]]></issn>
##     <htmlFlag><![CDATA[1]]></htmlFlag>
##     <arnumber><![CDATA[7005374]]></arnumber>
##     <doi><![CDATA[10.1109/ACCESS.2015.2389659]]></doi>
##     <publicationId><![CDATA[7005374]]></publicationId>
##     <mdurl><![CDATA[http://ieeexplore.ieee.org/xpl/articleDetails.jsp?tp=&arnumber=7005374&contentType=Journals+%26+Magazines]]></mdurl>
##     <pdf><![CDATA[http://ieeexplore.ieee.org/stamp/stamp.jsp?arnumber=7005374]]></pdf>
##   </document>
##   <document>
##     <rank>226</rank>
##     <title><![CDATA[Imaging and analysis techniques for electrical trees using X-ray computed tomography]]></title>
##     <authors><![CDATA[Schurch, R.;  Rowland, S.M.;  Bradley, R.S.;  Withers, P.J.]]></authors>
##     <affiliations><![CDATA[Sch. of Electr. & Electron. Eng., Univ. of Manchester, Manchester, UK]]></affiliations>
##     <controlledterms>
##       <term><![CDATA[computerised tomography]]></term>
##       <term><![CDATA[electrical engineering computing]]></term>
##       <term><![CDATA[nondestructive testing]]></term>
##       <term><![CDATA[scanning electron microscopy]]></term>
##       <term><![CDATA[trees (electrical)]]></term>
##     </controlledterms>
##     <thesaurusterms>
##       <term><![CDATA[Image resolution]]></term>
##       <term><![CDATA[Materials]]></term>
##       <term><![CDATA[Optical imaging]]></term>
##       <term><![CDATA[Scanning electron microscopy]]></term>
##       <term><![CDATA[X-ray imaging]]></term>
##     </thesaurusterms>
##     <pubtitle><![CDATA[Dielectrics and Electrical Insulation, IEEE Transactions on]]></pubtitle>
##     <punumber><![CDATA[94]]></punumber>
##     <pubtype><![CDATA[Journals & Magazines]]></pubtype>
##     <publisher><![CDATA[IEEE]]></publisher>
##     <volume><![CDATA[21]]></volume>
##     <issue><![CDATA[1]]></issue>
##     <py><![CDATA[2014]]></py>
##     <spage><![CDATA[53]]></spage>
##     <epage><![CDATA[63]]></epage>
##     <abstract><![CDATA[Electrical treeing is one of the main mechanisms of degradation in polymeric high voltage insulation, a precursor of power equipment failure. Electrical trees have been previously imaged mostly using two-dimensional imaging techniques; thereby loosing valuable information. Here we review the techniques that have been previously used and present the novel application of X-ray computed tomography (XCT) for electrical tree imaging. This non-destructive technique is able to reveal electrical trees, providing a threedimensional (3-D) view and therefore, a more complete representation of the phenomenon can be achieved. Moreover, taking virtual slices through the replica so created brings the possibility of internal exploration of the electrical tree, without the destruction of the specimen. Here, laboratory created electrical trees have been scanned using XCT with phase contrast enhancement, and 3-D virtual replicas created through which the trees are analyzed. Serial Block-Face scanning electron microscopy (SBFSEM) is shown to be a successful complementary technique. Computed tomography enables quantification of electrical tree characteristics that previously were not available. Characteristics such as the diameter and tortuosity of tree channels, as well as the overall tree volume can be calculated. Through the cross-section analysis, the progression of the number of tree channels and the area covered by them can be investigated. Using this approach it is expected that a better understanding of electrical treeing phenomenon will be developed.]]></abstract>
##     <issn><![CDATA[1070-9878]]></issn>
##     <htmlFlag><![CDATA[1]]></htmlFlag>
##     <arnumber><![CDATA[6740725]]></arnumber>
##     <doi><![CDATA[10.1109/TDEI.2013.003911]]></doi>
##     <publicationId><![CDATA[6740725]]></publicationId>
##     <mdurl><![CDATA[http://ieeexplore.ieee.org/xpl/articleDetails.jsp?tp=&arnumber=6740725&contentType=Journals+%26+Magazines]]></mdurl>
##     <pdf><![CDATA[http://ieeexplore.ieee.org/stamp/stamp.jsp?arnumber=6740725]]></pdf>
##   </document>
##   <document>
##     <rank>227</rank>
##     <title><![CDATA[Enhanced Broadband Parametric Wavelength Conversion in Silicon Waveguide With the Multi-Period Grating]]></title>
##     <authors><![CDATA[Xianting Zhang;  Jinhui Yuan;  Jiaxiu Zou;  Boyuan Jin;  Xinzhu Sang;  Qiang Wu;  Chongxiu Yu;  Farrell, G.]]></authors>
##     <affiliations><![CDATA[State Key Lab. of Inf. Photonics & Opt. Commun., Beijing Univ. of Posts & Telecommun., Beijing, China]]></affiliations>
##     <controlledterms>
##       <term><![CDATA[diffraction gratings]]></term>
##       <term><![CDATA[elemental semiconductors]]></term>
##       <term><![CDATA[integrated optics]]></term>
##       <term><![CDATA[optical phase matching]]></term>
##       <term><![CDATA[optical pumping]]></term>
##       <term><![CDATA[optical waveguides]]></term>
##       <term><![CDATA[optical wavelength conversion]]></term>
##       <term><![CDATA[silicon]]></term>
##       <term><![CDATA[silicon-on-insulator]]></term>
##     </controlledterms>
##     <thesaurusterms>
##       <term><![CDATA[Absorption]]></term>
##       <term><![CDATA[Dispersion]]></term>
##       <term><![CDATA[Gratings]]></term>
##       <term><![CDATA[Optical waveguides]]></term>
##       <term><![CDATA[Optical wavelength conversion]]></term>
##       <term><![CDATA[Silicon]]></term>
##     </thesaurusterms>
##     <pubtitle><![CDATA[Photonics Journal, IEEE]]></pubtitle>
##     <punumber><![CDATA[4563994]]></punumber>
##     <pubtype><![CDATA[Journals & Magazines]]></pubtype>
##     <publisher><![CDATA[IEEE]]></publisher>
##     <volume><![CDATA[6]]></volume>
##     <issue><![CDATA[6]]></issue>
##     <py><![CDATA[2014]]></py>
##     <spage><![CDATA[1]]></spage>
##     <epage><![CDATA[14]]></epage>
##     <abstract><![CDATA[In this paper, a novel silicon-on-insulator (SOI) waveguide with the multi-period vertical grating is proposed to realize broadband parametric wavelength conversion with a quasi-phase matching technique. The grating period in the former part of the SOI waveguide is optimized to enhance the conversion efficiency at designated signal wavelength and the 3-dB bandwidth. When the continuous-wave pump at 1550 nm is used, the conversion efficiency of -14.0 dB at 1750 nm and the 3-dB bandwidth of 387 nm can be obtained. Compared to the constant-width waveguide, the improvements of 26.7 dB and 298 nm are achieved, respectively. The results show that this SOI waveguide is an ideal device for broadband wavelength conversion without dispersion engineering.]]></abstract>
##     <issn><![CDATA[1943-0655]]></issn>
##     <htmlFlag><![CDATA[1]]></htmlFlag>
##     <arnumber><![CDATA[6945346]]></arnumber>
##     <doi><![CDATA[10.1109/JPHOT.2014.2366150]]></doi>
##     <publicationId><![CDATA[6945346]]></publicationId>
##     <mdurl><![CDATA[http://ieeexplore.ieee.org/xpl/articleDetails.jsp?tp=&arnumber=6945346&contentType=Journals+%26+Magazines]]></mdurl>
##     <pdf><![CDATA[http://ieeexplore.ieee.org/stamp/stamp.jsp?arnumber=6945346]]></pdf>
##   </document>
##   <document>
##     <rank>228</rank>
##     <title><![CDATA[Low-Temperature Naturatron Sputtering System for Deposition of Indium Tin Oxide Film]]></title>
##     <authors><![CDATA[Thungsuk, N.;  Yuji, T.;  Kasayapanand, N.;  Mungkung, N.;  Nuachauy, P.;  Arunrungrusmi, S.;  Nakabayashi, K.;  Okamura, Y.;  Kinoshita, H.;  Kataoka, H.;  Suzaki, Y.;  Hirata, T.]]></authors>
##     <affiliations><![CDATA[Div. of Energy Technol., King Mongkut's Univ. of Technol. Thonburi, Bangkok, Thailand]]></affiliations>
##     <controlledterms>
##       <term><![CDATA[X-ray chemical analysis]]></term>
##       <term><![CDATA[indium compounds]]></term>
##       <term><![CDATA[scanning electron microscopy]]></term>
##       <term><![CDATA[sputter deposition]]></term>
##       <term><![CDATA[thin films]]></term>
##       <term><![CDATA[tin compounds]]></term>
##     </controlledterms>
##     <thesaurusterms>
##       <term><![CDATA[Educational institutions]]></term>
##       <term><![CDATA[Indium tin oxide]]></term>
##       <term><![CDATA[Photovoltaic cells]]></term>
##       <term><![CDATA[Plasmas]]></term>
##       <term><![CDATA[Sputtering]]></term>
##       <term><![CDATA[Substrates]]></term>
##       <term><![CDATA[Surface treatment]]></term>
##     </thesaurusterms>
##     <pubtitle><![CDATA[Plasma Science, IEEE Transactions on]]></pubtitle>
##     <punumber><![CDATA[27]]></punumber>
##     <pubtype><![CDATA[Journals & Magazines]]></pubtype>
##     <publisher><![CDATA[IEEE]]></publisher>
##     <volume><![CDATA[42]]></volume>
##     <issue><![CDATA[10]]></issue>
##     <part><![CDATA[3]]></part>
##     <py><![CDATA[2014]]></py>
##     <spage><![CDATA[3391]]></spage>
##     <epage><![CDATA[3396]]></epage>
##     <abstract><![CDATA[In this paper, we have newly developed a metal thin film-forming sputtering system using the Naturatron Sputtering method that can prevent the plastic film from suffering damage caused by the high-energy particles in plasma and carry out the low-temperature high-density metal deposition with a sputtering chamber and a film deposition chamber separated from each other. This system has made it possible to deposit the indium tin oxide (ITO) thin film on the poly(ethylene naphthalate) film as a substrate. As a result of energy-dispersive X-ray spectroscopy analysis or scanning electron microscope analysis performed for the ITO thin film, it has been proven that the uniform-surface ITO thin film can be deposited on a plastic film.]]></abstract>
##     <issn><![CDATA[0093-3813]]></issn>
##     <htmlFlag><![CDATA[1]]></htmlFlag>
##     <arnumber><![CDATA[6906299]]></arnumber>
##     <doi><![CDATA[10.1109/TPS.2014.2356332]]></doi>
##     <publicationId><![CDATA[6906299]]></publicationId>
##     <mdurl><![CDATA[http://ieeexplore.ieee.org/xpl/articleDetails.jsp?tp=&arnumber=6906299&contentType=Journals+%26+Magazines]]></mdurl>
##     <pdf><![CDATA[http://ieeexplore.ieee.org/stamp/stamp.jsp?arnumber=6906299]]></pdf>
##   </document>
##   <document>
##     <rank>229</rank>
##     <title><![CDATA[Distributed Client-Server Assignment for Online Social Network Applications]]></title>
##     <authors><![CDATA[Thuan Duong-Ba;  Thinh Nguyen;  Bose, B.;  Tran, D.A.]]></authors>
##     <affiliations><![CDATA[Dept. of Electr. Eng. & Comput. Sci., Oregon State Univ., Corvallis, OR, USA]]></affiliations>
##     <controlledterms>
##       <term><![CDATA[client-server systems]]></term>
##       <term><![CDATA[computational complexity]]></term>
##       <term><![CDATA[search problems]]></term>
##       <term><![CDATA[simulated annealing]]></term>
##       <term><![CDATA[social networking (online)]]></term>
##     </controlledterms>
##     <thesaurusterms>
##       <term><![CDATA[Approximation algorithms]]></term>
##       <term><![CDATA[Distributed algorithms]]></term>
##       <term><![CDATA[Distributed databases]]></term>
##       <term><![CDATA[Heuristic algorithms]]></term>
##       <term><![CDATA[Linear programming]]></term>
##       <term><![CDATA[Online services]]></term>
##       <term><![CDATA[Operations research]]></term>
##       <term><![CDATA[Partitioning algorithms]]></term>
##       <term><![CDATA[Simulated annealing]]></term>
##       <term><![CDATA[Social network services]]></term>
##     </thesaurusterms>
##     <pubtitle><![CDATA[Emerging Topics in Computing, IEEE Transactions on]]></pubtitle>
##     <punumber><![CDATA[6245516]]></punumber>
##     <pubtype><![CDATA[Journals & Magazines]]></pubtype>
##     <publisher><![CDATA[IEEE]]></publisher>
##     <volume><![CDATA[2]]></volume>
##     <issue><![CDATA[4]]></issue>
##     <py><![CDATA[2014]]></py>
##     <spage><![CDATA[422]]></spage>
##     <epage><![CDATA[435]]></epage>
##     <abstract><![CDATA[We study the problem of assigning users to servers with an emphasis on the distributed algorithmic solutions. Typical online social network applications, such as Facebook and Twitter, are built on top of an infrastructure of servers that provides the services on behalf of the users. For a given communication pattern among users, the loads of the servers depend critically on how the users are assigned to the servers. A good assignment will reduce the overall load of the system while balancing the loads among the servers. Unfortunately, this optimal assignment problem is NP-hard. Therefore, we investigate three heuristic algorithms for solving the user server assignment problem: 1) the centralized simulated annealing (CSA) algorithm; 2) the distributed simulated annealing (DSA) algorithm; and 3) the distributed perturbed greedy search (DPGS). The CSA algorithm produces good solution in the fastest time, however it relies on timely accurate global system information, and is practical only for small and static systems. In contrast, the two distributed algorithms, DSA and DPGS, exploit local information at each server during their search for the optimal assignment, and thus can scale well with the number of users and servers as well as adapting to the system dynamics. Simulation results show that the performance of the distributed algorithms, specifically the DPGS algorithm, is very competitive with that of the centralized algorithm while providing the advantage of naturally adapting to time-varying communication patterns of users.]]></abstract>
##     <issn><![CDATA[2168-6750]]></issn>
##     <htmlFlag><![CDATA[1]]></htmlFlag>
##     <arnumber><![CDATA[6909044]]></arnumber>
##     <doi><![CDATA[10.1109/TETC.2014.2358801]]></doi>
##     <publicationId><![CDATA[6909044]]></publicationId>
##     <mdurl><![CDATA[http://ieeexplore.ieee.org/xpl/articleDetails.jsp?tp=&arnumber=6909044&contentType=Journals+%26+Magazines]]></mdurl>
##     <pdf><![CDATA[http://ieeexplore.ieee.org/stamp/stamp.jsp?arnumber=6909044]]></pdf>
##   </document>
##   <document>
##     <rank>230</rank>
##     <title><![CDATA[High Coupling Efficiency Etched Facet Tapers in Silicon Waveguides]]></title>
##     <authors><![CDATA[Cardenas, J.;  Poitras, C.B.;  Luke, K.;  Lian-Wee Luo;  Morton, P.A.;  Lipson, M.]]></authors>
##     <affiliations><![CDATA[Sch. of Electr. & Comput. Eng., Cornell Univ., Ithaca, NY, USA]]></affiliations>
##     <controlledterms>
##       <term><![CDATA[etching]]></term>
##       <term><![CDATA[integrated optics]]></term>
##       <term><![CDATA[lenses]]></term>
##       <term><![CDATA[optical fibre couplers]]></term>
##       <term><![CDATA[optical fibre fabrication]]></term>
##       <term><![CDATA[optical fibre losses]]></term>
##       <term><![CDATA[silicon]]></term>
##     </controlledterms>
##     <thesaurusterms>
##       <term><![CDATA[Couplers]]></term>
##       <term><![CDATA[Couplings]]></term>
##       <term><![CDATA[Loss measurement]]></term>
##       <term><![CDATA[Optical fiber couplers]]></term>
##       <term><![CDATA[Silicon]]></term>
##     </thesaurusterms>
##     <pubtitle><![CDATA[Photonics Technology Letters, IEEE]]></pubtitle>
##     <punumber><![CDATA[68]]></punumber>
##     <pubtype><![CDATA[Journals & Magazines]]></pubtype>
##     <publisher><![CDATA[IEEE]]></publisher>
##     <volume><![CDATA[26]]></volume>
##     <issue><![CDATA[23]]></issue>
##     <py><![CDATA[2014]]></py>
##     <spage><![CDATA[2380]]></spage>
##     <epage><![CDATA[2382]]></epage>
##     <abstract><![CDATA[We demonstrate a platform based on etched facet silicon inverse tapers for waveguide-lensed fiber coupling with a loss as low as 0.7 dB/facet. This platform can be fabricated on a wafer scale enabling mass-production of silicon photonic devices with broadband, high-efficiency couplers.]]></abstract>
##     <issn><![CDATA[1041-1135]]></issn>
##     <htmlFlag><![CDATA[1]]></htmlFlag>
##     <arnumber><![CDATA[6895281]]></arnumber>
##     <doi><![CDATA[10.1109/LPT.2014.2357177]]></doi>
##     <publicationId><![CDATA[6895281]]></publicationId>
##     <mdurl><![CDATA[http://ieeexplore.ieee.org/xpl/articleDetails.jsp?tp=&arnumber=6895281&contentType=Journals+%26+Magazines]]></mdurl>
##     <pdf><![CDATA[http://ieeexplore.ieee.org/stamp/stamp.jsp?arnumber=6895281]]></pdf>
##   </document>
##   <document>
##     <rank>231</rank>
##     <title><![CDATA[Modeling, Control, and Implementation of DC&#x2013;DC Converters for Variable Frequency Operation]]></title>
##     <authors><![CDATA[Priewasser, R.;  Agostinelli, M.;  Unterrieder, C.;  Marsili, S.;  Huemer, M.]]></authors>
##     <affiliations><![CDATA[Infineon Technol. Austria, Villach, Austria]]></affiliations>
##     <controlledterms>
##       <term><![CDATA[CMOS integrated circuits]]></term>
##       <term><![CDATA[DC-DC power convertors]]></term>
##       <term><![CDATA[field programmable gate arrays]]></term>
##       <term><![CDATA[power system control]]></term>
##       <term><![CDATA[three-term control]]></term>
##     </controlledterms>
##     <thesaurusterms>
##       <term><![CDATA[Equations]]></term>
##       <term><![CDATA[Integrated circuit modeling]]></term>
##       <term><![CDATA[Load modeling]]></term>
##       <term><![CDATA[Mathematical model]]></term>
##       <term><![CDATA[Modulation]]></term>
##       <term><![CDATA[Switches]]></term>
##     </thesaurusterms>
##     <pubtitle><![CDATA[Power Electronics, IEEE Transactions on]]></pubtitle>
##     <punumber><![CDATA[63]]></punumber>
##     <pubtype><![CDATA[Journals & Magazines]]></pubtype>
##     <publisher><![CDATA[IEEE]]></publisher>
##     <volume><![CDATA[29]]></volume>
##     <issue><![CDATA[1]]></issue>
##     <py><![CDATA[2014]]></py>
##     <spage><![CDATA[287]]></spage>
##     <epage><![CDATA[301]]></epage>
##     <abstract><![CDATA[In this paper, novel small-signal averaged models for dc-dc converters operating at variable switching frequency are derived. This is achieved by separately considering the on-time and the off-time of the switching period. The derivation is shown in detail for a synchronous buck converter and the model for a boost converter is also presented. The model for the buck converter is then used for the design of two digital feedback controllers, which exploit the additional insight in the converter dynamics. First, a digital multiloop PID controller is implemented, where the design is based on loop-shaping of the proposed frequency-domain transfer functions. And second, the design and the implementation of a digital LQG state-feedback controller, based on the proposed time-domain state-space model, is presented for the same converter topology. Experimental results are given for the digital multiloop PID controller integrated on an application-specified integrated circuit in a 0.13 &#x03BC;m CMOS technology, as well as for the state-feedback controller implemented on an FPGA. Tight output voltage regulation and an excellent dynamic performance is achieved, as the dynamics of the converter under variable frequency operation are considered during the design of both implementations.]]></abstract>
##     <issn><![CDATA[0885-8993]]></issn>
##     <htmlFlag><![CDATA[1]]></htmlFlag>
##     <arnumber><![CDATA[6472322]]></arnumber>
##     <doi><![CDATA[10.1109/TPEL.2013.2248751]]></doi>
##     <publicationId><![CDATA[6472322]]></publicationId>
##     <mdurl><![CDATA[http://ieeexplore.ieee.org/xpl/articleDetails.jsp?tp=&arnumber=6472322&contentType=Journals+%26+Magazines]]></mdurl>
##     <pdf><![CDATA[http://ieeexplore.ieee.org/stamp/stamp.jsp?arnumber=6472322]]></pdf>
##   </document>
##   <document>
##     <rank>232</rank>
##     <title><![CDATA[Tuneable Dual-Mode Micro-Resonator Associating Photonic Crystal Membrane and Fabry&#x2013;Perot Cavity]]></title>
##     <authors><![CDATA[Kusiaku, K.;  Leclercq, J.-L.;  Viktorovitch, P.;  Letartre, X.]]></authors>
##     <affiliations><![CDATA[Inst. des Nanotechnol. de Lyon (INL), Univ. de Lyon, Ecully, France]]></affiliations>
##     <controlledterms>
##       <term><![CDATA[Fabry-Perot resonators]]></term>
##       <term><![CDATA[membranes]]></term>
##       <term><![CDATA[micromechanical resonators]]></term>
##       <term><![CDATA[optical tuning]]></term>
##       <term><![CDATA[photonic crystals]]></term>
##     </controlledterms>
##     <thesaurusterms>
##       <term><![CDATA[Cavity resonators]]></term>
##       <term><![CDATA[Couplings]]></term>
##       <term><![CDATA[Mirrors]]></term>
##       <term><![CDATA[Optical resonators]]></term>
##       <term><![CDATA[Phase change materials]]></term>
##       <term><![CDATA[Q-factor]]></term>
##       <term><![CDATA[Resonant frequency]]></term>
##     </thesaurusterms>
##     <pubtitle><![CDATA[Photonics Journal, IEEE]]></pubtitle>
##     <punumber><![CDATA[4563994]]></punumber>
##     <pubtype><![CDATA[Journals & Magazines]]></pubtype>
##     <publisher><![CDATA[IEEE]]></publisher>
##     <volume><![CDATA[6]]></volume>
##     <issue><![CDATA[2]]></issue>
##     <py><![CDATA[2014]]></py>
##     <spage><![CDATA[1]]></spage>
##     <epage><![CDATA[9]]></epage>
##     <abstract><![CDATA[We report on a tuneable dual-wavelength micro-resonator with a resonant photonic crystal membrane (PCM) inserted in a vertical Fabry-Perot (FP) cavity. Strong optical coupling between both resonators leads to dual-wavelength resonances. Their energy difference, determined by the overlap of both modes, can be tuned using micro-opto-electro-mechanical systems. Variations in spectral and field overlaps are considered separately with a phenomenological matrix method through, respectively, the FP cavity thickness and the PCM position change. This approach is compared with 2-D finite-difference time-domain simulations in the second case. Periodic evolution of both modes is observed with unsymmetrical amplitude, unlike with the model cause of its approximations.]]></abstract>
##     <issn><![CDATA[1943-0655]]></issn>
##     <htmlFlag><![CDATA[1]]></htmlFlag>
##     <arnumber><![CDATA[6744617]]></arnumber>
##     <doi><![CDATA[10.1109/JPHOT.2014.2306841]]></doi>
##     <publicationId><![CDATA[6744617]]></publicationId>
##     <mdurl><![CDATA[http://ieeexplore.ieee.org/xpl/articleDetails.jsp?tp=&arnumber=6744617&contentType=Journals+%26+Magazines]]></mdurl>
##     <pdf><![CDATA[http://ieeexplore.ieee.org/stamp/stamp.jsp?arnumber=6744617]]></pdf>
##   </document>
##   <document>
##     <rank>233</rank>
##     <title><![CDATA[A Method to Standardize Quantification of Left Atrial Scar From Delayed-Enhancement MR Images]]></title>
##     <authors><![CDATA[Karim, R.;  Arujuna, A.;  Housden, R.J.;  Gill, J.;  Cliffe, H.;  Matharu, K.;  Gill, J.;  Rindaldi, C.A.;  O'Neill, M.;  Rueckert, D.;  Razavi, R.;  Schaeffter, T.;  Rhode, K.]]></authors>
##     <affiliations><![CDATA[Div. of Imaging Sci. & Biomed. Eng., King's Coll. London, London, UK]]></affiliations>
##     <controlledterms>
##       <term><![CDATA[biomedical MRI]]></term>
##       <term><![CDATA[blood vessels]]></term>
##       <term><![CDATA[cardiovascular system]]></term>
##       <term><![CDATA[image segmentation]]></term>
##       <term><![CDATA[medical disorders]]></term>
##       <term><![CDATA[medical image processing]]></term>
##       <term><![CDATA[phantoms]]></term>
##       <term><![CDATA[radiation therapy]]></term>
##       <term><![CDATA[radiofrequency heating]]></term>
##       <term><![CDATA[standardisation]]></term>
##     </controlledterms>
##     <thesaurusterms>
##       <term><![CDATA[Adaptation models]]></term>
##       <term><![CDATA[Biolgical tissues]]></term>
##       <term><![CDATA[Biomedical imaging]]></term>
##       <term><![CDATA[Blood]]></term>
##       <term><![CDATA[Image segmentation]]></term>
##       <term><![CDATA[Magnetic resonance imaging]]></term>
##       <term><![CDATA[Myocardium]]></term>
##     </thesaurusterms>
##     <pubtitle><![CDATA[Translational Engineering in Health and Medicine, IEEE Journal of]]></pubtitle>
##     <punumber><![CDATA[6221039]]></punumber>
##     <pubtype><![CDATA[Journals & Magazines]]></pubtype>
##     <publisher><![CDATA[IEEE]]></publisher>
##     <volume><![CDATA[2]]></volume>
##     <py><![CDATA[2014]]></py>
##     <spage><![CDATA[1]]></spage>
##     <epage><![CDATA[15]]></epage>
##     <abstract><![CDATA[Delayed-enhancement magnetic resonance imaging (DE-MRI) is an effective technique for detecting left atrial (LA) fibrosis both pre and postradiofrequency ablation for the treatment of atrial fibrillation. Fixed thresholding models are frequently utilized clinically to segment and quantify scar in DE-MRI due to their simplicity. These methods fail to provide a standardized quantification due to interobserver variability. Quantification of scar can be used as an endpoint in clinical studies and therefore standardization is important. In this paper, we propose a segmentation algorithm for LA fibrosis quantification and investigate its performance. The algorithm was validated using numerical phantoms and 15 clinical data sets from patients undergoing LA ablation. We demonstrate that the approach produces good concordance with expert manual delineations. The method offers a standardized quantification technique for evaluation and interpretation of DE-MRI scans.]]></abstract>
##     <issn><![CDATA[2168-2372]]></issn>
##     <htmlFlag><![CDATA[1]]></htmlFlag>
##     <arnumber><![CDATA[6774956]]></arnumber>
##     <doi><![CDATA[10.1109/JTEHM.2014.2312191]]></doi>
##     <publicationId><![CDATA[6774956]]></publicationId>
##     <mdurl><![CDATA[http://ieeexplore.ieee.org/xpl/articleDetails.jsp?tp=&arnumber=6774956&contentType=Journals+%26+Magazines]]></mdurl>
##     <pdf><![CDATA[http://ieeexplore.ieee.org/stamp/stamp.jsp?arnumber=6774956]]></pdf>
##   </document>
##   <document>
##     <rank>234</rank>
##     <title><![CDATA[Rethinking the Data Center Networking: Architecture, Network Protocols, and Resource Sharing]]></title>
##     <authors><![CDATA[Ting Wang;  Zhiyang Su;  Yu Xia;  Hamdi, M.]]></authors>
##     <affiliations><![CDATA[Dept. of Comput. Sci. & Eng., Hong Kong Univ. of Sci. & Technol., Hong Kong, China]]></affiliations>
##     <controlledterms>
##       <term><![CDATA[cloud computing]]></term>
##       <term><![CDATA[computer centres]]></term>
##     </controlledterms>
##     <thesaurusterms>
##       <term><![CDATA[Cloud computing]]></term>
##       <term><![CDATA[Computer architecture]]></term>
##       <term><![CDATA[Computer security]]></term>
##       <term><![CDATA[Data centers]]></term>
##       <term><![CDATA[Design methodology]]></term>
##       <term><![CDATA[Information technology]]></term>
##       <term><![CDATA[Large-scale systems]]></term>
##       <term><![CDATA[Resource management]]></term>
##       <term><![CDATA[Servers]]></term>
##     </thesaurusterms>
##     <pubtitle><![CDATA[Access, IEEE]]></pubtitle>
##     <punumber><![CDATA[6287639]]></punumber>
##     <pubtype><![CDATA[Journals & Magazines]]></pubtype>
##     <publisher><![CDATA[IEEE]]></publisher>
##     <volume><![CDATA[2]]></volume>
##     <py><![CDATA[2014]]></py>
##     <spage><![CDATA[1481]]></spage>
##     <epage><![CDATA[1496]]></epage>
##     <abstract><![CDATA[Large-scale data centers enable the new era of cloud computing and provide the core infrastructure to meet the computing and storage requirements for both enterprise information technology needs and cloud-based services. To support the ever-growing cloud computing needs, the number of servers in today's data centers are increasing exponentially, which in turn leads to enormous challenges in designing an efficient and cost-effective data center network. With data availability and security at stake, the issues with data center networks are more critical than ever. Motivated by these challenges and critical issues, many novel and creative research works have been proposed in recent years. In this paper, we investigate in data center networks and provide a general overview and analysis of the literature covering various research areas, including data center network interconnection architectures, network protocols for data center networks, and network resource sharing in multitenant cloud data centers. We start with an overview on data center networks and together with its requirements navigate the data center network designs. We then present the research literature related to the aforementioned research topics in the subsequent sections. Finally, we draw the conclusions.]]></abstract>
##     <issn><![CDATA[2169-3536]]></issn>
##     <htmlFlag><![CDATA[1]]></htmlFlag>
##     <arnumber><![CDATA[6990724]]></arnumber>
##     <doi><![CDATA[10.1109/ACCESS.2014.2383439]]></doi>
##     <publicationId><![CDATA[6990724]]></publicationId>
##     <mdurl><![CDATA[http://ieeexplore.ieee.org/xpl/articleDetails.jsp?tp=&arnumber=6990724&contentType=Journals+%26+Magazines]]></mdurl>
##     <pdf><![CDATA[http://ieeexplore.ieee.org/stamp/stamp.jsp?arnumber=6990724]]></pdf>
##   </document>
##   <document>
##     <rank>235</rank>
##     <title><![CDATA[Max-Ratio Relay Selection in Secure Buffer-Aided Cooperative Wireless Networks]]></title>
##     <authors><![CDATA[Gaojie Chen;  Zhao Tian;  Yu Gong;  Zhi Chen;  Chambers, J.A.]]></authors>
##     <affiliations><![CDATA[Adv. Signal Process. Group, Loughborough Univ., Loughborough, UK]]></affiliations>
##     <controlledterms>
##       <term><![CDATA[buffer storage]]></term>
##       <term><![CDATA[cooperative communication]]></term>
##       <term><![CDATA[decode and forward communication]]></term>
##       <term><![CDATA[probability]]></term>
##       <term><![CDATA[relay networks (telecommunication)]]></term>
##       <term><![CDATA[telecommunication security]]></term>
##     </controlledterms>
##     <thesaurusterms>
##       <term><![CDATA[Educational institutions]]></term>
##       <term><![CDATA[Fading]]></term>
##       <term><![CDATA[Jamming]]></term>
##       <term><![CDATA[Relays]]></term>
##       <term><![CDATA[Security]]></term>
##       <term><![CDATA[Signal to noise ratio]]></term>
##       <term><![CDATA[Wireless communication]]></term>
##     </thesaurusterms>
##     <pubtitle><![CDATA[Information Forensics and Security, IEEE Transactions on]]></pubtitle>
##     <punumber><![CDATA[10206]]></punumber>
##     <pubtype><![CDATA[Journals & Magazines]]></pubtype>
##     <publisher><![CDATA[IEEE]]></publisher>
##     <volume><![CDATA[9]]></volume>
##     <issue><![CDATA[4]]></issue>
##     <py><![CDATA[2014]]></py>
##     <spage><![CDATA[719]]></spage>
##     <epage><![CDATA[729]]></epage>
##     <abstract><![CDATA[This paper considers the security of transmission in buffer-aided decode-and-forward cooperative wireless networks. An eavesdropper which can intercept the data transmission from both the source and relay nodes is considered to threaten the security of transmission. Finite size data buffers are assumed to be available at every relay in order to avoid having to select concurrently the best source-to-relay and relay-to-destination links. A new max-ratio relay selection policy is proposed to optimize the secrecy transmission by considering all the possible source-to-relay and relay-to-destination links and selecting the relay having the link which maximizes the signal to eavesdropper channel gain ratio. Two cases are considered in terms of knowledge of the eavesdropper channel strengths: exact and average gains, respectively. Closed-form expressions for the secrecy outage probability for both cases are obtained, which are verified by simulations. The proposed max-ratio relay selection scheme is shown to outperform one based on a max-min-ratio relay scheme.]]></abstract>
##     <issn><![CDATA[1556-6013]]></issn>
##     <htmlFlag><![CDATA[1]]></htmlFlag>
##     <arnumber><![CDATA[6746659]]></arnumber>
##     <doi><![CDATA[10.1109/TIFS.2014.2307672]]></doi>
##     <publicationId><![CDATA[6746659]]></publicationId>
##     <mdurl><![CDATA[http://ieeexplore.ieee.org/xpl/articleDetails.jsp?tp=&arnumber=6746659&contentType=Journals+%26+Magazines]]></mdurl>
##     <pdf><![CDATA[http://ieeexplore.ieee.org/stamp/stamp.jsp?arnumber=6746659]]></pdf>
##   </document>
##   <document>
##     <rank>236</rank>
##     <title><![CDATA[Gastric Contraction Imaging System Using a 3-D Endoscope]]></title>
##     <authors><![CDATA[Yoshimoto, K.;  Yamada, K.;  Watabe, K.;  Takeda, M.;  Nishimura, T.;  Kido, M.;  Nagakura, T.;  Takahashi, H.;  Nishida, T.;  Iijima, H.;  Tsujii, M.;  Takehara, T.;  Ohno, Y.]]></authors>
##     <affiliations><![CDATA[Grad. Sch. of Med., Osaka Univ., Suita, Japan]]></affiliations>
##     <controlledterms>
##       <term><![CDATA[Gaussian processes]]></term>
##       <term><![CDATA[biological organs]]></term>
##       <term><![CDATA[biomechanics]]></term>
##       <term><![CDATA[biomedical optical imaging]]></term>
##       <term><![CDATA[diseases]]></term>
##       <term><![CDATA[endoscopes]]></term>
##       <term><![CDATA[image motion analysis]]></term>
##       <term><![CDATA[medical image processing]]></term>
##       <term><![CDATA[physiological models]]></term>
##       <term><![CDATA[prototypes]]></term>
##       <term><![CDATA[wave propagation]]></term>
##     </controlledterms>
##     <thesaurusterms>
##       <term><![CDATA[Endoscopes]]></term>
##       <term><![CDATA[Gastrointestinal tract]]></term>
##       <term><![CDATA[Stomach]]></term>
##       <term><![CDATA[Surface reconstruction]]></term>
##       <term><![CDATA[Surface waves]]></term>
##       <term><![CDATA[Three dimensional displays]]></term>
##     </thesaurusterms>
##     <pubtitle><![CDATA[Translational Engineering in Health and Medicine, IEEE Journal of]]></pubtitle>
##     <punumber><![CDATA[6221039]]></punumber>
##     <pubtype><![CDATA[Journals & Magazines]]></pubtype>
##     <publisher><![CDATA[IEEE]]></publisher>
##     <volume><![CDATA[2]]></volume>
##     <py><![CDATA[2014]]></py>
##     <spage><![CDATA[1]]></spage>
##     <epage><![CDATA[8]]></epage>
##     <abstract><![CDATA[This paper presents a gastric contraction imaging system for assessment of gastric motility using a 3-D endoscope. Gastrointestinal diseases are mainly based on morphological abnormalities. However, gastrointestinal symptoms are sometimes apparent without visible abnormalities. One of the major factors for these diseases is abnormal gastrointestinal motility. For assessment of gastric motility, a gastric motility imaging system is needed. To assess the dynamic motility of the stomach, the proposed system measures 3-D gastric contractions derived from a 3-D profile of the stomach wall obtained with a developed 3-D endoscope. After obtaining contraction waves, their frequency, amplitude, and speed of propagation can be calculated using a Gaussian function. The proposed system was evaluated for 3-D measurements of several objects with known geometries. The results showed that the surface profiles could be obtained with an error of &lt;;10% of the distance between two different points on images. Subsequently, we evaluated the validity of a prototype system using a wave simulated model. In the experiment, the amplitude and position of waves could be measured with 1-mm accuracy. The present results suggest that the proposed system can measure the speed and amplitude of contractions. This system has low invasiveness and can assess the motility of the stomach wall directly in a 3-D manner. Our method can be used for examination of gastric morphological and functional abnormalities.]]></abstract>
##     <issn><![CDATA[2168-2372]]></issn>
##     <arnumber><![CDATA[6705622]]></arnumber>
##     <doi><![CDATA[10.1109/JTEHM.2014.2298852]]></doi>
##     <publicationId><![CDATA[6705622]]></publicationId>
##     <mdurl><![CDATA[http://ieeexplore.ieee.org/xpl/articleDetails.jsp?tp=&arnumber=6705622&contentType=Journals+%26+Magazines]]></mdurl>
##     <pdf><![CDATA[http://ieeexplore.ieee.org/stamp/stamp.jsp?arnumber=6705622]]></pdf>
##   </document>
##   <document>
##     <rank>237</rank>
##     <title><![CDATA[Enhanced Field Effect Mobility on 4H-SiC by Oxidation at 1500&#x00B0;C]]></title>
##     <authors><![CDATA[Thomas, S.M.;  Sharma, Y.K.;  Crouch, M.A.;  Fisher, C.A.;  Perez-Tomas, A.;  Jennings, M.R.;  Mawby, P.A.]]></authors>
##     <affiliations><![CDATA[Sch. of Eng., Univ. of Warwick, Coventry, UK]]></affiliations>
##     <controlledterms>
##       <term><![CDATA[MOSFET]]></term>
##       <term><![CDATA[oxidation]]></term>
##       <term><![CDATA[passivation]]></term>
##       <term><![CDATA[silicon compounds]]></term>
##       <term><![CDATA[wide band gap semiconductors]]></term>
##     </controlledterms>
##     <thesaurusterms>
##       <term><![CDATA[Capacitance measurement]]></term>
##       <term><![CDATA[Logic gates]]></term>
##       <term><![CDATA[MOSFET]]></term>
##       <term><![CDATA[Oxidation]]></term>
##       <term><![CDATA[Silicon]]></term>
##       <term><![CDATA[Silicon carbide]]></term>
##       <term><![CDATA[Temperature measurement]]></term>
##     </thesaurusterms>
##     <pubtitle><![CDATA[Electron Devices Society, IEEE Journal of the]]></pubtitle>
##     <punumber><![CDATA[6245494]]></punumber>
##     <pubtype><![CDATA[Journals & Magazines]]></pubtype>
##     <publisher><![CDATA[IEEE]]></publisher>
##     <volume><![CDATA[2]]></volume>
##     <issue><![CDATA[5]]></issue>
##     <py><![CDATA[2014]]></py>
##     <spage><![CDATA[114]]></spage>
##     <epage><![CDATA[117]]></epage>
##     <abstract><![CDATA[A novel 1500&#x00B0;C gate oxidation process has been demonstrated on Si face of 4H-SiC. Lateral channel metal-oxide-semiconductor-field-effect-transistors (MOSFETs) fabricated using this process have a maximum field effect mobility of approximately 40 cm<sup>2</sup> V<sup>-1</sup> s<sup>-1</sup> without post oxidation passivation. This is substantially higher than other reports of MOSFETs with thermally grown oxides (typically grown at the standard silicon temperature range of 1100-1200&#x00B0;C). This result shows the potential of a high temperature oxidation step for reducing the channel resistance (thus the overall conduction loss), in power 4H-SiC MOSFETs.]]></abstract>
##     <issn><![CDATA[2168-6734]]></issn>
##     <htmlFlag><![CDATA[1]]></htmlFlag>
##     <arnumber><![CDATA[6849425]]></arnumber>
##     <doi><![CDATA[10.1109/JEDS.2014.2330737]]></doi>
##     <publicationId><![CDATA[6849425]]></publicationId>
##     <mdurl><![CDATA[http://ieeexplore.ieee.org/xpl/articleDetails.jsp?tp=&arnumber=6849425&contentType=Journals+%26+Magazines]]></mdurl>
##     <pdf><![CDATA[http://ieeexplore.ieee.org/stamp/stamp.jsp?arnumber=6849425]]></pdf>
##   </document>
##   <document>
##     <rank>238</rank>
##     <title><![CDATA[Comparative Evaluation of Methodologies for T-Wave Alternans Mapping in Electrograms]]></title>
##     <authors><![CDATA[Orini, M.;  Hanson, B.;  Monasterio, V.;  Martinez, J.P.;  Hayward, M.;  Taggart, P.;  Lambiase, P.]]></authors>
##     <affiliations><![CDATA[Inst. of Cardiovascular Sci., Univ. Coll. London, London, UK]]></affiliations>
##     <controlledterms>
##       <term><![CDATA[electrocardiography]]></term>
##       <term><![CDATA[impulse noise]]></term>
##       <term><![CDATA[medical signal detection]]></term>
##       <term><![CDATA[moving average processes]]></term>
##       <term><![CDATA[pneumodynamics]]></term>
##       <term><![CDATA[spectral analysis]]></term>
##       <term><![CDATA[time-frequency analysis]]></term>
##     </controlledterms>
##     <thesaurusterms>
##       <term><![CDATA[Accuracy]]></term>
##       <term><![CDATA[Electrodes]]></term>
##       <term><![CDATA[Estimation]]></term>
##       <term><![CDATA[Heart beat]]></term>
##       <term><![CDATA[Morphology]]></term>
##       <term><![CDATA[Noise]]></term>
##       <term><![CDATA[Standards]]></term>
##     </thesaurusterms>
##     <pubtitle><![CDATA[Biomedical Engineering, IEEE Transactions on]]></pubtitle>
##     <punumber><![CDATA[10]]></punumber>
##     <pubtype><![CDATA[Journals & Magazines]]></pubtype>
##     <publisher><![CDATA[IEEE]]></publisher>
##     <volume><![CDATA[61]]></volume>
##     <issue><![CDATA[2]]></issue>
##     <py><![CDATA[2014]]></py>
##     <spage><![CDATA[308]]></spage>
##     <epage><![CDATA[316]]></epage>
##     <abstract><![CDATA[Electrograms (EGM) recorded from the surface of the myocardium are becoming more and more accessible. T-wave alternans (TWA) is associated with increased vulnerability to ventricular tachycardia/fibrillation and it occurs before the onset of ventricular arrhythmias. Thus, accurate methodologies for time-varying alternans estimation/detection in EGM are needed. In this paper, we perform a simulation study based on epicardial EGM recorded in vivo in humans to compare the accuracy of four methodologies: the spectral method (SM), modified moving average method, laplacian likelihood ratio method (LLR), and a novel method based on time-frequency distributions. A variety of effects are considered, which include the presence of wide band noise, respiration, and impulse artifacts. We found that 1) EGM-TWA can be detected accurately when the standard deviation of wide-band noise is equal or smaller than ten times the magnitude of EGM-TWA. 2) Respiration can be critical for EGM-TWA analysis, even at typical respiratory rates. 3) Impulse noise strongly reduces the accuracy of all methods, except LLR. 4) If depolarization time is used as a fiducial point, the localization of the T-wave is not critical for the accuracy of EGM-TWA detection. 5) According to this study, all methodologies provided accurate EGM-TWA detection/quantification in ideal conditions, while LLR was the most robust, providing better detection-rates in noisy conditions. Application on epicardial mapping of the in vivo human heart shows that EGM-TWA has heterogeneous spatio-temporal distribution.]]></abstract>
##     <issn><![CDATA[0018-9294]]></issn>
##     <htmlFlag><![CDATA[1]]></htmlFlag>
##     <arnumber><![CDATA[6656846]]></arnumber>
##     <doi><![CDATA[10.1109/TBME.2013.2289304]]></doi>
##     <publicationId><![CDATA[6656846]]></publicationId>
##     <mdurl><![CDATA[http://ieeexplore.ieee.org/xpl/articleDetails.jsp?tp=&arnumber=6656846&contentType=Journals+%26+Magazines]]></mdurl>
##     <pdf><![CDATA[http://ieeexplore.ieee.org/stamp/stamp.jsp?arnumber=6656846]]></pdf>
##   </document>
##   <document>
##     <rank>239</rank>
##     <title><![CDATA[Trustworthy Sensing for Public Safety in Cloud-Centric Internet of Things]]></title>
##     <authors><![CDATA[Kantarci, B.;  Mouftah, H.T.]]></authors>
##     <affiliations><![CDATA[Sch. of Electr. Eng. & Comput. Sci., Univ. of Ottawa, Ottawa, ON, Canada]]></affiliations>
##     <controlledterms>
##       <term><![CDATA[Internet of Things]]></term>
##       <term><![CDATA[cloud computing]]></term>
##       <term><![CDATA[safety]]></term>
##       <term><![CDATA[software architecture]]></term>
##       <term><![CDATA[trusted computing]]></term>
##     </controlledterms>
##     <thesaurusterms>
##       <term><![CDATA[Cloud computing]]></term>
##       <term><![CDATA[Clouds]]></term>
##       <term><![CDATA[Intelligent sensors]]></term>
##       <term><![CDATA[Safety]]></term>
##       <term><![CDATA[Smart phones]]></term>
##       <term><![CDATA[Social network services]]></term>
##     </thesaurusterms>
##     <pubtitle><![CDATA[Internet of Things Journal, IEEE]]></pubtitle>
##     <punumber><![CDATA[6488907]]></punumber>
##     <pubtype><![CDATA[Journals & Magazines]]></pubtype>
##     <publisher><![CDATA[IEEE]]></publisher>
##     <volume><![CDATA[1]]></volume>
##     <issue><![CDATA[4]]></issue>
##     <py><![CDATA[2014]]></py>
##     <spage><![CDATA[360]]></spage>
##     <epage><![CDATA[368]]></epage>
##     <abstract><![CDATA[The Internet of Things (IoT) paradigm stands for virtually interconnected objects that are identifiable and equipped with sensing, computing, and communication capabilities. Implementation of services and applications over the IoT architecture can take benefit of the cloud computing concept. Sensing-as-a-Service (S<sup>2</sup> aaS) is a cloud-inspired service model which enables access to the IoT. In this paper, we present a framework where IoT can enhance public safety by crowd management via sensing services that are provided by smart phones equipped with various types of sensors. In order to ensure trustworthiness in the presented framework, we propose a reputation-based (S<sup>2</sup> aaS) scheme, namely, Trustworthy Sensing for Crowd Management (TSCM) for front-end access to the IoT. TSCM collects sensing data based on a cloud model and an auction procedure which selects mobile devices for particular sensing tasks and determines the payments to the users of the mobile devices that provide data. Performance evaluation of TSCM shows that the impact of malicious users in the crowdsourced data can be degraded by 75% while trustworthiness of a malicious user converges to a value below 40% following few auctions. Moreover, we show that TSCM can enhance the utility of the public safety authority up to 85%.]]></abstract>
##     <issn><![CDATA[2327-4662]]></issn>
##     <htmlFlag><![CDATA[1]]></htmlFlag>
##     <arnumber><![CDATA[6851843]]></arnumber>
##     <doi><![CDATA[10.1109/JIOT.2014.2337886]]></doi>
##     <publicationId><![CDATA[6851843]]></publicationId>
##     <mdurl><![CDATA[http://ieeexplore.ieee.org/xpl/articleDetails.jsp?tp=&arnumber=6851843&contentType=Journals+%26+Magazines]]></mdurl>
##     <pdf><![CDATA[http://ieeexplore.ieee.org/stamp/stamp.jsp?arnumber=6851843]]></pdf>
##   </document>
##   <document>
##     <rank>240</rank>
##     <title><![CDATA[Symbolic Crosschecking of Data-Parallel Floating-Point Code]]></title>
##     <authors><![CDATA[Collingbourne, P.;  Cadar, C.;  Kelly, P.H.J.]]></authors>
##     <controlledterms>
##       <term><![CDATA[data handling]]></term>
##       <term><![CDATA[floating point arithmetic]]></term>
##       <term><![CDATA[parallel processing]]></term>
##       <term><![CDATA[program debugging]]></term>
##     </controlledterms>
##     <thesaurusterms>
##       <term><![CDATA[Computational modeling]]></term>
##       <term><![CDATA[Computer architecture]]></term>
##       <term><![CDATA[Kernel]]></term>
##       <term><![CDATA[Parallel processing]]></term>
##       <term><![CDATA[Programming]]></term>
##       <term><![CDATA[Semantics]]></term>
##       <term><![CDATA[Vectors]]></term>
##     </thesaurusterms>
##     <pubtitle><![CDATA[Software Engineering, IEEE Transactions on]]></pubtitle>
##     <punumber><![CDATA[32]]></punumber>
##     <pubtype><![CDATA[Journals & Magazines]]></pubtype>
##     <publisher><![CDATA[IEEE]]></publisher>
##     <volume><![CDATA[40]]></volume>
##     <issue><![CDATA[7]]></issue>
##     <py><![CDATA[2014]]></py>
##     <spage><![CDATA[710]]></spage>
##     <epage><![CDATA[737]]></epage>
##     <abstract><![CDATA[We present a symbolic execution-based technique for cross-checking programs accelerated using SIMD or OpenCL against an unaccelerated version, as well as a technique for detecting data races in OpenCL programs. Our techniques are implemented in KLEE-CL, a tool based on the symbolic execution engine KLEE that supports symbolic reasoning on the equivalence between expressions involving both integer and floating-point operations. While the current generation of constraint solvers provide effective support for integer arithmetic, the situation is different for floating-point arithmetic, due to the complexity inherent in such computations. The key insight behind our approach is that floating-point values are only reliably equal if they are essentially built by the same operations. This allows us to use an algorithm based on symbolic expression matching augmented with canonicalisation rules to determine path equivalence. Under symbolic execution, we have to verify equivalence along every feasible control-flow path. We reduce the branching factor of this process by aggressively merging conditionals, if-converting branches into select operations via an aggressive phi-node folding transformation. To support the Intel Streaming SIMD Extension (SSE) instruction set, we lower SSE instructions to equivalent generic vector operations, which in turn are interpreted in terms of primitive integer and floating-point operations. To support OpenCL programs, we symbolically model the OpenCL environment using an OpenCL runtime library targeted to symbolic execution. We detect data races by keeping track of all memory accesses using a memory log, and reporting a race whenever we detect that two accesses conflict. By representing the memory log symbolically, we are also able to detect races associated with symbolically-indexed accesses of memory objects. We used KLEE-CL to prove the bounded equivalence between scalar and data-parallel versions of floating-point programs and find a number - f issues in a variety of open source projects that use SSE and OpenCL, including mismatches between implementations, memory errors, race conditions and a compiler bug.]]></abstract>
##     <issn><![CDATA[0098-5589]]></issn>
##     <htmlFlag><![CDATA[1]]></htmlFlag>
##     <arnumber><![CDATA[6698391]]></arnumber>
##     <doi><![CDATA[10.1109/TSE.2013.2297120]]></doi>
##     <publicationId><![CDATA[6698391]]></publicationId>
##     <mdurl><![CDATA[http://ieeexplore.ieee.org/xpl/articleDetails.jsp?tp=&arnumber=6698391&contentType=Journals+%26+Magazines]]></mdurl>
##     <pdf><![CDATA[http://ieeexplore.ieee.org/stamp/stamp.jsp?arnumber=6698391]]></pdf>
##   </document>
##   <document>
##     <rank>241</rank>
##     <title><![CDATA[Automatically Generating Specification Properties From Task Models for the Formal Verification of Human&#x2013;Automation Interaction]]></title>
##     <authors><![CDATA[Bolton, M.L.;  Jimenez, N.;  van Paassen, M.M.;  Trujillo, M.]]></authors>
##     <affiliations><![CDATA[Dept. of Ind. & Syst. Eng., State Univ. of New York at Buffalo, Amherst, NY, USA]]></affiliations>
##     <controlledterms>
##       <term><![CDATA[formal specification]]></term>
##       <term><![CDATA[formal verification]]></term>
##       <term><![CDATA[human computer interaction]]></term>
##       <term><![CDATA[task analysis]]></term>
##     </controlledterms>
##     <thesaurusterms>
##       <term><![CDATA[Analytical models]]></term>
##       <term><![CDATA[Automation]]></term>
##       <term><![CDATA[Computational modeling]]></term>
##       <term><![CDATA[Model checking]]></term>
##       <term><![CDATA[Safety]]></term>
##       <term><![CDATA[Training]]></term>
##       <term><![CDATA[Visualization]]></term>
##     </thesaurusterms>
##     <pubtitle><![CDATA[Human-Machine Systems, IEEE Transactions on]]></pubtitle>
##     <punumber><![CDATA[6221037]]></punumber>
##     <pubtype><![CDATA[Journals & Magazines]]></pubtype>
##     <publisher><![CDATA[IEEE]]></publisher>
##     <volume><![CDATA[44]]></volume>
##     <issue><![CDATA[5]]></issue>
##     <py><![CDATA[2014]]></py>
##     <spage><![CDATA[561]]></spage>
##     <epage><![CDATA[575]]></epage>
##     <abstract><![CDATA[Human-automation interaction (HAI) is often a contributor to failures in complex systems. This is frequently due to system interactions that were not anticipated by designers and analysts. Model checking is a method of formal verification analysis that automatically proves whether or not a formal system model adheres to desirable specification properties. Task analytic models can be included in formal system models to allow HAI to be evaluated with model checking. However, previous work in this area has required analysts to manually formulate the properties to check. Such a practice can be prone to analyst error and oversight which can result in unexpected dangerous HAI conditions not being discovered. To address this, this paper presents a method for automatically generating specification properties from task models that enables analysts to use formal verification to check for system HAI problems they may not have anticipated. This paper describes the design and implementation of the method. An example (a pilot performing a before landing checklist) is presented to illustrate its utility. Limitations of this approach and future research directions are discussed.]]></abstract>
##     <issn><![CDATA[2168-2291]]></issn>
##     <htmlFlag><![CDATA[1]]></htmlFlag>
##     <arnumber><![CDATA[6843972]]></arnumber>
##     <doi><![CDATA[10.1109/THMS.2014.2329476]]></doi>
##     <publicationId><![CDATA[6843972]]></publicationId>
##     <mdurl><![CDATA[http://ieeexplore.ieee.org/xpl/articleDetails.jsp?tp=&arnumber=6843972&contentType=Journals+%26+Magazines]]></mdurl>
##     <pdf><![CDATA[http://ieeexplore.ieee.org/stamp/stamp.jsp?arnumber=6843972]]></pdf>
##   </document>
##   <document>
##     <rank>242</rank>
##     <title><![CDATA[Efficient Third-Harmonic Generation From 2 <formula formulatype="inline"> <img src="/images/tex/798.gif" alt="\mu \hbox {m}"> </formula> in Asymmetric Plasmonic Slot Waveguide]]></title>
##     <authors><![CDATA[Tianye Huang;  Xuguang Shao;  Zhifang Wu;  Lee, T.;  Tingting Wu;  Yunxu Sun;  Jing Zhang;  Huy Quoc Lam;  Brambilla, G.;  Shum, P.P.]]></authors>
##     <affiliations><![CDATA[NTU, Singapore, Singapore]]></affiliations>
##     <controlledterms>
##       <term><![CDATA[optical harmonic generation]]></term>
##       <term><![CDATA[optical phase matching]]></term>
##       <term><![CDATA[optical pumping]]></term>
##       <term><![CDATA[optical waveguides]]></term>
##     </controlledterms>
##     <thesaurusterms>
##       <term><![CDATA[Optical fiber devices]]></term>
##       <term><![CDATA[Optical fibers]]></term>
##       <term><![CDATA[Optical frequency conversion]]></term>
##       <term><![CDATA[Optimized production technology]]></term>
##       <term><![CDATA[Photonics]]></term>
##     </thesaurusterms>
##     <pubtitle><![CDATA[Photonics Journal, IEEE]]></pubtitle>
##     <punumber><![CDATA[4563994]]></punumber>
##     <pubtype><![CDATA[Journals & Magazines]]></pubtype>
##     <publisher><![CDATA[IEEE]]></publisher>
##     <volume><![CDATA[6]]></volume>
##     <issue><![CDATA[3]]></issue>
##     <py><![CDATA[2014]]></py>
##     <spage><![CDATA[1]]></spage>
##     <epage><![CDATA[7]]></epage>
##     <abstract><![CDATA[We propose the asymmetrical plasmonic slot waveguide (APSW) design for third-harmonic generation (THG) from 2.25 &#x03BC;m. In this configuration, the phase-matching condition is fulfilled between the zeroth-order mode at fundamental frequency (FF) and the first-order mode at third-harmonic frequency (THF). Due to the asymmetrical geometry, the mode overlap between the two involved modes is significantly enhanced, leading to an efficient THG process. According to the numerical calculation, the conversion efficiency is predicted up to 1.4% with 1-W pump power. The proposed APSW has the potential to realize an integrated efficient THG device in nanometer scale.]]></abstract>
##     <issn><![CDATA[1943-0655]]></issn>
##     <htmlFlag><![CDATA[1]]></htmlFlag>
##     <arnumber><![CDATA[6817551]]></arnumber>
##     <doi><![CDATA[10.1109/JPHOT.2014.2323302]]></doi>
##     <publicationId><![CDATA[6817551]]></publicationId>
##     <mdurl><![CDATA[http://ieeexplore.ieee.org/xpl/articleDetails.jsp?tp=&arnumber=6817551&contentType=Journals+%26+Magazines]]></mdurl>
##     <pdf><![CDATA[http://ieeexplore.ieee.org/stamp/stamp.jsp?arnumber=6817551]]></pdf>
##   </document>
##   <document>
##     <rank>243</rank>
##     <title><![CDATA[Simple, Broadband Material Characterization Using Dual-Ridged Waveguide to Rectangular Waveguide Transitions]]></title>
##     <authors><![CDATA[Hyde, M.W.;  Havrilla, M.J.]]></authors>
##     <affiliations><![CDATA[Dept. of Electr. & Comput. Eng., Air Force Inst. of Technol., Wright-Patterson AFB, OH, USA]]></affiliations>
##     <controlledterms>
##       <term><![CDATA[S-parameters]]></term>
##       <term><![CDATA[magnetic permeability]]></term>
##       <term><![CDATA[permittivity]]></term>
##       <term><![CDATA[ridge waveguides]]></term>
##       <term><![CDATA[waveguide transitions]]></term>
##     </controlledterms>
##     <thesaurusterms>
##       <term><![CDATA[Broadband communication]]></term>
##       <term><![CDATA[Electromagnetic waveguides]]></term>
##       <term><![CDATA[Microwave measurement]]></term>
##       <term><![CDATA[Scattering parameters]]></term>
##       <term><![CDATA[Solids]]></term>
##       <term><![CDATA[Time-domain analysis]]></term>
##     </thesaurusterms>
##     <pubtitle><![CDATA[Electromagnetic Compatibility, IEEE Transactions on]]></pubtitle>
##     <punumber><![CDATA[15]]></punumber>
##     <pubtype><![CDATA[Journals & Magazines]]></pubtype>
##     <publisher><![CDATA[IEEE]]></publisher>
##     <volume><![CDATA[56]]></volume>
##     <issue><![CDATA[1]]></issue>
##     <py><![CDATA[2014]]></py>
##     <spage><![CDATA[239]]></spage>
##     <epage><![CDATA[242]]></epage>
##     <abstract><![CDATA[A simple technique is presented which utilizes dual-ridged waveguide to rectangular waveguide transitions to provide broadband material characterization measurements. Compared to a recently published technique which used dual-ridged waveguides, the proposed method significantly simplifies specimen preparation while maintaining measurement bandwidth. The behavior of the fields in the dual-ridged waveguide to rectangular waveguide transitions is briefly discussed. In addition, a brief discussion on the derivation of the theoretical scattering parameters, required for the extraction of permittivity and permeability of the material under test, is provided. Experimental material characterization results of a magnetic absorbing material are presented and analyzed to validate the proposed technique.]]></abstract>
##     <issn><![CDATA[0018-9375]]></issn>
##     <htmlFlag><![CDATA[1]]></htmlFlag>
##     <arnumber><![CDATA[6576186]]></arnumber>
##     <doi><![CDATA[10.1109/TEMC.2013.2274898]]></doi>
##     <publicationId><![CDATA[6576186]]></publicationId>
##     <mdurl><![CDATA[http://ieeexplore.ieee.org/xpl/articleDetails.jsp?tp=&arnumber=6576186&contentType=Journals+%26+Magazines]]></mdurl>
##     <pdf><![CDATA[http://ieeexplore.ieee.org/stamp/stamp.jsp?arnumber=6576186]]></pdf>
##   </document>
##   <document>
##     <rank>244</rank>
##     <title><![CDATA[High-Speed Visualization of Filament Instabilities and Self-Organization Effect in RF Argon Plasma Jet at Atmospheric Pressure]]></title>
##     <authors><![CDATA[Scha&#x0308; fer, J.;  Sperkaa, J.;  Go&#x0308; tt, G.;  Zaji&#x0301; ckova&#x0301; , L.;  Foest, R.]]></authors>
##     <affiliations><![CDATA[Leibniz Inst. for Plasma Sci. & Technol. e.V., Greifswald, Germany]]></affiliations>
##     <controlledterms>
##       <term><![CDATA[cameras]]></term>
##       <term><![CDATA[filamentation instability]]></term>
##       <term><![CDATA[plasma diagnostics]]></term>
##       <term><![CDATA[plasma jets]]></term>
##       <term><![CDATA[stroboscopes]]></term>
##     </controlledterms>
##     <thesaurusterms>
##       <term><![CDATA[Argon]]></term>
##       <term><![CDATA[Cameras]]></term>
##       <term><![CDATA[Discharges (electric)]]></term>
##       <term><![CDATA[Electron tubes]]></term>
##       <term><![CDATA[Laser mode locking]]></term>
##       <term><![CDATA[Plasmas]]></term>
##       <term><![CDATA[Radio frequency]]></term>
##     </thesaurusterms>
##     <pubtitle><![CDATA[Plasma Science, IEEE Transactions on]]></pubtitle>
##     <punumber><![CDATA[27]]></punumber>
##     <pubtype><![CDATA[Journals & Magazines]]></pubtype>
##     <publisher><![CDATA[IEEE]]></publisher>
##     <volume><![CDATA[42]]></volume>
##     <issue><![CDATA[10]]></issue>
##     <part><![CDATA[1]]></part>
##     <py><![CDATA[2014]]></py>
##     <spage><![CDATA[2454]]></spage>
##     <epage><![CDATA[2455]]></epage>
##     <abstract><![CDATA[An RF argon plasma jet has been explored using high-speed camera imaging at 10000 frames/s. Small variations of gas flow and/or RF power lead to instabilities of the filament movement. Two types of instabilities have been observed depending on the interrelated azimuthal velocities of filaments. In the case of antiparallel filament velocities, one filament is collapsing and fuses with the other filament, while the collapsing filament exhibits a striated structure. In the case of parallel velocities, both filaments establish a symmetric configuration and rotate with constant velocity in the jet. Spatially and temporally resolved features are visualized with a time-colored stroboscopic image.]]></abstract>
##     <issn><![CDATA[0093-3813]]></issn>
##     <htmlFlag><![CDATA[1]]></htmlFlag>
##     <arnumber><![CDATA[6805202]]></arnumber>
##     <doi><![CDATA[10.1109/TPS.2014.2316375]]></doi>
##     <publicationId><![CDATA[6805202]]></publicationId>
##     <mdurl><![CDATA[http://ieeexplore.ieee.org/xpl/articleDetails.jsp?tp=&arnumber=6805202&contentType=Journals+%26+Magazines]]></mdurl>
##     <pdf><![CDATA[http://ieeexplore.ieee.org/stamp/stamp.jsp?arnumber=6805202]]></pdf>
##   </document>
##   <document>
##     <rank>245</rank>
##     <title><![CDATA[A Novel Iterative <formula formulatype="inline"> <img src="/images/tex/16594.gif" alt="\theta "> </formula>-Adaptive Dynamic Programming for Discrete-Time Nonlinear Systems]]></title>
##     <authors><![CDATA[Qinglai Wei;  Derong Liu]]></authors>
##     <affiliations><![CDATA[State Key Lab. of Manage. & Control for Complex Syst., Inst. of Autom., Beijing, China]]></affiliations>
##     <controlledterms>
##       <term><![CDATA[convergence]]></term>
##       <term><![CDATA[discrete time systems]]></term>
##       <term><![CDATA[dynamic programming]]></term>
##       <term><![CDATA[infinite horizon]]></term>
##       <term><![CDATA[iterative methods]]></term>
##       <term><![CDATA[neurocontrollers]]></term>
##       <term><![CDATA[nonlinear control systems]]></term>
##       <term><![CDATA[optimal control]]></term>
##       <term><![CDATA[performance index]]></term>
##       <term><![CDATA[stability]]></term>
##     </controlledterms>
##     <thesaurusterms>
##       <term><![CDATA[Dynamic programming]]></term>
##       <term><![CDATA[Learning (artificial intelligence)]]></term>
##       <term><![CDATA[Neural networks]]></term>
##       <term><![CDATA[Nonlinear systems]]></term>
##       <term><![CDATA[Optimal control]]></term>
##     </thesaurusterms>
##     <pubtitle><![CDATA[Automation Science and Engineering, IEEE Transactions on]]></pubtitle>
##     <punumber><![CDATA[8856]]></punumber>
##     <pubtype><![CDATA[Journals & Magazines]]></pubtype>
##     <publisher><![CDATA[IEEE]]></publisher>
##     <volume><![CDATA[11]]></volume>
##     <issue><![CDATA[4]]></issue>
##     <py><![CDATA[2014]]></py>
##     <spage><![CDATA[1176]]></spage>
##     <epage><![CDATA[1190]]></epage>
##     <abstract><![CDATA[This paper is concerned with a new iterative &#x03B8;-adaptive dynamic programming (ADP) technique to solve optimal control problems of infinite horizon discrete-time nonlinear systems. The idea is to use an iterative ADP algorithm to obtain the iterative control law which optimizes the iterative performance index function. In the present iterative &#x03B8;-ADP algorithm, the condition of initial admissible control in policy iteration algorithm is avoided. It is proved that all the iterative controls obtained in the iterative &#x03B8;-ADP algorithm can stabilize the nonlinear system which means that the iterative &#x03B8;-ADP algorithm is feasible for implementations both online and offline. Convergence analysis of the performance index function is presented to guarantee that the iterative performance index function will converge to the optimum monotonically. Neural networks are used to approximate the performance index function and compute the optimal control policy, respectively, for facilitating the implementation of the iterative &#x03B8;-ADP algorithm. Finally, two simulation examples are given to illustrate the performance of the established method.]]></abstract>
##     <issn><![CDATA[1545-5955]]></issn>
##     <htmlFlag><![CDATA[1]]></htmlFlag>
##     <arnumber><![CDATA[6609148]]></arnumber>
##     <doi><![CDATA[10.1109/TASE.2013.2280974]]></doi>
##     <publicationId><![CDATA[6609148]]></publicationId>
##     <mdurl><![CDATA[http://ieeexplore.ieee.org/xpl/articleDetails.jsp?tp=&arnumber=6609148&contentType=Journals+%26+Magazines]]></mdurl>
##     <pdf><![CDATA[http://ieeexplore.ieee.org/stamp/stamp.jsp?arnumber=6609148]]></pdf>
##   </document>
##   <document>
##     <rank>246</rank>
##     <title><![CDATA[BitTorrent Content Distribution in Optical Networks]]></title>
##     <authors><![CDATA[Lawey, A.Q.;  El-Gorashi, T.E.H.;  Elmirghani, J.M.H.]]></authors>
##     <affiliations><![CDATA[Sch. of Electron. & Electr. Eng., Univ. of Leeds, Leeds, UK]]></affiliations>
##     <controlledterms>
##       <term><![CDATA[IP networks]]></term>
##       <term><![CDATA[client-server systems]]></term>
##       <term><![CDATA[integer programming]]></term>
##       <term><![CDATA[linear programming]]></term>
##       <term><![CDATA[optical fibre networks]]></term>
##       <term><![CDATA[peer-to-peer computing]]></term>
##       <term><![CDATA[protocols]]></term>
##       <term><![CDATA[telecommunication network topology]]></term>
##       <term><![CDATA[telecommunication services]]></term>
##       <term><![CDATA[telecommunication traffic]]></term>
##       <term><![CDATA[video on demand]]></term>
##       <term><![CDATA[wavelength division multiplexing]]></term>
##     </controlledterms>
##     <thesaurusterms>
##       <term><![CDATA[Energy consumption]]></term>
##       <term><![CDATA[Energy efficiency]]></term>
##       <term><![CDATA[IP networks]]></term>
##       <term><![CDATA[Peer-to-peer computing]]></term>
##       <term><![CDATA[Power demand]]></term>
##       <term><![CDATA[Protocols]]></term>
##       <term><![CDATA[WDM networks]]></term>
##     </thesaurusterms>
##     <pubtitle><![CDATA[Lightwave Technology, Journal of]]></pubtitle>
##     <punumber><![CDATA[50]]></punumber>
##     <pubtype><![CDATA[Journals & Magazines]]></pubtype>
##     <publisher><![CDATA[IEEE]]></publisher>
##     <volume><![CDATA[32]]></volume>
##     <issue><![CDATA[21]]></issue>
##     <py><![CDATA[2014]]></py>
##     <spage><![CDATA[4209]]></spage>
##     <epage><![CDATA[4225]]></epage>
##     <abstract><![CDATA[In this paper, we extend our previous study on BitTorrent, the most popular peer-to-peer (P2P) protocol, to investigate different aspects related to its energy efficiency in IP over WDM (IP/WDM) networks, validating the power savings previously obtained by modeling and simulation through experimental results. Our contributions can be summarized as follows: First, we compare the energy consumption of our previously proposed energy efficient BitTorrent protocol to that of the original BitTorrent protocol and the client-server (C-S) schemes over bypass IP/WDM networks considering a range of network topologies with different number of nodes and average hop counts. Our results show that for a certain swarm size, the energy efficient BitTorrent protocol achieves higher power savings in networks with lower number of nodes as the opportunity to localize traffic increases. Second, we extend our previously developed energy efficient BitTorrent heuristic enhancing its performance by allowing peers to progressively traverse more hops in the network if the number of peers in the local node is not sufficient. Third, we extend our previously developed mixed integer linear programming model to optimize the location as well as the upload rates of operator controlled seeders (OCS) to mitigate the performance degradation caused by leechers leaving after finishing the downloading operation. Fourth, we compare the power consumption of video on demand (VoD) services delivered using content distribution networks (CDN), P2P, and a promising hybrid CDN-P2P architecture over bypass IP/WDM core networks. A MILP model is developed to carry out the comparison. We investigate two scenarios for the hybrid CDN-P2P architecture: the H-MinNPC model where the model minimizes the IP/WDM network power consumption and the H-MinTPC model where the model minimizes the total power consumption including the network and the CDN datacenters power consumption. Finally, we carry out an experimental evaluation - f the original and energy efficient BitTorrent heuristics.]]></abstract>
##     <issn><![CDATA[0733-8724]]></issn>
##     <htmlFlag><![CDATA[1]]></htmlFlag>
##     <arnumber><![CDATA[6882178]]></arnumber>
##     <doi><![CDATA[10.1109/JLT.2014.2351074]]></doi>
##     <publicationId><![CDATA[6882178]]></publicationId>
##     <mdurl><![CDATA[http://ieeexplore.ieee.org/xpl/articleDetails.jsp?tp=&arnumber=6882178&contentType=Journals+%26+Magazines]]></mdurl>
##     <pdf><![CDATA[http://ieeexplore.ieee.org/stamp/stamp.jsp?arnumber=6882178]]></pdf>
##   </document>
##   <document>
##     <rank>247</rank>
##     <title><![CDATA[Analysis and Optimization of Hybrid Plasmonic Waveguide as a Platform for Biosensing]]></title>
##     <authors><![CDATA[Alam, M.Z.;  Bahrami, F.;  Aitchison, J.S.;  Mojahedi, M.]]></authors>
##     <affiliations><![CDATA[Dept. of Electr. & Comput. Eng., Univ. of Toronto, Toronto, ON, Canada]]></affiliations>
##     <controlledterms>
##       <term><![CDATA[biosensors]]></term>
##       <term><![CDATA[nonlinear optics]]></term>
##       <term><![CDATA[optimisation]]></term>
##       <term><![CDATA[plasmonics]]></term>
##     </controlledterms>
##     <thesaurusterms>
##       <term><![CDATA[Biosensors]]></term>
##       <term><![CDATA[Indexes]]></term>
##       <term><![CDATA[Plasmons]]></term>
##       <term><![CDATA[Sensitivity]]></term>
##       <term><![CDATA[Silicon]]></term>
##       <term><![CDATA[Surface waves]]></term>
##     </thesaurusterms>
##     <pubtitle><![CDATA[Photonics Journal, IEEE]]></pubtitle>
##     <punumber><![CDATA[4563994]]></punumber>
##     <pubtype><![CDATA[Journals & Magazines]]></pubtype>
##     <publisher><![CDATA[IEEE]]></publisher>
##     <volume><![CDATA[6]]></volume>
##     <issue><![CDATA[4]]></issue>
##     <py><![CDATA[2014]]></py>
##     <spage><![CDATA[1]]></spage>
##     <epage><![CDATA[10]]></epage>
##     <abstract><![CDATA[Hybrid plasmonic waveguides (HPWGs) have received attention worldwide for many different kinds of applications, including on-chip polarization control, enhanced nonlinear optical effects, and biosensing. The HPWG sensor can achieve detection limit lower than possible with purely plasmonic sensors. It can be also used for obtaining additional information about complex biological samples. We analyze the effects of various parameters on such a sensor, optimize the sensor design, and predict the optimum performance achievable for an HPWG sensor in the Mach-Zehnder configuration. We also compare the performance of the HPWG sensor to those of other plasmonic sensors.]]></abstract>
##     <issn><![CDATA[1943-0655]]></issn>
##     <htmlFlag><![CDATA[1]]></htmlFlag>
##     <arnumber><![CDATA[6838972]]></arnumber>
##     <doi><![CDATA[10.1109/JPHOT.2014.2331232]]></doi>
##     <publicationId><![CDATA[6838972]]></publicationId>
##     <mdurl><![CDATA[http://ieeexplore.ieee.org/xpl/articleDetails.jsp?tp=&arnumber=6838972&contentType=Journals+%26+Magazines]]></mdurl>
##     <pdf><![CDATA[http://ieeexplore.ieee.org/stamp/stamp.jsp?arnumber=6838972]]></pdf>
##   </document>
##   <document>
##     <rank>248</rank>
##     <title><![CDATA[Electro-Optic Organic Crystal Silicon High-Speed Modulator]]></title>
##     <authors><![CDATA[Korn, D.;  Jazbinsek, M.;  Palmer, R.;  Baier, M.;  Alloatti, L.;  Yu, H.;  Bogaerts, W.;  Lepage, G.;  Verheyen, P.;  Absil, P.;  Guenter, P.;  Koos, C.;  Freude, W.;  Leuthold, J.]]></authors>
##     <affiliations><![CDATA[Inst. IPQ & IMT, Karlsruhe Inst. of Technol., Karlsruhe, Germany]]></affiliations>
##     <controlledterms>
##       <term><![CDATA[CMOS integrated circuits]]></term>
##       <term><![CDATA[claddings]]></term>
##       <term><![CDATA[electro-optical modulation]]></term>
##       <term><![CDATA[elemental semiconductors]]></term>
##       <term><![CDATA[integrated optics]]></term>
##       <term><![CDATA[optical fabrication]]></term>
##       <term><![CDATA[organic compounds]]></term>
##       <term><![CDATA[polymers]]></term>
##       <term><![CDATA[silicon]]></term>
##       <term><![CDATA[silicon-on-insulator]]></term>
##     </controlledterms>
##     <thesaurusterms>
##       <term><![CDATA[Crystals]]></term>
##       <term><![CDATA[Electrooptic modulators]]></term>
##       <term><![CDATA[Electrooptical waveguides]]></term>
##       <term><![CDATA[Glass]]></term>
##       <term><![CDATA[Silicon]]></term>
##     </thesaurusterms>
##     <pubtitle><![CDATA[Photonics Journal, IEEE]]></pubtitle>
##     <punumber><![CDATA[4563994]]></punumber>
##     <pubtype><![CDATA[Journals & Magazines]]></pubtype>
##     <publisher><![CDATA[IEEE]]></publisher>
##     <volume><![CDATA[6]]></volume>
##     <issue><![CDATA[2]]></issue>
##     <py><![CDATA[2014]]></py>
##     <spage><![CDATA[1]]></spage>
##     <epage><![CDATA[9]]></epage>
##     <abstract><![CDATA[Silicon waveguides can be functionalized with an organic &#x03C7;<sup>(2)</sup>-nonlinear cladding. This complements silicon photonics with the electro-optic (EO) effect originating from the cladding and enables functionalities such as pure phase modulation, parametric amplification, or THz-wave generation. Claddings based on a polymer matrix containing chromophores have been introduced, and their strong &#x03C7;<sup>(2)</sup> nonlinearity has already been used to demonstrate ultralow power consuming modulators. However, these silicon-organic hybrid (SOH) devices inherit not only the advantageous properties; these polymer claddings require an alignment procedure called poling and must be operated well below their glass transition temperature. This excludes some applications. In contrast, claddings made from organic crystals come with a different set of properties. In particular, there is no need for poling. This new class of claddings also promises stronger resilience to high temperatures, better long-term stability, and photo-chemical stability. We report on the deposition of an organic crystal cladding of N-benzyl-2-methyl-4-nitroaniline (BNA) on silicon-on-insulator (SOI) waveguides, which have a CMOS-like metal stack on top. Adhering to such an architecture, which preserves the principal advantage of using CMOS-based silicon photonic fabrication processes, permits the first demonstration of high-speed modulation at 12.5 Gbit/s in this material class, which proves the availability of the EO effect from BNA on SOI also for other applications.]]></abstract>
##     <issn><![CDATA[1943-0655]]></issn>
##     <htmlFlag><![CDATA[1]]></htmlFlag>
##     <arnumber><![CDATA[6783776]]></arnumber>
##     <doi><![CDATA[10.1109/JPHOT.2014.2314113]]></doi>
##     <publicationId><![CDATA[6783776]]></publicationId>
##     <mdurl><![CDATA[http://ieeexplore.ieee.org/xpl/articleDetails.jsp?tp=&arnumber=6783776&contentType=Journals+%26+Magazines]]></mdurl>
##     <pdf><![CDATA[http://ieeexplore.ieee.org/stamp/stamp.jsp?arnumber=6783776]]></pdf>
##   </document>
##   <document>
##     <rank>249</rank>
##     <title><![CDATA[CAO-SIR: Channel Aware Ordered Successive Relaying]]></title>
##     <authors><![CDATA[Wei Chen]]></authors>
##     <affiliations><![CDATA[Dept. of Electron. Eng., Tsinghua Univ., Beijing, China]]></affiliations>
##     <controlledterms>
##       <term><![CDATA[MIMO communication]]></term>
##       <term><![CDATA[computational complexity]]></term>
##       <term><![CDATA[cooperative communication]]></term>
##       <term><![CDATA[decision feedback equalisers]]></term>
##       <term><![CDATA[decoding]]></term>
##       <term><![CDATA[diversity reception]]></term>
##       <term><![CDATA[interference suppression]]></term>
##       <term><![CDATA[protocols]]></term>
##       <term><![CDATA[radio links]]></term>
##       <term><![CDATA[relay networks (telecommunication)]]></term>
##       <term><![CDATA[wireless channels]]></term>
##     </controlledterms>
##     <thesaurusterms>
##       <term><![CDATA[Cooperative communication]]></term>
##       <term><![CDATA[Decoding]]></term>
##       <term><![CDATA[Interference cancellation]]></term>
##       <term><![CDATA[Power distribution]]></term>
##       <term><![CDATA[Relays]]></term>
##       <term><![CDATA[Wireless communication]]></term>
##     </thesaurusterms>
##     <pubtitle><![CDATA[Wireless Communications, IEEE Transactions on]]></pubtitle>
##     <punumber><![CDATA[7693]]></punumber>
##     <pubtype><![CDATA[Journals & Magazines]]></pubtype>
##     <publisher><![CDATA[IEEE]]></publisher>
##     <volume><![CDATA[13]]></volume>
##     <issue><![CDATA[12]]></issue>
##     <py><![CDATA[2014]]></py>
##     <spage><![CDATA[6513]]></spage>
##     <epage><![CDATA[6527]]></epage>
##     <abstract><![CDATA[Cooperative communication suffers from multiplexing loss and low spectral efficiency due to the half duplex constraint of relays. To improve the multiplexing gain, successive relaying, which allows concurrent transmission of the source and relays, has been proposed. However, the severe inter-relay interference becomes a key challenge. In this paper, we propose a channel aware successive relaying protocol, also referred to as CAO-SIR, which is capable of thoroughly mitigating inter-relay interference by carefully adapting relays' transmission order and rate. In particular, a relay having a poorer link to the source is scheduled first to forward a message, the data rate of which is adapted to the link quality of the source-relay and relay-destination channels. By this means, each relay may decode the messages intended for the preceding relays, and then cancel these relays' interference in a low complexity which is equal to that of Decision Feedback Equalizer (DFE). To further optimize and analyze CAO-SIR, we present its equivalent parallel relay channel model, based upon which the adaptive relay selection and power allocation schemes are proposed. By employing M half duplex relays, CAO-SIR is capable of achieving an diversity-multiplexing tradeoff (DMT) given by d(r) = max (M + 1) 1- {(M+2/M+1r)} , (1 - r) , where d(r) and r denote the diversity and multiplexing gains, respectively. Its DMT asymptotically approaches the DMT upper bound achieved by (M + 1) &#x00D7; 1 MISO systems or M full duplex relays, when M is large.]]></abstract>
##     <issn><![CDATA[1536-1276]]></issn>
##     <htmlFlag><![CDATA[1]]></htmlFlag>
##     <arnumber><![CDATA[6926805]]></arnumber>
##     <doi><![CDATA[10.1109/TWC.2014.2363453]]></doi>
##     <publicationId><![CDATA[6926805]]></publicationId>
##     <mdurl><![CDATA[http://ieeexplore.ieee.org/xpl/articleDetails.jsp?tp=&arnumber=6926805&contentType=Journals+%26+Magazines]]></mdurl>
##     <pdf><![CDATA[http://ieeexplore.ieee.org/stamp/stamp.jsp?arnumber=6926805]]></pdf>
##   </document>
##   <document>
##     <rank>250</rank>
##     <title><![CDATA[Periodic Nanopillar N-I-P Amorphous Si Photovoltaic Cells Using Carbon Nanotube Scaffolds]]></title>
##     <authors><![CDATA[Hang Zhou;  Fei Tao;  Hiralal, P.;  Ahnood, A.;  Unalan, H.E.;  Nathan, A.;  Amaratunga, G.A.J.]]></authors>
##     <affiliations><![CDATA[Shenzhen Grad. Sch., Sch. of Electron. & Comput. Eng., Peking Univ., Shenzhen, China]]></affiliations>
##     <controlledterms>
##       <term><![CDATA[amorphous semiconductors]]></term>
##       <term><![CDATA[carbon nanotubes]]></term>
##       <term><![CDATA[elemental semiconductors]]></term>
##       <term><![CDATA[nanofabrication]]></term>
##       <term><![CDATA[photoconductivity]]></term>
##       <term><![CDATA[reflectivity]]></term>
##       <term><![CDATA[semiconductor thin films]]></term>
##       <term><![CDATA[silicon]]></term>
##       <term><![CDATA[solar cells]]></term>
##     </controlledterms>
##     <thesaurusterms>
##       <term><![CDATA[Arrays]]></term>
##       <term><![CDATA[Nanoscale devices]]></term>
##       <term><![CDATA[Photovoltaic cells]]></term>
##       <term><![CDATA[Photovoltaic systems]]></term>
##       <term><![CDATA[Reflectivity]]></term>
##       <term><![CDATA[Silicon]]></term>
##     </thesaurusterms>
##     <pubtitle><![CDATA[Nanotechnology, IEEE Transactions on]]></pubtitle>
##     <punumber><![CDATA[7729]]></punumber>
##     <pubtype><![CDATA[Journals & Magazines]]></pubtype>
##     <publisher><![CDATA[IEEE]]></publisher>
##     <volume><![CDATA[13]]></volume>
##     <issue><![CDATA[5]]></issue>
##     <py><![CDATA[2014]]></py>
##     <spage><![CDATA[997]]></spage>
##     <epage><![CDATA[1004]]></epage>
##     <abstract><![CDATA[Arrays of periodic one-dimensional nanomaterials offer tunable optical properties in terms of light-matter interaction which are attractive for designing efficient optoelectronic devices. This paper presents a fabrication of bottom-up grown nanopillar (NP) array solar cells based on n-i-p thin-film amorphous silicon using scaffolds of vertically aligned carbon nanotube (CNT) array. The effects of varying the CNT spacing over the range from 800 to 2000 nm on optical and electrical properties of the solar cells were investigated. The NP solar cell with CNT spacing of 800 nm exhibited `moth-eye' broadband antireflection behavior, showing an average reflectance value lower than 10%. The enhanced optical absorption translated to significant enhancements in photocurrent and quantum efficiency compared to a conventional planar solar cell under low light condition. The open-circuit voltage (V<sub>oc</sub>) of the NP solar cell was found systematically correlated with the CNT spacing and the illumination condition. The results presented here is of importance for developing high efficiency one-dimensional nanostructured solar cells.]]></abstract>
##     <issn><![CDATA[1536-125X]]></issn>
##     <htmlFlag><![CDATA[1]]></htmlFlag>
##     <arnumber><![CDATA[6866173]]></arnumber>
##     <doi><![CDATA[10.1109/TNANO.2014.2343256]]></doi>
##     <publicationId><![CDATA[6866173]]></publicationId>
##     <mdurl><![CDATA[http://ieeexplore.ieee.org/xpl/articleDetails.jsp?tp=&arnumber=6866173&contentType=Journals+%26+Magazines]]></mdurl>
##     <pdf><![CDATA[http://ieeexplore.ieee.org/stamp/stamp.jsp?arnumber=6866173]]></pdf>
##   </document>
##   <document>
##     <rank>251</rank>
##     <title><![CDATA[Commuter Route Optimized Energy Management of Hybrid Electric Vehicles]]></title>
##     <authors><![CDATA[Larsson, V.;  Johannesson M&#x00E5; rdh, L.;  Egardt, B.;  Karlsson, S.]]></authors>
##     <affiliations><![CDATA[Dept. of Signals & Syst., Chalmers Univ. of Technol., Goteborg, Sweden]]></affiliations>
##     <controlledterms>
##       <term><![CDATA[dynamic programming]]></term>
##       <term><![CDATA[hybrid electric vehicles]]></term>
##       <term><![CDATA[pattern clustering]]></term>
##       <term><![CDATA[table lookup]]></term>
##       <term><![CDATA[traffic engineering computing]]></term>
##     </controlledterms>
##     <thesaurusterms>
##       <term><![CDATA[Batteries]]></term>
##       <term><![CDATA[Computational modeling]]></term>
##       <term><![CDATA[Energy management]]></term>
##       <term><![CDATA[Engines]]></term>
##       <term><![CDATA[Mathematical model]]></term>
##       <term><![CDATA[Torque]]></term>
##       <term><![CDATA[Vehicles]]></term>
##     </thesaurusterms>
##     <pubtitle><![CDATA[Intelligent Transportation Systems, IEEE Transactions on]]></pubtitle>
##     <punumber><![CDATA[6979]]></punumber>
##     <pubtype><![CDATA[Journals & Magazines]]></pubtype>
##     <publisher><![CDATA[IEEE]]></publisher>
##     <volume><![CDATA[15]]></volume>
##     <issue><![CDATA[3]]></issue>
##     <py><![CDATA[2014]]></py>
##     <spage><![CDATA[1145]]></spage>
##     <epage><![CDATA[1154]]></epage>
##     <abstract><![CDATA[Optimal energy management of hybrid electric vehicles requires a priori information regarding future driving conditions; the acquisition and processing of this information is nevertheless often neglected in academic research. This paper introduces a commuter route optimized energy management system, where the bulk of the computations are performed on a server. The idea is to identify commuter routes from historical driving data, using hierarchical agglomerative clustering, and then precompute an optimal solution to the energy management control problem with dynamic programming; the obtained solution can then be transmitted to the vehicle in the form of a lookup table. To investigate the potential of such a system, a simulation study is performed using a detailed vehicle model implemented in the Autonomie simulation environment for MATLAB/Simulink. The simulation results for a plug-in hybrid electric vehicle indicate that the average fuel consumption along the commuter route(s) can be reduced by 4%-9% and battery usage by 10%-15%.]]></abstract>
##     <issn><![CDATA[1524-9050]]></issn>
##     <htmlFlag><![CDATA[1]]></htmlFlag>
##     <arnumber><![CDATA[6719539]]></arnumber>
##     <doi><![CDATA[10.1109/TITS.2013.2294723]]></doi>
##     <publicationId><![CDATA[6719539]]></publicationId>
##     <mdurl><![CDATA[http://ieeexplore.ieee.org/xpl/articleDetails.jsp?tp=&arnumber=6719539&contentType=Journals+%26+Magazines]]></mdurl>
##     <pdf><![CDATA[http://ieeexplore.ieee.org/stamp/stamp.jsp?arnumber=6719539]]></pdf>
##   </document>
##   <document>
##     <rank>252</rank>
##     <title><![CDATA[Design, Fabrication, and Experimental Validation of Novel Flexible Silicon-Based Dry Sensors for Electroencephalography Signal Measurements]]></title>
##     <authors><![CDATA[Yi-Hsin Yu;  Shao-Wei Lu;  Lun-De Liao;  Chin-Teng Lin]]></authors>
##     <affiliations><![CDATA[Brain Res. Center, Nat. Chiao Tung Univ., Hsinchu, Taiwan]]></affiliations>
##     <controlledterms>
##       <term><![CDATA[cellular biophysics]]></term>
##       <term><![CDATA[electric sensing devices]]></term>
##       <term><![CDATA[electroencephalography]]></term>
##       <term><![CDATA[elemental semiconductors]]></term>
##       <term><![CDATA[silicon]]></term>
##       <term><![CDATA[skin]]></term>
##       <term><![CDATA[toxicology]]></term>
##     </controlledterms>
##     <thesaurusterms>
##       <term><![CDATA[Electrodes]]></term>
##       <term><![CDATA[Electroencephalography]]></term>
##       <term><![CDATA[Impedance]]></term>
##       <term><![CDATA[Impedance measurement]]></term>
##       <term><![CDATA[Materials]]></term>
##       <term><![CDATA[Sensors]]></term>
##       <term><![CDATA[Skin]]></term>
##     </thesaurusterms>
##     <pubtitle><![CDATA[Translational Engineering in Health and Medicine, IEEE Journal of]]></pubtitle>
##     <punumber><![CDATA[6221039]]></punumber>
##     <pubtype><![CDATA[Journals & Magazines]]></pubtype>
##     <publisher><![CDATA[IEEE]]></publisher>
##     <volume><![CDATA[2]]></volume>
##     <py><![CDATA[2014]]></py>
##     <spage><![CDATA[1]]></spage>
##     <epage><![CDATA[7]]></epage>
##     <abstract><![CDATA[Many commercially available electroencephalography (EEG) sensors, including conventional wet and dry sensors, can cause skin irritation and user discomfort owing to the foreign material. The EEG products, especially sensors, highly prioritize the comfort level during devices wear. To overcome these drawbacks for EEG sensors, this paper designs Societe Generale de Surveillance S &#x00B7; A &#x00B7; (SGS)-certified, silicon-based dry-contact EEG sensors (SBDSs) for EEG signal measurements. According to the SGS testing report, SBDSs extract does not irritate skin or induce noncytotoxic effects on L929 cells according to ISO10993-5. The SBDS is also lightweight, flexible, and nonirritating to the skin, as well as capable of easily fitting to scalps without any skin preparation or use of a conductive gel. For forehead and hairy sites, EEG signals can be measured reliably with the designed SBDSs. In particular, for EEG signal measurements at hairy sites, the acicular and flexible design of SBDS can push the hair aside to achieve satisfactory scalp contact, as well as maintain low skin-electrode interface impedance. Results of this paper demonstrate that the proposed sensors perform well in the EEG measurements and are feasible for practical applications.]]></abstract>
##     <issn><![CDATA[2168-2372]]></issn>
##     <htmlFlag><![CDATA[1]]></htmlFlag>
##     <arnumber><![CDATA[6949076]]></arnumber>
##     <doi><![CDATA[10.1109/JTEHM.2014.2367518]]></doi>
##     <publicationId><![CDATA[6949076]]></publicationId>
##     <mdurl><![CDATA[http://ieeexplore.ieee.org/xpl/articleDetails.jsp?tp=&arnumber=6949076&contentType=Journals+%26+Magazines]]></mdurl>
##     <pdf><![CDATA[http://ieeexplore.ieee.org/stamp/stamp.jsp?arnumber=6949076]]></pdf>
##   </document>
##   <document>
##     <rank>253</rank>
##     <title><![CDATA[A Food Recognition System for Diabetic Patients Based on an Optimized Bag-of-Features Model]]></title>
##     <authors><![CDATA[Anthimopoulos, M.M.;  Gianola, L.;  Scarnato, L.;  Diem, P.;  Mougiakakou, S.G.]]></authors>
##     <affiliations><![CDATA[ARTORG Center for Biomed. Eng. Res., Univ. of Bern, Bern, Switzerland]]></affiliations>
##     <controlledterms>
##       <term><![CDATA[computer vision]]></term>
##       <term><![CDATA[feature extraction]]></term>
##       <term><![CDATA[image classification]]></term>
##       <term><![CDATA[medical image processing]]></term>
##       <term><![CDATA[pattern clustering]]></term>
##       <term><![CDATA[support vector machines]]></term>
##       <term><![CDATA[transforms]]></term>
##     </controlledterms>
##     <thesaurusterms>
##       <term><![CDATA[Dictionaries]]></term>
##       <term><![CDATA[Feature extraction]]></term>
##       <term><![CDATA[Histograms]]></term>
##       <term><![CDATA[Image color analysis]]></term>
##       <term><![CDATA[Support vector machines]]></term>
##       <term><![CDATA[Training]]></term>
##       <term><![CDATA[Visualization]]></term>
##     </thesaurusterms>
##     <pubtitle><![CDATA[Biomedical and Health Informatics, IEEE Journal of]]></pubtitle>
##     <punumber><![CDATA[6221020]]></punumber>
##     <pubtype><![CDATA[Journals & Magazines]]></pubtype>
##     <publisher><![CDATA[IEEE]]></publisher>
##     <volume><![CDATA[18]]></volume>
##     <issue><![CDATA[4]]></issue>
##     <py><![CDATA[2014]]></py>
##     <spage><![CDATA[1261]]></spage>
##     <epage><![CDATA[1271]]></epage>
##     <abstract><![CDATA[Computer vision-based food recognition could be used to estimate a meal's carbohydrate content for diabetic patients. This study proposes a methodology for automatic food recognition, based on the bag-of-features (BoF) model. An extensive technical investigation was conducted for the identification and optimization of the best performing components involved in the BoF architecture, as well as the estimation of the corresponding parameters. For the design and evaluation of the prototype system, a visual dataset with nearly 5000 food images was created and organized into 11 classes. The optimized system computes dense local features, using the scale-invariant feature transform on the HSV color space, builds a visual dictionary of 10000 visual words by using the hierarchical k-means clustering and finally classifies the food images with a linear support vector machine classifier. The system achieved classification accuracy of the order of 78%, thus proving the feasibility of the proposed approach in a very challenging image dataset.]]></abstract>
##     <issn><![CDATA[2168-2194]]></issn>
##     <arnumber><![CDATA[6762879]]></arnumber>
##     <doi><![CDATA[10.1109/JBHI.2014.2308928]]></doi>
##     <publicationId><![CDATA[6762879]]></publicationId>
##     <mdurl><![CDATA[http://ieeexplore.ieee.org/xpl/articleDetails.jsp?tp=&arnumber=6762879&contentType=Journals+%26+Magazines]]></mdurl>
##     <pdf><![CDATA[http://ieeexplore.ieee.org/stamp/stamp.jsp?arnumber=6762879]]></pdf>
##   </document>
##   <document>
##     <rank>254</rank>
##     <title><![CDATA[Consistency of Measurements of Wavelength Position From Hyperspectral Imagery: Use of the Ferric Iron Crystal Field Absorption at <inline-formula> <img src="/images/tex/559.gif" alt="\sim"> </inline-formula>900 nm as an Indicator of Mineralogy]]></title>
##     <authors><![CDATA[Murphy, R.J.;  Schneider, S.;  Monteiro, S.T.]]></authors>
##     <affiliations><![CDATA[Dept. of Aerosp., Univ. of Sydney, Sydney, NSW, Australia]]></affiliations>
##     <controlledterms>
##       <term><![CDATA[geophysical image processing]]></term>
##       <term><![CDATA[hyperspectral imaging]]></term>
##       <term><![CDATA[minerals]]></term>
##       <term><![CDATA[position measurement]]></term>
##       <term><![CDATA[rocks]]></term>
##     </controlledterms>
##     <pubtitle><![CDATA[Geoscience and Remote Sensing, IEEE Transactions on]]></pubtitle>
##     <punumber><![CDATA[36]]></punumber>
##     <pubtype><![CDATA[Journals & Magazines]]></pubtype>
##     <publisher><![CDATA[IEEE]]></publisher>
##     <volume><![CDATA[52]]></volume>
##     <issue><![CDATA[5]]></issue>
##     <py><![CDATA[2014]]></py>
##     <spage><![CDATA[2843]]></spage>
##     <epage><![CDATA[2857]]></epage>
##     <abstract><![CDATA[Several environmental and sensor effects make the determination of the wavelength position of absorption features in the visible near infrared (VNIR) (400-1200 nm) from hyperspectral imagery more difficult than from nonimaging spectrometers. To evaluate this, we focus on the ferric iron crystal field absorption, located at about 900 nm (F900), because it is impacted by both environmental and sensor effects. The consistency with which the wavelength position of F900 can be determined from imagery acquired in laboratory and field settings is evaluated under artificial and natural illumination, respectively. The wavelength position of F900, determined from laboratory imagery, is also evaluated as an indicator of the proportion of goethite in mixtures of crushed rock. Results are compared with those from a high-resolution field spectrometer. Images describing the wavelength position of F900 showed large amounts of spatial variability and contained an artifact-a consistent shift in the wavelength position of F900 to longer wavelengths. These effects were greatly reduced or removed when wavelength position was determined from a polynomial fit to the data, enabling wavelength position to be used to map hematite and goethite in samples of ore and on a vertical surface (a mine face). The wavelength position of F900 from a polynomial fit was strongly positively correlated with the proportion of goethite (R<sup>2</sup>=0.97). Taken together, these findings indicate that the wavelength position of absorption features from VNIR imagery should be determined from a polynomial (or equivalent) fit to the original data and not from the original data themselves.]]></abstract>
##     <issn><![CDATA[0196-2892]]></issn>
##     <htmlFlag><![CDATA[1]]></htmlFlag>
##     <arnumber><![CDATA[6553103]]></arnumber>
##     <doi><![CDATA[10.1109/TGRS.2013.2266672]]></doi>
##     <publicationId><![CDATA[6553103]]></publicationId>
##     <mdurl><![CDATA[http://ieeexplore.ieee.org/xpl/articleDetails.jsp?tp=&arnumber=6553103&contentType=Journals+%26+Magazines]]></mdurl>
##     <pdf><![CDATA[http://ieeexplore.ieee.org/stamp/stamp.jsp?arnumber=6553103]]></pdf>
##   </document>
##   <document>
##     <rank>255</rank>
##     <title><![CDATA[Guided-Mode Resonant <formula formulatype="inline"> <img src="/images/tex/594.gif" alt="\hbox {HfO}_{2}"> </formula> Grating at Visible Wavelength Range]]></title>
##     <authors><![CDATA[Yongjin Wang;  Xumin Gao;  Zheng Shi;  Lifeng Chen;  Lopez Garcia, M.;  Hueting, N.A.;  Cryan, M.;  Xin Li;  Miao Zhang;  Hongbo Zhu]]></authors>
##     <affiliations><![CDATA[Gruenberg Res. Centre, Nanjing Univ. of Posts & Telecommun., Nanjing, China]]></affiliations>
##     <controlledterms>
##       <term><![CDATA[diffraction gratings]]></term>
##       <term><![CDATA[hafnium compounds]]></term>
##       <term><![CDATA[light polarisation]]></term>
##       <term><![CDATA[membranes]]></term>
##       <term><![CDATA[reflectivity]]></term>
##     </controlledterms>
##     <thesaurusterms>
##       <term><![CDATA[Filling]]></term>
##       <term><![CDATA[Gratings]]></term>
##       <term><![CDATA[Hafnium compounds]]></term>
##       <term><![CDATA[Reflectivity]]></term>
##       <term><![CDATA[Refractive index]]></term>
##       <term><![CDATA[Silicon]]></term>
##     </thesaurusterms>
##     <pubtitle><![CDATA[Photonics Journal, IEEE]]></pubtitle>
##     <punumber><![CDATA[4563994]]></punumber>
##     <pubtype><![CDATA[Journals & Magazines]]></pubtype>
##     <publisher><![CDATA[IEEE]]></publisher>
##     <volume><![CDATA[6]]></volume>
##     <issue><![CDATA[2]]></issue>
##     <py><![CDATA[2014]]></py>
##     <spage><![CDATA[1]]></spage>
##     <epage><![CDATA[7]]></epage>
##     <abstract><![CDATA[Subwavelength HfO<sub>2</sub> gratings are realized on a freestanding 200-nm-thick HfO<sub>2</sub> membrane with air as the low refractive index materials on top and bottom. Strong coupling between the incident light and HfO<sub>2</sub> grating is characterized by angular-resolved reflectivity measurement, and guided-mode resonances are experimentally demonstrated with the sensitivities to the parameters and shapes of grating and the polarization of incident beam. The experimental results are consistent with numerical simulation. This work opens the way to fabricate guided-mode resonant HfO<sub>2</sub> photonic devices in the visible wavelength range.]]></abstract>
##     <issn><![CDATA[1943-0655]]></issn>
##     <htmlFlag><![CDATA[1]]></htmlFlag>
##     <arnumber><![CDATA[6758364]]></arnumber>
##     <doi><![CDATA[10.1109/JPHOT.2014.2309641]]></doi>
##     <publicationId><![CDATA[6758364]]></publicationId>
##     <mdurl><![CDATA[http://ieeexplore.ieee.org/xpl/articleDetails.jsp?tp=&arnumber=6758364&contentType=Journals+%26+Magazines]]></mdurl>
##     <pdf><![CDATA[http://ieeexplore.ieee.org/stamp/stamp.jsp?arnumber=6758364]]></pdf>
##   </document>
##   <document>
##     <rank>256</rank>
##     <title><![CDATA[Influence of Collimation on Alignment Accuracy in Proximity Lithography]]></title>
##     <authors><![CDATA[Nan Wang;  Wei Jiang;  Jiangping Zhu;  Yan Tang;  Wei Yan;  Junmin Tong;  Song Hu]]></authors>
##     <affiliations><![CDATA[State Key Lab. of Opt. Technol. for Microfabrication, Inst. of Opt. & Electron., Chengdu, China]]></affiliations>
##     <controlledterms>
##       <term><![CDATA[displacement measurement]]></term>
##       <term><![CDATA[error correction]]></term>
##       <term><![CDATA[masks]]></term>
##       <term><![CDATA[moire fringes]]></term>
##       <term><![CDATA[optical collimators]]></term>
##       <term><![CDATA[photolithography]]></term>
##     </controlledterms>
##     <thesaurusterms>
##       <term><![CDATA[Accuracy]]></term>
##       <term><![CDATA[Adaptive optics]]></term>
##       <term><![CDATA[Convergence]]></term>
##       <term><![CDATA[Gratings]]></term>
##       <term><![CDATA[Lithography]]></term>
##       <term><![CDATA[Optical device fabrication]]></term>
##       <term><![CDATA[Optical imaging]]></term>
##     </thesaurusterms>
##     <pubtitle><![CDATA[Photonics Journal, IEEE]]></pubtitle>
##     <punumber><![CDATA[4563994]]></punumber>
##     <pubtype><![CDATA[Journals & Magazines]]></pubtype>
##     <publisher><![CDATA[IEEE]]></publisher>
##     <volume><![CDATA[6]]></volume>
##     <issue><![CDATA[4]]></issue>
##     <py><![CDATA[2014]]></py>
##     <spage><![CDATA[1]]></spage>
##     <epage><![CDATA[10]]></epage>
##     <abstract><![CDATA[The alignment method based on moire&#x0301; imaging has been wildly used for its high accuracy of physical measurement, which utilized the phase of moire&#x0301; fringe to measure the relative linear displacement between the mask and the wafer. However, this method is only theoretically accurate due to the affection of certain parameters, such as the optical beam collimation. In this paper, the influence of collimation on the alignment accuracy is thoroughly analyzed. The theoretical analyses and simulation results indicate that the alignment accuracy, which was observed just behind the test grating, is sensitive to the divergence or convergence angle of incident light. On this basis, the method for the error correction is proposed and confirmed.]]></abstract>
##     <issn><![CDATA[1943-0655]]></issn>
##     <htmlFlag><![CDATA[1]]></htmlFlag>
##     <arnumber><![CDATA[6873225]]></arnumber>
##     <doi><![CDATA[10.1109/JPHOT.2014.2345878]]></doi>
##     <publicationId><![CDATA[6873225]]></publicationId>
##     <mdurl><![CDATA[http://ieeexplore.ieee.org/xpl/articleDetails.jsp?tp=&arnumber=6873225&contentType=Journals+%26+Magazines]]></mdurl>
##     <pdf><![CDATA[http://ieeexplore.ieee.org/stamp/stamp.jsp?arnumber=6873225]]></pdf>
##   </document>
##   <document>
##     <rank>257</rank>
##     <title><![CDATA[Cofabrication of Planar Gunn Diode and HEMT on InP Substrate]]></title>
##     <authors><![CDATA[Papageorgiou, V.;  Khalid, A.;  Chong Li;  Cumming, D.R.S.]]></authors>
##     <affiliations><![CDATA[Sch. of Eng., Univ. of Glasgow, Glasgow, UK]]></affiliations>
##     <controlledterms>
##       <term><![CDATA[Gunn diodes]]></term>
##       <term><![CDATA[III-V semiconductors]]></term>
##       <term><![CDATA[electron beam lithography]]></term>
##       <term><![CDATA[high electron mobility transistors]]></term>
##       <term><![CDATA[indium compounds]]></term>
##       <term><![CDATA[millimetre wave diodes]]></term>
##       <term><![CDATA[millimetre wave transistors]]></term>
##     </controlledterms>
##     <thesaurusterms>
##       <term><![CDATA[Fabrication]]></term>
##       <term><![CDATA[HEMTs]]></term>
##       <term><![CDATA[Indium phosphide]]></term>
##       <term><![CDATA[Light emitting diodes]]></term>
##       <term><![CDATA[Oscillators]]></term>
##       <term><![CDATA[Semiconductor diodes]]></term>
##       <term><![CDATA[Substrates]]></term>
##     </thesaurusterms>
##     <pubtitle><![CDATA[Electron Devices, IEEE Transactions on]]></pubtitle>
##     <punumber><![CDATA[16]]></punumber>
##     <pubtype><![CDATA[Journals & Magazines]]></pubtype>
##     <publisher><![CDATA[IEEE]]></publisher>
##     <volume><![CDATA[61]]></volume>
##     <issue><![CDATA[8]]></issue>
##     <py><![CDATA[2014]]></py>
##     <spage><![CDATA[2779]]></spage>
##     <epage><![CDATA[2784]]></epage>
##     <abstract><![CDATA[We present the cofabrication of planar Gunn diodes and high-electron mobility transistors (HEMTs) on an indium phosphide substrate for the first time. Electron beam lithography has been used extensively for the complete fabrication procedure and a 70-nm T-gate technology was incorporated for the enhancement of the small-signal characteristics of the HEMT. Diodes with anode-to-cathode separation (L<sub>ac</sub>) down to 1and 120-&#x03BC;m width were shown to oscillate up to 204 GHz. The transistor presents a cutoff frequency ( f<sub>T</sub> ) of 220 GHz, with power gain up to 330 GHz ( f<sub>max</sub>). The integration of the two devices creates the potential for the realization of high-power, high-frequency MMIC Gunn oscillators, circuits, and systems.]]></abstract>
##     <issn><![CDATA[0018-9383]]></issn>
##     <htmlFlag><![CDATA[1]]></htmlFlag>
##     <arnumber><![CDATA[6847171]]></arnumber>
##     <doi><![CDATA[10.1109/TED.2014.2331368]]></doi>
##     <publicationId><![CDATA[6847171]]></publicationId>
##     <mdurl><![CDATA[http://ieeexplore.ieee.org/xpl/articleDetails.jsp?tp=&arnumber=6847171&contentType=Journals+%26+Magazines]]></mdurl>
##     <pdf><![CDATA[http://ieeexplore.ieee.org/stamp/stamp.jsp?arnumber=6847171]]></pdf>
##   </document>
##   <document>
##     <rank>258</rank>
##     <title><![CDATA[Coupling Graphs, Efficient Algorithms and B-Cell Epitope Prediction]]></title>
##     <authors><![CDATA[Liang Zhao;  Hoi, S.C.H.;  Zhenhua Li;  Limsoon Wong;  Hung Nguyen;  Jinyan Li]]></authors>
##     <affiliations><![CDATA[Dept. of Pediatrics, Baylor Coll. of Med., Houston, TX, USA]]></affiliations>
##     <controlledterms>
##       <term><![CDATA[bioinformatics]]></term>
##       <term><![CDATA[cellular biophysics]]></term>
##       <term><![CDATA[data mining]]></term>
##       <term><![CDATA[graph theory]]></term>
##       <term><![CDATA[molecular biophysics]]></term>
##       <term><![CDATA[proteins]]></term>
##     </controlledterms>
##     <thesaurusterms>
##       <term><![CDATA[Bioinformatics]]></term>
##       <term><![CDATA[Computational biology]]></term>
##       <term><![CDATA[Couplings]]></term>
##       <term><![CDATA[Databases]]></term>
##       <term><![CDATA[Fuel processing industries]]></term>
##       <term><![CDATA[Prediction algorithms]]></term>
##       <term><![CDATA[Proteins]]></term>
##     </thesaurusterms>
##     <pubtitle><![CDATA[Computational Biology and Bioinformatics, IEEE/ACM Transactions on]]></pubtitle>
##     <punumber><![CDATA[8857]]></punumber>
##     <pubtype><![CDATA[Journals & Magazines]]></pubtype>
##     <publisher><![CDATA[IEEE]]></publisher>
##     <volume><![CDATA[11]]></volume>
##     <issue><![CDATA[1]]></issue>
##     <py><![CDATA[2014]]></py>
##     <spage><![CDATA[7]]></spage>
##     <epage><![CDATA[16]]></epage>
##     <abstract><![CDATA[Coupling graphs are newly introduced in this paper to meet many application needs particularly in the field of bioinformatics. A coupling graph is a two-layer graph complex, in which each node from one layer of the graph complex has at least one connection with the nodes in the other layer, and vice versa. The coupling graph model is sufficiently powerful to capture strong and inherent associations between subgraph pairs in complicated applications. The focus of this paper is on mining algorithms of frequent coupling subgraphs and bioinformatics application. Although existing frequent subgraph mining algorithms are competent to identify frequent subgraphs from a graph database, they perform poorly on frequent coupling subgraph mining because they generate many irrelevant subgraphs. We propose a novel graph transformation technique to transform a coupling graph into a generic graph. Based on the transformed coupling graphs, existing graph mining methods are then utilized to discover frequent coupling subgraphs. We prove that the transformation is precise and complete and that the restoration is reversible. Experiments carried out on a database containing 10,511 coupling graphs show that our proposed algorithm reduces the mining time very much in comparison with the existing subgraph mining algorithms. Moreover, we demonstrate the usefulness of frequent coupling subgraphs by applying our algorithm to make accurate predictions of epitopes in antibody-antigen binding.]]></abstract>
##     <issn><![CDATA[1545-5963]]></issn>
##     <htmlFlag><![CDATA[1]]></htmlFlag>
##     <arnumber><![CDATA[6654150]]></arnumber>
##     <doi><![CDATA[10.1109/TCBB.2013.136]]></doi>
##     <publicationId><![CDATA[6654150]]></publicationId>
##     <mdurl><![CDATA[http://ieeexplore.ieee.org/xpl/articleDetails.jsp?tp=&arnumber=6654150&contentType=Journals+%26+Magazines]]></mdurl>
##     <pdf><![CDATA[http://ieeexplore.ieee.org/stamp/stamp.jsp?arnumber=6654150]]></pdf>
##   </document>
##   <document>
##     <rank>259</rank>
##     <title><![CDATA[A Robust Infrared Small Target Detection Algorithm Based on Human Visual System]]></title>
##     <authors><![CDATA[Jinhui Han;  Yong Ma;  Bo Zhou;  Fan Fan;  Kun Liang;  Yu Fang]]></authors>
##     <affiliations><![CDATA[Dept. of Electron. & Inf. Eng., Huazhong Univ. of Sci. & Technol., Wuhan, China]]></affiliations>
##     <controlledterms>
##       <term><![CDATA[infrared detectors]]></term>
##       <term><![CDATA[object detection]]></term>
##     </controlledterms>
##     <thesaurusterms>
##       <term><![CDATA[Arrays]]></term>
##       <term><![CDATA[Brightness]]></term>
##       <term><![CDATA[Object detection]]></term>
##       <term><![CDATA[Resists]]></term>
##       <term><![CDATA[Robustness]]></term>
##       <term><![CDATA[Visual systems]]></term>
##       <term><![CDATA[Visualization]]></term>
##     </thesaurusterms>
##     <pubtitle><![CDATA[Geoscience and Remote Sensing Letters, IEEE]]></pubtitle>
##     <punumber><![CDATA[8859]]></punumber>
##     <pubtype><![CDATA[Journals & Magazines]]></pubtype>
##     <publisher><![CDATA[IEEE]]></publisher>
##     <volume><![CDATA[11]]></volume>
##     <issue><![CDATA[12]]></issue>
##     <py><![CDATA[2014]]></py>
##     <spage><![CDATA[2168]]></spage>
##     <epage><![CDATA[2172]]></epage>
##     <abstract><![CDATA[Robust human visual system (HVS) properties can effectively improve the infrared (IR) small target detection capabilities, such as detection rate, false alarm rate, speed, etc. However, current algorithms based on HVS usually improve one or two of the aforementioned detection capabilities while sacrificing the others. In this letter, a robust IR small target detection algorithm based on HVS is proposed to pursue good performance in detection rate, false alarm rate, and speed simultaneously. First, an HVS size-adaptation process is used, and the IR image after preprocessing is divided into subblocks to improve detection speed. Then, based on HVS contrast mechanism, the improved local contrast measure, which can improve detection rate and reduce false alarm rate, is proposed to calculate the saliency map, and a threshold operation along with a rapid traversal mechanism based on HVS attention shift mechanism is used to get the target subblocks quickly. Experimental results show the proposed algorithm has good robustness and efficiency for real IR small target detection applications.]]></abstract>
##     <issn><![CDATA[1545-598X]]></issn>
##     <htmlFlag><![CDATA[1]]></htmlFlag>
##     <arnumber><![CDATA[6819810]]></arnumber>
##     <doi><![CDATA[10.1109/LGRS.2014.2323236]]></doi>
##     <publicationId><![CDATA[6819810]]></publicationId>
##     <mdurl><![CDATA[http://ieeexplore.ieee.org/xpl/articleDetails.jsp?tp=&arnumber=6819810&contentType=Journals+%26+Magazines]]></mdurl>
##     <pdf><![CDATA[http://ieeexplore.ieee.org/stamp/stamp.jsp?arnumber=6819810]]></pdf>
##   </document>
##   <document>
##     <rank>260</rank>
##     <title><![CDATA[Attenuation Correction Synthesis for Hybrid PET-MR Scanners: Application to Brain Studies]]></title>
##     <authors><![CDATA[Burgos, N.;  Cardoso, M.J.;  Thielemans, K.;  Modat, M.;  Pedemonte, S.;  Dickson, J.;  Barnes, A.;  Ahmed, R.;  Mahoney, C.J.;  Schott, J.M.;  Duncan, J.S.;  Atkinson, D.;  Arridge, S.R.;  Hutton, B.F.;  Ourselin, S.]]></authors>
##     <affiliations><![CDATA[Centre for Med. Image Comput., Univ. Coll. London, London, UK]]></affiliations>
##     <controlledterms>
##       <term><![CDATA[biomedical MRI]]></term>
##       <term><![CDATA[biomedical equipment]]></term>
##       <term><![CDATA[brain]]></term>
##       <term><![CDATA[computerised tomography]]></term>
##       <term><![CDATA[data acquisition]]></term>
##       <term><![CDATA[image matching]]></term>
##       <term><![CDATA[image reconstruction]]></term>
##       <term><![CDATA[medical image processing]]></term>
##       <term><![CDATA[neurophysiology]]></term>
##       <term><![CDATA[positron emission tomography]]></term>
##       <term><![CDATA[visual databases]]></term>
##     </controlledterms>
##     <thesaurusterms>
##       <term><![CDATA[Accuracy]]></term>
##       <term><![CDATA[Attenuation]]></term>
##       <term><![CDATA[Bones]]></term>
##       <term><![CDATA[Computed tomography]]></term>
##       <term><![CDATA[Databases]]></term>
##       <term><![CDATA[Magnetic resonance imaging]]></term>
##       <term><![CDATA[Positron emission tomography]]></term>
##     </thesaurusterms>
##     <pubtitle><![CDATA[Medical Imaging, IEEE Transactions on]]></pubtitle>
##     <punumber><![CDATA[42]]></punumber>
##     <pubtype><![CDATA[Journals & Magazines]]></pubtype>
##     <publisher><![CDATA[IEEE]]></publisher>
##     <volume><![CDATA[33]]></volume>
##     <issue><![CDATA[12]]></issue>
##     <py><![CDATA[2014]]></py>
##     <spage><![CDATA[2332]]></spage>
##     <epage><![CDATA[2341]]></epage>
##     <abstract><![CDATA[Attenuation correction is an essential requirement for quantification of positron emission tomography (PET) data. In PET/CT acquisition systems, attenuation maps are derived from computed tomography (CT) images. However, in hybrid PET/MR scanners, magnetic resonance imaging (MRI) images do not directly provide a patient-specific attenuation map. The aim of the proposed work is to improve attenuation correction for PET/MR scanners by generating synthetic CTs and attenuation maps. The synthetic images are generated through a multi-atlas information propagation scheme, locally matching the MRI-derived patient's morphology to a database of MRI/CT pairs, using a local image similarity measure. Results show significant improvements in CT synthesis and PET reconstruction accuracy when compared to a segmentation method using an ultrashort-echo-time MRI sequence and to a simplified atlas-based method.]]></abstract>
##     <issn><![CDATA[0278-0062]]></issn>
##     <htmlFlag><![CDATA[1]]></htmlFlag>
##     <arnumber><![CDATA[6858020]]></arnumber>
##     <doi><![CDATA[10.1109/TMI.2014.2340135]]></doi>
##     <publicationId><![CDATA[6858020]]></publicationId>
##     <mdurl><![CDATA[http://ieeexplore.ieee.org/xpl/articleDetails.jsp?tp=&arnumber=6858020&contentType=Journals+%26+Magazines]]></mdurl>
##     <pdf><![CDATA[http://ieeexplore.ieee.org/stamp/stamp.jsp?arnumber=6858020]]></pdf>
##   </document>
##   <document>
##     <rank>261</rank>
##     <title><![CDATA[On a Mathematical Model for Low-Rate Shrew DDoS]]></title>
##     <authors><![CDATA[Jingtang Luo;  Xiaolong Yang;  Jin Wang;  Jie Xu;  Jian Sun;  Keping Long]]></authors>
##     <affiliations><![CDATA[Sch. of Commun. & Inf. Eng., Univ. of Electron. Sci. & Technol. of China, Chengdu, China]]></affiliations>
##     <controlledterms>
##       <term><![CDATA[Internet]]></term>
##       <term><![CDATA[computer network security]]></term>
##       <term><![CDATA[transport protocols]]></term>
##     </controlledterms>
##     <thesaurusterms>
##       <term><![CDATA[Adaptation models]]></term>
##       <term><![CDATA[Bandwidth]]></term>
##       <term><![CDATA[Computer crime]]></term>
##       <term><![CDATA[Delays]]></term>
##       <term><![CDATA[Mathematical model]]></term>
##       <term><![CDATA[Packet loss]]></term>
##       <term><![CDATA[Throughput]]></term>
##     </thesaurusterms>
##     <pubtitle><![CDATA[Information Forensics and Security, IEEE Transactions on]]></pubtitle>
##     <punumber><![CDATA[10206]]></punumber>
##     <pubtype><![CDATA[Journals & Magazines]]></pubtype>
##     <publisher><![CDATA[IEEE]]></publisher>
##     <volume><![CDATA[9]]></volume>
##     <issue><![CDATA[7]]></issue>
##     <py><![CDATA[2014]]></py>
##     <spage><![CDATA[1069]]></spage>
##     <epage><![CDATA[1083]]></epage>
##     <abstract><![CDATA[The shrew distributed denial of service (DDoS) attack is very detrimental for many applications, since it can throttle TCP flows to a small fraction of their ideal rate at very low attack cost. Earlier works mainly focused on empirical studies of defending against the shrew DDoS, and very few of them provided analytic results about the attack itself. In this paper, we propose a mathematical model for estimating attack effect of this stealthy type of DDoS. By originally capturing the adjustment behaviors of victim TCPs congestion window, our model can comprehensively evaluate the combined impact of attack pattern (i.e., how the attack is configured) and network environment on attack effect (the existing models failed to consider the impact of network environment). Henceforth, our model has higher accuracy over a wider range of network environments. The relative error of our model remains around 10% for most attack patterns and network environments, whereas the relative error of the benchmark model in previous works has a mean value of 69.57%, and it could be more than 180% in some cases. More importantly, our model reveals some novel properties of the shrew attack from the interaction between attack pattern and network environment, such as the minimum cost formula to launch a successful attack, and the maximum effect formula of a shrew attack. With them, we are able to find out how to adaptively tune the attack parameters (e.g., the DoS burst length) to improve its attack effect in a given network environment, and how to reconfigure the network resource (e.g., the bottleneck buffer size) to mitigate the shrew DDoS with a given attack pattern. Finally, based on our theoretical results, we put forward a simple strategy to defend the shrew attack. The simulation results indicate that this strategy can remarkably increase TCP throughput by nearly half of the bottleneck bandwidth (and can be higher) for general attack patterns.]]></abstract>
##     <issn><![CDATA[1556-6013]]></issn>
##     <htmlFlag><![CDATA[1]]></htmlFlag>
##     <arnumber><![CDATA[6807757]]></arnumber>
##     <doi><![CDATA[10.1109/TIFS.2014.2321034]]></doi>
##     <publicationId><![CDATA[6807757]]></publicationId>
##     <mdurl><![CDATA[http://ieeexplore.ieee.org/xpl/articleDetails.jsp?tp=&arnumber=6807757&contentType=Journals+%26+Magazines]]></mdurl>
##     <pdf><![CDATA[http://ieeexplore.ieee.org/stamp/stamp.jsp?arnumber=6807757]]></pdf>
##   </document>
##   <document>
##     <rank>262</rank>
##     <title><![CDATA[Study of GaN LED ITO Nano-Gratings With Standing Wave Analysis]]></title>
##     <authors><![CDATA[Halpin, G.;  Robinson, T.;  Xiaomin Jin;  Xiang-Ning Kang;  Guo-Ying Zhang]]></authors>
##     <affiliations><![CDATA[Electr. Eng. Dept., California Polytech. State Univ., San Luis Obispo, AZ, USA]]></affiliations>
##     <controlledterms>
##       <term><![CDATA[III-V semiconductors]]></term>
##       <term><![CDATA[diffraction gratings]]></term>
##       <term><![CDATA[gallium compounds]]></term>
##       <term><![CDATA[indium compounds]]></term>
##       <term><![CDATA[light emitting diodes]]></term>
##       <term><![CDATA[nanophotonics]]></term>
##       <term><![CDATA[sapphire]]></term>
##       <term><![CDATA[wide band gap semiconductors]]></term>
##     </controlledterms>
##     <thesaurusterms>
##       <term><![CDATA[Finite difference methods]]></term>
##       <term><![CDATA[Gallium nitride]]></term>
##       <term><![CDATA[Gratings]]></term>
##       <term><![CDATA[Indium tin oxide]]></term>
##       <term><![CDATA[Light emitting diodes]]></term>
##       <term><![CDATA[Substrates]]></term>
##       <term><![CDATA[Time-domain analysis]]></term>
##     </thesaurusterms>
##     <pubtitle><![CDATA[Photonics Journal, IEEE]]></pubtitle>
##     <punumber><![CDATA[4563994]]></punumber>
##     <pubtype><![CDATA[Journals & Magazines]]></pubtype>
##     <publisher><![CDATA[IEEE]]></publisher>
##     <volume><![CDATA[6]]></volume>
##     <issue><![CDATA[3]]></issue>
##     <py><![CDATA[2014]]></py>
##     <spage><![CDATA[1]]></spage>
##     <epage><![CDATA[10]]></epage>
##     <abstract><![CDATA[This study reveals the effect of nanoscale ITO transmission gratings on light emission from the top, sides, and bottom of a GaN light-emitting diode (LED), based on the substrate standing wave analysis. First, we show that sapphire substrate thickness affects the standing wave pattern in the LED and find the best- and worst-case sapphire thicknesses. Second, we find that adding nanoscale ITO transmission gratings can improve light extraction by 222% or 253%, depending on the reference chosen. Third, we observe that maximizing top light emission with the nano-grating can significantly reduce bottom and side light emissions. Finally, we study grating performance over different wavelengths and generate the LED spectrum.]]></abstract>
##     <issn><![CDATA[1943-0655]]></issn>
##     <htmlFlag><![CDATA[1]]></htmlFlag>
##     <arnumber><![CDATA[6814847]]></arnumber>
##     <doi><![CDATA[10.1109/JPHOT.2014.2323296]]></doi>
##     <publicationId><![CDATA[6814847]]></publicationId>
##     <mdurl><![CDATA[http://ieeexplore.ieee.org/xpl/articleDetails.jsp?tp=&arnumber=6814847&contentType=Journals+%26+Magazines]]></mdurl>
##     <pdf><![CDATA[http://ieeexplore.ieee.org/stamp/stamp.jsp?arnumber=6814847]]></pdf>
##   </document>
##   <document>
##     <rank>263</rank>
##     <title><![CDATA[Design of an Alignment Tolerant Miniature Optical Subassembly Module]]></title>
##     <authors><![CDATA[Hak-Soon Lee;  Wenjing Yue;  Sang-Shin Lee]]></authors>
##     <affiliations><![CDATA[Dept. of Electron. Eng., Kwangwoon Univ., Seoul, South Korea]]></affiliations>
##     <controlledterms>
##       <term><![CDATA[microlenses]]></term>
##       <term><![CDATA[optical design techniques]]></term>
##       <term><![CDATA[optical fibre couplers]]></term>
##     </controlledterms>
##     <thesaurusterms>
##       <term><![CDATA[Apertures]]></term>
##       <term><![CDATA[Couplings]]></term>
##       <term><![CDATA[Lenses]]></term>
##       <term><![CDATA[Light sources]]></term>
##       <term><![CDATA[Optical coupling]]></term>
##       <term><![CDATA[Optical fibers]]></term>
##     </thesaurusterms>
##     <pubtitle><![CDATA[Photonics Journal, IEEE]]></pubtitle>
##     <punumber><![CDATA[4563994]]></punumber>
##     <pubtype><![CDATA[Journals & Magazines]]></pubtype>
##     <publisher><![CDATA[IEEE]]></publisher>
##     <volume><![CDATA[6]]></volume>
##     <issue><![CDATA[2]]></issue>
##     <py><![CDATA[2014]]></py>
##     <spage><![CDATA[1]]></spage>
##     <epage><![CDATA[9]]></epage>
##     <abstract><![CDATA[A miniature optical subassembly module incorporating a pair of collimating and focusing lenses was proposed and designed to provide affordable alignment tolerance in terms of optical coupling efficiency. The light source was specifically modeled taking advantage of multiple point sources placed on the emitting aperture. This scheme was practically verified to be superior to the conventional scheme involving a single point source by estimating the focal length and spot size of a focused beam. The performance of the proposed source modeling was scrutinized with respect to the number of point sources. The optical coupling of the subassembly to a fiber was investigated theoretically and experimentally in light of the characteristics of the displacement of the focused beam relative to the lateral displacement of the light source. Finally, a tradeoff between the positional 3-dB tolerance of the light source and the maximum available coupling efficiency was established with certitude.]]></abstract>
##     <issn><![CDATA[1943-0655]]></issn>
##     <htmlFlag><![CDATA[1]]></htmlFlag>
##     <arnumber><![CDATA[6755455]]></arnumber>
##     <doi><![CDATA[10.1109/JPHOT.2014.2309644]]></doi>
##     <publicationId><![CDATA[6755455]]></publicationId>
##     <mdurl><![CDATA[http://ieeexplore.ieee.org/xpl/articleDetails.jsp?tp=&arnumber=6755455&contentType=Journals+%26+Magazines]]></mdurl>
##     <pdf><![CDATA[http://ieeexplore.ieee.org/stamp/stamp.jsp?arnumber=6755455]]></pdf>
##   </document>
##   <document>
##     <rank>264</rank>
##     <title><![CDATA[Analysis and Design of a High-Order Discrete-Time Passive IIR Low-Pass Filter]]></title>
##     <authors><![CDATA[Tohidian, M.;  Madadi, I.;  Staszewski, R.B.]]></authors>
##     <affiliations><![CDATA[Electron. Res. Lab./DIMES, Delft Univ. of Technol., Delft, Netherlands]]></affiliations>
##     <controlledterms>
##       <term><![CDATA[IIR filters]]></term>
##       <term><![CDATA[logic design]]></term>
##       <term><![CDATA[low-pass filters]]></term>
##     </controlledterms>
##     <thesaurusterms>
##       <term><![CDATA[Capacitors]]></term>
##       <term><![CDATA[Clocks]]></term>
##       <term><![CDATA[Equations]]></term>
##       <term><![CDATA[History]]></term>
##       <term><![CDATA[Linearity]]></term>
##       <term><![CDATA[Noise]]></term>
##       <term><![CDATA[Transfer functions]]></term>
##     </thesaurusterms>
##     <pubtitle><![CDATA[Solid-State Circuits, IEEE Journal of]]></pubtitle>
##     <punumber><![CDATA[4]]></punumber>
##     <pubtype><![CDATA[Journals & Magazines]]></pubtype>
##     <publisher><![CDATA[IEEE]]></publisher>
##     <volume><![CDATA[49]]></volume>
##     <issue><![CDATA[11]]></issue>
##     <py><![CDATA[2014]]></py>
##     <spage><![CDATA[2575]]></spage>
##     <epage><![CDATA[2587]]></epage>
##     <abstract><![CDATA[In this paper, we propose a discrete-time IIR low-pass filter that achieves a high-order of filtering through a charge-sharing rotation. Its sampling rate is then multiplied through pipelining. The first stage of the filter can operate in either a voltage-sampling or charge-sampling mode. It uses switches, capacitors and a simple gm-cell, rather than opamps, thus being compatible with digital nanoscale technology. In the voltage-sampling mode, the gm-cell is bypassed so the filter is fully passive. A 7th-order filter prototype operating at 800 MS/s sampling rate is implemented in TSMC 65 nm CMOS. Bandwidth of this filter is programmable between 400 kHz to 30 MHz with 100 dB maximum stop-band rejection. Its IIP3 is +21 dBm and the averaged spot noise is 4.57 nV/&#x221A;Hz. It consumes 2 mW at 1.2 V and occupies 0.42 mm<sup>2</sup>.]]></abstract>
##     <issn><![CDATA[0018-9200]]></issn>
##     <htmlFlag><![CDATA[1]]></htmlFlag>
##     <arnumber><![CDATA[6922157]]></arnumber>
##     <doi><![CDATA[10.1109/JSSC.2014.2359656]]></doi>
##     <publicationId><![CDATA[6922157]]></publicationId>
##     <mdurl><![CDATA[http://ieeexplore.ieee.org/xpl/articleDetails.jsp?tp=&arnumber=6922157&contentType=Journals+%26+Magazines]]></mdurl>
##     <pdf><![CDATA[http://ieeexplore.ieee.org/stamp/stamp.jsp?arnumber=6922157]]></pdf>
##   </document>
##   <document>
##     <rank>265</rank>
##     <title><![CDATA[Optical-Time-Division Demultiplexing of 172 Gb/s to 43 Gb/s in a-Si:H Waveguides]]></title>
##     <authors><![CDATA[Suda, S.;  Tanizawa, K.;  Kurosu, T.;  Kamei, T.;  Sakakibara, Y.;  Takei, R.;  Mori, M.;  Kawashima, H.;  Namiki, S.]]></authors>
##     <affiliations><![CDATA[Photonics Res. Inst., Nat. Inst. of Adv. Ind. Sci. & Technol., Tsukuba, Japan]]></affiliations>
##     <controlledterms>
##       <term><![CDATA[amorphous semiconductors]]></term>
##       <term><![CDATA[demultiplexing]]></term>
##       <term><![CDATA[elemental semiconductors]]></term>
##       <term><![CDATA[error statistics]]></term>
##       <term><![CDATA[hydrogen]]></term>
##       <term><![CDATA[multiwave mixing]]></term>
##       <term><![CDATA[optical information processing]]></term>
##       <term><![CDATA[optical losses]]></term>
##       <term><![CDATA[optical waveguides]]></term>
##       <term><![CDATA[silicon compounds]]></term>
##       <term><![CDATA[two-photon spectroscopy]]></term>
##     </controlledterms>
##     <thesaurusterms>
##       <term><![CDATA[Adaptive optics]]></term>
##       <term><![CDATA[Nonlinear optics]]></term>
##       <term><![CDATA[Optical amplifiers]]></term>
##       <term><![CDATA[Optical pulses]]></term>
##       <term><![CDATA[Optical pumping]]></term>
##       <term><![CDATA[Optical waveguides]]></term>
##       <term><![CDATA[Optical wavelength conversion]]></term>
##     </thesaurusterms>
##     <pubtitle><![CDATA[Photonics Technology Letters, IEEE]]></pubtitle>
##     <punumber><![CDATA[68]]></punumber>
##     <pubtype><![CDATA[Journals & Magazines]]></pubtype>
##     <publisher><![CDATA[IEEE]]></publisher>
##     <volume><![CDATA[26]]></volume>
##     <issue><![CDATA[5]]></issue>
##     <py><![CDATA[2014]]></py>
##     <spage><![CDATA[426]]></spage>
##     <epage><![CDATA[429]]></epage>
##     <abstract><![CDATA[We demonstrate optical-time-division demultiplexing of 172-43 Gb/s based on degenerate four-wave mixing in a 17-mm-long hydrogenated amorphous silicon (a-Si:H) wire waveguide. The a-Si:H wire waveguide has a record-low propagation loss of 1.0 dB/cm and a low two-photon-absorption loss of 0.2 cm/GW, and the nonlinear figure of merit is estimated to be 0.97, which is more than twice as effective as conventional crystalline silicon wire waveguides. We achieved successful error-free demultiplexing with an ON/OFF conversion efficiency of -10 dB and a power penalty of only 1.1 dB at a bit error rate of 10<sup>-9</sup>. No optical damage to the a-Si:H waveguide was observed even after a continuous pump injection at an average power of several tens of mW for a couple of hours.]]></abstract>
##     <issn><![CDATA[1041-1135]]></issn>
##     <htmlFlag><![CDATA[1]]></htmlFlag>
##     <arnumber><![CDATA[6693721]]></arnumber>
##     <doi><![CDATA[10.1109/LPT.2013.2294954]]></doi>
##     <publicationId><![CDATA[6693721]]></publicationId>
##     <mdurl><![CDATA[http://ieeexplore.ieee.org/xpl/articleDetails.jsp?tp=&arnumber=6693721&contentType=Journals+%26+Magazines]]></mdurl>
##     <pdf><![CDATA[http://ieeexplore.ieee.org/stamp/stamp.jsp?arnumber=6693721]]></pdf>
##   </document>
##   <document>
##     <rank>266</rank>
##     <title><![CDATA[A Compact Low-Power 320-Gb/s WDM Transmitter Based on Silicon Microrings]]></title>
##     <authors><![CDATA[Ran Ding;  Yang Liu;  Qi Li;  Zhe Xuan;  Yangjin Ma;  Yisu Yang;  Lim, A.E.-J.;  Guo-Qiang Lo;  Bergman, K.;  Baehr-Jones, T.;  Hochberg, M.]]></authors>
##     <affiliations><![CDATA[Dept. of Electr. & Comput. Eng., Univ. of Delaware, Newark, DE, USA]]></affiliations>
##     <controlledterms>
##       <term><![CDATA[electro-optical modulation]]></term>
##       <term><![CDATA[elemental semiconductors]]></term>
##       <term><![CDATA[error statistics]]></term>
##       <term><![CDATA[integrated optics]]></term>
##       <term><![CDATA[integrated optoelectronics]]></term>
##       <term><![CDATA[lithium compounds]]></term>
##       <term><![CDATA[micro-optics]]></term>
##       <term><![CDATA[optical design techniques]]></term>
##       <term><![CDATA[optical noise]]></term>
##       <term><![CDATA[optical transmitters]]></term>
##       <term><![CDATA[silicon]]></term>
##       <term><![CDATA[silicon-on-insulator]]></term>
##       <term><![CDATA[telecommunication channels]]></term>
##       <term><![CDATA[wavelength division multiplexing]]></term>
##     </controlledterms>
##     <thesaurusterms>
##       <term><![CDATA[Junctions]]></term>
##       <term><![CDATA[Modulation]]></term>
##       <term><![CDATA[Optical transmitters]]></term>
##       <term><![CDATA[Optical waveguides]]></term>
##       <term><![CDATA[Silicon]]></term>
##       <term><![CDATA[Tuning]]></term>
##       <term><![CDATA[Wavelength division multiplexing]]></term>
##     </thesaurusterms>
##     <pubtitle><![CDATA[Photonics Journal, IEEE]]></pubtitle>
##     <punumber><![CDATA[4563994]]></punumber>
##     <pubtype><![CDATA[Journals & Magazines]]></pubtype>
##     <publisher><![CDATA[IEEE]]></publisher>
##     <volume><![CDATA[6]]></volume>
##     <issue><![CDATA[3]]></issue>
##     <py><![CDATA[2014]]></py>
##     <spage><![CDATA[1]]></spage>
##     <epage><![CDATA[8]]></epage>
##     <abstract><![CDATA[We demonstrate a compact and low-power wavelength-division multiplexing transmitter near a 1550-nm wavelength using silicon microrings. The transmitter is implemented on a silicon-on-insulator photonics platform with a compact footprint of 0.5 mm<sup>2</sup>. The transmitter incorporates 8 wavelength channels with 200-GHz spacing. Each channel achieved error-free operation at 40 Gb/s, resulting in an aggregated data transmission capability of 320 Gb/s. To our knowledge, this is the highest aggregated data rate demonstrated in silicon wavelength-division multiplexing transmitters. Owing to the small device capacitance and the efficient pn-junction modulator design, the transmitter achieves low energy-per-bit values of 36 fJ/bit under 2.4 Vpp drive and 144 fJ/bit under 4.8 Vpp drive. Comparisons are made to a commercial lithium niobate modulator in terms of bit-error-rate versus optical signal-to-noise ratio.]]></abstract>
##     <issn><![CDATA[1943-0655]]></issn>
##     <htmlFlag><![CDATA[1]]></htmlFlag>
##     <arnumber><![CDATA[6820737]]></arnumber>
##     <doi><![CDATA[10.1109/JPHOT.2014.2326656]]></doi>
##     <publicationId><![CDATA[6820737]]></publicationId>
##     <mdurl><![CDATA[http://ieeexplore.ieee.org/xpl/articleDetails.jsp?tp=&arnumber=6820737&contentType=Journals+%26+Magazines]]></mdurl>
##     <pdf><![CDATA[http://ieeexplore.ieee.org/stamp/stamp.jsp?arnumber=6820737]]></pdf>
##   </document>
##   <document>
##     <rank>267</rank>
##     <title><![CDATA[Dynamics of 1550-nm VCSELs With Positive Optoelectronic Feedback: Theory and Experiments]]></title>
##     <authors><![CDATA[Yi-Yuan Xie;  Hong-Jun Che;  Wei-Lun Zhao;  Ye-Xiong Huang;  Wei-Hua Xu;  Xin Li;  Qian Kan;  Jia-Chao Li]]></authors>
##     <affiliations><![CDATA[Sch. of Electron. & Inf. Eng., Southwest Univ., Chongqing, China]]></affiliations>
##     <controlledterms>
##       <term><![CDATA[delays]]></term>
##       <term><![CDATA[integrated optics]]></term>
##       <term><![CDATA[integrated optoelectronics]]></term>
##       <term><![CDATA[laser cavity resonators]]></term>
##       <term><![CDATA[nonlinear optics]]></term>
##       <term><![CDATA[optical chaos]]></term>
##       <term><![CDATA[optical feedback]]></term>
##       <term><![CDATA[optical pulse generation]]></term>
##       <term><![CDATA[surface emitting lasers]]></term>
##     </controlledterms>
##     <thesaurusterms>
##       <term><![CDATA[Delays]]></term>
##       <term><![CDATA[Laser feedback]]></term>
##       <term><![CDATA[Mathematical model]]></term>
##       <term><![CDATA[Nonlinear dynamical systems]]></term>
##       <term><![CDATA[Optical feedback]]></term>
##       <term><![CDATA[Vertical cavity surface emitting lasers]]></term>
##     </thesaurusterms>
##     <pubtitle><![CDATA[Photonics Journal, IEEE]]></pubtitle>
##     <punumber><![CDATA[4563994]]></punumber>
##     <pubtype><![CDATA[Journals & Magazines]]></pubtype>
##     <publisher><![CDATA[IEEE]]></publisher>
##     <volume><![CDATA[6]]></volume>
##     <issue><![CDATA[6]]></issue>
##     <py><![CDATA[2014]]></py>
##     <spage><![CDATA[1]]></spage>
##     <epage><![CDATA[8]]></epage>
##     <abstract><![CDATA[Nonlinear dynamics of a 1550-nm vertical-cavity surface-emitting laser with positive optoelectronic feedback are studied both numerically and experimentally. A mapping of dynamical states is presented in the parameter space of feedback delay time and feedback strength, where different states are identified and shown. A bifurcation diagram of the extrema of output peak series versus the feedback delay time is plotted. Various nonlinear dynamical behaviors, including regular pulsing, quasi-periodic pulsing, and chaotic pulsing, have been numerically and experimentally observed. Both numerical simulation and experimental observation indicate that the laser enters a chaotic pulsing state at certain delay times of the feedback loop through a quasi-periodic route.]]></abstract>
##     <issn><![CDATA[1943-0655]]></issn>
##     <htmlFlag><![CDATA[1]]></htmlFlag>
##     <arnumber><![CDATA[6949615]]></arnumber>
##     <doi><![CDATA[10.1109/JPHOT.2014.2368775]]></doi>
##     <publicationId><![CDATA[6949615]]></publicationId>
##     <mdurl><![CDATA[http://ieeexplore.ieee.org/xpl/articleDetails.jsp?tp=&arnumber=6949615&contentType=Journals+%26+Magazines]]></mdurl>
##     <pdf><![CDATA[http://ieeexplore.ieee.org/stamp/stamp.jsp?arnumber=6949615]]></pdf>
##   </document>
##   <document>
##     <rank>268</rank>
##     <title><![CDATA[An Active Damper for Stabilizing Power-Electronics-Based AC Systems]]></title>
##     <authors><![CDATA[Xiongfei Wang;  Blaabjerg, F.;  Liserre, M.;  Zhe Chen;  Jinwei He;  Yunwei Li]]></authors>
##     <affiliations><![CDATA[Dept. of Energy Technol., Aalborg Univ., Aalborg, Denmark]]></affiliations>
##     <controlledterms>
##       <term><![CDATA[power convertors]]></term>
##       <term><![CDATA[power grids]]></term>
##       <term><![CDATA[power supply quality]]></term>
##       <term><![CDATA[power system interconnection]]></term>
##       <term><![CDATA[power system stability]]></term>
##       <term><![CDATA[vibration control]]></term>
##     </controlledterms>
##     <thesaurusterms>
##       <term><![CDATA[Current control]]></term>
##       <term><![CDATA[Gain]]></term>
##       <term><![CDATA[Harmonic analysis]]></term>
##       <term><![CDATA[Impedance]]></term>
##       <term><![CDATA[Rectifiers]]></term>
##       <term><![CDATA[Shock absorbers]]></term>
##       <term><![CDATA[Stability analysis]]></term>
##     </thesaurusterms>
##     <pubtitle><![CDATA[Power Electronics, IEEE Transactions on]]></pubtitle>
##     <punumber><![CDATA[63]]></punumber>
##     <pubtype><![CDATA[Journals & Magazines]]></pubtype>
##     <publisher><![CDATA[IEEE]]></publisher>
##     <volume><![CDATA[29]]></volume>
##     <issue><![CDATA[7]]></issue>
##     <py><![CDATA[2014]]></py>
##     <spage><![CDATA[3318]]></spage>
##     <epage><![CDATA[3329]]></epage>
##     <abstract><![CDATA[The interactions among the parallel grid-connected converters coupled through the grid impedance tend to result in stability and power quality problems. To address them, this paper proposes an active damper based on a high bandwidth power electronics converter. The general idea behind this proposal is to dynamically reshape the grid impedance profile seen from the point of common coupling of the converters, such that the potential oscillations and resonance propagation in the parallel grid-connected converters can be mitigated. To validate the effectiveness of the active damper, simulations and experimental tests on a three-converter-based setup are carried out. The results show that the active damper can become a promising way to stabilize the power-electronics-based ac power systems.]]></abstract>
##     <issn><![CDATA[0885-8993]]></issn>
##     <htmlFlag><![CDATA[1]]></htmlFlag>
##     <arnumber><![CDATA[6583340]]></arnumber>
##     <doi><![CDATA[10.1109/TPEL.2013.2278716]]></doi>
##     <publicationId><![CDATA[6583340]]></publicationId>
##     <mdurl><![CDATA[http://ieeexplore.ieee.org/xpl/articleDetails.jsp?tp=&arnumber=6583340&contentType=Journals+%26+Magazines]]></mdurl>
##     <pdf><![CDATA[http://ieeexplore.ieee.org/stamp/stamp.jsp?arnumber=6583340]]></pdf>
##   </document>
##   <document>
##     <rank>269</rank>
##     <title><![CDATA[Nanoscale Low Crosstalk Photonic Crystal Integrated Sensor Array]]></title>
##     <authors><![CDATA[Daquan Yang;  Huiping Tian;  Yuefeng Ji]]></authors>
##     <affiliations><![CDATA[State Key Lab. of Inf. Photonics & Opt. Commun., Beijing Univ. of Posts & Telecommun., Beijing, China]]></affiliations>
##     <controlledterms>
##       <term><![CDATA[Q-factor]]></term>
##       <term><![CDATA[finite difference time-domain analysis]]></term>
##       <term><![CDATA[nanophotonics]]></term>
##       <term><![CDATA[optical crosstalk]]></term>
##       <term><![CDATA[optical sensors]]></term>
##       <term><![CDATA[photonic crystals]]></term>
##       <term><![CDATA[sensor arrays]]></term>
##     </controlledterms>
##     <thesaurusterms>
##       <term><![CDATA[Cavity resonators]]></term>
##       <term><![CDATA[Crosstalk]]></term>
##       <term><![CDATA[Optical sensors]]></term>
##       <term><![CDATA[Photonic crystals]]></term>
##       <term><![CDATA[Sensor arrays]]></term>
##     </thesaurusterms>
##     <pubtitle><![CDATA[Photonics Journal, IEEE]]></pubtitle>
##     <punumber><![CDATA[4563994]]></punumber>
##     <pubtype><![CDATA[Journals & Magazines]]></pubtype>
##     <publisher><![CDATA[IEEE]]></publisher>
##     <volume><![CDATA[6]]></volume>
##     <issue><![CDATA[1]]></issue>
##     <py><![CDATA[2014]]></py>
##     <spage><![CDATA[1]]></spage>
##     <epage><![CDATA[7]]></epage>
##     <abstract><![CDATA[We theoretically investigate a flexible design of building nanoscale photonic crystal (PhC) integrated sensor array with low crosstalk. The proposed device consists of array of side-coupled PhC resonant cavities with high Q-factors over 2 &#x00D7;10<sup>3</sup>. The extinction ratio of well-defined single resonance exceeds 30 dB. Each resonant cavity has different resonant wavelengths and independently shifts its resonance in response to the refractive index variations. With three-dimensional finite-difference time-domain (3D-FDTD) method, simulation results demonstrate that the proposed sensor array is desirable to perform monolithically integrated sensing and multiplexed detection. Particularly, the design method here makes it possible to effectively enhance sensor array integration density and simultaneously restrain crosstalk between each other adjacent sensors. The refractive index sensitivity of 100 nm/RIU and the crosstalk lower than -4 dB are observed, respectively. Both the specific result and the general idea are promising in future optical multiplexed sensing and nanophotonic integration.]]></abstract>
##     <issn><![CDATA[1943-0655]]></issn>
##     <htmlFlag><![CDATA[1]]></htmlFlag>
##     <arnumber><![CDATA[6725597]]></arnumber>
##     <doi><![CDATA[10.1109/JPHOT.2014.2302805]]></doi>
##     <publicationId><![CDATA[6725597]]></publicationId>
##     <mdurl><![CDATA[http://ieeexplore.ieee.org/xpl/articleDetails.jsp?tp=&arnumber=6725597&contentType=Journals+%26+Magazines]]></mdurl>
##     <pdf><![CDATA[http://ieeexplore.ieee.org/stamp/stamp.jsp?arnumber=6725597]]></pdf>
##   </document>
##   <document>
##     <rank>270</rank>
##     <title><![CDATA[Quality Improvement of GaN on Si Substrate for Ultraviolet Photodetector Application]]></title>
##     <authors><![CDATA[Chao-Wei Hsu;  Yung-Feng Chen;  Yan-Kuin Su]]></authors>
##     <affiliations><![CDATA[Dept. of Electr. Eng., Nat. Cheng Kung Univ., Tainan, Taiwan]]></affiliations>
##     <controlledterms>
##       <term><![CDATA[III-V semiconductors]]></term>
##       <term><![CDATA[X-ray diffraction]]></term>
##       <term><![CDATA[aluminium compounds]]></term>
##       <term><![CDATA[gallium compounds]]></term>
##       <term><![CDATA[nucleation]]></term>
##       <term><![CDATA[optical fabrication]]></term>
##       <term><![CDATA[photodetectors]]></term>
##       <term><![CDATA[photoluminescence]]></term>
##       <term><![CDATA[sapphire]]></term>
##       <term><![CDATA[silicon]]></term>
##       <term><![CDATA[wide band gap semiconductors]]></term>
##     </controlledterms>
##     <pubtitle><![CDATA[Quantum Electronics, IEEE Journal of]]></pubtitle>
##     <punumber><![CDATA[3]]></punumber>
##     <pubtype><![CDATA[Journals & Magazines]]></pubtype>
##     <publisher><![CDATA[IEEE]]></publisher>
##     <volume><![CDATA[50]]></volume>
##     <issue><![CDATA[1]]></issue>
##     <py><![CDATA[2014]]></py>
##     <spage><![CDATA[35]]></spage>
##     <epage><![CDATA[41]]></epage>
##     <abstract><![CDATA[GaN is grown on an Si substrate using metal-organic vapor-phase epitaxy. Compared with the full width at half maximum values from X-ray diffraction patterns and photoluminescence spectra of conventional GaN on the Si substrate, those of GaN on the Si substrate with the insertion of various-temperature AlN nucleation layers and Al<sub>0.3</sub>Ga<sub>0.7</sub>N/GaN superlattice intermediate layers are reduced by 34.9% and 25.6%, respectively. In addition, Raman spectra show that residual stress on the GaN epilayers decreased by 0.35 GPa. The c-lattice parameter of the GaN epilayer is 5.1844 &#x212B;, which is close to that of an unstrained GaN layer. Ultraviolet metal-semiconductor-metal photodetectors are fabricated on an almost-crack-free GaN surface. The dark current of a photodetector on the Si substrate is 2.4 &#x00D7;10<sup>-11</sup> A at a 9 V applied bias, which is one order of magnitude smaller than that of a photodetector on a conventional sapphire substrate. The maximum quantum efficiency value of a photodetector on the Si substrate is ~ 97% with an incident light wavelength of 360 nm and a 9 V applied bias.]]></abstract>
##     <issn><![CDATA[0018-9197]]></issn>
##     <htmlFlag><![CDATA[1]]></htmlFlag>
##     <arnumber><![CDATA[6674047]]></arnumber>
##     <doi><![CDATA[10.1109/JQE.2013.2292502]]></doi>
##     <publicationId><![CDATA[6674047]]></publicationId>
##     <mdurl><![CDATA[http://ieeexplore.ieee.org/xpl/articleDetails.jsp?tp=&arnumber=6674047&contentType=Journals+%26+Magazines]]></mdurl>
##     <pdf><![CDATA[http://ieeexplore.ieee.org/stamp/stamp.jsp?arnumber=6674047]]></pdf>
##   </document>
##   <document>
##     <rank>271</rank>
##     <title><![CDATA[High-Frequency Operation of a DC/AC/DC System for HVDC Applications]]></title>
##     <authors><![CDATA[Luth, T.;  Merlin, M.M.C.;  Green, T.C.;  Hassan, F.;  Barker, C.D.]]></authors>
##     <affiliations><![CDATA[Imperial Coll. London, London, UK]]></affiliations>
##     <controlledterms>
##       <term><![CDATA[AC-DC power convertors]]></term>
##       <term><![CDATA[DC-AC power convertors]]></term>
##       <term><![CDATA[DC-DC power convertors]]></term>
##       <term><![CDATA[HVDC power convertors]]></term>
##       <term><![CDATA[capacitors]]></term>
##       <term><![CDATA[switching convertors]]></term>
##       <term><![CDATA[transformers]]></term>
##     </controlledterms>
##     <thesaurusterms>
##       <term><![CDATA[Capacitors]]></term>
##       <term><![CDATA[HVDC transmission]]></term>
##       <term><![CDATA[Power conversion]]></term>
##       <term><![CDATA[Switching loss]]></term>
##       <term><![CDATA[Topology]]></term>
##       <term><![CDATA[Valves]]></term>
##     </thesaurusterms>
##     <pubtitle><![CDATA[Power Electronics, IEEE Transactions on]]></pubtitle>
##     <punumber><![CDATA[63]]></punumber>
##     <pubtype><![CDATA[Journals & Magazines]]></pubtype>
##     <publisher><![CDATA[IEEE]]></publisher>
##     <volume><![CDATA[29]]></volume>
##     <issue><![CDATA[8]]></issue>
##     <py><![CDATA[2014]]></py>
##     <spage><![CDATA[4107]]></spage>
##     <epage><![CDATA[4115]]></epage>
##     <abstract><![CDATA[Voltage ratings for HVdc point-to-point connections are not standardized and tend to depend on the latest available cable technology. DC/DC conversion at HV is required for interconnection of such HVdc schemes as well as to interface dc wind farms. Modular multilevel voltage source converters (VSCs), such as the modular multilevel converter (MMC) or the alternate arm converter (AAC), have been shown to incur significantly lower switching losses than previous two- or three-level VSCs. This paper presents a dc/ac/dc system using a transformer coupling two modular multilevel VSCs. In such a system, the capacitors occupy a large fraction of the volume of the cells but a significant reduction in volume can be achieved by raising the ac frequency. Using high frequency can also bring benefits to other passive components such as the transformer but also results in higher switching losses due to the higher number of waveform steps per second. This leads to a tradeoff between volume and losses which has been explored in this study and verified by simulation results with a transistor level model of 30-MW case study. The outcome of the study shows that a frequency of 350 Hz provides a significant improvement in volume but also a penalty in losses compared to 50 Hz.]]></abstract>
##     <issn><![CDATA[0885-8993]]></issn>
##     <htmlFlag><![CDATA[1]]></htmlFlag>
##     <arnumber><![CDATA[6675087]]></arnumber>
##     <doi><![CDATA[10.1109/TPEL.2013.2292614]]></doi>
##     <publicationId><![CDATA[6675087]]></publicationId>
##     <mdurl><![CDATA[http://ieeexplore.ieee.org/xpl/articleDetails.jsp?tp=&arnumber=6675087&contentType=Journals+%26+Magazines]]></mdurl>
##     <pdf><![CDATA[http://ieeexplore.ieee.org/stamp/stamp.jsp?arnumber=6675087]]></pdf>
##   </document>
##   <document>
##     <rank>272</rank>
##     <title><![CDATA[Design and Evaluation of Multispectral LiDAR for the Recovery of Arboreal Parameters]]></title>
##     <authors><![CDATA[Wallace, A.M.;  McCarthy, A.;  Nichol, C.J.;  Ximing Ren;  Morak, S.;  Martinez-Ramirez, D.;  Woodhouse, I.H.;  Buller, G.S.]]></authors>
##     <affiliations><![CDATA[Sch. of Eng. & Phys. Sci., Heriot-Watt Univ., Edinburgh, UK]]></affiliations>
##     <controlledterms>
##       <term><![CDATA[Markov processes]]></term>
##       <term><![CDATA[Monte Carlo methods]]></term>
##       <term><![CDATA[geophysical techniques]]></term>
##       <term><![CDATA[optical radar]]></term>
##       <term><![CDATA[remote sensing by radar]]></term>
##       <term><![CDATA[vegetation mapping]]></term>
##     </controlledterms>
##     <thesaurusterms>
##       <term><![CDATA[Instruments]]></term>
##       <term><![CDATA[Laser radar]]></term>
##       <term><![CDATA[Optical imaging]]></term>
##       <term><![CDATA[Optical sensors]]></term>
##       <term><![CDATA[Physiology]]></term>
##       <term><![CDATA[Vegetation]]></term>
##       <term><![CDATA[Wavelength measurement]]></term>
##     </thesaurusterms>
##     <pubtitle><![CDATA[Geoscience and Remote Sensing, IEEE Transactions on]]></pubtitle>
##     <punumber><![CDATA[36]]></punumber>
##     <pubtype><![CDATA[Journals & Magazines]]></pubtype>
##     <publisher><![CDATA[IEEE]]></publisher>
##     <volume><![CDATA[52]]></volume>
##     <issue><![CDATA[8]]></issue>
##     <py><![CDATA[2014]]></py>
##     <spage><![CDATA[4942]]></spage>
##     <epage><![CDATA[4954]]></epage>
##     <abstract><![CDATA[Multispectral light detection and ranging (LiDAR) has the potential to recover structural and physiological data from arboreal samples and, by extension, from forest canopies when deployed on aerial or space platforms. In this paper, we describe the design and evaluation of a prototype multispectral LiDAR system and demonstrate the measurement of leaf and bark area and abundance profiles using a series of experiments on tree samples &#x201C;viewed from above&#x201D; by tilting living conifers such that the apex is directed on the viewing axis. As the complete recovery of all structural and physiological parameters is ill posed with a restricted set of four wavelengths, we used leaf and bark spectra measured in the laboratory to constrain parameter inversion by an extended reversible jump Markov chain Monte Carlo algorithm. However, we also show in a separate experiment how the multispectral LiDAR can recover directly a profile of Normalized Difference Vegetation Index (NDVI), which is verified against the laboratory spectral measurements. Our work shows the potential of multispectral LiDAR to recover both structural and physiological data and also highlights the fine spatial resolution that can be achieved with time-correlated single-photon counting.]]></abstract>
##     <issn><![CDATA[0196-2892]]></issn>
##     <htmlFlag><![CDATA[1]]></htmlFlag>
##     <arnumber><![CDATA[6672004]]></arnumber>
##     <doi><![CDATA[10.1109/TGRS.2013.2285942]]></doi>
##     <publicationId><![CDATA[6672004]]></publicationId>
##     <mdurl><![CDATA[http://ieeexplore.ieee.org/xpl/articleDetails.jsp?tp=&arnumber=6672004&contentType=Journals+%26+Magazines]]></mdurl>
##     <pdf><![CDATA[http://ieeexplore.ieee.org/stamp/stamp.jsp?arnumber=6672004]]></pdf>
##   </document>
##   <document>
##     <rank>273</rank>
##     <title><![CDATA[Shielding Heterogeneous MPSoCs From Untrustworthy 3PIPs Through Security- Driven Task Scheduling]]></title>
##     <authors><![CDATA[Chen Liu;  Rajendran, J.;  Chengmo Yang;  Karri, R.]]></authors>
##     <affiliations><![CDATA[Dept. of Electr. & Comput. Eng., Univ. of Delaware, Newark, DE, USA]]></affiliations>
##     <controlledterms>
##       <term><![CDATA[logic design]]></term>
##       <term><![CDATA[multiprocessing systems]]></term>
##       <term><![CDATA[optimisation]]></term>
##       <term><![CDATA[scheduling]]></term>
##       <term><![CDATA[system-on-chip]]></term>
##       <term><![CDATA[trusted computing]]></term>
##     </controlledterms>
##     <thesaurusterms>
##       <term><![CDATA[Image color analysis]]></term>
##       <term><![CDATA[Network security]]></term>
##       <term><![CDATA[Optimal scheduling]]></term>
##       <term><![CDATA[Scheduling]]></term>
##       <term><![CDATA[Trojan horses]]></term>
##       <term><![CDATA[Trust management]]></term>
##     </thesaurusterms>
##     <pubtitle><![CDATA[Emerging Topics in Computing, IEEE Transactions on]]></pubtitle>
##     <punumber><![CDATA[6245516]]></punumber>
##     <pubtype><![CDATA[Journals & Magazines]]></pubtype>
##     <publisher><![CDATA[IEEE]]></publisher>
##     <volume><![CDATA[2]]></volume>
##     <issue><![CDATA[4]]></issue>
##     <py><![CDATA[2014]]></py>
##     <spage><![CDATA[461]]></spage>
##     <epage><![CDATA[472]]></epage>
##     <abstract><![CDATA[Multiprocessor system-on-chip (MPSoC) platforms face some of the most demanding security concerns, as they process, store, and communicate sensitive information using third-party intellectual property (3PIP) cores. The complexity of MPSoC makes it expensive and time consuming to fully analyze and test during the design stage. This has given rise to the trend of outsourcing design and fabrication of 3PIP components, that may not be trustworthy. To protect MPSoCs against malicious modifications, we impose a set of security-driven diversity constraints into the task scheduling step of the MPSoC design process, enabling the system to detect the presence of malicious modifications or to mute their effects during application execution. We pose the security-constrained MPSoC task scheduling as a multidimensional optimization problem, and propose a set of heuristics to ensure that the introduced security constraints can be fulfilled with a minimum impact on the other design goals such as performance and hardware. Experimental results show that without any extra cores, security constraints can be fulfilled within four vendors and 81% overhead in schedule length.]]></abstract>
##     <issn><![CDATA[2168-6750]]></issn>
##     <htmlFlag><![CDATA[1]]></htmlFlag>
##     <arnumber><![CDATA[6879429]]></arnumber>
##     <doi><![CDATA[10.1109/TETC.2014.2348182]]></doi>
##     <publicationId><![CDATA[6879429]]></publicationId>
##     <mdurl><![CDATA[http://ieeexplore.ieee.org/xpl/articleDetails.jsp?tp=&arnumber=6879429&contentType=Journals+%26+Magazines]]></mdurl>
##     <pdf><![CDATA[http://ieeexplore.ieee.org/stamp/stamp.jsp?arnumber=6879429]]></pdf>
##   </document>
##   <document>
##     <rank>274</rank>
##     <title><![CDATA[Checkerboard Nanoplasmonic Gold Structure for Long-Wave Infrared Absorption Enhancement]]></title>
##     <authors><![CDATA[Awad, E.;  Abdel-Rahman, M.;  Zia, M.F.]]></authors>
##     <affiliations><![CDATA[Dept. of Electr. Eng., King Saud Univ., Riyadh, Saudi Arabia]]></affiliations>
##     <controlledterms>
##       <term><![CDATA[electron beam lithography]]></term>
##       <term><![CDATA[finite difference time-domain analysis]]></term>
##       <term><![CDATA[gold]]></term>
##       <term><![CDATA[infrared spectra]]></term>
##       <term><![CDATA[nanofabrication]]></term>
##       <term><![CDATA[nanolithography]]></term>
##       <term><![CDATA[nanostructured materials]]></term>
##       <term><![CDATA[near-field scanning optical microscopy]]></term>
##       <term><![CDATA[plasmonics]]></term>
##       <term><![CDATA[silicon compounds]]></term>
##       <term><![CDATA[sputter deposition]]></term>
##     </controlledterms>
##     <thesaurusterms>
##       <term><![CDATA[Absorption]]></term>
##       <term><![CDATA[Electric fields]]></term>
##       <term><![CDATA[Gold]]></term>
##       <term><![CDATA[Optical polarization]]></term>
##       <term><![CDATA[Optical surface waves]]></term>
##       <term><![CDATA[Plasmons]]></term>
##       <term><![CDATA[Scanning electron microscopy]]></term>
##     </thesaurusterms>
##     <pubtitle><![CDATA[Photonics Journal, IEEE]]></pubtitle>
##     <punumber><![CDATA[4563994]]></punumber>
##     <pubtype><![CDATA[Journals & Magazines]]></pubtype>
##     <publisher><![CDATA[IEEE]]></publisher>
##     <volume><![CDATA[6]]></volume>
##     <issue><![CDATA[4]]></issue>
##     <py><![CDATA[2014]]></py>
##     <spage><![CDATA[1]]></spage>
##     <epage><![CDATA[7]]></epage>
##     <abstract><![CDATA[A localized nanoplasmonic induced absorption enhancement in silicon nitride (Si<sub>3</sub>N<sub>4</sub>) dielectric material using a nanoscale novel checkerboard gold (Au) structure is demonstrated. The checkerboard structure is fabricated on a Si<sub>3</sub>N<sub>4</sub> layer using electron-beam lithography and sputter deposition techniques. The plasmonic electric field and optical absorption enhancement are measured using scanning near-field optical microscopy. Finite-difference time-domain simulations are utilized to characterize the absorption spectral response enhancement together with its dependence on incidence angle and polarization. The checkerboard shows a broadband average spectral absorption enhancement of 63.2% over the wavelength range 8-12 &#x03BC;m with a maximum enhancement of 107% at 8 &#x03BC;m and a minimum enhancement of 24.8% at 12 &#x03BC;m. The degradation of enhanced absorption with incidence angle variation (00-60&#x03BF;) is less than 1.6% at 10.6-&#x03BC;m wavelength. The checkerboard device shows polarization-independent absorption enhancement with incidence angles.]]></abstract>
##     <issn><![CDATA[1943-0655]]></issn>
##     <htmlFlag><![CDATA[1]]></htmlFlag>
##     <arnumber><![CDATA[6872517]]></arnumber>
##     <doi><![CDATA[10.1109/JPHOT.2014.2345879]]></doi>
##     <publicationId><![CDATA[6872517]]></publicationId>
##     <mdurl><![CDATA[http://ieeexplore.ieee.org/xpl/articleDetails.jsp?tp=&arnumber=6872517&contentType=Journals+%26+Magazines]]></mdurl>
##     <pdf><![CDATA[http://ieeexplore.ieee.org/stamp/stamp.jsp?arnumber=6872517]]></pdf>
##   </document>
##   <document>
##     <rank>275</rank>
##     <title><![CDATA[Modeling the high-frequency complex modulus of silicone rubber using standing lamb waves and an inverse finite element method]]></title>
##     <authors><![CDATA[Jonsson, U.;  Lindahl, O.;  Andersson, B.]]></authors>
##     <affiliations><![CDATA[Dept. of Appl. Phys. & Electron., Umea Univ., Umea, Sweden]]></affiliations>
##     <controlledterms>
##       <term><![CDATA[elastic moduli]]></term>
##       <term><![CDATA[finite element analysis]]></term>
##       <term><![CDATA[piezoelectric materials]]></term>
##       <term><![CDATA[piezoelectricity]]></term>
##       <term><![CDATA[silicone rubber]]></term>
##       <term><![CDATA[surface acoustic waves]]></term>
##       <term><![CDATA[viscoelasticity]]></term>
##     </controlledterms>
##     <thesaurusterms>
##       <term><![CDATA[Equations]]></term>
##       <term><![CDATA[Finite element analysis]]></term>
##       <term><![CDATA[Materials]]></term>
##       <term><![CDATA[Mathematical model]]></term>
##       <term><![CDATA[Rubber]]></term>
##       <term><![CDATA[Strain]]></term>
##       <term><![CDATA[Stress]]></term>
##     </thesaurusterms>
##     <pubtitle><![CDATA[Ultrasonics, Ferroelectrics, and Frequency Control, IEEE Transactions on]]></pubtitle>
##     <punumber><![CDATA[58]]></punumber>
##     <pubtype><![CDATA[Journals & Magazines]]></pubtype>
##     <publisher><![CDATA[IEEE]]></publisher>
##     <volume><![CDATA[61]]></volume>
##     <issue><![CDATA[12]]></issue>
##     <py><![CDATA[2014]]></py>
##     <spage><![CDATA[2106]]></spage>
##     <epage><![CDATA[2120]]></epage>
##     <abstract><![CDATA[To gain an understanding of the high-frequency elastic properties of silicone rubber, a finite element model of a cylindrical piezoelectric element, in contact with a silicone rubber disk, was constructed. The frequency-dependent elastic modulus of the silicone rubber was modeled by a four-parameter fractional derivative viscoelastic model in the 100 to 250 kHz frequency range. The calculations were carried out in the range of the first radial resonance frequency of the sensor. At the resonance, the hyperelastic effect of the silicone rubber was modeled by a hyperelastic compensating function. The calculated response was matched to the measured response by using the transitional peaks in the impedance spectrum that originates from the switching of standing Lamb wave modes in the silicone rubber. To validate the results, the impedance responses of three 5-mm-thick silicone rubber disks, with different radial lengths, were measured. The calculated and measured transitional frequencies have been compared in detail. The comparison showed very good agreement, with average relative differences of 0.7%, 0.6%, and 0.7% for the silicone rubber samples with radial lengths of 38.0, 21.4, and 11.0 mm, respectively. The average complex elastic moduli of the samples were (0.97 + 0.009i) GPa at 100 kHz and (0.97 + 0.005i) GPa at 250 kHz.]]></abstract>
##     <issn><![CDATA[0885-3010]]></issn>
##     <htmlFlag><![CDATA[1]]></htmlFlag>
##     <arnumber><![CDATA[6968704]]></arnumber>
##     <doi><![CDATA[10.1109/TUFFC.2014.006471]]></doi>
##     <publicationId><![CDATA[6968704]]></publicationId>
##     <mdurl><![CDATA[http://ieeexplore.ieee.org/xpl/articleDetails.jsp?tp=&arnumber=6968704&contentType=Journals+%26+Magazines]]></mdurl>
##     <pdf><![CDATA[http://ieeexplore.ieee.org/stamp/stamp.jsp?arnumber=6968704]]></pdf>
##   </document>
##   <document>
##     <rank>276</rank>
##     <title><![CDATA[Integration of Micro Resistance Thermometer Detectors in AlGaN/GaN Devices]]></title>
##     <authors><![CDATA[Arenas, O.;  Al Alam, E.;  Thevenot, A.;  Cordier, Y.;  Jaouad, A.;  Aimez, V.;  Maher, H.;  Ares, R.;  Boone, F.]]></authors>
##     <affiliations><![CDATA[Lab. Nanotechnol. Nanosystemes, Univ. de Sherbrooke, Sherbrooke, QC, Canada]]></affiliations>
##     <controlledterms>
##       <term><![CDATA[III-V semiconductors]]></term>
##       <term><![CDATA[aluminium compounds]]></term>
##       <term><![CDATA[calibration]]></term>
##       <term><![CDATA[finite element analysis]]></term>
##       <term><![CDATA[gallium compounds]]></term>
##       <term><![CDATA[high electron mobility transistors]]></term>
##       <term><![CDATA[platinum]]></term>
##       <term><![CDATA[semiconductor device measurement]]></term>
##       <term><![CDATA[semiconductor device reliability]]></term>
##       <term><![CDATA[thermometers]]></term>
##       <term><![CDATA[wide band gap semiconductors]]></term>
##     </controlledterms>
##     <thesaurusterms>
##       <term><![CDATA[Aluminum gallium nitride]]></term>
##       <term><![CDATA[Gallium nitride]]></term>
##       <term><![CDATA[HEMTs]]></term>
##       <term><![CDATA[MODFETs]]></term>
##       <term><![CDATA[Resistance]]></term>
##       <term><![CDATA[Temperature measurement]]></term>
##       <term><![CDATA[Temperature sensors]]></term>
##     </thesaurusterms>
##     <pubtitle><![CDATA[Electron Devices Society, IEEE Journal of the]]></pubtitle>
##     <punumber><![CDATA[6245494]]></punumber>
##     <pubtype><![CDATA[Journals & Magazines]]></pubtype>
##     <publisher><![CDATA[IEEE]]></publisher>
##     <volume><![CDATA[2]]></volume>
##     <issue><![CDATA[6]]></issue>
##     <py><![CDATA[2014]]></py>
##     <spage><![CDATA[145]]></spage>
##     <epage><![CDATA[148]]></epage>
##     <abstract><![CDATA[Temperature measurements in AlGaN/GaN high electron mobility transistors are required for proper device design, modeling and achieving appropriate reliability. These measurements usually require sophisticated equipment and extensive calibration. This study evaluates the feasibility of temperature measurements by integration of a Pt resistance thermal detector (RTD) in an &#x201C;un-gated&#x201D; transistor and evaluating their electrical interactions. The integrated RTD presents the advantage of being independent of the device. Micro RTD showed a linear response in the calibration interval (0 to 206 &#x00B0;C). Measured temperature values using the micro RTD are in agreement with 3D finite element simulations at multiple bias conditions in the &#x201C;un-gated&#x201D; transistor. Measurements show no noticeable electrical perturbation between the device and RTD under simultaneous operation.]]></abstract>
##     <issn><![CDATA[2168-6734]]></issn>
##     <htmlFlag><![CDATA[1]]></htmlFlag>
##     <arnumber><![CDATA[6874488]]></arnumber>
##     <doi><![CDATA[10.1109/JEDS.2014.2346391]]></doi>
##     <publicationId><![CDATA[6874488]]></publicationId>
##     <mdurl><![CDATA[http://ieeexplore.ieee.org/xpl/articleDetails.jsp?tp=&arnumber=6874488&contentType=Journals+%26+Magazines]]></mdurl>
##     <pdf><![CDATA[http://ieeexplore.ieee.org/stamp/stamp.jsp?arnumber=6874488]]></pdf>
##   </document>
##   <document>
##     <rank>277</rank>
##     <title><![CDATA[Comparison of Optical Self-Phase Locked Loop Techniques for Frequency Stabilization of Oscillators]]></title>
##     <authors><![CDATA[Li Zhang;  Poddar, A.K.;  Rohde, U.L.;  Daryoush, A.S.]]></authors>
##     <affiliations><![CDATA[Electr. & Comput. Eng. Dept., Drexel Univ., Philadelphia, PA, USA]]></affiliations>
##     <controlledterms>
##       <term><![CDATA[frequency stability]]></term>
##       <term><![CDATA[integrated optics]]></term>
##       <term><![CDATA[integrated optoelectronics]]></term>
##       <term><![CDATA[optical communication equipment]]></term>
##       <term><![CDATA[optical delay lines]]></term>
##       <term><![CDATA[optical fibre communication]]></term>
##       <term><![CDATA[optical modulation]]></term>
##       <term><![CDATA[optical noise]]></term>
##       <term><![CDATA[optical phase locked loops]]></term>
##       <term><![CDATA[optical phase shifters]]></term>
##       <term><![CDATA[phase locked oscillators]]></term>
##       <term><![CDATA[phase noise]]></term>
##     </controlledterms>
##     <thesaurusterms>
##       <term><![CDATA[Delay lines]]></term>
##       <term><![CDATA[Optical fibers]]></term>
##       <term><![CDATA[Phase locked loops]]></term>
##       <term><![CDATA[Phase noise]]></term>
##       <term><![CDATA[Voltage-controlled oscillators]]></term>
##     </thesaurusterms>
##     <pubtitle><![CDATA[Photonics Journal, IEEE]]></pubtitle>
##     <punumber><![CDATA[4563994]]></punumber>
##     <pubtype><![CDATA[Journals & Magazines]]></pubtype>
##     <publisher><![CDATA[IEEE]]></publisher>
##     <volume><![CDATA[6]]></volume>
##     <issue><![CDATA[5]]></issue>
##     <py><![CDATA[2014]]></py>
##     <spage><![CDATA[1]]></spage>
##     <epage><![CDATA[15]]></epage>
##     <abstract><![CDATA[Self-phase locked loop (SPLL) through long optical delay lines is investigated for improving the phase noise of the optoelectronic oscillator (OEO) and the voltage-controlled oscillator (VCO). This paper features analytical modeling of close-in to carrier phase noise of oscillators employing SPLL technique using single and multiple delay loops. Experimental results are also provided to verify the analytical modeling and explore critical circuit design parameters to achieve substantial phase noise reduction. Performance comparison is presented for SPLL using electrical phase shifter, Mach-Zehnder modulator (MZM)-based phase shifting, and electrical VCO. Phase noise reduction of 65 dB at 1-kHz offset is achieved for an electrical VCO employing a dual-loop SPLL with 3- and 5-km optical fiber delay lines, whereas the phase locking based on MZM and electrical phase shifter provided 20- and 38-dB reductions at 1-kHz offset for an OEO, respectively.]]></abstract>
##     <issn><![CDATA[1943-0655]]></issn>
##     <htmlFlag><![CDATA[1]]></htmlFlag>
##     <arnumber><![CDATA[6910224]]></arnumber>
##     <doi><![CDATA[10.1109/JPHOT.2014.2360294]]></doi>
##     <publicationId><![CDATA[6910224]]></publicationId>
##     <mdurl><![CDATA[http://ieeexplore.ieee.org/xpl/articleDetails.jsp?tp=&arnumber=6910224&contentType=Journals+%26+Magazines]]></mdurl>
##     <pdf><![CDATA[http://ieeexplore.ieee.org/stamp/stamp.jsp?arnumber=6910224]]></pdf>
##   </document>
##   <document>
##     <rank>278</rank>
##     <title><![CDATA[On Recognizing Face Images With Weight and Age Variations]]></title>
##     <authors><![CDATA[Singh, M.;  Nagpal, S.;  Singh, R.;  Vatsa, M.]]></authors>
##     <affiliations><![CDATA[Indraprastha Inst. of Inf. Technol., New Delhi, India]]></affiliations>
##     <controlledterms>
##       <term><![CDATA[age issues]]></term>
##       <term><![CDATA[biometrics (access control)]]></term>
##       <term><![CDATA[decision trees]]></term>
##       <term><![CDATA[face recognition]]></term>
##       <term><![CDATA[learning (artificial intelligence)]]></term>
##       <term><![CDATA[neural nets]]></term>
##     </controlledterms>
##     <thesaurusterms>
##       <term><![CDATA[Aging]]></term>
##       <term><![CDATA[Biomedical imaging]]></term>
##       <term><![CDATA[Face recognition]]></term>
##       <term><![CDATA[Muscles]]></term>
##       <term><![CDATA[Skeleton]]></term>
##     </thesaurusterms>
##     <pubtitle><![CDATA[Access, IEEE]]></pubtitle>
##     <punumber><![CDATA[6287639]]></punumber>
##     <pubtype><![CDATA[Journals & Magazines]]></pubtype>
##     <publisher><![CDATA[IEEE]]></publisher>
##     <volume><![CDATA[2]]></volume>
##     <py><![CDATA[2014]]></py>
##     <spage><![CDATA[822]]></spage>
##     <epage><![CDATA[830]]></epage>
##     <abstract><![CDATA[With the increase in age, there are changes in skeletal structure, muscle mass, and body fat. For recognizing faces with age variations, researchers have generally focused on the skeletal structure and muscle mass. However, the effect of change in body fat has not been studied with respect to face recognition. In this paper, we incorporate weight information to improve the performance of face recognition with age variations. The proposed algorithm utilizes neural network and random decision forest to encode age variations across different weight categories. The results are reported on the WhoIsIt database prepared by the authors containing 1109 images from 110 individuals with age and weight variations. The comparison with existing state-of-the-art algorithms and commercial system on WhoIsIt and FG-Net databases shows that the proposed algorithm outperforms existing algorithms significantly.]]></abstract>
##     <issn><![CDATA[2169-3536]]></issn>
##     <htmlFlag><![CDATA[1]]></htmlFlag>
##     <arnumber><![CDATA[6871282]]></arnumber>
##     <doi><![CDATA[10.1109/ACCESS.2014.2344667]]></doi>
##     <publicationId><![CDATA[6871282]]></publicationId>
##     <mdurl><![CDATA[http://ieeexplore.ieee.org/xpl/articleDetails.jsp?tp=&arnumber=6871282&contentType=Journals+%26+Magazines]]></mdurl>
##     <pdf><![CDATA[http://ieeexplore.ieee.org/stamp/stamp.jsp?arnumber=6871282]]></pdf>
##   </document>
##   <document>
##     <rank>279</rank>
##     <title><![CDATA[Robust and Reverse-Engineering Resilient PUF Authentication and Key-Exchange by Substring Matching]]></title>
##     <authors><![CDATA[Rostami, M.;  Majzoobi, M.;  Koushanfar, F.;  Wallach, D.S.;  Devadas, S.]]></authors>
##     <affiliations><![CDATA[Dept. of Electr. & Comput. Eng., Rice Univ., Houston, TX, USA]]></affiliations>
##     <controlledterms>
##       <term><![CDATA[cryptographic protocols]]></term>
##       <term><![CDATA[learning (artificial intelligence)]]></term>
##       <term><![CDATA[reverse engineering]]></term>
##       <term><![CDATA[statistical analysis]]></term>
##     </controlledterms>
##     <thesaurusterms>
##       <term><![CDATA[Authentication]]></term>
##       <term><![CDATA[Error correction]]></term>
##       <term><![CDATA[Licenses]]></term>
##       <term><![CDATA[Network security]]></term>
##       <term><![CDATA[Protocols]]></term>
##       <term><![CDATA[Reverse engineering]]></term>
##     </thesaurusterms>
##     <pubtitle><![CDATA[Emerging Topics in Computing, IEEE Transactions on]]></pubtitle>
##     <punumber><![CDATA[6245516]]></punumber>
##     <pubtype><![CDATA[Journals & Magazines]]></pubtype>
##     <publisher><![CDATA[IEEE]]></publisher>
##     <volume><![CDATA[2]]></volume>
##     <issue><![CDATA[1]]></issue>
##     <py><![CDATA[2014]]></py>
##     <spage><![CDATA[37]]></spage>
##     <epage><![CDATA[49]]></epage>
##     <abstract><![CDATA[This paper proposes novel robust and low-overhead physical unclonable function (PUF) authentication and key exchange protocols that are resilient against reverse-engineering attacks. The protocols are executed between a party with access to a physical PUF (prover) and a trusted party who has access to the PUF compact model (verifier). The proposed protocols do not follow the classic paradigm of exposing the full PUF responses or a transformation of them. Instead, random subsets of the PUF response strings are sent to the verifier so the exact position of the subset is obfuscated for the third-party channel observers. Authentication of the responses at the verifier side is done by matching the substring to the available full response string; the index of the matching point is the actual obfuscated secret (or key) and not the response substring itself. We perform a thorough analysis of resiliency of the protocols against various adversarial acts, including machine learning and statistical attacks. The attack analysis guides us in tuning the parameters of the protocol for an efficient and secure implementation. The low overhead and practicality of the protocols are evaluated and confirmed by hardware implementation.]]></abstract>
##     <issn><![CDATA[2168-6750]]></issn>
##     <htmlFlag><![CDATA[1]]></htmlFlag>
##     <arnumber><![CDATA[6714458]]></arnumber>
##     <doi><![CDATA[10.1109/TETC.2014.2300635]]></doi>
##     <publicationId><![CDATA[6714458]]></publicationId>
##     <mdurl><![CDATA[http://ieeexplore.ieee.org/xpl/articleDetails.jsp?tp=&arnumber=6714458&contentType=Journals+%26+Magazines]]></mdurl>
##     <pdf><![CDATA[http://ieeexplore.ieee.org/stamp/stamp.jsp?arnumber=6714458]]></pdf>
##   </document>
##   <document>
##     <rank>280</rank>
##     <title><![CDATA[Little Knowledge Isn&#x2019;t Always Dangerous&#x2014;Understanding Water Distribution Networks Using Centrality Metrics]]></title>
##     <authors><![CDATA[Narayanan, I.;  Vasan, A.;  Sarangan, V.;  Kadengal, J.;  Sivasubramaniam, A.]]></authors>
##     <affiliations><![CDATA[Dept. of Comput. Sci. & Eng., Pennsylvania State Univ., University Park, PA, USA]]></affiliations>
##     <controlledterms>
##       <term><![CDATA[failure analysis]]></term>
##       <term><![CDATA[metering]]></term>
##       <term><![CDATA[pipe flow]]></term>
##       <term><![CDATA[water supply]]></term>
##     </controlledterms>
##     <thesaurusterms>
##       <term><![CDATA[Analytical models]]></term>
##       <term><![CDATA[Benchmark testing]]></term>
##       <term><![CDATA[Instruments]]></term>
##       <term><![CDATA[Measurement]]></term>
##       <term><![CDATA[Monitoring]]></term>
##       <term><![CDATA[Topology]]></term>
##       <term><![CDATA[Water conservation]]></term>
##     </thesaurusterms>
##     <pubtitle><![CDATA[Emerging Topics in Computing, IEEE Transactions on]]></pubtitle>
##     <punumber><![CDATA[6245516]]></punumber>
##     <pubtype><![CDATA[Journals & Magazines]]></pubtype>
##     <publisher><![CDATA[IEEE]]></publisher>
##     <volume><![CDATA[2]]></volume>
##     <issue><![CDATA[2]]></issue>
##     <py><![CDATA[2014]]></py>
##     <spage><![CDATA[225]]></spage>
##     <epage><![CDATA[238]]></epage>
##     <abstract><![CDATA[Addressing nonrevenue water, a major issue for water utilities, requires identification of strategic metering locations using calibrated hydraulic models of the water network. However, calibrated hydraulic models use both static and dynamic network data and are often prohibitively expensive. We present an approach to understand water network operations that uses only the static information of the network. Specifically, we analyze water networks using augmented centrality measures. We use readily available static information about network elements (e.g., diameters of pipes) rather than calibrated dynamic information (e.g., roughness coefficients of pipes, demands at nodes), and model each network element appropriately for analysis using customized centrality measures. Our approach identifies: 1) pipes carrying higher flows; 2) nodes with higher delivery heads; and 3) pipes with higher failure impact. Each of the above helps in determining strategic instrumentation locations. We validate our analysis by comparison with fully calibrated hydraulic models for three benchmark topologies. Our experimental evaluation shows that centrality analysis yields results which have a match of more than 85% with those obtained using calibrated hydraulic models on benchmark networks without significant over-provisioning. We also present results from a real-life case study where our approach matched 78% with locations picked by experts.]]></abstract>
##     <issn><![CDATA[2168-6750]]></issn>
##     <htmlFlag><![CDATA[1]]></htmlFlag>
##     <arnumber><![CDATA[6731535]]></arnumber>
##     <doi><![CDATA[10.1109/TETC.2014.2304502]]></doi>
##     <publicationId><![CDATA[6731535]]></publicationId>
##     <mdurl><![CDATA[http://ieeexplore.ieee.org/xpl/articleDetails.jsp?tp=&arnumber=6731535&contentType=Journals+%26+Magazines]]></mdurl>
##     <pdf><![CDATA[http://ieeexplore.ieee.org/stamp/stamp.jsp?arnumber=6731535]]></pdf>
##   </document>
##   <document>
##     <rank>281</rank>
##     <title><![CDATA[Optimization for Centralized and Decentralized Cognitive Radio Networks]]></title>
##     <authors><![CDATA[Hasegawa, M.;  Hirai, H.;  Nagano, K.;  Harada, H.;  Aihara, K.]]></authors>
##     <affiliations><![CDATA[Dept. of Electr. Eng., Univ. of Sci., Tokyo, Japan]]></affiliations>
##     <controlledterms>
##       <term><![CDATA[Hopfield neural nets]]></term>
##       <term><![CDATA[cognitive radio]]></term>
##       <term><![CDATA[decision making]]></term>
##       <term><![CDATA[distributed algorithms]]></term>
##       <term><![CDATA[minimisation]]></term>
##       <term><![CDATA[polynomials]]></term>
##       <term><![CDATA[radio spectrum management]]></term>
##       <term><![CDATA[telecommunication computing]]></term>
##       <term><![CDATA[telecommunication network management]]></term>
##     </controlledterms>
##     <thesaurusterms>
##       <term><![CDATA[Cognitive radio]]></term>
##       <term><![CDATA[Cognitive science]]></term>
##       <term><![CDATA[Linear programming]]></term>
##       <term><![CDATA[Mobile communication]]></term>
##       <term><![CDATA[Optimization]]></term>
##       <term><![CDATA[Wireless LAN]]></term>
##       <term><![CDATA[Wireless networks]]></term>
##     </thesaurusterms>
##     <pubtitle><![CDATA[Proceedings of the IEEE]]></pubtitle>
##     <punumber><![CDATA[5]]></punumber>
##     <pubtype><![CDATA[Journals & Magazines]]></pubtype>
##     <publisher><![CDATA[IEEE]]></publisher>
##     <volume><![CDATA[102]]></volume>
##     <issue><![CDATA[4]]></issue>
##     <py><![CDATA[2014]]></py>
##     <spage><![CDATA[574]]></spage>
##     <epage><![CDATA[584]]></epage>
##     <abstract><![CDATA[Cognitive radio technology improves radio resource usage by reconfiguring the wireless connection settings according to the optimum decisions, which are made on the basis of the collected context information. This paper focuses on optimization algorithms for decision making to optimize radio resource usage in heterogeneous cognitive wireless networks. For networks with centralized management, we proposed a novel optimization algorithm whose solution is guaranteed to be exactly optimal. In order to avoid an exponential increase of computational complexity in large-scale wireless networks, we model the target optimization problem as a minimum cost-flow problem and find the solution of the problem in polynomial time. For the networks with decentralized management, we propose a distributed algorithm using the distributed energy minimization dynamics of the Hopfield-Tank neural network. Our algorithm minimizes a given objective function without any centralized calculation. We derive the decision-making rule for each terminal to optimize the entire network. We demonstrate the validity of the proposed algorithms by several numerical simulations and the feasibility of the proposed schemes by designing and implementing them on experimental cognitive radio network systems.]]></abstract>
##     <issn><![CDATA[0018-9219]]></issn>
##     <htmlFlag><![CDATA[1]]></htmlFlag>
##     <arnumber><![CDATA[6767084]]></arnumber>
##     <doi><![CDATA[10.1109/JPROC.2014.2306255]]></doi>
##     <publicationId><![CDATA[6767084]]></publicationId>
##     <mdurl><![CDATA[http://ieeexplore.ieee.org/xpl/articleDetails.jsp?tp=&arnumber=6767084&contentType=Journals+%26+Magazines]]></mdurl>
##     <pdf><![CDATA[http://ieeexplore.ieee.org/stamp/stamp.jsp?arnumber=6767084]]></pdf>
##   </document>
##   <document>
##     <rank>282</rank>
##     <title><![CDATA[Sparse Beamforming and User-Centric Clustering for Downlink Cloud Radio Access Network]]></title>
##     <authors><![CDATA[Binbin Dai;  Wei Yu]]></authors>
##     <affiliations><![CDATA[Edward S. Rogers Sr. Dept. of Electr. & Comput. Eng., Univ. of Toronto, Toronto, ON, Canada]]></affiliations>
##     <controlledterms>
##       <term><![CDATA[array signal processing]]></term>
##       <term><![CDATA[cloud computing]]></term>
##       <term><![CDATA[least mean squares methods]]></term>
##       <term><![CDATA[optimisation]]></term>
##       <term><![CDATA[pattern clustering]]></term>
##       <term><![CDATA[radio access networks]]></term>
##     </controlledterms>
##     <thesaurusterms>
##       <term><![CDATA[Approximation methods]]></term>
##       <term><![CDATA[Array signal processing]]></term>
##       <term><![CDATA[Cloud computing]]></term>
##       <term><![CDATA[Downlink]]></term>
##       <term><![CDATA[Dynamic scheduling]]></term>
##       <term><![CDATA[Heuristic algorithms]]></term>
##       <term><![CDATA[Radio access networks]]></term>
##     </thesaurusterms>
##     <pubtitle><![CDATA[Access, IEEE]]></pubtitle>
##     <punumber><![CDATA[6287639]]></punumber>
##     <pubtype><![CDATA[Journals & Magazines]]></pubtype>
##     <publisher><![CDATA[IEEE]]></publisher>
##     <volume><![CDATA[2]]></volume>
##     <py><![CDATA[2014]]></py>
##     <spage><![CDATA[1326]]></spage>
##     <epage><![CDATA[1339]]></epage>
##     <abstract><![CDATA[This paper considers a downlink cloud radio access network (C-RAN) in which all the base-stations (BSs) are connected to a central computing cloud via digital backhaul links with finite capacities. Each user is associated with a user-centric cluster of BSs; the central processor shares the user's data with the BSs in the cluster, which then cooperatively serve the user through joint beamforming. Under this setup, this paper investigates the user scheduling, BS clustering, and beamforming design problem from a network utility maximization perspective. Differing from previous works, this paper explicitly considers the per-BS backhaul capacity constraints. We formulate the network utility maximization problem for the downlink C-RAN under two different models depending on whether the BS clustering for each user is dynamic or static over different user scheduling time slots. In the former case, the user-centric BS cluster is dynamically optimized for each scheduled user along with the beamforming vector in each time-frequency slot, whereas in the latter case, the user-centric BS cluster is fixed for each user and we jointly optimize the user scheduling and the beamforming vector to account for the backhaul constraints. In both cases, the nonconvex per-BS backhaul constraints are approximated using the reweighted &#x2113;<sub>1</sub>-norm technique. This approximation allows us to reformulate the per-BS backhaul constraints into weighted per-BS power constraints and solve the weighted sum rate maximization problem through a generalized weighted minimum mean square error approach. This paper shows that the proposed dynamic clustering algorithm can achieve significant performance gain over existing naive clustering schemes. This paper also proposes two heuristic static clustering schemes that can already achieve a substantial portion of the gain.]]></abstract>
##     <issn><![CDATA[2169-3536]]></issn>
##     <htmlFlag><![CDATA[1]]></htmlFlag>
##     <arnumber><![CDATA[6920005]]></arnumber>
##     <doi><![CDATA[10.1109/ACCESS.2014.2362860]]></doi>
##     <publicationId><![CDATA[6920005]]></publicationId>
##     <mdurl><![CDATA[http://ieeexplore.ieee.org/xpl/articleDetails.jsp?tp=&arnumber=6920005&contentType=Journals+%26+Magazines]]></mdurl>
##     <pdf><![CDATA[http://ieeexplore.ieee.org/stamp/stamp.jsp?arnumber=6920005]]></pdf>
##   </document>
##   <document>
##     <rank>283</rank>
##     <title><![CDATA[Power Metering for Virtual Machine in Cloud Computing-Challenges and Opportunities]]></title>
##     <authors><![CDATA[Gu, C.;  Huang, H.;  Jia, X.]]></authors>
##     <affiliations><![CDATA[Department of Computer Science and Technology, Harbin Institute of Technology Shenzhen Graduate School, Shenzhen, China]]></affiliations>
##     <thesaurusterms>
##       <term><![CDATA[Cloud computing]]></term>
##       <term><![CDATA[Meter reading]]></term>
##       <term><![CDATA[Power demand]]></term>
##       <term><![CDATA[Power distribution]]></term>
##       <term><![CDATA[Power system economics]]></term>
##       <term><![CDATA[Power system measurements]]></term>
##       <term><![CDATA[Resource management]]></term>
##       <term><![CDATA[Virtual machining]]></term>
##       <term><![CDATA[Virtualization]]></term>
##     </thesaurusterms>
##     <pubtitle><![CDATA[Access, IEEE]]></pubtitle>
##     <punumber><![CDATA[6287639]]></punumber>
##     <pubtype><![CDATA[Journals & Magazines]]></pubtype>
##     <publisher><![CDATA[IEEE]]></publisher>
##     <volume><![CDATA[2]]></volume>
##     <py><![CDATA[2014]]></py>
##     <spage><![CDATA[1106]]></spage>
##     <epage><![CDATA[1116]]></epage>
##     <abstract><![CDATA[The virtual machine (VM) is the most basic unit for virtualization and resource allocation. The study of VM power metering is the key to reducing the power consumption of data centers. In this paper, we make a comprehensive investigation in issues regarding VM power metering, including server models, sampling, VM power metering methods, and the accuracy of the methods. We will review many up-to-date power metering methods in this paper, and analyze their efficiencies, as well as evaluate their performance. Open research issues, such as VM service billing, power budgeting, and energy saving scheduling, are discussed, with an objective to spark new research interests in this field.]]></abstract>
##     <issn><![CDATA[2169-3536]]></issn>
##     <htmlFlag><![CDATA[1]]></htmlFlag>
##     <arnumber><![CDATA[6905704]]></arnumber>
##     <doi><![CDATA[10.1109/ACCESS.2014.2358992]]></doi>
##     <publicationId><![CDATA[6905704]]></publicationId>
##     <mdurl><![CDATA[http://ieeexplore.ieee.org/xpl/articleDetails.jsp?tp=&arnumber=6905704&contentType=Journals+%26+Magazines]]></mdurl>
##     <pdf><![CDATA[http://ieeexplore.ieee.org/stamp/stamp.jsp?arnumber=6905704]]></pdf>
##   </document>
##   <document>
##     <rank>284</rank>
##     <title><![CDATA[Operating Strategies for a GB Integrated Gas and Electricity Network Considering the Uncertainty in Wind Power Forecasts]]></title>
##     <authors><![CDATA[Qadrdan, M.;  Jianzhong Wu;  Jenkins, N.;  Ekanayake, J.]]></authors>
##     <affiliations><![CDATA[Electr. & Electron. Eng., Cardiff Univ., Cardiff, UK]]></affiliations>
##     <controlledterms>
##       <term><![CDATA[integer programming]]></term>
##       <term><![CDATA[linear programming]]></term>
##       <term><![CDATA[pipes]]></term>
##       <term><![CDATA[power generation dispatch]]></term>
##       <term><![CDATA[power generation planning]]></term>
##       <term><![CDATA[pumped-storage power stations]]></term>
##       <term><![CDATA[stochastic programming]]></term>
##       <term><![CDATA[thermal power stations]]></term>
##       <term><![CDATA[transmission networks]]></term>
##       <term><![CDATA[wind power]]></term>
##     </controlledterms>
##     <thesaurusterms>
##       <term><![CDATA[Compressors]]></term>
##       <term><![CDATA[Electricity]]></term>
##       <term><![CDATA[Programming]]></term>
##       <term><![CDATA[Stochastic processes]]></term>
##       <term><![CDATA[Uncertainty]]></term>
##       <term><![CDATA[Wind forecasting]]></term>
##       <term><![CDATA[Wind power generation]]></term>
##     </thesaurusterms>
##     <pubtitle><![CDATA[Sustainable Energy, IEEE Transactions on]]></pubtitle>
##     <punumber><![CDATA[5165391]]></punumber>
##     <pubtype><![CDATA[Journals & Magazines]]></pubtype>
##     <publisher><![CDATA[IEEE]]></publisher>
##     <volume><![CDATA[5]]></volume>
##     <issue><![CDATA[1]]></issue>
##     <py><![CDATA[2014]]></py>
##     <spage><![CDATA[128]]></spage>
##     <epage><![CDATA[138]]></epage>
##     <abstract><![CDATA[In many power systems, in particular in Great Britain (GB), significant wind generation is anticipated and gas-fired generation will continue to play an important role. Gas-fired generating units act as a link between the gas and electricity networks. The variability of wind power is, therefore, transferred to the gas network by influencing the gas demand for electricity generation. Operation of a GB integrated gas and electricity network considering the uncertainty in wind power forecast was investigated using three operational planning methods: deterministic, two-stage stochastic programming, and multistage stochastic programming. These methods were benchmarked against a perfect foresight model which has no uncertainty associated with the wind power forecast. In all the methods, thermal generators were controlled through an integrated unit commitment and economic dispatch algorithm that used mixed integer programming. The nonlinear characteristics of the gas network, including the gas flow along pipes and the operation of compressors, were taken into account and the resultant nonlinear problem was solved using successive linear programming. The operational strategies determined by the stochastic programming methods showed reductions of the operational costs compared to the solution of the deterministic method by almost 1%. Also, using the stochastic programming methods to schedule the thermal units was shown to make a better use of pumped storage plants to mitigate the variability and uncertainty of the net demand.]]></abstract>
##     <issn><![CDATA[1949-3029]]></issn>
##     <htmlFlag><![CDATA[1]]></htmlFlag>
##     <arnumber><![CDATA[6589136]]></arnumber>
##     <doi><![CDATA[10.1109/TSTE.2013.2274818]]></doi>
##     <publicationId><![CDATA[6589136]]></publicationId>
##     <mdurl><![CDATA[http://ieeexplore.ieee.org/xpl/articleDetails.jsp?tp=&arnumber=6589136&contentType=Journals+%26+Magazines]]></mdurl>
##     <pdf><![CDATA[http://ieeexplore.ieee.org/stamp/stamp.jsp?arnumber=6589136]]></pdf>
##   </document>
##   <document>
##     <rank>285</rank>
##     <title><![CDATA[RTT Prediction in Heavy Tailed Networks]]></title>
##     <authors><![CDATA[Rizo-Dominguez, L.;  Munoz-Rodriguez, D.;  Vargas-Rosales, C.;  Torres-Roman, D.;  Ramirez-Pacheco, J.]]></authors>
##     <affiliations><![CDATA[ITESO, Mexico]]></affiliations>
##     <controlledterms>
##       <term><![CDATA[Internet]]></term>
##       <term><![CDATA[delays]]></term>
##       <term><![CDATA[telecommunication traffic]]></term>
##       <term><![CDATA[transport protocols]]></term>
##     </controlledterms>
##     <thesaurusterms>
##       <term><![CDATA[Delays]]></term>
##       <term><![CDATA[Dispersion]]></term>
##       <term><![CDATA[Internet]]></term>
##       <term><![CDATA[Jacobian matrices]]></term>
##       <term><![CDATA[Jitter]]></term>
##       <term><![CDATA[Prediction algorithms]]></term>
##       <term><![CDATA[Quality of service]]></term>
##     </thesaurusterms>
##     <pubtitle><![CDATA[Communications Letters, IEEE]]></pubtitle>
##     <punumber><![CDATA[4234]]></punumber>
##     <pubtype><![CDATA[Journals & Magazines]]></pubtype>
##     <publisher><![CDATA[IEEE]]></publisher>
##     <volume><![CDATA[18]]></volume>
##     <issue><![CDATA[4]]></issue>
##     <py><![CDATA[2014]]></py>
##     <spage><![CDATA[700]]></spage>
##     <epage><![CDATA[703]]></epage>
##     <abstract><![CDATA[TCP is the most used transport protocol in the Internet and it relies on RTT (Round Trip Time) predictions for the retransmission control algorithm. Most of the algorithms reported in the literature consider memoryless traffic characteristics and do not study the performance under heavy tailed scenarios present in the Internet. In this paper, an algorithm for RTT prediction in a heavy-tailed environment is introduced, and it is shown to follow closely and accurately the actual RTT. The proposed algorithm is simple and permits online implementations. Results are compared with those obtained with other methodologies for real trace sets. It is shown that the proposed algorithm leads to a lower prediction error.]]></abstract>
##     <issn><![CDATA[1089-7798]]></issn>
##     <htmlFlag><![CDATA[1]]></htmlFlag>
##     <arnumber><![CDATA[6739048]]></arnumber>
##     <doi><![CDATA[10.1109/LCOMM.2014.013114.132668]]></doi>
##     <publicationId><![CDATA[6739048]]></publicationId>
##     <mdurl><![CDATA[http://ieeexplore.ieee.org/xpl/articleDetails.jsp?tp=&arnumber=6739048&contentType=Journals+%26+Magazines]]></mdurl>
##     <pdf><![CDATA[http://ieeexplore.ieee.org/stamp/stamp.jsp?arnumber=6739048]]></pdf>
##   </document>
##   <document>
##     <rank>286</rank>
##     <title><![CDATA[QuaDRiGa: A 3-D Multi-Cell Channel Model With Time Evolution for Enabling Virtual Field Trials]]></title>
##     <authors><![CDATA[Jaeckel, S.;  Raschkowski, L.;  Borner, K.;  Thiele, L.]]></authors>
##     <affiliations><![CDATA[Heinrich Hertz Inst., Fraunhofer Inst. for Telecommun., Berlin, Germany]]></affiliations>
##     <controlledterms>
##       <term><![CDATA[Long Term Evolution]]></term>
##       <term><![CDATA[antenna radiation patterns]]></term>
##       <term><![CDATA[computational complexity]]></term>
##       <term><![CDATA[electromagnetic wave propagation]]></term>
##       <term><![CDATA[mobile communication]]></term>
##     </controlledterms>
##     <thesaurusterms>
##       <term><![CDATA[Antennas]]></term>
##       <term><![CDATA[Biological system modeling]]></term>
##       <term><![CDATA[Channel models]]></term>
##       <term><![CDATA[Delays]]></term>
##       <term><![CDATA[Scattering]]></term>
##       <term><![CDATA[Solid modeling]]></term>
##     </thesaurusterms>
##     <pubtitle><![CDATA[Antennas and Propagation, IEEE Transactions on]]></pubtitle>
##     <punumber><![CDATA[8]]></punumber>
##     <pubtype><![CDATA[Journals & Magazines]]></pubtype>
##     <publisher><![CDATA[IEEE]]></publisher>
##     <volume><![CDATA[62]]></volume>
##     <issue><![CDATA[6]]></issue>
##     <py><![CDATA[2014]]></py>
##     <spage><![CDATA[3242]]></spage>
##     <epage><![CDATA[3256]]></epage>
##     <abstract><![CDATA[Channel models are important tools to evaluate the performance of new concepts in mobile communications. However, there is a tradeoff between complexity and accuracy. In this paper, we extend the popular Wireless World Initiative for New Radio (WINNER) channel model with new features to make it as realistic as possible. Our approach enables more realistic evaluation results at an early stage of algorithm development. The new model supports 3-D propagation, 3-D antenna patterns, time evolving channel traces of arbitrary length, scenario transitions and variable terminal speeds. We validated the model by measurements in a coherent LTE advanced testbed in downtown Berlin, Germany. We then reproduced the same scenario in the model and compared several channel parameters (delay spread, path gain, K-factor, geometry factor and capacity). The results match very well and we can accurately predict the performance for an urban macro-cell setup with commercial high-gain antennas. At the same time, the computational complexity does not increase significantly and we can use all existing WINNER parameter tables. These artificial channels, having equivalent characteristics as measured data, enable virtual field trials long before prototypes are available.]]></abstract>
##     <issn><![CDATA[0018-926X]]></issn>
##     <htmlFlag><![CDATA[1]]></htmlFlag>
##     <arnumber><![CDATA[6758357]]></arnumber>
##     <doi><![CDATA[10.1109/TAP.2014.2310220]]></doi>
##     <publicationId><![CDATA[6758357]]></publicationId>
##     <mdurl><![CDATA[http://ieeexplore.ieee.org/xpl/articleDetails.jsp?tp=&arnumber=6758357&contentType=Journals+%26+Magazines]]></mdurl>
##     <pdf><![CDATA[http://ieeexplore.ieee.org/stamp/stamp.jsp?arnumber=6758357]]></pdf>
##   </document>
##   <document>
##     <rank>287</rank>
##     <title><![CDATA[Impact of Detector Spatial Uniformity on the Measurement of Averaged LED Intensity]]></title>
##     <authors><![CDATA[Jian Liu;  Bao-zhou Zhang;  Hui Liu;  Shan-shan Zeng;  Wei-qiang Zhao;  Li-gen Lu]]></authors>
##     <affiliations><![CDATA[Dept. of Phys., Beijing Normal Univ., Beijing, China]]></affiliations>
##     <controlledterms>
##       <term><![CDATA[calibration]]></term>
##       <term><![CDATA[light emitting diodes]]></term>
##       <term><![CDATA[measurement errors]]></term>
##       <term><![CDATA[photodetectors]]></term>
##       <term><![CDATA[photometers]]></term>
##     </controlledterms>
##     <thesaurusterms>
##       <term><![CDATA[Apertures]]></term>
##       <term><![CDATA[Detectors]]></term>
##       <term><![CDATA[Light emitting diodes]]></term>
##       <term><![CDATA[Light sources]]></term>
##       <term><![CDATA[Optical filters]]></term>
##       <term><![CDATA[Semiconductor device measurement]]></term>
##       <term><![CDATA[Standards]]></term>
##     </thesaurusterms>
##     <pubtitle><![CDATA[Photonics Journal, IEEE]]></pubtitle>
##     <punumber><![CDATA[4563994]]></punumber>
##     <pubtype><![CDATA[Journals & Magazines]]></pubtype>
##     <publisher><![CDATA[IEEE]]></publisher>
##     <volume><![CDATA[6]]></volume>
##     <issue><![CDATA[1]]></issue>
##     <py><![CDATA[2014]]></py>
##     <spage><![CDATA[1]]></spage>
##     <epage><![CDATA[7]]></epage>
##     <abstract><![CDATA[Photometers are widely used for the measurement of the average LED intensity as defined by CIE 127, under the circumstance that LEDs have many kinds of intensity spatial distributions. Significant measurement errors can be resulted due to the following facts: 1) a photometer with a diffuser generally has a spatial response that is stronger in the center of the detector; and 2) LED angular distributions of intensity often have a sidelobe about the central axis rather than a central peak. A special measuring facility for illuminance intensity distribution of LEDs and spatial response distribution of photometers has been designed. An evaluation factor &#x201C; c&#x201D; has been introduced to describe the spatial response distribution uniformity of photometers. Three kinds of photometers with different spatial response distribution were used to measure the averaged LED intensity individually. Experiment shows that the photometer with a diffuser has a response distribution of a cosine function, and the illuminance intensity distribution of LEDs generally presents a sine, trapezoidal, or cosine function with the 6.5 <sup>&#x00B0;</sup> field of view for CIE Condition B. The results show that the measurement errors could be more than -5.10% when using the photometer with a diffuser to measure the average intensity of LEDs with a narrow beam angle. It is very important that the uniformity of spatial response distribution of the photometer should be considered when calibrating Averaged LED Intensity as it can introduce a significant error.]]></abstract>
##     <issn><![CDATA[1943-0655]]></issn>
##     <htmlFlag><![CDATA[1]]></htmlFlag>
##     <arnumber><![CDATA[6690231]]></arnumber>
##     <doi><![CDATA[10.1109/JPHOT.2013.2295458]]></doi>
##     <publicationId><![CDATA[6690231]]></publicationId>
##     <mdurl><![CDATA[http://ieeexplore.ieee.org/xpl/articleDetails.jsp?tp=&arnumber=6690231&contentType=Journals+%26+Magazines]]></mdurl>
##     <pdf><![CDATA[http://ieeexplore.ieee.org/stamp/stamp.jsp?arnumber=6690231]]></pdf>
##   </document>
##   <document>
##     <rank>288</rank>
##     <title><![CDATA[New Inference for Constant-Stress Accelerated Life Tests With Weibull Distribution and Progressively Type-II Censoring]]></title>
##     <authors><![CDATA[Bing Xing Wang;  Keming Yu;  Zhuo Sheng]]></authors>
##     <affiliations><![CDATA[Dept. of Stat., Zhejiang Gongshang Univ., Hangzhou, China]]></affiliations>
##     <controlledterms>
##       <term><![CDATA[Weibull distribution]]></term>
##       <term><![CDATA[life testing]]></term>
##       <term><![CDATA[reliability]]></term>
##       <term><![CDATA[statistical analysis]]></term>
##       <term><![CDATA[stress analysis]]></term>
##     </controlledterms>
##     <thesaurusterms>
##       <term><![CDATA[Life estimation]]></term>
##       <term><![CDATA[Maximum likelihood estimation]]></term>
##       <term><![CDATA[Random variables]]></term>
##       <term><![CDATA[Shape]]></term>
##       <term><![CDATA[Stress]]></term>
##       <term><![CDATA[Weibull distribution]]></term>
##     </thesaurusterms>
##     <pubtitle><![CDATA[Reliability, IEEE Transactions on]]></pubtitle>
##     <punumber><![CDATA[24]]></punumber>
##     <pubtype><![CDATA[Journals & Magazines]]></pubtype>
##     <publisher><![CDATA[IEEE]]></publisher>
##     <volume><![CDATA[63]]></volume>
##     <issue><![CDATA[3]]></issue>
##     <py><![CDATA[2014]]></py>
##     <spage><![CDATA[807]]></spage>
##     <epage><![CDATA[815]]></epage>
##     <abstract><![CDATA[Constant-stress procedures based on parametric lifetime distributions and models are often used for accelerated life testing in product reliability experiments. Maximum likelihood estimation (MLE) is the typical statistical inference method. This paper presents a new inference method, named the random variable transformation (RVT) method, for Weibull constant-stress accelerated life tests with progressively Type-II right censoring (including ordinary Type-II right censoring). A two-parameter Weibull life distribution with a scale parameter that is a log-linear function of stress is used. RVT inference life distribution parameters and the log-linear function coefficients are provided. Exact confidence intervals for these parameters are also explored. Numerical comparisons of RVT-based estimates to MLE show that the proposed RVT inference is promising, in particular for small sample sizes.]]></abstract>
##     <issn><![CDATA[0018-9529]]></issn>
##     <htmlFlag><![CDATA[1]]></htmlFlag>
##     <arnumber><![CDATA[6782725]]></arnumber>
##     <doi><![CDATA[10.1109/TR.2014.2313804]]></doi>
##     <publicationId><![CDATA[6782725]]></publicationId>
##     <mdurl><![CDATA[http://ieeexplore.ieee.org/xpl/articleDetails.jsp?tp=&arnumber=6782725&contentType=Journals+%26+Magazines]]></mdurl>
##     <pdf><![CDATA[http://ieeexplore.ieee.org/stamp/stamp.jsp?arnumber=6782725]]></pdf>
##   </document>
##   <document>
##     <rank>289</rank>
##     <title><![CDATA[First Demonstration of Ultra-Thin SiGe-Channel Junctionless Accumulation-Mode (JAM) Bulk FinFETs on Si Substrate with PN Junction-Isolation Scheme]]></title>
##     <authors><![CDATA[Dong-Hyun Kim;  Tae Kyun Kim;  Young Gwang Yoon;  Byeong-Woon Hwang;  Yang-Kyu Choi;  Byung Jin Cho;  Seok-Hee Lee]]></authors>
##     <affiliations><![CDATA[Dept. of Electr. Eng., Korea Adv. Inst. of Sci. & Technol., Daejeon, South Korea]]></affiliations>
##     <controlledterms>
##       <term><![CDATA[Ge-Si alloys]]></term>
##       <term><![CDATA[MOSFET]]></term>
##       <term><![CDATA[elemental semiconductors]]></term>
##       <term><![CDATA[isolation technology]]></term>
##       <term><![CDATA[p-n junctions]]></term>
##       <term><![CDATA[silicon]]></term>
##     </controlledterms>
##     <thesaurusterms>
##       <term><![CDATA[FinFETs]]></term>
##       <term><![CDATA[Licenses]]></term>
##       <term><![CDATA[Logic gates]]></term>
##       <term><![CDATA[Silicon]]></term>
##       <term><![CDATA[Silicon germanium]]></term>
##       <term><![CDATA[Substrates]]></term>
##     </thesaurusterms>
##     <pubtitle><![CDATA[Electron Devices Society, IEEE Journal of the]]></pubtitle>
##     <punumber><![CDATA[6245494]]></punumber>
##     <pubtype><![CDATA[Journals & Magazines]]></pubtype>
##     <publisher><![CDATA[IEEE]]></publisher>
##     <volume><![CDATA[2]]></volume>
##     <issue><![CDATA[5]]></issue>
##     <py><![CDATA[2014]]></py>
##     <spage><![CDATA[123]]></spage>
##     <epage><![CDATA[127]]></epage>
##     <abstract><![CDATA[A SiGe-channel junctionless-accumulation-mode (JAM) PMOS bulk FinFETs were successfully demonstrated on Si substrate with PN junction-isolation scheme for the first time. The JAM bulk FinFETs with fin width of 18 nm exhibits excellent subthreshold characteristics such as subthreshold swing of 64 mV/decade, drain-induced barrier lowering (DIBL) of 40 mV/V and high I<sub>on</sub>/I<sub>off</sub> current ratio (&gt;1 &#x00D7; 10<sup>5</sup>). The change of substrate bias from 0 to 5 V leads to the threshold voltage shift of 53 mV by modulating the effective channel thickness. When compared to the Si-channel bulk FinFETs with fin width of 18 nm, Si and SiGe channel devices exhibits comparable subthreshold swing and DIBL. For devices with longer fin width, SiGe channel devices exhibits much lower DIBL, indicating superior top-gate controllability and robustness to substrate bias compared to the Si channel devices. A zero temperature coefficient point was observed in the transfer curves as temperature increases from -120 to 120&#x00B0;C, confirming that mobility degradation is dominantly affected by phonon scattering mechanism.]]></abstract>
##     <issn><![CDATA[2168-6734]]></issn>
##     <htmlFlag><![CDATA[1]]></htmlFlag>
##     <arnumber><![CDATA[6819776]]></arnumber>
##     <doi><![CDATA[10.1109/JEDS.2014.2326560]]></doi>
##     <publicationId><![CDATA[6819776]]></publicationId>
##     <mdurl><![CDATA[http://ieeexplore.ieee.org/xpl/articleDetails.jsp?tp=&arnumber=6819776&contentType=Journals+%26+Magazines]]></mdurl>
##     <pdf><![CDATA[http://ieeexplore.ieee.org/stamp/stamp.jsp?arnumber=6819776]]></pdf>
##   </document>
##   <document>
##     <rank>290</rank>
##     <title><![CDATA[The HelioClim-1 Database of Daily Solar Radiation at Earth Surface: An Example of the Benefits of GEOSS Data-CORE]]></title>
##     <authors><![CDATA[Lefevre, M.;  Blanc, P.;  Espinar, B.;  Gschwind, B.;  Menard, L.;  Ranchin, T.;  Wald, L.;  Saboret, L.;  Thomas, C.;  Wey, E.]]></authors>
##     <affiliations><![CDATA[Center Obs., Impacts, Energy, Mines ParisTech, Sophia Antipolis, France]]></affiliations>
##     <controlledterms>
##       <term><![CDATA[atmospheric radiation]]></term>
##       <term><![CDATA[atmospheric techniques]]></term>
##       <term><![CDATA[geographic information systems]]></term>
##     </controlledterms>
##     <thesaurusterms>
##       <term><![CDATA[Africa]]></term>
##       <term><![CDATA[Databases]]></term>
##       <term><![CDATA[Europe]]></term>
##       <term><![CDATA[Global Earth Observation System of Systems]]></term>
##       <term><![CDATA[Meteorology]]></term>
##       <term><![CDATA[Solar radiation]]></term>
##     </thesaurusterms>
##     <pubtitle><![CDATA[Selected Topics in Applied Earth Observations and Remote Sensing, IEEE Journal of]]></pubtitle>
##     <punumber><![CDATA[4609443]]></punumber>
##     <pubtype><![CDATA[Journals & Magazines]]></pubtype>
##     <publisher><![CDATA[IEEE]]></publisher>
##     <volume><![CDATA[7]]></volume>
##     <issue><![CDATA[5]]></issue>
##     <py><![CDATA[2014]]></py>
##     <spage><![CDATA[1745]]></spage>
##     <epage><![CDATA[1753]]></epage>
##     <abstract><![CDATA[The HelioClim-1 database contains daily values of the solar radiation reaching the ground. This GEOSS (Global Earth Observation System of Systems) Data Collection of Open Resources for Everyone (Data-CORE) covers Europe, Africa and the Atlantic Ocean, from 1985 to 2005. It is freely accessible at no cost through the SoDa Service (www.soda-is.com). Several assessments of the HelioClim-1 data against measurements made in meteorological networks reveal that the HelioClim-1 database offers a reliable and accurate knowledge of the solar radiation and its daily, seasonal and annual variations over recent years. The HelioClim-1 data may help in qualifying in situ measurements and may supplement them, thus offering 21 years of accurate daily means of surface solar irradiance. Several published works benefited from openness, availability and accuracy of the HelioClim-1 database in various domains: oceanography, climate, energy production, life cycle analysis, agriculture, forestry, architecture, health and air quality. This demonstration of the benefit of the HelioClim-1 database draws attention to resources open to everyone such as those labeled GEOSS Data-CORE.]]></abstract>
##     <issn><![CDATA[1939-1404]]></issn>
##     <htmlFlag><![CDATA[1]]></htmlFlag>
##     <arnumber><![CDATA[6631471]]></arnumber>
##     <doi><![CDATA[10.1109/JSTARS.2013.2283791]]></doi>
##     <publicationId><![CDATA[6631471]]></publicationId>
##     <mdurl><![CDATA[http://ieeexplore.ieee.org/xpl/articleDetails.jsp?tp=&arnumber=6631471&contentType=Journals+%26+Magazines]]></mdurl>
##     <pdf><![CDATA[http://ieeexplore.ieee.org/stamp/stamp.jsp?arnumber=6631471]]></pdf>
##   </document>
##   <document>
##     <rank>291</rank>
##     <title><![CDATA[Improvements on Remote Diffuser-Phosphor-Packaged Light-Emitting Diode Systems]]></title>
##     <authors><![CDATA[Hua Xiao;  Yi-Jun Lu;  Tien-Mo Shih;  Li-Hong Zhu;  Si-Qi Lin;  Pagni, P.J.;  Zhong Chen]]></authors>
##     <affiliations><![CDATA[Dept. of Electron. Sci., Xiamen Univ., Xiamen, China]]></affiliations>
##     <controlledterms>
##       <term><![CDATA[Monte Carlo methods]]></term>
##       <term><![CDATA[light emitting diodes]]></term>
##       <term><![CDATA[phosphors]]></term>
##     </controlledterms>
##     <thesaurusterms>
##       <term><![CDATA[Educational institutions]]></term>
##       <term><![CDATA[Light emitting diodes]]></term>
##       <term><![CDATA[Optical reflection]]></term>
##       <term><![CDATA[Optical sensors]]></term>
##       <term><![CDATA[Phosphors]]></term>
##       <term><![CDATA[Three-dimensional displays]]></term>
##     </thesaurusterms>
##     <pubtitle><![CDATA[Photonics Journal, IEEE]]></pubtitle>
##     <punumber><![CDATA[4563994]]></punumber>
##     <pubtype><![CDATA[Journals & Magazines]]></pubtype>
##     <publisher><![CDATA[IEEE]]></publisher>
##     <volume><![CDATA[6]]></volume>
##     <issue><![CDATA[2]]></issue>
##     <py><![CDATA[2014]]></py>
##     <spage><![CDATA[1]]></spage>
##     <epage><![CDATA[8]]></epage>
##     <abstract><![CDATA[By modifying traditional remote phosphor-diffuser-packaged light-emitting diode systems, we have managed to increase the luminous efficacy from 145.7 to 162.3 lm/W. One mechanism responsible for this achievement is associated with randomizing the directions of light beams transmitting through an interior diffuser, whose position is optimized based on an overall merit. The other mechanism is identified as the gradual attenuation of the undesirable blue-ring image along the distance from the diffuser toward the phosphor base. The identification of these two mechanisms is verified by luminous efficacy measurements, 3-D image plots, and combined Monte Carlo algorithm ray tracing simulation results. In addition, merits related to correlated color temperatures and issues pertaining to costs are briefly discussed.]]></abstract>
##     <issn><![CDATA[1943-0655]]></issn>
##     <htmlFlag><![CDATA[1]]></htmlFlag>
##     <arnumber><![CDATA[6678195]]></arnumber>
##     <doi><![CDATA[10.1109/JPHOT.2013.2293612]]></doi>
##     <publicationId><![CDATA[6678195]]></publicationId>
##     <mdurl><![CDATA[http://ieeexplore.ieee.org/xpl/articleDetails.jsp?tp=&arnumber=6678195&contentType=Journals+%26+Magazines]]></mdurl>
##     <pdf><![CDATA[http://ieeexplore.ieee.org/stamp/stamp.jsp?arnumber=6678195]]></pdf>
##   </document>
##   <document>
##     <rank>292</rank>
##     <title><![CDATA[Wheel Torque Distribution Criteria for Electric Vehicles With Torque-Vectoring Differentials]]></title>
##     <authors><![CDATA[De Novellis, L.;  Sorniotti, A.;  Gruber, P.]]></authors>
##     <affiliations><![CDATA[Dept. of Mech. Eng. Sci., Univ. of Surrey, Guildford, UK]]></affiliations>
##     <controlledterms>
##       <term><![CDATA[braking]]></term>
##       <term><![CDATA[electric vehicles]]></term>
##       <term><![CDATA[optimisation]]></term>
##       <term><![CDATA[torque]]></term>
##       <term><![CDATA[torque control]]></term>
##       <term><![CDATA[wheels]]></term>
##     </controlledterms>
##     <thesaurusterms>
##       <term><![CDATA[Linear programming]]></term>
##       <term><![CDATA[Optimization]]></term>
##       <term><![CDATA[TV]]></term>
##       <term><![CDATA[Tires]]></term>
##       <term><![CDATA[Torque]]></term>
##       <term><![CDATA[Vehicles]]></term>
##       <term><![CDATA[Wheels]]></term>
##     </thesaurusterms>
##     <pubtitle><![CDATA[Vehicular Technology, IEEE Transactions on]]></pubtitle>
##     <punumber><![CDATA[25]]></punumber>
##     <pubtype><![CDATA[Journals & Magazines]]></pubtype>
##     <publisher><![CDATA[IEEE]]></publisher>
##     <volume><![CDATA[63]]></volume>
##     <issue><![CDATA[4]]></issue>
##     <py><![CDATA[2014]]></py>
##     <spage><![CDATA[1593]]></spage>
##     <epage><![CDATA[1602]]></epage>
##     <abstract><![CDATA[The continuous and precise modulation of the driving and braking torques of each wheel is considered the ultimate goal for controlling the performance of a vehicle in steady-state and transient conditions. To do so, dedicated torque-vectoring (TV) controllers that allow optimal wheel torque distribution under all possible driving conditions have to be developed. Commonly, vehicle TV controllers are based on a hierarchical approach, consisting of a high-level supervisory controller that evaluates a corrective yaw moment and a low-level controller that defines the individual wheel torque reference values. The problem of the optimal individual wheel torque distribution for a particular driving condition can be solved through an optimization-based control-allocation (CA) algorithm, which must rely on the appropriate selection of the objective function. With a newly developed offline optimization procedure, this paper assesses the performance of alternative objective functions for the optimal wheel torque distribution of a four-wheel-drive (4WD) fully electric vehicle. Results show that objective functions based on the minimum tire slip criterion provide better control performance than functions based on energy efficiency.]]></abstract>
##     <issn><![CDATA[0018-9545]]></issn>
##     <arnumber><![CDATA[6656947]]></arnumber>
##     <doi><![CDATA[10.1109/TVT.2013.2289371]]></doi>
##     <publicationId><![CDATA[6656947]]></publicationId>
##     <mdurl><![CDATA[http://ieeexplore.ieee.org/xpl/articleDetails.jsp?tp=&arnumber=6656947&contentType=Journals+%26+Magazines]]></mdurl>
##     <pdf><![CDATA[http://ieeexplore.ieee.org/stamp/stamp.jsp?arnumber=6656947]]></pdf>
##   </document>
##   <document>
##     <rank>293</rank>
##     <title><![CDATA[Congestion Aware Routing in Nonlinear Elastic Optical Networks]]></title>
##     <authors><![CDATA[Savory, S.J.]]></authors>
##     <affiliations><![CDATA[Dept. of Electron. & Electr. Eng., Univ. Coll. London, London, UK]]></affiliations>
##     <controlledterms>
##       <term><![CDATA[error statistics]]></term>
##       <term><![CDATA[nonlinear optics]]></term>
##       <term><![CDATA[optical fibre networks]]></term>
##       <term><![CDATA[optical modulation]]></term>
##       <term><![CDATA[optical transceivers]]></term>
##       <term><![CDATA[telecommunication network routing]]></term>
##     </controlledterms>
##     <thesaurusterms>
##       <term><![CDATA[Fiber nonlinear optics]]></term>
##       <term><![CDATA[Optical fiber amplifiers]]></term>
##       <term><![CDATA[Optical fiber networks]]></term>
##       <term><![CDATA[Routing]]></term>
##       <term><![CDATA[Signal to noise ratio]]></term>
##     </thesaurusterms>
##     <pubtitle><![CDATA[Photonics Technology Letters, IEEE]]></pubtitle>
##     <punumber><![CDATA[68]]></punumber>
##     <pubtype><![CDATA[Journals & Magazines]]></pubtype>
##     <publisher><![CDATA[IEEE]]></publisher>
##     <volume><![CDATA[26]]></volume>
##     <issue><![CDATA[10]]></issue>
##     <py><![CDATA[2014]]></py>
##     <spage><![CDATA[1057]]></spage>
##     <epage><![CDATA[1060]]></epage>
##     <abstract><![CDATA[In elastic optical networks, digital coherent transceivers modify their symbol rate, modulation format, and forward error correction to best serve the network demands. In a nonlinear elastic optical network, these parameters are inherently coupled with the routing algorithm. We propose to use congestion aware routing in a nonlinear elastic optical network and demonstrate its efficacy for the NSFNET reference network (14 nodes, 22 links). The network is sequentially loaded with 100 GbE demands until a demand becomes blocked, this procedure being repeated 10000 times to estimate the network blocking probability (NBP). Three routing algorithms are considered: 1) shortest path routing; 2) simple congestion aware algorithm; and 3) weighted congestion aware routing algorithm with 50, 25, 12.5, and 6.25 GHz resolution flexgrids. For NBP = 1% using a 50 GHz grid, congestion aware routing doubles the network capacity compared with the shortest path routing. When congestion aware routing is combined with a 6.25 GHz resolution flexgrid, a fivefold increase in network capacity is afforded.]]></abstract>
##     <issn><![CDATA[1041-1135]]></issn>
##     <htmlFlag><![CDATA[1]]></htmlFlag>
##     <arnumber><![CDATA[6781032]]></arnumber>
##     <doi><![CDATA[10.1109/LPT.2014.2314438]]></doi>
##     <publicationId><![CDATA[6781032]]></publicationId>
##     <mdurl><![CDATA[http://ieeexplore.ieee.org/xpl/articleDetails.jsp?tp=&arnumber=6781032&contentType=Journals+%26+Magazines]]></mdurl>
##     <pdf><![CDATA[http://ieeexplore.ieee.org/stamp/stamp.jsp?arnumber=6781032]]></pdf>
##   </document>
##   <document>
##     <rank>294</rank>
##     <title><![CDATA[Challenges and Opportunities of Underwater Cognitive Acoustic Networks]]></title>
##     <authors><![CDATA[Yu Luo;  Pu, L.;  Zuba, M.;  Zheng Peng;  Jun-Hong Cui]]></authors>
##     <affiliations><![CDATA[Dept. of Comput. Sci. & Eng., Univ. of Connecticut, Storrs, CT, USA]]></affiliations>
##     <controlledterms>
##       <term><![CDATA[acoustic signal processing]]></term>
##       <term><![CDATA[cognitive radio]]></term>
##       <term><![CDATA[radio spectrum management]]></term>
##       <term><![CDATA[underwater acoustic communication]]></term>
##     </controlledterms>
##     <thesaurusterms>
##       <term><![CDATA[Acoustic devices]]></term>
##       <term><![CDATA[Green's function methods]]></term>
##       <term><![CDATA[Impedance]]></term>
##       <term><![CDATA[Integral equations]]></term>
##       <term><![CDATA[Interference]]></term>
##       <term><![CDATA[Sonar navigation]]></term>
##       <term><![CDATA[Underwater acoustics]]></term>
##     </thesaurusterms>
##     <pubtitle><![CDATA[Emerging Topics in Computing, IEEE Transactions on]]></pubtitle>
##     <punumber><![CDATA[6245516]]></punumber>
##     <pubtype><![CDATA[Journals & Magazines]]></pubtype>
##     <publisher><![CDATA[IEEE]]></publisher>
##     <volume><![CDATA[2]]></volume>
##     <issue><![CDATA[2]]></issue>
##     <py><![CDATA[2014]]></py>
##     <spage><![CDATA[198]]></spage>
##     <epage><![CDATA[211]]></epage>
##     <abstract><![CDATA[In oceans, both the natural acoustic systems (such as marine mammals) and artificial acoustic systems [like underwater acoustic networks (UANs) and sonar users] use acoustic signal for communication, echolocation, sensing, and detection. This makes the channel spectrum heavily shared by various underwater acoustic systems. Nevertheless, the precious spectrum resource is still underutilized temporally and spatially in underwater environments. To efficiently utilize the spectrum while avoiding harmful interference with other acoustic systems, a smart UAN should be aware of the surrounding environment and reconfigure their operation parameters. Unfortunately, existing UAN designs have mainly focused on the single network scenario, and very few studies have considered the presence of nearby acoustic activities. In this paper, we advocate cognitive acoustic as a promising technique to develop an environment-friendly UAN with high spectrum utilization. However, underwater cognitive acoustic networks (UCANs) also pose grand challenges due to the unique features of underwater channel and acoustic systems. In this paper, we comprehensively investigate these unique characteristics and their impact on the UCAN design. Finally, possible solutions to tackle such challenges are advocated.]]></abstract>
##     <issn><![CDATA[2168-6750]]></issn>
##     <htmlFlag><![CDATA[1]]></htmlFlag>
##     <arnumber><![CDATA[6763113]]></arnumber>
##     <doi><![CDATA[10.1109/TETC.2014.2310457]]></doi>
##     <publicationId><![CDATA[6763113]]></publicationId>
##     <mdurl><![CDATA[http://ieeexplore.ieee.org/xpl/articleDetails.jsp?tp=&arnumber=6763113&contentType=Journals+%26+Magazines]]></mdurl>
##     <pdf><![CDATA[http://ieeexplore.ieee.org/stamp/stamp.jsp?arnumber=6763113]]></pdf>
##   </document>
##   <document>
##     <rank>295</rank>
##     <title><![CDATA[Microwave System for the Early Stage Detection of Congestive Heart Failure]]></title>
##     <authors><![CDATA[Rezaeieh, S.A.;  Bialkowski, K.S.;  Abbosh, A.M.]]></authors>
##     <affiliations><![CDATA[Sch. of Inf. Technol. & Electr. Eng., Univ. of Queensland, Brisbane, QLD, Australia]]></affiliations>
##     <controlledterms>
##       <term><![CDATA[Fourier transforms]]></term>
##       <term><![CDATA[UHF devices]]></term>
##       <term><![CDATA[bioelectric phenomena]]></term>
##       <term><![CDATA[biological tissues]]></term>
##       <term><![CDATA[biomedical equipment]]></term>
##       <term><![CDATA[broadband antennas]]></term>
##       <term><![CDATA[cardiology]]></term>
##       <term><![CDATA[data acquisition]]></term>
##       <term><![CDATA[data visualisation]]></term>
##       <term><![CDATA[hardware-software codesign]]></term>
##       <term><![CDATA[image classification]]></term>
##       <term><![CDATA[laptop computers]]></term>
##       <term><![CDATA[lung]]></term>
##       <term><![CDATA[medical disorders]]></term>
##       <term><![CDATA[medical image processing]]></term>
##       <term><![CDATA[microwave imaging]]></term>
##       <term><![CDATA[time-domain analysis]]></term>
##       <term><![CDATA[transceivers]]></term>
##     </controlledterms>
##     <thesaurusterms>
##       <term><![CDATA[Biomedical image processing]]></term>
##       <term><![CDATA[Cardiac arrest]]></term>
##       <term><![CDATA[Fluids]]></term>
##       <term><![CDATA[Lungs]]></term>
##       <term><![CDATA[Microwave antennas]]></term>
##       <term><![CDATA[Microwave imaging]]></term>
##       <term><![CDATA[Microwave theory and techniques]]></term>
##     </thesaurusterms>
##     <pubtitle><![CDATA[Access, IEEE]]></pubtitle>
##     <punumber><![CDATA[6287639]]></punumber>
##     <pubtype><![CDATA[Journals & Magazines]]></pubtype>
##     <publisher><![CDATA[IEEE]]></publisher>
##     <volume><![CDATA[2]]></volume>
##     <py><![CDATA[2014]]></py>
##     <spage><![CDATA[921]]></spage>
##     <epage><![CDATA[929]]></epage>
##     <abstract><![CDATA[Fluid accumulation inside the lungs, known as cardiac pulmonary edema, is one of the main early symptoms of congestive heart failure (CHF). That accumulation causes significant changes in the electrical properties of the lung tissues, which in turn can be detected using microwave techniques. To that end, the design and implementation of an automated ultrahigh-frequency microwave-based system for CHF detection and monitoring is presented. The hardware of the system consists of a wideband folded antenna attached to a fully automated vertical scanning platform, compact microwave transceiver, and laptop. The system includes software in the form of operational control, signal processing, and visualizing algorithms. To detect CHF, the system is designed to vertically scan the rear side of the human torso in a monostatic radar approach. The collected data from the scanning is then visualized in the time domain using the inverse Fourier transform. These images show the intensity of the reflected signals from different parts of the torso. Using a differential based detection technique, a threshold is defined to differentiate between healthy and unhealthy cases. This paper includes details of developing the automated platform, designing the antenna with the required properties imposed by the system, developing a signal processing algorithm, and introducing differential detection technique besides investigating miscellaneous probable CHF cases.]]></abstract>
##     <issn><![CDATA[2169-3536]]></issn>
##     <htmlFlag><![CDATA[1]]></htmlFlag>
##     <arnumber><![CDATA[6884768]]></arnumber>
##     <doi><![CDATA[10.1109/ACCESS.2014.2352614]]></doi>
##     <publicationId><![CDATA[6884768]]></publicationId>
##     <mdurl><![CDATA[http://ieeexplore.ieee.org/xpl/articleDetails.jsp?tp=&arnumber=6884768&contentType=Journals+%26+Magazines]]></mdurl>
##     <pdf><![CDATA[http://ieeexplore.ieee.org/stamp/stamp.jsp?arnumber=6884768]]></pdf>
##   </document>
##   <document>
##     <rank>296</rank>
##     <title><![CDATA[Real-Time Waveform Characterization by Using Frequency-Resolved Optical Gating Capable of Carrier-Envelope Phase Determination]]></title>
##     <authors><![CDATA[Shirai, H.;  Nomura, Y.;  Fuji, T.]]></authors>
##     <affiliations><![CDATA[Inst. for Mol. Sci., Okazaki, Japan]]></affiliations>
##     <controlledterms>
##       <term><![CDATA[electro-optical effects]]></term>
##       <term><![CDATA[high-speed optical techniques]]></term>
##       <term><![CDATA[waveform analysis]]></term>
##     </controlledterms>
##     <thesaurusterms>
##       <term><![CDATA[Delays]]></term>
##       <term><![CDATA[Earth Observing System]]></term>
##       <term><![CDATA[Imaging]]></term>
##       <term><![CDATA[Optical distortion]]></term>
##       <term><![CDATA[Pulse measurements]]></term>
##       <term><![CDATA[Real-time systems]]></term>
##       <term><![CDATA[Ultraviolet sources]]></term>
##     </thesaurusterms>
##     <pubtitle><![CDATA[Photonics Journal, IEEE]]></pubtitle>
##     <punumber><![CDATA[4563994]]></punumber>
##     <pubtype><![CDATA[Journals & Magazines]]></pubtype>
##     <publisher><![CDATA[IEEE]]></publisher>
##     <volume><![CDATA[6]]></volume>
##     <issue><![CDATA[3]]></issue>
##     <py><![CDATA[2014]]></py>
##     <spage><![CDATA[1]]></spage>
##     <epage><![CDATA[12]]></epage>
##     <abstract><![CDATA[We present real-time waveform characterization of ultrashort pulses by the use of a pulse-front tilt in frequency-resolved optical gating capable of carrier-envelope phase determination. Simultaneous measurement of cross-correlation frequency-resolved optical gating and electro-optic sampling signals has been realized without scanning any delays. Complete waveform characterization of single-cycle infrared pulses in real time has been demonstrated. The method has been also applied for characterization of more complex pulses, namely, a strongly chirped pulse after passing through a solid crystal or a distorted pulse due to the absorption of ambient air.]]></abstract>
##     <issn><![CDATA[1943-0655]]></issn>
##     <htmlFlag><![CDATA[1]]></htmlFlag>
##     <arnumber><![CDATA[6803855]]></arnumber>
##     <doi><![CDATA[10.1109/JPHOT.2014.2319091]]></doi>
##     <publicationId><![CDATA[6803855]]></publicationId>
##     <mdurl><![CDATA[http://ieeexplore.ieee.org/xpl/articleDetails.jsp?tp=&arnumber=6803855&contentType=Journals+%26+Magazines]]></mdurl>
##     <pdf><![CDATA[http://ieeexplore.ieee.org/stamp/stamp.jsp?arnumber=6803855]]></pdf>
##   </document>
##   <document>
##     <rank>297</rank>
##     <title><![CDATA[Integrated Optical Chemical Sensor Based on an SOI Ring Resonator Using Phase-Interrogation]]></title>
##     <authors><![CDATA[Jin Liu;  Xi Zhou;  Zhi Qiao;  Jianhao Zhang;  Chenzhao Zhang;  Tuowen Xiang;  Lingling Shui;  Yaocheng Shi;  Liu Liu]]></authors>
##     <affiliations><![CDATA[Centre for Opt. & Electromagn. Res., South China Normal Univ., Guangzhou, China]]></affiliations>
##     <controlledterms>
##       <term><![CDATA[chemical sensors]]></term>
##       <term><![CDATA[integrated optics]]></term>
##       <term><![CDATA[optical sensors]]></term>
##       <term><![CDATA[phase measurement]]></term>
##       <term><![CDATA[refractive index measurement]]></term>
##       <term><![CDATA[resonators]]></term>
##       <term><![CDATA[silicon-on-insulator]]></term>
##     </controlledterms>
##     <thesaurusterms>
##       <term><![CDATA[Measurement by laser beam]]></term>
##       <term><![CDATA[Optical ring resonators]]></term>
##       <term><![CDATA[Optical sensors]]></term>
##       <term><![CDATA[Optical waveguides]]></term>
##       <term><![CDATA[Ring lasers]]></term>
##       <term><![CDATA[Wavelength measurement]]></term>
##     </thesaurusterms>
##     <pubtitle><![CDATA[Photonics Journal, IEEE]]></pubtitle>
##     <punumber><![CDATA[4563994]]></punumber>
##     <pubtype><![CDATA[Journals & Magazines]]></pubtype>
##     <publisher><![CDATA[IEEE]]></publisher>
##     <volume><![CDATA[6]]></volume>
##     <issue><![CDATA[5]]></issue>
##     <py><![CDATA[2014]]></py>
##     <spage><![CDATA[1]]></spage>
##     <epage><![CDATA[7]]></epage>
##     <abstract><![CDATA[A phase-interrogation approach for the bulk refractive index sensing based on a silicon-on-insulator (SOI) ring resonator is introduced. The rapid phase variation around the resonance of the ring resonator is interrogated, and a single-sideband generation and coherent detection technology is adopted for the phase measurement. In the proposed approach, most of the intensity noise can be shielded, which leads to an ultra-stable reading for the phase signals. A sensitivity of 6 &#x00D7; 10<sup>3</sup> rad/RIU and a detection limit of 2.5 &#x00D7; 10<sup>-6</sup> RIU are demonstrated.]]></abstract>
##     <issn><![CDATA[1943-0655]]></issn>
##     <htmlFlag><![CDATA[1]]></htmlFlag>
##     <arnumber><![CDATA[6892939]]></arnumber>
##     <doi><![CDATA[10.1109/JPHOT.2014.2352973]]></doi>
##     <publicationId><![CDATA[6892939]]></publicationId>
##     <mdurl><![CDATA[http://ieeexplore.ieee.org/xpl/articleDetails.jsp?tp=&arnumber=6892939&contentType=Journals+%26+Magazines]]></mdurl>
##     <pdf><![CDATA[http://ieeexplore.ieee.org/stamp/stamp.jsp?arnumber=6892939]]></pdf>
##   </document>
##   <document>
##     <rank>298</rank>
##     <title><![CDATA[Constrained Turbo Block Convolutional Codes for 100 G and Beyond Optical Transmissions]]></title>
##     <authors><![CDATA[Sang Ik Han;  Fonseka, J.P.;  Dowling, E.M.]]></authors>
##     <affiliations><![CDATA[Dept. of Electr. Eng., Univ. of Texas at Dallas, Richardson, TX, USA]]></affiliations>
##     <controlledterms>
##       <term><![CDATA[Hamming codes]]></term>
##       <term><![CDATA[block codes]]></term>
##       <term><![CDATA[concatenated codes]]></term>
##       <term><![CDATA[convolutional codes]]></term>
##       <term><![CDATA[decoding]]></term>
##       <term><![CDATA[interleaved codes]]></term>
##       <term><![CDATA[optical fibre networks]]></term>
##       <term><![CDATA[turbo codes]]></term>
##     </controlledterms>
##     <thesaurusterms>
##       <term><![CDATA[Complexity theory]]></term>
##       <term><![CDATA[Convolutional codes]]></term>
##       <term><![CDATA[Decoding]]></term>
##       <term><![CDATA[Forward error correction]]></term>
##       <term><![CDATA[Iterative decoding]]></term>
##       <term><![CDATA[Magnetohydrodynamics]]></term>
##     </thesaurusterms>
##     <pubtitle><![CDATA[Photonics Technology Letters, IEEE]]></pubtitle>
##     <punumber><![CDATA[68]]></punumber>
##     <pubtype><![CDATA[Journals & Magazines]]></pubtype>
##     <publisher><![CDATA[IEEE]]></publisher>
##     <volume><![CDATA[26]]></volume>
##     <issue><![CDATA[10]]></issue>
##     <py><![CDATA[2014]]></py>
##     <spage><![CDATA[995]]></spage>
##     <epage><![CDATA[998]]></epage>
##     <abstract><![CDATA[Constrained turbo block convolutional (CTBC) codes are developed for 100 G and beyond optical transmissions. The CTBC codes developed herein each fit within one optical transport network (OTN) frame. The CTBC codes involve a simple outer block code that is serially concatenated with a simple inner recursive convolutional code using a constrained interleaver that simultaneously delivers a high interleaver gain and a high minimum Hamming distance. Codes with 11.1 dB net coding gain (NCG) at 12.5% overhead (OH), 11.3 dB NCG at 15% OH, 11.6 dB NCG at 20% OH, and 11.9 dB NCG at 23.4% OH are reported. Compared with other codes that have been previously proposed for OTN applications, CTBC codes have much lower encoding/decoding complexity, improved NCG/OH tradeoffs, and avoid negative error floor effects.]]></abstract>
##     <issn><![CDATA[1041-1135]]></issn>
##     <arnumber><![CDATA[6767056]]></arnumber>
##     <doi><![CDATA[10.1109/LPT.2014.2311998]]></doi>
##     <publicationId><![CDATA[6767056]]></publicationId>
##     <mdurl><![CDATA[http://ieeexplore.ieee.org/xpl/articleDetails.jsp?tp=&arnumber=6767056&contentType=Journals+%26+Magazines]]></mdurl>
##     <pdf><![CDATA[http://ieeexplore.ieee.org/stamp/stamp.jsp?arnumber=6767056]]></pdf>
##   </document>
##   <document>
##     <rank>299</rank>
##     <title><![CDATA[Physical Origins and Analysis of Negative-Bias Stress Instability Mechanism in Polymer-Based Thin-Film Transistors]]></title>
##     <authors><![CDATA[Jaewook Lee;  Jaeman Jang;  Hyeongjung Kim;  Jiyoul Lee;  Bang-Lin Lee;  Sung-Jin Choi;  Dong Myong Kim;  Dae Hwan Kim;  Kyung Rok Kim]]></authors>
##     <affiliations><![CDATA[Sch. of Electr. Eng., Kookmin Univ., Seoul, South Korea]]></affiliations>
##     <controlledterms>
##       <term><![CDATA[Schottky barriers]]></term>
##       <term><![CDATA[technology CAD (electronics)]]></term>
##       <term><![CDATA[thin film transistors]]></term>
##     </controlledterms>
##     <thesaurusterms>
##       <term><![CDATA[Charge carrier processes]]></term>
##       <term><![CDATA[Degradation]]></term>
##       <term><![CDATA[Logic gates]]></term>
##       <term><![CDATA[NIST]]></term>
##       <term><![CDATA[Polymers]]></term>
##       <term><![CDATA[Stress]]></term>
##       <term><![CDATA[Transistors]]></term>
##     </thesaurusterms>
##     <pubtitle><![CDATA[Electron Device Letters, IEEE]]></pubtitle>
##     <punumber><![CDATA[55]]></punumber>
##     <pubtype><![CDATA[Journals & Magazines]]></pubtype>
##     <publisher><![CDATA[IEEE]]></publisher>
##     <volume><![CDATA[35]]></volume>
##     <issue><![CDATA[3]]></issue>
##     <py><![CDATA[2014]]></py>
##     <spage><![CDATA[396]]></spage>
##     <epage><![CDATA[398]]></epage>
##     <abstract><![CDATA[The physical origins of the negative-bias stress (NBS) instability in polymer-based thin-film transistors have been characterized. Through the quantitative analysis by TCAD simulation for the NBS time-dependent experimental results, the threshold voltage (V<sub>T</sub>)-shift by sub-bandgap density-of-states redistribution forms 70% and 78% for the measured total V<sub>T</sub>-shift while V<sub>T</sub>-shift by gate oxide charge trapping only takes 30% and 22% at NBS time of 3000 and 7000 s, respectively. In addition, the increase of source/drain Schottky contact resistance (R<sub>SD</sub>) is the main reason for NBS-induced on -current (I<sub>ON</sub>) degradation.]]></abstract>
##     <issn><![CDATA[0741-3106]]></issn>
##     <htmlFlag><![CDATA[1]]></htmlFlag>
##     <arnumber><![CDATA[6720145]]></arnumber>
##     <doi><![CDATA[10.1109/LED.2014.2298861]]></doi>
##     <publicationId><![CDATA[6720145]]></publicationId>
##     <mdurl><![CDATA[http://ieeexplore.ieee.org/xpl/articleDetails.jsp?tp=&arnumber=6720145&contentType=Journals+%26+Magazines]]></mdurl>
##     <pdf><![CDATA[http://ieeexplore.ieee.org/stamp/stamp.jsp?arnumber=6720145]]></pdf>
##   </document>
##   <document>
##     <rank>300</rank>
##     <title><![CDATA[Hardware Trojan Attacks: Threat Analysis and Countermeasures]]></title>
##     <authors><![CDATA[Bhunia, S.;  Hsiao, M.S.;  Banga, M.;  Narasimhan, S.]]></authors>
##     <affiliations><![CDATA[Case Western Reserve Univ., Cleveland, OH, USA]]></affiliations>
##     <controlledterms>
##       <term><![CDATA[electronics industry]]></term>
##       <term><![CDATA[invasive software]]></term>
##     </controlledterms>
##     <thesaurusterms>
##       <term><![CDATA[Circuit faults]]></term>
##       <term><![CDATA[Computer security]]></term>
##       <term><![CDATA[Fabrication]]></term>
##       <term><![CDATA[Hardware]]></term>
##       <term><![CDATA[Integrated circuit modeling]]></term>
##       <term><![CDATA[Integrated circuits]]></term>
##       <term><![CDATA[Trojan horses]]></term>
##     </thesaurusterms>
##     <pubtitle><![CDATA[Proceedings of the IEEE]]></pubtitle>
##     <punumber><![CDATA[5]]></punumber>
##     <pubtype><![CDATA[Journals & Magazines]]></pubtype>
##     <publisher><![CDATA[IEEE]]></publisher>
##     <volume><![CDATA[102]]></volume>
##     <issue><![CDATA[8]]></issue>
##     <py><![CDATA[2014]]></py>
##     <spage><![CDATA[1229]]></spage>
##     <epage><![CDATA[1247]]></epage>
##     <abstract><![CDATA[Security of a computer system has been traditionally related to the security of the software or the information being processed. The underlying hardware used for information processing has been considered trusted. The emergence of hardware Trojan attacks violates this root of trust. These attacks, in the form of malicious modifications of electronic hardware at different stages of its life cycle, pose major security concerns in the electronics industry. An adversary can mount such an attack with an objective to cause operational failure or to leak secret information from inside a chip-e.g., the key in a cryptographic chip, during field operation. Global economic trend that encourages increased reliance on untrusted entities in the hardware design and fabrication process is rapidly enhancing the vulnerability to such attacks. In this paper, we analyze the threat of hardware Trojan attacks; present attack models, types, and scenarios; discuss different forms of protection approaches, both proactive and reactive; and describe emerging attack modes, defenses, and future research pathways.]]></abstract>
##     <issn><![CDATA[0018-9219]]></issn>
##     <htmlFlag><![CDATA[1]]></htmlFlag>
##     <arnumber><![CDATA[6856140]]></arnumber>
##     <doi><![CDATA[10.1109/JPROC.2014.2334493]]></doi>
##     <publicationId><![CDATA[6856140]]></publicationId>
##     <mdurl><![CDATA[http://ieeexplore.ieee.org/xpl/articleDetails.jsp?tp=&arnumber=6856140&contentType=Journals+%26+Magazines]]></mdurl>
##     <pdf><![CDATA[http://ieeexplore.ieee.org/stamp/stamp.jsp?arnumber=6856140]]></pdf>
##   </document>
##   <document>
##     <rank>301</rank>
##     <title><![CDATA[The Technologically Integrated Oncosimulator: Combining Multiscale Cancer Modeling With Information Technology in the In Silico Oncology Context]]></title>
##     <authors><![CDATA[Stamatakos, G.;  Dionysiou, D.;  Lunzer, A.;  Belleman, R.;  Kolokotroni, E.;  Georgiadi, E.;  Erdt, M.;  Pukacki, J.;  Rueping, S.;  Giatili, S.;  d'Onofrio, A.;  Sfakianakis, S.;  Marias, K.;  Desmedt, C.;  Tsiknakis, M.;  Graf, N.]]></authors>
##     <affiliations><![CDATA[In Silico Oncology Group, Nat. Tech. Univ. of Athens, Zografos, Greece]]></affiliations>
##     <controlledterms>
##       <term><![CDATA[biomedical engineering]]></term>
##       <term><![CDATA[cancer]]></term>
##       <term><![CDATA[medical image processing]]></term>
##       <term><![CDATA[patient treatment]]></term>
##       <term><![CDATA[tumours]]></term>
##     </controlledterms>
##     <thesaurusterms>
##       <term><![CDATA[Adaptation models]]></term>
##       <term><![CDATA[Breast cancer]]></term>
##       <term><![CDATA[Chemotherapy]]></term>
##       <term><![CDATA[Computational modeling]]></term>
##       <term><![CDATA[Oncology]]></term>
##       <term><![CDATA[Tumors]]></term>
##     </thesaurusterms>
##     <pubtitle><![CDATA[Biomedical and Health Informatics, IEEE Journal of]]></pubtitle>
##     <punumber><![CDATA[6221020]]></punumber>
##     <pubtype><![CDATA[Journals & Magazines]]></pubtype>
##     <publisher><![CDATA[IEEE]]></publisher>
##     <volume><![CDATA[18]]></volume>
##     <issue><![CDATA[3]]></issue>
##     <py><![CDATA[2014]]></py>
##     <spage><![CDATA[840]]></spage>
##     <epage><![CDATA[854]]></epage>
##     <abstract><![CDATA[This paper outlines the major components and function of the technologically integrated oncosimulator developed primarily within the Advancing Clinico Genomic Trials on Cancer (ACGT) project. The Oncosimulator is defined as an information technology system simulating in vivo tumor response to therapeutic modalities within the clinical trial context. Chemotherapy in the neoadjuvant setting, according to two real clinical trials concerning nephroblastoma and breast cancer, has been considered. The spatiotemporal simulation module embedded in the Oncosimulator is based on the multiscale, predominantly top-down, discrete entity-discrete event cancer simulation technique developed by the In Silico Oncology Group, National Technical University of Athens. The technology modules include multiscale data handling, image processing, invocation of code execution via a spreadsheet-inspired environment portal, execution of the code on the grid, and the visualization of the predictions. A refining scenario for the eventual coupling of the oncosimulator with immunological models is also presented. Parameter values have been adapted to multiscale clinical trial data in a consistent way, thus supporting the predictive potential of the oncosimulator. Indicative results demonstrating various aspects of the clinical adaptation and validation process are presented. Completion of these processes is expected to pave the way for the clinical translation of the system.]]></abstract>
##     <issn><![CDATA[2168-2194]]></issn>
##     <htmlFlag><![CDATA[1]]></htmlFlag>
##     <arnumber><![CDATA[6617658]]></arnumber>
##     <doi><![CDATA[10.1109/JBHI.2013.2284276]]></doi>
##     <publicationId><![CDATA[6617658]]></publicationId>
##     <mdurl><![CDATA[http://ieeexplore.ieee.org/xpl/articleDetails.jsp?tp=&arnumber=6617658&contentType=Journals+%26+Magazines]]></mdurl>
##     <pdf><![CDATA[http://ieeexplore.ieee.org/stamp/stamp.jsp?arnumber=6617658]]></pdf>
##   </document>
##   <document>
##     <rank>302</rank>
##     <title><![CDATA[Superconducting Tunnel Junction Detectors for Analytical Sciences]]></title>
##     <authors><![CDATA[Ohkubo, M.;  Shigetomo, S.;  Ukibe, M.;  Fujii, G.;  Matsubayashi, N.]]></authors>
##     <affiliations><![CDATA[Tsukuba Innovation Arena Headquarter (TIA), Nat. Inst. of Adv. Ind. Sci. & Technol. (AIST), Tsukuba, Japan]]></affiliations>
##     <controlledterms>
##       <term><![CDATA[EXAFS]]></term>
##       <term><![CDATA[mass spectroscopy]]></term>
##       <term><![CDATA[superconducting energy gap]]></term>
##       <term><![CDATA[superconducting junction devices]]></term>
##       <term><![CDATA[superconducting photodetectors]]></term>
##     </controlledterms>
##     <thesaurusterms>
##       <term><![CDATA[Detectors]]></term>
##       <term><![CDATA[Energy resolution]]></term>
##       <term><![CDATA[Josephson junctions]]></term>
##       <term><![CDATA[Junctions]]></term>
##       <term><![CDATA[Niobium]]></term>
##       <term><![CDATA[Photonics]]></term>
##       <term><![CDATA[Tunneling]]></term>
##     </thesaurusterms>
##     <pubtitle><![CDATA[Applied Superconductivity, IEEE Transactions on]]></pubtitle>
##     <punumber><![CDATA[77]]></punumber>
##     <pubtype><![CDATA[Journals & Magazines]]></pubtype>
##     <publisher><![CDATA[IEEE]]></publisher>
##     <volume><![CDATA[24]]></volume>
##     <issue><![CDATA[4]]></issue>
##     <py><![CDATA[2014]]></py>
##     <spage><![CDATA[1]]></spage>
##     <epage><![CDATA[8]]></epage>
##     <abstract><![CDATA[Superconducting tunnel junction (STJ) detectors exhibit superior detection performance for photons and particles at a high spectroscopic resolution of ~ 10 eV, a short dead-time (decay time) of ~ &#x03BC;s, a high quantum efficiency of ~ 100%, and a low detection threshold energy of less than 1 eV, which cannot be achieved by conventional detectors. The outstanding detection performance originates from a small superconducting energy gap of ~ meV, which is three orders of magnitude smaller than ~ eV in semiconductors. This paper reports our recent progress in two applications of STJ detectors to fluorescence-yield X-ray absorption fine structure (XAFS) spectrometry for trace light elements in matrices and mass spectrometry (MS) for ions with the same mass/charge-number ratio (m/z) but different charge states and neutral fragments.]]></abstract>
##     <issn><![CDATA[1051-8223]]></issn>
##     <htmlFlag><![CDATA[1]]></htmlFlag>
##     <arnumber><![CDATA[6803065]]></arnumber>
##     <doi><![CDATA[10.1109/TASC.2014.2318316]]></doi>
##     <publicationId><![CDATA[6803065]]></publicationId>
##     <mdurl><![CDATA[http://ieeexplore.ieee.org/xpl/articleDetails.jsp?tp=&arnumber=6803065&contentType=Journals+%26+Magazines]]></mdurl>
##     <pdf><![CDATA[http://ieeexplore.ieee.org/stamp/stamp.jsp?arnumber=6803065]]></pdf>
##   </document>
##   <document>
##     <rank>303</rank>
##     <title><![CDATA[Enhancing Voice over WLAN via Rate Adaptation and Retry Scheduling]]></title>
##     <authors><![CDATA[Hyewon Lee;  Seongho Byeon;  Byoungjin Kim;  Kwang Bok Lee;  Sunghyun Choi]]></authors>
##     <affiliations><![CDATA[Dept. of Electr. & Comput. Eng., Seoul Nat. Univ., Seoul, South Korea]]></affiliations>
##     <controlledterms>
##       <term><![CDATA[Internet telephony]]></term>
##       <term><![CDATA[quality of service]]></term>
##       <term><![CDATA[telecommunication traffic]]></term>
##       <term><![CDATA[wireless LAN]]></term>
##       <term><![CDATA[wireless channels]]></term>
##     </controlledterms>
##     <thesaurusterms>
##       <term><![CDATA[Algorithm design and analysis]]></term>
##       <term><![CDATA[IP networks]]></term>
##       <term><![CDATA[Internet telephony]]></term>
##       <term><![CDATA[Local area networks]]></term>
##       <term><![CDATA[Quality of service]]></term>
##       <term><![CDATA[Scheduling]]></term>
##       <term><![CDATA[Wireless LAN]]></term>
##     </thesaurusterms>
##     <pubtitle><![CDATA[Mobile Computing, IEEE Transactions on]]></pubtitle>
##     <punumber><![CDATA[7755]]></punumber>
##     <pubtype><![CDATA[Journals & Magazines]]></pubtype>
##     <publisher><![CDATA[IEEE]]></publisher>
##     <volume><![CDATA[13]]></volume>
##     <issue><![CDATA[12]]></issue>
##     <py><![CDATA[2014]]></py>
##     <spage><![CDATA[2791]]></spage>
##     <epage><![CDATA[2805]]></epage>
##     <abstract><![CDATA[Today, voice over IP (VoIP) service is emerging as a popular and important application in wireless local area networks (WLANs). While rate adaptation (or link adaptation) has been identified as a key factor determining the performance of WLANs, we have observed that most (if not all) rate adaptation algorithms have been developed to improve the throughput of data traffic, not the quality of service (QoS) of VoIP traffic. Accordingly, in this paper, we investigate the characteristics of VoIP traffic and the limitations of state-of-the-art rate adaptation algorithms, and then enhance the QoS of voice over WLAN (VoWLAN) by ameliorating the existing rate adaptation algorithms. Specifically, we design fast decrease to control the transmission rate of retransmissions, and retry scheduling to avoid the deep fading of the wireless channel as well as hidden terminal interference. We comparatively evaluate the QoS of the revised rate adaptation algorithms via ns-3 simulations and MadWiFi implementations in various communication environments, and demonstrate that the proposed schemes improve the R-score performance by up to 80 percent depending on the network scenarios.]]></abstract>
##     <issn><![CDATA[1536-1233]]></issn>
##     <arnumber><![CDATA[6517180]]></arnumber>
##     <doi><![CDATA[10.1109/TMC.2013.54]]></doi>
##     <publicationId><![CDATA[6517180]]></publicationId>
##     <mdurl><![CDATA[http://ieeexplore.ieee.org/xpl/articleDetails.jsp?tp=&arnumber=6517180&contentType=Journals+%26+Magazines]]></mdurl>
##     <pdf><![CDATA[http://ieeexplore.ieee.org/stamp/stamp.jsp?arnumber=6517180]]></pdf>
##   </document>
##   <document>
##     <rank>304</rank>
##     <title><![CDATA[Design and Analysis of InP-Based Waveguide Uni-Traveling Carrier Photodiode Integrated on Silicon-on-Insulator Through <inline-formula> <img src="/images/tex/1010.gif" alt="\hbox {Al}_{2}\hbox {O}_{3}"> </inline-formula> Bonding Layer]]></title>
##     <authors><![CDATA[Gao, B.;  Wang, H.;  Liu, C.Y.;  Meng, Q.Q.;  Tian, Y.;  Ang, K.S.;  Si, J.H.]]></authors>
##     <affiliations><![CDATA[Sch. of Electron. & Inf. Eng., Xi'an Jiaotong Univ., Xian, China]]></affiliations>
##     <controlledterms>
##       <term><![CDATA[III-V semiconductors]]></term>
##       <term><![CDATA[alumina]]></term>
##       <term><![CDATA[indium compounds]]></term>
##       <term><![CDATA[optical waveguides]]></term>
##       <term><![CDATA[photodiodes]]></term>
##       <term><![CDATA[silicon-on-insulator]]></term>
##       <term><![CDATA[thermal resistance]]></term>
##     </controlledterms>
##     <thesaurusterms>
##       <term><![CDATA[Aluminum oxide]]></term>
##       <term><![CDATA[Bonding]]></term>
##       <term><![CDATA[Indium phosphide]]></term>
##       <term><![CDATA[Optical waveguides]]></term>
##       <term><![CDATA[Silicon]]></term>
##       <term><![CDATA[Thermal resistance]]></term>
##     </thesaurusterms>
##     <pubtitle><![CDATA[Photonics Journal, IEEE]]></pubtitle>
##     <punumber><![CDATA[4563994]]></punumber>
##     <pubtype><![CDATA[Journals & Magazines]]></pubtype>
##     <publisher><![CDATA[IEEE]]></publisher>
##     <volume><![CDATA[6]]></volume>
##     <issue><![CDATA[5]]></issue>
##     <py><![CDATA[2014]]></py>
##     <spage><![CDATA[1]]></spage>
##     <epage><![CDATA[6]]></epage>
##     <abstract><![CDATA[Thermal resistance and optical evanescent coupling of InP-based waveguide uni-traveling carrier photodiode (UTC-PD) integrated on silicon on insulator (SOI) through novel Al<sub>2</sub>O<sub>3</sub> bonding layer have been investigated with a constant heat spreading model and optical beam propagation method (BPM), respectively. Compared to UTC-PD integrated on SOI through conventional SiO<sub>2</sub> bonding layer, there is a significant reduction (up to 70.41%) in terms of the total thermal resistance for the same structure through Al<sub>2</sub>O<sub>3</sub> bonding layer. On the other hand, based on the evanescent coupling simulation and analysis with BPM, as compared to SiO<sub>2</sub> bonding scheme, no compromise in optical coupling efficiency was found by using Al<sub>2</sub>O<sub>3</sub> bonding layer. Our results suggest that Al<sub>2</sub>O<sub>3</sub> bonding layer could be a promising candidate for high-power and high-speed III-V photonic devices integrated on SOI, where thermal dissipation is a major concern.]]></abstract>
##     <issn><![CDATA[1943-0655]]></issn>
##     <htmlFlag><![CDATA[1]]></htmlFlag>
##     <arnumber><![CDATA[6887294]]></arnumber>
##     <doi><![CDATA[10.1109/JPHOT.2014.2352640]]></doi>
##     <publicationId><![CDATA[6887294]]></publicationId>
##     <mdurl><![CDATA[http://ieeexplore.ieee.org/xpl/articleDetails.jsp?tp=&arnumber=6887294&contentType=Journals+%26+Magazines]]></mdurl>
##     <pdf><![CDATA[http://ieeexplore.ieee.org/stamp/stamp.jsp?arnumber=6887294]]></pdf>
##   </document>
##   <document>
##     <rank>305</rank>
##     <title><![CDATA[An Active Sensing Principle for Haptic Interaction With Dynamical Systems]]></title>
##     <authors><![CDATA[Furmanov, T.;  Karniel, A.]]></authors>
##     <affiliations><![CDATA[Dept. of Biomed. Eng., Ben-Gurion Univ. of the Negev, Beer-Sheva, Israel]]></affiliations>
##     <controlledterms>
##       <term><![CDATA[haptic interfaces]]></term>
##       <term><![CDATA[sensitivity analysis]]></term>
##     </controlledterms>
##     <thesaurusterms>
##       <term><![CDATA[Audio systems]]></term>
##       <term><![CDATA[Data mining]]></term>
##       <term><![CDATA[Haptic interfaces]]></term>
##       <term><![CDATA[Mechanical systems]]></term>
##       <term><![CDATA[Sensitivity]]></term>
##       <term><![CDATA[Sensors]]></term>
##       <term><![CDATA[Testing]]></term>
##     </thesaurusterms>
##     <pubtitle><![CDATA[Access, IEEE]]></pubtitle>
##     <punumber><![CDATA[6287639]]></punumber>
##     <pubtype><![CDATA[Journals & Magazines]]></pubtype>
##     <publisher><![CDATA[IEEE]]></publisher>
##     <volume><![CDATA[2]]></volume>
##     <py><![CDATA[2014]]></py>
##     <spage><![CDATA[243]]></spage>
##     <epage><![CDATA[257]]></epage>
##     <abstract><![CDATA[What is it inside the colorfully wrapped present? You pick it up to ear level and listen, shake it, and then listen again. To you, the basic principle of active sensing is quite clear - first absorb, then, if there is no movement or sound, shake it, and then reabsorb. We propose an extremely basic hypothesis for the active sensing of haptic interaction with dynamical systems. Our hypothesis asserts that in order to improve the efficiency of extracting information from a probed system, the sensor should act according to the following basic principle: if the probed system is passive, the sensor should be active; conversely, when the probed system is active, the sensor should be passive. We proved the proposed principle for interaction with a second-order mechanical system with the goal to enhance classification performance between two possible sine power sources. We showed that the addition of an active power source to a passive testing sensor leads to decreased sensitivity in the amplitude and frequency of the tested power source. Further, an extension of this principle is provided, presenting the conditions for reduced sensitivity to spring and damper parameters. To test its applicability for a linear system in a noisy environment, a computer simulation was performed demonstrating that classification performance improved by following the proposed principle. At last, ten subjects probed an active virtual system under either active or passive conditions. A comparison of the mean just-noticeable difference of both conditions indicated significantly better sensitivity was obtained by following the principle.]]></abstract>
##     <issn><![CDATA[2169-3536]]></issn>
##     <htmlFlag><![CDATA[1]]></htmlFlag>
##     <arnumber><![CDATA[6778786]]></arnumber>
##     <doi><![CDATA[10.1109/ACCESS.2014.2313581]]></doi>
##     <publicationId><![CDATA[6778786]]></publicationId>
##     <mdurl><![CDATA[http://ieeexplore.ieee.org/xpl/articleDetails.jsp?tp=&arnumber=6778786&contentType=Journals+%26+Magazines]]></mdurl>
##     <pdf><![CDATA[http://ieeexplore.ieee.org/stamp/stamp.jsp?arnumber=6778786]]></pdf>
##   </document>
##   <document>
##     <rank>306</rank>
##     <title><![CDATA[On Switched Regulator Design of Uncertain Nonlinear Systems Using Takagi&#x2013;Sugeno Fuzzy Models]]></title>
##     <authors><![CDATA[de Souza, W.A.;  Teixeira, M.C.M.;  Cardim, R.;  Assuncao, E.]]></authors>
##     <affiliations><![CDATA[Dept. of Acad. Areas of Jatai, IFG (Fed. Inst. of Educ., Sci. & Technol. of Goias), Jatai, Brazil]]></affiliations>
##     <controlledterms>
##       <term><![CDATA[Lyapunov methods]]></term>
##       <term><![CDATA[control system synthesis]]></term>
##       <term><![CDATA[fuzzy control]]></term>
##       <term><![CDATA[linear matrix inequalities]]></term>
##       <term><![CDATA[nonlinear control systems]]></term>
##       <term><![CDATA[uncertain systems]]></term>
##     </controlledterms>
##     <thesaurusterms>
##       <term><![CDATA[Indexes]]></term>
##       <term><![CDATA[Lyapunov methods]]></term>
##       <term><![CDATA[Nonlinear systems]]></term>
##       <term><![CDATA[Switches]]></term>
##       <term><![CDATA[Symmetric matrices]]></term>
##       <term><![CDATA[Takagi-Sugeno model]]></term>
##     </thesaurusterms>
##     <pubtitle><![CDATA[Fuzzy Systems, IEEE Transactions on]]></pubtitle>
##     <punumber><![CDATA[91]]></punumber>
##     <pubtype><![CDATA[Journals & Magazines]]></pubtype>
##     <publisher><![CDATA[IEEE]]></publisher>
##     <volume><![CDATA[22]]></volume>
##     <issue><![CDATA[6]]></issue>
##     <py><![CDATA[2014]]></py>
##     <spage><![CDATA[1720]]></spage>
##     <epage><![CDATA[1727]]></epage>
##     <abstract><![CDATA[This paper is concerned with the design of state-feedback switched controllers for a class of uncertain nonlinear plants described by Takagi-Sugeno (T-S) fuzzy models. The proposed methodology eliminates the need to find the membership function expressions to implement the control law, which is relevant in cases where the membership function depends on uncertain parameters. The design of the switched controllers is based on a minimum-type Lyapunov function and the minimization of the time derivative of this Lyapunov function. The conditions of the new stability criterion are represented by a kind of bilinear matrix inequalities (BMIs) that has been solved by the path-following method. A numerical example and the nonlinear control design of a magnetic levitator with uncertainties illustrate the procedure.]]></abstract>
##     <issn><![CDATA[1063-6706]]></issn>
##     <htmlFlag><![CDATA[1]]></htmlFlag>
##     <arnumber><![CDATA[6722917]]></arnumber>
##     <doi><![CDATA[10.1109/TFUZZ.2014.2302494]]></doi>
##     <publicationId><![CDATA[6722917]]></publicationId>
##     <mdurl><![CDATA[http://ieeexplore.ieee.org/xpl/articleDetails.jsp?tp=&arnumber=6722917&contentType=Journals+%26+Magazines]]></mdurl>
##     <pdf><![CDATA[http://ieeexplore.ieee.org/stamp/stamp.jsp?arnumber=6722917]]></pdf>
##   </document>
##   <document>
##     <rank>307</rank>
##     <title><![CDATA[Remote Radar Based on Chaos Generation and Radio Over Fiber]]></title>
##     <authors><![CDATA[Mingjiang Zhang;  Yongning Ji;  Yongning Zhang;  Yuan Wu;  Hang Xu;  Weipeng Xu]]></authors>
##     <affiliations><![CDATA[Key Lab. of Adv. Transducers & Intell. Control Syst., Taiyuan Univ. of Technol., Taiyuan, China]]></affiliations>
##     <controlledterms>
##       <term><![CDATA[antennas]]></term>
##       <term><![CDATA[laser feedback]]></term>
##       <term><![CDATA[optical chaos]]></term>
##       <term><![CDATA[optical fibre communication]]></term>
##       <term><![CDATA[photodetectors]]></term>
##       <term><![CDATA[radio-over-fibre]]></term>
##       <term><![CDATA[remote sensing by radar]]></term>
##       <term><![CDATA[semiconductor lasers]]></term>
##       <term><![CDATA[signal generators]]></term>
##       <term><![CDATA[ultra wideband radar]]></term>
##     </controlledterms>
##     <thesaurusterms>
##       <term><![CDATA[Chaos]]></term>
##       <term><![CDATA[Distance measurement]]></term>
##       <term><![CDATA[Laser radar]]></term>
##       <term><![CDATA[Optical fiber amplifiers]]></term>
##       <term><![CDATA[Radar antennas]]></term>
##       <term><![CDATA[Ultra wideband radar]]></term>
##     </thesaurusterms>
##     <pubtitle><![CDATA[Photonics Journal, IEEE]]></pubtitle>
##     <punumber><![CDATA[4563994]]></punumber>
##     <pubtype><![CDATA[Journals & Magazines]]></pubtype>
##     <publisher><![CDATA[IEEE]]></publisher>
##     <volume><![CDATA[6]]></volume>
##     <issue><![CDATA[5]]></issue>
##     <py><![CDATA[2014]]></py>
##     <spage><![CDATA[1]]></spage>
##     <epage><![CDATA[12]]></epage>
##     <abstract><![CDATA[An ultrawideband (UWB) radar system for remote ranging based on microwave-photonic chaotic signal generation and fiber-optic distribution is proposed and demonstrated experimentally. In this system, an optical-feedback semiconductor laser with optical injection in the central office generates photonic UWB chaos as probing signal, and two single-mode fibers transport the optical signal to the remote antenna monitoring terminal and return the corresponding echoed signal back to the central office. In the remote antenna terminal, the photonic signal is transformed into microwave chaos by a fast photodetector and then launched to target by a transmitting antenna, and the echoed signal received by another antenna is converted into optical domain by modulating a laser diode. The target ranging is achieved at the central office by correlating the echoed signal with the reference signal. We experimentally realize a detection range of 8 m, for a free-space target, after 24-km remote distance, and achieve a ranging resolution of 3 cm for single target and 8 cm for double targets. In another fiber link branch with 15-km fiber transmission, we obtained the 2-cm ranging resolution for a single target.]]></abstract>
##     <issn><![CDATA[1943-0655]]></issn>
##     <htmlFlag><![CDATA[1]]></htmlFlag>
##     <arnumber><![CDATA[6894267]]></arnumber>
##     <doi><![CDATA[10.1109/JPHOT.2014.2352628]]></doi>
##     <publicationId><![CDATA[6894267]]></publicationId>
##     <mdurl><![CDATA[http://ieeexplore.ieee.org/xpl/articleDetails.jsp?tp=&arnumber=6894267&contentType=Journals+%26+Magazines]]></mdurl>
##     <pdf><![CDATA[http://ieeexplore.ieee.org/stamp/stamp.jsp?arnumber=6894267]]></pdf>
##   </document>
##   <document>
##     <rank>308</rank>
##     <title><![CDATA[High Sensitivity of Temperature Sensor Based on Ultracompact Photonics Crystal Fibers]]></title>
##     <authors><![CDATA[Hailiang Chen;  Shuguang Li;  Jianshe Li;  Ying Han;  Yidong Wu]]></authors>
##     <affiliations><![CDATA[State Key Lab. of Metastable Mater. Sci. & Technol., Yanshan Univ., Qinhuangdao, China]]></affiliations>
##     <controlledterms>
##       <term><![CDATA[fibre optic sensors]]></term>
##       <term><![CDATA[finite element analysis]]></term>
##       <term><![CDATA[holey fibres]]></term>
##       <term><![CDATA[optical losses]]></term>
##       <term><![CDATA[optical phase matching]]></term>
##       <term><![CDATA[photonic crystals]]></term>
##       <term><![CDATA[temperature sensors]]></term>
##     </controlledterms>
##     <thesaurusterms>
##       <term><![CDATA[Frequency conversion]]></term>
##       <term><![CDATA[Materials]]></term>
##       <term><![CDATA[Photonic crystals]]></term>
##       <term><![CDATA[Refractive index]]></term>
##       <term><![CDATA[Sensitivity]]></term>
##       <term><![CDATA[Temperature sensors]]></term>
##     </thesaurusterms>
##     <pubtitle><![CDATA[Photonics Journal, IEEE]]></pubtitle>
##     <punumber><![CDATA[4563994]]></punumber>
##     <pubtype><![CDATA[Journals & Magazines]]></pubtype>
##     <publisher><![CDATA[IEEE]]></publisher>
##     <volume><![CDATA[6]]></volume>
##     <issue><![CDATA[6]]></issue>
##     <py><![CDATA[2014]]></py>
##     <spage><![CDATA[1]]></spage>
##     <epage><![CDATA[6]]></epage>
##     <abstract><![CDATA[A temperature sensor with high sensitivity based on ultracompact photonics crystal fibers is proposed and analyzed by the finite-element method. The temperature-sensitive materials are injected into one cladding air hole, which shows high confinement loss and works as a defect core. As the phase-matched condition is satisfied, the power in the transferring core couples to the defect core. The temperature sensitivity and figure of merit reach to 2.82 nm/&#x00B0;C, 0.105/&#x00B0;C and 1.99 nm/&#x00B0;C, 0.048/&#x00B0;C, for the y-polarized and x-polarized directions, respectively, which are one to two orders of magnitude better than other reported sensors. The performance characteristics can be further improved by optimizing the structure parameters and infilling materials.]]></abstract>
##     <issn><![CDATA[1943-0655]]></issn>
##     <htmlFlag><![CDATA[1]]></htmlFlag>
##     <arnumber><![CDATA[6942146]]></arnumber>
##     <doi><![CDATA[10.1109/JPHOT.2014.2366157]]></doi>
##     <publicationId><![CDATA[6942146]]></publicationId>
##     <mdurl><![CDATA[http://ieeexplore.ieee.org/xpl/articleDetails.jsp?tp=&arnumber=6942146&contentType=Journals+%26+Magazines]]></mdurl>
##     <pdf><![CDATA[http://ieeexplore.ieee.org/stamp/stamp.jsp?arnumber=6942146]]></pdf>
##   </document>
##   <document>
##     <rank>309</rank>
##     <title><![CDATA[Simultaneous Visualization of a Paraelectric Mn:KNTN Crystal and Measurement of Its Kerr Coefficient by Digital Holographic Interferometry]]></title>
##     <authors><![CDATA[Lu Qieni;  Zhao Shuang;  Dai Haitao;  Zhang Yimo]]></authors>
##     <affiliations><![CDATA[Coll. of Optoelectron. & Precision Instrum. Eng., Tianjin Univ., Tianjin, China]]></affiliations>
##     <controlledterms>
##       <term><![CDATA[electro-optical effects]]></term>
##       <term><![CDATA[holographic interferometry]]></term>
##       <term><![CDATA[manganese]]></term>
##       <term><![CDATA[optical Kerr effect]]></term>
##       <term><![CDATA[phase transformations]]></term>
##       <term><![CDATA[photonic crystals]]></term>
##       <term><![CDATA[refractive index]]></term>
##     </controlledterms>
##     <thesaurusterms>
##       <term><![CDATA[Crystals]]></term>
##       <term><![CDATA[Holographic optical components]]></term>
##       <term><![CDATA[Holography]]></term>
##       <term><![CDATA[Laser beams]]></term>
##       <term><![CDATA[Optical refraction]]></term>
##       <term><![CDATA[Optical variables control]]></term>
##       <term><![CDATA[Refractive index]]></term>
##     </thesaurusterms>
##     <pubtitle><![CDATA[Photonics Journal, IEEE]]></pubtitle>
##     <punumber><![CDATA[4563994]]></punumber>
##     <pubtype><![CDATA[Journals & Magazines]]></pubtype>
##     <publisher><![CDATA[IEEE]]></publisher>
##     <volume><![CDATA[6]]></volume>
##     <issue><![CDATA[1]]></issue>
##     <py><![CDATA[2014]]></py>
##     <spage><![CDATA[1]]></spage>
##     <epage><![CDATA[10]]></epage>
##     <abstract><![CDATA[A method of visualizing a paraelectric Mn:KNTN crystal and measuring its Kerr electrooptic coefficient by digital holographic interferometry is presented. Digital holograms of the crystal with different voltages are recorded; a series of sequential phase maps of the KNTN crystal is numerically reconstructed from the holograms in different states to visualize the refractive index distribution. The spatial distribution of the induced refractive index change can be obtained by subtracting the phase of different voltages, and the effective electrooptic coefficient is calculated by means of the refractive index change. We obtain the Kerr coefficient of the Mn:KNTN crystal with R<sub>11</sub> = 11.260 &#x00D7;10<sup>-16</sup> m<sup>2</sup>/V<sup>2</sup> and R<sub>12</sub> = -1.351 &#x00D7;10<sup>-16</sup> m<sup>2</sup>/V<sup>2</sup> near its phase-transition temperature, and the research results show that the method presented in this paper is successful and feasible.]]></abstract>
##     <issn><![CDATA[1943-0655]]></issn>
##     <htmlFlag><![CDATA[1]]></htmlFlag>
##     <arnumber><![CDATA[6714404]]></arnumber>
##     <doi><![CDATA[10.1109/JPHOT.2014.2300472]]></doi>
##     <publicationId><![CDATA[6714404]]></publicationId>
##     <mdurl><![CDATA[http://ieeexplore.ieee.org/xpl/articleDetails.jsp?tp=&arnumber=6714404&contentType=Journals+%26+Magazines]]></mdurl>
##     <pdf><![CDATA[http://ieeexplore.ieee.org/stamp/stamp.jsp?arnumber=6714404]]></pdf>
##   </document>
##   <document>
##     <rank>310</rank>
##     <title><![CDATA[Efficient Attribute-Based Signatures for Non-Monotone Predicates in the Standard Model]]></title>
##     <authors><![CDATA[Okamoto, T.;  Takashima, K.]]></authors>
##     <affiliations><![CDATA[Secure Platform Labs., NTT, Musashino, Japan]]></affiliations>
##     <controlledterms>
##       <term><![CDATA[cryptography]]></term>
##       <term><![CDATA[digital signatures]]></term>
##     </controlledterms>
##     <thesaurusterms>
##       <term><![CDATA[Cloud computing]]></term>
##       <term><![CDATA[Computational modeling]]></term>
##       <term><![CDATA[Computer security]]></term>
##       <term><![CDATA[Logic gates]]></term>
##       <term><![CDATA[Privacy]]></term>
##       <term><![CDATA[Standards]]></term>
##     </thesaurusterms>
##     <pubtitle><![CDATA[Cloud Computing, IEEE Transactions on]]></pubtitle>
##     <punumber><![CDATA[6245519]]></punumber>
##     <pubtype><![CDATA[Journals & Magazines]]></pubtype>
##     <publisher><![CDATA[IEEE]]></publisher>
##     <volume><![CDATA[2]]></volume>
##     <issue><![CDATA[4]]></issue>
##     <py><![CDATA[2014]]></py>
##     <spage><![CDATA[409]]></spage>
##     <epage><![CDATA[421]]></epage>
##     <abstract><![CDATA[This paper presents a fully secure (adaptive-predicate unforgeable and private) attribute-based signature (ABS) scheme in the standard model. The security of the proposed ABS scheme is proven under standard assumptions, the decisional linear (DLIN) assumption and the existence of collision resistant (CR) hash functions. The admissible predicates of the proposed ABS scheme are more general than those of the existing ABS schemes, i.e., the proposed ABS scheme is the first to support general non-monotone predicates, which can be expressed using NOT gates as well as AND, OR, and Threshold gates, while the existing ABS schemes only support monotone predicates. The proposed ABS scheme is comparably as efficient as (several times worse than) one of the most efficient ABS schemes, which is proven to be secure in the generic group model.]]></abstract>
##     <issn><![CDATA[2168-7161]]></issn>
##     <htmlFlag><![CDATA[1]]></htmlFlag>
##     <arnumber><![CDATA[6887292]]></arnumber>
##     <doi><![CDATA[10.1109/TCC.2014.2353053]]></doi>
##     <publicationId><![CDATA[6887292]]></publicationId>
##     <mdurl><![CDATA[http://ieeexplore.ieee.org/xpl/articleDetails.jsp?tp=&arnumber=6887292&contentType=Journals+%26+Magazines]]></mdurl>
##     <pdf><![CDATA[http://ieeexplore.ieee.org/stamp/stamp.jsp?arnumber=6887292]]></pdf>
##   </document>
##   <document>
##     <rank>311</rank>
##     <title><![CDATA[Development of a 16.8&#x0025; Efficient 18-&#x03BC;m Silicon Solar Cell on Steel]]></title>
##     <authors><![CDATA[Lu Wang;  Lochtefeld, A.;  Jianshu Han;  Gerger, A.P.;  Carroll, M.;  Jingjia Ji;  Lennon, A.;  Hongzhao Li;  Opila, R.;  Barnett, A.]]></authors>
##     <affiliations><![CDATA[Univ. of New South Wales, Sydney, NSW, Australia]]></affiliations>
##     <controlledterms>
##       <term><![CDATA[current density]]></term>
##       <term><![CDATA[elemental semiconductors]]></term>
##       <term><![CDATA[epitaxial growth]]></term>
##       <term><![CDATA[semiconductor epitaxial layers]]></term>
##       <term><![CDATA[semiconductor growth]]></term>
##       <term><![CDATA[short-circuit currents]]></term>
##       <term><![CDATA[silicon]]></term>
##       <term><![CDATA[solar cells]]></term>
##       <term><![CDATA[surface texture]]></term>
##     </controlledterms>
##     <thesaurusterms>
##       <term><![CDATA[Epitaxial growth]]></term>
##       <term><![CDATA[Photovoltaic cells]]></term>
##       <term><![CDATA[Silicon]]></term>
##       <term><![CDATA[Steel]]></term>
##       <term><![CDATA[Substrates]]></term>
##       <term><![CDATA[Surface texture]]></term>
##       <term><![CDATA[Surface treatment]]></term>
##     </thesaurusterms>
##     <pubtitle><![CDATA[Photovoltaics, IEEE Journal of]]></pubtitle>
##     <punumber><![CDATA[5503869]]></punumber>
##     <pubtype><![CDATA[Journals & Magazines]]></pubtype>
##     <publisher><![CDATA[IEEE]]></publisher>
##     <volume><![CDATA[4]]></volume>
##     <issue><![CDATA[6]]></issue>
##     <py><![CDATA[2014]]></py>
##     <spage><![CDATA[1397]]></spage>
##     <epage><![CDATA[1404]]></epage>
##     <abstract><![CDATA[Thin crystalline silicon solar cells have the potential to achieve high efficiency due to the potential for increased voltage. Thin silicon wafers are fragile; therefore, means of support must be provided. This paper reports the design, development, and analysis of an 18-&#x03BC;m crystalline silicon solar cell electrically integrated with a steel alloy substrate. This ultrathin silicon is epitaxially grown on porous silicon and then transferred onto the steel substrate. This method allows the independent processing of each surface. The steel substrate enables robust handling and provides a conductive back plane. Three groups of cells with planar and textured structures are compared; significant improvements in J<sub>sc</sub>, V<sub>oc</sub>, and fill factor (FF) are achieved. The best cell shows an efficiency of 16.8% with an open-circuit voltage of 632 mV and a short-circuit current density of 34.5 mA/cm<sup>2</sup>.]]></abstract>
##     <issn><![CDATA[2156-3381]]></issn>
##     <htmlFlag><![CDATA[1]]></htmlFlag>
##     <arnumber><![CDATA[6880361]]></arnumber>
##     <doi><![CDATA[10.1109/JPHOTOV.2014.2344769]]></doi>
##     <publicationId><![CDATA[6880361]]></publicationId>
##     <mdurl><![CDATA[http://ieeexplore.ieee.org/xpl/articleDetails.jsp?tp=&arnumber=6880361&contentType=Journals+%26+Magazines]]></mdurl>
##     <pdf><![CDATA[http://ieeexplore.ieee.org/stamp/stamp.jsp?arnumber=6880361]]></pdf>
##   </document>
##   <document>
##     <rank>312</rank>
##     <title><![CDATA[Predictive Monitoring of Mobile Patients by Combining Clinical Observations With Data From Wearable Sensors]]></title>
##     <authors><![CDATA[Clifton, L.;  Clifton, D.A.;  Pimentel, M.A.F.;  Watkinson, P.J.;  Tarassenko, L.]]></authors>
##     <affiliations><![CDATA[Dept. of Eng. Sci., Univ. of Oxford, Oxford, UK]]></affiliations>
##     <controlledterms>
##       <term><![CDATA[biomedical electronics]]></term>
##       <term><![CDATA[biomedical telemetry]]></term>
##       <term><![CDATA[body sensor networks]]></term>
##       <term><![CDATA[data acquisition]]></term>
##       <term><![CDATA[data analysis]]></term>
##       <term><![CDATA[information services]]></term>
##       <term><![CDATA[learning (artificial intelligence)]]></term>
##       <term><![CDATA[low-power electronics]]></term>
##       <term><![CDATA[patient care]]></term>
##       <term><![CDATA[patient monitoring]]></term>
##       <term><![CDATA[support vector machines]]></term>
##       <term><![CDATA[telemedicine]]></term>
##     </controlledterms>
##     <thesaurusterms>
##       <term><![CDATA[Biomedical monitoring]]></term>
##       <term><![CDATA[Hospitals]]></term>
##       <term><![CDATA[Kernel]]></term>
##       <term><![CDATA[Manuals]]></term>
##       <term><![CDATA[Monitoring]]></term>
##       <term><![CDATA[Support vector machines]]></term>
##       <term><![CDATA[Wearable sensors]]></term>
##     </thesaurusterms>
##     <pubtitle><![CDATA[Biomedical and Health Informatics, IEEE Journal of]]></pubtitle>
##     <punumber><![CDATA[6221020]]></punumber>
##     <pubtype><![CDATA[Journals & Magazines]]></pubtype>
##     <publisher><![CDATA[IEEE]]></publisher>
##     <volume><![CDATA[18]]></volume>
##     <issue><![CDATA[3]]></issue>
##     <py><![CDATA[2014]]></py>
##     <spage><![CDATA[722]]></spage>
##     <epage><![CDATA[730]]></epage>
##     <abstract><![CDATA[The majority of patients in the hospital are ambulatory and would benefit significantly from predictive and personalized monitoring systems. Such patients are well suited to having their physiological condition monitored using low-power, minimally intrusive wearable sensors. Despite data-collection systems now being manufactured commercially, allowing physiological data to be acquired from mobile patients, little work has been undertaken on the use of the resultant data in a principled manner for robust patient care, including predictive monitoring. Most current devices generate so many false-positive alerts that devices cannot be used for routine clinical practice. This paper explores principled machine learning approaches to interpreting large quantities of continuously acquired, multivariate physiological data, using wearable patient monitors, where the goal is to provide early warning of serious physiological determination, such that a degree of predictive care may be provided. We adopt a one-class support vector machine formulation, proposing a formulation for determining the free parameters of the model using partial area under the ROC curve, a method arising from the unique requirements of performing online analysis with data from patient-worn sensors. There are few clinical evaluations of machine learning techniques in the literature, so we present results from a study at the Oxford University Hospitals NHS Trust devised to investigate the large-scale clinical use of patient-worn sensors for predictive monitoring in a ward with a high incidence of patient mortality. We show that our system can combine routine manual observations made by clinical staff with the continuous data acquired from wearable sensors. Practical considerations and recommendations based on our experiences of this clinical study are discussed, in the context of a framework for personalized monitoring.]]></abstract>
##     <issn><![CDATA[2168-2194]]></issn>
##     <htmlFlag><![CDATA[1]]></htmlFlag>
##     <arnumber><![CDATA[6675775]]></arnumber>
##     <doi><![CDATA[10.1109/JBHI.2013.2293059]]></doi>
##     <publicationId><![CDATA[6675775]]></publicationId>
##     <mdurl><![CDATA[http://ieeexplore.ieee.org/xpl/articleDetails.jsp?tp=&arnumber=6675775&contentType=Journals+%26+Magazines]]></mdurl>
##     <pdf><![CDATA[http://ieeexplore.ieee.org/stamp/stamp.jsp?arnumber=6675775]]></pdf>
##   </document>
##   <document>
##     <rank>313</rank>
##     <title><![CDATA[A Novel Serial Multimodal Biometrics Framework Based on Semisupervised Learning Techniques]]></title>
##     <authors><![CDATA[Qing Zhang;  Yilong Yin;  De-Chuan Zhan;  Jingliang Peng]]></authors>
##     <affiliations><![CDATA[Machine Learning & Data Min. Lab., Shandong Univ., Jinan, China]]></affiliations>
##     <controlledterms>
##       <term><![CDATA[face recognition]]></term>
##       <term><![CDATA[fingerprint identification]]></term>
##       <term><![CDATA[gesture recognition]]></term>
##       <term><![CDATA[image fusion]]></term>
##       <term><![CDATA[learning (artificial intelligence)]]></term>
##     </controlledterms>
##     <thesaurusterms>
##       <term><![CDATA[Authentication]]></term>
##       <term><![CDATA[Biometrics (access control)]]></term>
##       <term><![CDATA[Educational institutions]]></term>
##       <term><![CDATA[Face]]></term>
##       <term><![CDATA[Face recognition]]></term>
##       <term><![CDATA[Feature extraction]]></term>
##       <term><![CDATA[Reliability]]></term>
##     </thesaurusterms>
##     <pubtitle><![CDATA[Information Forensics and Security, IEEE Transactions on]]></pubtitle>
##     <punumber><![CDATA[10206]]></punumber>
##     <pubtype><![CDATA[Journals & Magazines]]></pubtype>
##     <publisher><![CDATA[IEEE]]></publisher>
##     <volume><![CDATA[9]]></volume>
##     <issue><![CDATA[10]]></issue>
##     <py><![CDATA[2014]]></py>
##     <spage><![CDATA[1681]]></spage>
##     <epage><![CDATA[1694]]></epage>
##     <abstract><![CDATA[We propose in this paper a novel framework for serial multimodal biometric systems based on semisupervised learning techniques. The proposed framework addresses the inherent issues of user inconvenience and system inefficiency in parallel multimodal biometric systems. Further, it advances the serial multimodal biometric systems by promoting the discriminating power of the weaker but more user convenient trait(s) and saving the use of the stronger but less user convenient trait(s) whenever possible. This is in contrast to other existing serial multimodal biometric systems that suggest optimized orderings of the traits deployed and parameterizations of the corresponding matchers but ignore the most important requirements of common applications. In terms of methodology, we propose to use semisupervised learning techniques to strengthen the matcher(s) on the weaker trait(s), utilizing the coupling relationship between the weaker and the stronger traits. A dimensionality reduction method for the weaker trait(s) based on dependence maximization is proposed to achieve this purpose. Experiments on two prototype systems clearly demonstrate the advantages of the proposed framework and methodology.]]></abstract>
##     <issn><![CDATA[1556-6013]]></issn>
##     <htmlFlag><![CDATA[1]]></htmlFlag>
##     <arnumber><![CDATA[6874510]]></arnumber>
##     <doi><![CDATA[10.1109/TIFS.2014.2346703]]></doi>
##     <publicationId><![CDATA[6874510]]></publicationId>
##     <mdurl><![CDATA[http://ieeexplore.ieee.org/xpl/articleDetails.jsp?tp=&arnumber=6874510&contentType=Journals+%26+Magazines]]></mdurl>
##     <pdf><![CDATA[http://ieeexplore.ieee.org/stamp/stamp.jsp?arnumber=6874510]]></pdf>
##   </document>
##   <document>
##     <rank>314</rank>
##     <title><![CDATA[Design and Construction of Arduino-Hacked Variable Gating Distortion Pedal]]></title>
##     <authors><![CDATA[Murthy, A.A.;  Rao, N.;  Beemaiah, Y.R.;  Shandilya, S.D.;  Siddegowda, R.B.]]></authors>
##     <affiliations><![CDATA[Dept. of Mechatron., Manipal Univ., Manipal, India]]></affiliations>
##     <controlledterms>
##       <term><![CDATA[acoustic distortion]]></term>
##       <term><![CDATA[acoustic signal processing]]></term>
##       <term><![CDATA[acoustic waves]]></term>
##       <term><![CDATA[audio signal processing]]></term>
##       <term><![CDATA[microcontrollers]]></term>
##       <term><![CDATA[musical instruments]]></term>
##       <term><![CDATA[potentiometers]]></term>
##     </controlledterms>
##     <thesaurusterms>
##       <term><![CDATA[Audio systems]]></term>
##       <term><![CDATA[Distortion]]></term>
##       <term><![CDATA[Music]]></term>
##       <term><![CDATA[Potentiometers]]></term>
##       <term><![CDATA[Printed circuits]]></term>
##     </thesaurusterms>
##     <pubtitle><![CDATA[Access, IEEE]]></pubtitle>
##     <punumber><![CDATA[6287639]]></punumber>
##     <pubtype><![CDATA[Journals & Magazines]]></pubtype>
##     <publisher><![CDATA[IEEE]]></publisher>
##     <volume><![CDATA[2]]></volume>
##     <py><![CDATA[2014]]></py>
##     <spage><![CDATA[1409]]></spage>
##     <epage><![CDATA[1417]]></epage>
##     <abstract><![CDATA[This paper describes the distortion effects often used in an electric guitar. Distortion is an added effect in an electric guitar, which compresses the peaks of the sound waves produced by the musical instrument, to produce a large number of added overtones, which here is done by rigging up a circuit in collaboration with the Arduino UNO circuit board. The digital potentiometer controlled by the Arduino (microcontroller) was an improvement and was able to produce satisfactory results, as compared with the analog potentiometer without the Arduino control. The complex circuitry of a three-stage distortion circuit with the analog potentiometer was replaced by a digital potentiometer controlled by a microcontroller, with better results. This variable-gating distortion pedal has an added advantage of being compact, light, and inexpensive.]]></abstract>
##     <issn><![CDATA[2169-3536]]></issn>
##     <htmlFlag><![CDATA[1]]></htmlFlag>
##     <arnumber><![CDATA[6965623]]></arnumber>
##     <doi><![CDATA[10.1109/ACCESS.2014.2374195]]></doi>
##     <publicationId><![CDATA[6965623]]></publicationId>
##     <mdurl><![CDATA[http://ieeexplore.ieee.org/xpl/articleDetails.jsp?tp=&arnumber=6965623&contentType=Journals+%26+Magazines]]></mdurl>
##     <pdf><![CDATA[http://ieeexplore.ieee.org/stamp/stamp.jsp?arnumber=6965623]]></pdf>
##   </document>
##   <document>
##     <rank>315</rank>
##     <title><![CDATA[Evolution of Laser-Induced Specific Nanostructures on SiGe Compounds via Laser Irradiation Intensity Tuning]]></title>
##     <authors><![CDATA[Dongfeng Qi;  Xin Li;  Peng Wang;  Songyan Chen;  Wei Huang;  Cheng Li;  Kai Huang;  Hongkai Lai]]></authors>
##     <affiliations><![CDATA[Dept. of Phys., Xiamen Univ., Xiamen, China]]></affiliations>
##     <controlledterms>
##       <term><![CDATA[Ge-Si alloys]]></term>
##       <term><![CDATA[high-speed optical techniques]]></term>
##       <term><![CDATA[laser beam effects]]></term>
##       <term><![CDATA[laser tuning]]></term>
##       <term><![CDATA[light interference]]></term>
##       <term><![CDATA[semiconductor thin films]]></term>
##     </controlledterms>
##     <thesaurusterms>
##       <term><![CDATA[Interference]]></term>
##       <term><![CDATA[Nanostructures]]></term>
##       <term><![CDATA[Radiation effects]]></term>
##       <term><![CDATA[Silicon germanium]]></term>
##       <term><![CDATA[Surface emitting lasers]]></term>
##       <term><![CDATA[Surface morphology]]></term>
##     </thesaurusterms>
##     <pubtitle><![CDATA[Photonics Journal, IEEE]]></pubtitle>
##     <punumber><![CDATA[4563994]]></punumber>
##     <pubtype><![CDATA[Journals & Magazines]]></pubtype>
##     <publisher><![CDATA[IEEE]]></publisher>
##     <volume><![CDATA[6]]></volume>
##     <issue><![CDATA[1]]></issue>
##     <py><![CDATA[2014]]></py>
##     <spage><![CDATA[1]]></spage>
##     <epage><![CDATA[5]]></epage>
##     <abstract><![CDATA[We have presented the mechanisms of nanosecond laser-induced specific nanostructures on a SiGe film by changing laser irradiation intensities. Our experimental results show that the types of the formed nanostructures were sensitive to the laser intensity because the structures such as cone-like protrusions, islands, and droplet-like ripples upon the irradiated SiGe film successively transform to ripple structures as the irradiation intensity decreases. Moreover, we obtain the threshold of the laser intensity for each nanostructure. Further studies give compelling evidence that the laser irradiation undergoes a transition from an initial thermal effect to an optical interference, which results in the evolution of SiGe nanostructures.]]></abstract>
##     <issn><![CDATA[1943-0655]]></issn>
##     <htmlFlag><![CDATA[1]]></htmlFlag>
##     <arnumber><![CDATA[6685818]]></arnumber>
##     <doi><![CDATA[10.1109/JPHOT.2013.2294631]]></doi>
##     <publicationId><![CDATA[6685818]]></publicationId>
##     <mdurl><![CDATA[http://ieeexplore.ieee.org/xpl/articleDetails.jsp?tp=&arnumber=6685818&contentType=Journals+%26+Magazines]]></mdurl>
##     <pdf><![CDATA[http://ieeexplore.ieee.org/stamp/stamp.jsp?arnumber=6685818]]></pdf>
##   </document>
##   <document>
##     <rank>316</rank>
##     <title><![CDATA[Lensless Miniature Portable Fluorometer for Measurement of Chlorophyll and CDOM in Water Using Fluorescence Contact Imaging]]></title>
##     <authors><![CDATA[Blockstein, L.;  Yadid-Pecht, O.]]></authors>
##     <affiliations><![CDATA[Dept. of Electr. & Comput. Eng., Univ. of Calgary, Calgary, AB, Canada]]></affiliations>
##     <controlledterms>
##       <term><![CDATA[fluorescence spectroscopy]]></term>
##       <term><![CDATA[light emitting diodes]]></term>
##       <term><![CDATA[optical filters]]></term>
##       <term><![CDATA[organic compounds]]></term>
##       <term><![CDATA[portable instruments]]></term>
##       <term><![CDATA[water quality]]></term>
##     </controlledterms>
##     <thesaurusterms>
##       <term><![CDATA[Absorption]]></term>
##       <term><![CDATA[Arrays]]></term>
##       <term><![CDATA[Fluorescence]]></term>
##       <term><![CDATA[Monitoring]]></term>
##       <term><![CDATA[Optical filters]]></term>
##       <term><![CDATA[Sea surface]]></term>
##       <term><![CDATA[Water resources]]></term>
##     </thesaurusterms>
##     <pubtitle><![CDATA[Photonics Journal, IEEE]]></pubtitle>
##     <punumber><![CDATA[4563994]]></punumber>
##     <pubtype><![CDATA[Journals & Magazines]]></pubtype>
##     <publisher><![CDATA[IEEE]]></publisher>
##     <volume><![CDATA[6]]></volume>
##     <issue><![CDATA[3]]></issue>
##     <py><![CDATA[2014]]></py>
##     <spage><![CDATA[1]]></spage>
##     <epage><![CDATA[16]]></epage>
##     <abstract><![CDATA[We report on the design, fabrication, and verification of a proof-of-concept miniature fluorometer, which is designed to measure chlorophyll and colored dissolved organic matter (CDOM) concentration in aquatic environment. The system utilizes light emitting diodes (LEDs) for fluorescence excitation and absorption filters for excitation light attenuation. The excitation LED for chlorophyll has peak emission at 465 nm, and the excitation LED for CDOM has peak emission at 341 nm. Our device demonstrates the concept of attaching two different absorption filters on a single sensor array for measuring the fluorescence signal from two different fluorescent dyes. We have tested the system's ability to detect fluorescence from various concentrations of fluorescein as a close simulant and the calibration standard for chlorophyll, and quinine sulfate dihydrate (QSD) as a simulant and calibration standard for CDOM. We have successfully acquired the fluorescent signal for fluorescein between 0.7 and 1000 nM and for QSD between 2.6 and 638 nM.]]></abstract>
##     <issn><![CDATA[1943-0655]]></issn>
##     <htmlFlag><![CDATA[1]]></htmlFlag>
##     <arnumber><![CDATA[6824241]]></arnumber>
##     <doi><![CDATA[10.1109/JPHOT.2014.2326665]]></doi>
##     <publicationId><![CDATA[6824241]]></publicationId>
##     <mdurl><![CDATA[http://ieeexplore.ieee.org/xpl/articleDetails.jsp?tp=&arnumber=6824241&contentType=Journals+%26+Magazines]]></mdurl>
##     <pdf><![CDATA[http://ieeexplore.ieee.org/stamp/stamp.jsp?arnumber=6824241]]></pdf>
##   </document>
##   <document>
##     <rank>317</rank>
##     <title><![CDATA[A LISP-Based Implementation of Follow Me Cloud]]></title>
##     <authors><![CDATA[Ksentini, A.;  Taleb, T.;  Messaoudi, F.]]></authors>
##     <affiliations><![CDATA[Inst. de Rech. en Inf. et Syst. Aleatoires, Univ. of Rennes 1, Rennes, France]]></affiliations>
##     <controlledterms>
##       <term><![CDATA[cloud computing]]></term>
##       <term><![CDATA[digital subscriber lines]]></term>
##       <term><![CDATA[internetworking]]></term>
##       <term><![CDATA[mobile computing]]></term>
##       <term><![CDATA[protocols]]></term>
##       <term><![CDATA[virtualisation]]></term>
##       <term><![CDATA[wireless LAN]]></term>
##     </controlledterms>
##     <thesaurusterms>
##       <term><![CDATA[3G mobile communication]]></term>
##       <term><![CDATA[Cloud computing]]></term>
##       <term><![CDATA[Data centers]]></term>
##       <term><![CDATA[IEEE 802.11 Standards]]></term>
##       <term><![CDATA[Logic gates]]></term>
##       <term><![CDATA[Mobile communication]]></term>
##       <term><![CDATA[Mobile computing]]></term>
##       <term><![CDATA[Virtualization]]></term>
##     </thesaurusterms>
##     <pubtitle><![CDATA[Access, IEEE]]></pubtitle>
##     <punumber><![CDATA[6287639]]></punumber>
##     <pubtype><![CDATA[Journals & Magazines]]></pubtype>
##     <publisher><![CDATA[IEEE]]></publisher>
##     <volume><![CDATA[2]]></volume>
##     <py><![CDATA[2014]]></py>
##     <spage><![CDATA[1340]]></spage>
##     <epage><![CDATA[1347]]></epage>
##     <abstract><![CDATA[The follow me cloud (FMC) concept enables service mobility, wherein not only content/data, but also services follow their respective users. The FMC allows mobile users to be always connected via the optimal data anchor and mobility gateways to access their data and services from optimal data centers. The FMC was initially designed to support user mobility, particularly in 3rd Generation Partnership Project (3GPP) networks. In this paper, FMC is further tailored to support mobile users connected from other network types, such as public WiFi or Asymetric Digital Subscriber Line (ADSL) fixed networks. Indeed, this paper presents an implementation of FMC based on local/identifier separation protocol (LISP), whereby the main goal is to render FMC independent from the underlying technology. To simplify further the deployment, all FMC entities (including LISP entities) are virtualized considering the network function virtualization principle.]]></abstract>
##     <issn><![CDATA[2169-3536]]></issn>
##     <htmlFlag><![CDATA[1]]></htmlFlag>
##     <arnumber><![CDATA[6910231]]></arnumber>
##     <doi><![CDATA[10.1109/ACCESS.2014.2360352]]></doi>
##     <publicationId><![CDATA[6910231]]></publicationId>
##     <mdurl><![CDATA[http://ieeexplore.ieee.org/xpl/articleDetails.jsp?tp=&arnumber=6910231&contentType=Journals+%26+Magazines]]></mdurl>
##     <pdf><![CDATA[http://ieeexplore.ieee.org/stamp/stamp.jsp?arnumber=6910231]]></pdf>
##   </document>
##   <document>
##     <rank>318</rank>
##     <title><![CDATA[Optimal, Nonlinear, and Distributed Designs of Droop Controls for DC Microgrids]]></title>
##     <authors><![CDATA[Maknouninejad, A.;  Zhihua Qu;  Lewis, F.L.;  Davoudi, A.]]></authors>
##     <affiliations><![CDATA[Dept. of Electr. Eng. & Comput. Sci., Univ. of Central Florida, Orlando, FL, USA]]></affiliations>
##     <controlledterms>
##       <term><![CDATA[distributed power generation]]></term>
##       <term><![CDATA[nonlinear control systems]]></term>
##       <term><![CDATA[optimal control]]></term>
##       <term><![CDATA[power control]]></term>
##       <term><![CDATA[power distribution control]]></term>
##       <term><![CDATA[power generation control]]></term>
##       <term><![CDATA[voltage control]]></term>
##     </controlledterms>
##     <thesaurusterms>
##       <term><![CDATA[Adaptive control]]></term>
##       <term><![CDATA[Microgrids]]></term>
##       <term><![CDATA[Optimal control]]></term>
##       <term><![CDATA[Sensors]]></term>
##       <term><![CDATA[Stability analysis]]></term>
##       <term><![CDATA[Symmetric matrices]]></term>
##       <term><![CDATA[Voltage control]]></term>
##     </thesaurusterms>
##     <pubtitle><![CDATA[Smart Grid, IEEE Transactions on]]></pubtitle>
##     <punumber><![CDATA[5165411]]></punumber>
##     <pubtype><![CDATA[Journals & Magazines]]></pubtype>
##     <publisher><![CDATA[IEEE]]></publisher>
##     <volume><![CDATA[5]]></volume>
##     <issue><![CDATA[5]]></issue>
##     <py><![CDATA[2014]]></py>
##     <spage><![CDATA[2508]]></spage>
##     <epage><![CDATA[2516]]></epage>
##     <abstract><![CDATA[In this paper, the problem of optimal voltage and power regulation is formulated for distributed generators (DGs) in DC microgrids. It is shown that the resulting control is optimal but would require the full information of the microgrid. Relaxation of information requirement reduces the optimal control into several controls including the conventional droop control. The general setting of a DC microgrid equipped with local sensing/communication network calls for the design and implementation of a cooperative droop control that uses the available local information and coordinates voltage control in a distributed manner. The proposed cooperative droop control is shown to include other controls as special cases, its performance is superior to the conventional droop control, and it is robust with respect to uncertain changes in both distribution network and sensing/communication network. These features make the proposed control an effective scheme for operating a DC microgrid with intermittent and distributed generation.]]></abstract>
##     <issn><![CDATA[1949-3053]]></issn>
##     <htmlFlag><![CDATA[1]]></htmlFlag>
##     <arnumber><![CDATA[6894269]]></arnumber>
##     <doi><![CDATA[10.1109/TSG.2014.2325855]]></doi>
##     <publicationId><![CDATA[6894269]]></publicationId>
##     <mdurl><![CDATA[http://ieeexplore.ieee.org/xpl/articleDetails.jsp?tp=&arnumber=6894269&contentType=Journals+%26+Magazines]]></mdurl>
##     <pdf><![CDATA[http://ieeexplore.ieee.org/stamp/stamp.jsp?arnumber=6894269]]></pdf>
##   </document>
##   <document>
##     <rank>319</rank>
##     <title><![CDATA[Breakthroughs in Photonics 2013: Quantitative Phase Imaging: Metrology Meets Biology]]></title>
##     <authors><![CDATA[Taewoo Kim;  Renjie Zhou;  Goddard, L.L.;  Popescu, G.]]></authors>
##     <affiliations><![CDATA[Dept. of Electr. & Comput. Eng., Univ. of Illinois at Urbana-Champaign, Champaign, IL, USA]]></affiliations>
##     <controlledterms>
##       <term><![CDATA[biological tissues]]></term>
##       <term><![CDATA[biomedical optical imaging]]></term>
##       <term><![CDATA[blood]]></term>
##       <term><![CDATA[cellular biophysics]]></term>
##       <term><![CDATA[light diffraction]]></term>
##       <term><![CDATA[optical tomography]]></term>
##       <term><![CDATA[phase measurement]]></term>
##       <term><![CDATA[transparency]]></term>
##     </controlledterms>
##     <thesaurusterms>
##       <term><![CDATA[Biology]]></term>
##       <term><![CDATA[Deconvolution]]></term>
##       <term><![CDATA[Diffraction]]></term>
##       <term><![CDATA[Microscopy]]></term>
##       <term><![CDATA[Phase measurement]]></term>
##       <term><![CDATA[Three-dimensional displays]]></term>
##     </thesaurusterms>
##     <pubtitle><![CDATA[Photonics Journal, IEEE]]></pubtitle>
##     <punumber><![CDATA[4563994]]></punumber>
##     <pubtype><![CDATA[Journals & Magazines]]></pubtype>
##     <publisher><![CDATA[IEEE]]></publisher>
##     <volume><![CDATA[6]]></volume>
##     <issue><![CDATA[2]]></issue>
##     <py><![CDATA[2014]]></py>
##     <spage><![CDATA[1]]></spage>
##     <epage><![CDATA[9]]></epage>
##     <abstract><![CDATA[Quantitative phase imaging (QPI) is an emerging optical approach that measures the optical path length of a transparent specimen noninvasively. Therefore, it is suitable for studying unstained biological tissues and cells with high sensitivity and resolution. This capability of QPI has fueled itself to grow rapidly as an active field of study for the past two decades. With this trend, QPI has experienced some breakthroughs in methods and applications in the past year. We briefly review some of these breakthroughs in method, including QPI through silicon marker-free phase nanoscopy and white-light diffraction tomography. Furthermore, some of the applications, such as quantitative phase measurement of cell growth and real-time blood testing, are introduced to show the importance and applicability of the field.]]></abstract>
##     <issn><![CDATA[1943-0655]]></issn>
##     <htmlFlag><![CDATA[1]]></htmlFlag>
##     <arnumber><![CDATA[6762854]]></arnumber>
##     <doi><![CDATA[10.1109/JPHOT.2014.2309647]]></doi>
##     <publicationId><![CDATA[6762854]]></publicationId>
##     <mdurl><![CDATA[http://ieeexplore.ieee.org/xpl/articleDetails.jsp?tp=&arnumber=6762854&contentType=Journals+%26+Magazines]]></mdurl>
##     <pdf><![CDATA[http://ieeexplore.ieee.org/stamp/stamp.jsp?arnumber=6762854]]></pdf>
##   </document>
##   <document>
##     <rank>320</rank>
##     <title><![CDATA[Thermal features of low current discharges and energy transfer to insulation surfaces]]></title>
##     <authors><![CDATA[An Xiao;  Rowland, S.M.;  Xin Tu;  Whitehead, J.C.]]></authors>
##     <affiliations><![CDATA[Sch. of Electr. of Electron. Eng., Univ. of Manchester, Manchester, UK]]></affiliations>
##     <controlledterms>
##       <term><![CDATA[drops]]></term>
##       <term><![CDATA[finite element analysis]]></term>
##       <term><![CDATA[heat transfer]]></term>
##       <term><![CDATA[infrared imaging]]></term>
##       <term><![CDATA[silicone insulation]]></term>
##       <term><![CDATA[spectroscopy]]></term>
##       <term><![CDATA[temperature distribution]]></term>
##       <term><![CDATA[thermal conductivity]]></term>
##     </controlledterms>
##     <thesaurusterms>
##       <term><![CDATA[Discharges (electric)]]></term>
##       <term><![CDATA[Plasma temperature]]></term>
##       <term><![CDATA[Surface discharges]]></term>
##       <term><![CDATA[Temperature]]></term>
##       <term><![CDATA[Temperature measurement]]></term>
##       <term><![CDATA[Temperature sensors]]></term>
##     </thesaurusterms>
##     <pubtitle><![CDATA[Dielectrics and Electrical Insulation, IEEE Transactions on]]></pubtitle>
##     <punumber><![CDATA[94]]></punumber>
##     <pubtype><![CDATA[Journals & Magazines]]></pubtype>
##     <publisher><![CDATA[IEEE]]></publisher>
##     <volume><![CDATA[21]]></volume>
##     <issue><![CDATA[6]]></issue>
##     <py><![CDATA[2014]]></py>
##     <spage><![CDATA[2466]]></spage>
##     <epage><![CDATA[2475]]></epage>
##     <abstract><![CDATA[The thermal features of low current AC discharges, of between 2 and 5 mA, between water droplets are presented in this paper. The `Best-fit' method of optical emission spectroscopy (OES) analysis is applied to measure the discharge temperature. The discharge temperature increased from 1200 K to 1500 K when the current level rose from 2.5 mA to 5 mA, and its temperature was less sensitive to arc length than arc current. Measurements of the distribution of temperature along a 4 mm discharge showed that the regions including the arc roots do not have a higher temperature than the discharge column, despite appearing brighter. The introduction of a silicone rubber insulation surface adjacent to the discharge increased the discharge temperature, typically by ~160 K, and its energy per half cycle by 150%. The resulting temperature distribution on the insulation surface was captured by infrared thermal imaging. Finite element analysis (FEA) software successfully simulated heat transfer between discharge and surface, and showed good agreement with experimental results: highest temperatures being seen under the centre of the arc, where damage is also first seen in practice. FEA can thus be used to model, for example, the impact of thermal conductivity on insulation surface temperature.]]></abstract>
##     <issn><![CDATA[1070-9878]]></issn>
##     <htmlFlag><![CDATA[1]]></htmlFlag>
##     <arnumber><![CDATA[7031495]]></arnumber>
##     <doi><![CDATA[10.1109/TDEI.2014.004587]]></doi>
##     <publicationId><![CDATA[7031495]]></publicationId>
##     <mdurl><![CDATA[http://ieeexplore.ieee.org/xpl/articleDetails.jsp?tp=&arnumber=7031495&contentType=Journals+%26+Magazines]]></mdurl>
##     <pdf><![CDATA[http://ieeexplore.ieee.org/stamp/stamp.jsp?arnumber=7031495]]></pdf>
##   </document>
##   <document>
##     <rank>321</rank>
##     <title><![CDATA[Micromachined Thick Mesh Filters for Millimeter-Wave and Terahertz Applications]]></title>
##     <authors><![CDATA[Yi Wang;  Bin Yang;  Yingtao Tian;  Donnan, R.S.;  Lancaster, M.J.]]></authors>
##     <affiliations><![CDATA[Dept. of Electron. Electr. Comput. Eng., Univ. of Greenwich, Chatham, UK]]></affiliations>
##     <controlledterms>
##       <term><![CDATA[band-pass filters]]></term>
##       <term><![CDATA[electromagnetic coupling]]></term>
##       <term><![CDATA[micromachining]]></term>
##       <term><![CDATA[micromechanical resonators]]></term>
##       <term><![CDATA[millimetre wave filters]]></term>
##       <term><![CDATA[network synthesis]]></term>
##       <term><![CDATA[resonator filters]]></term>
##       <term><![CDATA[submillimetre wave filters]]></term>
##     </controlledterms>
##     <thesaurusterms>
##       <term><![CDATA[Bandwidth]]></term>
##       <term><![CDATA[Couplings]]></term>
##       <term><![CDATA[Educational institutions]]></term>
##       <term><![CDATA[Materials]]></term>
##       <term><![CDATA[Millimeter wave technology]]></term>
##       <term><![CDATA[Resonant frequency]]></term>
##       <term><![CDATA[Sensitivity]]></term>
##     </thesaurusterms>
##     <pubtitle><![CDATA[Terahertz Science and Technology, IEEE Transactions on]]></pubtitle>
##     <punumber><![CDATA[5503871]]></punumber>
##     <pubtype><![CDATA[Journals & Magazines]]></pubtype>
##     <publisher><![CDATA[IEEE]]></publisher>
##     <volume><![CDATA[4]]></volume>
##     <issue><![CDATA[2]]></issue>
##     <py><![CDATA[2014]]></py>
##     <spage><![CDATA[247]]></spage>
##     <epage><![CDATA[253]]></epage>
##     <abstract><![CDATA[This paper presents several freestanding bandpass mesh filters fabricated using an SU-8-based micromachining technique. The important geometric feature of the filters, which SU8 is able to increase, is the thickness of the cross-shaped micromachined slots. This is five times its width. This thickness offers an extra degree of control over the resonance characteristics. The large thickness not only strengthens the structures, but also enhances the resonance quality factor ( Q-factor). A 0.3-mm-thick, single-layer, mesh filter resonant at 300 GHz has been designed and fabricated and its performance verified. The measured Q-factor is 16.3 and the insertion loss is 0.98 dB. Two multi-layer filter structures have also been demonstrated. The first one is a stacked structure of two single mesh filters producing a double thickness, which achieved a further increased Q-factor of 27. This is over six times higher than a thin mesh filter. The second multilayer filter is an electromagnetically coupled structure forming a two-pole filter. The coupling characteristics are discussed based on experimental and simulation results. These thick mesh filters can potentially be used for sensing and material characterization at millimeter-wave and terahertz frequencies.]]></abstract>
##     <issn><![CDATA[2156-342X]]></issn>
##     <htmlFlag><![CDATA[1]]></htmlFlag>
##     <arnumber><![CDATA[6705669]]></arnumber>
##     <doi><![CDATA[10.1109/TTHZ.2013.2296564]]></doi>
##     <publicationId><![CDATA[6705669]]></publicationId>
##     <mdurl><![CDATA[http://ieeexplore.ieee.org/xpl/articleDetails.jsp?tp=&arnumber=6705669&contentType=Journals+%26+Magazines]]></mdurl>
##     <pdf><![CDATA[http://ieeexplore.ieee.org/stamp/stamp.jsp?arnumber=6705669]]></pdf>
##   </document>
##   <document>
##     <rank>322</rank>
##     <title><![CDATA[Photonics-Assisted Serial Channelized Radio-Frequency Measurement System With Nyquist-Bandwidth Detection]]></title>
##     <authors><![CDATA[Hongwei Chen;  Ruiyue Li;  Cheng Lei;  Ying Yu;  Minghua Chen;  Sigang Yang;  Shizhong Xie]]></authors>
##     <affiliations><![CDATA[Tsinghua Nat. Lab. for Inf. Sci. & Technol. (TNList), Beijing, China]]></affiliations>
##     <controlledterms>
##       <term><![CDATA[frequency modulation]]></term>
##       <term><![CDATA[light coherence]]></term>
##       <term><![CDATA[microwave photonics]]></term>
##       <term><![CDATA[modules]]></term>
##       <term><![CDATA[optical receivers]]></term>
##       <term><![CDATA[photodetectors]]></term>
##       <term><![CDATA[telecommunication channels]]></term>
##     </controlledterms>
##     <thesaurusterms>
##       <term><![CDATA[Bandwidth]]></term>
##       <term><![CDATA[Frequency modulation]]></term>
##       <term><![CDATA[Optical fibers]]></term>
##       <term><![CDATA[Optical filters]]></term>
##       <term><![CDATA[Radio frequency]]></term>
##       <term><![CDATA[Time-frequency analysis]]></term>
##     </thesaurusterms>
##     <pubtitle><![CDATA[Photonics Journal, IEEE]]></pubtitle>
##     <punumber><![CDATA[4563994]]></punumber>
##     <pubtype><![CDATA[Journals & Magazines]]></pubtype>
##     <publisher><![CDATA[IEEE]]></publisher>
##     <volume><![CDATA[6]]></volume>
##     <issue><![CDATA[6]]></issue>
##     <py><![CDATA[2014]]></py>
##     <spage><![CDATA[1]]></spage>
##     <epage><![CDATA[7]]></epage>
##     <abstract><![CDATA[Based on serial channelization and coherent detection, a radio-frequency (RF) measurement scheme with a Nyquist-bandwidth detector is proposed and experimentally demonstrated. With a wavelength scanning structure, multiple RF channels serial in the time domain are implemented. A coherent receiving module based on an optical hybrid and balanced photodetectors (BPDs) is constructed to reduce the receiver bandwidth and the bandwidth of the follow-up electronic devices. In this paper, a six-channel 3-GHz-spacing channelizer, with 18-GHz receiving bandwidth and 1.5-GHz BPD, is demonstrated. In addition, multifrequency signals and a linear frequency modulation signal with the slope of 4.53 MHz/&#x03BC;s are tested.]]></abstract>
##     <issn><![CDATA[1943-0655]]></issn>
##     <htmlFlag><![CDATA[1]]></htmlFlag>
##     <arnumber><![CDATA[6955890]]></arnumber>
##     <doi><![CDATA[10.1109/JPHOT.2014.2366168]]></doi>
##     <publicationId><![CDATA[6955890]]></publicationId>
##     <mdurl><![CDATA[http://ieeexplore.ieee.org/xpl/articleDetails.jsp?tp=&arnumber=6955890&contentType=Journals+%26+Magazines]]></mdurl>
##     <pdf><![CDATA[http://ieeexplore.ieee.org/stamp/stamp.jsp?arnumber=6955890]]></pdf>
##   </document>
##   <document>
##     <rank>323</rank>
##     <title><![CDATA[Simplified Configuration of Brillouin Optical Correlation-Domain Reflectometry]]></title>
##     <authors><![CDATA[Hayashi, N.;  Mizuno, Y.;  Nakamura, K.]]></authors>
##     <affiliations><![CDATA[Precision & Intell. Lab., Tokyo Inst. of Technol., Yokohama, Japan]]></affiliations>
##     <controlledterms>
##       <term><![CDATA[Brillouin spectra]]></term>
##       <term><![CDATA[fibre optic sensors]]></term>
##       <term><![CDATA[light reflection]]></term>
##       <term><![CDATA[reflectometry]]></term>
##       <term><![CDATA[strain measurement]]></term>
##     </controlledterms>
##     <thesaurusterms>
##       <term><![CDATA[Frequency modulation]]></term>
##       <term><![CDATA[Optical fiber sensors]]></term>
##       <term><![CDATA[Optical fibers]]></term>
##       <term><![CDATA[Scattering]]></term>
##       <term><![CDATA[Spatial resolution]]></term>
##       <term><![CDATA[Temperature measurement]]></term>
##     </thesaurusterms>
##     <pubtitle><![CDATA[Photonics Journal, IEEE]]></pubtitle>
##     <punumber><![CDATA[4563994]]></punumber>
##     <pubtype><![CDATA[Journals & Magazines]]></pubtype>
##     <publisher><![CDATA[IEEE]]></publisher>
##     <volume><![CDATA[6]]></volume>
##     <issue><![CDATA[5]]></issue>
##     <py><![CDATA[2014]]></py>
##     <spage><![CDATA[1]]></spage>
##     <epage><![CDATA[7]]></epage>
##     <abstract><![CDATA[We develop a simple and cost-efficient configuration of Brillouin optical correlation-domain reflectometry (BOCDR), the setup of which does not include an additional reference path used in standard BOCDR systems. The Fresnel-reflected light from an open end of a sensing fiber is used as reference light. The limitations of spatial resolution, measurement range, and their ratio are theoretically clarified, and then, a distributed strain measurement with a &lt;; 100-mm spatial resolution and a 4.1-m measurement range is demonstrated with a high signal-to-noise ratio.]]></abstract>
##     <issn><![CDATA[1943-0655]]></issn>
##     <htmlFlag><![CDATA[1]]></htmlFlag>
##     <arnumber><![CDATA[6917205]]></arnumber>
##     <doi><![CDATA[10.1109/JPHOT.2014.2361638]]></doi>
##     <publicationId><![CDATA[6917205]]></publicationId>
##     <mdurl><![CDATA[http://ieeexplore.ieee.org/xpl/articleDetails.jsp?tp=&arnumber=6917205&contentType=Journals+%26+Magazines]]></mdurl>
##     <pdf><![CDATA[http://ieeexplore.ieee.org/stamp/stamp.jsp?arnumber=6917205]]></pdf>
##   </document>
##   <document>
##     <rank>324</rank>
##     <title><![CDATA[Acoustic Analysis of Inhaler Sounds From Community-Dwelling Asthmatic Patients for Automatic Assessment of Adherence]]></title>
##     <authors><![CDATA[Holmes, M.S.;  D'arcy, S.;  Costello, R.W.;  Reilly, R.B.]]></authors>
##     <affiliations><![CDATA[Trinity Centre for Bioeng., Trinity Coll. Dublin, Dublin, Ireland]]></affiliations>
##     <controlledterms>
##       <term><![CDATA[bioacoustics]]></term>
##       <term><![CDATA[biomedical equipment]]></term>
##       <term><![CDATA[biomedical measurement]]></term>
##       <term><![CDATA[diseases]]></term>
##       <term><![CDATA[drug delivery systems]]></term>
##       <term><![CDATA[feedback]]></term>
##       <term><![CDATA[medical signal detection]]></term>
##       <term><![CDATA[medical signal processing]]></term>
##       <term><![CDATA[patient monitoring]]></term>
##       <term><![CDATA[pneumodynamics]]></term>
##       <term><![CDATA[psychology]]></term>
##       <term><![CDATA[signal classification]]></term>
##       <term><![CDATA[telemedicine]]></term>
##       <term><![CDATA[telemetry]]></term>
##     </controlledterms>
##     <thesaurusterms>
##       <term><![CDATA[Acoustics]]></term>
##       <term><![CDATA[Algorithm design and analysis]]></term>
##       <term><![CDATA[Biomedical monitoring]]></term>
##       <term><![CDATA[Classification algorithms]]></term>
##       <term><![CDATA[Diseases]]></term>
##       <term><![CDATA[Respiratory diseases]]></term>
##       <term><![CDATA[Training]]></term>
##     </thesaurusterms>
##     <pubtitle><![CDATA[Translational Engineering in Health and Medicine, IEEE Journal of]]></pubtitle>
##     <punumber><![CDATA[6221039]]></punumber>
##     <pubtype><![CDATA[Journals & Magazines]]></pubtype>
##     <publisher><![CDATA[IEEE]]></publisher>
##     <volume><![CDATA[2]]></volume>
##     <py><![CDATA[2014]]></py>
##     <spage><![CDATA[1]]></spage>
##     <epage><![CDATA[10]]></epage>
##     <abstract><![CDATA[Inhalers are devices which deliver medication to the airways in the treatment of chronic respiratory diseases. When used correctly inhalers relieve and improve patients' symptoms. However, adherence to inhaler medication has been demonstrated to be poor, leading to reduced clinical outcomes, wasted medication, and higher healthcare costs. There is a clinical need for a system that can accurately monitor inhaler adherence as currently no method exists to evaluate how patients use their inhalers between clinic visits. This paper presents a method of automatically evaluating inhaler adherence through acoustic analysis of inhaler sounds. An acoustic monitoring device was employed to record the sounds patients produce while using a Diskus dry powder inhaler, in addition to the time and date patients use the inhaler. An algorithm was designed and developed to automatically detect inhaler events from the audio signals and provide feedback regarding patient adherence. The algorithm was evaluated on 407 audio files obtained from 12 community dwelling asthmatic patients. Results of the automatic classification were compared against two expert human raters. For patient data for whom the human raters Cohen's kappa agreement score was , results indicated that the algorithm's accuracy was 83% in determining the correct inhaler technique score compared with the raters. This paper has several clinical implications as it demonstrates the feasibility of using acoustics to objectively monitor patient inhaler adherence and provide real-time personalized medical care for a chronic respiratory illness.]]></abstract>
##     <issn><![CDATA[2168-2372]]></issn>
##     <htmlFlag><![CDATA[1]]></htmlFlag>
##     <arnumber><![CDATA[6762909]]></arnumber>
##     <doi><![CDATA[10.1109/JTEHM.2014.2310480]]></doi>
##     <publicationId><![CDATA[6762909]]></publicationId>
##     <mdurl><![CDATA[http://ieeexplore.ieee.org/xpl/articleDetails.jsp?tp=&arnumber=6762909&contentType=Journals+%26+Magazines]]></mdurl>
##     <pdf><![CDATA[http://ieeexplore.ieee.org/stamp/stamp.jsp?arnumber=6762909]]></pdf>
##   </document>
##   <document>
##     <rank>325</rank>
##     <title><![CDATA[Fabrication Attacks: Zero-Overhead Malicious Modifications Enabling Modern Microprocessor Privilege Escalation]]></title>
##     <authors><![CDATA[Tsoutsos, N.G.;  Maniatakos, M.]]></authors>
##     <affiliations><![CDATA[Polytech. Sch. of Eng., Dept. of Comput. Sci. & Eng., New York Univ., New York, NY, USA]]></affiliations>
##     <controlledterms>
##       <term><![CDATA[invasive software]]></term>
##       <term><![CDATA[multiprocessing systems]]></term>
##     </controlledterms>
##     <thesaurusterms>
##       <term><![CDATA[Computer architecture]]></term>
##       <term><![CDATA[Embedded systems]]></term>
##       <term><![CDATA[Fabrication]]></term>
##       <term><![CDATA[Hardware]]></term>
##       <term><![CDATA[Logic gates]]></term>
##       <term><![CDATA[Microprocessors]]></term>
##       <term><![CDATA[Trojan horses]]></term>
##     </thesaurusterms>
##     <pubtitle><![CDATA[Emerging Topics in Computing, IEEE Transactions on]]></pubtitle>
##     <punumber><![CDATA[6245516]]></punumber>
##     <pubtype><![CDATA[Journals & Magazines]]></pubtype>
##     <publisher><![CDATA[IEEE]]></publisher>
##     <volume><![CDATA[2]]></volume>
##     <issue><![CDATA[1]]></issue>
##     <py><![CDATA[2014]]></py>
##     <spage><![CDATA[81]]></spage>
##     <epage><![CDATA[93]]></epage>
##     <abstract><![CDATA[The wide deployment of general purpose and embedded microprocessors has emphasized the need for defenses against cyber-attacks. Due to the globalized supply chain, however, there are several stages where a processor can be maliciously modified. The most promising stage, and the hardest during which to inject the hardware trojan, is the fabrication stage. As modern microprocessor chips are characterized by very dense, billion-transistor designs, such attacks must be very carefully crafted. In this paper, we demonstrate zero overhead malicious modifications on both high-performance and embedded microprocessors. These hardware trojans enable privilege escalation through execution of an instruction stream that excites the necessary conditions to make the modification appear. The minimal footprint, however, comes at the cost of a small window of attack opportunities. Experimental results show that malicious users can gain escalated privileges within a few million clock cycles. In addition, no system crashes were reported during normal operation, rendering the modifications transparent to the end user.]]></abstract>
##     <issn><![CDATA[2168-6750]]></issn>
##     <htmlFlag><![CDATA[1]]></htmlFlag>
##     <arnumber><![CDATA[6646239]]></arnumber>
##     <doi><![CDATA[10.1109/TETC.2013.2287186]]></doi>
##     <publicationId><![CDATA[6646239]]></publicationId>
##     <mdurl><![CDATA[http://ieeexplore.ieee.org/xpl/articleDetails.jsp?tp=&arnumber=6646239&contentType=Journals+%26+Magazines]]></mdurl>
##     <pdf><![CDATA[http://ieeexplore.ieee.org/stamp/stamp.jsp?arnumber=6646239]]></pdf>
##   </document>
##   <document>
##     <rank>326</rank>
##     <title><![CDATA[Optical Gaussian Notch Filter Based on Periodic Microbent Fiber Bragg Grating]]></title>
##     <authors><![CDATA[Ali, M.M.;  Lim, K.S.;  Becir, A.;  Lai, M.H.;  Ahmad, H.]]></authors>
##     <affiliations><![CDATA[Photonics Res. Centre, Univ. of Malaya, Kuala Lumpur, Malaysia]]></affiliations>
##     <controlledterms>
##       <term><![CDATA[Bragg gratings]]></term>
##       <term><![CDATA[notch filters]]></term>
##       <term><![CDATA[optical filters]]></term>
##     </controlledterms>
##     <thesaurusterms>
##       <term><![CDATA[Fiber gratings]]></term>
##       <term><![CDATA[Modulation]]></term>
##       <term><![CDATA[Optical fibers]]></term>
##       <term><![CDATA[Optical filters]]></term>
##       <term><![CDATA[Optical reflection]]></term>
##     </thesaurusterms>
##     <pubtitle><![CDATA[Photonics Journal, IEEE]]></pubtitle>
##     <punumber><![CDATA[4563994]]></punumber>
##     <pubtype><![CDATA[Journals & Magazines]]></pubtype>
##     <publisher><![CDATA[IEEE]]></publisher>
##     <volume><![CDATA[6]]></volume>
##     <issue><![CDATA[1]]></issue>
##     <py><![CDATA[2014]]></py>
##     <spage><![CDATA[1]]></spage>
##     <epage><![CDATA[8]]></epage>
##     <abstract><![CDATA[In this paper, we demonstrated an optical notch filter constructed from a periodic microbent fiber Bragg grating attained by using two copper-wire-wound slabs. In the reflection spectrum, sideband peaks are created as a result of mechanically induced modulation on the grating period, and a higher number of peaks are observed when greater modulation is applied. The peak wavelength spacing depends upon the period of microbending, which can be varied by changing the diameter of the winding wire or the angle of placement of fiber Bragg grating placed in between the two wound slabs. Moreover, the trend of Bragg transmission loss (BTL) for the transmission spectrum changes with the changing of modulation period and amplitude. The proposed technique is very stable, and it can be used as an optical Gaussian notch filter.]]></abstract>
##     <issn><![CDATA[1943-0655]]></issn>
##     <htmlFlag><![CDATA[1]]></htmlFlag>
##     <arnumber><![CDATA[6690101]]></arnumber>
##     <doi><![CDATA[10.1109/JPHOT.2013.2295466]]></doi>
##     <publicationId><![CDATA[6690101]]></publicationId>
##     <mdurl><![CDATA[http://ieeexplore.ieee.org/xpl/articleDetails.jsp?tp=&arnumber=6690101&contentType=Journals+%26+Magazines]]></mdurl>
##     <pdf><![CDATA[http://ieeexplore.ieee.org/stamp/stamp.jsp?arnumber=6690101]]></pdf>
##   </document>
##   <document>
##     <rank>327</rank>
##     <title><![CDATA[CFD Analysis of Arc&#x2013;Flow Interaction in a High-Voltage Gas Circuit Breaker Using an Overset Method]]></title>
##     <authors><![CDATA[Jung Ho Park;  Kyu Hong Kim;  Chang Ho Yeo;  Hong Kyu Kim]]></authors>
##     <affiliations><![CDATA[Aerosp. Eng. Dept., Seoul Nat. Univ., Seoul, South Korea]]></affiliations>
##     <controlledterms>
##       <term><![CDATA[arcs (electric)]]></term>
##       <term><![CDATA[circuit breakers]]></term>
##       <term><![CDATA[computational fluid dynamics]]></term>
##       <term><![CDATA[nozzles]]></term>
##     </controlledterms>
##     <thesaurusterms>
##       <term><![CDATA[Absorption]]></term>
##       <term><![CDATA[Computational fluid dynamics]]></term>
##       <term><![CDATA[Conductivity]]></term>
##       <term><![CDATA[Electric potential]]></term>
##       <term><![CDATA[Equations]]></term>
##       <term><![CDATA[Heating]]></term>
##       <term><![CDATA[Mathematical model]]></term>
##     </thesaurusterms>
##     <pubtitle><![CDATA[Plasma Science, IEEE Transactions on]]></pubtitle>
##     <punumber><![CDATA[27]]></punumber>
##     <pubtype><![CDATA[Journals & Magazines]]></pubtype>
##     <publisher><![CDATA[IEEE]]></publisher>
##     <volume><![CDATA[42]]></volume>
##     <issue><![CDATA[1]]></issue>
##     <py><![CDATA[2014]]></py>
##     <spage><![CDATA[175]]></spage>
##     <epage><![CDATA[184]]></epage>
##     <abstract><![CDATA[A computational analysis of the gas flow in a gas circuit breaker is presented in this paper. A structured grid with a high-order conservative overset method is used for an accurate flow analysis. In addition, this paper includes the modeling of the arc, radiation, and ablation of the nozzle. The results of the flow analysis are compared with the experimental data, showing good agreement with the experimental data.]]></abstract>
##     <issn><![CDATA[0093-3813]]></issn>
##     <htmlFlag><![CDATA[1]]></htmlFlag>
##     <arnumber><![CDATA[6662454]]></arnumber>
##     <doi><![CDATA[10.1109/TPS.2013.2288102]]></doi>
##     <publicationId><![CDATA[6662454]]></publicationId>
##     <mdurl><![CDATA[http://ieeexplore.ieee.org/xpl/articleDetails.jsp?tp=&arnumber=6662454&contentType=Journals+%26+Magazines]]></mdurl>
##     <pdf><![CDATA[http://ieeexplore.ieee.org/stamp/stamp.jsp?arnumber=6662454]]></pdf>
##   </document>
##   <document>
##     <rank>328</rank>
##     <title><![CDATA[Ensemble Learning for Large-Scale Workload Prediction]]></title>
##     <authors><![CDATA[Singh, N.;  Rao, S.]]></authors>
##     <affiliations><![CDATA[Int. Inst. of Inf. Technol. - Bangalore (IIIT-B), Bangalore, India]]></affiliations>
##     <controlledterms>
##       <term><![CDATA[learning (artificial intelligence)]]></term>
##       <term><![CDATA[network servers]]></term>
##       <term><![CDATA[power aware computing]]></term>
##       <term><![CDATA[resource allocation]]></term>
##     </controlledterms>
##     <thesaurusterms>
##       <term><![CDATA[Approximation algorithms]]></term>
##       <term><![CDATA[Computational modeling]]></term>
##       <term><![CDATA[Data models]]></term>
##       <term><![CDATA[Energy consumption]]></term>
##       <term><![CDATA[Large-scale systems]]></term>
##       <term><![CDATA[Prediction algorithms]]></term>
##       <term><![CDATA[Predictive models]]></term>
##     </thesaurusterms>
##     <pubtitle><![CDATA[Emerging Topics in Computing, IEEE Transactions on]]></pubtitle>
##     <punumber><![CDATA[6245516]]></punumber>
##     <pubtype><![CDATA[Journals & Magazines]]></pubtype>
##     <publisher><![CDATA[IEEE]]></publisher>
##     <volume><![CDATA[2]]></volume>
##     <issue><![CDATA[2]]></issue>
##     <py><![CDATA[2014]]></py>
##     <spage><![CDATA[149]]></spage>
##     <epage><![CDATA[165]]></epage>
##     <abstract><![CDATA[Increasing energy costs of large-scale server systems have led to a demand for innovative methods for optimizing resource utilization in these systems. Such methods aim to reduce server energy consumption, cooling requirements, carbon footprint, and so on, thereby leading to improved holistic sustainability of the overall server infrastructure. At the core of many of these methods lie reliable workload-prediction techniques that guide in identifying servers, time intervals, and other parameters that are needed for building sustainability solutions based on techniques like virtualization and server consolidation for server systems. Many workload prediction methods have been proposed in the recent paper, but unfortunately they do not deal adequately with the issues that arise specifically in large-scale server systems, viz., extensive nonstationarity of server workloads, and massive online streaming data. In this paper, we fill this gap by proposing two online ensemble learning methods for workload prediction, which address these issues in large-scale server systems. The proposed algorithms are motivated from the weighted majority and simulatable experts approaches, which we extend and adapt to the large-scale workload prediction problem. We demonstrate the effectiveness of our algorithms using real and synthetic data sets, and show that using the proposed algorithms, the workloads of 91% of servers in a real data center can be predicted with accuracy &gt; 89%, whereas using baseline approaches, the workloads of only 13%-24% of the servers can be predicted with similar accuracy.]]></abstract>
##     <issn><![CDATA[2168-6750]]></issn>
##     <htmlFlag><![CDATA[1]]></htmlFlag>
##     <arnumber><![CDATA[6784514]]></arnumber>
##     <doi><![CDATA[10.1109/TETC.2014.2310455]]></doi>
##     <publicationId><![CDATA[6784514]]></publicationId>
##     <mdurl><![CDATA[http://ieeexplore.ieee.org/xpl/articleDetails.jsp?tp=&arnumber=6784514&contentType=Journals+%26+Magazines]]></mdurl>
##     <pdf><![CDATA[http://ieeexplore.ieee.org/stamp/stamp.jsp?arnumber=6784514]]></pdf>
##   </document>
##   <document>
##     <rank>329</rank>
##     <title><![CDATA[Decentralized Multi-Agent System-Based Cooperative Frequency Control for Autonomous Microgrids With Communication Constraints]]></title>
##     <authors><![CDATA[Wei Liu;  Wei Gu;  Wanxing Sheng;  Xiaoli Meng;  Zaijun Wu;  Wu Chen]]></authors>
##     <affiliations><![CDATA[Sch. of Electr. Eng., Southeast Univ., Nanjing, China]]></affiliations>
##     <controlledterms>
##       <term><![CDATA[distributed power generation]]></term>
##       <term><![CDATA[frequency control]]></term>
##       <term><![CDATA[load shedding]]></term>
##       <term><![CDATA[multi-agent systems]]></term>
##       <term><![CDATA[power system stability]]></term>
##     </controlledterms>
##     <thesaurusterms>
##       <term><![CDATA[Delay effects]]></term>
##       <term><![CDATA[Frequency control]]></term>
##       <term><![CDATA[Information management]]></term>
##       <term><![CDATA[Microgrids]]></term>
##       <term><![CDATA[PSCAD]]></term>
##       <term><![CDATA[Power system stability]]></term>
##       <term><![CDATA[Stability analysis]]></term>
##     </thesaurusterms>
##     <pubtitle><![CDATA[Sustainable Energy, IEEE Transactions on]]></pubtitle>
##     <punumber><![CDATA[5165391]]></punumber>
##     <pubtype><![CDATA[Journals & Magazines]]></pubtype>
##     <publisher><![CDATA[IEEE]]></publisher>
##     <volume><![CDATA[5]]></volume>
##     <issue><![CDATA[2]]></issue>
##     <py><![CDATA[2014]]></py>
##     <spage><![CDATA[446]]></spage>
##     <epage><![CDATA[456]]></epage>
##     <abstract><![CDATA[Based on power line carrier communication technology, a decentralized multi-agent system (DMAS)-based frequency control strategy is proposed and investigated in this study on an autonomous microgrid with communication constraints, where each agent can only communicate with its neighboring agents. Using the optimized average consensus algorithm, the global information (i.e., total active power deficiency of the microgrid) can be accurately shared in a decentralized way. Depending on the discovered global information, the cooperative frequency control strategy, which involves primary and secondary frequency control and multi-stage load shedding, is executed to achieve a cooperative frequency recovery. Simulation results indicate that the proposed frequency control approach can guarantee the consensus and coordination of the distributed agents and maintain the frequency stability of islanded microgrids even in emergency conditions.]]></abstract>
##     <issn><![CDATA[1949-3029]]></issn>
##     <htmlFlag><![CDATA[1]]></htmlFlag>
##     <arnumber><![CDATA[6709671]]></arnumber>
##     <doi><![CDATA[10.1109/TSTE.2013.2293148]]></doi>
##     <publicationId><![CDATA[6709671]]></publicationId>
##     <mdurl><![CDATA[http://ieeexplore.ieee.org/xpl/articleDetails.jsp?tp=&arnumber=6709671&contentType=Journals+%26+Magazines]]></mdurl>
##     <pdf><![CDATA[http://ieeexplore.ieee.org/stamp/stamp.jsp?arnumber=6709671]]></pdf>
##   </document>
##   <document>
##     <rank>330</rank>
##     <title><![CDATA[Time-Domain Absorbing Boundary Terminations for Waveguide Ports Based on State-Space Models]]></title>
##     <authors><![CDATA[Flisgen, T.;  Heller, J.;  van Rienen, U.]]></authors>
##     <affiliations><![CDATA[IEF, Univ. Rostock, Rostock, Germany]]></affiliations>
##     <controlledterms>
##       <term><![CDATA[electromagnetic wave absorption]]></term>
##       <term><![CDATA[modal analysis]]></term>
##       <term><![CDATA[waveguide components]]></term>
##     </controlledterms>
##     <thesaurusterms>
##       <term><![CDATA[Approximation methods]]></term>
##       <term><![CDATA[Cutoff frequency]]></term>
##       <term><![CDATA[Electromagnetic waveguides]]></term>
##       <term><![CDATA[Ports (Computers)]]></term>
##       <term><![CDATA[Radio frequency]]></term>
##       <term><![CDATA[Time-domain analysis]]></term>
##       <term><![CDATA[Transient analysis]]></term>
##     </thesaurusterms>
##     <pubtitle><![CDATA[Magnetics, IEEE Transactions on]]></pubtitle>
##     <punumber><![CDATA[20]]></punumber>
##     <pubtype><![CDATA[Journals & Magazines]]></pubtype>
##     <publisher><![CDATA[IEEE]]></publisher>
##     <volume><![CDATA[50]]></volume>
##     <issue><![CDATA[2]]></issue>
##     <py><![CDATA[2014]]></py>
##     <spage><![CDATA[145]]></spage>
##     <epage><![CDATA[148]]></epage>
##     <abstract><![CDATA[Absorbing boundary conditions for waveguide ports in time domain are important elements of transient approaches to treat RF structures. A successful way to implement these termination conditions is the decomposition of the transient fields in the absorbing plane in terms of modal field patterns. The absorbing condition is then accomplished by transferring the wave impedances (or admittances) of the modes to time domain, which leads to convolution operations involving Bessel functions and integrals of Bessel functions. This paper presents a new alternative approach: the convolution operations are approximated by appropriate state-space models whose system responses can be conveniently computed by standard integration schemes. These schemes are indispensable for transient simulations anyhow. Sufficiently far away from the cutoff frequency, a wideband match is achieved.]]></abstract>
##     <issn><![CDATA[0018-9464]]></issn>
##     <htmlFlag><![CDATA[1]]></htmlFlag>
##     <arnumber><![CDATA[6749267]]></arnumber>
##     <doi><![CDATA[10.1109/TMAG.2013.2283065]]></doi>
##     <publicationId><![CDATA[6749267]]></publicationId>
##     <mdurl><![CDATA[http://ieeexplore.ieee.org/xpl/articleDetails.jsp?tp=&arnumber=6749267&contentType=Journals+%26+Magazines]]></mdurl>
##     <pdf><![CDATA[http://ieeexplore.ieee.org/stamp/stamp.jsp?arnumber=6749267]]></pdf>
##   </document>
##   <document>
##     <rank>331</rank>
##     <title><![CDATA[Sampling-Based Robot Motion Planning: A Review]]></title>
##     <authors><![CDATA[Elbanhawi, M.;  Simic, M.]]></authors>
##     <affiliations><![CDATA[Sch. of Aerosp., Mech. & Manuf. Eng., RMIT Univ., Melbourne, VIC, Australia]]></affiliations>
##     <controlledterms>
##       <term><![CDATA[mobile robots]]></term>
##       <term><![CDATA[path planning]]></term>
##       <term><![CDATA[robot dynamics]]></term>
##       <term><![CDATA[robot kinematics]]></term>
##       <term><![CDATA[sampling methods]]></term>
##       <term><![CDATA[uncertain systems]]></term>
##     </controlledterms>
##     <thesaurusterms>
##       <term><![CDATA[Heuristic algorithms]]></term>
##       <term><![CDATA[Measurement]]></term>
##       <term><![CDATA[Path planning]]></term>
##       <term><![CDATA[Planning]]></term>
##       <term><![CDATA[Robot sensing systems]]></term>
##       <term><![CDATA[Vegetation]]></term>
##     </thesaurusterms>
##     <pubtitle><![CDATA[Access, IEEE]]></pubtitle>
##     <punumber><![CDATA[6287639]]></punumber>
##     <pubtype><![CDATA[Journals & Magazines]]></pubtype>
##     <publisher><![CDATA[IEEE]]></publisher>
##     <volume><![CDATA[2]]></volume>
##     <py><![CDATA[2014]]></py>
##     <spage><![CDATA[56]]></spage>
##     <epage><![CDATA[77]]></epage>
##     <abstract><![CDATA[Motion planning is a fundamental research area in robotics. Sampling-based methods offer an efficient solution for what is otherwise a rather challenging dilemma of path planning. Consequently, these methods have been extended further away from basic robot planning into further difficult scenarios and diverse applications. A comprehensive survey of the growing body of work in sampling-based planning is given here. Simulations are executed to evaluate some of the proposed planners and highlight some of the implementation details that are often left unspecified. An emphasis is placed on contemporary research directions in this field. We address planners that tackle current issues in robotics. For instance, real-life kinodynamic planning, optimal planning, replanning in dynamic environments, and planning under uncertainty are discussed. The aim of this paper is to survey the state of the art in motion planning and to assess selected planners, examine implementation details and above all shed a light on the current challenges in motion planning and the promising approaches that will potentially overcome those problems.]]></abstract>
##     <issn><![CDATA[2169-3536]]></issn>
##     <htmlFlag><![CDATA[1]]></htmlFlag>
##     <arnumber><![CDATA[6722915]]></arnumber>
##     <doi><![CDATA[10.1109/ACCESS.2014.2302442]]></doi>
##     <publicationId><![CDATA[6722915]]></publicationId>
##     <mdurl><![CDATA[http://ieeexplore.ieee.org/xpl/articleDetails.jsp?tp=&arnumber=6722915&contentType=Journals+%26+Magazines]]></mdurl>
##     <pdf><![CDATA[http://ieeexplore.ieee.org/stamp/stamp.jsp?arnumber=6722915]]></pdf>
##   </document>
##   <document>
##     <rank>332</rank>
##     <title><![CDATA[ASIN-Based UWB Radar for Sludge Monitoring]]></title>
##     <authors><![CDATA[Sardar, S.;  Mishra, A.K.]]></authors>
##     <affiliations><![CDATA[Indian Inst. of Technol. Guwahati, Guwahati, India]]></affiliations>
##     <controlledterms>
##       <term><![CDATA[computerised monitoring]]></term>
##       <term><![CDATA[data visualisation]]></term>
##       <term><![CDATA[finite difference time-domain analysis]]></term>
##       <term><![CDATA[image reconstruction]]></term>
##       <term><![CDATA[oil technology]]></term>
##       <term><![CDATA[radar imaging]]></term>
##       <term><![CDATA[radial basis function networks]]></term>
##       <term><![CDATA[radiotelemetry]]></term>
##       <term><![CDATA[regression analysis]]></term>
##       <term><![CDATA[sludge treatment]]></term>
##       <term><![CDATA[tanks (containers)]]></term>
##       <term><![CDATA[ultra wideband radar]]></term>
##     </controlledterms>
##     <thesaurusterms>
##       <term><![CDATA[Environmental factors]]></term>
##       <term><![CDATA[Finite difference methods]]></term>
##       <term><![CDATA[Fuel storage]]></term>
##       <term><![CDATA[Monitoring]]></term>
##       <term><![CDATA[Oil tanks]]></term>
##       <term><![CDATA[Radar imaging]]></term>
##       <term><![CDATA[Sludge treatment]]></term>
##       <term><![CDATA[Time-domain analysis]]></term>
##       <term><![CDATA[Ultra wideband radar]]></term>
##     </thesaurusterms>
##     <pubtitle><![CDATA[Access, IEEE]]></pubtitle>
##     <punumber><![CDATA[6287639]]></punumber>
##     <pubtype><![CDATA[Journals & Magazines]]></pubtype>
##     <publisher><![CDATA[IEEE]]></publisher>
##     <volume><![CDATA[2]]></volume>
##     <py><![CDATA[2014]]></py>
##     <spage><![CDATA[290]]></spage>
##     <epage><![CDATA[300]]></epage>
##     <abstract><![CDATA[In this paper, we propose an application specific instrument (ASIN)-based ultrawideband (UWB) radar system for sludge monitoring from scattering signatures from the bottom of industrial oil tanks. The method is validated by successful estimation of sludge volume in oil tanks using simulated and real data. First, as a demonstration of the conventional system, image reconstruction algorithms are used for tank-bottom sludge profile imaging for symmetrical and asymmetrical sludge profiles, where the setup is modeled in finite difference time domain method with reduced dimensions of the tank. A 3-D imaging algorithm is used for the 3-D simulation of real life targets. To get the volume of the sludge, ASIN-based UWB radar system is then applied and its effectiveness is demonstrated. In this framework, to get information about the sludge at the bottom of industrial tank, first, a scheme is proposed to differentiate between two sets of data which correspond to two different set of volumes. This method is validated using a commercial UWB kit, in which, practical experiments were performed. The data obtained is visualized using multidimensional scaling procedure and analyzed. Then, regression analysis using radial basis function artificial neuron network is performed, so that given a particular data, it can be predicted that, to which volume it best corresponds.]]></abstract>
##     <issn><![CDATA[2169-3536]]></issn>
##     <htmlFlag><![CDATA[1]]></htmlFlag>
##     <arnumber><![CDATA[6778048]]></arnumber>
##     <doi><![CDATA[10.1109/ACCESS.2014.2313601]]></doi>
##     <publicationId><![CDATA[6778048]]></publicationId>
##     <mdurl><![CDATA[http://ieeexplore.ieee.org/xpl/articleDetails.jsp?tp=&arnumber=6778048&contentType=Journals+%26+Magazines]]></mdurl>
##     <pdf><![CDATA[http://ieeexplore.ieee.org/stamp/stamp.jsp?arnumber=6778048]]></pdf>
##   </document>
##   <document>
##     <rank>333</rank>
##     <title><![CDATA[Secondary Electron Energy Deposition in Thin Polymeric Films for Neutron-Photon Discrimination]]></title>
##     <authors><![CDATA[Miller, L.F.;  Urffer, M.J.;  Mabe, A.;  Uppal, R.;  Penumadu, D.;  Schweitzer, G.]]></authors>
##     <affiliations><![CDATA[Dept. of Nucl. Eng., Univ. of Tennessee, Knoxville, TN, USA]]></affiliations>
##     <controlledterms>
##       <term><![CDATA[Compton effect]]></term>
##       <term><![CDATA[Monte Carlo methods]]></term>
##       <term><![CDATA[radiation monitoring]]></term>
##       <term><![CDATA[thin films]]></term>
##     </controlledterms>
##     <thesaurusterms>
##       <term><![CDATA[Detectors]]></term>
##       <term><![CDATA[Energy measurement]]></term>
##       <term><![CDATA[Neutrons]]></term>
##       <term><![CDATA[Photonics]]></term>
##       <term><![CDATA[Polymers]]></term>
##       <term><![CDATA[Rendering (computer graphics)]]></term>
##     </thesaurusterms>
##     <pubtitle><![CDATA[Nuclear Science, IEEE Transactions on]]></pubtitle>
##     <punumber><![CDATA[23]]></punumber>
##     <pubtype><![CDATA[Journals & Magazines]]></pubtype>
##     <publisher><![CDATA[IEEE]]></publisher>
##     <volume><![CDATA[61]]></volume>
##     <issue><![CDATA[3]]></issue>
##     <part><![CDATA[2]]></part>
##     <py><![CDATA[2014]]></py>
##     <spage><![CDATA[1381]]></spage>
##     <epage><![CDATA[1388]]></epage>
##     <abstract><![CDATA[Thin polymeric films are evaluated in this research as a potential replacement technology for radiation portal monitors where specific attention is given to the physical basis for neutron-photon discrimination. It is shown that the difference in the energy deposition mechanics from charged particle reaction products and from the Compton scattered electrons allows for the effective discrimination between neutrons and gammas. One goal of this study was to establish optimal thickness for polymeric films that maximize the neutron interactions and simultaneously minimize the measured interaction of photons. Polymeric films ranging from 15 &#x03BC;m to 600 &#x03BC;m containing <sup>6</sup>LiF were fabricated and tested for their capability to satisfy criteria established by the Monte Carlo simulations with the GEANT4 code and data from measurements confirm the technical basis for our proposed understanding of neutron-photon discrimination characteristics for thin films.Department of Homeland Security. Results from Monte Carlo simulations with the GEANT4 code and data from measurements confirm the technical basis for our proposed understanding of neutron-photon discrimination characteristics for thin films.]]></abstract>
##     <issn><![CDATA[0018-9499]]></issn>
##     <htmlFlag><![CDATA[1]]></htmlFlag>
##     <arnumber><![CDATA[6818409]]></arnumber>
##     <doi><![CDATA[10.1109/TNS.2014.2312822]]></doi>
##     <publicationId><![CDATA[6818409]]></publicationId>
##     <mdurl><![CDATA[http://ieeexplore.ieee.org/xpl/articleDetails.jsp?tp=&arnumber=6818409&contentType=Journals+%26+Magazines]]></mdurl>
##     <pdf><![CDATA[http://ieeexplore.ieee.org/stamp/stamp.jsp?arnumber=6818409]]></pdf>
##   </document>
##   <document>
##     <rank>334</rank>
##     <title><![CDATA[Influence of Data Patterns on Reader Performance at Off-Track Reading]]></title>
##     <authors><![CDATA[Zhejie Liu;  Zhimin Yuan;  Chun-Lian Ong;  Shiming Ang]]></authors>
##     <affiliations><![CDATA[Data Storage Inst., Agency for Sci., Technol. & Res., Singapore, Singapore]]></affiliations>
##     <controlledterms>
##       <term><![CDATA[gyromagnetic effect]]></term>
##       <term><![CDATA[magnetic fields]]></term>
##       <term><![CDATA[magnetic heads]]></term>
##       <term><![CDATA[magnetic recording]]></term>
##       <term><![CDATA[magnetic sensors]]></term>
##       <term><![CDATA[magnetisation]]></term>
##       <term><![CDATA[micromagnetics]]></term>
##     </controlledterms>
##     <thesaurusterms>
##       <term><![CDATA[Magnetic noise]]></term>
##       <term><![CDATA[Magnetic tunneling]]></term>
##       <term><![CDATA[Magnetization]]></term>
##       <term><![CDATA[Media]]></term>
##       <term><![CDATA[Resistance]]></term>
##       <term><![CDATA[Saturation magnetization]]></term>
##     </thesaurusterms>
##     <pubtitle><![CDATA[Magnetics, IEEE Transactions on]]></pubtitle>
##     <punumber><![CDATA[20]]></punumber>
##     <pubtype><![CDATA[Journals & Magazines]]></pubtype>
##     <publisher><![CDATA[IEEE]]></publisher>
##     <volume><![CDATA[50]]></volume>
##     <issue><![CDATA[11]]></issue>
##     <py><![CDATA[2014]]></py>
##     <spage><![CDATA[1]]></spage>
##     <epage><![CDATA[4]]></epage>
##     <abstract><![CDATA[In this paper, we focus our attention on the performance of read heads under the off-track reading condition when the reader is under the influence of the recorded magnetization patterns on the medium, and analyze how the magnetic field due to various data patterns impacts the read head behavior. The analysis is based on the micromagnetic modeling of the state of magnetization in read sensor considering its external magnetic fields due to both the hard bias and the media magnetization pattern. The effects of various magnetization patterns on media are analyzed. The effect of thermal magnetic agitation of the gyromagnetical precession of magnetizations is also evaluated. It is shown that to account for such effect is important for evaluation of magnetic recording schemes for extremely high density.]]></abstract>
##     <issn><![CDATA[0018-9464]]></issn>
##     <htmlFlag><![CDATA[1]]></htmlFlag>
##     <arnumber><![CDATA[6971789]]></arnumber>
##     <doi><![CDATA[10.1109/TMAG.2014.2329813]]></doi>
##     <publicationId><![CDATA[6971789]]></publicationId>
##     <mdurl><![CDATA[http://ieeexplore.ieee.org/xpl/articleDetails.jsp?tp=&arnumber=6971789&contentType=Journals+%26+Magazines]]></mdurl>
##     <pdf><![CDATA[http://ieeexplore.ieee.org/stamp/stamp.jsp?arnumber=6971789]]></pdf>
##   </document>
##   <document>
##     <rank>335</rank>
##     <title><![CDATA[Tapered-EDF-Based Mach&#x2013;Zehnder Interferometer for Dual-Wavelength Fiber Laser]]></title>
##     <authors><![CDATA[Md Ali, M.I.;  Ibrahim, S.A.;  Abu Bakar, M.H.;  Noor, A.S.M.;  Ahmad Anas, S.B.;  Zamzuri, A.K.;  Mahdi, M.A.]]></authors>
##     <affiliations><![CDATA[Wireless & Photonics Networks Res. Centre, Univ. Putra Malaysia, Serdang, Malaysia]]></affiliations>
##     <controlledterms>
##       <term><![CDATA[Mach-Zehnder interferometers]]></term>
##       <term><![CDATA[comb filters]]></term>
##       <term><![CDATA[erbium]]></term>
##       <term><![CDATA[fibre lasers]]></term>
##       <term><![CDATA[optical fibre filters]]></term>
##     </controlledterms>
##     <thesaurusterms>
##       <term><![CDATA[Fiber lasers]]></term>
##       <term><![CDATA[Optical fiber filters]]></term>
##       <term><![CDATA[Optical fiber networks]]></term>
##       <term><![CDATA[Optical fibers]]></term>
##       <term><![CDATA[Pump lasers]]></term>
##     </thesaurusterms>
##     <pubtitle><![CDATA[Photonics Journal, IEEE]]></pubtitle>
##     <punumber><![CDATA[4563994]]></punumber>
##     <pubtype><![CDATA[Journals & Magazines]]></pubtype>
##     <publisher><![CDATA[IEEE]]></publisher>
##     <volume><![CDATA[6]]></volume>
##     <issue><![CDATA[5]]></issue>
##     <py><![CDATA[2014]]></py>
##     <spage><![CDATA[1]]></spage>
##     <epage><![CDATA[9]]></epage>
##     <abstract><![CDATA[An all-fiber comb filter using a tapered-erbium-doped fiber in a Mach-Zehnder interferometer structure is presented. The free spectral range, extinction ratio, bandwidth, and interference pattern of the comb filter can be shaped by controlling the taper waist length and the length of up and down taper transition regions. By varying the taper waist length from 5 to 25 mm, the free spectral range changes from 14.7 to 1.0 nm, and the linewidth varies from 3.3 to 0.3 nm, respectively. We demonstrate a tunable dual-wavelength laser by using the tapered-erbium-doped fiber as a gain medium as well as a wavelength-selective element. The laser can be tuned at a resolution of 0.2 nm with a side-mode suppression ratio of up to 46.88 dB and a linewidth of 0.09 nm.]]></abstract>
##     <issn><![CDATA[1943-0655]]></issn>
##     <htmlFlag><![CDATA[1]]></htmlFlag>
##     <arnumber><![CDATA[6920014]]></arnumber>
##     <doi><![CDATA[10.1109/JPHOT.2014.2361642]]></doi>
##     <publicationId><![CDATA[6920014]]></publicationId>
##     <mdurl><![CDATA[http://ieeexplore.ieee.org/xpl/articleDetails.jsp?tp=&arnumber=6920014&contentType=Journals+%26+Magazines]]></mdurl>
##     <pdf><![CDATA[http://ieeexplore.ieee.org/stamp/stamp.jsp?arnumber=6920014]]></pdf>
##   </document>
##   <document>
##     <rank>336</rank>
##     <title><![CDATA[Overcoming the Equivalent Mutant Problem: A Systematic Literature Review and a Comparative Experiment of Second Order Mutation]]></title>
##     <authors><![CDATA[Madeyski, L.;  Orzeszyna, W.;  Torkar, R.;  Jozala, M.]]></authors>
##     <affiliations><![CDATA[Inst. of Inf., Wroclaw Univ. of Technol., Wroclaw, Poland]]></affiliations>
##     <controlledterms>
##       <term><![CDATA[digital libraries]]></term>
##       <term><![CDATA[program testing]]></term>
##     </controlledterms>
##     <thesaurusterms>
##       <term><![CDATA[Databases]]></term>
##       <term><![CDATA[Educational institutions]]></term>
##       <term><![CDATA[Informatics]]></term>
##       <term><![CDATA[Java]]></term>
##       <term><![CDATA[Libraries]]></term>
##       <term><![CDATA[Systematics]]></term>
##       <term><![CDATA[Testing]]></term>
##     </thesaurusterms>
##     <pubtitle><![CDATA[Software Engineering, IEEE Transactions on]]></pubtitle>
##     <punumber><![CDATA[32]]></punumber>
##     <pubtype><![CDATA[Journals & Magazines]]></pubtype>
##     <publisher><![CDATA[IEEE]]></publisher>
##     <volume><![CDATA[40]]></volume>
##     <issue><![CDATA[1]]></issue>
##     <py><![CDATA[2014]]></py>
##     <spage><![CDATA[23]]></spage>
##     <epage><![CDATA[42]]></epage>
##     <abstract><![CDATA[Context. The equivalent mutant problem (EMP) is one of the crucial problems in mutation testing widely studied over decades. Objectives. The objectives are: to present a systematic literature review (SLR) in the field of EMP; to identify, classify and improve the existing, or implement new, methods which try to overcome EMP and evaluate them. Method. We performed SLR based on the search of digital libraries. We implemented four second order mutation (SOM) strategies, in addition to first order mutation (FOM), and compared them from different perspectives. Results. Our SLR identified 17 relevant techniques (in 22 articles) and three categories of techniques: detecting (DEM); suggesting (SEM); and avoiding equivalent mutant generation (AEMG). The experiment indicated that SOM in general and JudyDiffOp strategy in particular provide the best results in the following areas: total number of mutants generated; the association between the type of mutation strategy and whether the generated mutants were equivalent or not; the number of not killed mutants; mutation testing time; time needed for manual classification. Conclusions . The results in the DEM category are still far from perfect. Thus, the SEM and AEMG categories have been developed. The JudyDiffOp algorithm achieved good results in many areas.]]></abstract>
##     <issn><![CDATA[0098-5589]]></issn>
##     <htmlFlag><![CDATA[1]]></htmlFlag>
##     <arnumber><![CDATA[6613487]]></arnumber>
##     <doi><![CDATA[10.1109/TSE.2013.44]]></doi>
##     <publicationId><![CDATA[6613487]]></publicationId>
##     <mdurl><![CDATA[http://ieeexplore.ieee.org/xpl/articleDetails.jsp?tp=&arnumber=6613487&contentType=Journals+%26+Magazines]]></mdurl>
##     <pdf><![CDATA[http://ieeexplore.ieee.org/stamp/stamp.jsp?arnumber=6613487]]></pdf>
##   </document>
##   <document>
##     <rank>337</rank>
##     <title><![CDATA[Double-Layer Crystalline Silicon on Insulator Material Platform for Integrated Photonic Applications]]></title>
##     <authors><![CDATA[Moradinejad, H.;  Atabaki, A.H.;  Hosseinnia, A.H.;  Eftekhar, A.A.;  Adibi, A.]]></authors>
##     <affiliations><![CDATA[Sch. of Electr. & Comput. Eng., Georgia Inst. of Technol., Atlanta, GA, USA]]></affiliations>
##     <controlledterms>
##       <term><![CDATA[Q-factor]]></term>
##       <term><![CDATA[integrated optics]]></term>
##       <term><![CDATA[micro-optics]]></term>
##       <term><![CDATA[microcavities]]></term>
##       <term><![CDATA[optical fabrication]]></term>
##       <term><![CDATA[optical filters]]></term>
##       <term><![CDATA[optical modulation]]></term>
##       <term><![CDATA[optical multilayers]]></term>
##       <term><![CDATA[optical resonators]]></term>
##       <term><![CDATA[optical switches]]></term>
##       <term><![CDATA[silicon-on-insulator]]></term>
##       <term><![CDATA[wafer bonding]]></term>
##     </controlledterms>
##     <thesaurusterms>
##       <term><![CDATA[Bonding]]></term>
##       <term><![CDATA[Optical resonators]]></term>
##       <term><![CDATA[Optical waveguides]]></term>
##       <term><![CDATA[Photonics]]></term>
##       <term><![CDATA[Silicon]]></term>
##       <term><![CDATA[Waveguide discontinuities]]></term>
##     </thesaurusterms>
##     <pubtitle><![CDATA[Photonics Journal, IEEE]]></pubtitle>
##     <punumber><![CDATA[4563994]]></punumber>
##     <pubtype><![CDATA[Journals & Magazines]]></pubtype>
##     <publisher><![CDATA[IEEE]]></publisher>
##     <volume><![CDATA[6]]></volume>
##     <issue><![CDATA[6]]></issue>
##     <py><![CDATA[2014]]></py>
##     <spage><![CDATA[1]]></spage>
##     <epage><![CDATA[8]]></epage>
##     <abstract><![CDATA[We report a high optical-quality double-layer silicon (Si) material platform with a thin interface oxide using a low-temperature wafer bonding technique. To assess the quality of the platform, resonators with different radii are fabricated, and their quality factors (Q) are measured. Q's of 25 k and 350 k are demonstrated in 2-&#x03BC;m- and 20-&#x03BC;m-radius microring resonators, respectively. The former is the most compact high-Q resonator demonstrated in any type of double-layer Si platform, and the latter is the highest Q demonstrated in a double-layer Si platform to date. This material platform enables a new set of integrated optical devices for a wide range of applications, including high-speed modulators, tunable filters, and low-power switches.]]></abstract>
##     <issn><![CDATA[1943-0655]]></issn>
##     <htmlFlag><![CDATA[1]]></htmlFlag>
##     <arnumber><![CDATA[6945248]]></arnumber>
##     <doi><![CDATA[10.1109/JPHOT.2014.2366169]]></doi>
##     <publicationId><![CDATA[6945248]]></publicationId>
##     <mdurl><![CDATA[http://ieeexplore.ieee.org/xpl/articleDetails.jsp?tp=&arnumber=6945248&contentType=Journals+%26+Magazines]]></mdurl>
##     <pdf><![CDATA[http://ieeexplore.ieee.org/stamp/stamp.jsp?arnumber=6945248]]></pdf>
##   </document>
##   <document>
##     <rank>338</rank>
##     <title><![CDATA[Cognitive Ability-Demand Gap Analysis With Latent Response Models]]></title>
##     <authors><![CDATA[Hossain, G.;  Yeasin, M.]]></authors>
##     <affiliations><![CDATA[Dept. of Electr. & Comput. Eng., Univ. of Memphis, Memphis, TN, USA]]></affiliations>
##     <controlledterms>
##       <term><![CDATA[cognition]]></term>
##       <term><![CDATA[ergonomics]]></term>
##       <term><![CDATA[feedback]]></term>
##     </controlledterms>
##     <thesaurusterms>
##       <term><![CDATA[Adaptation models]]></term>
##       <term><![CDATA[Assistive technology]]></term>
##       <term><![CDATA[Cognition]]></term>
##       <term><![CDATA[Data models]]></term>
##       <term><![CDATA[Human factors]]></term>
##       <term><![CDATA[Market research]]></term>
##       <term><![CDATA[Rasch models]]></term>
##     </thesaurusterms>
##     <pubtitle><![CDATA[Access, IEEE]]></pubtitle>
##     <punumber><![CDATA[6287639]]></punumber>
##     <pubtype><![CDATA[Journals & Magazines]]></pubtype>
##     <publisher><![CDATA[IEEE]]></publisher>
##     <volume><![CDATA[2]]></volume>
##     <py><![CDATA[2014]]></py>
##     <spage><![CDATA[711]]></spage>
##     <epage><![CDATA[724]]></epage>
##     <abstract><![CDATA[A better understanding of human cognitive ability-demand gap (ADG) is critical in designing assistive technology solution that is accurate and adaptive over a wide range of human-agent interaction. The main goal is to design systems that can adapt with the user's abilities and needs over a range of cognitive tasks. It will also enable the system to provide feedback consistent with the situation. However, the latent structure and relationship between human ability to respond to cognitive task (demand on human by the agent) remains unknown. Robust modeling of cognitive ADG will be a paradigm shift from the current trends in assistive technology design. The key idea is to estimate the gap, based on human-agent cognitive task interaction. In particular, latent response model was adopted to quantify the gap. First, we used one parameter Rasch model and extended Rasch model (rating scale model, partial credit model) with dichotomous and polytomous responses, respectively. Residues between expected and observed ability scores were considered as gap parameter in case of dichotomous response. In extended Rasch modeling, response latitudes are considered as an indicator of the gap. Additionally, we performed model fitting, standard error measurement, kernel density estimation, and differential item functioning to test the suitability of Rasch model. Empirical analyses on a number of data set show that proposed analytical method can model the cognitive ADG from dichotomous and polytomous responses. In dichotomous case, the model better fits for mixed responses (combination of easy, medium, and hard) data set rather than monotonic (e.g., only easy) data. Results show that Rasch model can be reliably used to estimate cognitive gap with different cognitive task types.]]></abstract>
##     <issn><![CDATA[2169-3536]]></issn>
##     <htmlFlag><![CDATA[1]]></htmlFlag>
##     <arnumber><![CDATA[6857320]]></arnumber>
##     <doi><![CDATA[10.1109/ACCESS.2014.2339328]]></doi>
##     <publicationId><![CDATA[6857320]]></publicationId>
##     <mdurl><![CDATA[http://ieeexplore.ieee.org/xpl/articleDetails.jsp?tp=&arnumber=6857320&contentType=Journals+%26+Magazines]]></mdurl>
##     <pdf><![CDATA[http://ieeexplore.ieee.org/stamp/stamp.jsp?arnumber=6857320]]></pdf>
##   </document>
##   <document>
##     <rank>339</rank>
##     <title><![CDATA[Classification of Proteomic MS Data as Bayesian Solution of an Inverse Problem]]></title>
##     <authors><![CDATA[Szacherski, P.;  Giovannelli, J.-F.;  Gerfault, L.;  Mahe, P.;  Charrier, J.-P.;  Giremus, A.;  Lacroix, B.;  Grangeat, P.]]></authors>
##     <affiliations><![CDATA[Univ. Grenoble Alpes, Grenoble, France]]></affiliations>
##     <controlledterms>
##       <term><![CDATA[Bayes methods]]></term>
##       <term><![CDATA[biology computing]]></term>
##       <term><![CDATA[cellular biophysics]]></term>
##       <term><![CDATA[inverse problems]]></term>
##       <term><![CDATA[medical computing]]></term>
##       <term><![CDATA[patient diagnosis]]></term>
##       <term><![CDATA[patient treatment]]></term>
##       <term><![CDATA[proteins]]></term>
##       <term><![CDATA[proteomics]]></term>
##     </controlledterms>
##     <thesaurusterms>
##       <term><![CDATA[Biological cells]]></term>
##       <term><![CDATA[Biomedical monitoring]]></term>
##       <term><![CDATA[Biomedical signal processing]]></term>
##       <term><![CDATA[Cells (biology)]]></term>
##       <term><![CDATA[Chromatography]]></term>
##       <term><![CDATA[Classification algorithms]]></term>
##       <term><![CDATA[Inverse problems]]></term>
##       <term><![CDATA[Liquid chromatography]]></term>
##       <term><![CDATA[Probability]]></term>
##       <term><![CDATA[Proteins]]></term>
##       <term><![CDATA[Signal processing]]></term>
##       <term><![CDATA[Statistical analysis]]></term>
##     </thesaurusterms>
##     <pubtitle><![CDATA[Access, IEEE]]></pubtitle>
##     <punumber><![CDATA[6287639]]></punumber>
##     <pubtype><![CDATA[Journals & Magazines]]></pubtype>
##     <publisher><![CDATA[IEEE]]></publisher>
##     <volume><![CDATA[2]]></volume>
##     <py><![CDATA[2014]]></py>
##     <spage><![CDATA[1248]]></spage>
##     <epage><![CDATA[1262]]></epage>
##     <abstract><![CDATA[The cells in an organism emit different amounts of proteins according to their clinical state (healthy/pathological, for instance). The resulting proteomic profile can be used for early detection, diagnosis, and therapy planning. In this paper, we study the classification of a proteomic sample from the point of view of an inverse problem with a joint Bayesian solution, called inversion-classification. We propose a hierarchical physical forward model and present encouraging results from both simulation and clinical data.]]></abstract>
##     <issn><![CDATA[2169-3536]]></issn>
##     <htmlFlag><![CDATA[1]]></htmlFlag>
##     <arnumber><![CDATA[6910218]]></arnumber>
##     <doi><![CDATA[10.1109/ACCESS.2014.2359979]]></doi>
##     <publicationId><![CDATA[6910218]]></publicationId>
##     <mdurl><![CDATA[http://ieeexplore.ieee.org/xpl/articleDetails.jsp?tp=&arnumber=6910218&contentType=Journals+%26+Magazines]]></mdurl>
##     <pdf><![CDATA[http://ieeexplore.ieee.org/stamp/stamp.jsp?arnumber=6910218]]></pdf>
##   </document>
##   <document>
##     <rank>340</rank>
##     <title><![CDATA[Core Mode-Cladding Supermode Modal Interferometer and High-Temperature Sensing Application Based on All-Solid Photonic Bandgap Fiber]]></title>
##     <authors><![CDATA[Tan, X.L.;  Geng, Y.F.;  Li, X.J.;  Yu, Y.Q.;  Deng, Y.L.;  Yin, Z.;  Gao, R.]]></authors>
##     <affiliations><![CDATA[Shenzhen Key Lab. of Sensor Technol., Shenzhen Univ., Shenzhen, China]]></affiliations>
##     <controlledterms>
##       <term><![CDATA[fibre optic sensors]]></term>
##       <term><![CDATA[high-temperature techniques]]></term>
##       <term><![CDATA[light interferometers]]></term>
##       <term><![CDATA[photonic band gap]]></term>
##       <term><![CDATA[thermo-optical devices]]></term>
##     </controlledterms>
##     <thesaurusterms>
##       <term><![CDATA[Interference]]></term>
##       <term><![CDATA[Optical fiber sensors]]></term>
##       <term><![CDATA[Optical interferometry]]></term>
##       <term><![CDATA[Sensitivity]]></term>
##       <term><![CDATA[Temperature measurement]]></term>
##       <term><![CDATA[Temperature sensors]]></term>
##     </thesaurusterms>
##     <pubtitle><![CDATA[Photonics Journal, IEEE]]></pubtitle>
##     <punumber><![CDATA[4563994]]></punumber>
##     <pubtype><![CDATA[Journals & Magazines]]></pubtype>
##     <publisher><![CDATA[IEEE]]></publisher>
##     <volume><![CDATA[6]]></volume>
##     <issue><![CDATA[1]]></issue>
##     <py><![CDATA[2014]]></py>
##     <spage><![CDATA[1]]></spage>
##     <epage><![CDATA[7]]></epage>
##     <abstract><![CDATA[A core-mode-cladding-supermode modal interferometer with all-solid photonic bandgap fiber (AS-PBF) is constructed, and a reflective Michelson-type high-temperature sensor is fabricated. Such a fiber sensor is constituted by a small segment of AS-PBF and a leading single-mode fiber. The splice region of the two fibers is weakly tapered to excite the cladding supermode. Both the interference spectra and the near-field infrared CCD images verify that the LP01 cladding supermode is effectively excited and interferes with the LP01 core mode, which agrees well with theoretical results. Benefiting from a large effective thermooptic coefficient between the two modes, temperature sensitivity up to 0.111 nm/&#x00B0;C at 500 <sup>&#x00B0;</sup>C is obtained in experiment. The proposed sensor is compact and easy to fabricate, which makes it very attractive for high-temperature sensing applications.]]></abstract>
##     <issn><![CDATA[1943-0655]]></issn>
##     <htmlFlag><![CDATA[1]]></htmlFlag>
##     <arnumber><![CDATA[6728608]]></arnumber>
##     <doi><![CDATA[10.1109/JPHOT.2014.2303571]]></doi>
##     <publicationId><![CDATA[6728608]]></publicationId>
##     <mdurl><![CDATA[http://ieeexplore.ieee.org/xpl/articleDetails.jsp?tp=&arnumber=6728608&contentType=Journals+%26+Magazines]]></mdurl>
##     <pdf><![CDATA[http://ieeexplore.ieee.org/stamp/stamp.jsp?arnumber=6728608]]></pdf>
##   </document>
##   <document>
##     <rank>341</rank>
##     <title><![CDATA[Bandwidth Studies on Multimode Polymer Waveguides for <inline-formula> <img src="/images/tex/21804.gif" alt="({\ge }25)"> </inline-formula> Gb/s Optical Interconnects]]></title>
##     <authors><![CDATA[Bamiedakis, N.;  Jian Chen;  Penty, R.V.;  White, I.H.]]></authors>
##     <affiliations><![CDATA[Dept. of Electr. Eng. Div., Univ. of Cambridge, Cambridge, UK]]></affiliations>
##     <controlledterms>
##       <term><![CDATA[optical communication equipment]]></term>
##       <term><![CDATA[optical interconnections]]></term>
##       <term><![CDATA[optical polymers]]></term>
##       <term><![CDATA[optical waveguides]]></term>
##       <term><![CDATA[surface emitting lasers]]></term>
##     </controlledterms>
##     <thesaurusterms>
##       <term><![CDATA[Adaptive optics]]></term>
##       <term><![CDATA[Bandwidth]]></term>
##       <term><![CDATA[Optical interconnections]]></term>
##       <term><![CDATA[Optical waveguides]]></term>
##       <term><![CDATA[Polymers]]></term>
##       <term><![CDATA[Spirals]]></term>
##       <term><![CDATA[Vertical cavity surface emitting lasers]]></term>
##     </thesaurusterms>
##     <pubtitle><![CDATA[Photonics Technology Letters, IEEE]]></pubtitle>
##     <punumber><![CDATA[68]]></punumber>
##     <pubtype><![CDATA[Journals & Magazines]]></pubtype>
##     <publisher><![CDATA[IEEE]]></publisher>
##     <volume><![CDATA[26]]></volume>
##     <issue><![CDATA[20]]></issue>
##     <py><![CDATA[2014]]></py>
##     <spage><![CDATA[2004]]></spage>
##     <epage><![CDATA[2007]]></epage>
##     <abstract><![CDATA[Multimode polymer waveguides constitute a promising technology for use in board-level optical interconnects. However, the continuous improvements in high-speed performance of VCSELs raise important questions about their ability to support such high data rates due to their inherent highly multimoded nature. Thorough experimental studies on the bandwidth of a 1.4-m-long multimode spiral waveguide are presented in this letter, indicating a bandwidth-length product of at least 35 GHz&#x00D7;m even in the case of an overfilled launch. No significant transmission impairments are observed for spatial input offsets, while error-free (BER &lt;; 10<sup>-12</sup>) data transmission over the 1.4-m-long spiral waveguide is demonstrated at 25 Gb/s.]]></abstract>
##     <issn><![CDATA[1041-1135]]></issn>
##     <htmlFlag><![CDATA[1]]></htmlFlag>
##     <arnumber><![CDATA[6866155]]></arnumber>
##     <doi><![CDATA[10.1109/LPT.2014.2342881]]></doi>
##     <publicationId><![CDATA[6866155]]></publicationId>
##     <mdurl><![CDATA[http://ieeexplore.ieee.org/xpl/articleDetails.jsp?tp=&arnumber=6866155&contentType=Journals+%26+Magazines]]></mdurl>
##     <pdf><![CDATA[http://ieeexplore.ieee.org/stamp/stamp.jsp?arnumber=6866155]]></pdf>
##   </document>
##   <document>
##     <rank>342</rank>
##     <title><![CDATA[Challenges, Opportunities, and Future Trends of Emerging Techniques for Augmented Reality-Based Maintenance]]></title>
##     <authors><![CDATA[Lamberti, F.;  Manuri, F.;  Sanna, A.;  Paravati, G.;  Pezzolla, P.;  Montuschi, P.]]></authors>
##     <affiliations><![CDATA[Dipt. di Autom. e Inf., Politec. di Torino, Turin, Italy]]></affiliations>
##     <controlledterms>
##       <term><![CDATA[augmented reality]]></term>
##       <term><![CDATA[maintenance engineering]]></term>
##       <term><![CDATA[production engineering computing]]></term>
##     </controlledterms>
##     <thesaurusterms>
##       <term><![CDATA[Augmented reality]]></term>
##       <term><![CDATA[Cameras]]></term>
##       <term><![CDATA[Maintenance engineering]]></term>
##       <term><![CDATA[Mobile handsets]]></term>
##       <term><![CDATA[Solid modeling]]></term>
##       <term><![CDATA[Technology forecasting]]></term>
##       <term><![CDATA[Three-dimensional displays]]></term>
##     </thesaurusterms>
##     <pubtitle><![CDATA[Emerging Topics in Computing, IEEE Transactions on]]></pubtitle>
##     <punumber><![CDATA[6245516]]></punumber>
##     <pubtype><![CDATA[Journals & Magazines]]></pubtype>
##     <publisher><![CDATA[IEEE]]></publisher>
##     <volume><![CDATA[2]]></volume>
##     <issue><![CDATA[4]]></issue>
##     <py><![CDATA[2014]]></py>
##     <spage><![CDATA[411]]></spage>
##     <epage><![CDATA[421]]></epage>
##     <abstract><![CDATA[Augmented reality (AR) is a well-known technology that can be exploited to provide mass-market users an effective and customizable support in a large spectrum of personal applications, by overlapping computer-generated hints to the real world. Mobile devices, such as smartphones and tablets, are playing a key role in the exponential growth of this kind of solutions. Nonetheless, there exists some application domains that just started to take advantage from the AR systems. Maintenance, repair, and assembly have been considered as strategic fields for the application of the AR technology from the 1990s, but often only specialists using ad hoc hardware were involved in limited experimental tests. Nowadays, AR-based maintenance and repair procedures are available also for end-users on consumer electronics devices. This paper aims to explore new challenges and opportunities of this technology, by also presenting the software framework that is being developed in the EASE-R<sup>3</sup> project by exploiting reconfigurable AR procedures and tele-assistance to overcome some of the limitations of current solutions.]]></abstract>
##     <issn><![CDATA[2168-6750]]></issn>
##     <htmlFlag><![CDATA[1]]></htmlFlag>
##     <arnumber><![CDATA[7004838]]></arnumber>
##     <doi><![CDATA[10.1109/TETC.2014.2368833]]></doi>
##     <publicationId><![CDATA[7004838]]></publicationId>
##     <mdurl><![CDATA[http://ieeexplore.ieee.org/xpl/articleDetails.jsp?tp=&arnumber=7004838&contentType=Journals+%26+Magazines]]></mdurl>
##     <pdf><![CDATA[http://ieeexplore.ieee.org/stamp/stamp.jsp?arnumber=7004838]]></pdf>
##   </document>
##   <document>
##     <rank>343</rank>
##     <title><![CDATA[Sharp Plasmon-Mediated Resonant Reflection From an Undulated Metal Layer]]></title>
##     <authors><![CDATA[Jourlin, Y.;  Tonchev, S.;  Tishchenko, A.V.;  Parriaux, O.]]></authors>
##     <affiliations><![CDATA[Lab. Hubert Curien, Univ. de Lyon, St. Etienne, France]]></affiliations>
##     <controlledterms>
##       <term><![CDATA[X-ray spectra]]></term>
##       <term><![CDATA[gold]]></term>
##       <term><![CDATA[metallic thin films]]></term>
##       <term><![CDATA[optical films]]></term>
##       <term><![CDATA[optical losses]]></term>
##       <term><![CDATA[plasmons]]></term>
##     </controlledterms>
##     <thesaurusterms>
##       <term><![CDATA[Films]]></term>
##       <term><![CDATA[Gold]]></term>
##       <term><![CDATA[Gratings]]></term>
##       <term><![CDATA[Plasmons]]></term>
##       <term><![CDATA[Propagation losses]]></term>
##       <term><![CDATA[Reflection]]></term>
##     </thesaurusterms>
##     <pubtitle><![CDATA[Photonics Journal, IEEE]]></pubtitle>
##     <punumber><![CDATA[4563994]]></punumber>
##     <pubtype><![CDATA[Journals & Magazines]]></pubtype>
##     <publisher><![CDATA[IEEE]]></publisher>
##     <volume><![CDATA[6]]></volume>
##     <issue><![CDATA[5]]></issue>
##     <py><![CDATA[2014]]></py>
##     <spage><![CDATA[1]]></spage>
##     <epage><![CDATA[6]]></epage>
##     <abstract><![CDATA[A close-to-the-theoretically-largest TM resonant reflection of a free-space wave from a free-standing undulated gold layer immersed in a liquid host medium is demonstrated experimentally. It is mediated by the grating-excited long-range plasmon mode propagating along the continuous metal film with particularly low loss.]]></abstract>
##     <issn><![CDATA[1943-0655]]></issn>
##     <htmlFlag><![CDATA[1]]></htmlFlag>
##     <arnumber><![CDATA[6926775]]></arnumber>
##     <doi><![CDATA[10.1109/JPHOT.2014.2361664]]></doi>
##     <publicationId><![CDATA[6926775]]></publicationId>
##     <mdurl><![CDATA[http://ieeexplore.ieee.org/xpl/articleDetails.jsp?tp=&arnumber=6926775&contentType=Journals+%26+Magazines]]></mdurl>
##     <pdf><![CDATA[http://ieeexplore.ieee.org/stamp/stamp.jsp?arnumber=6926775]]></pdf>
##   </document>
##   <document>
##     <rank>344</rank>
##     <title><![CDATA[Reliable and Fast Estimation of Recombination Rates by Convergence Diagnosis and Parallel Markov Chain Monte Carlo]]></title>
##     <authors><![CDATA[Jing Guo;  Jain, R.;  Peng Yang;  Rui Fan;  Chee Keong Kwoh;  Jie Zheng]]></authors>
##     <affiliations><![CDATA[Sch. of Comput. Eng., Nanyang Technol. Univ., Singapore, Singapore]]></affiliations>
##     <controlledterms>
##       <term><![CDATA[Markov processes]]></term>
##       <term><![CDATA[Monte Carlo methods]]></term>
##       <term><![CDATA[biology computing]]></term>
##       <term><![CDATA[genetics]]></term>
##       <term><![CDATA[parallel processing]]></term>
##       <term><![CDATA[statistical analysis]]></term>
##     </controlledterms>
##     <thesaurusterms>
##       <term><![CDATA[Algorithm design and analysis]]></term>
##       <term><![CDATA[Bioinformatics]]></term>
##       <term><![CDATA[Convergence]]></term>
##       <term><![CDATA[Estimation]]></term>
##       <term><![CDATA[Markov processes]]></term>
##       <term><![CDATA[Prediction algorithms]]></term>
##       <term><![CDATA[Program processors]]></term>
##     </thesaurusterms>
##     <pubtitle><![CDATA[Computational Biology and Bioinformatics, IEEE/ACM Transactions on]]></pubtitle>
##     <punumber><![CDATA[8857]]></punumber>
##     <pubtype><![CDATA[Journals & Magazines]]></pubtype>
##     <publisher><![CDATA[IEEE]]></publisher>
##     <volume><![CDATA[11]]></volume>
##     <issue><![CDATA[1]]></issue>
##     <py><![CDATA[2014]]></py>
##     <spage><![CDATA[63]]></spage>
##     <epage><![CDATA[72]]></epage>
##     <abstract><![CDATA[Genetic recombination is an essential event during the process of meiosis resulting in an exchange of segments between paired chromosomes. Estimating recombination rate is crucial for understanding the process of recombination. Experimental methods are normally difficult and limited to small scale estimations. Thus statistical methods using population genetics data are important for large-scale analysis. LDhat is an extensively used statistical method using rjMCMC algorithm to predict recombination rates. Due to the complexity of rjMCMC scheme, LDhat may take a long time for large SNP data sets. In addition, rjMCMC parameters should be manually defined in the original program which directly impact results. To address these issues, we designed an improved algorithm based on LDhat implementing MCMC convergence diagnostic algorithms to automatically predict values of parameters and monitor the mixing process. Then parallel computation methods were employed to further accelerate the new program. The new algorithms have been tested on ten samples from HapMap phase 2 data set. The results were compared with previous code and showed nearly identical output. However, our new methods achieved significant acceleration proving that they are more efficient and reliable for the estimation of recombination rates. The stand-alone package is freely available for download http://www.ntu.edu.sg/home/zhengjie/software/CPLDhat.]]></abstract>
##     <issn><![CDATA[1545-5963]]></issn>
##     <htmlFlag><![CDATA[1]]></htmlFlag>
##     <arnumber><![CDATA[6646172]]></arnumber>
##     <doi><![CDATA[10.1109/TCBB.2013.133]]></doi>
##     <publicationId><![CDATA[6646172]]></publicationId>
##     <mdurl><![CDATA[http://ieeexplore.ieee.org/xpl/articleDetails.jsp?tp=&arnumber=6646172&contentType=Journals+%26+Magazines]]></mdurl>
##     <pdf><![CDATA[http://ieeexplore.ieee.org/stamp/stamp.jsp?arnumber=6646172]]></pdf>
##   </document>
##   <document>
##     <rank>345</rank>
##     <title><![CDATA[Maximum Likelihood Acoustic Factor Analysis Models for Robust Speaker Verification in Noise]]></title>
##     <authors><![CDATA[Hasan, T.;  Hansen, J.H.L.]]></authors>
##     <affiliations><![CDATA[Center for Robust Speech Syst. (CRSS), Univ. of Texas at Dallas, Richardson, TX, USA]]></affiliations>
##     <controlledterms>
##       <term><![CDATA[Gaussian processes]]></term>
##       <term><![CDATA[expectation-maximisation algorithm]]></term>
##       <term><![CDATA[mixture models]]></term>
##       <term><![CDATA[speaker recognition]]></term>
##     </controlledterms>
##     <thesaurusterms>
##       <term><![CDATA[Acoustics]]></term>
##       <term><![CDATA[Analytical models]]></term>
##       <term><![CDATA[Covariance matrices]]></term>
##       <term><![CDATA[Feature extraction]]></term>
##       <term><![CDATA[NIST]]></term>
##       <term><![CDATA[Noise]]></term>
##       <term><![CDATA[Noise measurement]]></term>
##     </thesaurusterms>
##     <pubtitle><![CDATA[Audio, Speech, and Language Processing, IEEE/ACM Transactions on]]></pubtitle>
##     <punumber><![CDATA[6570655]]></punumber>
##     <pubtype><![CDATA[Journals & Magazines]]></pubtype>
##     <publisher><![CDATA[IEEE]]></publisher>
##     <volume><![CDATA[22]]></volume>
##     <issue><![CDATA[2]]></issue>
##     <py><![CDATA[2014]]></py>
##     <spage><![CDATA[381]]></spage>
##     <epage><![CDATA[391]]></epage>
##     <abstract><![CDATA[Recent speaker recognition/verification systems generally utilize an utterance dependent fixed dimensional vector as features to Bayesian classifiers. These vectors, known as i-Vectors, are lower dimensional representations of Gaussian Mixture Model (GMM) mean super-vectors adapted from a Universal Background Model (UBM) using speech utterance features, and extracted utilizing a Factor Analysis (FA) framework. This method is based on the assumption that the speaker dependent information resides in a lower dimensional sub-space. In this study, we utilize a mixture of Acoustic Factor Analyzers (AFA) to model the acoustic features instead of a GMM-UBM. Following our previously proposed AFA technique (&#x201C;Acoustic factor analysis for robust speaker verification,&#x201D; by Hasan and Hansen, IEEE Trans. Audio, Speech, Lang. Process., vol. 21, no. 4, April 2013), this model is based on the assumption that the speaker relevant information lies in a lower dimensional subspace in the multi-dimensional feature space localized by the mixture components. Unlike our previous method, here we train the AFA-UBM model directly from the data using an Expectation-Maximization (EM) algorithm. This method shows improved robustness to noise as the nuisance dimensions are removed in each EM iteration. Two variants of the AFA model are considered utilizing an isotropic and diagonal covariance residual term. The method is integrated within a standard i-Vector system where the hidden variables of the model, termed as acoustic factors, are utilized as the input for total variability modeling. Experimental results obtained on the 2012 National Institute of Standards and Technology (NIST) Speaker Recognition Evaluation (SRE) core-extended trials indicate the effectiveness of the proposed strategy in both clean and noisy conditions.]]></abstract>
##     <issn><![CDATA[2329-9290]]></issn>
##     <htmlFlag><![CDATA[1]]></htmlFlag>
##     <arnumber><![CDATA[6674091]]></arnumber>
##     <doi><![CDATA[10.1109/TASLP.2013.2292356]]></doi>
##     <publicationId><![CDATA[6674091]]></publicationId>
##     <mdurl><![CDATA[http://ieeexplore.ieee.org/xpl/articleDetails.jsp?tp=&arnumber=6674091&contentType=Journals+%26+Magazines]]></mdurl>
##     <pdf><![CDATA[http://ieeexplore.ieee.org/stamp/stamp.jsp?arnumber=6674091]]></pdf>
##   </document>
##   <document>
##     <rank>346</rank>
##     <title><![CDATA[Bilateral PIN Diode for Fast Neutron Dose Measurement]]></title>
##     <authors><![CDATA[Hu, A.Q.;  Yu, M.;  Zhou, C.Z.;  Fan, C.;  Liu, C.C.;  Wang, S.N.;  Shi, B.H.;  Qi, L.;  Wang, J.Y.;  Jin, Y.F.]]></authors>
##     <affiliations><![CDATA[Nat. Key Lab. of Nano/Micro Fabrication Technol., Peking Univ., Beijing, China]]></affiliations>
##     <controlledterms>
##       <term><![CDATA[current distribution]]></term>
##       <term><![CDATA[neutron detection]]></term>
##       <term><![CDATA[p-i-n diodes]]></term>
##       <term><![CDATA[silicon radiation detectors]]></term>
##     </controlledterms>
##     <thesaurusterms>
##       <term><![CDATA[Current density]]></term>
##       <term><![CDATA[Neutrons]]></term>
##       <term><![CDATA[PIN photodiodes]]></term>
##       <term><![CDATA[Semiconductor diodes]]></term>
##       <term><![CDATA[Sensitivity]]></term>
##       <term><![CDATA[Silicon]]></term>
##     </thesaurusterms>
##     <pubtitle><![CDATA[Nuclear Science, IEEE Transactions on]]></pubtitle>
##     <punumber><![CDATA[23]]></punumber>
##     <pubtype><![CDATA[Journals & Magazines]]></pubtype>
##     <publisher><![CDATA[IEEE]]></publisher>
##     <volume><![CDATA[61]]></volume>
##     <issue><![CDATA[3]]></issue>
##     <part><![CDATA[2]]></part>
##     <py><![CDATA[2014]]></py>
##     <spage><![CDATA[1311]]></spage>
##     <epage><![CDATA[1315]]></epage>
##     <abstract><![CDATA[A silicon-based bilateral diode for fast neutron dose measurement is presented to take advantage of vertical and lateral current distributions and to achieve high uniform current distribution. The structure is designed to place rectangle p+ and n+ contacts on each side of the n-Si wafer. Diodes with different structure parameters are fabricated and the sensitivity to neutron dose is measured. It is found that, in this research, the increase in the lateral space between the two contacts can effectively increase sensitivity. Furthermore, the decrease of the contact length and the increase of current density can also increase sensitivity. The measured sensitivity data are verified with the model.]]></abstract>
##     <issn><![CDATA[0018-9499]]></issn>
##     <htmlFlag><![CDATA[1]]></htmlFlag>
##     <arnumber><![CDATA[6819063]]></arnumber>
##     <doi><![CDATA[10.1109/TNS.2014.2317757]]></doi>
##     <publicationId><![CDATA[6819063]]></publicationId>
##     <mdurl><![CDATA[http://ieeexplore.ieee.org/xpl/articleDetails.jsp?tp=&arnumber=6819063&contentType=Journals+%26+Magazines]]></mdurl>
##     <pdf><![CDATA[http://ieeexplore.ieee.org/stamp/stamp.jsp?arnumber=6819063]]></pdf>
##   </document>
##   <document>
##     <rank>347</rank>
##     <title><![CDATA[Reconfigurable Time Slot Interchange Based on Four-Wave Mixing and a Programmable Planar Lightwave Circuit]]></title>
##     <authors><![CDATA[Dizaji, M.R.;  Kostko, I.A.;  Shia, B.;  Callender, C.L.;  Dumais, P.;  Jacob, S.;  Chen, L.R.]]></authors>
##     <affiliations><![CDATA[Dept. of Electr. & Comput. Eng., McGill Univ., Montreal, QC, Canada]]></affiliations>
##     <controlledterms>
##       <term><![CDATA[multiwave mixing]]></term>
##       <term><![CDATA[optical information processing]]></term>
##       <term><![CDATA[optical wavelength conversion]]></term>
##     </controlledterms>
##     <thesaurusterms>
##       <term><![CDATA[Band-pass filters]]></term>
##       <term><![CDATA[Bit error rate]]></term>
##       <term><![CDATA[Delays]]></term>
##       <term><![CDATA[Erbium-doped fiber amplifiers]]></term>
##       <term><![CDATA[Optical buffering]]></term>
##       <term><![CDATA[Optical packet switching]]></term>
##       <term><![CDATA[Optical wavelength conversion]]></term>
##     </thesaurusterms>
##     <pubtitle><![CDATA[Photonics Journal, IEEE]]></pubtitle>
##     <punumber><![CDATA[4563994]]></punumber>
##     <pubtype><![CDATA[Journals & Magazines]]></pubtype>
##     <publisher><![CDATA[IEEE]]></publisher>
##     <volume><![CDATA[6]]></volume>
##     <issue><![CDATA[5]]></issue>
##     <py><![CDATA[2014]]></py>
##     <spage><![CDATA[1]]></spage>
##     <epage><![CDATA[7]]></epage>
##     <abstract><![CDATA[We demonstrate the all-optical reconfigurable time slot interchange (TSI) of individual bits at 40 Gb/s based on four-wave mixing (FWM) and a programmable planar lightwave circuit (PLC). The PLC is used to generate different control signals (masks) that determine which bits undergo TSI. By programming the PLC to generate two different masks, two different TSI patterns are obtained. TSI is achieved using FWM between the data signal and the desired mask with bidirectional propagation in a length of highly nonlinear fiber. Error-free operation is obtained for both of the TSI patterns, with a power penalty &lt;; 5.2 dB, at a bit error rate of 10<sup>-9</sup>.]]></abstract>
##     <issn><![CDATA[1943-0655]]></issn>
##     <htmlFlag><![CDATA[1]]></htmlFlag>
##     <arnumber><![CDATA[6895127]]></arnumber>
##     <doi><![CDATA[10.1109/JPHOT.2014.2356509]]></doi>
##     <publicationId><![CDATA[6895127]]></publicationId>
##     <mdurl><![CDATA[http://ieeexplore.ieee.org/xpl/articleDetails.jsp?tp=&arnumber=6895127&contentType=Journals+%26+Magazines]]></mdurl>
##     <pdf><![CDATA[http://ieeexplore.ieee.org/stamp/stamp.jsp?arnumber=6895127]]></pdf>
##   </document>
##   <document>
##     <rank>348</rank>
##     <title><![CDATA[The Optimal Hard Threshold for Singular Values is <inline-formula> <img src="/images/tex/21744.gif" alt="4/\sqrt {3}"> </inline-formula>]]></title>
##     <authors><![CDATA[Gavish, M.;  Donoho, D.L.]]></authors>
##     <affiliations><![CDATA[Dept. of Stat., Stanford Univ., Stanford, CA, USA]]></affiliations>
##     <controlledterms>
##       <term><![CDATA[mean square error methods]]></term>
##       <term><![CDATA[singular value decomposition]]></term>
##       <term><![CDATA[white noise]]></term>
##     </controlledterms>
##     <thesaurusterms>
##       <term><![CDATA[Approximation methods]]></term>
##       <term><![CDATA[Information theory]]></term>
##       <term><![CDATA[Noise level]]></term>
##       <term><![CDATA[Noise reduction]]></term>
##       <term><![CDATA[Signal to noise ratio]]></term>
##       <term><![CDATA[Vectors]]></term>
##       <term><![CDATA[White noise]]></term>
##     </thesaurusterms>
##     <pubtitle><![CDATA[Information Theory, IEEE Transactions on]]></pubtitle>
##     <punumber><![CDATA[18]]></punumber>
##     <pubtype><![CDATA[Journals & Magazines]]></pubtype>
##     <publisher><![CDATA[IEEE]]></publisher>
##     <volume><![CDATA[60]]></volume>
##     <issue><![CDATA[8]]></issue>
##     <py><![CDATA[2014]]></py>
##     <spage><![CDATA[5040]]></spage>
##     <epage><![CDATA[5053]]></epage>
##     <abstract><![CDATA[We consider recovery of low-rank matrices from noisy data by hard thresholding of singular values, in which empirical singular values below a threshold &#x03BB; are set to 0. We study the asymptotic mean squared error (AMSE) in a framework, where the matrix size is large compared with the rank of the matrix to be recovered, and the signal-to-noise ratio of the low-rank piece stays constant. The AMSE-optimal choice of hard threshold, in the case of n-by-n matrix in white noise of level &#x03C3;, is simply (4/&#x221A;3)&#x221A;n&#x03C3; &#x2248; 2.309&#x221A;n&#x03C3; when &#x03C3; is known, or simply 2.858 &#x00B7; y<sub>med</sub> when &#x03C3; is unknown, where y<sub>med</sub> is the median empirical singular value. For nonsquare, m by n matrices with m &#x2260; n the thresholding coefficients 4/&#x221A;3 and 2.858 are replaced with different provided constants that depend on m/n. Asymptotically, this thresholding rule adapts to unknown rank and unknown noise level in an optimal manner: it is always better than hard thresholding at any other value, and is always better than ideal truncated singular value decomposition (TSVD), which truncates at the true rank of the low-rank matrix we are trying to recover. Hard thresholding at the recommended value to recover an n-by-n matrix of rank r guarantees an AMSE at most 3 nr&#x03C3;<sup>2</sup>. In comparison, the guarantees provided by TSVD, optimally tuned singular value soft thresholding and the best guarantee achievable by any shrinkage of the data singular values are 5 nr&#x03C3;<sup>2</sup>, 6 nr&#x03C3;<sup>2</sup>, and 2 nr&#x03C3;<sup>2</sup>, respectively. The recommended value for hard threshold also offers, among hard thresholds, the best possible AMSE guarantees for recovering matrices with bounded nuclear norm. Empirical evidence suggests that performance improvement over TSVD and other popular shrinkage rules can be substantial, for different noise distributions, even in relatively small n.]]></abstract>
##     <issn><![CDATA[0018-9448]]></issn>
##     <htmlFlag><![CDATA[1]]></htmlFlag>
##     <arnumber><![CDATA[6846297]]></arnumber>
##     <doi><![CDATA[10.1109/TIT.2014.2323359]]></doi>
##     <publicationId><![CDATA[6846297]]></publicationId>
##     <mdurl><![CDATA[http://ieeexplore.ieee.org/xpl/articleDetails.jsp?tp=&arnumber=6846297&contentType=Journals+%26+Magazines]]></mdurl>
##     <pdf><![CDATA[http://ieeexplore.ieee.org/stamp/stamp.jsp?arnumber=6846297]]></pdf>
##   </document>
##   <document>
##     <rank>349</rank>
##     <title><![CDATA[Half-Wave-Coupled Ring-FP Laser With 50-Channel 100-GHz-Spaced Wavelength Tuning]]></title>
##     <authors><![CDATA[Lin Wu;  Zhipeng Hu;  Xiaolu Liao;  Jian-Jun He]]></authors>
##     <affiliations><![CDATA[Dept. of Opt. Eng., Zhejiang Univ., Hangzhou, China]]></affiliations>
##     <controlledterms>
##       <term><![CDATA[laser tuning]]></term>
##       <term><![CDATA[optical modulation]]></term>
##       <term><![CDATA[semiconductor lasers]]></term>
##       <term><![CDATA[thermo-optical effects]]></term>
##       <term><![CDATA[thermoelectric cooling]]></term>
##     </controlledterms>
##     <thesaurusterms>
##       <term><![CDATA[Couplers]]></term>
##       <term><![CDATA[Electrodes]]></term>
##       <term><![CDATA[Laser modes]]></term>
##       <term><![CDATA[Laser tuning]]></term>
##       <term><![CDATA[Ring lasers]]></term>
##       <term><![CDATA[Waveguide lasers]]></term>
##     </thesaurusterms>
##     <pubtitle><![CDATA[Photonics Journal, IEEE]]></pubtitle>
##     <punumber><![CDATA[4563994]]></punumber>
##     <pubtype><![CDATA[Journals & Magazines]]></pubtype>
##     <publisher><![CDATA[IEEE]]></publisher>
##     <volume><![CDATA[6]]></volume>
##     <issue><![CDATA[4]]></issue>
##     <py><![CDATA[2014]]></py>
##     <spage><![CDATA[1]]></spage>
##     <epage><![CDATA[8]]></epage>
##     <abstract><![CDATA[We report our latest experimental results of the half-wave-coupled rectangular ring-FP laser. All the waveguides of the device are active, and the thermooptic effect is employed to extend the tuning range beyond the free-spectral-range limit. Singleelectrode tuning of 20 channels with 100-GHz spacing is obtained. By combining five single-electrode-controlled tuning curves with different bias currents on fixed electrodes and with the thermoelectric cooler fixed at 20 &#x00B0;C, wavelength tuning of 50 channels with 100-GHz spacing covering the L-band is achieved with a side-mode suppression ratio up to 41 dB. Direct modulation at 2.5 Gb/s is also demonstrated. The simple and compact ring-FP laser has great potential as a low-cost tunable laser for many applications.]]></abstract>
##     <issn><![CDATA[1943-0655]]></issn>
##     <htmlFlag><![CDATA[1]]></htmlFlag>
##     <arnumber><![CDATA[6872520]]></arnumber>
##     <doi><![CDATA[10.1109/JPHOT.2014.2345876]]></doi>
##     <publicationId><![CDATA[6872520]]></publicationId>
##     <mdurl><![CDATA[http://ieeexplore.ieee.org/xpl/articleDetails.jsp?tp=&arnumber=6872520&contentType=Journals+%26+Magazines]]></mdurl>
##     <pdf><![CDATA[http://ieeexplore.ieee.org/stamp/stamp.jsp?arnumber=6872520]]></pdf>
##   </document>
##   <document>
##     <rank>350</rank>
##     <title><![CDATA[Modeling Hotspots of Climate Change in the Sahel Using Object-Based Regionalization of Multidimensional Gridded Datasets]]></title>
##     <authors><![CDATA[Hagenlocher, M.;  Lang, S.;  Holbling, D.;  Tiede, D.;  Kienberger, S.]]></authors>
##     <controlledterms>
##       <term><![CDATA[atmospheric techniques]]></term>
##       <term><![CDATA[atmospheric temperature]]></term>
##       <term><![CDATA[climatology]]></term>
##       <term><![CDATA[geophysical image processing]]></term>
##       <term><![CDATA[image segmentation]]></term>
##       <term><![CDATA[remote sensing]]></term>
##     </controlledterms>
##     <thesaurusterms>
##       <term><![CDATA[Africa]]></term>
##       <term><![CDATA[Earth]]></term>
##       <term><![CDATA[Indexes]]></term>
##       <term><![CDATA[Market research]]></term>
##       <term><![CDATA[Meteorology]]></term>
##       <term><![CDATA[Remote sensing]]></term>
##       <term><![CDATA[Sociology]]></term>
##     </thesaurusterms>
##     <pubtitle><![CDATA[Selected Topics in Applied Earth Observations and Remote Sensing, IEEE Journal of]]></pubtitle>
##     <punumber><![CDATA[4609443]]></punumber>
##     <pubtype><![CDATA[Journals & Magazines]]></pubtype>
##     <publisher><![CDATA[IEEE]]></publisher>
##     <volume><![CDATA[7]]></volume>
##     <issue><![CDATA[1]]></issue>
##     <py><![CDATA[2014]]></py>
##     <spage><![CDATA[229]]></spage>
##     <epage><![CDATA[234]]></epage>
##     <abstract><![CDATA[The population of subsaharan Africa, and particularly of the countries of the Sahel and western Africa, is one of the most vulnerable to climate change and climate-related extreme events. To provide updated information for targeted climate change adaptation measures, we modeled hotspots of climate change and related extreme events in an integrative manner. This was achieved by constructing a spatial composite indicator of cumulative climate change impact, which integrates four climate- and hazard-related subindicators: seasonal temperature trends, seasonal precipitation trends, drought occurrences, and major flood events. The analysis is based on time-series of freely available continuous, gridded geo-spatial datasets, including remote sensing data. The aggregation of the four subindicators was performed by making use of a regionalization approach, based on segmentation techniques widely used in the remote sensing community in the field of object-based image analysis. Following the approach presented in this paper, 19 hotspots with most severe climatic changes were identified, evaluated, and mapped. The method enables not only the prioritization of intervention areas, but also allows decomposing the identified hotspots into their underlying subindicators, and thus additional information for effective climate change adaptation measures can be provided.]]></abstract>
##     <issn><![CDATA[1939-1404]]></issn>
##     <htmlFlag><![CDATA[1]]></htmlFlag>
##     <arnumber><![CDATA[6515704]]></arnumber>
##     <doi><![CDATA[10.1109/JSTARS.2013.2259579]]></doi>
##     <publicationId><![CDATA[6515704]]></publicationId>
##     <mdurl><![CDATA[http://ieeexplore.ieee.org/xpl/articleDetails.jsp?tp=&arnumber=6515704&contentType=Journals+%26+Magazines]]></mdurl>
##     <pdf><![CDATA[http://ieeexplore.ieee.org/stamp/stamp.jsp?arnumber=6515704]]></pdf>
##   </document>
##   <document>
##     <rank>351</rank>
##     <title><![CDATA[Tunable Bandpass Filter With Two Adjustable Transmission Poles and Compensable Coupling]]></title>
##     <authors><![CDATA[Xun Luo;  Sheng Sun;  Staszewski, R.B.]]></authors>
##     <affiliations><![CDATA[Microwave Branch, Huawei Technol. Co. Ltd., Shenzhen, China]]></affiliations>
##     <controlledterms>
##       <term><![CDATA[UHF filters]]></term>
##       <term><![CDATA[band-pass filters]]></term>
##       <term><![CDATA[microstrip filters]]></term>
##       <term><![CDATA[varactors]]></term>
##     </controlledterms>
##     <thesaurusterms>
##       <term><![CDATA[Bandwidth]]></term>
##       <term><![CDATA[Couplings]]></term>
##       <term><![CDATA[Microstrip]]></term>
##       <term><![CDATA[Passband]]></term>
##       <term><![CDATA[Resonant frequency]]></term>
##       <term><![CDATA[Tuning]]></term>
##       <term><![CDATA[Varactors]]></term>
##     </thesaurusterms>
##     <pubtitle><![CDATA[Microwave Theory and Techniques, IEEE Transactions on]]></pubtitle>
##     <punumber><![CDATA[22]]></punumber>
##     <pubtype><![CDATA[Journals & Magazines]]></pubtype>
##     <publisher><![CDATA[IEEE]]></publisher>
##     <volume><![CDATA[62]]></volume>
##     <issue><![CDATA[9]]></issue>
##     <py><![CDATA[2014]]></py>
##     <spage><![CDATA[2003]]></spage>
##     <epage><![CDATA[2013]]></epage>
##     <abstract><![CDATA[In this paper, tunable microstrip bandpass filters with two adjustable transmission poles and compensable coupling are proposed. The fundamental structure is based on a half-wavelength (&#x03BB;/2) resonator with a center-tapped open-stub. Microwave varactors placed at various internal nodes separately adjust the filter's center frequency and bandwidth over a wide tuning range. The constant absolute bandwidth is achieved at different center frequencies by maintaining the distance between the in-band transmission poles. Meanwhile, the coupling strength could be compensable by tuning varactors that are side and embedding loaded in the parallel coupled microstrip lines (PCMLs). As a demonstrator, a second-order filter with seven tuning varactors is implemented and verified. A frequency range of 0.58-0.91 GHz with a 1-dB bandwidth tuning from 115 to 315 MHz (i.e., 12.6%-54.3% fractional bandwidth) is demonstrated. Specifically, the return loss of passbands with different operating center frequencies can be achieved with same level, i.e., about 13.1 and 11.6 dB for narrow and wide passband responses, respectively. To further verify the etch-tolerance characteristics of the proposed prototype filter, another second-order filter with nine tuning varactors is proposed and fabricated. The measured results exhibit that the tunable fitler with the embedded varactor-loaded PCML has less sensitivity to fabrication tolerances. Meanwhile, the passband return loss can be achieved with same level of 20 dB for narrow and wide passband responses, respectively.]]></abstract>
##     <issn><![CDATA[0018-9480]]></issn>
##     <htmlFlag><![CDATA[1]]></htmlFlag>
##     <arnumber><![CDATA[6862071]]></arnumber>
##     <doi><![CDATA[10.1109/TMTT.2014.2337287]]></doi>
##     <publicationId><![CDATA[6862071]]></publicationId>
##     <mdurl><![CDATA[http://ieeexplore.ieee.org/xpl/articleDetails.jsp?tp=&arnumber=6862071&contentType=Journals+%26+Magazines]]></mdurl>
##     <pdf><![CDATA[http://ieeexplore.ieee.org/stamp/stamp.jsp?arnumber=6862071]]></pdf>
##   </document>
##   <document>
##     <rank>352</rank>
##     <title><![CDATA[Ball Lightning Events Explained as Self-Stable Spinning High-Density Plasma Toroids or Atmospheric Spheromaks]]></title>
##     <authors><![CDATA[Seward, C.]]></authors>
##     <affiliations><![CDATA[Electron Power Syst., Inc., Acton, MA, USA]]></affiliations>
##     <controlledterms>
##       <term><![CDATA[ion density]]></term>
##       <term><![CDATA[lightning]]></term>
##       <term><![CDATA[plasma density]]></term>
##       <term><![CDATA[reversed field pinch]]></term>
##     </controlledterms>
##     <thesaurusterms>
##       <term><![CDATA[Atmospheric modeling]]></term>
##       <term><![CDATA[Ball lightning]]></term>
##       <term><![CDATA[Ions]]></term>
##       <term><![CDATA[Lightning]]></term>
##       <term><![CDATA[Plasmas]]></term>
##       <term><![CDATA[Toroidal magnetic fields]]></term>
##     </thesaurusterms>
##     <pubtitle><![CDATA[Access, IEEE]]></pubtitle>
##     <punumber><![CDATA[6287639]]></punumber>
##     <pubtype><![CDATA[Journals & Magazines]]></pubtype>
##     <publisher><![CDATA[IEEE]]></publisher>
##     <volume><![CDATA[2]]></volume>
##     <py><![CDATA[2014]]></py>
##     <spage><![CDATA[153]]></spage>
##     <epage><![CDATA[159]]></epage>
##     <abstract><![CDATA[Spinning plasma toroids, or spinning spheromaks, are reported as forming in partial atmosphere during high-power electric arc experiments. They are a new class of spheromaks because they are observed to be stable in partial atmosphere with no confining external toroidal magnetic fields, and are observed to endure for more than 600 ms. Included in this paper is a model that explains these stable plasma toroids (spheromaks); they are hollow plasma toroids with a thin outer shell of electrons and ions that all travel in parallel paths orthogonal to the toroid circumference - in effect, spiraling around the toroid. These toroids include sufficient ions to neutralize the space charge of the electrons. This model leads to the name Electron Spiral Toroid Spheromak (ESTS). The discovery of this new class of spheromaks resulted from work to explain ball lightning. A comparison is made between the experimental observations of spheromaks in partial atmosphere and reported ball lightning observations; strong similarities are reported. The ESTS is also found to have a high ion density of ~ 10<sup>19</sup> ions/cm<sup>3</sup> without needing any external toroidal magnetic field for containment, compared, for example, to tokamaks, with ion density limits of ~ 10<sup>15</sup> ions/cm<sup>3</sup>. This high ion density is a defining characteristic and opens the potential to be useful in applications. The ESTS is a field reversed configuration plasma toroid.]]></abstract>
##     <issn><![CDATA[2169-3536]]></issn>
##     <htmlFlag><![CDATA[1]]></htmlFlag>
##     <arnumber><![CDATA[6748850]]></arnumber>
##     <doi><![CDATA[10.1109/ACCESS.2014.2308476]]></doi>
##     <publicationId><![CDATA[6748850]]></publicationId>
##     <mdurl><![CDATA[http://ieeexplore.ieee.org/xpl/articleDetails.jsp?tp=&arnumber=6748850&contentType=Journals+%26+Magazines]]></mdurl>
##     <pdf><![CDATA[http://ieeexplore.ieee.org/stamp/stamp.jsp?arnumber=6748850]]></pdf>
##   </document>
##   <document>
##     <rank>353</rank>
##     <title><![CDATA[Energy Efficiency Benefits of RAN-as-a-Service Concept for a Cloud-Based 5G Mobile Network Infrastructure]]></title>
##     <authors><![CDATA[Sabella, D.;  De Domenico, A.;  Katranaras, E.;  Imran, M.A.;  di Girolamo, M.;  Salim, U.;  Lalam, M.;  Samdanis, K.;  Maeder, A.]]></authors>
##     <affiliations><![CDATA[Telecom Italia, Turin, Italy]]></affiliations>
##     <controlledterms>
##       <term><![CDATA[5G mobile communication]]></term>
##       <term><![CDATA[Long Term Evolution]]></term>
##       <term><![CDATA[cloud computing]]></term>
##       <term><![CDATA[energy conservation]]></term>
##       <term><![CDATA[mobile computing]]></term>
##       <term><![CDATA[radio access networks]]></term>
##     </controlledterms>
##     <thesaurusterms>
##       <term><![CDATA[5G mobile communication]]></term>
##       <term><![CDATA[Cloud computing]]></term>
##       <term><![CDATA[Energy efficiency]]></term>
##       <term><![CDATA[Long Term Evolution]]></term>
##       <term><![CDATA[Mobile communication]]></term>
##       <term><![CDATA[Network architecture]]></term>
##       <term><![CDATA[Radio access network]]></term>
##       <term><![CDATA[Virtualization]]></term>
##     </thesaurusterms>
##     <pubtitle><![CDATA[Access, IEEE]]></pubtitle>
##     <punumber><![CDATA[6287639]]></punumber>
##     <pubtype><![CDATA[Journals & Magazines]]></pubtype>
##     <publisher><![CDATA[IEEE]]></publisher>
##     <volume><![CDATA[2]]></volume>
##     <py><![CDATA[2014]]></py>
##     <spage><![CDATA[1586]]></spage>
##     <epage><![CDATA[1597]]></epage>
##     <abstract><![CDATA[This paper focuses on energy efficiency aspects and related benefits of radio-access-network-as-a-service (RANaaS) implementation (using commodity hardware) as architectural evolution of LTE-advanced networks toward 5G infrastructure. RANaaS is a novel concept introduced recently, which enables the partial centralization of RAN functionalities depending on the actual needs as well as on network characteristics. In the view of future definition of 5G systems, this cloud-based design is an important solution in terms of efficient usage of network resources. The aim of this paper is to give a vision of the advantages of the RANaaS, to present its benefits in terms of energy efficiency and to propose a consistent system-level power model as a reference for assessing innovative functionalities toward 5G systems. The incremental benefits through the years are also discussed in perspective, by considering technological evolution of IT platforms and the increasing matching between their capabilities and the need for progressive virtualization of RAN functionalities. The description is complemented by an exemplary evaluation in terms of energy efficiency, analyzing the achievable gains associated with the RANaaS paradigm.]]></abstract>
##     <issn><![CDATA[2169-3536]]></issn>
##     <htmlFlag><![CDATA[1]]></htmlFlag>
##     <arnumber><![CDATA[6990725]]></arnumber>
##     <doi><![CDATA[10.1109/ACCESS.2014.2381215]]></doi>
##     <publicationId><![CDATA[6990725]]></publicationId>
##     <mdurl><![CDATA[http://ieeexplore.ieee.org/xpl/articleDetails.jsp?tp=&arnumber=6990725&contentType=Journals+%26+Magazines]]></mdurl>
##     <pdf><![CDATA[http://ieeexplore.ieee.org/stamp/stamp.jsp?arnumber=6990725]]></pdf>
##   </document>
##   <document>
##     <rank>354</rank>
##     <title><![CDATA[Exact Analytical Solution for the Mutual Compensation of Astigmatism Using Curved Mirrors in a Folded Resonator Laser]]></title>
##     <authors><![CDATA[Wen Qiao;  Liang Guowen;  Zhang Xiaojun;  Liang Zongsen;  Wang Yonggang;  Li Ji;  Niu Hanben]]></authors>
##     <affiliations><![CDATA[Key Lab. of Optoelectron. Devices & Syst. of the Minist. of Educ. & Guangdong Province, Shenzhen Univ., Shenzhen, China]]></affiliations>
##     <controlledterms>
##       <term><![CDATA[aberrations]]></term>
##       <term><![CDATA[laser beams]]></term>
##       <term><![CDATA[laser cavity resonators]]></term>
##       <term><![CDATA[laser mirrors]]></term>
##       <term><![CDATA[neodymium]]></term>
##       <term><![CDATA[solid lasers]]></term>
##     </controlledterms>
##     <thesaurusterms>
##       <term><![CDATA[Laser beams]]></term>
##       <term><![CDATA[Laser cavity resonators]]></term>
##       <term><![CDATA[Laser noise]]></term>
##       <term><![CDATA[Lenses]]></term>
##       <term><![CDATA[Mirrors]]></term>
##       <term><![CDATA[Optical resonators]]></term>
##     </thesaurusterms>
##     <pubtitle><![CDATA[Photonics Journal, IEEE]]></pubtitle>
##     <punumber><![CDATA[4563994]]></punumber>
##     <pubtype><![CDATA[Journals & Magazines]]></pubtype>
##     <publisher><![CDATA[IEEE]]></publisher>
##     <volume><![CDATA[6]]></volume>
##     <issue><![CDATA[6]]></issue>
##     <py><![CDATA[2014]]></py>
##     <spage><![CDATA[1]]></spage>
##     <epage><![CDATA[13]]></epage>
##     <abstract><![CDATA[For the first time, to the best of our knowledge, analytical expressions for mutually compensating the curved mirror astigmatism in the two terminal arms of a folded resonator are deduced and experimentally verified. The analytical expressions are derived using the theory of the propagation and transformation of Gaussian beams under few assumptions. Our exact analytical expressions describe the necessary and sufficient conditions for simultaneously compensating astigmatism in two arms of a sequence of two off-axis curved mirrors. The theoretical results indicate that, when the astigmatism compensation expressions are satisfied, the astigmatism introduced by a folded curved mirror can be mutually compensated by another folded curved mirror, even if there is no additional Brewster element inside the cavity. The astigmatism can only be successfully eliminated by using a pair of concave mirrors or convex mirrors. Our analytical expressions can be used to design astigmatically compensated folded resonators without a Brewster element. A typical side-pumped z-shaped cavity Nd:YAG laser is employed to demonstrate the astigmatic compensation. Experimental measurements of the pattern of the laser output beam show that not only the spot intensity profile deformation but, in addition, the phase distortion in the two terminal arms can be simultaneously compensated completely in the cavity, which is in good agreement with the analytical predictions.]]></abstract>
##     <issn><![CDATA[1943-0655]]></issn>
##     <htmlFlag><![CDATA[1]]></htmlFlag>
##     <arnumber><![CDATA[6945338]]></arnumber>
##     <doi><![CDATA[10.1109/JPHOT.2014.2366160]]></doi>
##     <publicationId><![CDATA[6945338]]></publicationId>
##     <mdurl><![CDATA[http://ieeexplore.ieee.org/xpl/articleDetails.jsp?tp=&arnumber=6945338&contentType=Journals+%26+Magazines]]></mdurl>
##     <pdf><![CDATA[http://ieeexplore.ieee.org/stamp/stamp.jsp?arnumber=6945338]]></pdf>
##   </document>
##   <document>
##     <rank>355</rank>
##     <title><![CDATA[Through-Silicon Hole Interposers for 3-D IC Integration]]></title>
##     <authors><![CDATA[Lau, J.H.;  Ching-Kuan Lee;  Chau-Jie Zhan;  Sheng-Tsai Wu;  Yu-Lin Chao;  Ming-Ji Dai;  Ra-Min Tain;  Heng-Chieh Chien;  Jui-Feng Hung;  Chun-Hsien Chien;  Ren-Shing Cheng;  Yu-wei Huang;  Yu-Mei Cheng;  Li-Ling Liao;  Wei-Chung Lo;  Ming-Jer Kao]]></authors>
##     <affiliations><![CDATA[Electron. & Optoelectron. Res. Lab., Ind. Technol. Res. Inst., Hsinchu, Taiwan]]></affiliations>
##     <controlledterms>
##       <term><![CDATA[system-in-package]]></term>
##       <term><![CDATA[test equipment]]></term>
##       <term><![CDATA[three-dimensional integrated circuits]]></term>
##     </controlledterms>
##     <thesaurusterms>
##       <term><![CDATA[Resists]]></term>
##       <term><![CDATA[Silicon]]></term>
##       <term><![CDATA[Substrates]]></term>
##       <term><![CDATA[Through-silicon vias]]></term>
##       <term><![CDATA[Tin]]></term>
##       <term><![CDATA[Vehicles]]></term>
##     </thesaurusterms>
##     <pubtitle><![CDATA[Components, Packaging and Manufacturing Technology, IEEE Transactions on]]></pubtitle>
##     <punumber><![CDATA[5503870]]></punumber>
##     <pubtype><![CDATA[Journals & Magazines]]></pubtype>
##     <publisher><![CDATA[IEEE]]></publisher>
##     <volume><![CDATA[4]]></volume>
##     <issue><![CDATA[9]]></issue>
##     <py><![CDATA[2014]]></py>
##     <spage><![CDATA[1407]]></spage>
##     <epage><![CDATA[1419]]></epage>
##     <abstract><![CDATA[In this investigation, a system-in-package (SiP) that consists of a very low-cost interposer with through-silicon holes (TSHs) and with chips on its top and bottom sides (a real 3-D IC integration) is studied. Emphasis is placed on the fabrication of a test vehicle to demonstrate the feasibility of this SiP technology. The design, materials, and process of the top chip, bottom chip, TSH interposer, and final assembly will be presented. Shock and thermal cycling tests will be performed to demonstrate the integrity of the SiP structure.]]></abstract>
##     <issn><![CDATA[2156-3950]]></issn>
##     <htmlFlag><![CDATA[1]]></htmlFlag>
##     <arnumber><![CDATA[6877650]]></arnumber>
##     <doi><![CDATA[10.1109/TCPMT.2014.2339832]]></doi>
##     <publicationId><![CDATA[6877650]]></publicationId>
##     <mdurl><![CDATA[http://ieeexplore.ieee.org/xpl/articleDetails.jsp?tp=&arnumber=6877650&contentType=Journals+%26+Magazines]]></mdurl>
##     <pdf><![CDATA[http://ieeexplore.ieee.org/stamp/stamp.jsp?arnumber=6877650]]></pdf>
##   </document>
##   <document>
##     <rank>356</rank>
##     <title><![CDATA[Investigating the Applicability of Cartosat-1 DEMs and Topographic Maps to Localize Large-Area Urban Mass Concentrations]]></title>
##     <authors><![CDATA[Wurm, M.;  dAngelo, P.;  Reinartz, P.;  Taubenbock, H.]]></authors>
##     <affiliations><![CDATA[Deutsches Fernerkundungsdatenzentrum, Deutsches Zentrum fur Luft- und Raumfahrt, Wessling, Germany]]></affiliations>
##     <controlledterms>
##       <term><![CDATA[cartography]]></term>
##       <term><![CDATA[digital elevation models]]></term>
##       <term><![CDATA[geophysical image processing]]></term>
##       <term><![CDATA[image classification]]></term>
##       <term><![CDATA[optical radar]]></term>
##       <term><![CDATA[remote sensing by radar]]></term>
##       <term><![CDATA[spaceborne radar]]></term>
##       <term><![CDATA[stereo image processing]]></term>
##       <term><![CDATA[topography (Earth)]]></term>
##     </controlledterms>
##     <thesaurusterms>
##       <term><![CDATA[Accuracy]]></term>
##       <term><![CDATA[Buildings]]></term>
##       <term><![CDATA[Cities and towns]]></term>
##       <term><![CDATA[Data mining]]></term>
##       <term><![CDATA[Data models]]></term>
##       <term><![CDATA[Remote sensing]]></term>
##       <term><![CDATA[Urban areas]]></term>
##     </thesaurusterms>
##     <pubtitle><![CDATA[Selected Topics in Applied Earth Observations and Remote Sensing, IEEE Journal of]]></pubtitle>
##     <punumber><![CDATA[4609443]]></punumber>
##     <pubtype><![CDATA[Journals & Magazines]]></pubtype>
##     <publisher><![CDATA[IEEE]]></publisher>
##     <volume><![CDATA[7]]></volume>
##     <issue><![CDATA[10]]></issue>
##     <py><![CDATA[2014]]></py>
##     <spage><![CDATA[4138]]></spage>
##     <epage><![CDATA[4152]]></epage>
##     <abstract><![CDATA[Building models are a valuable information source for urban studies and in particular for analyses of urban mass concentrations (UMCS). Most commonly, light detection and ranging (LiDAR) is used for their generation. The trade-off for the high geometric detail of these data is the low spatial coverage, comparably high costs and low actualization rates. Spaceborne stereo data from Cartosat-1 are able to cover large areas on the one hand, but hold a lower geometric resolution on the other hand. In this paper, we investigate to which extent the geometric shortcomings of Cartosat-1 can be overcome integrating building footprints from topographic maps for the derivation of large-area building models. Therefore, we describe the methodology to derive digital surface models (DSMs) from Cartosat-1 data and the derivation of building footprints from topographic maps at 1:25 000 (DTK25). Both data are fused to generate building block models for four metropolitan regions in Germany with an area of ~ 16 000 km<sup>2</sup>. Building block models are further aggregated to 1 &#x00D7; 1 km grid cells and volume densities are computed. Volume densities are classified to various levels of UMCs. Performance evaluation of the building block models reveals that the building footprints are larger in the DTK-25, and building heights are lower with a mean absolute error of 3.21 m. Both factors influence the building volume, which is linearly lower than the reference. However, this error does not affect the classification of UMC, which can be classified with accuracies between 77% and 97%.]]></abstract>
##     <issn><![CDATA[1939-1404]]></issn>
##     <htmlFlag><![CDATA[1]]></htmlFlag>
##     <arnumber><![CDATA[6887291]]></arnumber>
##     <doi><![CDATA[10.1109/JSTARS.2014.2346655]]></doi>
##     <publicationId><![CDATA[6887291]]></publicationId>
##     <mdurl><![CDATA[http://ieeexplore.ieee.org/xpl/articleDetails.jsp?tp=&arnumber=6887291&contentType=Journals+%26+Magazines]]></mdurl>
##     <pdf><![CDATA[http://ieeexplore.ieee.org/stamp/stamp.jsp?arnumber=6887291]]></pdf>
##   </document>
##   <document>
##     <rank>357</rank>
##     <title><![CDATA[Comparison of Feedback Control Techniques for Torque-Vectoring Control of Fully Electric Vehicles]]></title>
##     <authors><![CDATA[De Novellis, L.;  Sorniotti, A.;  Gruber, P.;  Pennycott, A.]]></authors>
##     <affiliations><![CDATA[Univ. of Surrey, Guildford, UK]]></affiliations>
##     <controlledterms>
##       <term><![CDATA[adaptive control]]></term>
##       <term><![CDATA[electric vehicles]]></term>
##       <term><![CDATA[feedback]]></term>
##       <term><![CDATA[feedforward]]></term>
##       <term><![CDATA[power transmission (mechanical)]]></term>
##       <term><![CDATA[road safety]]></term>
##       <term><![CDATA[three-term control]]></term>
##       <term><![CDATA[torque control]]></term>
##       <term><![CDATA[variable structure systems]]></term>
##     </controlledterms>
##     <thesaurusterms>
##       <term><![CDATA[Acceleration]]></term>
##       <term><![CDATA[Feedforward neural networks]]></term>
##       <term><![CDATA[Friction]]></term>
##       <term><![CDATA[Mathematical model]]></term>
##       <term><![CDATA[Tires]]></term>
##       <term><![CDATA[Vehicle dynamics]]></term>
##       <term><![CDATA[Vehicles]]></term>
##     </thesaurusterms>
##     <pubtitle><![CDATA[Vehicular Technology, IEEE Transactions on]]></pubtitle>
##     <punumber><![CDATA[25]]></punumber>
##     <pubtype><![CDATA[Journals & Magazines]]></pubtype>
##     <publisher><![CDATA[IEEE]]></publisher>
##     <volume><![CDATA[63]]></volume>
##     <issue><![CDATA[8]]></issue>
##     <py><![CDATA[2014]]></py>
##     <spage><![CDATA[3612]]></spage>
##     <epage><![CDATA[3623]]></epage>
##     <abstract><![CDATA[Fully electric vehicles (FEVs) with individually controlled powertrains can significantly enhance vehicle response to steering-wheel inputs in both steady-state and transient conditions, thereby improving vehicle handling and, thus, active safety and the fun-to-drive element. This paper presents a comparison between different torque-vectoring control structures for the yaw moment control of FEVs. Two second-order sliding-mode controllers are evaluated against a feedforward controller combined with either a conventional or an adaptive proportional-integral-derivative (PID) controller. Furthermore, the potential performance and robustness benefits arising from the integration of a body sideslip controller with the yaw rate feedback control system are assessed. The results show that all the evaluated controllers are able to significantly change the understeer behavior with respect to the baseline vehicle. The PID-based controllers achieve very good vehicle performance in steady-state and transient conditions, whereas the controllers based on the sliding-mode approach demonstrate a high level of robustness against variations in the vehicle parameters. The integrated sideslip controller effectively maintains the sideslip angle within acceptable limits in the case of an erroneous estimation of the tire-road friction coefficient.]]></abstract>
##     <issn><![CDATA[0018-9545]]></issn>
##     <htmlFlag><![CDATA[1]]></htmlFlag>
##     <arnumber><![CDATA[6736063]]></arnumber>
##     <doi><![CDATA[10.1109/TVT.2014.2305475]]></doi>
##     <publicationId><![CDATA[6736063]]></publicationId>
##     <mdurl><![CDATA[http://ieeexplore.ieee.org/xpl/articleDetails.jsp?tp=&arnumber=6736063&contentType=Journals+%26+Magazines]]></mdurl>
##     <pdf><![CDATA[http://ieeexplore.ieee.org/stamp/stamp.jsp?arnumber=6736063]]></pdf>
##   </document>
##   <document>
##     <rank>358</rank>
##     <title><![CDATA[Demonstration of Large Coupling-Induced Phase Delay in Silicon Directional Cross-Couplers]]></title>
##     <authors><![CDATA[Westerveld, W.J.;  Pozo, J.;  Leinders, S.M.;  Yousefi, M.;  Urbach, H.P.]]></authors>
##     <affiliations><![CDATA[Opt. Res. Group, Delft Univ. of Technol., Delft, Netherlands]]></affiliations>
##     <controlledterms>
##       <term><![CDATA[elemental semiconductors]]></term>
##       <term><![CDATA[integrated optics]]></term>
##       <term><![CDATA[optical delay lines]]></term>
##       <term><![CDATA[optical directional couplers]]></term>
##       <term><![CDATA[optical resonators]]></term>
##       <term><![CDATA[silicon-on-insulator]]></term>
##     </controlledterms>
##     <thesaurusterms>
##       <term><![CDATA[Delays]]></term>
##       <term><![CDATA[Directional couplers]]></term>
##       <term><![CDATA[Indexes]]></term>
##       <term><![CDATA[Optical ring resonators]]></term>
##       <term><![CDATA[Optical waveguides]]></term>
##       <term><![CDATA[Photonics]]></term>
##     </thesaurusterms>
##     <pubtitle><![CDATA[Selected Topics in Quantum Electronics, IEEE Journal of]]></pubtitle>
##     <punumber><![CDATA[2944]]></punumber>
##     <pubtype><![CDATA[Journals & Magazines]]></pubtype>
##     <publisher><![CDATA[IEEE]]></publisher>
##     <volume><![CDATA[20]]></volume>
##     <issue><![CDATA[4]]></issue>
##     <py><![CDATA[2014]]></py>
##     <spage><![CDATA[1]]></spage>
##     <epage><![CDATA[6]]></epage>
##     <abstract><![CDATA[We investigate directional couplers in silicon-on-insulator photonic technology. We theoretically and experimentally demonstrate a large coupling-induced phase delay that occurs when nearly all light is coupled from one waveguide to the other, i.e., when the coupler operates as a cross-coupler. We show that even a tiny asymmetry in the two waveguides of the coupler causes a significant additional phase delay. The observed change in the free-spectral-range of ring resonators from 5.0 nm to 6.4 nm is explained by a small 0.1% difference in the propagation constants of the two waveguides. Such a difference can be caused by for example a 1 nm difference in the widths of the two waveguides.]]></abstract>
##     <issn><![CDATA[1077-260X]]></issn>
##     <htmlFlag><![CDATA[1]]></htmlFlag>
##     <arnumber><![CDATA[6675800]]></arnumber>
##     <doi><![CDATA[10.1109/JSTQE.2013.2292874]]></doi>
##     <publicationId><![CDATA[6675800]]></publicationId>
##     <mdurl><![CDATA[http://ieeexplore.ieee.org/xpl/articleDetails.jsp?tp=&arnumber=6675800&contentType=Journals+%26+Magazines]]></mdurl>
##     <pdf><![CDATA[http://ieeexplore.ieee.org/stamp/stamp.jsp?arnumber=6675800]]></pdf>
##   </document>
##   <document>
##     <rank>359</rank>
##     <title><![CDATA[Performance of Pitch and Stall Regulated Tidal Stream Turbines]]></title>
##     <authors><![CDATA[Whitby, B.;  Ugalde-Loo, C.E.]]></authors>
##     <affiliations><![CDATA[Cardiff Univ., Cardiff, UK]]></affiliations>
##     <controlledterms>
##       <term><![CDATA[blades]]></term>
##       <term><![CDATA[control system synthesis]]></term>
##       <term><![CDATA[hydraulic turbines]]></term>
##       <term><![CDATA[power control]]></term>
##       <term><![CDATA[power generation control]]></term>
##       <term><![CDATA[turbulence]]></term>
##     </controlledterms>
##     <thesaurusterms>
##       <term><![CDATA[Blades]]></term>
##       <term><![CDATA[Generators]]></term>
##       <term><![CDATA[Hydrodynamics]]></term>
##       <term><![CDATA[Resonant frequency]]></term>
##       <term><![CDATA[Rotors]]></term>
##       <term><![CDATA[Torque]]></term>
##       <term><![CDATA[Turbines]]></term>
##     </thesaurusterms>
##     <pubtitle><![CDATA[Sustainable Energy, IEEE Transactions on]]></pubtitle>
##     <punumber><![CDATA[5165391]]></punumber>
##     <pubtype><![CDATA[Journals & Magazines]]></pubtype>
##     <publisher><![CDATA[IEEE]]></publisher>
##     <volume><![CDATA[5]]></volume>
##     <issue><![CDATA[1]]></issue>
##     <py><![CDATA[2014]]></py>
##     <spage><![CDATA[64]]></spage>
##     <epage><![CDATA[72]]></epage>
##     <abstract><![CDATA[Controllers for a pitch and a stall regulated horizontal axial flow, variable-speed tidal stream turbine are developed, and a performance comparison is carried out. Below rated flow speed, both turbines are operated in variable-speed mode so that the optimum tip-speed ratio is maintained. One of the turbines has variable pitch blades, which above rated speed are pitched to feather in order to regulate power. The other turbine has fixed pitch blades and uses speed-assisted stall to regulate power. The control system design behind both strategies is examined in MATLAB, with the performance under turbulent flows, loading and energy yield analysis being evaluated in GH Tidal Bladed. Both strategies provide a satisfactory performance, but the out-of-plane loads on the stall regulated turbine were higher over the entire range of operation. In addition, the dynamic characteristics of the stall regulated turbine require a more complex control design. The results suggest that the pitch regulated turbine would be a more attractive solution for turbine developers.]]></abstract>
##     <issn><![CDATA[1949-3029]]></issn>
##     <htmlFlag><![CDATA[1]]></htmlFlag>
##     <arnumber><![CDATA[6578587]]></arnumber>
##     <doi><![CDATA[10.1109/TSTE.2013.2272653]]></doi>
##     <publicationId><![CDATA[6578587]]></publicationId>
##     <mdurl><![CDATA[http://ieeexplore.ieee.org/xpl/articleDetails.jsp?tp=&arnumber=6578587&contentType=Journals+%26+Magazines]]></mdurl>
##     <pdf><![CDATA[http://ieeexplore.ieee.org/stamp/stamp.jsp?arnumber=6578587]]></pdf>
##   </document>
##   <document>
##     <rank>360</rank>
##     <title><![CDATA[Strain Sensor Based on Fiber Ring Cavity Laser With Photonic Crystal Fiber In-Line Mach&#x2013;Zehnder Interferometer]]></title>
##     <authors><![CDATA[Xuekun Bai;  Dengfeng Fan;  Shaofei Wang;  Shengli Pu;  Xianglong Zeng]]></authors>
##     <affiliations><![CDATA[Key Lab. of Specialty Fiber Opt. & Opt. Access Network, Shanghai Univ., Shanghai, China]]></affiliations>
##     <controlledterms>
##       <term><![CDATA[Mach-Zehnder interferometers]]></term>
##       <term><![CDATA[fibre lasers]]></term>
##       <term><![CDATA[fibre optic sensors]]></term>
##       <term><![CDATA[holey fibres]]></term>
##       <term><![CDATA[laser cavity resonators]]></term>
##       <term><![CDATA[measurement by laser beam]]></term>
##       <term><![CDATA[photonic crystals]]></term>
##       <term><![CDATA[ring lasers]]></term>
##       <term><![CDATA[strain measurement]]></term>
##       <term><![CDATA[strain sensors]]></term>
##     </controlledterms>
##     <thesaurusterms>
##       <term><![CDATA[Cavity resonators]]></term>
##       <term><![CDATA[Lead]]></term>
##       <term><![CDATA[Measurement by laser beam]]></term>
##       <term><![CDATA[Ring lasers]]></term>
##       <term><![CDATA[Sensitivity]]></term>
##       <term><![CDATA[Sensors]]></term>
##       <term><![CDATA[Strain]]></term>
##     </thesaurusterms>
##     <pubtitle><![CDATA[Photonics Journal, IEEE]]></pubtitle>
##     <punumber><![CDATA[4563994]]></punumber>
##     <pubtype><![CDATA[Journals & Magazines]]></pubtype>
##     <publisher><![CDATA[IEEE]]></publisher>
##     <volume><![CDATA[6]]></volume>
##     <issue><![CDATA[4]]></issue>
##     <py><![CDATA[2014]]></py>
##     <spage><![CDATA[1]]></spage>
##     <epage><![CDATA[8]]></epage>
##     <abstract><![CDATA[We experimentally demonstrated a strain sensor based on fiber ring cavity laser with a photonic crystal fiber (PCF) in-line Mach-Zehnder interferometer (MZI) structure, which is used as an optical band-pass filter and acts as the strain sensing component. The fiber ring cavity laser plays the role of enhancing the visibility of the resonant spectrum and narrowing the corresponding 3-dB bandwidth, thus improving the comprehensive sensing performance. The induced axial strain on the structure is measured by monitoring the central wavelengths of the laser output. A high strain sensing sensitivity of 2.1 pm/&#x03BC;&#x03B5; is successfully achieved in the linear strain range of 0-2100 &#x03BC;&#x03B5;. A parameter Q value describing the overall sensing performance is introduced by including the strain sensing sensitivity, sensing sensitivity relative to 3-dB bandwidth of the resonant spectrum and the corresponding visibility. Comparing with the reported strain measurements based on a PCF in-line MZI structure, the experimental results based on fiber ring cavity laser sensor present more than nine times larger Q value.]]></abstract>
##     <issn><![CDATA[1943-0655]]></issn>
##     <htmlFlag><![CDATA[1]]></htmlFlag>
##     <arnumber><![CDATA[6842630]]></arnumber>
##     <doi><![CDATA[10.1109/JPHOT.2014.2332454]]></doi>
##     <publicationId><![CDATA[6842630]]></publicationId>
##     <mdurl><![CDATA[http://ieeexplore.ieee.org/xpl/articleDetails.jsp?tp=&arnumber=6842630&contentType=Journals+%26+Magazines]]></mdurl>
##     <pdf><![CDATA[http://ieeexplore.ieee.org/stamp/stamp.jsp?arnumber=6842630]]></pdf>
##   </document>
##   <document>
##     <rank>361</rank>
##     <title><![CDATA[Automated Stereo Retrieval of Smoke Plume Injection Heights and Retrieval of Smoke Plume Masks From AATSR and Their Assessment With CALIPSO and MISR]]></title>
##     <authors><![CDATA[Fisher, D.;  Muller, J.-P.;  Yershov, V.N.]]></authors>
##     <affiliations><![CDATA[Univ. Coll. London, Dorking, UK]]></affiliations>
##     <controlledterms>
##       <term><![CDATA[geophysical techniques]]></term>
##       <term><![CDATA[smoke]]></term>
##       <term><![CDATA[wildfires]]></term>
##     </controlledterms>
##     <pubtitle><![CDATA[Geoscience and Remote Sensing, IEEE Transactions on]]></pubtitle>
##     <punumber><![CDATA[36]]></punumber>
##     <pubtype><![CDATA[Journals & Magazines]]></pubtype>
##     <publisher><![CDATA[IEEE]]></publisher>
##     <volume><![CDATA[52]]></volume>
##     <issue><![CDATA[2]]></issue>
##     <py><![CDATA[2014]]></py>
##     <spage><![CDATA[1249]]></spage>
##     <epage><![CDATA[1258]]></epage>
##     <abstract><![CDATA[The longevity and dispersion of smoke and associated chemical constituents released from wildfire events are dependent on several factors, crucially including the height at which the smoke is injected into the atmosphere. The aim here is to provide improved emission data for the initialization of chemical transport models in order to better predict aerosol and trace gas dispersion following injection into the free atmosphere. A new stereo-matching algorithm, named M6, which can effectively resolve smoke plume injection heights (SPIH), is presented here. M6 is extensively validated against two alternative spaceborne earth observation SPIH data sources and demonstrates good agreement. Further, due to the spectral and dual-view configuration of the Advanced Along-Track Scanning Radiometer imaging system, it is possible to automatically differentiate smoke from other atmospheric features effectively-a feat, which currently no other algorithm can achieve. Additionally, as the M6 algorithm shares a heritage with the other M-series matchers, it is here compared against one of its predecessors, M4, which, for the determination of SPIH, M6 is shown to substantially outperform.]]></abstract>
##     <issn><![CDATA[0196-2892]]></issn>
##     <htmlFlag><![CDATA[1]]></htmlFlag>
##     <arnumber><![CDATA[6487400]]></arnumber>
##     <doi><![CDATA[10.1109/TGRS.2013.2249073]]></doi>
##     <publicationId><![CDATA[6487400]]></publicationId>
##     <mdurl><![CDATA[http://ieeexplore.ieee.org/xpl/articleDetails.jsp?tp=&arnumber=6487400&contentType=Journals+%26+Magazines]]></mdurl>
##     <pdf><![CDATA[http://ieeexplore.ieee.org/stamp/stamp.jsp?arnumber=6487400]]></pdf>
##   </document>
##   <document>
##     <rank>362</rank>
##     <title><![CDATA[Diagnosis of Eccentric Rotor in Synchronous Machines by Analysis of Split-Phase Currents&#x2014;Part I: Theoretical Analysis]]></title>
##     <authors><![CDATA[Bruzzese, C.]]></authors>
##     <affiliations><![CDATA[Dept. of Astronaut., Electr., & Energy Eng., Univ. of Rome &#x201C;La Sapienza, Rome, Italy]]></affiliations>
##     <controlledterms>
##       <term><![CDATA[air gaps]]></term>
##       <term><![CDATA[electric current measurement]]></term>
##       <term><![CDATA[electric reactance]]></term>
##       <term><![CDATA[electrical faults]]></term>
##       <term><![CDATA[rotors]]></term>
##       <term><![CDATA[stators]]></term>
##       <term><![CDATA[synchronous generators]]></term>
##     </controlledterms>
##     <thesaurusterms>
##       <term><![CDATA[Couplings]]></term>
##       <term><![CDATA[Monitoring]]></term>
##       <term><![CDATA[Rotors]]></term>
##       <term><![CDATA[Stator windings]]></term>
##       <term><![CDATA[Vectors]]></term>
##       <term><![CDATA[Windings]]></term>
##     </thesaurusterms>
##     <pubtitle><![CDATA[Industrial Electronics, IEEE Transactions on]]></pubtitle>
##     <punumber><![CDATA[41]]></punumber>
##     <pubtype><![CDATA[Journals & Magazines]]></pubtype>
##     <publisher><![CDATA[IEEE]]></publisher>
##     <volume><![CDATA[61]]></volume>
##     <issue><![CDATA[8]]></issue>
##     <py><![CDATA[2014]]></py>
##     <spage><![CDATA[4193]]></spage>
##     <epage><![CDATA[4205]]></epage>
##     <abstract><![CDATA[This work shows a method to quantify rotor eccentricities in synchronous machines by exploiting the unbalance caused in the split-phase currents. The paper first develops a machine model comprehensive of eccentricities and parallel circuits in the stator, by using symmetrical components. Then, the model is used for formal calculation of the unbalanced currents. Finally, the equations are reversed to obtain eccentricity degrees from current measurements. Practical formulas are given for fault assessment, only requiring machine line voltage and synchronous reactance. The method can be applied on load. This paper provides full details of the theory underlying the method. The theory also clarifies some aspects about split-phase currents, not deepened before. It is proven that the air gap flux modulation due to eccentricities, acting through additional 2(p &#x00B1;1) -pole flux waves in 2p-pole machines, stimulates additional currents, which circulate in the stator and turn into 2(p &#x00B1;1)-pole rotating space vectors in the complex domain. Vector trajectories have shape and amplitude dictated by eccentricity type and degree, respectively. This study is limited to 2p-pole machines with p &#x2265; 2. The theory is corroborated by simulations of a practical 1950-kVA generator in this paper. Experimental proofs and simulations of a laboratory 17-kVA machine are provided in a sequel of this paper.]]></abstract>
##     <issn><![CDATA[0278-0046]]></issn>
##     <htmlFlag><![CDATA[1]]></htmlFlag>
##     <arnumber><![CDATA[6616567]]></arnumber>
##     <doi><![CDATA[10.1109/TIE.2013.2284141]]></doi>
##     <publicationId><![CDATA[6616567]]></publicationId>
##     <mdurl><![CDATA[http://ieeexplore.ieee.org/xpl/articleDetails.jsp?tp=&arnumber=6616567&contentType=Journals+%26+Magazines]]></mdurl>
##     <pdf><![CDATA[http://ieeexplore.ieee.org/stamp/stamp.jsp?arnumber=6616567]]></pdf>
##   </document>
##   <document>
##     <rank>363</rank>
##     <title><![CDATA[Regional Variance for Multi-Object Filtering]]></title>
##     <authors><![CDATA[Delande, E.;  Uney, M.;  Houssineau, J.;  Clark, D.]]></authors>
##     <affiliations><![CDATA[Sch. of Eng. & Phys. Sci., Heriot-Watt Univ. (HWU), Edinburgh, UK]]></affiliations>
##     <controlledterms>
##       <term><![CDATA[higher order statistics]]></term>
##       <term><![CDATA[object detection]]></term>
##       <term><![CDATA[signal processing]]></term>
##     </controlledterms>
##     <thesaurusterms>
##       <term><![CDATA[Bayes methods]]></term>
##       <term><![CDATA[Higher order statistics]]></term>
##       <term><![CDATA[Random variables]]></term>
##       <term><![CDATA[Reactive power]]></term>
##       <term><![CDATA[Signal processing algorithms]]></term>
##       <term><![CDATA[Target tracking]]></term>
##     </thesaurusterms>
##     <pubtitle><![CDATA[Signal Processing, IEEE Transactions on]]></pubtitle>
##     <punumber><![CDATA[78]]></punumber>
##     <pubtype><![CDATA[Journals & Magazines]]></pubtype>
##     <publisher><![CDATA[IEEE]]></publisher>
##     <volume><![CDATA[62]]></volume>
##     <issue><![CDATA[13]]></issue>
##     <py><![CDATA[2014]]></py>
##     <spage><![CDATA[3415]]></spage>
##     <epage><![CDATA[3428]]></epage>
##     <abstract><![CDATA[Recent progress in multi-object filtering has led to algorithms that compute the first-order moment of multi-object distributions based on sensor measurements. The number of targets in arbitrarily selected regions can be estimated using the first-order moment. In this work, we introduce explicit formulae for the computation of the second-order statistic on the target number. The proposed concept of regional variance quantifies the level of confidence on target number estimates in arbitrary regions and facilitates information-based decisions. We provide algorithms for its computation for the probability hypothesis density (PHD) and the cardinalized probability hypothesis density (CPHD) filters. We demonstrate the behaviour of the regional statistics through simulation examples.]]></abstract>
##     <issn><![CDATA[1053-587X]]></issn>
##     <htmlFlag><![CDATA[1]]></htmlFlag>
##     <arnumber><![CDATA[6824846]]></arnumber>
##     <doi><![CDATA[10.1109/TSP.2014.2328326]]></doi>
##     <publicationId><![CDATA[6824846]]></publicationId>
##     <mdurl><![CDATA[http://ieeexplore.ieee.org/xpl/articleDetails.jsp?tp=&arnumber=6824846&contentType=Journals+%26+Magazines]]></mdurl>
##     <pdf><![CDATA[http://ieeexplore.ieee.org/stamp/stamp.jsp?arnumber=6824846]]></pdf>
##   </document>
##   <document>
##     <rank>364</rank>
##     <title><![CDATA[Fiber Optic Pressure Sensing Arrays for Monitoring Horizontal and Vertical Pressures Generated by Traveling Water Waves]]></title>
##     <authors><![CDATA[Arkwright, J.W.;  Underhill, I.D.;  Maunder, S.A.;  Jafari, A.;  Cartwright, N.;  Lemckert, C.]]></authors>
##     <affiliations><![CDATA[Dept. of Mater. Sci. & Eng., CSIRO, West Lindfield, NSW, Australia]]></affiliations>
##     <controlledterms>
##       <term><![CDATA[Bragg gratings]]></term>
##       <term><![CDATA[distributed sensors]]></term>
##       <term><![CDATA[electromagnetic interference]]></term>
##       <term><![CDATA[fibre optic sensors]]></term>
##       <term><![CDATA[image sensors]]></term>
##       <term><![CDATA[piezoelectric transducers]]></term>
##       <term><![CDATA[pressure measurement]]></term>
##       <term><![CDATA[pressure sensors]]></term>
##       <term><![CDATA[sensor arrays]]></term>
##     </controlledterms>
##     <thesaurusterms>
##       <term><![CDATA[Optical arrays]]></term>
##       <term><![CDATA[Optical fiber sensors]]></term>
##       <term><![CDATA[Optical fibers]]></term>
##       <term><![CDATA[Sensor arrays]]></term>
##       <term><![CDATA[Transducers]]></term>
##     </thesaurusterms>
##     <pubtitle><![CDATA[Sensors Journal, IEEE]]></pubtitle>
##     <punumber><![CDATA[7361]]></punumber>
##     <pubtype><![CDATA[Journals & Magazines]]></pubtype>
##     <publisher><![CDATA[IEEE]]></publisher>
##     <volume><![CDATA[14]]></volume>
##     <issue><![CDATA[8]]></issue>
##     <py><![CDATA[2014]]></py>
##     <spage><![CDATA[2739]]></spage>
##     <epage><![CDATA[2742]]></epage>
##     <abstract><![CDATA[Distributed pressure sensing arrays fabricated from fiber Bragg gratings have been demonstrated for real-time monitoring of the dynamic subsurface pressures beneath water waves in a wave tank. Two sensing arrays were used to monitor horizontal and vertical pressures in the tank as periodic wave trains passed overhead. The horizontal and vertical arrays contained 90 and 35 sensing elements, respectively, spaced at 1-cm intervals allowing highly accurate spatial resolution to be achieved in both orientations. The wave tank paddle was programmed to generate wave-trains varying from ~5 to 30-cm peak-to-trough and the pressures measured using the fiber optic array were validated using commercial piezo-electric pressure sensors and video image analysis. The length and sensor separation of the fiber optic sensing array can be varied to suit the location under test, and the fiber optic elements make the devices inherently resistant to corrosion and electromagnetic interference.]]></abstract>
##     <issn><![CDATA[1530-437X]]></issn>
##     <htmlFlag><![CDATA[1]]></htmlFlag>
##     <arnumber><![CDATA[6766669]]></arnumber>
##     <doi><![CDATA[10.1109/JSEN.2014.2311806]]></doi>
##     <publicationId><![CDATA[6766669]]></publicationId>
##     <mdurl><![CDATA[http://ieeexplore.ieee.org/xpl/articleDetails.jsp?tp=&arnumber=6766669&contentType=Journals+%26+Magazines]]></mdurl>
##     <pdf><![CDATA[http://ieeexplore.ieee.org/stamp/stamp.jsp?arnumber=6766669]]></pdf>
##   </document>
##   <document>
##     <rank>365</rank>
##     <title><![CDATA[Fast Mode Decision Algorithm Through Inter-View Rate-Distortion Prediction for Multiview Video Coding System]]></title>
##     <authors><![CDATA[Chia-Hung Yeh;  Ming-Feng Li;  Mei-Juan Chen;  Ming-Chieh Chi;  Xin-Xian Huang;  Hao-Wen Chi]]></authors>
##     <affiliations><![CDATA[Dept. of Electr. Eng., Nat. Sun Yat-sen Univ., Kaohsiung, Taiwan]]></affiliations>
##     <controlledterms>
##       <term><![CDATA[IP networks]]></term>
##       <term><![CDATA[computational complexity]]></term>
##       <term><![CDATA[video coding]]></term>
##     </controlledterms>
##     <thesaurusterms>
##       <term><![CDATA[Computational complexity]]></term>
##       <term><![CDATA[Correlation]]></term>
##       <term><![CDATA[Encoding]]></term>
##       <term><![CDATA[Estimation]]></term>
##       <term><![CDATA[Motion estimation]]></term>
##       <term><![CDATA[Prediction algorithms]]></term>
##       <term><![CDATA[Video coding]]></term>
##     </thesaurusterms>
##     <pubtitle><![CDATA[Industrial Informatics, IEEE Transactions on]]></pubtitle>
##     <punumber><![CDATA[9424]]></punumber>
##     <pubtype><![CDATA[Journals & Magazines]]></pubtype>
##     <publisher><![CDATA[IEEE]]></publisher>
##     <volume><![CDATA[10]]></volume>
##     <issue><![CDATA[1]]></issue>
##     <py><![CDATA[2014]]></py>
##     <spage><![CDATA[594]]></spage>
##     <epage><![CDATA[603]]></epage>
##     <abstract><![CDATA[Multiview video coding (MVC) has attracted great attention from industries and research institutes. MVC is used to encode stereoscopic video streams for 3D playout systems such as 3D television, digital cinema, and IP network applications. MVC is an extended version of H.264/AVC that improves the performance of multiview videos. Yet, when compared with single-view video coding, MVC consumes much more time when encoding large amounts of data. Speed-up algorithms, therefore, are essential for realizing related applications. This paper presents a fast mode decision algorithm to avoid the high computational complexity of MVC. The proposed approach aims to reduce candidate modes and make mode decision process more efficient. The minimum and maximum values of rate-distortion cost (RD cost) in the previously encoded view are used to compute a threshold for each mode in the current view. Compared with joint multiview video coding, the experimental results demonstrate that the proposed algorithm provides an average of 79% in time savings with negligible bit rate increase and peak signal-to-noise ratio decrease.]]></abstract>
##     <issn><![CDATA[1551-3203]]></issn>
##     <htmlFlag><![CDATA[1]]></htmlFlag>
##     <arnumber><![CDATA[6558811]]></arnumber>
##     <doi><![CDATA[10.1109/TII.2013.2273308]]></doi>
##     <publicationId><![CDATA[6558811]]></publicationId>
##     <mdurl><![CDATA[http://ieeexplore.ieee.org/xpl/articleDetails.jsp?tp=&arnumber=6558811&contentType=Journals+%26+Magazines]]></mdurl>
##     <pdf><![CDATA[http://ieeexplore.ieee.org/stamp/stamp.jsp?arnumber=6558811]]></pdf>
##   </document>
##   <document>
##     <rank>366</rank>
##     <title><![CDATA[On-Wafer Optical Loss Measurements Using Ring Resonators With Integrated Sources and Detectors]]></title>
##     <authors><![CDATA[Bitincka, E.;  Gilardi, G.;  Smit, M.K.]]></authors>
##     <affiliations><![CDATA[Dept. of Electr. Eng., Univ. of Technol. Eindhoven, Eindhoven, Netherlands]]></affiliations>
##     <controlledterms>
##       <term><![CDATA[III-V semiconductors]]></term>
##       <term><![CDATA[indium compounds]]></term>
##       <term><![CDATA[integrated optics]]></term>
##       <term><![CDATA[optical losses]]></term>
##       <term><![CDATA[optical resonators]]></term>
##       <term><![CDATA[optical variables measurement]]></term>
##     </controlledterms>
##     <thesaurusterms>
##       <term><![CDATA[Couplings]]></term>
##       <term><![CDATA[Loss measurement]]></term>
##       <term><![CDATA[Optical device fabrication]]></term>
##       <term><![CDATA[Optical ring resonators]]></term>
##       <term><![CDATA[Optical variables measurement]]></term>
##       <term><![CDATA[Propagation losses]]></term>
##       <term><![CDATA[Semiconductor device measurement]]></term>
##     </thesaurusterms>
##     <pubtitle><![CDATA[Photonics Journal, IEEE]]></pubtitle>
##     <punumber><![CDATA[4563994]]></punumber>
##     <pubtype><![CDATA[Journals & Magazines]]></pubtype>
##     <publisher><![CDATA[IEEE]]></publisher>
##     <volume><![CDATA[6]]></volume>
##     <issue><![CDATA[5]]></issue>
##     <py><![CDATA[2014]]></py>
##     <spage><![CDATA[1]]></spage>
##     <epage><![CDATA[12]]></epage>
##     <abstract><![CDATA[We demonstrate for the first time a fully integrated test structure dedicated to on-wafer propagation loss measurement. An integrated light source is used in combination with a resonant cavity and a full absorbing detector. It is fabricated in a multi project wafer run in an InP based foundry process. The probing of the integrated light source and detector with electrical signals avoids the reproducibility issues and time-overhead associated with high-precision optical alignment. The measurement accuracy, estimated to be ~&#x00B1;0.2, the compact footprint (~1.5 mm<sup>2</sup>), and the simple and fast measurement procedure make this approach an ideal candidate for the future characterization of propagation losses in both research and manufacturing environments.]]></abstract>
##     <issn><![CDATA[1943-0655]]></issn>
##     <htmlFlag><![CDATA[1]]></htmlFlag>
##     <arnumber><![CDATA[6887349]]></arnumber>
##     <doi><![CDATA[10.1109/JPHOT.2014.2352627]]></doi>
##     <publicationId><![CDATA[6887349]]></publicationId>
##     <mdurl><![CDATA[http://ieeexplore.ieee.org/xpl/articleDetails.jsp?tp=&arnumber=6887349&contentType=Journals+%26+Magazines]]></mdurl>
##     <pdf><![CDATA[http://ieeexplore.ieee.org/stamp/stamp.jsp?arnumber=6887349]]></pdf>
##   </document>
##   <document>
##     <rank>367</rank>
##     <title><![CDATA[Breakthroughs in Photonics 2013: Organic Nanostructures for Antireflection]]></title>
##     <authors><![CDATA[Schulz, U.;  Munzert, P.;  Rickelt, F.;  Kaiser, N.]]></authors>
##     <affiliations><![CDATA[Fraunhofer Inst. for Appl. Opt. & Precision Eng., Jena, Germany]]></affiliations>
##     <controlledterms>
##       <term><![CDATA[antireflection coatings]]></term>
##       <term><![CDATA[nanofabrication]]></term>
##       <term><![CDATA[nanostructured materials]]></term>
##       <term><![CDATA[polymer films]]></term>
##       <term><![CDATA[sputter etching]]></term>
##       <term><![CDATA[vacuum deposition]]></term>
##     </controlledterms>
##     <thesaurusterms>
##       <term><![CDATA[Coatings]]></term>
##       <term><![CDATA[Etching]]></term>
##       <term><![CDATA[Nanostructures]]></term>
##       <term><![CDATA[Plasmas]]></term>
##       <term><![CDATA[Polymers]]></term>
##       <term><![CDATA[Substrates]]></term>
##     </thesaurusterms>
##     <pubtitle><![CDATA[Photonics Journal, IEEE]]></pubtitle>
##     <punumber><![CDATA[4563994]]></punumber>
##     <pubtype><![CDATA[Journals & Magazines]]></pubtype>
##     <publisher><![CDATA[IEEE]]></publisher>
##     <volume><![CDATA[6]]></volume>
##     <issue><![CDATA[2]]></issue>
##     <py><![CDATA[2014]]></py>
##     <spage><![CDATA[1]]></spage>
##     <epage><![CDATA[5]]></epage>
##     <abstract><![CDATA[The processing of organic substances by vacuum deposition is opening new possibilities for the properties of optical surfaces. Organic small molecules can be evaporated and deposited like optical thin films. Plasma etching, which has been successfully applied to generate nanostructures on polymer substrates, is now used for obtaining antireflective nanostructures on top of interference stacks on glass. The latest results in achieving excellent antireflective properties for a wide range of light incidence angles were accomplished through multiple etching of polymer substrates and organic layers.]]></abstract>
##     <issn><![CDATA[1943-0655]]></issn>
##     <htmlFlag><![CDATA[1]]></htmlFlag>
##     <arnumber><![CDATA[6766193]]></arnumber>
##     <doi><![CDATA[10.1109/JPHOT.2014.2311432]]></doi>
##     <publicationId><![CDATA[6766193]]></publicationId>
##     <mdurl><![CDATA[http://ieeexplore.ieee.org/xpl/articleDetails.jsp?tp=&arnumber=6766193&contentType=Journals+%26+Magazines]]></mdurl>
##     <pdf><![CDATA[http://ieeexplore.ieee.org/stamp/stamp.jsp?arnumber=6766193]]></pdf>
##   </document>
##   <document>
##     <rank>368</rank>
##     <title><![CDATA[All-Optical Generation of Two IEEE802.11n Signals for 2 <inline-formula> <img src="/images/tex/326.gif" alt="\times"> </inline-formula> 2 MIMO-RoF via MRR System]]></title>
##     <authors><![CDATA[Amiri, I.S.;  Alavi, S.E.;  Fisal, N.;  Supa'at, A.S.M.;  Ahmad, H.]]></authors>
##     <affiliations><![CDATA[Photonics Res. Centre, Univ. of Malaya, Kuala Lumpur, Malaysia]]></affiliations>
##     <controlledterms>
##       <term><![CDATA[MIMO communication]]></term>
##       <term><![CDATA[micromechanical resonators]]></term>
##       <term><![CDATA[multiplexing]]></term>
##       <term><![CDATA[radio-over-fibre]]></term>
##       <term><![CDATA[transmitting antennas]]></term>
##       <term><![CDATA[wireless LAN]]></term>
##     </controlledterms>
##     <thesaurusterms>
##       <term><![CDATA[MIMO]]></term>
##       <term><![CDATA[OFDM]]></term>
##       <term><![CDATA[Optical fibers]]></term>
##       <term><![CDATA[Optical pulses]]></term>
##       <term><![CDATA[Optical ring resonators]]></term>
##       <term><![CDATA[Optical transmitters]]></term>
##       <term><![CDATA[Solitons]]></term>
##     </thesaurusterms>
##     <pubtitle><![CDATA[Photonics Journal, IEEE]]></pubtitle>
##     <punumber><![CDATA[4563994]]></punumber>
##     <pubtype><![CDATA[Journals & Magazines]]></pubtype>
##     <publisher><![CDATA[IEEE]]></publisher>
##     <volume><![CDATA[6]]></volume>
##     <issue><![CDATA[6]]></issue>
##     <py><![CDATA[2014]]></py>
##     <spage><![CDATA[1]]></spage>
##     <epage><![CDATA[11]]></epage>
##     <abstract><![CDATA[A radio-over-fiber system capable of very spectrally efficient data transmission and based on multiple-input multiple-output (MIMO) and orthogonal frequency-division multiplexing (OFDM) is presented here. Carrier generation is the basic building block for implementation of OFDM transmission, and multicarriers can be generated using the microring resonator (MRR) system. A series of MRRs incorporated with an add/drop filter system was utilized to generate multicarriers in the 193.00999-193.01001-THz range, which were used to all-optically generate two MIMO wireless local area network radio frequency (RF) signals suitable for the IEEE802.11n standard communication systems, and single wavelengths at frequencies of 193.08, 193.1, and 193.12 THz with free spectral range of 20 GHz used to optically transport the separated MIMO signals over a single-mode fiber (SMF). The error vector magnitude (EVM) and bit error rate of the overall system performance are discussed. Results show that the generated RF signals wirelessly propagated through the MIMO channel using the 2 &#x00D7; 2 MIMO Tx antennas. There is an acceptable EVM variation for wireless distance lower than 70, 30, and 15 m. It can be concluded that the transmission of both MIMO RF signals is feasible for up to a 50-km SMF path length and a wireless distance of 15 m.]]></abstract>
##     <issn><![CDATA[1943-0655]]></issn>
##     <htmlFlag><![CDATA[1]]></htmlFlag>
##     <arnumber><![CDATA[6923468]]></arnumber>
##     <doi><![CDATA[10.1109/JPHOT.2014.2363437]]></doi>
##     <publicationId><![CDATA[6923468]]></publicationId>
##     <mdurl><![CDATA[http://ieeexplore.ieee.org/xpl/articleDetails.jsp?tp=&arnumber=6923468&contentType=Journals+%26+Magazines]]></mdurl>
##     <pdf><![CDATA[http://ieeexplore.ieee.org/stamp/stamp.jsp?arnumber=6923468]]></pdf>
##   </document>
##   <document>
##     <rank>369</rank>
##     <title><![CDATA[Compact MOSFET Modeling for Process Variability-Aware VLSI Circuit Design]]></title>
##     <authors><![CDATA[Saha, S.K.]]></authors>
##     <affiliations><![CDATA[Prospicient Devices, Milpitas, CA, USA]]></affiliations>
##     <controlledterms>
##       <term><![CDATA[CMOS integrated circuits]]></term>
##       <term><![CDATA[MOSFET]]></term>
##       <term><![CDATA[integrated circuit design]]></term>
##       <term><![CDATA[semiconductor device models]]></term>
##     </controlledterms>
##     <thesaurusterms>
##       <term><![CDATA[Analytical models]]></term>
##       <term><![CDATA[Data models]]></term>
##       <term><![CDATA[Integrated circuit modeling]]></term>
##       <term><![CDATA[MOSFET]]></term>
##       <term><![CDATA[Semiconductor device modeling]]></term>
##       <term><![CDATA[Solid modeling]]></term>
##       <term><![CDATA[Very large scale integration]]></term>
##     </thesaurusterms>
##     <pubtitle><![CDATA[Access, IEEE]]></pubtitle>
##     <punumber><![CDATA[6287639]]></punumber>
##     <pubtype><![CDATA[Journals & Magazines]]></pubtype>
##     <publisher><![CDATA[IEEE]]></publisher>
##     <volume><![CDATA[2]]></volume>
##     <py><![CDATA[2014]]></py>
##     <spage><![CDATA[104]]></spage>
##     <epage><![CDATA[115]]></epage>
##     <abstract><![CDATA[This paper presents a systematic methodology to develop compact MOSFET models for process variability-aware VLSI circuit design. Process variability in scaled CMOS technologies severely impacts the functionality, yield, and reliability of advanced integrated circuit devices, circuits, and systems. Therefore, variability-aware circuit design techniques are required for realistic assessment of the impact of random and systematic process variability in advanced VLSI circuit performance. However, variability-aware circuit design requires compact MOSFET variability models for computer analysis of the impact of process variability in VLSI circuit design. This paper describes a generalized methodology to determine the major set of device parameters sensitive to random and systematic process variability in nanoscale MOSFET devices, map each variability-sensitive device parameter to the corresponding compact model parameter of the target compact model, and generate statistical compact MOSFET models for variability-aware VLSI circuit design.]]></abstract>
##     <issn><![CDATA[2169-3536]]></issn>
##     <htmlFlag><![CDATA[1]]></htmlFlag>
##     <arnumber><![CDATA[6732881]]></arnumber>
##     <doi><![CDATA[10.1109/ACCESS.2014.2304568]]></doi>
##     <publicationId><![CDATA[6732881]]></publicationId>
##     <mdurl><![CDATA[http://ieeexplore.ieee.org/xpl/articleDetails.jsp?tp=&arnumber=6732881&contentType=Journals+%26+Magazines]]></mdurl>
##     <pdf><![CDATA[http://ieeexplore.ieee.org/stamp/stamp.jsp?arnumber=6732881]]></pdf>
##   </document>
##   <document>
##     <rank>370</rank>
##     <title><![CDATA[Millimeter Wave Channel Modeling and Cellular Capacity Evaluation]]></title>
##     <authors><![CDATA[Akdeniz, M.R.;  Yuanpeng Liu;  Samimi, M.K.;  Shu Sun;  Rangan, S.;  Rappaport, T.S.;  Erkip, E.]]></authors>
##     <affiliations><![CDATA[NYU WIRELESS, New York Univ. Polytech. Sch. of Eng., Brooklyn, NY, USA]]></affiliations>
##     <controlledterms>
##       <term><![CDATA[4G mobile communication]]></term>
##       <term><![CDATA[array signal processing]]></term>
##       <term><![CDATA[microcellular radio]]></term>
##       <term><![CDATA[microwave antenna arrays]]></term>
##       <term><![CDATA[millimetre wave antenna arrays]]></term>
##       <term><![CDATA[next generation networks]]></term>
##       <term><![CDATA[picocellular radio]]></term>
##       <term><![CDATA[radio spectrum management]]></term>
##       <term><![CDATA[signal detection]]></term>
##       <term><![CDATA[space division multiplexing]]></term>
##       <term><![CDATA[statistical analysis]]></term>
##       <term><![CDATA[submillimetre wave antennas]]></term>
##       <term><![CDATA[wireless channels]]></term>
##     </controlledterms>
##     <thesaurusterms>
##       <term><![CDATA[Antenna measurements]]></term>
##       <term><![CDATA[Clustering algorithms]]></term>
##       <term><![CDATA[Delays]]></term>
##       <term><![CDATA[Gain]]></term>
##       <term><![CDATA[Mobile communication]]></term>
##       <term><![CDATA[Power measurement]]></term>
##       <term><![CDATA[Standards]]></term>
##     </thesaurusterms>
##     <pubtitle><![CDATA[Selected Areas in Communications, IEEE Journal on]]></pubtitle>
##     <punumber><![CDATA[49]]></punumber>
##     <pubtype><![CDATA[Journals & Magazines]]></pubtype>
##     <publisher><![CDATA[IEEE]]></publisher>
##     <volume><![CDATA[32]]></volume>
##     <issue><![CDATA[6]]></issue>
##     <py><![CDATA[2014]]></py>
##     <spage><![CDATA[1164]]></spage>
##     <epage><![CDATA[1179]]></epage>
##     <abstract><![CDATA[With the severe spectrum shortage in conventional cellular bands, millimeter wave (mmW) frequencies between 30 and 300 GHz have been attracting growing attention as a possible candidate for next-generation micro- and picocellular wireless networks. The mmW bands offer orders of magnitude greater spectrum than current cellular allocations and enable very high-dimensional antenna arrays for further gains via beamforming and spatial multiplexing. This paper uses recent real-world measurements at 28 and 73 GHz in New York, NY, USA, to derive detailed spatial statistical models of the channels and uses these models to provide a realistic assessment of mmW micro- and picocellular networks in a dense urban deployment. Statistical models are derived for key channel parameters, including the path loss, number of spatial clusters, angular dispersion, and outage. It is found that, even in highly non-line-of-sight environments, strong signals can be detected 100-200 m from potential cell sites, potentially with multiple clusters to support spatial multiplexing. Moreover, a system simulation based on the models predicts that mmW systems can offer an order of magnitude increase in capacity over current state-of-the-art 4G cellular networks with no increase in cell density from current urban deployments.]]></abstract>
##     <issn><![CDATA[0733-8716]]></issn>
##     <htmlFlag><![CDATA[1]]></htmlFlag>
##     <arnumber><![CDATA[6834753]]></arnumber>
##     <doi><![CDATA[10.1109/JSAC.2014.2328154]]></doi>
##     <publicationId><![CDATA[6834753]]></publicationId>
##     <mdurl><![CDATA[http://ieeexplore.ieee.org/xpl/articleDetails.jsp?tp=&arnumber=6834753&contentType=Journals+%26+Magazines]]></mdurl>
##     <pdf><![CDATA[http://ieeexplore.ieee.org/stamp/stamp.jsp?arnumber=6834753]]></pdf>
##   </document>
##   <document>
##     <rank>371</rank>
##     <title><![CDATA[Influence of Transfer Gate Design and Bias on the Radiation Hardness of Pinned Photodiode CMOS Image Sensors]]></title>
##     <authors><![CDATA[Goiffon, V.;  Estribeau, M.;  Cervantes, P.;  Molina, R.;  Gaillardin, M.;  Magnan, P.]]></authors>
##     <affiliations><![CDATA[ISAE, Univ. de Toulouse, Toulouse, France]]></affiliations>
##     <controlledterms>
##       <term><![CDATA[CMOS image sensors]]></term>
##       <term><![CDATA[charge exchange]]></term>
##       <term><![CDATA[photodiodes]]></term>
##       <term><![CDATA[radiation hardening (electronics)]]></term>
##     </controlledterms>
##     <thesaurusterms>
##       <term><![CDATA[Active pixel sensors]]></term>
##       <term><![CDATA[CMOS image sensors]]></term>
##       <term><![CDATA[Dark current]]></term>
##       <term><![CDATA[Ionizing radiation]]></term>
##       <term><![CDATA[Photodiodes]]></term>
##       <term><![CDATA[Radiation effects]]></term>
##       <term><![CDATA[Radiation hardening (electronics)]]></term>
##     </thesaurusterms>
##     <pubtitle><![CDATA[Nuclear Science, IEEE Transactions on]]></pubtitle>
##     <punumber><![CDATA[23]]></punumber>
##     <pubtype><![CDATA[Journals & Magazines]]></pubtype>
##     <publisher><![CDATA[IEEE]]></publisher>
##     <volume><![CDATA[61]]></volume>
##     <issue><![CDATA[6]]></issue>
##     <part><![CDATA[1]]></part>
##     <py><![CDATA[2014]]></py>
##     <spage><![CDATA[3290]]></spage>
##     <epage><![CDATA[3301]]></epage>
##     <abstract><![CDATA[The effects of Cobalt 60 gamma-ray irradiation on pinned photodiode (PPD) CMOS image sensors (CIS) are investigated by comparing the total ionizing dose (TID) response of several transfer gate (TG) and PPD designs manufactured using a 180 nm CIS process. The TID induced variations of charge transfer efficiency (CTE), pinning voltage, equilibrium full well capacity (EFWC), full well capacity (FWC) and dark current measured on the different pixel designs lead to the conclusion that only three degradation sources are responsible for all the observed radiation effects: the pre-metal dielectric (PMD) positive trapped charge, the TG sidewall spacer positive trapped charge and, with less influence, the TG channel shallow trench isolation (STI) trapped charge. The different FWC evolutions with TID presented here are in very good agreement with a recently proposed analytical model. This work also demonstrates that the peripheral STI is not responsible for the observed degradations and thus that the enclosed layout TG design does not improve the radiation hardness of PPD CIS. The results of this study also lead to the conclusion that the TG OFF voltage bias during irradiation has no influence on the radiation effects. Alternative design and process solutions to improve the radiation hardness of PPD CIS are discussed.]]></abstract>
##     <issn><![CDATA[0018-9499]]></issn>
##     <htmlFlag><![CDATA[1]]></htmlFlag>
##     <arnumber><![CDATA[6939744]]></arnumber>
##     <doi><![CDATA[10.1109/TNS.2014.2360773]]></doi>
##     <publicationId><![CDATA[6939744]]></publicationId>
##     <mdurl><![CDATA[http://ieeexplore.ieee.org/xpl/articleDetails.jsp?tp=&arnumber=6939744&contentType=Journals+%26+Magazines]]></mdurl>
##     <pdf><![CDATA[http://ieeexplore.ieee.org/stamp/stamp.jsp?arnumber=6939744]]></pdf>
##   </document>
##   <document>
##     <rank>372</rank>
##     <title><![CDATA[Direct Power Control of Doubly Fed Induction Generator Under Distorted Grid Voltage]]></title>
##     <authors><![CDATA[Heng Nian;  Yipeng Song]]></authors>
##     <affiliations><![CDATA[Coll. of Electr. Eng., Zhejiang Univ., Hangzhou, China]]></affiliations>
##     <controlledterms>
##       <term><![CDATA[PI control]]></term>
##       <term><![CDATA[asynchronous generators]]></term>
##       <term><![CDATA[closed loop systems]]></term>
##       <term><![CDATA[harmonic distortion]]></term>
##       <term><![CDATA[machine vector control]]></term>
##       <term><![CDATA[power generation control]]></term>
##       <term><![CDATA[power grids]]></term>
##       <term><![CDATA[reactive power control]]></term>
##       <term><![CDATA[stability]]></term>
##       <term><![CDATA[wind power plants]]></term>
##     </controlledterms>
##     <thesaurusterms>
##       <term><![CDATA[Harmonic analysis]]></term>
##       <term><![CDATA[Harmonic distortion]]></term>
##       <term><![CDATA[Reactive power]]></term>
##       <term><![CDATA[Regulators]]></term>
##       <term><![CDATA[Rotors]]></term>
##       <term><![CDATA[Stators]]></term>
##       <term><![CDATA[Voltage control]]></term>
##     </thesaurusterms>
##     <pubtitle><![CDATA[Power Electronics, IEEE Transactions on]]></pubtitle>
##     <punumber><![CDATA[63]]></punumber>
##     <pubtype><![CDATA[Journals & Magazines]]></pubtype>
##     <publisher><![CDATA[IEEE]]></publisher>
##     <volume><![CDATA[29]]></volume>
##     <issue><![CDATA[2]]></issue>
##     <py><![CDATA[2014]]></py>
##     <spage><![CDATA[894]]></spage>
##     <epage><![CDATA[905]]></epage>
##     <abstract><![CDATA[This paper presents a direct power control (DPC) strategy for a doubly fed induction generator (DFIG)-based wind power generation system under distorted grid voltage. By analyzing the six times grid frequency power pulsation produced by the fifth and seventh grid voltage harmonic components, a novel DPC strategy with vector proportional integrated (VPI) regulator has been proposed to implement the smooth active and reactive power output of DFIG. The performance analysis of the proposed DPC strategy, including the steady and dynamic state performance, closed-loop operation stability, and rejection capability for the grid voltage distorted component and back EMF compensation item has been investigated. The availability of the proposed DPC strategy with a VPI regulator is verified by experiment results of DFIG system under harmonically distorted grid condition.]]></abstract>
##     <issn><![CDATA[0885-8993]]></issn>
##     <htmlFlag><![CDATA[1]]></htmlFlag>
##     <arnumber><![CDATA[6507348]]></arnumber>
##     <doi><![CDATA[10.1109/TPEL.2013.2258943]]></doi>
##     <publicationId><![CDATA[6507348]]></publicationId>
##     <mdurl><![CDATA[http://ieeexplore.ieee.org/xpl/articleDetails.jsp?tp=&arnumber=6507348&contentType=Journals+%26+Magazines]]></mdurl>
##     <pdf><![CDATA[http://ieeexplore.ieee.org/stamp/stamp.jsp?arnumber=6507348]]></pdf>
##   </document>
##   <document>
##     <rank>373</rank>
##     <title><![CDATA[Stability Analysis of a Hierarchical Architecture for Discrete-Time Sensor-Based Control of Robotic Systems]]></title>
##     <authors><![CDATA[Bjerkeng, M.;  Falco, P.;  Natale, C.;  Pettersen, K.Y.]]></authors>
##     <affiliations><![CDATA[Dept. of Eng. Cybern., Norwegian Univ. of Sci. & Technol., Trondheim, Norway]]></affiliations>
##     <controlledterms>
##       <term><![CDATA[delays]]></term>
##       <term><![CDATA[discrete time systems]]></term>
##       <term><![CDATA[feedback]]></term>
##       <term><![CDATA[industrial manipulators]]></term>
##       <term><![CDATA[manipulator kinematics]]></term>
##       <term><![CDATA[motion control]]></term>
##       <term><![CDATA[robot vision]]></term>
##       <term><![CDATA[sensors]]></term>
##       <term><![CDATA[stability]]></term>
##       <term><![CDATA[synchronisation]]></term>
##       <term><![CDATA[visual servoing]]></term>
##     </controlledterms>
##     <thesaurusterms>
##       <term><![CDATA[Delays]]></term>
##       <term><![CDATA[Joints]]></term>
##       <term><![CDATA[Robot kinematics]]></term>
##       <term><![CDATA[Robot sensing systems]]></term>
##       <term><![CDATA[Stability criteria]]></term>
##     </thesaurusterms>
##     <pubtitle><![CDATA[Robotics, IEEE Transactions on]]></pubtitle>
##     <punumber><![CDATA[8860]]></punumber>
##     <pubtype><![CDATA[Journals & Magazines]]></pubtype>
##     <publisher><![CDATA[IEEE]]></publisher>
##     <volume><![CDATA[30]]></volume>
##     <issue><![CDATA[3]]></issue>
##     <py><![CDATA[2014]]></py>
##     <spage><![CDATA[745]]></spage>
##     <epage><![CDATA[753]]></epage>
##     <abstract><![CDATA[The stability of discrete time kinematic sensor-based control of robots is investigated in this paper. A hierarchical inner-loop/outer-loop control architecture common for a generic robotic system is considered. The inner loop is composed of a servo-level joint controller and higher level kinematic feedback is performed in the outer loop. Stability results derived in this paper are of interest in several applications including visual servoing problems, redundancy control, and coordination/synchronization problems. The stability of the overall system is investigated taking into account input/output delays and the inner loop dynamics. A necessary and sufficient condition that the gain of the outer feedback loop has to satisfy to ensure local stability is derived. Experiments on a Kuka K-R16 manipulator have been performed in order to validate the theoretical findings on a real robotic system and show their practical relevance.]]></abstract>
##     <issn><![CDATA[1552-3098]]></issn>
##     <htmlFlag><![CDATA[1]]></htmlFlag>
##     <arnumber><![CDATA[6710135]]></arnumber>
##     <doi><![CDATA[10.1109/TRO.2013.2294882]]></doi>
##     <publicationId><![CDATA[6710135]]></publicationId>
##     <mdurl><![CDATA[http://ieeexplore.ieee.org/xpl/articleDetails.jsp?tp=&arnumber=6710135&contentType=Journals+%26+Magazines]]></mdurl>
##     <pdf><![CDATA[http://ieeexplore.ieee.org/stamp/stamp.jsp?arnumber=6710135]]></pdf>
##   </document>
##   <document>
##     <rank>374</rank>
##     <title><![CDATA[Flexible Optical Cross-Connect Structures Supporting WDM Multicast With Multiple Pumps for Multiple Channels]]></title>
##     <authors><![CDATA[Danshi Wang;  Min Zhang;  Ze Li;  Yue Cui]]></authors>
##     <affiliations><![CDATA[State Key Lab. of Inf. Photonics & Opt. Commun., Beijing Univ. of Posts & Telecommun., Beijing, China]]></affiliations>
##     <controlledterms>
##       <term><![CDATA[multicast communication]]></term>
##       <term><![CDATA[multiwave mixing]]></term>
##       <term><![CDATA[optical couplers]]></term>
##       <term><![CDATA[optical interconnections]]></term>
##       <term><![CDATA[quadrature phase shift keying]]></term>
##       <term><![CDATA[semiconductor optical amplifiers]]></term>
##       <term><![CDATA[wavelength division multiplexing]]></term>
##     </controlledterms>
##     <thesaurusterms>
##       <term><![CDATA[Four-wave mixing]]></term>
##       <term><![CDATA[Multicast communication]]></term>
##       <term><![CDATA[Nonlinear optics]]></term>
##       <term><![CDATA[Optical pumping]]></term>
##       <term><![CDATA[Optical signal processing]]></term>
##       <term><![CDATA[Phase shift keying]]></term>
##       <term><![CDATA[Semiconductor optical amplifiers]]></term>
##       <term><![CDATA[Wavelength division multiplexing]]></term>
##     </thesaurusterms>
##     <pubtitle><![CDATA[Photonics Journal, IEEE]]></pubtitle>
##     <punumber><![CDATA[4563994]]></punumber>
##     <pubtype><![CDATA[Journals & Magazines]]></pubtype>
##     <publisher><![CDATA[IEEE]]></publisher>
##     <volume><![CDATA[6]]></volume>
##     <issue><![CDATA[6]]></issue>
##     <py><![CDATA[2014]]></py>
##     <spage><![CDATA[1]]></spage>
##     <epage><![CDATA[12]]></epage>
##     <abstract><![CDATA[We not only propose three flexible optical cross-connect (OXC) structures, with the ability of wavelength-division multiplexing (WDM) multicast, but also experimentally demonstrate two WDM multicast schemes based on four-wave mixing in a semiconductor optical amplifier (SOA). One-to-ten WDM multicast of 25-Gb/s QPSK signals is achieved with three pumps, and the power penalties for all signals are less than 1.7 dB. For the first time, dual-channel WDM multicasts are simultaneously realized with only two pumps. One-to-six multicast for one input with the largest power penalty of 1.28 dB is obtained, whereas one-to-three multicast for the other input with the power penalty of 1.27 dB is also realized. The demonstrated schemes can be applied to the proposed OXC structures, according to the different requirements. The effect of bias current on conversion efficiency and optical signal-to-noise ratio is also discussed, and the optimal value is given.]]></abstract>
##     <issn><![CDATA[1943-0655]]></issn>
##     <htmlFlag><![CDATA[1]]></htmlFlag>
##     <arnumber><![CDATA[6966737]]></arnumber>
##     <doi><![CDATA[10.1109/JPHOT.2014.2374614]]></doi>
##     <publicationId><![CDATA[6966737]]></publicationId>
##     <mdurl><![CDATA[http://ieeexplore.ieee.org/xpl/articleDetails.jsp?tp=&arnumber=6966737&contentType=Journals+%26+Magazines]]></mdurl>
##     <pdf><![CDATA[http://ieeexplore.ieee.org/stamp/stamp.jsp?arnumber=6966737]]></pdf>
##   </document>
##   <document>
##     <rank>375</rank>
##     <title><![CDATA[Triangular Microwave Waveform Generation Based on Stimulated Brillouin Scattering]]></title>
##     <authors><![CDATA[Wen Hui Sun;  Wei Li;  Wen Ting Wang;  Wei Yu Wang;  Jian Guo Liu;  Ning Hua Zhu]]></authors>
##     <affiliations><![CDATA[State Key Lab. on Integrated Optoelectron., Inst. of Semicond., Beijing, China]]></affiliations>
##     <controlledterms>
##       <term><![CDATA[amplitude modulation]]></term>
##       <term><![CDATA[microwave generation]]></term>
##       <term><![CDATA[microwave photonics]]></term>
##       <term><![CDATA[optical harmonic generation]]></term>
##       <term><![CDATA[optical modulation]]></term>
##       <term><![CDATA[optical pumping]]></term>
##       <term><![CDATA[phase modulation]]></term>
##       <term><![CDATA[stimulated Brillouin scattering]]></term>
##     </controlledterms>
##     <thesaurusterms>
##       <term><![CDATA[Microwave photonics]]></term>
##       <term><![CDATA[Optical attenuators]]></term>
##       <term><![CDATA[Optical fibers]]></term>
##       <term><![CDATA[Optical filters]]></term>
##       <term><![CDATA[Optical polarization]]></term>
##       <term><![CDATA[Optical pumping]]></term>
##     </thesaurusterms>
##     <pubtitle><![CDATA[Photonics Journal, IEEE]]></pubtitle>
##     <punumber><![CDATA[4563994]]></punumber>
##     <pubtype><![CDATA[Journals & Magazines]]></pubtype>
##     <publisher><![CDATA[IEEE]]></publisher>
##     <volume><![CDATA[6]]></volume>
##     <issue><![CDATA[6]]></issue>
##     <py><![CDATA[2014]]></py>
##     <spage><![CDATA[1]]></spage>
##     <epage><![CDATA[7]]></epage>
##     <abstract><![CDATA[We proposed a new approach to generate full-duty-cycle triangular waveforms using a sinusoidal radio-frequency (RF) signal based on stimulated Brillouin scattering. In order to generate a triangular waveform, the even-order harmonics of the RF signal has to be removed. In our scheme, we use both the pump and Stokes waves to manipulate the amplitudes and phases of the optical components. In this way, the amplitudes and phases of the fundamental RF tone and its harmonics can be controlled. The proposed method is theoretically analyzed and experimentally verified. For a proof-of-concept demonstration, triangular waveforms at 8.5, 10.5, and 12.5 GHz are generated. The measured results agree well with the simulated ones.]]></abstract>
##     <issn><![CDATA[1943-0655]]></issn>
##     <htmlFlag><![CDATA[1]]></htmlFlag>
##     <arnumber><![CDATA[6924724]]></arnumber>
##     <doi><![CDATA[10.1109/JPHOT.2014.2361606]]></doi>
##     <publicationId><![CDATA[6924724]]></publicationId>
##     <mdurl><![CDATA[http://ieeexplore.ieee.org/xpl/articleDetails.jsp?tp=&arnumber=6924724&contentType=Journals+%26+Magazines]]></mdurl>
##     <pdf><![CDATA[http://ieeexplore.ieee.org/stamp/stamp.jsp?arnumber=6924724]]></pdf>
##   </document>
##   <document>
##     <rank>376</rank>
##     <title><![CDATA[Analysis and Suppression of Leakage Current in Cascaded-Multilevel-Inverter-Based PV Systems]]></title>
##     <authors><![CDATA[Yan Zhou;  Hui Li]]></authors>
##     <affiliations><![CDATA[Dept. of Electr. & Comput. Eng., Florida State Univ., Tallahassee, FL, USA]]></affiliations>
##     <controlledterms>
##       <term><![CDATA[invertors]]></term>
##       <term><![CDATA[leakage currents]]></term>
##       <term><![CDATA[photovoltaic power systems]]></term>
##     </controlledterms>
##     <thesaurusterms>
##       <term><![CDATA[Capacitors]]></term>
##       <term><![CDATA[Electromagnetic interference]]></term>
##       <term><![CDATA[Harmonic analysis]]></term>
##       <term><![CDATA[Inductors]]></term>
##       <term><![CDATA[Inverters]]></term>
##       <term><![CDATA[Leakage currents]]></term>
##       <term><![CDATA[Power harmonic filters]]></term>
##     </thesaurusterms>
##     <pubtitle><![CDATA[Power Electronics, IEEE Transactions on]]></pubtitle>
##     <punumber><![CDATA[63]]></punumber>
##     <pubtype><![CDATA[Journals & Magazines]]></pubtype>
##     <publisher><![CDATA[IEEE]]></publisher>
##     <volume><![CDATA[29]]></volume>
##     <issue><![CDATA[10]]></issue>
##     <py><![CDATA[2014]]></py>
##     <spage><![CDATA[5265]]></spage>
##     <epage><![CDATA[5277]]></epage>
##     <abstract><![CDATA[The transformerless cascaded multilevel inverter (CMI) is considered to be a promising topology alternative for low-cost and high-efficiency photovoltaic (PV) systems. However, the leakage current issue resulted from the parasitic capacitors between the PV panels and the earth remains a challenging in designing a reliable CMI-based PV system. In this paper, the leakage current paths in PV CMI are analyzed and the unique features are discussed. Two filter-based suppression solutions are then presented to tackle the leakage current issue in different PV CMI applications. Simplified leakage current analytical models are derived to study the suppression mechanisms and design the suppression filters. Study cases are demonstrated for each of the solutions with filter design example, simulation and experimental verifications.]]></abstract>
##     <issn><![CDATA[0885-8993]]></issn>
##     <htmlFlag><![CDATA[1]]></htmlFlag>
##     <arnumber><![CDATA[6656962]]></arnumber>
##     <doi><![CDATA[10.1109/TPEL.2013.2289939]]></doi>
##     <publicationId><![CDATA[6656962]]></publicationId>
##     <mdurl><![CDATA[http://ieeexplore.ieee.org/xpl/articleDetails.jsp?tp=&arnumber=6656962&contentType=Journals+%26+Magazines]]></mdurl>
##     <pdf><![CDATA[http://ieeexplore.ieee.org/stamp/stamp.jsp?arnumber=6656962]]></pdf>
##   </document>
##   <document>
##     <rank>377</rank>
##     <title><![CDATA[A Bicluster-Based Bayesian Principal Component Analysis Method for Microarray Missing Value Estimation]]></title>
##     <authors><![CDATA[Fanchi Meng;  Cheng Cai;  Hong Yan]]></authors>
##     <affiliations><![CDATA[Dept. of Comput. Sci., Northwest A&F Univ., Yangling, China]]></affiliations>
##     <controlledterms>
##       <term><![CDATA[DNA]]></term>
##       <term><![CDATA[biology computing]]></term>
##       <term><![CDATA[lab-on-a-chip]]></term>
##       <term><![CDATA[learning (artificial intelligence)]]></term>
##       <term><![CDATA[mean square error methods]]></term>
##       <term><![CDATA[molecular biophysics]]></term>
##       <term><![CDATA[principal component analysis]]></term>
##       <term><![CDATA[singular value decomposition]]></term>
##     </controlledterms>
##     <thesaurusterms>
##       <term><![CDATA[Bayes methods]]></term>
##       <term><![CDATA[Correlation]]></term>
##       <term><![CDATA[Estimation]]></term>
##       <term><![CDATA[Informatics]]></term>
##       <term><![CDATA[Mathematical model]]></term>
##       <term><![CDATA[Principal component analysis]]></term>
##       <term><![CDATA[Vectors]]></term>
##     </thesaurusterms>
##     <pubtitle><![CDATA[Biomedical and Health Informatics, IEEE Journal of]]></pubtitle>
##     <punumber><![CDATA[6221020]]></punumber>
##     <pubtype><![CDATA[Journals & Magazines]]></pubtype>
##     <publisher><![CDATA[IEEE]]></publisher>
##     <volume><![CDATA[18]]></volume>
##     <issue><![CDATA[3]]></issue>
##     <py><![CDATA[2014]]></py>
##     <spage><![CDATA[863]]></spage>
##     <epage><![CDATA[871]]></epage>
##     <abstract><![CDATA[Data generated from microarray experiments often suffer from missing values. As most downstream analyses need full matrices as input, these missing values have to be estimated. Bayesian principal component analysis (BPCA) is a well-known microarray missing value estimation method, but its performance is not satisfactory on datasets with strong local similarity structure. A bicluster-based BPCA (bi-BPCA) method is proposed in this paper to fully exploit local structure of the matrix. In a bicluster, the most correlated genes and experimental conditions with the missing entry are identified, and BPCA is conducted on these biclusters to estimate the missing values. An automatic parameter learning scheme is also developed to obtain optimal parameters. Experimental results on four real microarray matrices indicate that bi-BPCA obtains the lowest normalized root-mean-square error on 82.14% of all missing rates.]]></abstract>
##     <issn><![CDATA[2168-2194]]></issn>
##     <htmlFlag><![CDATA[1]]></htmlFlag>
##     <arnumber><![CDATA[6630054]]></arnumber>
##     <doi><![CDATA[10.1109/JBHI.2013.2284795]]></doi>
##     <publicationId><![CDATA[6630054]]></publicationId>
##     <mdurl><![CDATA[http://ieeexplore.ieee.org/xpl/articleDetails.jsp?tp=&arnumber=6630054&contentType=Journals+%26+Magazines]]></mdurl>
##     <pdf><![CDATA[http://ieeexplore.ieee.org/stamp/stamp.jsp?arnumber=6630054]]></pdf>
##   </document>
##   <document>
##     <rank>378</rank>
##     <title><![CDATA[CMOS Compatible Fabrication Processes for the Digital Micromirror Device]]></title>
##     <authors><![CDATA[Cuiling Gong;  Hogan, T.]]></authors>
##     <affiliations><![CDATA[Dept. of Eng., Texas Christian Univ., Fort Worth, TX, USA]]></affiliations>
##     <controlledterms>
##       <term><![CDATA[CMOS digital integrated circuits]]></term>
##       <term><![CDATA[microfabrication]]></term>
##       <term><![CDATA[micromirrors]]></term>
##     </controlledterms>
##     <thesaurusterms>
##       <term><![CDATA[Fabrication]]></term>
##       <term><![CDATA[Fasteners]]></term>
##       <term><![CDATA[Films]]></term>
##       <term><![CDATA[Metals]]></term>
##       <term><![CDATA[Micromirrors]]></term>
##       <term><![CDATA[Resists]]></term>
##     </thesaurusterms>
##     <pubtitle><![CDATA[Electron Devices Society, IEEE Journal of the]]></pubtitle>
##     <punumber><![CDATA[6245494]]></punumber>
##     <pubtype><![CDATA[Journals & Magazines]]></pubtype>
##     <publisher><![CDATA[IEEE]]></publisher>
##     <volume><![CDATA[2]]></volume>
##     <issue><![CDATA[3]]></issue>
##     <py><![CDATA[2014]]></py>
##     <spage><![CDATA[27]]></spage>
##     <epage><![CDATA[32]]></epage>
##     <abstract><![CDATA[DLP&#x00AE; technology has been widely used in the display products since it was first introduced to the world in 1996 by Texas Instruments. Projectors powered by DLP&#x00AE; technology range from cinema projectors that light up large movie theater screens to palm-sized &#x201C;Pico&#x201D; projectors. The heart of the technology is the digital micromirror device (DMD) that features an addressable array of up to 8 million microscopic mirrors. DMDs are fabricated using standard semiconductor processing equipment. However due to the unique nature of MOEMS application and digital operation of the DMDs, special CMOS-compatible fabrication processes have been developed to produce highly reflective digital micromirrors with robust operation margin and long term reliability. This paper will present an overview of the fabrication processes of the DMDs.]]></abstract>
##     <issn><![CDATA[2168-6734]]></issn>
##     <arnumber><![CDATA[6774433]]></arnumber>
##     <doi><![CDATA[10.1109/JEDS.2014.2309129]]></doi>
##     <publicationId><![CDATA[6774433]]></publicationId>
##     <mdurl><![CDATA[http://ieeexplore.ieee.org/xpl/articleDetails.jsp?tp=&arnumber=6774433&contentType=Journals+%26+Magazines]]></mdurl>
##     <pdf><![CDATA[http://ieeexplore.ieee.org/stamp/stamp.jsp?arnumber=6774433]]></pdf>
##   </document>
##   <document>
##     <rank>379</rank>
##     <title><![CDATA[A Survey of Radio Propagation Modeling for Tunnels]]></title>
##     <authors><![CDATA[Hrovat, A.;  Kandus, G.;  Javornik, T.]]></authors>
##     <affiliations><![CDATA[Dept. of Commun. Syst., Jovzef Stefan Inst., Ljubljana, Slovenia]]></affiliations>
##     <controlledterms>
##       <term><![CDATA[radiowave propagation]]></term>
##       <term><![CDATA[tunnels]]></term>
##       <term><![CDATA[underground communication]]></term>
##     </controlledterms>
##     <thesaurusterms>
##       <term><![CDATA[Attenuation]]></term>
##       <term><![CDATA[Channel models]]></term>
##       <term><![CDATA[Electromagnetic waveguides]]></term>
##       <term><![CDATA[Magnetic tunneling]]></term>
##       <term><![CDATA[Mathematical model]]></term>
##       <term><![CDATA[Radio propagation]]></term>
##       <term><![CDATA[Ray tracing]]></term>
##     </thesaurusterms>
##     <pubtitle><![CDATA[Communications Surveys & Tutorials, IEEE]]></pubtitle>
##     <punumber><![CDATA[9739]]></punumber>
##     <pubtype><![CDATA[Journals & Magazines]]></pubtype>
##     <publisher><![CDATA[IEEE]]></publisher>
##     <volume><![CDATA[16]]></volume>
##     <issue><![CDATA[2]]></issue>
##     <py><![CDATA[2014]]></py>
##     <spage><![CDATA[658]]></spage>
##     <epage><![CDATA[669]]></epage>
##     <abstract><![CDATA[Radio signal propagation modeling plays an important role in designing wireless communication systems. The propagation models are used to calculate the number and position of base stations and predict the radio coverage. Different models have been developed to predict radio propagation behavior for wireless communication systems in different operating environments. In this paper we shall limit our discussion to the latest achievements in radio propagation modeling related to tunnels. The main modeling approaches used for propagation in tunnels are reviewed, namely, numerical methods for solving Maxwell equations, waveguide or modal approach, ray tracing based methods and two-slope path loss modeling. They are discussed in terms of modeling complexity and required information on the environment including tunnel geometry and electric as well as magnetic properties of walls.]]></abstract>
##     <issn><![CDATA[1553-877X]]></issn>
##     <htmlFlag><![CDATA[1]]></htmlFlag>
##     <arnumber><![CDATA[6616687]]></arnumber>
##     <doi><![CDATA[10.1109/SURV.2013.091213.00175]]></doi>
##     <publicationId><![CDATA[6616687]]></publicationId>
##     <mdurl><![CDATA[http://ieeexplore.ieee.org/xpl/articleDetails.jsp?tp=&arnumber=6616687&contentType=Journals+%26+Magazines]]></mdurl>
##     <pdf><![CDATA[http://ieeexplore.ieee.org/stamp/stamp.jsp?arnumber=6616687]]></pdf>
##   </document>
##   <document>
##     <rank>380</rank>
##     <title><![CDATA[Requirements of an Open Data Based Business Ecosystem]]></title>
##     <authors><![CDATA[Immonen, A.;  Palviainen, M.;  Ovaska, E.]]></authors>
##     <affiliations><![CDATA[VTT Tech. Res. Centre of Finland, Oulu, Finland]]></affiliations>
##     <controlledterms>
##       <term><![CDATA[business data processing]]></term>
##       <term><![CDATA[groupware]]></term>
##     </controlledterms>
##     <thesaurusterms>
##       <term><![CDATA[Biological system modeling]]></term>
##       <term><![CDATA[Companies]]></term>
##       <term><![CDATA[Data models]]></term>
##       <term><![CDATA[Ecosystems]]></term>
##       <term><![CDATA[Licenses]]></term>
##     </thesaurusterms>
##     <pubtitle><![CDATA[Access, IEEE]]></pubtitle>
##     <punumber><![CDATA[6287639]]></punumber>
##     <pubtype><![CDATA[Journals & Magazines]]></pubtype>
##     <publisher><![CDATA[IEEE]]></publisher>
##     <volume><![CDATA[2]]></volume>
##     <py><![CDATA[2014]]></py>
##     <spage><![CDATA[88]]></spage>
##     <epage><![CDATA[103]]></epage>
##     <abstract><![CDATA[Emerging opportunities for open data based business have been recognized around the world. Open data can provide new business opportunities for actors that provide data, for actors that consume data, and for actors that develop innovative services and applications around the data. Open data based business requires business models and a collaborative environment-called an ecosystem-to support businesses based on open data, services, and applications. This paper outlines the open data ecosystem (ODE) from the business viewpoint and then defines the requirements of such an ecosystem. The outline and requirements are based on the state-of-the-art knowledge explored from the literature and the state of the practice on data-based business in the industry collected through interviews. The interviews revealed several motives and advantages of the ODE. However, there are also obstacles that should be carefully considered and solved. This paper defines the actors of the ODE and their roles in the ecosystem as well as the business model elements and services that are needed in open data based business. According to the interviews, the interest in open data and open data ecosystems is high at this moment. However, further research work is required to establish and validate the ODE in the near future.]]></abstract>
##     <issn><![CDATA[2169-3536]]></issn>
##     <htmlFlag><![CDATA[1]]></htmlFlag>
##     <arnumber><![CDATA[6730652]]></arnumber>
##     <doi><![CDATA[10.1109/ACCESS.2014.2302872]]></doi>
##     <publicationId><![CDATA[6730652]]></publicationId>
##     <mdurl><![CDATA[http://ieeexplore.ieee.org/xpl/articleDetails.jsp?tp=&arnumber=6730652&contentType=Journals+%26+Magazines]]></mdurl>
##     <pdf><![CDATA[http://ieeexplore.ieee.org/stamp/stamp.jsp?arnumber=6730652]]></pdf>
##   </document>
##   <document>
##     <rank>381</rank>
##     <title><![CDATA[Noise Interpolation for Unique Word OFDM]]></title>
##     <authors><![CDATA[Onic, A.;  Huemer, M.]]></authors>
##     <affiliations><![CDATA[Inst. of Signal Process., Johannes Kepler Univ. Linz, Linz, Austria]]></affiliations>
##     <controlledterms>
##       <term><![CDATA[OFDM modulation]]></term>
##       <term><![CDATA[error statistics]]></term>
##       <term><![CDATA[interpolation]]></term>
##       <term><![CDATA[stochastic processes]]></term>
##     </controlledterms>
##     <thesaurusterms>
##       <term><![CDATA[Decoding]]></term>
##       <term><![CDATA[Estimation]]></term>
##       <term><![CDATA[Generators]]></term>
##       <term><![CDATA[Interpolation]]></term>
##       <term><![CDATA[Noise]]></term>
##       <term><![CDATA[OFDM]]></term>
##       <term><![CDATA[Vectors]]></term>
##     </thesaurusterms>
##     <pubtitle><![CDATA[Signal Processing Letters, IEEE]]></pubtitle>
##     <punumber><![CDATA[97]]></punumber>
##     <pubtype><![CDATA[Journals & Magazines]]></pubtype>
##     <publisher><![CDATA[IEEE]]></publisher>
##     <volume><![CDATA[21]]></volume>
##     <issue><![CDATA[7]]></issue>
##     <py><![CDATA[2014]]></py>
##     <spage><![CDATA[814]]></spage>
##     <epage><![CDATA[818]]></epage>
##     <abstract><![CDATA[Unique word-orthogonal frequency division multiplexing (UW-OFDM) is known to feature an excellent bit error ratio performance when compared to conventional OFDM using cyclic prefixes (CP). In recent papers classical and Bayesian linear data estimators as well as the non-linear sphere decoding (SD) approach have been investigated in the UW-OFDM context. In general, the remaining error samples after a linear data estimation in UW-OFDM are correlated. In this work, noise interpolation (NI) is proposed, which uses preliminary data decisions together with Wiener interpolation to exploit these correlations. It turns out that the selection of samples to be used for NI is crucial. With the sample selection policy suggested in this work, NI clearly outperforms LMMSE data estimation, and due to its lower complexity compared to SD, it can be regarded as an attractive compromise for UW-OFDM reception.]]></abstract>
##     <issn><![CDATA[1070-9908]]></issn>
##     <htmlFlag><![CDATA[1]]></htmlFlag>
##     <arnumber><![CDATA[6798702]]></arnumber>
##     <doi><![CDATA[10.1109/LSP.2014.2317512]]></doi>
##     <publicationId><![CDATA[6798702]]></publicationId>
##     <mdurl><![CDATA[http://ieeexplore.ieee.org/xpl/articleDetails.jsp?tp=&arnumber=6798702&contentType=Journals+%26+Magazines]]></mdurl>
##     <pdf><![CDATA[http://ieeexplore.ieee.org/stamp/stamp.jsp?arnumber=6798702]]></pdf>
##   </document>
##   <document>
##     <rank>382</rank>
##     <title><![CDATA[Distributed Energy Efficient Clouds Over Core Networks]]></title>
##     <authors><![CDATA[Lawey, A.Q.;  El-Gorashi, T.E.H.;  Elmirghani, J.M.H.]]></authors>
##     <affiliations><![CDATA[Sch. of Electron. & Electr. Eng., Univ. of Leeds, Leeds, UK]]></affiliations>
##     <controlledterms>
##       <term><![CDATA[IP networks]]></term>
##       <term><![CDATA[cloud computing]]></term>
##       <term><![CDATA[linear programming]]></term>
##       <term><![CDATA[optical communication equipment]]></term>
##       <term><![CDATA[optical switches]]></term>
##       <term><![CDATA[telecommunication network routing]]></term>
##       <term><![CDATA[virtual machines]]></term>
##       <term><![CDATA[wavelength division multiplexing]]></term>
##     </controlledterms>
##     <thesaurusterms>
##       <term><![CDATA[Cloud computing]]></term>
##       <term><![CDATA[Energy efficiency]]></term>
##       <term><![CDATA[IP networks]]></term>
##       <term><![CDATA[Optical switches]]></term>
##       <term><![CDATA[Power demand]]></term>
##       <term><![CDATA[Servers]]></term>
##       <term><![CDATA[WDM networks]]></term>
##     </thesaurusterms>
##     <pubtitle><![CDATA[Lightwave Technology, Journal of]]></pubtitle>
##     <punumber><![CDATA[50]]></punumber>
##     <pubtype><![CDATA[Journals & Magazines]]></pubtype>
##     <publisher><![CDATA[IEEE]]></publisher>
##     <volume><![CDATA[32]]></volume>
##     <issue><![CDATA[7]]></issue>
##     <py><![CDATA[2014]]></py>
##     <spage><![CDATA[1261]]></spage>
##     <epage><![CDATA[1281]]></epage>
##     <abstract><![CDATA[In this paper, we introduce a framework for designing energy efficient cloud computing services over non-bypass IP/WDM core networks. We investigate network related factors including the centralization versus distribution of clouds and the impact of demand, content popularity and access frequency on the clouds placement, and cloud capability factors including the number of servers, switches and routers and amount of storage required in each cloud. We study the optimization of three cloud services: cloud content delivery, storage as a service (StaaS), and virtual machines (VMS) placement for processing applications. First, we develop a mixed integer linear programming (MILP) model to optimize cloud content delivery services. Our results indicate that replicating content into multiple clouds based on content popularity yields 43% total saving in power consumption compared to power un-aware centralized content delivery. Based on the model insights, we develop an energy efficient cloud content delivery heuristic, DEER-CD, with comparable power efficiency to the MILP results. Second, we extend the content delivery model to optimize StaaS applications. The results show that migrating content according to its access frequency yields up to 48% network power savings compared to serving content from a single central location. Third, we optimize the placement of VMs to minimize the total power consumption. Our results show that slicing the VMs into smaller VMs and placing them in proximity to their users saves 25% of the total power compared to a single virtualized cloud scenario. We also develop a heuristic for real time VM placement (DEER-VM) that achieves comparable power savings.]]></abstract>
##     <issn><![CDATA[0733-8724]]></issn>
##     <htmlFlag><![CDATA[1]]></htmlFlag>
##     <arnumber><![CDATA[6716972]]></arnumber>
##     <doi><![CDATA[10.1109/JLT.2014.2301450]]></doi>
##     <publicationId><![CDATA[6716972]]></publicationId>
##     <mdurl><![CDATA[http://ieeexplore.ieee.org/xpl/articleDetails.jsp?tp=&arnumber=6716972&contentType=Journals+%26+Magazines]]></mdurl>
##     <pdf><![CDATA[http://ieeexplore.ieee.org/stamp/stamp.jsp?arnumber=6716972]]></pdf>
##   </document>
##   <document>
##     <rank>383</rank>
##     <title><![CDATA[Stochastic Extraction of Elongated Curvilinear Structures With Applications]]></title>
##     <authors><![CDATA[Krylov, V.A.;  Nelson, J.D.B.]]></authors>
##     <affiliations><![CDATA[Dept. of Electr., Electron., Telecommun. Eng. & Naval Archit., Univ. of Genoa, Genoa, Italy]]></affiliations>
##     <controlledterms>
##       <term><![CDATA[Radon transforms]]></term>
##       <term><![CDATA[geophysical image processing]]></term>
##       <term><![CDATA[image recognition]]></term>
##       <term><![CDATA[mammography]]></term>
##       <term><![CDATA[medical image processing]]></term>
##       <term><![CDATA[palmprint recognition]]></term>
##       <term><![CDATA[remote sensing]]></term>
##       <term><![CDATA[stochastic processes]]></term>
##     </controlledterms>
##     <thesaurusterms>
##       <term><![CDATA[Feature extraction]]></term>
##       <term><![CDATA[Hidden Markov models]]></term>
##       <term><![CDATA[Image segmentation]]></term>
##       <term><![CDATA[Noise]]></term>
##       <term><![CDATA[Roads]]></term>
##       <term><![CDATA[Standards]]></term>
##       <term><![CDATA[Transforms]]></term>
##     </thesaurusterms>
##     <pubtitle><![CDATA[Image Processing, IEEE Transactions on]]></pubtitle>
##     <punumber><![CDATA[83]]></punumber>
##     <pubtype><![CDATA[Journals & Magazines]]></pubtype>
##     <publisher><![CDATA[IEEE]]></publisher>
##     <volume><![CDATA[23]]></volume>
##     <issue><![CDATA[12]]></issue>
##     <py><![CDATA[2014]]></py>
##     <spage><![CDATA[5360]]></spage>
##     <epage><![CDATA[5373]]></epage>
##     <abstract><![CDATA[The automatic extraction of elongated curvilinear structures (CLSs) is an important task in various image processing applications, including numerous remote sensing, and biometrical and medical problems. To address this task, we develop a stochastic approach that relies on a fixed-grid, localized Radon transform for line segment extraction and a conditional random field model to incorporate local interactions and refine the extracted CLSs. We propose several different energy data terms, the appropriate choice of which allows us to process images with different noise and geometry properties. The contribution of this paper is the design of a flexible and robust elongated CLS extraction framework that is comparatively fast due to the use of a fixed-grid configuration and fast deterministic Radon-based line detector. We present several different applications of the developed approach, namely: 1) CLS extraction in mammographic images; 2) road networks extraction from optical remotely sensed images; and 3) line extraction from palmprint images. The experimental results demonstrate that the method is fairly robust to CLS curvature and can accurately extract blurred and low-contrast elongated CLS.]]></abstract>
##     <issn><![CDATA[1057-7149]]></issn>
##     <htmlFlag><![CDATA[1]]></htmlFlag>
##     <arnumber><![CDATA[6924805]]></arnumber>
##     <doi><![CDATA[10.1109/TIP.2014.2363612]]></doi>
##     <publicationId><![CDATA[6924805]]></publicationId>
##     <mdurl><![CDATA[http://ieeexplore.ieee.org/xpl/articleDetails.jsp?tp=&arnumber=6924805&contentType=Journals+%26+Magazines]]></mdurl>
##     <pdf><![CDATA[http://ieeexplore.ieee.org/stamp/stamp.jsp?arnumber=6924805]]></pdf>
##   </document>
##   <document>
##     <rank>384</rank>
##     <title><![CDATA[Persistent Scatterer Pair Interferometry: Approach and Application to COSMO-SkyMed SAR Data]]></title>
##     <authors><![CDATA[Costantini, M.;  Falco, S.;  Malvarosa, F.;  Minati, F.;  Trillo, F.;  Vecchioli, F.]]></authors>
##     <affiliations><![CDATA[e-GEOS-an Italian Space Agency (ASI), Telespazio Co., Rome, Italy]]></affiliations>
##     <controlledterms>
##       <term><![CDATA[backscatter]]></term>
##       <term><![CDATA[displacement measurement]]></term>
##       <term><![CDATA[environmental monitoring (geophysics)]]></term>
##       <term><![CDATA[geomorphology]]></term>
##       <term><![CDATA[radar interferometry]]></term>
##       <term><![CDATA[remote sensing by radar]]></term>
##       <term><![CDATA[synthetic aperture radar]]></term>
##       <term><![CDATA[terrain mapping]]></term>
##     </controlledterms>
##     <thesaurusterms>
##       <term><![CDATA[Atmospheric measurements]]></term>
##       <term><![CDATA[Coherence]]></term>
##       <term><![CDATA[Displacement measurement]]></term>
##       <term><![CDATA[Extraterrestrial measurements]]></term>
##       <term><![CDATA[Interferometry]]></term>
##       <term><![CDATA[Synthetic aperture radar]]></term>
##       <term><![CDATA[Terrain factors]]></term>
##     </thesaurusterms>
##     <pubtitle><![CDATA[Selected Topics in Applied Earth Observations and Remote Sensing, IEEE Journal of]]></pubtitle>
##     <punumber><![CDATA[4609443]]></punumber>
##     <pubtype><![CDATA[Journals & Magazines]]></pubtype>
##     <publisher><![CDATA[IEEE]]></publisher>
##     <volume><![CDATA[7]]></volume>
##     <issue><![CDATA[7]]></issue>
##     <py><![CDATA[2014]]></py>
##     <spage><![CDATA[2869]]></spage>
##     <epage><![CDATA[2879]]></epage>
##     <abstract><![CDATA[Persistent scatterer interferometry is a widely used technique to detect and monitor slow terrain movements, with millimetric accuracy, from satellite synthetic aperture radar (SAR) data. We have recently proposed a method, named persistent scatterer pair (PSP), aimed at overcoming some limitations of standard techniques. The PSP method is characterized by the fact of exploiting only the relative properties of neighboring pairs of points for both detection and analysis of persistent scatterers (PSs), intended in the general sense of scatterers that exhibit interferometric coherence for the time period and baseline span of the acquisitions, including both point-like and distributed scatterers. Thanks to the pair-of-point approach, the PSP technique is intrinsically not affected by artifacts slowly variable in space, like those depending on atmosphere or orbits. Moreover, by exploiting a very redundant set of pair-of-point connections, the PSP approach guarantees extremely dense and accurate displacement and elevation measurements, both in correspondence of structures and when the backscattering is weak or distributed as in the case of natural terrains. In all cases, the measurements keep the full resolution of the input SAR images. In this work, the qualifying characteristics of the PSP technique are described, and several application examples and validation tests based on COSMO-SkyMed data are reported, which demonstrate the validity of the proposed approach.]]></abstract>
##     <issn><![CDATA[1939-1404]]></issn>
##     <htmlFlag><![CDATA[1]]></htmlFlag>
##     <arnumber><![CDATA[6881816]]></arnumber>
##     <doi><![CDATA[10.1109/JSTARS.2014.2343915]]></doi>
##     <publicationId><![CDATA[6881816]]></publicationId>
##     <mdurl><![CDATA[http://ieeexplore.ieee.org/xpl/articleDetails.jsp?tp=&arnumber=6881816&contentType=Journals+%26+Magazines]]></mdurl>
##     <pdf><![CDATA[http://ieeexplore.ieee.org/stamp/stamp.jsp?arnumber=6881816]]></pdf>
##   </document>
##   <document>
##     <rank>385</rank>
##     <title><![CDATA[A 3&#x2013;10 GHz IR-UWB CMOS Pulse Generator With 6 mW Peak Power Dissipation Using A Slow-Charge Fast-Discharge Technique]]></title>
##     <authors><![CDATA[Ming Shen;  Ying-Zheng Yin;  Hao Jiang;  Tong Tian;  Mikkelsen, J.H.]]></authors>
##     <affiliations><![CDATA[Dept. of Electron. Syst., Aalborg Univ., Aalborg, Denmark]]></affiliations>
##     <controlledterms>
##       <term><![CDATA[CMOS integrated circuits]]></term>
##       <term><![CDATA[pulse generators]]></term>
##       <term><![CDATA[ultra wideband communication]]></term>
##     </controlledterms>
##     <thesaurusterms>
##       <term><![CDATA[Bandwidth]]></term>
##       <term><![CDATA[CMOS integrated circuits]]></term>
##       <term><![CDATA[Discharges (electric)]]></term>
##       <term><![CDATA[Power dissipation]]></term>
##       <term><![CDATA[Pulse generation]]></term>
##       <term><![CDATA[Wireless communication]]></term>
##       <term><![CDATA[Wireless sensor networks]]></term>
##     </thesaurusterms>
##     <pubtitle><![CDATA[Microwave and Wireless Components Letters, IEEE]]></pubtitle>
##     <punumber><![CDATA[7260]]></punumber>
##     <pubtype><![CDATA[Journals & Magazines]]></pubtype>
##     <publisher><![CDATA[IEEE]]></publisher>
##     <volume><![CDATA[24]]></volume>
##     <issue><![CDATA[9]]></issue>
##     <py><![CDATA[2014]]></py>
##     <spage><![CDATA[634]]></spage>
##     <epage><![CDATA[636]]></epage>
##     <abstract><![CDATA[This letter proposes a UWB pulse generator topology featuring low peak power dissipation for applications with stringent instantaneous power requirements. This is accomplished by employing a new slow-charge fast-discharge approach to extend the time duration of the generator's peak current so that the peak value of the current is significantly reduced, while maintaining the waveform of the generated UWB pulse signal. A prototype pulse generator has been implemented using the UMC 0.18 &#x03BC;m CMOS process for validation. The pulse generator offers a 3-10 GHz bandwidth, a maximum pulse repetitive rate of 1 Gpps, a minimum peak power consumption of 6 mW, and a low energy consumption of 5 pJ/pulse. The fabricated pulse generator measures 0.16 mm<sup>2</sup>.]]></abstract>
##     <issn><![CDATA[1531-1309]]></issn>
##     <htmlFlag><![CDATA[1]]></htmlFlag>
##     <arnumber><![CDATA[6851212]]></arnumber>
##     <doi><![CDATA[10.1109/LMWC.2014.2332057]]></doi>
##     <publicationId><![CDATA[6851212]]></publicationId>
##     <mdurl><![CDATA[http://ieeexplore.ieee.org/xpl/articleDetails.jsp?tp=&arnumber=6851212&contentType=Journals+%26+Magazines]]></mdurl>
##     <pdf><![CDATA[http://ieeexplore.ieee.org/stamp/stamp.jsp?arnumber=6851212]]></pdf>
##   </document>
##   <document>
##     <rank>386</rank>
##     <title><![CDATA[Cooperative Down-Conversion Luminescence in <formula formulatype="inline"> <img src="/images/tex/21382.gif" alt=" \hbox {Tb}^{3+}/ \hbox {Yb}^{3+}"> </formula> Co-Doped <formula formulatype="inline"> <img src="/images/tex/21383.gif" alt="\hbox {LiYF}_{4}"> </formula> Single Crystals]]></title>
##     <authors><![CDATA[Li Fu;  Haiping Xia;  Yanmin Dong;  Shanshan Li;  Haochuan Jiang;  Baojiu Chen]]></authors>
##     <affiliations><![CDATA[Key Lab. of Photo-Electron. Mater., Ningbo Univ., Ningbo, China]]></affiliations>
##     <controlledterms>
##       <term><![CDATA[crystal growth]]></term>
##       <term><![CDATA[lithium compounds]]></term>
##       <term><![CDATA[luminescence]]></term>
##       <term><![CDATA[photoluminescence]]></term>
##       <term><![CDATA[photonic crystals]]></term>
##       <term><![CDATA[photons]]></term>
##       <term><![CDATA[quantum optics]]></term>
##       <term><![CDATA[terbium]]></term>
##       <term><![CDATA[ytterbium]]></term>
##     </controlledterms>
##     <thesaurusterms>
##       <term><![CDATA[Absorption]]></term>
##       <term><![CDATA[Crystals]]></term>
##       <term><![CDATA[Ions]]></term>
##       <term><![CDATA[Licenses]]></term>
##       <term><![CDATA[Photonics]]></term>
##       <term><![CDATA[Photovoltaic cells]]></term>
##     </thesaurusterms>
##     <pubtitle><![CDATA[Photonics Journal, IEEE]]></pubtitle>
##     <punumber><![CDATA[4563994]]></punumber>
##     <pubtype><![CDATA[Journals & Magazines]]></pubtype>
##     <publisher><![CDATA[IEEE]]></publisher>
##     <volume><![CDATA[6]]></volume>
##     <issue><![CDATA[1]]></issue>
##     <py><![CDATA[2014]]></py>
##     <spage><![CDATA[1]]></spage>
##     <epage><![CDATA[9]]></epage>
##     <abstract><![CDATA[Cooperative down-conversion (DC) with emission of two near-infrared photons for each blue photon absorbed was achieved in Tb<sup>3+</sup>/Yb<sup>3+</sup> co-doped yttrium lithium fluoride single crystals grown by an improved Bridgman method. The luminescent properties of the crystals were investigated through photoluminescence excitation, emission spectra, and decay curves. With the excitation of Tb<sup>3+</sup> ion by a 486 nm light, emission between 980 and 1030 nm from the Yb<sup>3+</sup>:<sup>2</sup>F<sub>5/2</sub> &#x2192; <sup>2</sup>F<sub>7/2</sub> transition was observed, and this emission originated from the DC between Tb<sup>3+</sup> and Yb<sup>3+</sup> ions. The energy transfer processes are studied based on the Inokuti-Hirayama's model, and the interaction between Tb<sup>3+</sup> and Yb<sup>3+</sup> is confirmed to be electric dipole-dipole. The large quantum cutting efficiency approaches up to 166.7% for 0.32 mol% of Tb<sup>3+</sup> and 7.98 mol% of Yb<sup>3+</sup> co-doped LiYF<sub>4</sub>, which is potentially used as a DC layer in silicon-based solar cells.]]></abstract>
##     <issn><![CDATA[1943-0655]]></issn>
##     <htmlFlag><![CDATA[1]]></htmlFlag>
##     <arnumber><![CDATA[6714393]]></arnumber>
##     <doi><![CDATA[10.1109/JPHOT.2014.2300495]]></doi>
##     <publicationId><![CDATA[6714393]]></publicationId>
##     <mdurl><![CDATA[http://ieeexplore.ieee.org/xpl/articleDetails.jsp?tp=&arnumber=6714393&contentType=Journals+%26+Magazines]]></mdurl>
##     <pdf><![CDATA[http://ieeexplore.ieee.org/stamp/stamp.jsp?arnumber=6714393]]></pdf>
##   </document>
##   <document>
##     <rank>387</rank>
##     <title><![CDATA[Effective CPR Procedure With Real Time Evaluation and Feedback Using Smartphones]]></title>
##     <authors><![CDATA[Gupta, N.K.;  Dantu, V.;  Dantu, R.]]></authors>
##     <affiliations><![CDATA[Univ. of North Texas, Denton, TX, USA]]></affiliations>
##     <controlledterms>
##       <term><![CDATA[blood]]></term>
##       <term><![CDATA[cardiology]]></term>
##       <term><![CDATA[feedback]]></term>
##       <term><![CDATA[medical computing]]></term>
##       <term><![CDATA[pneumodynamics]]></term>
##       <term><![CDATA[real-time systems]]></term>
##       <term><![CDATA[smart phones]]></term>
##     </controlledterms>
##     <thesaurusterms>
##       <term><![CDATA[Acceleration]]></term>
##       <term><![CDATA[Accelerometers]]></term>
##       <term><![CDATA[Blood]]></term>
##       <term><![CDATA[Frequency measurement]]></term>
##       <term><![CDATA[Oxygen]]></term>
##       <term><![CDATA[Real-time systems]]></term>
##       <term><![CDATA[Smart phones]]></term>
##     </thesaurusterms>
##     <pubtitle><![CDATA[Translational Engineering in Health and Medicine, IEEE Journal of]]></pubtitle>
##     <punumber><![CDATA[6221039]]></punumber>
##     <pubtype><![CDATA[Journals & Magazines]]></pubtype>
##     <publisher><![CDATA[IEEE]]></publisher>
##     <volume><![CDATA[2]]></volume>
##     <py><![CDATA[2014]]></py>
##     <spage><![CDATA[1]]></spage>
##     <epage><![CDATA[11]]></epage>
##     <abstract><![CDATA[Timely cardio pulmonary resuscitation (CPR) can mean the difference between life and death. A trained person may not be available at emergency sites to give CPR. Normally, a 9-1-1 operator gives verbal instructions over the phone to a person giving CPR. In this paper, we discuss the use of smartphones to assist in administering CPR more efficiently and accurately. The two important CPR parameters are the frequency and depth of compressions. In this paper, we used smartphones to calculate these factors and to give real-time guidance to improve CPR. In addition, we used an application to measure oxygen saturation in blood. If blood oxygen saturation falls below an acceptable threshold, the person giving CPR can be asked to do mouth-to-mouth breathing. The 9-1-1 operator receives this information real time and can further guide the person giving CPR. Our experiments show accuracy &gt;90% for compression frequency, depth, and oxygen saturation.]]></abstract>
##     <issn><![CDATA[2168-2372]]></issn>
##     <htmlFlag><![CDATA[1]]></htmlFlag>
##     <arnumber><![CDATA[6823662]]></arnumber>
##     <doi><![CDATA[10.1109/JTEHM.2014.2327612]]></doi>
##     <publicationId><![CDATA[6823662]]></publicationId>
##     <mdurl><![CDATA[http://ieeexplore.ieee.org/xpl/articleDetails.jsp?tp=&arnumber=6823662&contentType=Journals+%26+Magazines]]></mdurl>
##     <pdf><![CDATA[http://ieeexplore.ieee.org/stamp/stamp.jsp?arnumber=6823662]]></pdf>
##   </document>
##   <document>
##     <rank>388</rank>
##     <title><![CDATA[High Broadband Transmission of Light Through a Multi-Layer Sandwich Microstructure Containing a Seamless Metallic Film]]></title>
##     <authors><![CDATA[Hu Quan Li;  Jinsong Liu;  Zhengang Yang;  Kejia Wang]]></authors>
##     <affiliations><![CDATA[Sch. of Opt. & Electron. Inf., Huazhong Univ. of Sci. & Technol., Wuhan, China]]></affiliations>
##     <controlledterms>
##       <term><![CDATA[MIS structures]]></term>
##       <term><![CDATA[composite materials]]></term>
##       <term><![CDATA[crystal microstructure]]></term>
##       <term><![CDATA[dielectric thin films]]></term>
##       <term><![CDATA[electrical resistivity]]></term>
##       <term><![CDATA[elemental semiconductors]]></term>
##       <term><![CDATA[infrared spectra]]></term>
##       <term><![CDATA[light polarisation]]></term>
##       <term><![CDATA[metallic thin films]]></term>
##       <term><![CDATA[multilayers]]></term>
##       <term><![CDATA[sandwich structures]]></term>
##       <term><![CDATA[semiconductor thin films]]></term>
##       <term><![CDATA[silicon]]></term>
##       <term><![CDATA[silicon compounds]]></term>
##       <term><![CDATA[silver]]></term>
##       <term><![CDATA[surface plasmon resonance]]></term>
##       <term><![CDATA[transfer function matrices]]></term>
##       <term><![CDATA[visible spectra]]></term>
##     </controlledterms>
##     <thesaurusterms>
##       <term><![CDATA[Broadband communication]]></term>
##       <term><![CDATA[Cascading style sheets]]></term>
##       <term><![CDATA[Dielectric films]]></term>
##       <term><![CDATA[Optical films]]></term>
##       <term><![CDATA[Optical filters]]></term>
##       <term><![CDATA[Plasmons]]></term>
##     </thesaurusterms>
##     <pubtitle><![CDATA[Photonics Journal, IEEE]]></pubtitle>
##     <punumber><![CDATA[4563994]]></punumber>
##     <pubtype><![CDATA[Journals & Magazines]]></pubtype>
##     <publisher><![CDATA[IEEE]]></publisher>
##     <volume><![CDATA[6]]></volume>
##     <issue><![CDATA[5]]></issue>
##     <py><![CDATA[2014]]></py>
##     <spage><![CDATA[1]]></spage>
##     <epage><![CDATA[7]]></epage>
##     <abstract><![CDATA[A category of microstructure is proposed, which consists of a seamless metallic film sandwiched in multi-layer dielectric films of Si and SiO<sub>2</sub>. Such a structure can support unimaginable high broadband electromagnetic transmission from visible to near infrared. The transfer matrix theory is used to perform the numerical experiment and analyze the transmission characteristics. It has been revealed that Fabry-Pe&#x0301;rot and bulk plasmon resonant modes excited in the composite structure contribute to the high broadband transmission that is weakly dependent on the geometry parameters and the polarization and incident angle of the light. When using a 15-nm Ag film, the calculated structure's sheet resistance is ~2 &#x03A9;/sq, which is superior to that of indium tin oxide glasses, thus making promising applications in optoelectronic devices, especially in photovoltaics.]]></abstract>
##     <issn><![CDATA[1943-0655]]></issn>
##     <htmlFlag><![CDATA[1]]></htmlFlag>
##     <arnumber><![CDATA[6926778]]></arnumber>
##     <doi><![CDATA[10.1109/JPHOT.2014.2361627]]></doi>
##     <publicationId><![CDATA[6926778]]></publicationId>
##     <mdurl><![CDATA[http://ieeexplore.ieee.org/xpl/articleDetails.jsp?tp=&arnumber=6926778&contentType=Journals+%26+Magazines]]></mdurl>
##     <pdf><![CDATA[http://ieeexplore.ieee.org/stamp/stamp.jsp?arnumber=6926778]]></pdf>
##   </document>
##   <document>
##     <rank>389</rank>
##     <title><![CDATA[Stand-Alone Three-Dimensional Optical Tweezers Based on Fibred Bowtie Nanoaperture]]></title>
##     <authors><![CDATA[Hameed, N.M.;  El Eter, A.;  Grosjean, T.;  Baida, F.I.]]></authors>
##     <affiliations><![CDATA[Dept. d'Opt. P.M. Duffieux, Univ. de Franche-Comte, Besancon, France]]></affiliations>
##     <controlledterms>
##       <term><![CDATA[aperture antennas]]></term>
##       <term><![CDATA[bow-tie antennas]]></term>
##       <term><![CDATA[nanoparticles]]></term>
##       <term><![CDATA[nanophotonics]]></term>
##       <term><![CDATA[optical fibres]]></term>
##       <term><![CDATA[optical microscopes]]></term>
##       <term><![CDATA[radiation pressure]]></term>
##       <term><![CDATA[water]]></term>
##     </controlledterms>
##     <thesaurusterms>
##       <term><![CDATA[Biomedical optical imaging]]></term>
##       <term><![CDATA[Charge carrier processes]]></term>
##       <term><![CDATA[Force]]></term>
##       <term><![CDATA[Optical diffraction]]></term>
##       <term><![CDATA[Optical imaging]]></term>
##       <term><![CDATA[Optical polarization]]></term>
##       <term><![CDATA[Optical resonators]]></term>
##     </thesaurusterms>
##     <pubtitle><![CDATA[Photonics Journal, IEEE]]></pubtitle>
##     <punumber><![CDATA[4563994]]></punumber>
##     <pubtype><![CDATA[Journals & Magazines]]></pubtype>
##     <publisher><![CDATA[IEEE]]></publisher>
##     <volume><![CDATA[6]]></volume>
##     <issue><![CDATA[4]]></issue>
##     <py><![CDATA[2014]]></py>
##     <spage><![CDATA[1]]></spage>
##     <epage><![CDATA[10]]></epage>
##     <abstract><![CDATA[We study optical trapping of small particles based on the use of a bowtie nanoaperture antenna (BNA) engraved at the end of a metal-coated near-field optical microscope tip. Within the obtained light confinement, a 3-D trapping of latex nanoparticles is studied and quantified as a function of the incident light power. Good agreement between experimental and numerical results is obtained for a BNA operating in water at (&#x03BB; = 1064 nm) that faithfully traps 250-nm-radius latex particles. Further numerical investigations are performed to study the dynamic of the trapping process in comparison with experimental results. In addition, numerical results for R = 100 nm and R = 30 nm-radii latex particles are presented and show that such a configuration has the potential to trap latex particles as small as 30 nm in radius.]]></abstract>
##     <issn><![CDATA[1943-0655]]></issn>
##     <htmlFlag><![CDATA[1]]></htmlFlag>
##     <arnumber><![CDATA[6863628]]></arnumber>
##     <doi><![CDATA[10.1109/JPHOT.2014.2341011]]></doi>
##     <publicationId><![CDATA[6863628]]></publicationId>
##     <mdurl><![CDATA[http://ieeexplore.ieee.org/xpl/articleDetails.jsp?tp=&arnumber=6863628&contentType=Journals+%26+Magazines]]></mdurl>
##     <pdf><![CDATA[http://ieeexplore.ieee.org/stamp/stamp.jsp?arnumber=6863628]]></pdf>
##   </document>
##   <document>
##     <rank>390</rank>
##     <title><![CDATA[Efficient Search and Localization of Human Actions in Video Databases]]></title>
##     <authors><![CDATA[Ling Shao;  Jones, S.;  Xuelong Li]]></authors>
##     <affiliations><![CDATA[Coll. of Electron. & Inf. Eng., Nanjing Univ. of Inf. Sci. & Technol., Nanjing, China]]></affiliations>
##     <controlledterms>
##       <term><![CDATA[image motion analysis]]></term>
##       <term><![CDATA[search problems]]></term>
##       <term><![CDATA[video databases]]></term>
##       <term><![CDATA[video retrieval]]></term>
##     </controlledterms>
##     <thesaurusterms>
##       <term><![CDATA[Accuracy]]></term>
##       <term><![CDATA[Databases]]></term>
##       <term><![CDATA[Detectors]]></term>
##       <term><![CDATA[Educational institutions]]></term>
##       <term><![CDATA[Electronic mail]]></term>
##       <term><![CDATA[Feature extraction]]></term>
##       <term><![CDATA[Histograms]]></term>
##     </thesaurusterms>
##     <pubtitle><![CDATA[Circuits and Systems for Video Technology, IEEE Transactions on]]></pubtitle>
##     <punumber><![CDATA[76]]></punumber>
##     <pubtype><![CDATA[Journals & Magazines]]></pubtype>
##     <publisher><![CDATA[IEEE]]></publisher>
##     <volume><![CDATA[24]]></volume>
##     <issue><![CDATA[3]]></issue>
##     <py><![CDATA[2014]]></py>
##     <spage><![CDATA[504]]></spage>
##     <epage><![CDATA[512]]></epage>
##     <abstract><![CDATA[As digital video databases grow, so grows the problem of effectively navigating through them. In this paper we propose a novel content-based video retrieval approach to searching such video databases, specifically those involving human actions, incorporating spatio-temporal localization. We outline a novel, highly efficient localization model that first performs temporal localization based on histograms of evenly spaced time-slices, then spatial localization based on histograms of a 2-D spatial grid. We further argue that our retrieval model, based on the aforementioned localization, followed by relevance ranking, results in a highly discriminative system, while remaining an order of magnitude faster than the current state-of-the-art method. We also show how relevance feedback can be applied to our localization and ranking algorithms. As a result, the presented system is more directly applicable to real-world problems than any prior content-based video retrieval system.]]></abstract>
##     <issn><![CDATA[1051-8215]]></issn>
##     <htmlFlag><![CDATA[1]]></htmlFlag>
##     <arnumber><![CDATA[6575133]]></arnumber>
##     <doi><![CDATA[10.1109/TCSVT.2013.2276700]]></doi>
##     <publicationId><![CDATA[6575133]]></publicationId>
##     <mdurl><![CDATA[http://ieeexplore.ieee.org/xpl/articleDetails.jsp?tp=&arnumber=6575133&contentType=Journals+%26+Magazines]]></mdurl>
##     <pdf><![CDATA[http://ieeexplore.ieee.org/stamp/stamp.jsp?arnumber=6575133]]></pdf>
##   </document>
##   <document>
##     <rank>391</rank>
##     <title><![CDATA[Absorption and Losses in One-Dimensional Photonic-Crystal-Based Absorbers Incorporating Graphene]]></title>
##     <authors><![CDATA[Grande, M.;  Vincenti, M.A.;  Stomeo, T.;  de Ceglia, D.;  Petruzzelli, V.;  De Vittorio, M.;  Scalora, M.;  D'Orazio, A.]]></authors>
##     <affiliations><![CDATA[Dipt. di Ing. Elettr. e dell'Inf., Politec. di Bari, Bari, Italy]]></affiliations>
##     <controlledterms>
##       <term><![CDATA[dielectric materials]]></term>
##       <term><![CDATA[graphene]]></term>
##       <term><![CDATA[high-speed optical techniques]]></term>
##       <term><![CDATA[optical losses]]></term>
##       <term><![CDATA[optical saturable absorption]]></term>
##       <term><![CDATA[photodetectors]]></term>
##       <term><![CDATA[photonic crystals]]></term>
##       <term><![CDATA[slabs]]></term>
##     </controlledterms>
##     <thesaurusterms>
##       <term><![CDATA[Absorption]]></term>
##       <term><![CDATA[Dielectric losses]]></term>
##       <term><![CDATA[Extinction coefficients]]></term>
##       <term><![CDATA[Graphene]]></term>
##       <term><![CDATA[Mirrors]]></term>
##       <term><![CDATA[Modeling]]></term>
##       <term><![CDATA[Photonic crystals]]></term>
##     </thesaurusterms>
##     <pubtitle><![CDATA[Photonics Journal, IEEE]]></pubtitle>
##     <punumber><![CDATA[4563994]]></punumber>
##     <pubtype><![CDATA[Journals & Magazines]]></pubtype>
##     <publisher><![CDATA[IEEE]]></publisher>
##     <volume><![CDATA[6]]></volume>
##     <issue><![CDATA[6]]></issue>
##     <py><![CDATA[2014]]></py>
##     <spage><![CDATA[1]]></spage>
##     <epage><![CDATA[8]]></epage>
##     <abstract><![CDATA[We investigate the linear response of single and multiple graphene sheets embedded in quarter-wave one-dimensional photonic crystals (PhCs) in terms of absorption and losses. In particular, we show that it is possible to achieve near-perfect narrowband absorption when a single monolayer graphene is sandwiched between two PhC mirrors with optimized pair numbers. The simulations reveal that the resonant wavelength and the total absorption frequency may be tuned by tilting the angle of incidence of the impinging source. We also show that the losses, related to the dielectric materials constituting the one-dimensional PhC, can degrade the optical performance of the device. Conversely, by arranging the same dielectric slabs in a different order (supercell), it is possible to achieve a broadband absorption that is almost constant over a wide range of angle of incidence. In this configuration, the absorption and the bandwidth can be tuned by varying the supercell geometry. These features make these devices attractive for different applications ranging from tunable and saturable absorbers for short-pulse lasers to graphene-based photodetectors.]]></abstract>
##     <issn><![CDATA[1943-0655]]></issn>
##     <htmlFlag><![CDATA[1]]></htmlFlag>
##     <arnumber><![CDATA[6895106]]></arnumber>
##     <doi><![CDATA[10.1109/JPHOT.2014.2356495]]></doi>
##     <publicationId><![CDATA[6895106]]></publicationId>
##     <mdurl><![CDATA[http://ieeexplore.ieee.org/xpl/articleDetails.jsp?tp=&arnumber=6895106&contentType=Journals+%26+Magazines]]></mdurl>
##     <pdf><![CDATA[http://ieeexplore.ieee.org/stamp/stamp.jsp?arnumber=6895106]]></pdf>
##   </document>
##   <document>
##     <rank>392</rank>
##     <title><![CDATA[All-Electrical Thermal Monitoring of Terahertz Quantum Cascade Lasers]]></title>
##     <authors><![CDATA[Krall, M.;  Bachmann, D.;  Deutsch, C.;  Brandstetter, M.;  Detz, H.;  Andrews, A.M.;  Schrenk, W.;  Strasser, G.;  Unterrainer, K.]]></authors>
##     <affiliations><![CDATA[Photonics Inst. & the Center for Micro- & Nanostruct., Vienna Univ. of Technol., Vienna, Austria]]></affiliations>
##     <controlledterms>
##       <term><![CDATA[III-V semiconductors]]></term>
##       <term><![CDATA[aluminium compounds]]></term>
##       <term><![CDATA[gallium arsenide]]></term>
##       <term><![CDATA[laser variables measurement]]></term>
##       <term><![CDATA[microwave photonics]]></term>
##       <term><![CDATA[quantum cascade lasers]]></term>
##       <term><![CDATA[temperature measurement]]></term>
##       <term><![CDATA[temperature sensors]]></term>
##       <term><![CDATA[terahertz wave devices]]></term>
##       <term><![CDATA[thermal conductivity]]></term>
##       <term><![CDATA[waveguide lasers]]></term>
##     </controlledterms>
##     <thesaurusterms>
##       <term><![CDATA[Conductivity]]></term>
##       <term><![CDATA[Current measurement]]></term>
##       <term><![CDATA[Electrical resistance measurement]]></term>
##       <term><![CDATA[Heat sinks]]></term>
##       <term><![CDATA[Quantum cascade lasers]]></term>
##       <term><![CDATA[Resistance]]></term>
##       <term><![CDATA[Temperature measurement]]></term>
##     </thesaurusterms>
##     <pubtitle><![CDATA[Photonics Technology Letters, IEEE]]></pubtitle>
##     <punumber><![CDATA[68]]></punumber>
##     <pubtype><![CDATA[Journals & Magazines]]></pubtype>
##     <publisher><![CDATA[IEEE]]></publisher>
##     <volume><![CDATA[26]]></volume>
##     <issue><![CDATA[14]]></issue>
##     <py><![CDATA[2014]]></py>
##     <spage><![CDATA[1470]]></spage>
##     <epage><![CDATA[1473]]></epage>
##     <abstract><![CDATA[A key limitation, especially for the continuous-wave operation of terahertz quantum cascade lasers, is the large amount of heat dissipated in the active region. We demonstrate an all-electrical technique for monitoring the lattice temperature and characterizing the thermal properties of the active region, using the waveguide of the device as a temperature sensor. We report a measured temperature difference between the heat sink and top waveguide layer of up to 27 K during the continuous-wave operation of GaAs/Al<sub>0.15</sub>Ga<sub>0.85</sub>As-based devices, lasing at 2.4 THz. A thermal model of the devices is used to determine the thermal conductivity of the active region perpendicular to the semiconductor interfaces to be 7.3 W/(m&#x00B7;K).]]></abstract>
##     <issn><![CDATA[1041-1135]]></issn>
##     <htmlFlag><![CDATA[1]]></htmlFlag>
##     <arnumber><![CDATA[6819404]]></arnumber>
##     <doi><![CDATA[10.1109/LPT.2014.2326043]]></doi>
##     <publicationId><![CDATA[6819404]]></publicationId>
##     <mdurl><![CDATA[http://ieeexplore.ieee.org/xpl/articleDetails.jsp?tp=&arnumber=6819404&contentType=Journals+%26+Magazines]]></mdurl>
##     <pdf><![CDATA[http://ieeexplore.ieee.org/stamp/stamp.jsp?arnumber=6819404]]></pdf>
##   </document>
##   <document>
##     <rank>393</rank>
##     <title><![CDATA[Geometry Design Optimization of Large-Scale Broadband Antenna Array Systems]]></title>
##     <authors><![CDATA[El-Makadema, A.;  Rashid, L.;  Brown, A.K.]]></authors>
##     <affiliations><![CDATA[Sch. of EEE, Univ. of Manchester, Manchester, UK]]></affiliations>
##     <controlledterms>
##       <term><![CDATA[antenna arrays]]></term>
##       <term><![CDATA[broadband antennas]]></term>
##       <term><![CDATA[genetic algorithms]]></term>
##     </controlledterms>
##     <thesaurusterms>
##       <term><![CDATA[Antenna arrays]]></term>
##       <term><![CDATA[Apertures]]></term>
##       <term><![CDATA[Arrays]]></term>
##       <term><![CDATA[Broadband antennas]]></term>
##       <term><![CDATA[Broadband communication]]></term>
##       <term><![CDATA[Geometry]]></term>
##       <term><![CDATA[Noise]]></term>
##     </thesaurusterms>
##     <pubtitle><![CDATA[Antennas and Propagation, IEEE Transactions on]]></pubtitle>
##     <punumber><![CDATA[8]]></punumber>
##     <pubtype><![CDATA[Journals & Magazines]]></pubtype>
##     <publisher><![CDATA[IEEE]]></publisher>
##     <volume><![CDATA[62]]></volume>
##     <issue><![CDATA[4]]></issue>
##     <part><![CDATA[1]]></part>
##     <py><![CDATA[2014]]></py>
##     <spage><![CDATA[1673]]></spage>
##     <epage><![CDATA[1680]]></epage>
##     <abstract><![CDATA[The next generation radio telescopes such as the Square Kilometer Array (SKA) are expected to contain thousands of antenna array elements operating over a broad frequency range where the signals from each antenna element are combined and processed simultaneously providing high sensitivity with multiple beams providing a wide field of view. One crucial design aspect influencing both the performance and the cost of such systems is the array geometry. Due to the large bandwidth and number of broadband antenna elements, the optimization of such array system is difficult to achieve with the current array geometry optimization techniques which rely mainly on genetic algorithms and pattern search techniques. This paper provides a study of the effects of array geometry on the performance broadband array system. In addition, it provides a method where the array geometry can be more easily optimized for different applications. This is demonstrated for optimizing a typical SKA station in the frequency band between (70-450 MHz).]]></abstract>
##     <issn><![CDATA[0018-926X]]></issn>
##     <htmlFlag><![CDATA[1]]></htmlFlag>
##     <arnumber><![CDATA[6556975]]></arnumber>
##     <doi><![CDATA[10.1109/TAP.2013.2272571]]></doi>
##     <publicationId><![CDATA[6556975]]></publicationId>
##     <mdurl><![CDATA[http://ieeexplore.ieee.org/xpl/articleDetails.jsp?tp=&arnumber=6556975&contentType=Journals+%26+Magazines]]></mdurl>
##     <pdf><![CDATA[http://ieeexplore.ieee.org/stamp/stamp.jsp?arnumber=6556975]]></pdf>
##   </document>
##   <document>
##     <rank>394</rank>
##     <title><![CDATA[Using 3-D Video Game Technology in Channel Modeling]]></title>
##     <authors><![CDATA[Navarro Cadavid, A.;  Guevara Ibarra, D.;  Londono Salcedo, S.]]></authors>
##     <affiliations><![CDATA[Univ. Icesi, Cali, Colombia]]></affiliations>
##     <controlledterms>
##       <term><![CDATA[channel estimation]]></term>
##       <term><![CDATA[computer games]]></term>
##       <term><![CDATA[indoor radio]]></term>
##       <term><![CDATA[multipath channels]]></term>
##       <term><![CDATA[ray tracing]]></term>
##       <term><![CDATA[telecommunication computing]]></term>
##       <term><![CDATA[wireless channels]]></term>
##     </controlledterms>
##     <thesaurusterms>
##       <term><![CDATA[Computational modeling]]></term>
##       <term><![CDATA[Computer graphics]]></term>
##       <term><![CDATA[Game theory]]></term>
##       <term><![CDATA[Graphics processing units]]></term>
##       <term><![CDATA[Physics]]></term>
##       <term><![CDATA[Ray tracing]]></term>
##       <term><![CDATA[Three-dimensional displays]]></term>
##     </thesaurusterms>
##     <pubtitle><![CDATA[Access, IEEE]]></pubtitle>
##     <punumber><![CDATA[6287639]]></punumber>
##     <pubtype><![CDATA[Journals & Magazines]]></pubtype>
##     <publisher><![CDATA[IEEE]]></publisher>
##     <volume><![CDATA[2]]></volume>
##     <py><![CDATA[2014]]></py>
##     <spage><![CDATA[1652]]></spage>
##     <epage><![CDATA[1659]]></epage>
##     <abstract><![CDATA[We present here a paper on the potential use of game engines and graphic technologies for 3-D ray-based technologies used to simulate multipath channels. Our approach harnesses the power of video game development engines to provide an urban 3-D ray-based model for exploration and analysis of multipath channels for a wide frequency range in different complex outdoor and indoor scenarios. Game technologies offer a variety of options for exploiting the capabilities of graphical processing units and provide high performance in computing time with accurate results for channel modeling in current and future wireless technologies. We show the usefulness of this approach using our 3-D ray-based system in different applications.]]></abstract>
##     <issn><![CDATA[2169-3536]]></issn>
##     <htmlFlag><![CDATA[1]]></htmlFlag>
##     <arnumber><![CDATA[6955706]]></arnumber>
##     <doi><![CDATA[10.1109/ACCESS.2014.2370758]]></doi>
##     <publicationId><![CDATA[6955706]]></publicationId>
##     <mdurl><![CDATA[http://ieeexplore.ieee.org/xpl/articleDetails.jsp?tp=&arnumber=6955706&contentType=Journals+%26+Magazines]]></mdurl>
##     <pdf><![CDATA[http://ieeexplore.ieee.org/stamp/stamp.jsp?arnumber=6955706]]></pdf>
##   </document>
##   <document>
##     <rank>395</rank>
##     <title><![CDATA[Efficient Exploratory Learning of Inverse Kinematics on a Bionic Elephant Trunk]]></title>
##     <authors><![CDATA[Rolf, M.;  Steil, J.J.]]></authors>
##     <affiliations><![CDATA[Res. Inst. for Cognition & Robot., Bielefeld Univ., Bielefeld, Germany]]></affiliations>
##     <controlledterms>
##       <term><![CDATA[feedback]]></term>
##       <term><![CDATA[learning (artificial intelligence)]]></term>
##       <term><![CDATA[manipulator kinematics]]></term>
##       <term><![CDATA[statistical analysis]]></term>
##     </controlledterms>
##     <thesaurusterms>
##       <term><![CDATA[Accuracy]]></term>
##       <term><![CDATA[Actuators]]></term>
##       <term><![CDATA[Bellows]]></term>
##       <term><![CDATA[Inverse problems]]></term>
##       <term><![CDATA[Kinematics]]></term>
##       <term><![CDATA[Robot sensing systems]]></term>
##     </thesaurusterms>
##     <pubtitle><![CDATA[Neural Networks and Learning Systems, IEEE Transactions on]]></pubtitle>
##     <punumber><![CDATA[5962385]]></punumber>
##     <pubtype><![CDATA[Journals & Magazines]]></pubtype>
##     <publisher><![CDATA[IEEE]]></publisher>
##     <volume><![CDATA[25]]></volume>
##     <issue><![CDATA[6]]></issue>
##     <py><![CDATA[2014]]></py>
##     <spage><![CDATA[1147]]></spage>
##     <epage><![CDATA[1160]]></epage>
##     <abstract><![CDATA[We present an approach to learn the inverse kinematics of the &#x201C;bionic handling assistant&#x201D;-an elephant trunk robot. This task comprises substantial challenges including high dimensionality, restrictive and unknown actuation ranges, and nonstationary system behavior. We use a recent exploration scheme, online goal babbling, which deals with these challenges by bootstrapping and adapting the inverse kinematics on the fly. We show the success of the method in extensive real-world experiments on the nonstationary robot, including a novel combination of learning and traditional feedback control. Simulations further investigate the impact of nonstationary actuation ranges, drifting sensors, and morphological changes. The experiments provide the first substantial quantitative real-world evidence for the success of goal-directed bootstrapping schemes, moreover with the challenge of nonstationary system behavior. We thereby provide the first functioning control concept for this challenging robot platform.]]></abstract>
##     <issn><![CDATA[2162-237X]]></issn>
##     <arnumber><![CDATA[6671436]]></arnumber>
##     <doi><![CDATA[10.1109/TNNLS.2013.2287890]]></doi>
##     <publicationId><![CDATA[6671436]]></publicationId>
##     <mdurl><![CDATA[http://ieeexplore.ieee.org/xpl/articleDetails.jsp?tp=&arnumber=6671436&contentType=Journals+%26+Magazines]]></mdurl>
##     <pdf><![CDATA[http://ieeexplore.ieee.org/stamp/stamp.jsp?arnumber=6671436]]></pdf>
##   </document>
##   <document>
##     <rank>396</rank>
##     <title><![CDATA[Cascaded Cladding Light Extracting Strippers for High Power Fiber Lasers and Amplifiers]]></title>
##     <authors><![CDATA[Wei Guo;  Zilun Chen;  Hang Zhou;  Jie Li;  Jing Hou]]></authors>
##     <affiliations><![CDATA[Coll. of Optoelectron. Sci. & Eng., Nat. Univ. of Defense Technol., Changsha, China]]></affiliations>
##     <controlledterms>
##       <term><![CDATA[fibre lasers]]></term>
##       <term><![CDATA[optical fibre amplifiers]]></term>
##       <term><![CDATA[optical fibre cladding]]></term>
##     </controlledterms>
##     <thesaurusterms>
##       <term><![CDATA[Attenuation]]></term>
##       <term><![CDATA[Diode lasers]]></term>
##       <term><![CDATA[Fiber lasers]]></term>
##       <term><![CDATA[Optical fiber amplifiers]]></term>
##       <term><![CDATA[Optical fiber devices]]></term>
##       <term><![CDATA[Plastics]]></term>
##       <term><![CDATA[Refractive index]]></term>
##     </thesaurusterms>
##     <pubtitle><![CDATA[Photonics Journal, IEEE]]></pubtitle>
##     <punumber><![CDATA[4563994]]></punumber>
##     <pubtype><![CDATA[Journals & Magazines]]></pubtype>
##     <publisher><![CDATA[IEEE]]></publisher>
##     <volume><![CDATA[6]]></volume>
##     <issue><![CDATA[3]]></issue>
##     <py><![CDATA[2014]]></py>
##     <spage><![CDATA[1]]></spage>
##     <epage><![CDATA[6]]></epage>
##     <abstract><![CDATA[We demonstrate efficient cladding light extracting through cascaded strippers for high-power fiber lasers and amplifiers. A selected part of the fiber is divided into five segments, and the fluoroacrylate jackets of three interval segments are gradually removed and replaced with different higher index polymers. It is shown that such cascaded strippers can overcome the challenging problems of localized heating and thermal degradation of the recoating materials, which are limiting factors for the conventional high power cladding light strippers. The power-handling capability of the device is tested up to cladding light of power up to 150 W. A high attenuation of 18 dB has been achieved, whereas the maximum local temperature is less than 64 <sup>&#x00B0;</sup>C. We anticipate that this simple configuration offers a practical solution for cladding light extracting in high power fiber lasers and amplifiers.]]></abstract>
##     <issn><![CDATA[1943-0655]]></issn>
##     <htmlFlag><![CDATA[1]]></htmlFlag>
##     <arnumber><![CDATA[6810787]]></arnumber>
##     <doi><![CDATA[10.1109/JPHOT.2014.2320736]]></doi>
##     <publicationId><![CDATA[6810787]]></publicationId>
##     <mdurl><![CDATA[http://ieeexplore.ieee.org/xpl/articleDetails.jsp?tp=&arnumber=6810787&contentType=Journals+%26+Magazines]]></mdurl>
##     <pdf><![CDATA[http://ieeexplore.ieee.org/stamp/stamp.jsp?arnumber=6810787]]></pdf>
##   </document>
##   <document>
##     <rank>397</rank>
##     <title><![CDATA[Denial-of-Service Attacks in Bloom-Filter-Based Forwarding]]></title>
##     <authors><![CDATA[Antikainen, M.;  Aura, T.;  Sarela, M.]]></authors>
##     <affiliations><![CDATA[Sch. of Sci., Aalto Univ., Espoo, Finland]]></affiliations>
##     <controlledterms>
##       <term><![CDATA[Internet]]></term>
##       <term><![CDATA[computer network security]]></term>
##       <term><![CDATA[data structures]]></term>
##       <term><![CDATA[routing protocols]]></term>
##     </controlledterms>
##     <thesaurusterms>
##       <term><![CDATA[Computer crime]]></term>
##       <term><![CDATA[Network topology]]></term>
##       <term><![CDATA[Routing]]></term>
##       <term><![CDATA[Routing protocols]]></term>
##       <term><![CDATA[Topology]]></term>
##     </thesaurusterms>
##     <pubtitle><![CDATA[Networking, IEEE/ACM Transactions on]]></pubtitle>
##     <punumber><![CDATA[90]]></punumber>
##     <pubtype><![CDATA[Journals & Magazines]]></pubtype>
##     <publisher><![CDATA[IEEE]]></publisher>
##     <volume><![CDATA[22]]></volume>
##     <issue><![CDATA[5]]></issue>
##     <py><![CDATA[2014]]></py>
##     <spage><![CDATA[1463]]></spage>
##     <epage><![CDATA[1476]]></epage>
##     <abstract><![CDATA[Bloom-filter-based forwarding has been suggested to solve several fundamental problems in the current Internet, such as routing-table growth, multicast scalability issues, and denial-of-service (DoS) attacks by botnets. The proposed protocols are source-routed and include the delivery tree encoded as a Bloom filter in each packet. The network nodes forward packets based on this in-packet information without consulting routing tables and without storing per-flow state. We show that these protocols have critical vulnerabilities and make several false security assumptions. In particular, we present DoS attacks against broad classes of Bloom-filter-based protocols and conclude that the protocols are not ready for deployment on open networks. The results also help us understand the limitations and design options for Bloom-filter forwarding.]]></abstract>
##     <issn><![CDATA[1063-6692]]></issn>
##     <htmlFlag><![CDATA[1]]></htmlFlag>
##     <arnumber><![CDATA[6616021]]></arnumber>
##     <doi><![CDATA[10.1109/TNET.2013.2281614]]></doi>
##     <publicationId><![CDATA[6616021]]></publicationId>
##     <mdurl><![CDATA[http://ieeexplore.ieee.org/xpl/articleDetails.jsp?tp=&arnumber=6616021&contentType=Journals+%26+Magazines]]></mdurl>
##     <pdf><![CDATA[http://ieeexplore.ieee.org/stamp/stamp.jsp?arnumber=6616021]]></pdf>
##   </document>
##   <document>
##     <rank>398</rank>
##     <title><![CDATA[Deep-Subwavelength MIMO Using Graphene-Based Nanoscale Communication Channel]]></title>
##     <authors><![CDATA[Sugiura, S.;  Iizuka, H.]]></authors>
##     <affiliations><![CDATA[Dept. of Comput. & Inf. Sci., Tokyo Univ. of Agric. & Technol., Koganei, Japan]]></affiliations>
##     <controlledterms>
##       <term><![CDATA[MIMO communication]]></term>
##       <term><![CDATA[graphene]]></term>
##       <term><![CDATA[plasmons]]></term>
##       <term><![CDATA[radio transceivers]]></term>
##       <term><![CDATA[radiowave propagation]]></term>
##       <term><![CDATA[wireless channels]]></term>
##     </controlledterms>
##     <thesaurusterms>
##       <term><![CDATA[Antenna arrays]]></term>
##       <term><![CDATA[Graphene]]></term>
##       <term><![CDATA[MIMO]]></term>
##       <term><![CDATA[Nanoscale devices]]></term>
##       <term><![CDATA[Plasmons]]></term>
##       <term><![CDATA[Transceivers]]></term>
##       <term><![CDATA[Transmitters]]></term>
##       <term><![CDATA[Wireless communication]]></term>
##     </thesaurusterms>
##     <pubtitle><![CDATA[Access, IEEE]]></pubtitle>
##     <punumber><![CDATA[6287639]]></punumber>
##     <pubtype><![CDATA[Journals & Magazines]]></pubtype>
##     <publisher><![CDATA[IEEE]]></publisher>
##     <volume><![CDATA[2]]></volume>
##     <py><![CDATA[2014]]></py>
##     <spage><![CDATA[1240]]></spage>
##     <epage><![CDATA[1247]]></epage>
##     <abstract><![CDATA[In this paper, a novel graphene-based multiple-input multiple-output (MIMO) concept is proposed for high-rate nanoscale wireless communications between transceivers, which are nano/micrometers apart from each other. In particular, the proposed MIMO architecture considers exploiting a deep-subwavelength propagation channel made of graphene. This allows us to increase the number of transmitted symbol streams, while using a deep-subwavelength arrangement of individual plasmonic nanotransmit/receive elements in which the spacing between the transmitters and/or the receivers is tens of times smaller than the wavelength. This exclusive benefit is achieved with the aid of the phenomenon of graphene plasmons, where graphene offers the extremely confined and low-loss plasmon propagation. Hence, the proposed graphene-based MIMO system is capable of combating the fundamental limitations imposed on the classic MIMO configuration. We also present a novel graphene-specific channel adaptation technique, where the chemical potential of the graphene channel is varied to improve the power of the received signals.]]></abstract>
##     <issn><![CDATA[2169-3536]]></issn>
##     <htmlFlag><![CDATA[1]]></htmlFlag>
##     <arnumber><![CDATA[6930729]]></arnumber>
##     <doi><![CDATA[10.1109/ACCESS.2014.2364091]]></doi>
##     <publicationId><![CDATA[6930729]]></publicationId>
##     <mdurl><![CDATA[http://ieeexplore.ieee.org/xpl/articleDetails.jsp?tp=&arnumber=6930729&contentType=Journals+%26+Magazines]]></mdurl>
##     <pdf><![CDATA[http://ieeexplore.ieee.org/stamp/stamp.jsp?arnumber=6930729]]></pdf>
##   </document>
##   <document>
##     <rank>399</rank>
##     <title><![CDATA[Aging of Silane Crosslinked Polyethylene]]></title>
##     <authors><![CDATA[Hellstrom, S.;  Bergfors, F.;  Laurenson, P.;  Robinson, J.]]></authors>
##     <affiliations><![CDATA[Borealis AB, Stenungsund, Sweden]]></affiliations>
##     <controlledterms>
##       <term><![CDATA[ageing]]></term>
##       <term><![CDATA[elongation]]></term>
##       <term><![CDATA[fracture mechanics]]></term>
##       <term><![CDATA[polymer blends]]></term>
##       <term><![CDATA[statistical analysis]]></term>
##       <term><![CDATA[tensile strength]]></term>
##       <term><![CDATA[tensile testing]]></term>
##     </controlledterms>
##     <thesaurusterms>
##       <term><![CDATA[Aging]]></term>
##       <term><![CDATA[Cable insulation]]></term>
##       <term><![CDATA[Mechanical factors]]></term>
##       <term><![CDATA[Polyethylene]]></term>
##       <term><![CDATA[Power cables]]></term>
##     </thesaurusterms>
##     <pubtitle><![CDATA[Access, IEEE]]></pubtitle>
##     <punumber><![CDATA[6287639]]></punumber>
##     <pubtype><![CDATA[Journals & Magazines]]></pubtype>
##     <publisher><![CDATA[IEEE]]></publisher>
##     <volume><![CDATA[2]]></volume>
##     <py><![CDATA[2014]]></py>
##     <spage><![CDATA[177]]></spage>
##     <epage><![CDATA[182]]></epage>
##     <abstract><![CDATA[Silane crosslinked polyethylene cable insulation occasionally fails to meet the aging requirements given in technical standards. The purpose of this paper is to investigate this phenomena and establish whether the safety margin of aging tests can be increased by changes in manufacture or test procedures. Using a number of cable types with different compositions and dimensions, the evolution of the absolute values of tensile strength and elongation at break upon aging was obtained. The results show that the major changes in mechanical properties happen within the first 24-48 h. This finding is valid both for ethylene vinylsilane copolymers and for grafted silane systems. In general, the effect is more pronounced for the 100&#x00B0;C compatibility test. Statistical analysis shows that insulation crosslinked in a hot waterbath will exhibit this behavior to a lesser extent, thus increasing the safety margin in aging tests, compared with ambient curing. This paper demonstrates that preconditioning at 70&#x00B0;C has no significant impact on aging properties. In addition, only small variations in mechanical properties were seen when changing the process parameters. It is concluded that further crosslinking is the principal cause of the phenomena under investigation.]]></abstract>
##     <issn><![CDATA[2169-3536]]></issn>
##     <htmlFlag><![CDATA[1]]></htmlFlag>
##     <arnumber><![CDATA[6758336]]></arnumber>
##     <doi><![CDATA[10.1109/ACCESS.2014.2308919]]></doi>
##     <publicationId><![CDATA[6758336]]></publicationId>
##     <mdurl><![CDATA[http://ieeexplore.ieee.org/xpl/articleDetails.jsp?tp=&arnumber=6758336&contentType=Journals+%26+Magazines]]></mdurl>
##     <pdf><![CDATA[http://ieeexplore.ieee.org/stamp/stamp.jsp?arnumber=6758336]]></pdf>
##   </document>
##   <document>
##     <rank>400</rank>
##     <title><![CDATA[Patterns of Shoulder Muscle Coordination Vary Between Wheelchair Propulsion Techniques]]></title>
##     <authors><![CDATA[Liping Qi;  Wakeling, J.;  Grange, S.;  Ferguson-Pell, M.]]></authors>
##     <affiliations><![CDATA[Fac. of Rehabilitation Med., Univ. of Alberta, Edmonton, AB, Canada]]></affiliations>
##     <controlledterms>
##       <term><![CDATA[biomechanics]]></term>
##       <term><![CDATA[electromyography]]></term>
##       <term><![CDATA[handicapped aids]]></term>
##       <term><![CDATA[injuries]]></term>
##       <term><![CDATA[principal component analysis]]></term>
##       <term><![CDATA[propulsion]]></term>
##       <term><![CDATA[wheelchairs]]></term>
##     </controlledterms>
##     <pubtitle><![CDATA[Neural Systems and Rehabilitation Engineering, IEEE Transactions on]]></pubtitle>
##     <punumber><![CDATA[7333]]></punumber>
##     <pubtype><![CDATA[Journals & Magazines]]></pubtype>
##     <publisher><![CDATA[IEEE]]></publisher>
##     <volume><![CDATA[22]]></volume>
##     <issue><![CDATA[3]]></issue>
##     <py><![CDATA[2014]]></py>
##     <spage><![CDATA[559]]></spage>
##     <epage><![CDATA[566]]></epage>
##     <abstract><![CDATA[This study investigated changes in the coordination patterns of shoulder muscles and wheelchair kinetics with different propulsion techniques by comparing wheelchair users' self-selected propulsion patterns with a semicircular pattern adopted after instruction. Wheelchair kinetics data were recorded by Smart <sup>Wheel</sup> on an ergometer, while EMG activity of seven muscles was recorded with surface electrodes on 15 able-bodied inexperienced participants. The performance data in two sessions, first using a self-selected and then the learned semicircular pattern, were compared with a paired t-test. Muscle coordination patterns across seven muscles were analyzed by principal component analysis. The semicircular pattern was characterized by significantly lower push frequency, significantly longer push length, push duration and push distance (p &lt;; 0.05, all cases) without a significant increase in push force, when compared with the self-selected pattern. In addition, our results show that in the semicircular propulsion technique, synergistic muscles were recruited in distinct phases and displayed a clearer separation between activities in the push phase and recovery phase muscles. An instruction session in semicircular propulsion technique is recommended for the initial use of a wheelchair after an injury.]]></abstract>
##     <issn><![CDATA[1534-4320]]></issn>
##     <htmlFlag><![CDATA[1]]></htmlFlag>
##     <arnumber><![CDATA[6542669]]></arnumber>
##     <doi><![CDATA[10.1109/TNSRE.2013.2266136]]></doi>
##     <publicationId><![CDATA[6542669]]></publicationId>
##     <mdurl><![CDATA[http://ieeexplore.ieee.org/xpl/articleDetails.jsp?tp=&arnumber=6542669&contentType=Journals+%26+Magazines]]></mdurl>
##     <pdf><![CDATA[http://ieeexplore.ieee.org/stamp/stamp.jsp?arnumber=6542669]]></pdf>
##   </document>
##   <document>
##     <rank>401</rank>
##     <title><![CDATA[Fast Prediction of Transmission Line Radiated Emissions Using the Hertzian Dipole Method and Line-End Discontinuity Models]]></title>
##     <authors><![CDATA[Jin Meng;  Yu Xian Teo;  Thomas, D.W.P.;  Christopoulos, C.]]></authors>
##     <affiliations><![CDATA[George Green Inst. for Electromagn. Res., Univ. of Nottingham, Nottingham, UK]]></affiliations>
##     <controlledterms>
##       <term><![CDATA[approximation theory]]></term>
##       <term><![CDATA[coaxial cables]]></term>
##       <term><![CDATA[network analysers]]></term>
##       <term><![CDATA[power cables]]></term>
##       <term><![CDATA[power system interconnection]]></term>
##       <term><![CDATA[power transmission lines]]></term>
##     </controlledterms>
##     <thesaurusterms>
##       <term><![CDATA[Computational modeling]]></term>
##       <term><![CDATA[Connectors]]></term>
##       <term><![CDATA[Impedance]]></term>
##       <term><![CDATA[Integrated circuit modeling]]></term>
##       <term><![CDATA[Load modeling]]></term>
##       <term><![CDATA[Transmission line measurements]]></term>
##       <term><![CDATA[Wires]]></term>
##     </thesaurusterms>
##     <pubtitle><![CDATA[Electromagnetic Compatibility, IEEE Transactions on]]></pubtitle>
##     <punumber><![CDATA[15]]></punumber>
##     <pubtype><![CDATA[Journals & Magazines]]></pubtype>
##     <publisher><![CDATA[IEEE]]></publisher>
##     <volume><![CDATA[56]]></volume>
##     <issue><![CDATA[6]]></issue>
##     <py><![CDATA[2014]]></py>
##     <spage><![CDATA[1295]]></spage>
##     <epage><![CDATA[1303]]></epage>
##     <abstract><![CDATA[High-frequency signals on interconnects can cause significant radiated electromagnetic emissions. An intermediate level modeling method aimed at providing a faster solution with less computing resources to allow designers to obtain rapid approximations is desirable. This paper presents a modeling technique to speed up the evaluation of radiated fields from interconnect cables. Based on the Hertzian dipole radiation theory and transmission-line frequency-dependant solutions, the radiating source is modeled by the sum of a large number of short dipoles. This model allows the contributions of line-end discontinuities to be included through a vector network analyzer measurement together with a monopole approximation. The proposed method is verified by open-line and RG 58 coaxial cable measurements.]]></abstract>
##     <issn><![CDATA[0018-9375]]></issn>
##     <htmlFlag><![CDATA[1]]></htmlFlag>
##     <arnumber><![CDATA[6813629]]></arnumber>
##     <doi><![CDATA[10.1109/TEMC.2014.2318720]]></doi>
##     <publicationId><![CDATA[6813629]]></publicationId>
##     <mdurl><![CDATA[http://ieeexplore.ieee.org/xpl/articleDetails.jsp?tp=&arnumber=6813629&contentType=Journals+%26+Magazines]]></mdurl>
##     <pdf><![CDATA[http://ieeexplore.ieee.org/stamp/stamp.jsp?arnumber=6813629]]></pdf>
##   </document>
##   <document>
##     <rank>402</rank>
##     <title><![CDATA[An InGaAlAs&#x2013;InGaAs Two-Color Photodetector for Ratio Thermometry]]></title>
##     <authors><![CDATA[Xinxin Zhou;  Hobbs, M.J.;  White, B.S.;  David, J.P.R.;  Willmott, J.R.;  Chee Hing Tan]]></authors>
##     <affiliations><![CDATA[Dept. of Electron. & Electr. Eng., Univ. of Sheffield, Sheffield, UK]]></affiliations>
##     <controlledterms>
##       <term><![CDATA[III-V semiconductors]]></term>
##       <term><![CDATA[aluminium compounds]]></term>
##       <term><![CDATA[gallium arsenide]]></term>
##       <term><![CDATA[gallium compounds]]></term>
##       <term><![CDATA[indium compounds]]></term>
##       <term><![CDATA[infrared imaging]]></term>
##       <term><![CDATA[p-i-n photodiodes]]></term>
##       <term><![CDATA[photodetectors]]></term>
##       <term><![CDATA[temperature measurement]]></term>
##     </controlledterms>
##     <thesaurusterms>
##       <term><![CDATA[Indium gallium arsenide]]></term>
##       <term><![CDATA[Photodetectors]]></term>
##       <term><![CDATA[Signal to noise ratio]]></term>
##       <term><![CDATA[Silicon]]></term>
##       <term><![CDATA[Temperature measurement]]></term>
##       <term><![CDATA[Wavelength measurement]]></term>
##     </thesaurusterms>
##     <pubtitle><![CDATA[Electron Devices, IEEE Transactions on]]></pubtitle>
##     <punumber><![CDATA[16]]></punumber>
##     <pubtype><![CDATA[Journals & Magazines]]></pubtype>
##     <publisher><![CDATA[IEEE]]></publisher>
##     <volume><![CDATA[61]]></volume>
##     <issue><![CDATA[3]]></issue>
##     <py><![CDATA[2014]]></py>
##     <spage><![CDATA[838]]></spage>
##     <epage><![CDATA[843]]></epage>
##     <abstract><![CDATA[We report the evaluation of a molecular-beam epitaxy grown two-color photodetector for radiation thermometry. This two-color photodetector consists of two p<sup>+</sup>in<sup>+</sup> diodes, an In<sub>0.53</sub>Ga<sub>0.25</sub>Al<sub>0.22</sub>As (hereafter InGaAlAs) p<sup>+</sup>in<sup>+</sup> diode, which has a cutoff wavelength of 1180 nm, and an In<sub>0.53</sub>Ga<sub>0.47</sub>As (hereafter InGaAs) p<sup>+</sup>in<sup>+</sup> diode with a cutoff wavelength of 1700 nm. Our simple monolithic integrated two-color photodetector achieved comparable output signal and signal-to-noise (SNR) ratio to that of a commercial two-color Si-InGaAs photodetector. The InGaAlAs and InGaAs diodes detect blackbody temperature as low as 275&#x00B0;C and 125&#x00B0;C, respectively, with an SNR above 10. The temperature errors extracted from our data are 4&#x00B0;C at 275&#x00B0;C for the InGaAlAs diode and 2.3&#x00B0;C at 125&#x00B0;C for the InGaAs diode. As a ratio thermometer, our two-color photodetector achieves a temperature error of 12.8&#x00B0;C at 275&#x00B0;C, but this improves with temperature to 0.1&#x00B0;C at 450&#x00B0;C. These results demonstrated the potential of InGaAlAs-InGaAs two-color photodetector for the development of high performance two-color array detectors for radiation thermometry and thermal imaging of hot objects.]]></abstract>
##     <issn><![CDATA[0018-9383]]></issn>
##     <htmlFlag><![CDATA[1]]></htmlFlag>
##     <arnumber><![CDATA[6740070]]></arnumber>
##     <doi><![CDATA[10.1109/TED.2013.2297409]]></doi>
##     <publicationId><![CDATA[6740070]]></publicationId>
##     <mdurl><![CDATA[http://ieeexplore.ieee.org/xpl/articleDetails.jsp?tp=&arnumber=6740070&contentType=Journals+%26+Magazines]]></mdurl>
##     <pdf><![CDATA[http://ieeexplore.ieee.org/stamp/stamp.jsp?arnumber=6740070]]></pdf>
##   </document>
##   <document>
##     <rank>403</rank>
##     <title><![CDATA[Controlling Ambipolar Current in Tunneling FETs Using Overlapping Gate-on-Drain]]></title>
##     <authors><![CDATA[Abdi, D.B.;  Kumar, M.J.]]></authors>
##     <affiliations><![CDATA[Dept. of Electr. Eng., Indian Inst. of Technol., Delhi, New Delhi, India]]></affiliations>
##     <controlledterms>
##       <term><![CDATA[field effect transistors]]></term>
##       <term><![CDATA[tunnel transistors]]></term>
##     </controlledterms>
##     <thesaurusterms>
##       <term><![CDATA[Capacitance]]></term>
##       <term><![CDATA[Doping]]></term>
##       <term><![CDATA[Field effect transistors]]></term>
##       <term><![CDATA[Semiconductor process modeling]]></term>
##       <term><![CDATA[Tunneling]]></term>
##     </thesaurusterms>
##     <pubtitle><![CDATA[Electron Devices Society, IEEE Journal of the]]></pubtitle>
##     <punumber><![CDATA[6245494]]></punumber>
##     <pubtype><![CDATA[Journals & Magazines]]></pubtype>
##     <publisher><![CDATA[IEEE]]></publisher>
##     <volume><![CDATA[2]]></volume>
##     <issue><![CDATA[6]]></issue>
##     <py><![CDATA[2014]]></py>
##     <spage><![CDATA[187]]></spage>
##     <epage><![CDATA[190]]></epage>
##     <abstract><![CDATA[In this paper, we have demonstrated that overlapping the gate on the drain can suppress the ambipolar conduction, which is an inherent property of a tunnel field effect transistor (TFET). Unlike in the conventional TFET where the gate controls the tunneling barrier width at both source-channel and channel-drain interfaces for different polarity of gate voltage, overlapping the gate on the drain limits the gate to control only the tunneling barrier width at the source-channel interface irrespective of the polarity of the gate voltage. As a result, the proposed overlapping gate-on-drain TFET exhibits suppressed ambipolar conduction even when the drain doping is as high as 1 &#x00D7; 10<sup>19</sup> cm<sup>-3</sup>.]]></abstract>
##     <issn><![CDATA[2168-6734]]></issn>
##     <htmlFlag><![CDATA[1]]></htmlFlag>
##     <arnumber><![CDATA[6823663]]></arnumber>
##     <doi><![CDATA[10.1109/JEDS.2014.2327626]]></doi>
##     <publicationId><![CDATA[6823663]]></publicationId>
##     <mdurl><![CDATA[http://ieeexplore.ieee.org/xpl/articleDetails.jsp?tp=&arnumber=6823663&contentType=Journals+%26+Magazines]]></mdurl>
##     <pdf><![CDATA[http://ieeexplore.ieee.org/stamp/stamp.jsp?arnumber=6823663]]></pdf>
##   </document>
##   <document>
##     <rank>404</rank>
##     <title><![CDATA[Recent Advances on Singlemodal and Multimodal Face Recognition: A Survey]]></title>
##     <authors><![CDATA[Hailing Zhou;  Mian, A.;  Lei Wei;  Creighton, D.;  Hossny, M.;  Nahavandi, S.]]></authors>
##     <affiliations><![CDATA[Centre for Intell. Syst. Res., Deakin Univ., Geelong, VIC, Australia]]></affiliations>
##     <controlledterms>
##       <term><![CDATA[face recognition]]></term>
##       <term><![CDATA[infrared imaging]]></term>
##       <term><![CDATA[lighting]]></term>
##       <term><![CDATA[visual databases]]></term>
##     </controlledterms>
##     <thesaurusterms>
##       <term><![CDATA[Databases]]></term>
##       <term><![CDATA[Deformable models]]></term>
##       <term><![CDATA[Face recognition]]></term>
##       <term><![CDATA[Infrared imaging]]></term>
##       <term><![CDATA[Lighting]]></term>
##     </thesaurusterms>
##     <pubtitle><![CDATA[Human-Machine Systems, IEEE Transactions on]]></pubtitle>
##     <punumber><![CDATA[6221037]]></punumber>
##     <pubtype><![CDATA[Journals & Magazines]]></pubtype>
##     <publisher><![CDATA[IEEE]]></publisher>
##     <volume><![CDATA[44]]></volume>
##     <issue><![CDATA[6]]></issue>
##     <py><![CDATA[2014]]></py>
##     <spage><![CDATA[701]]></spage>
##     <epage><![CDATA[716]]></epage>
##     <abstract><![CDATA[High performance for face recognition systems occurs in controlled environments and degrades with variations in illumination, facial expression, and pose. Efforts have been made to explore alternate face modalities such as infrared (IR) and 3-D for face recognition. Studies also demonstrate that fusion of multiple face modalities improve performance as compared with singlemodal face recognition. This paper categorizes these algorithms into singlemodal and multimodal face recognition and evaluates methods within each category via detailed descriptions of representative work and summarizations in tables. Advantages and disadvantages of each modality for face recognition are analyzed. In addition, face databases and system evaluations are also covered.]]></abstract>
##     <issn><![CDATA[2168-2291]]></issn>
##     <htmlFlag><![CDATA[1]]></htmlFlag>
##     <arnumber><![CDATA[6876188]]></arnumber>
##     <doi><![CDATA[10.1109/THMS.2014.2340578]]></doi>
##     <publicationId><![CDATA[6876188]]></publicationId>
##     <mdurl><![CDATA[http://ieeexplore.ieee.org/xpl/articleDetails.jsp?tp=&arnumber=6876188&contentType=Journals+%26+Magazines]]></mdurl>
##     <pdf><![CDATA[http://ieeexplore.ieee.org/stamp/stamp.jsp?arnumber=6876188]]></pdf>
##   </document>
##   <document>
##     <rank>405</rank>
##     <title><![CDATA[A Remote-Controlled Airbag Device Can Improve Upper Airway Collapsibility by Producing Head Elevation With Jaw Closure in Normal Subjects Under Propofol Anesthesia]]></title>
##     <authors><![CDATA[Ishizaka, S.;  Moromugi, S.;  Kobayashi, M.;  Kajihara, H.;  Koga, K.;  Sugahara, H.;  Ishimatsu, T.;  Kurata, S.;  Kirkness, J.P.;  Oi, K.;  Ayuse, T.]]></authors>
##     <affiliations><![CDATA[Grad. Sch. of Biomed. Sci., Dept. of Clinical Physiol., Nagasaki Univ., Nagasaki, Japan]]></affiliations>
##     <controlledterms>
##       <term><![CDATA[biomedical equipment]]></term>
##       <term><![CDATA[biomedical measurement]]></term>
##       <term><![CDATA[bone]]></term>
##       <term><![CDATA[demography]]></term>
##       <term><![CDATA[drugs]]></term>
##       <term><![CDATA[inflatable structures]]></term>
##       <term><![CDATA[medical disorders]]></term>
##       <term><![CDATA[medical robotics]]></term>
##       <term><![CDATA[oxygen]]></term>
##       <term><![CDATA[patient treatment]]></term>
##       <term><![CDATA[pneumodynamics]]></term>
##       <term><![CDATA[position control]]></term>
##       <term><![CDATA[sleep]]></term>
##       <term><![CDATA[statistical analysis]]></term>
##       <term><![CDATA[telerobotics]]></term>
##     </controlledterms>
##     <thesaurusterms>
##       <term><![CDATA[Anesthesia]]></term>
##       <term><![CDATA[Head manipulation]]></term>
##       <term><![CDATA[Magnetic heads]]></term>
##       <term><![CDATA[Maintenance engineering]]></term>
##       <term><![CDATA[Robots]]></term>
##       <term><![CDATA[Sleep apnea]]></term>
##     </thesaurusterms>
##     <pubtitle><![CDATA[Translational Engineering in Health and Medicine, IEEE Journal of]]></pubtitle>
##     <punumber><![CDATA[6221039]]></punumber>
##     <pubtype><![CDATA[Journals & Magazines]]></pubtype>
##     <publisher><![CDATA[IEEE]]></publisher>
##     <volume><![CDATA[2]]></volume>
##     <py><![CDATA[2014]]></py>
##     <spage><![CDATA[1]]></spage>
##     <epage><![CDATA[9]]></epage>
##     <abstract><![CDATA[Continuous maintenance of an appropriate position of the mandible and head purely by manual manipulation is difficult, although the maneuver can restore airway patency during sleep and anesthesia. The aim of this paper was to examine the effect of head elevation with jaw closure using a remote-controlled airbag device, such as the airbag system, on passive upper airway collapsibility during propofol anesthesia. Seven male subjects were studied. Propofol infusion was used for anesthesia induction and maintenance, with a target blood propofol concentration of 1.5-2 &#x03BC;g/ml. Nasal mask pressure (P<sub>N</sub>) was intermittently reduced to evaluate upper airway collapsibility (passive P<sub>CRIT</sub>) and upstream resistance (R<sub>US</sub>) at three different head and jaw positions, jaw opening position in the supine position, jaw opening position in the sniffing position with 6-cm head elevation, and jaw closure at a 6-cm height sniffing position. The 6-cm height sniffing position with jaw closure was achieved by an airbag device that was attached to the subject's head-like headgear. Patient demographics, P<sub>CRIT</sub> and R<sub>US</sub> in each condition were compared using one-way ANOVA with a post hoc Tukey test. P &lt;; 0.05 was considered significant. We also confirmed the effects of our airbag device on improvement of upper airway collapsibility in three obstructive sleep apnea patients in a clinical study. The combination of 6-cm head elevation with jaw closure using the air-inflatable robotic airbag system decreased upper airway collapsibility (P<sub>CRIT</sub> ~ -3.4-cm H<sub>2</sub>O) compared with the baseline position (P<sub>CRIT</sub> ~ -0.8-cm H<sub>2</sub>O, P = 0.0001). In the clinical study, there was improvement of upper airway obstruction in sleep apnea patients, including decreased apnea and hypopnea duration and increased the lowest level of oxygen saturation. We demonstrated that establishment of head elevation with jaw closure achi- ved by a remote-controlled airbag device using an inflatable airbag system can produce substantial decreases in upper airway collapsibility and maintain upper airway patency during propofol anesthesia and sleep.]]></abstract>
##     <issn><![CDATA[2168-2372]]></issn>
##     <htmlFlag><![CDATA[1]]></htmlFlag>
##     <arnumber><![CDATA[6811201]]></arnumber>
##     <doi><![CDATA[10.1109/JTEHM.2014.2321062]]></doi>
##     <publicationId><![CDATA[6811201]]></publicationId>
##     <mdurl><![CDATA[http://ieeexplore.ieee.org/xpl/articleDetails.jsp?tp=&arnumber=6811201&contentType=Journals+%26+Magazines]]></mdurl>
##     <pdf><![CDATA[http://ieeexplore.ieee.org/stamp/stamp.jsp?arnumber=6811201]]></pdf>
##   </document>
##   <document>
##     <rank>406</rank>
##     <title><![CDATA[Behaviors of Electromagnetic Waves Directly Excited by Earthquakes]]></title>
##     <authors><![CDATA[Tsutsui, M.]]></authors>
##     <affiliations><![CDATA[Kyoto Sangyo Univ., Kyoto, Japan]]></affiliations>
##     <controlledterms>
##       <term><![CDATA[earthquakes]]></term>
##       <term><![CDATA[seismic waves]]></term>
##     </controlledterms>
##     <thesaurusterms>
##       <term><![CDATA[Earth]]></term>
##       <term><![CDATA[Earthquakes]]></term>
##       <term><![CDATA[Educational institutions]]></term>
##       <term><![CDATA[Seismic measurements]]></term>
##       <term><![CDATA[Seismic waves]]></term>
##       <term><![CDATA[Sensor systems]]></term>
##       <term><![CDATA[Stress]]></term>
##     </thesaurusterms>
##     <pubtitle><![CDATA[Geoscience and Remote Sensing Letters, IEEE]]></pubtitle>
##     <punumber><![CDATA[8859]]></punumber>
##     <pubtype><![CDATA[Journals & Magazines]]></pubtype>
##     <publisher><![CDATA[IEEE]]></publisher>
##     <volume><![CDATA[11]]></volume>
##     <issue><![CDATA[11]]></issue>
##     <py><![CDATA[2014]]></py>
##     <spage><![CDATA[1961]]></spage>
##     <epage><![CDATA[1965]]></epage>
##     <abstract><![CDATA[We detected electromagnetic (EM) waves directly excited by earthquakes in a deep borehole and confirmed them by simultaneous capturing of their waveforms and of seismic waves measured at the same observation site. Furthermore, the excitation mechanism of the EM pulse was confirmed as the piezoelectric effect by a laboratory experiment, in which a seismic P-wave was readily generated by a small stress impact, and the EM wave was simultaneously excited basically by the P-wave. Here, we show behaviors of seismic waves and of their excited EM waves when small and large earthquakes occurred. We also found that EM waves excited by seismic waves have leaked out of the ground surface.]]></abstract>
##     <issn><![CDATA[1545-598X]]></issn>
##     <htmlFlag><![CDATA[1]]></htmlFlag>
##     <arnumber><![CDATA[6810784]]></arnumber>
##     <doi><![CDATA[10.1109/LGRS.2014.2315208]]></doi>
##     <publicationId><![CDATA[6810784]]></publicationId>
##     <mdurl><![CDATA[http://ieeexplore.ieee.org/xpl/articleDetails.jsp?tp=&arnumber=6810784&contentType=Journals+%26+Magazines]]></mdurl>
##     <pdf><![CDATA[http://ieeexplore.ieee.org/stamp/stamp.jsp?arnumber=6810784]]></pdf>
##   </document>
##   <document>
##     <rank>407</rank>
##     <title><![CDATA[Development of a Compact Rectenna for Wireless Powering of a Head-Mountable Deep Brain Stimulation Device]]></title>
##     <authors><![CDATA[Hosain, M.K.;  Kouzani, A.Z.;  Tye, S.J.;  Abulseoud, O.A.;  Amiet, A.;  Galehdar, A.;  Kaynak, A.;  Berk, M.]]></authors>
##     <affiliations><![CDATA[Sch. of Eng., Deakin Univ., Geelong, VIC, Australia]]></affiliations>
##     <controlledterms>
##       <term><![CDATA[DC generators]]></term>
##       <term><![CDATA[bioelectric potentials]]></term>
##       <term><![CDATA[biomedical electronics]]></term>
##       <term><![CDATA[brain]]></term>
##       <term><![CDATA[impedance matching]]></term>
##       <term><![CDATA[low-pass filters]]></term>
##       <term><![CDATA[low-power electronics]]></term>
##       <term><![CDATA[patient treatment]]></term>
##       <term><![CDATA[planar inverted-F antennas]]></term>
##       <term><![CDATA[pulse generators]]></term>
##       <term><![CDATA[radiofrequency power transmission]]></term>
##       <term><![CDATA[rectennas]]></term>
##       <term><![CDATA[rectifying circuits]]></term>
##     </controlledterms>
##     <thesaurusterms>
##       <term><![CDATA[Brain modeling]]></term>
##       <term><![CDATA[DC machines]]></term>
##       <term><![CDATA[Impedance]]></term>
##       <term><![CDATA[Power transmisison]]></term>
##       <term><![CDATA[Rectennas]]></term>
##       <term><![CDATA[Satellite broadcasting]]></term>
##       <term><![CDATA[Wireless communication]]></term>
##     </thesaurusterms>
##     <pubtitle><![CDATA[Translational Engineering in Health and Medicine, IEEE Journal of]]></pubtitle>
##     <punumber><![CDATA[6221039]]></punumber>
##     <pubtype><![CDATA[Journals & Magazines]]></pubtype>
##     <publisher><![CDATA[IEEE]]></publisher>
##     <volume><![CDATA[2]]></volume>
##     <py><![CDATA[2014]]></py>
##     <spage><![CDATA[1]]></spage>
##     <epage><![CDATA[13]]></epage>
##     <abstract><![CDATA[Design of a rectangular spiral planar inverted-F antenna (PIFA) at 915 MHz for wireless power transmission applications is proposed. The antenna and rectifying circuitry form a rectenna, which can produce dc power from a distant radio frequency energy transmitter. The generated dc power is used to operate a low-power deep brain stimulation pulse generator. The proposed antenna has the dimensions of 10 mm &#x00D7; 12.5 mm &#x00D7; 1.5 mm and resonance frequency of 915 MHz with a measured bandwidth of 15 MHz at return loss of -10 dB. A dielectric substrate of FR-4 of &#x03B5;<sub>r</sub> = 4.8 and &#x03B4; = 0.015 with thickness of 1.5 mm is used for both antenna and rectifier circuit simulation and fabrication because of its availability and low cost. An L-section impedance matching circuit is used between the PIFA and voltage doubler rectifier. The impedance matching circuit also works as a low-pass filter for elimination of higher order harmonics. Maximum dc voltage at the rectenna output is 7.5 V in free space and this rectenna can drive a deep brain stimulation pulse generator at a distance of 30 cm from a radio frequency energy transmitter, which transmits power of 26.77 dBm.]]></abstract>
##     <issn><![CDATA[2168-2372]]></issn>
##     <htmlFlag><![CDATA[1]]></htmlFlag>
##     <arnumber><![CDATA[6778778]]></arnumber>
##     <doi><![CDATA[10.1109/JTEHM.2014.2313856]]></doi>
##     <publicationId><![CDATA[6778778]]></publicationId>
##     <mdurl><![CDATA[http://ieeexplore.ieee.org/xpl/articleDetails.jsp?tp=&arnumber=6778778&contentType=Journals+%26+Magazines]]></mdurl>
##     <pdf><![CDATA[http://ieeexplore.ieee.org/stamp/stamp.jsp?arnumber=6778778]]></pdf>
##   </document>
##   <document>
##     <rank>408</rank>
##     <title><![CDATA[On Using the Relative Configuration to Explore Cooperative Localization]]></title>
##     <authors><![CDATA[Ping Zhang;  Qiao Wang]]></authors>
##     <affiliations><![CDATA[Dept. of Radio Eng., Southeast Univ., Nanjing, China]]></affiliations>
##     <controlledterms>
##       <term><![CDATA[cooperative communication]]></term>
##       <term><![CDATA[distance measurement]]></term>
##       <term><![CDATA[maximum likelihood estimation]]></term>
##       <term><![CDATA[mobile computing]]></term>
##       <term><![CDATA[statistical analysis]]></term>
##       <term><![CDATA[wireless sensor networks]]></term>
##     </controlledterms>
##     <thesaurusterms>
##       <term><![CDATA[Accuracy]]></term>
##       <term><![CDATA[Distance measurement]]></term>
##       <term><![CDATA[Maximum likelihood estimation]]></term>
##       <term><![CDATA[Shape]]></term>
##       <term><![CDATA[Uncertainty]]></term>
##     </thesaurusterms>
##     <pubtitle><![CDATA[Signal Processing, IEEE Transactions on]]></pubtitle>
##     <punumber><![CDATA[78]]></punumber>
##     <pubtype><![CDATA[Journals & Magazines]]></pubtype>
##     <publisher><![CDATA[IEEE]]></publisher>
##     <volume><![CDATA[62]]></volume>
##     <issue><![CDATA[4]]></issue>
##     <py><![CDATA[2014]]></py>
##     <spage><![CDATA[968]]></spage>
##     <epage><![CDATA[980]]></epage>
##     <abstract><![CDATA[Cooperative localization differs from conventional localizations in using the measurements between the unknown nodes, which provide the relative location information of the nodes. This paper investigates cooperative localization by adopting the concept of relative configuration that describes the &#x201C;shape&#x201D; of the node network, without considering its absolute location, orientation, and/or scaling. Since the relative configuration is a non-Euclidean object, we introduce the Procrustes coordinates as a coordinate representation, suggest using the relative error as a coordinate independent error metric, and then derive the Crame&#x0301;r-Rao lower bound (CRLB) and a CRLB-type bound for the Procrustes coordinates and the relative error respectively. Three applications of the relative configuration are demonstrated: the first one gives the CRLB analysis for anchor-free localization; the second one discusses the optimal minimally constrained system (MCS) for deriving the absolute locations; and the third one refers to the anchor selection with consideration of anchor location uncertainty. These applications show the advantages of using the relative configuration to investigate cooperative localization.]]></abstract>
##     <issn><![CDATA[1053-587X]]></issn>
##     <htmlFlag><![CDATA[1]]></htmlFlag>
##     <arnumber><![CDATA[6701354]]></arnumber>
##     <doi><![CDATA[10.1109/TSP.2013.2297680]]></doi>
##     <publicationId><![CDATA[6701354]]></publicationId>
##     <mdurl><![CDATA[http://ieeexplore.ieee.org/xpl/articleDetails.jsp?tp=&arnumber=6701354&contentType=Journals+%26+Magazines]]></mdurl>
##     <pdf><![CDATA[http://ieeexplore.ieee.org/stamp/stamp.jsp?arnumber=6701354]]></pdf>
##   </document>
##   <document>
##     <rank>409</rank>
##     <title><![CDATA[Smart cymbal transducers with nitinol end caps tunable to multiple operating frequencies]]></title>
##     <authors><![CDATA[Feeney, A.;  Lucas, M.]]></authors>
##     <affiliations><![CDATA[Sch. of Eng., Univ. of Glasgow, Glasgow, UK]]></affiliations>
##     <controlledterms>
##       <term><![CDATA[differential scanning calorimetry]]></term>
##       <term><![CDATA[elasticity]]></term>
##       <term><![CDATA[nickel alloys]]></term>
##       <term><![CDATA[piezoelectric transducers]]></term>
##       <term><![CDATA[shape memory effects]]></term>
##       <term><![CDATA[titanium alloys]]></term>
##       <term><![CDATA[ultrasonic transducers]]></term>
##     </controlledterms>
##     <thesaurusterms>
##       <term><![CDATA[Acoustics]]></term>
##       <term><![CDATA[Materials]]></term>
##       <term><![CDATA[Metals]]></term>
##       <term><![CDATA[Resonant frequency]]></term>
##       <term><![CDATA[Temperature distribution]]></term>
##       <term><![CDATA[Transducers]]></term>
##       <term><![CDATA[Vibrations]]></term>
##     </thesaurusterms>
##     <pubtitle><![CDATA[Ultrasonics, Ferroelectrics, and Frequency Control, IEEE Transactions on]]></pubtitle>
##     <punumber><![CDATA[58]]></punumber>
##     <pubtype><![CDATA[Journals & Magazines]]></pubtype>
##     <publisher><![CDATA[IEEE]]></publisher>
##     <volume><![CDATA[61]]></volume>
##     <issue><![CDATA[10]]></issue>
##     <py><![CDATA[2014]]></py>
##     <spage><![CDATA[1709]]></spage>
##     <epage><![CDATA[1719]]></epage>
##     <abstract><![CDATA[Cymbal flextensional transducers have principally been adopted for sensing and actuation and their performance in higher power applications has only recently been investigated. Nitinol is a shape-memory alloy (SMA) with excellent strain recovery, durability, corrosion resistance, and fatigue strength. Although it has been incorporated in many applications, the implementation of nitinol, or any of the SMAs, in power ultrasonic applications is limited. Nitinol exhibits two phenomena, the first being the superelastic effect and the second being the shape-memory effect (SME). This paper assesses two cymbal transducers, one assembled with superelastic nitinol end caps and the other with shape-memory nitinol end caps. Characterization of the nitinol alloy before the design of such transducers is vital, so that they can be tuned to the desired operating frequencies. It is shown this can be achieved for shape-memory nitinol using differential scanning calorimetry (DSC); however, it is also shown that characterizing superelastic nitinol with DSC is problematic. Two transducers are assembled whose two operating frequencies can be tuned, and their dynamic behaviors are compared. Both transducers are shown to be tunable, with limitation for high-power applications largely being associated with the bond layer.]]></abstract>
##     <issn><![CDATA[0885-3010]]></issn>
##     <htmlFlag><![CDATA[1]]></htmlFlag>
##     <arnumber><![CDATA[6910381]]></arnumber>
##     <doi><![CDATA[10.1109/TUFFC.2013.006231]]></doi>
##     <publicationId><![CDATA[6910381]]></publicationId>
##     <mdurl><![CDATA[http://ieeexplore.ieee.org/xpl/articleDetails.jsp?tp=&arnumber=6910381&contentType=Journals+%26+Magazines]]></mdurl>
##     <pdf><![CDATA[http://ieeexplore.ieee.org/stamp/stamp.jsp?arnumber=6910381]]></pdf>
##   </document>
##   <document>
##     <rank>410</rank>
##     <title><![CDATA[In-Band Estimation of Optical Signal-to-Noise Ratio From Equalized Signals in Digital Coherent Receivers]]></title>
##     <authors><![CDATA[Faruk, M.S.;  Mori, Y.;  Kikuchi, K.]]></authors>
##     <affiliations><![CDATA[Dept. of Electr. & Electron. Eng., Dhaka Univ. of Eng. & Technol., Gazipur, Bangladesh]]></affiliations>
##     <controlledterms>
##       <term><![CDATA[optical information processing]]></term>
##       <term><![CDATA[optical receivers]]></term>
##       <term><![CDATA[quadrature amplitude modulation]]></term>
##       <term><![CDATA[quadrature phase shift keying]]></term>
##       <term><![CDATA[wavelength division multiplexing]]></term>
##     </controlledterms>
##     <thesaurusterms>
##       <term><![CDATA[Estimation]]></term>
##       <term><![CDATA[Optical noise]]></term>
##       <term><![CDATA[Optical receivers]]></term>
##       <term><![CDATA[Optical signal processing]]></term>
##       <term><![CDATA[Signal to noise ratio]]></term>
##       <term><![CDATA[Wavelength division multiplexing]]></term>
##     </thesaurusterms>
##     <pubtitle><![CDATA[Photonics Journal, IEEE]]></pubtitle>
##     <punumber><![CDATA[4563994]]></punumber>
##     <pubtype><![CDATA[Journals & Magazines]]></pubtype>
##     <publisher><![CDATA[IEEE]]></publisher>
##     <volume><![CDATA[6]]></volume>
##     <issue><![CDATA[1]]></issue>
##     <py><![CDATA[2014]]></py>
##     <spage><![CDATA[1]]></spage>
##     <epage><![CDATA[9]]></epage>
##     <abstract><![CDATA[We propose a novel method of in-band estimation of optical signal-to-noise ratio (OSNR) using a digital coherent receiver, where OSNR is determined from second- and fourth-order statistical moments of equalized signals in any modulation format. Our proposed method is especially important in recently-developed Nyquist wavelength-division multiplexed (WDM) systems and/or reconfigurable optical-add/drop-multiplexed (ROADM) networks, because in these systems and networks, we cannot apply the conventional OSNR estimation method based on optical-spectrum measurements of the in-band signal and the out-of-band noise. Effectiveness of the proposed method is validated with computer simulations of Nyquist-WDM systems and ROADM networks using 25-Gbaud quadrature phase-shift keying (QPSK) and 16 quadrature-amplitude modulation (16-QAM) formats.]]></abstract>
##     <issn><![CDATA[1943-0655]]></issn>
##     <htmlFlag><![CDATA[1]]></htmlFlag>
##     <arnumber><![CDATA[6731549]]></arnumber>
##     <doi><![CDATA[10.1109/JPHOT.2014.2304557]]></doi>
##     <publicationId><![CDATA[6731549]]></publicationId>
##     <mdurl><![CDATA[http://ieeexplore.ieee.org/xpl/articleDetails.jsp?tp=&arnumber=6731549&contentType=Journals+%26+Magazines]]></mdurl>
##     <pdf><![CDATA[http://ieeexplore.ieee.org/stamp/stamp.jsp?arnumber=6731549]]></pdf>
##   </document>
##   <document>
##     <rank>411</rank>
##     <title><![CDATA[Remote Photothermal Actuation for Calibration of In-Phase and Quadrature Readout in a Mechanically Amplified Fabry&#x2013;P&#x00E9;rot Accelerometer]]></title>
##     <authors><![CDATA[Davies, E.;  George, D.S.;  Holmes, A.S.]]></authors>
##     <affiliations><![CDATA[Dept. of Electr. & Electron. Eng. (EEE), Imperial Coll. London, London, UK]]></affiliations>
##     <controlledterms>
##       <term><![CDATA[Fabry-Perot interferometers]]></term>
##       <term><![CDATA[accelerometers]]></term>
##       <term><![CDATA[amplifiers]]></term>
##       <term><![CDATA[calibration]]></term>
##       <term><![CDATA[elemental semiconductors]]></term>
##       <term><![CDATA[gold]]></term>
##       <term><![CDATA[mirrors]]></term>
##       <term><![CDATA[optical fibres]]></term>
##       <term><![CDATA[photothermal effects]]></term>
##       <term><![CDATA[silicon]]></term>
##       <term><![CDATA[thermal analysis]]></term>
##       <term><![CDATA[thermal expansion]]></term>
##     </controlledterms>
##     <thesaurusterms>
##       <term><![CDATA[Accelerometers]]></term>
##       <term><![CDATA[Cavity resonators]]></term>
##       <term><![CDATA[Laser beams]]></term>
##       <term><![CDATA[Mirrors]]></term>
##       <term><![CDATA[Optical fibers]]></term>
##       <term><![CDATA[Silicon]]></term>
##     </thesaurusterms>
##     <pubtitle><![CDATA[Photonics Journal, IEEE]]></pubtitle>
##     <punumber><![CDATA[4563994]]></punumber>
##     <pubtype><![CDATA[Journals & Magazines]]></pubtype>
##     <publisher><![CDATA[IEEE]]></publisher>
##     <volume><![CDATA[6]]></volume>
##     <issue><![CDATA[3]]></issue>
##     <py><![CDATA[2014]]></py>
##     <spage><![CDATA[1]]></spage>
##     <epage><![CDATA[15]]></epage>
##     <abstract><![CDATA[A mechanically amplified Fabry-Pe&#x0301;rot optical accelerometer is reported in which photothermal actuation is used to calibrate the in-phase and quadrature (I&amp;Q) readout. The Fabry-Pe&#x0301;rot interferometer (FPI) is formed between a gold-coated silicon mirror, situated in the middle of a V-beam amplifier, and the end surface of a cleaved optical fiber. On the opposite side of the silicon mirror, a further cleaved optical fiber transmits near-infrared laser light (&#x03BB; = 785 nm), which is absorbed by the uncoated silicon causing heating. The thermal expansion of the V-beam is translated into an amplified change in cavity length of the FPI, large enough for the 2&#x03C0;-phase variation necessary for I&amp;Q calibration. A simple 1D thermal analysis of the structure has been developed to predict the relationship between laser power and change in cavity length. A device having a V-beam of length 1.8 mm, width 20 &#x03BC;m, and angle 2 &#x00B0; was found to undergo a cavity length change of 785 nm at 30 mW input power. The device response was approximately linear for input accelerations from 0.01 to 15 g. The noise was measured to be ~ 60 &#x03BC;g/&#x221A;Hz from 100 Hz to 3.0 kHz, whereas the limit of detection was 47.7 mg from dc to 3.0 kHz.]]></abstract>
##     <issn><![CDATA[1943-0655]]></issn>
##     <htmlFlag><![CDATA[1]]></htmlFlag>
##     <arnumber><![CDATA[6823683]]></arnumber>
##     <doi><![CDATA[10.1109/JPHOT.2014.2326652]]></doi>
##     <publicationId><![CDATA[6823683]]></publicationId>
##     <mdurl><![CDATA[http://ieeexplore.ieee.org/xpl/articleDetails.jsp?tp=&arnumber=6823683&contentType=Journals+%26+Magazines]]></mdurl>
##     <pdf><![CDATA[http://ieeexplore.ieee.org/stamp/stamp.jsp?arnumber=6823683]]></pdf>
##   </document>
##   <document>
##     <rank>412</rank>
##     <title><![CDATA[Energy-Efficient Future High-Definition TV]]></title>
##     <authors><![CDATA[Osman, N.I.;  El-Gorashi, T.;  Krug, L.;  Elmirghani, J.M.H.]]></authors>
##     <affiliations><![CDATA[Sch. of Electron. & Electr. Eng., Univ. of Leeds, Leeds, UK]]></affiliations>
##     <controlledterms>
##       <term><![CDATA[IPTV]]></term>
##       <term><![CDATA[cache storage]]></term>
##       <term><![CDATA[energy conservation]]></term>
##       <term><![CDATA[high definition television]]></term>
##     </controlledterms>
##     <thesaurusterms>
##       <term><![CDATA[HDTV]]></term>
##       <term><![CDATA[IP networks]]></term>
##       <term><![CDATA[IPTV]]></term>
##       <term><![CDATA[Power demand]]></term>
##       <term><![CDATA[Streaming media]]></term>
##       <term><![CDATA[Wavelength division multiplexing]]></term>
##     </thesaurusterms>
##     <pubtitle><![CDATA[Lightwave Technology, Journal of]]></pubtitle>
##     <punumber><![CDATA[50]]></punumber>
##     <pubtype><![CDATA[Journals & Magazines]]></pubtype>
##     <publisher><![CDATA[IEEE]]></publisher>
##     <volume><![CDATA[32]]></volume>
##     <issue><![CDATA[13]]></issue>
##     <py><![CDATA[2014]]></py>
##     <spage><![CDATA[2364]]></spage>
##     <epage><![CDATA[2381]]></epage>
##     <abstract><![CDATA[The rapidly growing IPTV market has resulted in increased traffic volumes raising concerns over Internet energy consumption. In this paper, we explore the dynamics of TV viewing behavior and program popularity in order to devise a strategy to minimize energy usage. We evaluate the impact of our strategy by calculating the power consumption of IPTV delivered over an IP-over-WDM network, considering both standard definition and high definition TV. Caches are used to reduce energy consumption by storing the most popular programs at nodes closer to the end user. We then use our knowledge of viewing behaviors to generate a time-driven content replacement strategy to maximize cache hit ratios and minimize energy use. We develop a mixed integer linear programming (MILP) model to evaluate the power consumption of the network while performing time-driven content replacements on caches and validate the results by simulation. Finally, we extend our model to perform content replacements on caches with sleep-mode capabilities which can save power by reducing their size. Our results show that time-based content replacements with such variable caches increase cache hit ratios and so reduces the overall power consumption by up to 86% compared to no caching. Our findings also show that more power savings are achieved for high definition TV compared to standard definition TV, so this strategy will be beneficial in the long term.]]></abstract>
##     <issn><![CDATA[0733-8724]]></issn>
##     <htmlFlag><![CDATA[1]]></htmlFlag>
##     <arnumber><![CDATA[6817544]]></arnumber>
##     <doi><![CDATA[10.1109/JLT.2014.2324634]]></doi>
##     <publicationId><![CDATA[6817544]]></publicationId>
##     <mdurl><![CDATA[http://ieeexplore.ieee.org/xpl/articleDetails.jsp?tp=&arnumber=6817544&contentType=Journals+%26+Magazines]]></mdurl>
##     <pdf><![CDATA[http://ieeexplore.ieee.org/stamp/stamp.jsp?arnumber=6817544]]></pdf>
##   </document>
##   <document>
##     <rank>413</rank>
##     <title><![CDATA[Optimization of Timing Jitter Reduction by Optical Feedback for a Passively Mode-Locked Laser]]></title>
##     <authors><![CDATA[Otto, C.;  Jaurigue, L.C.;  Scholl, E.;  Ludge, K.]]></authors>
##     <affiliations><![CDATA[Inst. of Theor. Phys., Tech. Univ. of Berlin, Berlin, Germany]]></affiliations>
##     <controlledterms>
##       <term><![CDATA[laser cavity resonators]]></term>
##       <term><![CDATA[laser feedback]]></term>
##       <term><![CDATA[laser mode locking]]></term>
##       <term><![CDATA[optimisation]]></term>
##       <term><![CDATA[semiconductor lasers]]></term>
##       <term><![CDATA[timing jitter]]></term>
##     </controlledterms>
##     <thesaurusterms>
##       <term><![CDATA[Cavity resonators]]></term>
##       <term><![CDATA[Delays]]></term>
##       <term><![CDATA[Laser feedback]]></term>
##       <term><![CDATA[Laser mode locking]]></term>
##       <term><![CDATA[Optical feedback]]></term>
##       <term><![CDATA[Timing jitter]]></term>
##     </thesaurusterms>
##     <pubtitle><![CDATA[Photonics Journal, IEEE]]></pubtitle>
##     <punumber><![CDATA[4563994]]></punumber>
##     <pubtype><![CDATA[Journals & Magazines]]></pubtype>
##     <publisher><![CDATA[IEEE]]></publisher>
##     <volume><![CDATA[6]]></volume>
##     <issue><![CDATA[5]]></issue>
##     <py><![CDATA[2014]]></py>
##     <spage><![CDATA[1]]></spage>
##     <epage><![CDATA[14]]></epage>
##     <abstract><![CDATA[We investigate the effect of optical self-feedback on the timing jitter of a passively mode-locked (ML) laser and compare the von Linde method with two pure time domain methods for calculating the timing jitter of such a system. We find that all three methods yield the same dependence of the timing jitter on the external cavity roundtrip time (delay time). Of these methods, the so-called long term jitter method is significantly less computationally expensive. We therefore use this method to investigate the influence of the delay time, feedback strength, and amplitude-phase coupling on the timing jitter. Our results show that, with vanishing amplitude-phase coupling, greater timing jitter reduction can be achieved with long delay times and larger feedback strengths, when feedback is near resonance. Off resonance, the timing jitter is increased. Non-vanishing amplitude-phase coupling can lead to destabilized pulse stream, even at the feedback main resonances.]]></abstract>
##     <issn><![CDATA[1943-0655]]></issn>
##     <htmlFlag><![CDATA[1]]></htmlFlag>
##     <arnumber><![CDATA[6884771]]></arnumber>
##     <doi><![CDATA[10.1109/JPHOT.2014.2352934]]></doi>
##     <publicationId><![CDATA[6884771]]></publicationId>
##     <mdurl><![CDATA[http://ieeexplore.ieee.org/xpl/articleDetails.jsp?tp=&arnumber=6884771&contentType=Journals+%26+Magazines]]></mdurl>
##     <pdf><![CDATA[http://ieeexplore.ieee.org/stamp/stamp.jsp?arnumber=6884771]]></pdf>
##   </document>
##   <document>
##     <rank>414</rank>
##     <title><![CDATA[Quantum-Assisted Routing Optimization for Self-Organizing Networks]]></title>
##     <authors><![CDATA[Alanis, D.;  Botsinis, P.;  Soon Xin Ng;  Hanzo, L.]]></authors>
##     <affiliations><![CDATA[Sch. of Electron. & Comput. Sci., Univ. of Southampton, Southampton, UK]]></affiliations>
##     <controlledterms>
##       <term><![CDATA[Pareto optimisation]]></term>
##       <term><![CDATA[quality of service]]></term>
##       <term><![CDATA[quantum computing]]></term>
##       <term><![CDATA[telecommunication network routing]]></term>
##       <term><![CDATA[wireless sensor networks]]></term>
##     </controlledterms>
##     <thesaurusterms>
##       <term><![CDATA[Biological system modeling]]></term>
##       <term><![CDATA[Complexity theory]]></term>
##       <term><![CDATA[Optimization]]></term>
##       <term><![CDATA[Quality of service]]></term>
##       <term><![CDATA[Quantum computing]]></term>
##       <term><![CDATA[Routing]]></term>
##       <term><![CDATA[Self-organizing networks]]></term>
##       <term><![CDATA[Wireless sensor networks]]></term>
##     </thesaurusterms>
##     <pubtitle><![CDATA[Access, IEEE]]></pubtitle>
##     <punumber><![CDATA[6287639]]></punumber>
##     <pubtype><![CDATA[Journals & Magazines]]></pubtype>
##     <publisher><![CDATA[IEEE]]></publisher>
##     <volume><![CDATA[2]]></volume>
##     <py><![CDATA[2014]]></py>
##     <spage><![CDATA[614]]></spage>
##     <epage><![CDATA[632]]></epage>
##     <abstract><![CDATA[Self-organizing networks act autonomously for the sake of achieving the best possible performance. The attainable routing depends on a delicate balance of diverse and often conflicting quality-of-service requirements. Finding the optimal solution typically becomes an nonolynomial-hard problem, as the network size increases in terms of the number of nodes. Moreover, the employment of user-defined utility functions for the aggregation of the different objective functions often leads to suboptimal solutions. On the other hand, Pareto optimality is capable of amalgamating the different design objectives by providing an element of elitism. Although there is a plethora of bioinspired algorithms that attempt to address this optimization problem, they often fail to generate all the points constituting the optimal Pareto front. As a remedy, we propose an optimal multiobjective quantum-assisted algorithm, namely the nondominated quantum optimization algorithm (NDQO), which evaluates the legitimate routes using the concept of Pareto optimality at a reduced complexity. We then compare the performance of the NDQO algorithm to the state-of-the-art evolutionary algorithms, demonstrating that the NDQO algorithm achieves a near-optimal performance. Furthermore, we analytically derive the upper and lower bounds of the NDQO algorithmic complexity, which is of the order of O(N) and O(N&#x221A;(N)) in the best and worst case scenario, respectively. This corresponds to a substantial complexity reduction of the NDQO from the order of O(N<sup>2</sup>) imposed by the brute-force method.]]></abstract>
##     <issn><![CDATA[2169-3536]]></issn>
##     <htmlFlag><![CDATA[1]]></htmlFlag>
##     <arnumber><![CDATA[6825798]]></arnumber>
##     <doi><![CDATA[10.1109/ACCESS.2014.2327596]]></doi>
##     <publicationId><![CDATA[6825798]]></publicationId>
##     <mdurl><![CDATA[http://ieeexplore.ieee.org/xpl/articleDetails.jsp?tp=&arnumber=6825798&contentType=Journals+%26+Magazines]]></mdurl>
##     <pdf><![CDATA[http://ieeexplore.ieee.org/stamp/stamp.jsp?arnumber=6825798]]></pdf>
##   </document>
##   <document>
##     <rank>415</rank>
##     <title><![CDATA[Study of the Further Reduction of Shaft Voltage of Brushless DC Motor With Insulated Rotor Driven by PWM Inverter]]></title>
##     <authors><![CDATA[Isomura, Y.;  Yamamoto, K.;  Morimoto, S.;  Maetani, T.;  Watanabe, A.;  Nakano, K.]]></authors>
##     <affiliations><![CDATA[Motor Bus. Div., Panasonic Corp., Daito, Japan]]></affiliations>
##     <controlledterms>
##       <term><![CDATA[DC motor drives]]></term>
##       <term><![CDATA[PWM invertors]]></term>
##       <term><![CDATA[brushless DC motors]]></term>
##       <term><![CDATA[equivalent circuits]]></term>
##       <term><![CDATA[machine insulation]]></term>
##       <term><![CDATA[resins]]></term>
##       <term><![CDATA[rotors]]></term>
##       <term><![CDATA[shafts]]></term>
##       <term><![CDATA[voltage measurement]]></term>
##     </controlledterms>
##     <thesaurusterms>
##       <term><![CDATA[Brushless DC motors]]></term>
##       <term><![CDATA[Capacitance]]></term>
##       <term><![CDATA[Magnetic cores]]></term>
##       <term><![CDATA[Rotors]]></term>
##       <term><![CDATA[Shafts]]></term>
##       <term><![CDATA[Stator cores]]></term>
##     </thesaurusterms>
##     <pubtitle><![CDATA[Industry Applications, IEEE Transactions on]]></pubtitle>
##     <punumber><![CDATA[28]]></punumber>
##     <pubtype><![CDATA[Journals & Magazines]]></pubtype>
##     <publisher><![CDATA[IEEE]]></publisher>
##     <volume><![CDATA[50]]></volume>
##     <issue><![CDATA[6]]></issue>
##     <py><![CDATA[2014]]></py>
##     <spage><![CDATA[3738]]></spage>
##     <epage><![CDATA[3743]]></epage>
##     <abstract><![CDATA[The authors previously succeeded in reducing the shaft voltage of a PWM driven motor with a rotor which had an outer core and an inner core (and the shaft), electrically insulated each other by a resin (hereafter, the insulated rotor). This paper proposes a new method for further reduction of the shaft voltage of a motor with an insulated rotor by adding a capacitor between brackets and N line of the dc link of the inverter. A common-mode equivalent circuit of the system with an ungrounded motor is examined, and the effect of further reduction by the new method is verified by calculation of the shaft voltage from the equivalent circuit and measurement of the shaft voltage of the motor.]]></abstract>
##     <issn><![CDATA[0093-9994]]></issn>
##     <htmlFlag><![CDATA[1]]></htmlFlag>
##     <arnumber><![CDATA[6766738]]></arnumber>
##     <doi><![CDATA[10.1109/TIA.2014.2311591]]></doi>
##     <publicationId><![CDATA[6766738]]></publicationId>
##     <mdurl><![CDATA[http://ieeexplore.ieee.org/xpl/articleDetails.jsp?tp=&arnumber=6766738&contentType=Journals+%26+Magazines]]></mdurl>
##     <pdf><![CDATA[http://ieeexplore.ieee.org/stamp/stamp.jsp?arnumber=6766738]]></pdf>
##   </document>
##   <document>
##     <rank>416</rank>
##     <title><![CDATA[Semi-Supervised Multiresolution Classification Using Adaptive Graph Filtering With Application to Indirect Bridge Structural Health Monitoring]]></title>
##     <authors><![CDATA[Siheng Chen;  Cerda, F.;  Rizzo, P.;  Bielak, J.;  Garrett, J.H.;  Kovacevic, J.]]></authors>
##     <affiliations><![CDATA[Dept. of Electr. & Comput. Eng., Carnegie Mellon Univ., Pittsburgh, PA, USA]]></affiliations>
##     <controlledterms>
##       <term><![CDATA[adaptive filters]]></term>
##       <term><![CDATA[bridges (structures)]]></term>
##       <term><![CDATA[condition monitoring]]></term>
##       <term><![CDATA[feature extraction]]></term>
##       <term><![CDATA[graph theory]]></term>
##       <term><![CDATA[learning (artificial intelligence)]]></term>
##       <term><![CDATA[signal classification]]></term>
##       <term><![CDATA[signal resolution]]></term>
##       <term><![CDATA[structural engineering computing]]></term>
##       <term><![CDATA[time-frequency analysis]]></term>
##     </controlledterms>
##     <thesaurusterms>
##       <term><![CDATA[Bridges]]></term>
##       <term><![CDATA[Feature extraction]]></term>
##       <term><![CDATA[Image resolution]]></term>
##       <term><![CDATA[Monitoring]]></term>
##       <term><![CDATA[Semisupervised learning]]></term>
##       <term><![CDATA[Signal processing algorithms]]></term>
##       <term><![CDATA[Signal resolution]]></term>
##     </thesaurusterms>
##     <pubtitle><![CDATA[Signal Processing, IEEE Transactions on]]></pubtitle>
##     <punumber><![CDATA[78]]></punumber>
##     <pubtype><![CDATA[Journals & Magazines]]></pubtype>
##     <publisher><![CDATA[IEEE]]></publisher>
##     <volume><![CDATA[62]]></volume>
##     <issue><![CDATA[11]]></issue>
##     <py><![CDATA[2014]]></py>
##     <spage><![CDATA[2879]]></spage>
##     <epage><![CDATA[2893]]></epage>
##     <abstract><![CDATA[We present a multiresolution classification framework with semi-supervised learning on graphs with application to the indirect bridge structural health monitoring. Classification in real-world applications faces two main challenges: reliable features can be hard to extract and few labeled signals are available for training. We propose a novel classification framework to address these problems: we use a multiresolution framework to deal with nonstationarities in the signals and extract features in each localized time-frequency region and semi-supervised learning to train on both labeled and unlabeled signals. We further propose an adaptive graph filter for semi-supervised classification that allows for classifying unlabeled as well as unseen signals and for correcting mislabeled signals. We validate the proposed framework on indirect bridge structural health monitoring and show that it performs significantly better than previous approaches.]]></abstract>
##     <issn><![CDATA[1053-587X]]></issn>
##     <htmlFlag><![CDATA[1]]></htmlFlag>
##     <arnumber><![CDATA[6778068]]></arnumber>
##     <doi><![CDATA[10.1109/TSP.2014.2313528]]></doi>
##     <publicationId><![CDATA[6778068]]></publicationId>
##     <mdurl><![CDATA[http://ieeexplore.ieee.org/xpl/articleDetails.jsp?tp=&arnumber=6778068&contentType=Journals+%26+Magazines]]></mdurl>
##     <pdf><![CDATA[http://ieeexplore.ieee.org/stamp/stamp.jsp?arnumber=6778068]]></pdf>
##   </document>
##   <document>
##     <rank>417</rank>
##     <title><![CDATA[Discussion of the Epsilon-Near-Zero Effect of Graphene in a Horizontal Slot Waveguide]]></title>
##     <authors><![CDATA[Min-Suk Kwon]]></authors>
##     <affiliations><![CDATA[Sch. of Electr. & Comput. Eng., Ulsan Nat. Inst. of Sci. & Technol. (UNIST), Ulsan, South Korea]]></affiliations>
##     <controlledterms>
##       <term><![CDATA[graphene]]></term>
##       <term><![CDATA[nanophotonics]]></term>
##       <term><![CDATA[optical materials]]></term>
##       <term><![CDATA[optical modulation]]></term>
##       <term><![CDATA[optical waveguides]]></term>
##       <term><![CDATA[permittivity]]></term>
##     </controlledterms>
##     <thesaurusterms>
##       <term><![CDATA[Dielectrics]]></term>
##       <term><![CDATA[Electric fields]]></term>
##       <term><![CDATA[Graphene]]></term>
##       <term><![CDATA[Mathematical model]]></term>
##       <term><![CDATA[Modulation]]></term>
##       <term><![CDATA[Optical attenuators]]></term>
##       <term><![CDATA[Optical waveguides]]></term>
##     </thesaurusterms>
##     <pubtitle><![CDATA[Photonics Journal, IEEE]]></pubtitle>
##     <punumber><![CDATA[4563994]]></punumber>
##     <pubtype><![CDATA[Journals & Magazines]]></pubtype>
##     <publisher><![CDATA[IEEE]]></publisher>
##     <volume><![CDATA[6]]></volume>
##     <issue><![CDATA[3]]></issue>
##     <py><![CDATA[2014]]></py>
##     <spage><![CDATA[1]]></spage>
##     <epage><![CDATA[9]]></epage>
##     <abstract><![CDATA[Horizontal slot waveguides based on graphene have been considered an attractive structure for optical waveguide modulators for transverse magnetic (TM) modes. Graphene is embedded in the slot region of a horizontal slot waveguide. If graphene were treated as an isotropic material and its dielectric constant were made close to zero by adjusting its Fermi level, the surface-normal electric field component of the fundamental TM mode of a horizontal slot waveguide might be highly enhanced in graphene. This could cause a large increase in the attenuation coefficient of the mode. This is called the epsilon-near-zero (ENZ) effect. This paper discusses that graphene needs to be treated as an anisotropic material that has an almost real surface-normal dielectric constant component. Then, the ENZ effect does not exist. Approximate analytic expressions and numerical simulation are used for the discussion, and they demonstrate that horizontal slot waveguides are not appropriate for graphene-based modulators for TM modes.]]></abstract>
##     <issn><![CDATA[1943-0655]]></issn>
##     <htmlFlag><![CDATA[1]]></htmlFlag>
##     <arnumber><![CDATA[6820729]]></arnumber>
##     <doi><![CDATA[10.1109/JPHOT.2014.2326667]]></doi>
##     <publicationId><![CDATA[6820729]]></publicationId>
##     <mdurl><![CDATA[http://ieeexplore.ieee.org/xpl/articleDetails.jsp?tp=&arnumber=6820729&contentType=Journals+%26+Magazines]]></mdurl>
##     <pdf><![CDATA[http://ieeexplore.ieee.org/stamp/stamp.jsp?arnumber=6820729]]></pdf>
##   </document>
##   <document>
##     <rank>418</rank>
##     <title><![CDATA[Corruptive Artifacts Suppression for Example-Based Color Transfer]]></title>
##     <authors><![CDATA[Zhuo Su;  Kun Zeng;  Li Liu;  Bo Li;  Xiaonan Luo]]></authors>
##     <affiliations><![CDATA[Nat. Eng. Res. Center of Digital Life, Sun Yat-sen Univ., Shenzhen, China]]></affiliations>
##     <controlledterms>
##       <term><![CDATA[filtering theory]]></term>
##       <term><![CDATA[image colour analysis]]></term>
##       <term><![CDATA[iterative methods]]></term>
##       <term><![CDATA[probability]]></term>
##     </controlledterms>
##     <thesaurusterms>
##       <term><![CDATA[Educational institutions]]></term>
##       <term><![CDATA[Histograms]]></term>
##       <term><![CDATA[Image color analysis]]></term>
##       <term><![CDATA[Probabilistic logic]]></term>
##       <term><![CDATA[Probability density function]]></term>
##       <term><![CDATA[Probability distribution]]></term>
##       <term><![CDATA[Sun]]></term>
##     </thesaurusterms>
##     <pubtitle><![CDATA[Multimedia, IEEE Transactions on]]></pubtitle>
##     <punumber><![CDATA[6046]]></punumber>
##     <pubtype><![CDATA[Journals & Magazines]]></pubtype>
##     <publisher><![CDATA[IEEE]]></publisher>
##     <volume><![CDATA[16]]></volume>
##     <issue><![CDATA[4]]></issue>
##     <py><![CDATA[2014]]></py>
##     <spage><![CDATA[988]]></spage>
##     <epage><![CDATA[999]]></epage>
##     <abstract><![CDATA[Example-based color transfer is a critical operation in image editing but easily suffers from some corruptive artifacts in the mapping process. In this paper, we propose a novel unified color transfer framework with corruptive artifacts suppression, which performs iterative probabilistic color mapping with self-learning filtering scheme and multiscale detail manipulation scheme in minimizing the normalized Kullback-Leibler distance. First, an iterative probabilistic color mapping is applied to construct the mapping relationship between the reference and target images. Then, a self-learning filtering scheme is applied into the transfer process to prevent from artifacts and extract details. The transferred output and the extracted multi-levels details are integrated by the measurement minimization to yield the final result. Our framework achieves a sound grain suppression, color fidelity and detail appearance seamlessly. For demonstration, a series of objective and subjective measurements are used to evaluate the quality in color transfer. Finally, a few extended applications are implemented to show the applicability of this framework.]]></abstract>
##     <issn><![CDATA[1520-9210]]></issn>
##     <htmlFlag><![CDATA[1]]></htmlFlag>
##     <arnumber><![CDATA[6739126]]></arnumber>
##     <doi><![CDATA[10.1109/TMM.2014.2305914]]></doi>
##     <publicationId><![CDATA[6739126]]></publicationId>
##     <mdurl><![CDATA[http://ieeexplore.ieee.org/xpl/articleDetails.jsp?tp=&arnumber=6739126&contentType=Journals+%26+Magazines]]></mdurl>
##     <pdf><![CDATA[http://ieeexplore.ieee.org/stamp/stamp.jsp?arnumber=6739126]]></pdf>
##   </document>
##   <document>
##     <rank>419</rank>
##     <title><![CDATA[UpSet: Visualization of Intersecting Sets]]></title>
##     <authors><![CDATA[Lex, A.;  Gehlenborg, N.;  Strobelt, H.;  Vuillemot, R.;  Pfister, H.]]></authors>
##     <affiliations><![CDATA[Hendrik Strobelt & Hanspeter Pfister, Harvard Univ., Cambridge, MA, USA]]></affiliations>
##     <controlledterms>
##       <term><![CDATA[Internet]]></term>
##       <term><![CDATA[combinatorial mathematics]]></term>
##       <term><![CDATA[data visualisation]]></term>
##       <term><![CDATA[duality (mathematics)]]></term>
##       <term><![CDATA[mathematics computing]]></term>
##       <term><![CDATA[matrix algebra]]></term>
##       <term><![CDATA[public domain software]]></term>
##       <term><![CDATA[set theory]]></term>
##       <term><![CDATA[sorting]]></term>
##     </controlledterms>
##     <thesaurusterms>
##       <term><![CDATA[Data visualization]]></term>
##       <term><![CDATA[Information analysis]]></term>
##       <term><![CDATA[Power generation]]></term>
##       <term><![CDATA[Sorting]]></term>
##       <term><![CDATA[Visualization]]></term>
##     </thesaurusterms>
##     <pubtitle><![CDATA[Visualization and Computer Graphics, IEEE Transactions on]]></pubtitle>
##     <punumber><![CDATA[2945]]></punumber>
##     <pubtype><![CDATA[Journals & Magazines]]></pubtype>
##     <publisher><![CDATA[IEEE]]></publisher>
##     <volume><![CDATA[20]]></volume>
##     <issue><![CDATA[12]]></issue>
##     <py><![CDATA[2014]]></py>
##     <spage><![CDATA[1983]]></spage>
##     <epage><![CDATA[1992]]></epage>
##     <abstract><![CDATA[Understanding relationships between sets is an important analysis task that has received widespread attention in the visualization community. The major challenge in this context is the combinatorial explosion of the number of set intersections if the number of sets exceeds a trivial threshold. In this paper we introduce UpSet, a novel visualization technique for the quantitative analysis of sets, their intersections, and aggregates of intersections. UpSet is focused on creating task-driven aggregates, communicating the size and properties of aggregates and intersections, and a duality between the visualization of the elements in a dataset and their set membership. UpSet visualizes set intersections in a matrix layout and introduces aggregates based on groupings and queries. The matrix layout enables the effective representation of associated data, such as the number of elements in the aggregates and intersections, as well as additional summary statistics derived from subset or element attributes. Sorting according to various measures enables a task-driven analysis of relevant intersections and aggregates. The elements represented in the sets and their associated attributes are visualized in a separate view. Queries based on containment in specific intersections, aggregates or driven by attribute filters are propagated between both views. We also introduce several advanced visual encodings and interaction methods to overcome the problems of varying scales and to address scalability. UpSet is web-based and open source. We demonstrate its general utility in multiple use cases from various domains.]]></abstract>
##     <issn><![CDATA[1077-2626]]></issn>
##     <htmlFlag><![CDATA[1]]></htmlFlag>
##     <arnumber><![CDATA[6876017]]></arnumber>
##     <doi><![CDATA[10.1109/TVCG.2014.2346248]]></doi>
##     <publicationId><![CDATA[6876017]]></publicationId>
##     <mdurl><![CDATA[http://ieeexplore.ieee.org/xpl/articleDetails.jsp?tp=&arnumber=6876017&contentType=Journals+%26+Magazines]]></mdurl>
##     <pdf><![CDATA[http://ieeexplore.ieee.org/stamp/stamp.jsp?arnumber=6876017]]></pdf>
##   </document>
##   <document>
##     <rank>420</rank>
##     <title><![CDATA[Data Requisites for Transformer Statistical Lifetime Modelling&#x2014;Part II: Combination of Random and Aging-Related Failures]]></title>
##     <authors><![CDATA[Dan Zhou;  Zhongdong Wang;  Jarman, P.;  Chengrong Li]]></authors>
##     <affiliations><![CDATA[North China Electr. Power Univ., Beijing, China]]></affiliations>
##     <controlledterms>
##       <term><![CDATA[Monte Carlo methods]]></term>
##       <term><![CDATA[ageing]]></term>
##       <term><![CDATA[failure analysis]]></term>
##       <term><![CDATA[power transformers]]></term>
##       <term><![CDATA[remaining life assessment]]></term>
##     </controlledterms>
##     <thesaurusterms>
##       <term><![CDATA[Accuracy]]></term>
##       <term><![CDATA[Aging]]></term>
##       <term><![CDATA[Analytical models]]></term>
##       <term><![CDATA[Data models]]></term>
##       <term><![CDATA[Power transformers]]></term>
##       <term><![CDATA[Shape]]></term>
##       <term><![CDATA[Stress]]></term>
##     </thesaurusterms>
##     <pubtitle><![CDATA[Power Delivery, IEEE Transactions on]]></pubtitle>
##     <punumber><![CDATA[61]]></punumber>
##     <pubtype><![CDATA[Journals & Magazines]]></pubtype>
##     <publisher><![CDATA[IEEE]]></publisher>
##     <volume><![CDATA[29]]></volume>
##     <issue><![CDATA[1]]></issue>
##     <py><![CDATA[2014]]></py>
##     <spage><![CDATA[154]]></spage>
##     <epage><![CDATA[160]]></epage>
##     <abstract><![CDATA[Statistical lifetime modeling is of importance for replacement management of aged power transformers. Survival data are recognized as important as failure data in improving the accuracy level of the lifetime models since transformer failures are rare events and most of the units are still in operating condition. This paper argues that differentiating random failures and aging-related failures is also important. Different data requisites for modeling random failures and aging-related failures are analyzed and compared through Monte Carlo simulations. The transformer life-cycle failure model can be built by combining the random and aging-related failure models. A case study is presented to show that through postmortem analysis, the two failure modes can be distinguished and, hence, it helps to improve the accuracy of the combined model.]]></abstract>
##     <issn><![CDATA[0885-8977]]></issn>
##     <htmlFlag><![CDATA[1]]></htmlFlag>
##     <arnumber><![CDATA[6555940]]></arnumber>
##     <doi><![CDATA[10.1109/TPWRD.2013.2270116]]></doi>
##     <publicationId><![CDATA[6555940]]></publicationId>
##     <mdurl><![CDATA[http://ieeexplore.ieee.org/xpl/articleDetails.jsp?tp=&arnumber=6555940&contentType=Journals+%26+Magazines]]></mdurl>
##     <pdf><![CDATA[http://ieeexplore.ieee.org/stamp/stamp.jsp?arnumber=6555940]]></pdf>
##   </document>
##   <document>
##     <rank>421</rank>
##     <title><![CDATA[Geometric Algebra for Electrical and Electronic Engineers]]></title>
##     <authors><![CDATA[Chappell, J.M.;  Drake, S.P.;  Seidel, C.L.;  Gunn, L.J.;  Iqbal, A.;  Allison, A.;  Abbott, D.]]></authors>
##     <thesaurusterms>
##       <term><![CDATA[Algebra]]></term>
##       <term><![CDATA[Electric fields]]></term>
##       <term><![CDATA[Geometry]]></term>
##       <term><![CDATA[Magnetic separation]]></term>
##       <term><![CDATA[Mathematics]]></term>
##       <term><![CDATA[Quaternions]]></term>
##     </thesaurusterms>
##     <pubtitle><![CDATA[Proceedings of the IEEE]]></pubtitle>
##     <punumber><![CDATA[5]]></punumber>
##     <pubtype><![CDATA[Journals & Magazines]]></pubtype>
##     <publisher><![CDATA[IEEE]]></publisher>
##     <volume><![CDATA[102]]></volume>
##     <issue><![CDATA[9]]></issue>
##     <py><![CDATA[2014]]></py>
##     <spage><![CDATA[1340]]></spage>
##     <epage><![CDATA[1363]]></epage>
##     <abstract><![CDATA[In this paper, we explicate the suggested benefits of Clifford's geometric algebra (GA) when applied to the field of electrical engineering. Engineers are always interested in keeping formulas as simple or compact as possible, and we illustrate that geometric algebra does provide such a simplified representation in many cases. We also demonstrate an additional structural check provided by GA for formulas in addition to the usual checking of physical dimensions. Naturally, there is an initial learning curve when applying a new method, but it appears to be worth the effort, as we show significantly simplified formulas, greater intuition, and improved problem solving in many cases.]]></abstract>
##     <issn><![CDATA[0018-9219]]></issn>
##     <htmlFlag><![CDATA[1]]></htmlFlag>
##     <arnumber><![CDATA[6876131]]></arnumber>
##     <doi><![CDATA[10.1109/JPROC.2014.2339299]]></doi>
##     <publicationId><![CDATA[6876131]]></publicationId>
##     <mdurl><![CDATA[http://ieeexplore.ieee.org/xpl/articleDetails.jsp?tp=&arnumber=6876131&contentType=Journals+%26+Magazines]]></mdurl>
##     <pdf><![CDATA[http://ieeexplore.ieee.org/stamp/stamp.jsp?arnumber=6876131]]></pdf>
##   </document>
##   <document>
##     <rank>422</rank>
##     <title><![CDATA[Compact Modeling of Flicker Noise in HEMTs]]></title>
##     <authors><![CDATA[Dasgupta, A.;  Khandelwal, S.;  Chauhan, Y.S.]]></authors>
##     <affiliations><![CDATA[Dept. of Electr. Eng., Indian Inst. of Technol. Kanpur, Kanpur, India]]></affiliations>
##     <controlledterms>
##       <term><![CDATA[carrier mobility]]></term>
##       <term><![CDATA[flicker noise]]></term>
##       <term><![CDATA[high electron mobility transistors]]></term>
##       <term><![CDATA[semiconductor device models]]></term>
##       <term><![CDATA[semiconductor device noise]]></term>
##     </controlledterms>
##     <thesaurusterms>
##       <term><![CDATA[1f noise]]></term>
##       <term><![CDATA[Gallium nitride]]></term>
##       <term><![CDATA[HEMTs]]></term>
##       <term><![CDATA[Integrated circuit modeling]]></term>
##       <term><![CDATA[Noise]]></term>
##     </thesaurusterms>
##     <pubtitle><![CDATA[Electron Devices Society, IEEE Journal of the]]></pubtitle>
##     <punumber><![CDATA[6245494]]></punumber>
##     <pubtype><![CDATA[Journals & Magazines]]></pubtype>
##     <publisher><![CDATA[IEEE]]></publisher>
##     <volume><![CDATA[2]]></volume>
##     <issue><![CDATA[6]]></issue>
##     <py><![CDATA[2014]]></py>
##     <spage><![CDATA[174]]></spage>
##     <epage><![CDATA[178]]></epage>
##     <abstract><![CDATA[In this paper, we present a physics-based compact model for low frequency noise in high electron mobility transistors (HEMTs). The model is derived considering the physical mechanisms of carrier number fluctuation and mobility fluctuation in the channel. The model is tunable and hence applicable to a wide range of HEMT devices of different geometries and construction. The model is in excellent agreement with experimental data and TCAD simulations.]]></abstract>
##     <issn><![CDATA[2168-6734]]></issn>
##     <htmlFlag><![CDATA[1]]></htmlFlag>
##     <arnumber><![CDATA[6878427]]></arnumber>
##     <doi><![CDATA[10.1109/JEDS.2014.2347991]]></doi>
##     <publicationId><![CDATA[6878427]]></publicationId>
##     <mdurl><![CDATA[http://ieeexplore.ieee.org/xpl/articleDetails.jsp?tp=&arnumber=6878427&contentType=Journals+%26+Magazines]]></mdurl>
##     <pdf><![CDATA[http://ieeexplore.ieee.org/stamp/stamp.jsp?arnumber=6878427]]></pdf>
##   </document>
##   <document>
##     <rank>423</rank>
##     <title><![CDATA[Optical Crosspoint Matrix Using Broadband Resonant Switches]]></title>
##     <authors><![CDATA[DasMahapatra, P.;  Stabile, R.;  Rohit, A.;  Williams, K.A.]]></authors>
##     <affiliations><![CDATA[Dept. of Electr. Eng., Eindhoven Univ. of Technol., Eindhoven, Netherlands]]></affiliations>
##     <controlledterms>
##       <term><![CDATA[integrated optoelectronics]]></term>
##       <term><![CDATA[optical arrays]]></term>
##       <term><![CDATA[optical design techniques]]></term>
##       <term><![CDATA[optical fabrication]]></term>
##       <term><![CDATA[optical losses]]></term>
##       <term><![CDATA[optical switches]]></term>
##       <term><![CDATA[optical transfer function]]></term>
##       <term><![CDATA[optical tuning]]></term>
##       <term><![CDATA[thermo-optical devices]]></term>
##     </controlledterms>
##     <thesaurusterms>
##       <term><![CDATA[Optical device fabrication]]></term>
##       <term><![CDATA[Optical ring resonators]]></term>
##       <term><![CDATA[Optical switches]]></term>
##       <term><![CDATA[Optical waveguides]]></term>
##       <term><![CDATA[Routing]]></term>
##       <term><![CDATA[Transmission line matrix methods]]></term>
##     </thesaurusterms>
##     <pubtitle><![CDATA[Selected Topics in Quantum Electronics, IEEE Journal of]]></pubtitle>
##     <punumber><![CDATA[2944]]></punumber>
##     <pubtype><![CDATA[Journals & Magazines]]></pubtype>
##     <publisher><![CDATA[IEEE]]></publisher>
##     <volume><![CDATA[20]]></volume>
##     <issue><![CDATA[4]]></issue>
##     <py><![CDATA[2014]]></py>
##     <spage><![CDATA[1]]></spage>
##     <epage><![CDATA[10]]></epage>
##     <abstract><![CDATA[A low loss, broadband crosspoint switch matrix using high-order resonant optical switch elements is designed, and fabricated for the first time. Multi-path routing is demonstrated for a broad range of representative paths across the circuit. Connections are assessed between eight inputs and four outputs to show losses as low as 0.9 dB per off state ring and 2.0 dB per on state ring. Analysis of the on-state and off-state transfer functions reveal switch extinction ratios exceeding 20 dB for operational bandwidths of 100 GHz for twenty-five different path combinations. Switching is implemented with thermo-optic tuning to give 100-GHz passbands and stopbands. Thermo-optic actuation with a 2-D array of on-chip microheaters allows rise and fall switching times of 17 and 4 microseconds respectively. Power penalties of less than 1.0 dB at 10 Gb/s are observed for twenty eight paths and comparable performance is observed for 40 Gb/s routing on representative paths through the switch matrix without significant signal degradation.]]></abstract>
##     <issn><![CDATA[1077-260X]]></issn>
##     <htmlFlag><![CDATA[1]]></htmlFlag>
##     <arnumber><![CDATA[6736076]]></arnumber>
##     <doi><![CDATA[10.1109/JSTQE.2013.2296746]]></doi>
##     <publicationId><![CDATA[6736076]]></publicationId>
##     <mdurl><![CDATA[http://ieeexplore.ieee.org/xpl/articleDetails.jsp?tp=&arnumber=6736076&contentType=Journals+%26+Magazines]]></mdurl>
##     <pdf><![CDATA[http://ieeexplore.ieee.org/stamp/stamp.jsp?arnumber=6736076]]></pdf>
##   </document>
##   <document>
##     <rank>424</rank>
##     <title><![CDATA[Integration of SDR and SDN for 5G]]></title>
##     <authors><![CDATA[Hsin-Hung Cho;  Chin-Feng Lai;  Shih, T.K.;  Han-Chieh Chao]]></authors>
##     <affiliations><![CDATA[Dept. of Comput. Sci. & Inf. Eng., Nat. Central Univ., Zhongli, Taiwan]]></affiliations>
##     <controlledterms>
##       <term><![CDATA[Internet]]></term>
##       <term><![CDATA[mobile radio]]></term>
##       <term><![CDATA[software radio]]></term>
##     </controlledterms>
##     <thesaurusterms>
##       <term><![CDATA[Bandwidth]]></term>
##       <term><![CDATA[Computer architecture]]></term>
##       <term><![CDATA[Control systems]]></term>
##       <term><![CDATA[Hardware]]></term>
##       <term><![CDATA[Monitoring]]></term>
##       <term><![CDATA[Protocols]]></term>
##       <term><![CDATA[Software radio]]></term>
##     </thesaurusterms>
##     <pubtitle><![CDATA[Access, IEEE]]></pubtitle>
##     <punumber><![CDATA[6287639]]></punumber>
##     <pubtype><![CDATA[Journals & Magazines]]></pubtype>
##     <publisher><![CDATA[IEEE]]></publisher>
##     <volume><![CDATA[2]]></volume>
##     <py><![CDATA[2014]]></py>
##     <spage><![CDATA[1196]]></spage>
##     <epage><![CDATA[1204]]></epage>
##     <abstract><![CDATA[Wireless networks have evolved from 1G to 4G networks, allowing smart devices to become important tools in daily life. The 5G network is a revolutionary technology that can change consumers' Internet use habits, as it creates a truly wireless environment. It is faster, with better quality, and is more secure. Most importantly, users can truly use network services anytime, anywhere. With increasing demand, the use of bandwidth and frequency spectrum resources is beyond expectations. This paper found that the frequency spectrum and network information have considerable relevance; thus, spectrum utilization and channel flow interactions should be simultaneously considered. We considered that software defined radio (SDR) and software defined networks (SDNs) are the best solution. We propose a cross-layer architecture combining SDR and SDN characteristics. As the simulation evaluation results suggest, the proposed architecture can effectively use the frequency spectrum and considerably enhance network performance. Based on the results, suggestions are proposed for follow-up studies on the proposed architecture.]]></abstract>
##     <issn><![CDATA[2169-3536]]></issn>
##     <htmlFlag><![CDATA[1]]></htmlFlag>
##     <arnumber><![CDATA[6895241]]></arnumber>
##     <doi><![CDATA[10.1109/ACCESS.2014.2357435]]></doi>
##     <publicationId><![CDATA[6895241]]></publicationId>
##     <mdurl><![CDATA[http://ieeexplore.ieee.org/xpl/articleDetails.jsp?tp=&arnumber=6895241&contentType=Journals+%26+Magazines]]></mdurl>
##     <pdf><![CDATA[http://ieeexplore.ieee.org/stamp/stamp.jsp?arnumber=6895241]]></pdf>
##   </document>
##   <document>
##     <rank>425</rank>
##     <title><![CDATA[Kernelized Bayesian Matrix Factorization]]></title>
##     <authors><![CDATA[Gonen, M.;  Kaski, S.]]></authors>
##     <affiliations><![CDATA[Sage Bionetworks, Seatle, WA, USA]]></affiliations>
##     <controlledterms>
##       <term><![CDATA[approximation theory]]></term>
##       <term><![CDATA[biology computing]]></term>
##       <term><![CDATA[learning (artificial intelligence)]]></term>
##       <term><![CDATA[matrix decomposition]]></term>
##       <term><![CDATA[pattern classification]]></term>
##       <term><![CDATA[probability]]></term>
##     </controlledterms>
##     <thesaurusterms>
##       <term><![CDATA[Approximation methods]]></term>
##       <term><![CDATA[Bayes methods]]></term>
##       <term><![CDATA[Computational modeling]]></term>
##       <term><![CDATA[Covariance matrices]]></term>
##       <term><![CDATA[Kernel]]></term>
##       <term><![CDATA[Prediction algorithms]]></term>
##       <term><![CDATA[Probabilistic logic]]></term>
##     </thesaurusterms>
##     <pubtitle><![CDATA[Pattern Analysis and Machine Intelligence, IEEE Transactions on]]></pubtitle>
##     <punumber><![CDATA[34]]></punumber>
##     <pubtype><![CDATA[Journals & Magazines]]></pubtype>
##     <publisher><![CDATA[IEEE]]></publisher>
##     <volume><![CDATA[36]]></volume>
##     <issue><![CDATA[10]]></issue>
##     <py><![CDATA[2014]]></py>
##     <spage><![CDATA[2047]]></spage>
##     <epage><![CDATA[2060]]></epage>
##     <abstract><![CDATA[We extend kernelized matrix factorization with a full-Bayesian treatment and with an ability to work with multiple side information sources expressed as different kernels. Kernels have been introduced to integrate side information about the rows and columns, which is necessary for making out-of-matrix predictions. We discuss specifically binary output matrices but extensions to realvalued matrices are straightforward. We extend the state of the art in two key aspects: (i) A full-conjugate probabilistic formulation of the kernelized matrix factorization enables an efficient variational approximation, whereas full-Bayesian treatments are not computationally feasible in the earlier approaches. (ii) Multiple side information sources are included, treated as different kernels in multiple kernel learning which additionally reveals which side sources are informative. We then show that the framework can also be used for supervised and semi-supervised multilabel classification and multi-output regression, by considering samples and outputs as the domains where matrix factorization operates. Our method outperforms alternatives in predicting drug-protein interactions on two data sets. On multilabel classification, our algorithm obtains the lowest Hamming losses on 10 out of 14 data sets compared to five state-of-the-art multilabel classification algorithms. We finally show that the proposed approach outperforms alternatives in multi-output regression experiments on a yeast cell cycle data set.]]></abstract>
##     <issn><![CDATA[0162-8828]]></issn>
##     <arnumber><![CDATA[6777351]]></arnumber>
##     <doi><![CDATA[10.1109/TPAMI.2014.2313125]]></doi>
##     <publicationId><![CDATA[6777351]]></publicationId>
##     <mdurl><![CDATA[http://ieeexplore.ieee.org/xpl/articleDetails.jsp?tp=&arnumber=6777351&contentType=Journals+%26+Magazines]]></mdurl>
##     <pdf><![CDATA[http://ieeexplore.ieee.org/stamp/stamp.jsp?arnumber=6777351]]></pdf>
##   </document>
##   <document>
##     <rank>426</rank>
##     <title><![CDATA[A Stationary-Sources and Rotating-Detectors Computed Tomography Architecture for Higher Temporal Resolution and Lower Radiation Dose]]></title>
##     <authors><![CDATA[Guohua Cao;  Baodong Liu;  Hao Gong;  Hengyong Yu;  Ge Wang]]></authors>
##     <affiliations><![CDATA[Sch. of Biomed. Eng. & Sci., Virginia Tech-Wake Forest Univ., Blacksburg, VA, USA]]></affiliations>
##     <controlledterms>
##       <term><![CDATA[X-ray tubes]]></term>
##       <term><![CDATA[biomedical equipment]]></term>
##       <term><![CDATA[cardiology]]></term>
##       <term><![CDATA[computerised tomography]]></term>
##       <term><![CDATA[dosimetry]]></term>
##       <term><![CDATA[image resolution]]></term>
##       <term><![CDATA[medical image processing]]></term>
##     </controlledterms>
##     <thesaurusterms>
##       <term><![CDATA[Biomedical imaging]]></term>
##       <term><![CDATA[Computed tomography]]></term>
##       <term><![CDATA[Computer architecture]]></term>
##       <term><![CDATA[Detectors]]></term>
##       <term><![CDATA[Electron tubes]]></term>
##       <term><![CDATA[Heart rate]]></term>
##       <term><![CDATA[Image resolution]]></term>
##       <term><![CDATA[Radiation monitoring]]></term>
##       <term><![CDATA[X-ray detectors]]></term>
##       <term><![CDATA[X-ray imaging]]></term>
##     </thesaurusterms>
##     <pubtitle><![CDATA[Access, IEEE]]></pubtitle>
##     <punumber><![CDATA[6287639]]></punumber>
##     <pubtype><![CDATA[Journals & Magazines]]></pubtype>
##     <publisher><![CDATA[IEEE]]></publisher>
##     <volume><![CDATA[2]]></volume>
##     <py><![CDATA[2014]]></py>
##     <spage><![CDATA[1263]]></spage>
##     <epage><![CDATA[1271]]></epage>
##     <abstract><![CDATA[In current computed tomography (CT) architecture, both X-ray tubes and X-ray detectors are rotated mechanically around an object to collect a sufficient number of projections. This architecture has been shown to not be fast enough for patients with high or irregular heart rates. Furthermore, both X-ray beams and detectors of the current architecture are made wide enough, so that the entire object is covered in the lateral direction without data truncation. Although novel acquisition protocols have recently been developed to reduce a radiation exposure, the high radiation dose from CT imaging remains a heightened public concern (especially for cardiac CT). The current CT architecture is a major bottleneck to further increase the temporal resolution and reduce the radiation dose. To overcome these problems, we present an innovative stationary-sources rotating-detectors CT (SSRD-CT) architecture based on the three stationary distributed X-ray sources and three smaller rotating X-ray detectors. Each distributed X-ray source has ~ 100 distinctive X-ray focal spots, and each detector has a narrower width compared with the conventional CT detectors. The SSRD-CT will have a field-of-view of 200 mm in diameter at isocenter, which is large enough to image many internal organs, including hearts. X-rays from the distributed sources are activated electronically to simulate the mechanical spinning of conventional single-beam X-ray sources with a high speed. The activation of individual X-ray beam will be synchronized to the corresponding rotating detector at the opposite end. Three source-detector chains can work in parallel to acquire three projections simultaneously and improve temporal resolution. Lower full-body radiation dose is expected for the proposed SSRD-CT because X-rays are restricted to irradiate a local smaller region. Taken together, the proposed SSRD-CT architecture will enable &#x2264;50-ms temporal resolution and reduce radiation dose significantly.]]></abstract>
##     <issn><![CDATA[2169-3536]]></issn>
##     <htmlFlag><![CDATA[1]]></htmlFlag>
##     <arnumber><![CDATA[6928482]]></arnumber>
##     <doi><![CDATA[10.1109/ACCESS.2014.2363367]]></doi>
##     <publicationId><![CDATA[6928482]]></publicationId>
##     <mdurl><![CDATA[http://ieeexplore.ieee.org/xpl/articleDetails.jsp?tp=&arnumber=6928482&contentType=Journals+%26+Magazines]]></mdurl>
##     <pdf><![CDATA[http://ieeexplore.ieee.org/stamp/stamp.jsp?arnumber=6928482]]></pdf>
##   </document>
##   <document>
##     <rank>427</rank>
##     <title><![CDATA[Silicon Photomultipliers Signal-to-Noise Ratio in the Continuous Wave Regime]]></title>
##     <authors><![CDATA[Adamo, G.;  Parisi, A.;  Stivala, S.;  Tomasino, A.;  Agro, D.;  Curcio, L.;  Giaconia, G.C.;  Busacca, A.;  Fallica, G.]]></authors>
##     <affiliations><![CDATA[Dept. of Energy, Inf. Eng. & Math. Models, Univ. of Palermo, Palermo, Italy]]></affiliations>
##     <controlledterms>
##       <term><![CDATA[dark conductivity]]></term>
##       <term><![CDATA[elemental semiconductors]]></term>
##       <term><![CDATA[photoconductivity]]></term>
##       <term><![CDATA[photomultipliers]]></term>
##       <term><![CDATA[shot noise]]></term>
##       <term><![CDATA[silicon]]></term>
##     </controlledterms>
##     <thesaurusterms>
##       <term><![CDATA[Current measurement]]></term>
##       <term><![CDATA[Noise measurement]]></term>
##       <term><![CDATA[Optical filters]]></term>
##       <term><![CDATA[Photonics]]></term>
##       <term><![CDATA[Signal to noise ratio]]></term>
##       <term><![CDATA[Temperature measurement]]></term>
##     </thesaurusterms>
##     <pubtitle><![CDATA[Selected Topics in Quantum Electronics, IEEE Journal of]]></pubtitle>
##     <punumber><![CDATA[2944]]></punumber>
##     <pubtype><![CDATA[Journals & Magazines]]></pubtype>
##     <publisher><![CDATA[IEEE]]></publisher>
##     <volume><![CDATA[20]]></volume>
##     <issue><![CDATA[6]]></issue>
##     <py><![CDATA[2014]]></py>
##     <spage><![CDATA[284]]></spage>
##     <epage><![CDATA[290]]></epage>
##     <abstract><![CDATA[We report on signal-to-noise ratio measurements carried out in the continuous wave regime, at different bias voltages, frequencies, and temperatures, on a class of silicon photomultipliers fabricated in planar technology on silicon p-type substrate. Signal-to-noise ratio has been measured as the ratio of the photogenerated current, filtered and averaged by a lock-in amplifier, and the root mean square deviation of the same current. The measured noise takes into account the shot noise, resulting from the photocurrent and the dark current. We have also performed a comparison between our SiPMs and a photomultiplier tube in terms of signal-to-noise ratio, as a function of the temperature of the SiPM package and at different bias voltages. Our results show the outstanding performance of this class of SiPMs even without the need of any cooling system.]]></abstract>
##     <issn><![CDATA[1077-260X]]></issn>
##     <htmlFlag><![CDATA[1]]></htmlFlag>
##     <arnumber><![CDATA[6874489]]></arnumber>
##     <doi><![CDATA[10.1109/JSTQE.2014.2346489]]></doi>
##     <publicationId><![CDATA[6874489]]></publicationId>
##     <mdurl><![CDATA[http://ieeexplore.ieee.org/xpl/articleDetails.jsp?tp=&arnumber=6874489&contentType=Journals+%26+Magazines]]></mdurl>
##     <pdf><![CDATA[http://ieeexplore.ieee.org/stamp/stamp.jsp?arnumber=6874489]]></pdf>
##   </document>
##   <document>
##     <rank>428</rank>
##     <title><![CDATA[LateBiclustering: Efficient Heuristic Algorithm for Time-Lagged Bicluster Identification]]></title>
##     <authors><![CDATA[Gonc&#x0327; alves, J.P.;  Madeira, S.C.]]></authors>
##     <affiliations><![CDATA[Centrum Wiskunde & Inf., Amsterdam, Netherlands]]></affiliations>
##     <controlledterms>
##       <term><![CDATA[biology computing]]></term>
##       <term><![CDATA[computational complexity]]></term>
##       <term><![CDATA[diseases]]></term>
##       <term><![CDATA[genetics]]></term>
##       <term><![CDATA[microorganisms]]></term>
##       <term><![CDATA[pattern clustering]]></term>
##       <term><![CDATA[string matching]]></term>
##       <term><![CDATA[time series]]></term>
##     </controlledterms>
##     <thesaurusterms>
##       <term><![CDATA[Bioinformatics]]></term>
##       <term><![CDATA[Computational biology]]></term>
##       <term><![CDATA[Gene expression]]></term>
##       <term><![CDATA[Pattern matching]]></term>
##       <term><![CDATA[Time series analysis]]></term>
##     </thesaurusterms>
##     <pubtitle><![CDATA[Computational Biology and Bioinformatics, IEEE/ACM Transactions on]]></pubtitle>
##     <punumber><![CDATA[8857]]></punumber>
##     <pubtype><![CDATA[Journals & Magazines]]></pubtype>
##     <publisher><![CDATA[IEEE]]></publisher>
##     <volume><![CDATA[11]]></volume>
##     <issue><![CDATA[5]]></issue>
##     <py><![CDATA[2014]]></py>
##     <spage><![CDATA[801]]></spage>
##     <epage><![CDATA[813]]></epage>
##     <abstract><![CDATA[Identifying patterns in temporal data is key to uncover meaningful relationships in diverse domains, from stock trading to social interactions. Also of great interest are clinical and biological applications, namely monitoring patient response to treatment or characterizing activity at the molecular level. In biology, researchers seek to gain insight into gene functions and dynamics of biological processes, as well as potential perturbations of these leading to disease, through the study of patterns emerging from gene expression time series. Clustering can group genes exhibiting similar expression profiles, but focuses on global patterns denoting rather broad, unspecific responses. Biclustering reveals local patterns, which more naturally capture the intricate collaboration between biological players, particularly under a temporal setting. Despite the general biclustering formulation being NP-hard, considering specific properties of time series has led to efficient solutions for the discovery of temporally aligned patterns. Notably, the identification of biclusters with time-lagged patterns, suggestive of transcriptional cascades, remains a challenge due to the combinatorial explosion of delayed occurrences. Herein, we propose LateBiclustering, a sensible heuristic algorithm enabling a polynomial rather than exponential time solution for the problem. We show that it identifies meaningful time-lagged biclusters relevant to the response of Saccharomyces cerevisiae to heat stress.]]></abstract>
##     <issn><![CDATA[1545-5963]]></issn>
##     <arnumber><![CDATA[6774461]]></arnumber>
##     <doi><![CDATA[10.1109/TCBB.2014.2312007]]></doi>
##     <publicationId><![CDATA[6774461]]></publicationId>
##     <mdurl><![CDATA[http://ieeexplore.ieee.org/xpl/articleDetails.jsp?tp=&arnumber=6774461&contentType=Journals+%26+Magazines]]></mdurl>
##     <pdf><![CDATA[http://ieeexplore.ieee.org/stamp/stamp.jsp?arnumber=6774461]]></pdf>
##   </document>
##   <document>
##     <rank>429</rank>
##     <title><![CDATA[Improvement in the Light Extraction of Blue InGaN/GaN-Based LEDs Using Patterned Metal Contacts]]></title>
##     <authors><![CDATA[Kadiyala, A.;  Kyoungnae Lee;  Rodak, L.E.;  Hornak, L.A.;  Korakakis, D.;  Dawson, J.M.]]></authors>
##     <affiliations><![CDATA[West Virginia Univ., Morgantown, WV, USA]]></affiliations>
##     <controlledterms>
##       <term><![CDATA[III-V semiconductors]]></term>
##       <term><![CDATA[electron beam lithography]]></term>
##       <term><![CDATA[gallium compounds]]></term>
##       <term><![CDATA[indium compounds]]></term>
##       <term><![CDATA[light emitting diodes]]></term>
##       <term><![CDATA[lighting]]></term>
##       <term><![CDATA[mathematics computing]]></term>
##       <term><![CDATA[photonic crystals]]></term>
##       <term><![CDATA[quantum well devices]]></term>
##       <term><![CDATA[silicon compounds]]></term>
##       <term><![CDATA[wide band gap semiconductors]]></term>
##     </controlledterms>
##     <thesaurusterms>
##       <term><![CDATA[Contacts]]></term>
##       <term><![CDATA[Fabrication]]></term>
##       <term><![CDATA[Gallium nitride]]></term>
##       <term><![CDATA[Light emitting diodes]]></term>
##       <term><![CDATA[Lithography]]></term>
##       <term><![CDATA[Metals]]></term>
##       <term><![CDATA[Photonic crystals]]></term>
##     </thesaurusterms>
##     <pubtitle><![CDATA[Electron Devices Society, IEEE Journal of the]]></pubtitle>
##     <punumber><![CDATA[6245494]]></punumber>
##     <pubtype><![CDATA[Journals & Magazines]]></pubtype>
##     <publisher><![CDATA[IEEE]]></publisher>
##     <volume><![CDATA[2]]></volume>
##     <issue><![CDATA[2]]></issue>
##     <py><![CDATA[2014]]></py>
##     <spage><![CDATA[16]]></spage>
##     <epage><![CDATA[22]]></epage>
##     <abstract><![CDATA[We demonstrate a method to improve the light extraction from an LED using photonic crystal (PhC)-like structures in metal contacts. A patterned metal contact with an array of Silicon Oxide (SiO<sub>x</sub>) pillars (440 nm in size) on an InGaN/GaN-based MQW LED has shown to increase output illumination uniformity through experimental characterization. Structural methods of improving light extraction using transparent contacts or dielectric photonic crystals typically require a tradeoff between improving light extraction and optimal electrical characteristics. The method presented here provides an alternate solution to provide a 15% directional improvement (surface normal) in the radiation profile and ~ 30% increase in the respective intensity profile without affecting the electrical characteristics of the device. Electron beam patterning of hydrogen silesquioxane (HSQ), a novel electron beam resist is used in patterning these metal contacts. After patterning, thermal curing of the patterned resist is done to form SiO<sub>x</sub> pillars. These SiO<sub>x</sub> pillars aid as a mask for transferring the pattern to the p-metal contact. Electrical and optical characterization results of LEDs fabricated with and without patterned contacts are presented. We present the radiation and intensity profiles of the planar and patterned devices extracted using Matlab-based image analysis technique from 200 &#x03BC;m (diameter) circular unpackaged LEDs.]]></abstract>
##     <issn><![CDATA[2168-6734]]></issn>
##     <htmlFlag><![CDATA[1]]></htmlFlag>
##     <arnumber><![CDATA[6656882]]></arnumber>
##     <doi><![CDATA[10.1109/JEDS.2013.2289308]]></doi>
##     <publicationId><![CDATA[6656882]]></publicationId>
##     <mdurl><![CDATA[http://ieeexplore.ieee.org/xpl/articleDetails.jsp?tp=&arnumber=6656882&contentType=Journals+%26+Magazines]]></mdurl>
##     <pdf><![CDATA[http://ieeexplore.ieee.org/stamp/stamp.jsp?arnumber=6656882]]></pdf>
##   </document>
##   <document>
##     <rank>430</rank>
##     <title><![CDATA[Wireless Spintronics Modulation With a Spin Torque Nano-Oscillator (STNO) Array]]></title>
##     <authors><![CDATA[Inn-yeal Oh;  Seung-young Park;  Doo-Hyung Kang;  Chul Soon Park]]></authors>
##     <affiliations><![CDATA[Dept. of Electr. Eng., Korea Adv. Inst. of Sci. & Technol. (KAIST), Daejeon, South Korea]]></affiliations>
##     <controlledterms>
##       <term><![CDATA[amplitude shift keying]]></term>
##       <term><![CDATA[frequency division multiplexing]]></term>
##       <term><![CDATA[high-pass filters]]></term>
##       <term><![CDATA[low-pass filters]]></term>
##       <term><![CDATA[magnetoelectronics]]></term>
##       <term><![CDATA[nanoelectronics]]></term>
##       <term><![CDATA[oscillators]]></term>
##       <term><![CDATA[torque]]></term>
##     </controlledterms>
##     <thesaurusterms>
##       <term><![CDATA[Arrays]]></term>
##       <term><![CDATA[Frequency division multiplexing]]></term>
##       <term><![CDATA[Frequency modulation]]></term>
##       <term><![CDATA[Magnetoelectronics]]></term>
##       <term><![CDATA[Oscillators]]></term>
##       <term><![CDATA[Wireless communication]]></term>
##     </thesaurusterms>
##     <pubtitle><![CDATA[Microwave and Wireless Components Letters, IEEE]]></pubtitle>
##     <punumber><![CDATA[7260]]></punumber>
##     <pubtype><![CDATA[Journals & Magazines]]></pubtype>
##     <publisher><![CDATA[IEEE]]></publisher>
##     <volume><![CDATA[24]]></volume>
##     <issue><![CDATA[7]]></issue>
##     <py><![CDATA[2014]]></py>
##     <spage><![CDATA[502]]></spage>
##     <epage><![CDATA[504]]></epage>
##     <abstract><![CDATA[This letter presents a spin RF-direct modulation using a frequency division multiplex (FDM) with a spin torque nano-oscillator (STNO) array. For the proposed modulation, two STNO are bonded at each branch of a T-junction on a PCB, where a high pass filter (HPF) and a low pass filter (LPF) are connected on each branch of the T-junction to transmit the modulated signal into one antenna. Each STNO modulates by on-off keying (OOK) with digital data directly after setting on the separation of channels for two STNOs into 700 MHz, one STNO at 3.5 GHz frequency and the other at 4.2 GHz, with consideration of the minimum sensitivity and interference related with isolation between channels. A data rate of up to 400 Kbps is obtained at a distance of 10 mm, and the dc power consumption is 3 mW per STNO, including logic circuit operation. The PCB size is as small as 28 &#x00D7; 27 mm, including the 4.2 &#x00D7; 2.1 mm STNO array.]]></abstract>
##     <issn><![CDATA[1531-1309]]></issn>
##     <htmlFlag><![CDATA[1]]></htmlFlag>
##     <arnumber><![CDATA[6827216]]></arnumber>
##     <doi><![CDATA[10.1109/LMWC.2014.2316494]]></doi>
##     <publicationId><![CDATA[6827216]]></publicationId>
##     <mdurl><![CDATA[http://ieeexplore.ieee.org/xpl/articleDetails.jsp?tp=&arnumber=6827216&contentType=Journals+%26+Magazines]]></mdurl>
##     <pdf><![CDATA[http://ieeexplore.ieee.org/stamp/stamp.jsp?arnumber=6827216]]></pdf>
##   </document>
##   <document>
##     <rank>431</rank>
##     <title><![CDATA[Depth-Based Human Fall Detection via Shape Features and Improved Extreme Learning Machine]]></title>
##     <authors><![CDATA[Xin Ma;  Haibo Wang;  Bingxia Xue;  Mingang Zhou;  Bing Ji;  Yibin Li]]></authors>
##     <affiliations><![CDATA[Sch. of Control Sci. & Eng., Shandong Univ., Jinan, China]]></affiliations>
##     <controlledterms>
##       <term><![CDATA[bending]]></term>
##       <term><![CDATA[biomedical optical imaging]]></term>
##       <term><![CDATA[cameras]]></term>
##       <term><![CDATA[feature extraction]]></term>
##       <term><![CDATA[gait analysis]]></term>
##       <term><![CDATA[geriatrics]]></term>
##       <term><![CDATA[image classification]]></term>
##       <term><![CDATA[image representation]]></term>
##       <term><![CDATA[learning (artificial intelligence)]]></term>
##       <term><![CDATA[mechanoception]]></term>
##       <term><![CDATA[medical image processing]]></term>
##       <term><![CDATA[particle swarm optimisation]]></term>
##     </controlledterms>
##     <thesaurusterms>
##       <term><![CDATA[Accuracy]]></term>
##       <term><![CDATA[Algorithm design and analysis]]></term>
##       <term><![CDATA[Cameras]]></term>
##       <term><![CDATA[Computer vision]]></term>
##       <term><![CDATA[Feature extraction]]></term>
##       <term><![CDATA[Particle swarm optimization]]></term>
##     </thesaurusterms>
##     <pubtitle><![CDATA[Biomedical and Health Informatics, IEEE Journal of]]></pubtitle>
##     <punumber><![CDATA[6221020]]></punumber>
##     <pubtype><![CDATA[Journals & Magazines]]></pubtype>
##     <publisher><![CDATA[IEEE]]></publisher>
##     <volume><![CDATA[18]]></volume>
##     <issue><![CDATA[6]]></issue>
##     <py><![CDATA[2014]]></py>
##     <spage><![CDATA[1915]]></spage>
##     <epage><![CDATA[1922]]></epage>
##     <abstract><![CDATA[Falls are one of the major causes leading to injury of elderly people. Using wearable devices for fall detection has a high cost and may cause inconvenience to the daily lives of the elderly. In this paper, we present an automated fall detection approach that requires only a low-cost depth camera. Our approach combines two computer vision techniques-shape-based fall characterization and a learning-based classifier to distinguish falls from other daily actions. Given a fall video clip, we extract curvature scale space (CSS) features of human silhouettes at each frame and represent the action by a bag of CSS words (BoCSS). Then, we utilize the extreme learning machine (ELM) classifier to identify the BoCSS representation of a fall from those of other actions. In order to eliminate the sensitivity of ELM to its hyperparameters, we present a variable-length particle swarm optimization algorithm to optimize the number of hidden neurons, corresponding input weights, and biases of ELM. Using a low-cost Kinect depth camera, we build an action dataset that consists of six types of actions (falling, bending, sitting, squatting, walking, and lying) from ten subjects. Experimenting with the dataset shows that our approach can achieve up to 91.15% sensitivity, 77.14% specificity, and 86.83% accuracy. On a public dataset, our approach performs comparably to state-of-the-art fall detection methods that need multiple cameras.]]></abstract>
##     <issn><![CDATA[2168-2194]]></issn>
##     <htmlFlag><![CDATA[1]]></htmlFlag>
##     <arnumber><![CDATA[6730899]]></arnumber>
##     <doi><![CDATA[10.1109/JBHI.2014.2304357]]></doi>
##     <publicationId><![CDATA[6730899]]></publicationId>
##     <mdurl><![CDATA[http://ieeexplore.ieee.org/xpl/articleDetails.jsp?tp=&arnumber=6730899&contentType=Journals+%26+Magazines]]></mdurl>
##     <pdf><![CDATA[http://ieeexplore.ieee.org/stamp/stamp.jsp?arnumber=6730899]]></pdf>
##   </document>
##   <document>
##     <rank>432</rank>
##     <title><![CDATA[A 240 &#x00D7; 180 130 dB 3 &#x00B5;s Latency Global Shutter Spatiotemporal Vision Sensor]]></title>
##     <authors><![CDATA[Brandli, C.;  Berner, R.;  Minhao Yang;  Shih-Chii Liu;  Delbruck, T.]]></authors>
##     <affiliations><![CDATA[Inst. of Neuroinf., Univ. & ETH Zurich, Zurich, Switzerland]]></affiliations>
##     <controlledterms>
##       <term><![CDATA[CMOS image sensors]]></term>
##       <term><![CDATA[photodetectors]]></term>
##       <term><![CDATA[photodiodes]]></term>
##       <term><![CDATA[sensor arrays]]></term>
##     </controlledterms>
##     <thesaurusterms>
##       <term><![CDATA[Cameras]]></term>
##       <term><![CDATA[Photoconductivity]]></term>
##       <term><![CDATA[Photodiodes]]></term>
##       <term><![CDATA[Photoreceptors]]></term>
##       <term><![CDATA[Robot sensing systems]]></term>
##       <term><![CDATA[Universal Serial Bus]]></term>
##       <term><![CDATA[Voltage control]]></term>
##     </thesaurusterms>
##     <pubtitle><![CDATA[Solid-State Circuits, IEEE Journal of]]></pubtitle>
##     <punumber><![CDATA[4]]></punumber>
##     <pubtype><![CDATA[Journals & Magazines]]></pubtype>
##     <publisher><![CDATA[IEEE]]></publisher>
##     <volume><![CDATA[49]]></volume>
##     <issue><![CDATA[10]]></issue>
##     <py><![CDATA[2014]]></py>
##     <spage><![CDATA[2333]]></spage>
##     <epage><![CDATA[2341]]></epage>
##     <abstract><![CDATA[Event-based dynamic vision sensors (DVSs) asynchronously report log intensity changes. Their high dynamic range, sub-ms latency and sparse output make them useful in applications such as robotics and real-time tracking. However they discard absolute intensity information which is useful for object recognition and classification. This paper presents a dynamic and active pixel vision sensor (DAVIS) which addresses this deficiency by outputting asynchronous DVS events and synchronous global shutter frames concurrently. The active pixel sensor (APS) circuits and the DVS circuits within a pixel share a single photodiode. Measurements from a 240&#x00D7;180 sensor array of 18.5 &#x03BC;m 2 pixels fabricated in a 0.18 &#x03BC;m 6M1P CMOS image sensor (CIS) technology show a dynamic range of 130 dB with 11% contrast detection threshold, minimum 3 &#x03BC;s latency, and 3.5% contrast matching for the DVS pathway; and a 51 dB dynamic range with 0.5% FPN for the APS readout.]]></abstract>
##     <issn><![CDATA[0018-9200]]></issn>
##     <htmlFlag><![CDATA[1]]></htmlFlag>
##     <arnumber><![CDATA[6889103]]></arnumber>
##     <doi><![CDATA[10.1109/JSSC.2014.2342715]]></doi>
##     <publicationId><![CDATA[6889103]]></publicationId>
##     <mdurl><![CDATA[http://ieeexplore.ieee.org/xpl/articleDetails.jsp?tp=&arnumber=6889103&contentType=Journals+%26+Magazines]]></mdurl>
##     <pdf><![CDATA[http://ieeexplore.ieee.org/stamp/stamp.jsp?arnumber=6889103]]></pdf>
##   </document>
##   <document>
##     <rank>433</rank>
##     <title><![CDATA[Sizing Strategy of Distributed Battery Storage System With High Penetration of Photovoltaic for Voltage Regulation and Peak Load Shaving]]></title>
##     <authors><![CDATA[Ye Yang;  Hui Li;  Aichhorn, A.;  Jianping Zheng;  Greenleaf, M.]]></authors>
##     <affiliations><![CDATA[Center for Adv. Power Syst., Florida State Univ., Tallahassee, FL, USA]]></affiliations>
##     <controlledterms>
##       <term><![CDATA[battery storage plants]]></term>
##       <term><![CDATA[cost-benefit analysis]]></term>
##       <term><![CDATA[energy management systems]]></term>
##       <term><![CDATA[photovoltaic power systems]]></term>
##       <term><![CDATA[power generation control]]></term>
##       <term><![CDATA[voltage control]]></term>
##     </controlledterms>
##     <thesaurusterms>
##       <term><![CDATA[Batteries]]></term>
##       <term><![CDATA[Discharges (electric)]]></term>
##       <term><![CDATA[Energy management]]></term>
##       <term><![CDATA[Load modeling]]></term>
##       <term><![CDATA[Power generation]]></term>
##       <term><![CDATA[Power systems]]></term>
##       <term><![CDATA[Voltage control]]></term>
##     </thesaurusterms>
##     <pubtitle><![CDATA[Smart Grid, IEEE Transactions on]]></pubtitle>
##     <punumber><![CDATA[5165411]]></punumber>
##     <pubtype><![CDATA[Journals & Magazines]]></pubtype>
##     <publisher><![CDATA[IEEE]]></publisher>
##     <volume><![CDATA[5]]></volume>
##     <issue><![CDATA[2]]></issue>
##     <py><![CDATA[2014]]></py>
##     <spage><![CDATA[982]]></spage>
##     <epage><![CDATA[991]]></epage>
##     <abstract><![CDATA[This paper proposes an effective sizing strategy for distributed battery energy storage system (BESS) in the distribution networks under high photovoltaic (PV) penetration level. The main objective of the proposed method is to optimize the size of the distributed BESS and derive the cost-benefit analysis when the distributed BESS is applied for voltage regulation and peak load shaving. In particular, a system model that includes a physical battery model and a voltage regulation and peak load shaving oriented energy management system (EMS) is developed to apply the proposed strategy. The cost-benefit analysis presented in this paper considers factors of BESS influence on the work stress of voltage regulation devices, load shifting and peaking power generation, as well as individual BESS cost with its lifetime estimation. Based on the cost-benefit analysis, the cost-benefit size can be determined for the distributed BESS.]]></abstract>
##     <issn><![CDATA[1949-3053]]></issn>
##     <htmlFlag><![CDATA[1]]></htmlFlag>
##     <arnumber><![CDATA[6609116]]></arnumber>
##     <doi><![CDATA[10.1109/TSG.2013.2282504]]></doi>
##     <publicationId><![CDATA[6609116]]></publicationId>
##     <mdurl><![CDATA[http://ieeexplore.ieee.org/xpl/articleDetails.jsp?tp=&arnumber=6609116&contentType=Journals+%26+Magazines]]></mdurl>
##     <pdf><![CDATA[http://ieeexplore.ieee.org/stamp/stamp.jsp?arnumber=6609116]]></pdf>
##   </document>
##   <document>
##     <rank>434</rank>
##     <title><![CDATA[Theoretical Study of Colliding Pulse Passively Mode-Locked Semiconductor Ring Lasers With an Intracavity Mach&#x2013;Zehnder Modulator]]></title>
##     <authors><![CDATA[Moskalenko, V.;  Javaloyes, J.;  Balle, S.;  Smit, M.K.;  Bente, E.A.J.M.]]></authors>
##     <affiliations><![CDATA[Dept. of Electr. Eng., Eindhoven Univ. of Technol., Eindhoven, Netherlands]]></affiliations>
##     <controlledterms>
##       <term><![CDATA[III-V semiconductors]]></term>
##       <term><![CDATA[Mach-Zehnder interferometers]]></term>
##       <term><![CDATA[gallium arsenide]]></term>
##       <term><![CDATA[indium compounds]]></term>
##       <term><![CDATA[laser cavity resonators]]></term>
##       <term><![CDATA[optical pulse generation]]></term>
##       <term><![CDATA[quantum well lasers]]></term>
##       <term><![CDATA[ring lasers]]></term>
##     </controlledterms>
##     <thesaurusterms>
##       <term><![CDATA[Cavity resonators]]></term>
##       <term><![CDATA[Gain]]></term>
##       <term><![CDATA[Laser mode locking]]></term>
##       <term><![CDATA[Modulation]]></term>
##       <term><![CDATA[Optical pulse shaping]]></term>
##       <term><![CDATA[Optical pulses]]></term>
##       <term><![CDATA[Ring lasers]]></term>
##     </thesaurusterms>
##     <pubtitle><![CDATA[Quantum Electronics, IEEE Journal of]]></pubtitle>
##     <punumber><![CDATA[3]]></punumber>
##     <pubtype><![CDATA[Journals & Magazines]]></pubtype>
##     <publisher><![CDATA[IEEE]]></publisher>
##     <volume><![CDATA[50]]></volume>
##     <issue><![CDATA[6]]></issue>
##     <py><![CDATA[2014]]></py>
##     <spage><![CDATA[415]]></spage>
##     <epage><![CDATA[422]]></epage>
##     <abstract><![CDATA[In this paper, we theoretically study the impact of an intracavity filter based on a Mach-Zehnder interferometer (MZI) on the pulses emitted by InGaAsP/InP passively mode-locked quantum well ring lasers. The filter allows to control the net gain curvature in the device, hereby providing for control over the modes that participate in the dynamics. Simulations of a traveling-wave model indicate that the pulsewidth can be controlled and reduced down to 500 fs. We present and verify a simple algorithm, which can be used for calculating the optimum values of the MZI parameters. The optimum parameters are then used in the study of an MZI passive-mode-locked laser under various operating conditions.]]></abstract>
##     <issn><![CDATA[0018-9197]]></issn>
##     <arnumber><![CDATA[6785958]]></arnumber>
##     <doi><![CDATA[10.1109/JQE.2014.2316339]]></doi>
##     <publicationId><![CDATA[6785958]]></publicationId>
##     <mdurl><![CDATA[http://ieeexplore.ieee.org/xpl/articleDetails.jsp?tp=&arnumber=6785958&contentType=Journals+%26+Magazines]]></mdurl>
##     <pdf><![CDATA[http://ieeexplore.ieee.org/stamp/stamp.jsp?arnumber=6785958]]></pdf>
##   </document>
##   <document>
##     <rank>435</rank>
##     <title><![CDATA[Hybrid Gate-Level Leakage Model for Monte Carlo Analysis on Multiple GPUs]]></title>
##     <authors><![CDATA[Jinwook Kim;  Young Hwan Kim]]></authors>
##     <affiliations><![CDATA[Pohang Univ. of Sci. & Technol., Pohang, South Korea]]></affiliations>
##     <controlledterms>
##       <term><![CDATA[Monte Carlo methods]]></term>
##       <term><![CDATA[graphics processing units]]></term>
##       <term><![CDATA[polynomials]]></term>
##       <term><![CDATA[table lookup]]></term>
##     </controlledterms>
##     <thesaurusterms>
##       <term><![CDATA[Analytical models]]></term>
##       <term><![CDATA[Computational modeling]]></term>
##       <term><![CDATA[Graphics processing units]]></term>
##       <term><![CDATA[Interpolation]]></term>
##       <term><![CDATA[Leakage currents]]></term>
##       <term><![CDATA[Logic gates]]></term>
##       <term><![CDATA[Monte Carlo methods]]></term>
##       <term><![CDATA[Table lookup]]></term>
##     </thesaurusterms>
##     <pubtitle><![CDATA[Access, IEEE]]></pubtitle>
##     <punumber><![CDATA[6287639]]></punumber>
##     <pubtype><![CDATA[Journals & Magazines]]></pubtype>
##     <publisher><![CDATA[IEEE]]></publisher>
##     <volume><![CDATA[2]]></volume>
##     <py><![CDATA[2014]]></py>
##     <spage><![CDATA[183]]></spage>
##     <epage><![CDATA[194]]></epage>
##     <abstract><![CDATA[This paper proposes a hybrid gate-level leakage model for the use with the Monte Carlo (MC) analysis approach, which combines a lookup table (LUT) model with a first-order exponential-polynomial model (first-order model, herein). For the process parameters having strong nonlinear relationships with the logarithm of leakage current, the proposed model uses the LUT approach for the sake of modeling accuracy. For the other process parameters, it uses the first-order model for increased efficiency. During the library characterization for each type of logic gates, the proposed approach determines the process parameters for which it will use the LUT model. And, it determines the number of LUT data points, which can maximize analysis efficiency with acceptable accuracy, based on the user-defined threshold. The proposed model was implemented for gate-level MC leakage analysis using three graphic processing units. In experiments, the proposed approach exhibited the average errors of &#x00A1;5% in both mean and standard deviation with reference to SPICE-level MC leakage analysis. In comparison, MC analysis with the first-order model exhibited more than 90% errors. In CPU times, the proposed hybrid approach took only two to five times longer runtimes. In comparison with the full LUT model, the proposed hybrid model was up to one hundred times faster while increasing the average errors by only 3%. Finally, the proposed approach completed a leakage analysis of an OpenSparc T2 core of 4.5 million gates with a runtime of .]]></abstract>
##     <issn><![CDATA[2169-3536]]></issn>
##     <htmlFlag><![CDATA[1]]></htmlFlag>
##     <arnumber><![CDATA[6762843]]></arnumber>
##     <doi><![CDATA[10.1109/ACCESS.2014.2308922]]></doi>
##     <publicationId><![CDATA[6762843]]></publicationId>
##     <mdurl><![CDATA[http://ieeexplore.ieee.org/xpl/articleDetails.jsp?tp=&arnumber=6762843&contentType=Journals+%26+Magazines]]></mdurl>
##     <pdf><![CDATA[http://ieeexplore.ieee.org/stamp/stamp.jsp?arnumber=6762843]]></pdf>
##   </document>
##   <document>
##     <rank>436</rank>
##     <title><![CDATA[Automatic Auroral Detection in Color All-Sky Camera Images]]></title>
##     <authors><![CDATA[Rao, J.;  Partamies, N.;  Amariutei, O.;  Syrjasuo, M.;  van de Sande, K.E.A.]]></authors>
##     <affiliations><![CDATA[Dept. of Inf. & Comput. Sci., Aalto Univ., Espoo, Finland]]></affiliations>
##     <controlledterms>
##       <term><![CDATA[atmospheric techniques]]></term>
##       <term><![CDATA[aurora]]></term>
##       <term><![CDATA[feature extraction]]></term>
##       <term><![CDATA[geophysical image processing]]></term>
##       <term><![CDATA[image classification]]></term>
##       <term><![CDATA[remote sensing]]></term>
##       <term><![CDATA[support vector machines]]></term>
##     </controlledterms>
##     <thesaurusterms>
##       <term><![CDATA[Cameras]]></term>
##       <term><![CDATA[Classification]]></term>
##       <term><![CDATA[Computer vision]]></term>
##       <term><![CDATA[Feature extraction]]></term>
##       <term><![CDATA[Histograms]]></term>
##       <term><![CDATA[Image color analysis]]></term>
##       <term><![CDATA[Scene detection]]></term>
##       <term><![CDATA[Support vector machines]]></term>
##     </thesaurusterms>
##     <pubtitle><![CDATA[Selected Topics in Applied Earth Observations and Remote Sensing, IEEE Journal of]]></pubtitle>
##     <punumber><![CDATA[4609443]]></punumber>
##     <pubtype><![CDATA[Journals & Magazines]]></pubtype>
##     <publisher><![CDATA[IEEE]]></publisher>
##     <volume><![CDATA[7]]></volume>
##     <issue><![CDATA[12]]></issue>
##     <py><![CDATA[2014]]></py>
##     <spage><![CDATA[4717]]></spage>
##     <epage><![CDATA[4725]]></epage>
##     <abstract><![CDATA[Every winter, the all-sky cameras (ASCs) in the MIRACLE network take images of the night sky at regular intervals of 10-20 s. This amounts to millions of images that not only need to be pruned, but there is also a need for efficient auroral activity detection techniques. In this paper, we describe a method for performing automated classification of ASC images into three mutually exclusive classes: aurora, no aurora, and cloudy. This not only reduces the amount of data to be processed, but also facilitates in building statistical models linking the magnetic fluctuations and auroral activity helping us to get a step closer to forecasting auroral activity. We experimented with different feature extraction techniques coupled with Support Vector Machines classification. Color variants of Scale Invariant Feature Transform (SIFT) features, specifically Opponent SIFT features, were found to perform better than other feature extraction techniques. With Opponent SIFT features, we were able to build a classification model with a cross-validation accuracy of 91%, which was further improved using temporal information and elimination of outliers which makes it accurate enough for operational data pruning purposes. Since the problem is essentially similar to scene detection, local point description features perform better than global- and texture-based feature descriptors.]]></abstract>
##     <issn><![CDATA[1939-1404]]></issn>
##     <htmlFlag><![CDATA[1]]></htmlFlag>
##     <arnumber><![CDATA[6817533]]></arnumber>
##     <doi><![CDATA[10.1109/JSTARS.2014.2321433]]></doi>
##     <publicationId><![CDATA[6817533]]></publicationId>
##     <mdurl><![CDATA[http://ieeexplore.ieee.org/xpl/articleDetails.jsp?tp=&arnumber=6817533&contentType=Journals+%26+Magazines]]></mdurl>
##     <pdf><![CDATA[http://ieeexplore.ieee.org/stamp/stamp.jsp?arnumber=6817533]]></pdf>
##   </document>
##   <document>
##     <rank>437</rank>
##     <title><![CDATA[Challenges Toward Wireless Communications for High-Speed Railway]]></title>
##     <authors><![CDATA[Bo Ai;  Xiang Cheng;  Kurner, T.;  Zhang-dui Zhong;  Ke Guan;  Rui-Si He;  Lei Xiong;  Matolak, D.W.;  Michelson, D.G.;  Briso-Rodriguez, C.]]></authors>
##     <affiliations><![CDATA[State Key Lab. of Rail Traffic Control & Safety, Beijing Jiaotong Univ., Beijing, China]]></affiliations>
##     <controlledterms>
##       <term><![CDATA[data communication]]></term>
##       <term><![CDATA[magnetic levitation]]></term>
##       <term><![CDATA[radiocommunication]]></term>
##       <term><![CDATA[railway communication]]></term>
##       <term><![CDATA[railways]]></term>
##       <term><![CDATA[wireless channels]]></term>
##     </controlledterms>
##     <thesaurusterms>
##       <term><![CDATA[Antenna measurements]]></term>
##       <term><![CDATA[Communication system security]]></term>
##       <term><![CDATA[Control systems]]></term>
##       <term><![CDATA[Mobile communication]]></term>
##       <term><![CDATA[Rail transportation]]></term>
##       <term><![CDATA[Standards]]></term>
##       <term><![CDATA[Wireless communication]]></term>
##     </thesaurusterms>
##     <pubtitle><![CDATA[Intelligent Transportation Systems, IEEE Transactions on]]></pubtitle>
##     <punumber><![CDATA[6979]]></punumber>
##     <pubtype><![CDATA[Journals & Magazines]]></pubtype>
##     <publisher><![CDATA[IEEE]]></publisher>
##     <volume><![CDATA[15]]></volume>
##     <issue><![CDATA[5]]></issue>
##     <py><![CDATA[2014]]></py>
##     <spage><![CDATA[2143]]></spage>
##     <epage><![CDATA[2158]]></epage>
##     <abstract><![CDATA[High-speed railway (HSR) brings convenience to peoples' lives and is generally considered as one of the most sustainable developments for ground transportation. One of the important parts of HSR construction is the signaling system, which is also called the &#x201C;operation control system,&#x201D; where wireless communications play a key role in the transmission of train control data. We discuss in detail the main differences in scientific research for wireless communications between the HSR operation scenarios and the conventional public land mobile scenarios. The latest research progress in wireless channel modeling in viaducts, cuttings, and tunnels scenarios are discussed. The characteristics of nonstationary channel and the line-of-sight (LOS) sparse and LOS multiple-input-multiple-output channels, which are the typical channels in HSR scenarios, are analyzed. Some novel concepts such as composite transportation and key challenging techniques such as train-to-train communication, vacuum maglev train techniques, the security for HSR, and the fifth-generation wireless communications related techniques for future HSR development for safer, more comfortable, and more secure HSR operation are also discussed.]]></abstract>
##     <issn><![CDATA[1524-9050]]></issn>
##     <htmlFlag><![CDATA[1]]></htmlFlag>
##     <arnumber><![CDATA[6808529]]></arnumber>
##     <doi><![CDATA[10.1109/TITS.2014.2310771]]></doi>
##     <publicationId><![CDATA[6808529]]></publicationId>
##     <mdurl><![CDATA[http://ieeexplore.ieee.org/xpl/articleDetails.jsp?tp=&arnumber=6808529&contentType=Journals+%26+Magazines]]></mdurl>
##     <pdf><![CDATA[http://ieeexplore.ieee.org/stamp/stamp.jsp?arnumber=6808529]]></pdf>
##   </document>
##   <document>
##     <rank>438</rank>
##     <title><![CDATA[LiDAR-Derived Surface Roughness Texture Mapping: Application to Mount St. Helens Pumice Plain Deposit Analysis]]></title>
##     <authors><![CDATA[Whelley, P.L.;  Glaze, L.S.;  Calder, E.S.;  Harding, D.J.]]></authors>
##     <affiliations><![CDATA[Dept. of Geol., Univ. at Buffalo SUNY, Buffalo, NY, USA]]></affiliations>
##     <controlledterms>
##       <term><![CDATA[optical radar]]></term>
##       <term><![CDATA[remote sensing by laser beam]]></term>
##       <term><![CDATA[rocks]]></term>
##       <term><![CDATA[statistical analysis]]></term>
##       <term><![CDATA[terrain mapping]]></term>
##       <term><![CDATA[volcanology]]></term>
##     </controlledterms>
##     <thesaurusterms>
##       <term><![CDATA[Laser radar]]></term>
##       <term><![CDATA[Rough surfaces]]></term>
##       <term><![CDATA[Surface morphology]]></term>
##       <term><![CDATA[Surface roughness]]></term>
##       <term><![CDATA[Surface texture]]></term>
##       <term><![CDATA[Surface topography]]></term>
##       <term><![CDATA[Surface treatment]]></term>
##     </thesaurusterms>
##     <pubtitle><![CDATA[Geoscience and Remote Sensing, IEEE Transactions on]]></pubtitle>
##     <punumber><![CDATA[36]]></punumber>
##     <pubtype><![CDATA[Journals & Magazines]]></pubtype>
##     <publisher><![CDATA[IEEE]]></publisher>
##     <volume><![CDATA[52]]></volume>
##     <issue><![CDATA[1]]></issue>
##     <part><![CDATA[2]]></part>
##     <py><![CDATA[2014]]></py>
##     <spage><![CDATA[426]]></spage>
##     <epage><![CDATA[438]]></epage>
##     <abstract><![CDATA[Statistical measures of patterns (textures) in surface roughness are used to quantitatively differentiate volcanic deposit facies on the Pumice Plain, on the northern flank of Mount St. Helens (MSH). Surface roughness values are derived from a Light Detection and Ranging (LiDAR) point cloud collected in 2004 from a fixed-wing airborne platform. Patterns in surface roughness are characterized using co-occurrence texture statistics. Pristine-pyroclastic, reworked-pyroclastic, mudflow, boulder beds, eroded lava flows, braided streams, and other units within the Pumice Plain are all found to have significantly distinct roughness textures. The MSH deposits are reasonably accessible, and the textural variations have been verified in the field. Results of this work indicate that by affecting the distribution of large clasts and tens-of-meter scale landforms, modification of pyroclastic deposits by lahars alters the morphology of the surface in detectable quantifiable ways. When a lahar erodes a pyroclastic deposit, surface roughness increases, as does the randomness in the deposit surface. Conversely, when a lahar deposits material, the resulting landforms are less rough but more random than pristine pumice-rich pyroclastic deposits. By mapping these relationships and others, volcanic deposit facies can be differentiated. This new method of mapping, based on roughness texture, has the potential to aid mapping efforts in more remote regions, both on this planet and elsewhere in the solar system.]]></abstract>
##     <issn><![CDATA[0196-2892]]></issn>
##     <htmlFlag><![CDATA[1]]></htmlFlag>
##     <arnumber><![CDATA[6476666]]></arnumber>
##     <doi><![CDATA[10.1109/TGRS.2013.2241443]]></doi>
##     <publicationId><![CDATA[6476666]]></publicationId>
##     <mdurl><![CDATA[http://ieeexplore.ieee.org/xpl/articleDetails.jsp?tp=&arnumber=6476666&contentType=Journals+%26+Magazines]]></mdurl>
##     <pdf><![CDATA[http://ieeexplore.ieee.org/stamp/stamp.jsp?arnumber=6476666]]></pdf>
##   </document>
##   <document>
##     <rank>439</rank>
##     <title><![CDATA[Design and Analysis of a Miniature Intensity Modulator Based on a Silicon-Polymer-Metal Hybrid Plasmonic Waveguide]]></title>
##     <authors><![CDATA[Xiaomeng Sun;  Linjie Zhou;  Haike Zhu;  Qianqian Wu;  Xinwan Li;  Jianping Chen]]></authors>
##     <affiliations><![CDATA[Dept. of Electron. Eng., Shanghai Jiao Tong Univ., Shanghai, China]]></affiliations>
##     <controlledterms>
##       <term><![CDATA[electro-optical effects]]></term>
##       <term><![CDATA[elemental semiconductors]]></term>
##       <term><![CDATA[finite element analysis]]></term>
##       <term><![CDATA[intensity modulation]]></term>
##       <term><![CDATA[nanophotonics]]></term>
##       <term><![CDATA[optical design techniques]]></term>
##       <term><![CDATA[optical polymers]]></term>
##       <term><![CDATA[optical waveguides]]></term>
##       <term><![CDATA[plasmonics]]></term>
##       <term><![CDATA[silicon]]></term>
##     </controlledterms>
##     <thesaurusterms>
##       <term><![CDATA[Modulation]]></term>
##       <term><![CDATA[Optical refraction]]></term>
##       <term><![CDATA[Optical variables control]]></term>
##       <term><![CDATA[Optical waveguides]]></term>
##       <term><![CDATA[Plasmons]]></term>
##       <term><![CDATA[Polymers]]></term>
##       <term><![CDATA[Silicon]]></term>
##     </thesaurusterms>
##     <pubtitle><![CDATA[Photonics Journal, IEEE]]></pubtitle>
##     <punumber><![CDATA[4563994]]></punumber>
##     <pubtype><![CDATA[Journals & Magazines]]></pubtype>
##     <publisher><![CDATA[IEEE]]></publisher>
##     <volume><![CDATA[6]]></volume>
##     <issue><![CDATA[3]]></issue>
##     <py><![CDATA[2014]]></py>
##     <spage><![CDATA[1]]></spage>
##     <epage><![CDATA[10]]></epage>
##     <abstract><![CDATA[We propose a miniature optical intensity modulator based on a silicon-polymer-metal hybrid plasmonic waveguide. Benefiting from the high mode confinement of hybrid plasmonic waveguide and the high linear electro-optic effect of polymer material, the intensity modulator is ultra-compact with a length of only ~ 13 &#x03BC;m. The device is optimized using numerical simulations based on the finite element method (FEM). The modulator exhibits a large modulation bandwidth of 90 GHz, a modulation depth of 12 dB at 6 V, and low power consumption of 24.3 fJ/bit.]]></abstract>
##     <issn><![CDATA[1943-0655]]></issn>
##     <htmlFlag><![CDATA[1]]></htmlFlag>
##     <arnumber><![CDATA[6827164]]></arnumber>
##     <doi><![CDATA[10.1109/JPHOT.2014.2329395]]></doi>
##     <publicationId><![CDATA[6827164]]></publicationId>
##     <mdurl><![CDATA[http://ieeexplore.ieee.org/xpl/articleDetails.jsp?tp=&arnumber=6827164&contentType=Journals+%26+Magazines]]></mdurl>
##     <pdf><![CDATA[http://ieeexplore.ieee.org/stamp/stamp.jsp?arnumber=6827164]]></pdf>
##   </document>
##   <document>
##     <rank>440</rank>
##     <title><![CDATA[An Efficient Implementation of the ITU-R Channel Model for Device-to-Device Simulation]]></title>
##     <authors><![CDATA[Kyung-Won Kim;  Seong-Jun Oh]]></authors>
##     <affiliations><![CDATA[Coll. of Inf. & Commun., Korea Univ., Seoul, South Korea]]></affiliations>
##     <controlledterms>
##       <term><![CDATA[4G mobile communication]]></term>
##       <term><![CDATA[cellular radio]]></term>
##       <term><![CDATA[fading channels]]></term>
##       <term><![CDATA[multipath channels]]></term>
##       <term><![CDATA[next generation networks]]></term>
##     </controlledterms>
##     <thesaurusterms>
##       <term><![CDATA[Channel models]]></term>
##       <term><![CDATA[Computational modeling]]></term>
##       <term><![CDATA[Doppler effect]]></term>
##       <term><![CDATA[Fading]]></term>
##       <term><![CDATA[Fourier transforms]]></term>
##       <term><![CDATA[Load modeling]]></term>
##       <term><![CDATA[Time-frequency analysis]]></term>
##     </thesaurusterms>
##     <pubtitle><![CDATA[Communications Letters, IEEE]]></pubtitle>
##     <punumber><![CDATA[4234]]></punumber>
##     <pubtype><![CDATA[Journals & Magazines]]></pubtype>
##     <publisher><![CDATA[IEEE]]></publisher>
##     <volume><![CDATA[18]]></volume>
##     <issue><![CDATA[9]]></issue>
##     <py><![CDATA[2014]]></py>
##     <spage><![CDATA[1633]]></spage>
##     <epage><![CDATA[1636]]></epage>
##     <abstract><![CDATA[To evaluate 4G cellular systems, a system-level simulation (SLS) with the ITU-R channel model is often used. For the evaluation of the next generation communication systems, a SLS including a device-to-device (D2D) system is needed. However, as the number of multi-paths in a D2D link is squared compared to the link between a base station and a mobile station, the simulation run-time has to increase greatly. In this paper, we propose the ITU-R channel model extension to the D2D channel and the Doppler response-based fast fading channel generation as a realization method for the D2D channel. The proposed method can significantly reduce the simulation run-time.]]></abstract>
##     <issn><![CDATA[1089-7798]]></issn>
##     <htmlFlag><![CDATA[1]]></htmlFlag>
##     <arnumber><![CDATA[6868206]]></arnumber>
##     <doi><![CDATA[10.1109/LCOMM.2014.2344053]]></doi>
##     <publicationId><![CDATA[6868206]]></publicationId>
##     <mdurl><![CDATA[http://ieeexplore.ieee.org/xpl/articleDetails.jsp?tp=&arnumber=6868206&contentType=Journals+%26+Magazines]]></mdurl>
##     <pdf><![CDATA[http://ieeexplore.ieee.org/stamp/stamp.jsp?arnumber=6868206]]></pdf>
##   </document>
##   <document>
##     <rank>441</rank>
##     <title><![CDATA[Random Bit Generator Using Delayed Self-Difference of Filtered Amplified Spontaneous Emission]]></title>
##     <authors><![CDATA[Lei Li;  Anbang Wang;  Pu Li;  Hang Xu;  Longsheng Wang;  Yuncai Wang]]></authors>
##     <affiliations><![CDATA[Key Lab. of Adv. Transducers & Intell. Control Syst., Taiyuan Univ. of Technol., Taiyuan, China]]></affiliations>
##     <controlledterms>
##       <term><![CDATA[delays]]></term>
##       <term><![CDATA[optical fibre communication]]></term>
##       <term><![CDATA[optical pulse generation]]></term>
##       <term><![CDATA[random sequences]]></term>
##       <term><![CDATA[superradiance]]></term>
##     </controlledterms>
##     <thesaurusterms>
##       <term><![CDATA[Bandwidth]]></term>
##       <term><![CDATA[Delays]]></term>
##       <term><![CDATA[Generators]]></term>
##       <term><![CDATA[NIST]]></term>
##       <term><![CDATA[Optical fiber filters]]></term>
##       <term><![CDATA[Random sequences]]></term>
##       <term><![CDATA[Superluminescent diodes]]></term>
##     </thesaurusterms>
##     <pubtitle><![CDATA[Photonics Journal, IEEE]]></pubtitle>
##     <punumber><![CDATA[4563994]]></punumber>
##     <pubtype><![CDATA[Journals & Magazines]]></pubtype>
##     <publisher><![CDATA[IEEE]]></publisher>
##     <volume><![CDATA[6]]></volume>
##     <issue><![CDATA[1]]></issue>
##     <py><![CDATA[2014]]></py>
##     <spage><![CDATA[1]]></spage>
##     <epage><![CDATA[9]]></epage>
##     <abstract><![CDATA[We propose a scheme for random bit generation with filtered amplified spontaneous emission (ASE). Using the filtered ASE, we can get larger signal fluctuations than when using directly detected ASE with a photodetector. Utilizing the delayed self-difference technique to improve the symmetry of filtered ASE, we experimentally achieve real-time 2.5-Gbit/s random sequence generation, which matches with the speeds of fiber communication for one-time pad encryption. Experimental results show that the generated random sequences can pass NIST tests as long as the delay time is beyond a threshold value, which is called the minimum delay time (MDT). Further simulation results indicate that there is a certain relationship between the MDT and bandwidth of the ASE signal.]]></abstract>
##     <issn><![CDATA[1943-0655]]></issn>
##     <htmlFlag><![CDATA[1]]></htmlFlag>
##     <arnumber><![CDATA[6737264]]></arnumber>
##     <doi><![CDATA[10.1109/JPHOT.2014.2304555]]></doi>
##     <publicationId><![CDATA[6737264]]></publicationId>
##     <mdurl><![CDATA[http://ieeexplore.ieee.org/xpl/articleDetails.jsp?tp=&arnumber=6737264&contentType=Journals+%26+Magazines]]></mdurl>
##     <pdf><![CDATA[http://ieeexplore.ieee.org/stamp/stamp.jsp?arnumber=6737264]]></pdf>
##   </document>
##   <document>
##     <rank>442</rank>
##     <title><![CDATA[An Accurate Frequency Tracking Method Based on Short Current Detection for Inductive Power Transfer System]]></title>
##     <authors><![CDATA[Xin Dai;  Yue Sun]]></authors>
##     <affiliations><![CDATA[Coll. of Autom., Chongqing Univ., Chongqing, China]]></affiliations>
##     <controlledterms>
##       <term><![CDATA[comparators (circuits)]]></term>
##       <term><![CDATA[inductive power transmission]]></term>
##       <term><![CDATA[zero current switching]]></term>
##       <term><![CDATA[zero voltage switching]]></term>
##     </controlledterms>
##     <pubtitle><![CDATA[Industrial Electronics, IEEE Transactions on]]></pubtitle>
##     <punumber><![CDATA[41]]></punumber>
##     <pubtype><![CDATA[Journals & Magazines]]></pubtype>
##     <publisher><![CDATA[IEEE]]></publisher>
##     <volume><![CDATA[61]]></volume>
##     <issue><![CDATA[2]]></issue>
##     <py><![CDATA[2014]]></py>
##     <spage><![CDATA[776]]></spage>
##     <epage><![CDATA[783]]></epage>
##     <abstract><![CDATA[Frequency drifting is a common problem in an inductive power transfer (IPT) system. Conventional autonomous oscillation method for maintaining soft switching is challenged due to the drawbacks of feedback delay and disturbance, resonant failure, and requirements of additional start-up circuit. A novel frequency tracking method based on short current detection is proposed for IPT applications. In addition, an instantaneous short current detection method utilizing cheap comparator is proposed. Furthermore, a fast and accurate tracking method is proposed to calculate the frequency mismatch and make a correction. The method can realize accurate frequency correction in several oscillation periods. Furthermore, the method is simple and economic for hardware implementation. Finally, the results of the experiment and comparison results verified the frequency tracking method.]]></abstract>
##     <issn><![CDATA[0278-0046]]></issn>
##     <htmlFlag><![CDATA[1]]></htmlFlag>
##     <arnumber><![CDATA[6494623]]></arnumber>
##     <doi><![CDATA[10.1109/TIE.2013.2257149]]></doi>
##     <publicationId><![CDATA[6494623]]></publicationId>
##     <mdurl><![CDATA[http://ieeexplore.ieee.org/xpl/articleDetails.jsp?tp=&arnumber=6494623&contentType=Journals+%26+Magazines]]></mdurl>
##     <pdf><![CDATA[http://ieeexplore.ieee.org/stamp/stamp.jsp?arnumber=6494623]]></pdf>
##   </document>
##   <document>
##     <rank>443</rank>
##     <title><![CDATA[A Tensor-Based Approach for Big Data Representation and Dimensionality Reduction]]></title>
##     <authors><![CDATA[Liwei Kuang;  Fei Hao;  Yang, L.T.;  Man Lin;  Changqing Luo;  Geyong Min]]></authors>
##     <affiliations><![CDATA[Sch. of Comput. Sci. & Technol., Huazhong Univ. of Sci. & Technol., Wuhan, China]]></affiliations>
##     <controlledterms>
##       <term><![CDATA[Big Data]]></term>
##       <term><![CDATA[approximation theory]]></term>
##       <term><![CDATA[computational complexity]]></term>
##       <term><![CDATA[data structures]]></term>
##       <term><![CDATA[database theory]]></term>
##       <term><![CDATA[singular value decomposition]]></term>
##       <term><![CDATA[tensors]]></term>
##     </controlledterms>
##     <thesaurusterms>
##       <term><![CDATA[Approximation methods]]></term>
##       <term><![CDATA[Big data]]></term>
##       <term><![CDATA[Data models]]></term>
##       <term><![CDATA[Large-scale systems]]></term>
##       <term><![CDATA[Tensile stress]]></term>
##       <term><![CDATA[XML]]></term>
##     </thesaurusterms>
##     <pubtitle><![CDATA[Emerging Topics in Computing, IEEE Transactions on]]></pubtitle>
##     <punumber><![CDATA[6245516]]></punumber>
##     <pubtype><![CDATA[Journals & Magazines]]></pubtype>
##     <publisher><![CDATA[IEEE]]></publisher>
##     <volume><![CDATA[2]]></volume>
##     <issue><![CDATA[3]]></issue>
##     <py><![CDATA[2014]]></py>
##     <spage><![CDATA[280]]></spage>
##     <epage><![CDATA[291]]></epage>
##     <abstract><![CDATA[Variety and veracity are two distinct characteristics of large-scale and heterogeneous data. It has been a great challenge to efficiently represent and process big data with a unified scheme. In this paper, a unified tensor model is proposed to represent the unstructured, semistructured, and structured data. With tensor extension operator, various types of data are represented as subtensors and then are merged to a unified tensor. In order to extract the core tensor which is small but contains valuable information, an incremental high order singular value decomposition (IHOSVD) method is presented. By recursively applying the incremental matrix decomposition algorithm, IHOSVD is able to update the orthogonal bases and compute the new core tensor. Analyzes in terms of time complexity, memory usage, and approximation accuracy of the proposed method are provided in this paper. A case study illustrates that approximate data reconstructed from the core set containing 18% elements can guarantee 93% accuracy in general. Theoretical analyzes and experimental results demonstrate that the proposed unified tensor model and IHOSVD method are efficient for big data representation and dimensionality reduction.]]></abstract>
##     <issn><![CDATA[2168-6750]]></issn>
##     <htmlFlag><![CDATA[1]]></htmlFlag>
##     <arnumber><![CDATA[6832490]]></arnumber>
##     <doi><![CDATA[10.1109/TETC.2014.2330516]]></doi>
##     <publicationId><![CDATA[6832490]]></publicationId>
##     <mdurl><![CDATA[http://ieeexplore.ieee.org/xpl/articleDetails.jsp?tp=&arnumber=6832490&contentType=Journals+%26+Magazines]]></mdurl>
##     <pdf><![CDATA[http://ieeexplore.ieee.org/stamp/stamp.jsp?arnumber=6832490]]></pdf>
##   </document>
##   <document>
##     <rank>444</rank>
##     <title><![CDATA[Application of Compressive Sensing to Refractivity Retrieval Using Networked Weather Radars]]></title>
##     <authors><![CDATA[Ozturk, S.;  Tian-You Yu;  Lei Ding;  Palmer, R.D.;  Gasperoni, N.A.]]></authors>
##     <affiliations><![CDATA[Sch. of Electr. & Comput. Eng., Univ. of Oklahoma, Norman, OK, USA]]></affiliations>
##     <controlledterms>
##       <term><![CDATA[atmospheric humidity]]></term>
##       <term><![CDATA[atmospheric measuring apparatus]]></term>
##       <term><![CDATA[atmospheric techniques]]></term>
##       <term><![CDATA[meteorological radar]]></term>
##       <term><![CDATA[remote sensing by radar]]></term>
##     </controlledterms>
##     <pubtitle><![CDATA[Geoscience and Remote Sensing, IEEE Transactions on]]></pubtitle>
##     <punumber><![CDATA[36]]></punumber>
##     <pubtype><![CDATA[Journals & Magazines]]></pubtype>
##     <publisher><![CDATA[IEEE]]></publisher>
##     <volume><![CDATA[52]]></volume>
##     <issue><![CDATA[5]]></issue>
##     <py><![CDATA[2014]]></py>
##     <spage><![CDATA[2799]]></spage>
##     <epage><![CDATA[2809]]></epage>
##     <abstract><![CDATA[Radar-derived refractivity from stationary ground targets can be used as a proxy of near-surface moisture field and has the potential to improve the forecast of convection initiation. Refractivity retrieval was originally developed for a single radar and was recently extended for a network of radars by solving a constrained least squares (CLS) minimization. In practice, the number of high-quality ground returns can be often limited, and consequently, the retrieval problem becomes ill-conditioned. In this paper, an emerging technology of compressive sensing (CS) is proposed to estimate the refractivity field using a network of radars. It has been shown that CS can provide an optimal solution for the underdetermined inverse problem under certain conditions and has been applied to different fields such as magnetic resonance imaging, radar imaging, etc. In this paper, a CS framework is developed to solve the inversion. The feasibility of CS for refractivity retrieval using single and multiple radars is demonstrated using simulations, where the model refractivity fields were obtained from the Advanced Regional Prediction System. The root-mean-squared error was introduced to quantify the performance of the retrieval. The performance of CS was assessed statistically and compared to the CLS estimates for various amounts of measurement errors, numbers of radars, and model refractivity fields. Our preliminary results have shown that CS can consistently provide relatively robust and high-quality estimates of the refractivity field.]]></abstract>
##     <issn><![CDATA[0196-2892]]></issn>
##     <htmlFlag><![CDATA[1]]></htmlFlag>
##     <arnumber><![CDATA[6553149]]></arnumber>
##     <doi><![CDATA[10.1109/TGRS.2013.2266277]]></doi>
##     <publicationId><![CDATA[6553149]]></publicationId>
##     <mdurl><![CDATA[http://ieeexplore.ieee.org/xpl/articleDetails.jsp?tp=&arnumber=6553149&contentType=Journals+%26+Magazines]]></mdurl>
##     <pdf><![CDATA[http://ieeexplore.ieee.org/stamp/stamp.jsp?arnumber=6553149]]></pdf>
##   </document>
##   <document>
##     <rank>445</rank>
##     <title><![CDATA[Multifeature-Based Surround Inhibition Improves Contour Detection in Natural Images]]></title>
##     <authors><![CDATA[Kai-Fu Yang;  Chao-Yi Li;  Yong-Jie Li]]></authors>
##     <affiliations><![CDATA[Sch. of Life Sci. & Technol., Univ. of Electron. Sci. & Technol. of China, Chengdu, China]]></affiliations>
##     <controlledterms>
##       <term><![CDATA[computer vision]]></term>
##       <term><![CDATA[object detection]]></term>
##       <term><![CDATA[physiology]]></term>
##     </controlledterms>
##     <thesaurusterms>
##       <term><![CDATA[Computational modeling]]></term>
##       <term><![CDATA[Feature extraction]]></term>
##       <term><![CDATA[Neurons]]></term>
##       <term><![CDATA[Vectors]]></term>
##       <term><![CDATA[Visual systems]]></term>
##       <term><![CDATA[Visualization]]></term>
##     </thesaurusterms>
##     <pubtitle><![CDATA[Image Processing, IEEE Transactions on]]></pubtitle>
##     <punumber><![CDATA[83]]></punumber>
##     <pubtype><![CDATA[Journals & Magazines]]></pubtype>
##     <publisher><![CDATA[IEEE]]></publisher>
##     <volume><![CDATA[23]]></volume>
##     <issue><![CDATA[12]]></issue>
##     <py><![CDATA[2014]]></py>
##     <spage><![CDATA[5020]]></spage>
##     <epage><![CDATA[5032]]></epage>
##     <abstract><![CDATA[To effectively perform visual tasks like detecting contours, the visual system normally needs to integrate multiple visual features. Sufficient physiological studies have revealed that for a large number of neurons in the primary visual cortex (V1) of monkeys and cats, neuronal responses elicited by the stimuli placed within the classical receptive field (CRF) are substantially modulated, normally inhibited, when difference exists between the CRF and its surround, namely, non-CRF, for various local features. The exquisite sensitivity of V1 neurons to the center-surround stimulus configuration is thought to serve important perceptual functions, including contour detection. In this paper, we propose a biologically motivated model to improve the performance of perceptually salient contour detection. The main contribution is the multifeature-based center-surround framework, in which the surround inhibition weights of individual features, including orientation, luminance, and luminance contrast, are combined according to a scale-guided strategy, and the combined weights are then used to modulate the final surround inhibition of the neurons. The performance was compared with that of single-cue-based models and other existing methods (especially other biologically motivated ones). The results show that combining multiple cues can substantially improve the performance of contour detection compared with the models using single cue. In general, luminance and luminance contrast contribute much more than orientation to the specific task of contour extraction, at least in gray-scale natural images.]]></abstract>
##     <issn><![CDATA[1057-7149]]></issn>
##     <htmlFlag><![CDATA[1]]></htmlFlag>
##     <arnumber><![CDATA[6914612]]></arnumber>
##     <doi><![CDATA[10.1109/TIP.2014.2361210]]></doi>
##     <publicationId><![CDATA[6914612]]></publicationId>
##     <mdurl><![CDATA[http://ieeexplore.ieee.org/xpl/articleDetails.jsp?tp=&arnumber=6914612&contentType=Journals+%26+Magazines]]></mdurl>
##     <pdf><![CDATA[http://ieeexplore.ieee.org/stamp/stamp.jsp?arnumber=6914612]]></pdf>
##   </document>
##   <document>
##     <rank>446</rank>
##     <title><![CDATA[All-Fiber Curvature Sensor Based on an Abrupt Tapered Fiber and a Fabry&#x2013;P&#x00E9;rot Interferometer]]></title>
##     <authors><![CDATA[Cano-Contreras, M.;  Guzman-Chavez, A.D.;  Mata-Chavez, R.I.;  Vargas-Rodriguez, E.;  Jauregui-Vazquez, D.;  Claudio-Gonzalez, D.;  Estudillo-Ayala, J.M.;  Rojas-Laguna, R.;  Huerta-Mascotte, E.]]></authors>
##     <affiliations><![CDATA[Dept. de Estudios Multidisciplinarios, Univ. de Guanajuato, Leon, Mexico]]></affiliations>
##     <controlledterms>
##       <term><![CDATA[Fabry-Perot interferometers]]></term>
##       <term><![CDATA[fibre optic sensors]]></term>
##       <term><![CDATA[micro-optics]]></term>
##       <term><![CDATA[microsensors]]></term>
##     </controlledterms>
##     <thesaurusterms>
##       <term><![CDATA[Fabry-Perot interferometers]]></term>
##       <term><![CDATA[Optical fiber sensors]]></term>
##       <term><![CDATA[Optical fibers]]></term>
##       <term><![CDATA[Optical interferometry]]></term>
##       <term><![CDATA[Sensitivity]]></term>
##     </thesaurusterms>
##     <pubtitle><![CDATA[Photonics Technology Letters, IEEE]]></pubtitle>
##     <punumber><![CDATA[68]]></punumber>
##     <pubtype><![CDATA[Journals & Magazines]]></pubtype>
##     <publisher><![CDATA[IEEE]]></publisher>
##     <volume><![CDATA[26]]></volume>
##     <issue><![CDATA[22]]></issue>
##     <py><![CDATA[2014]]></py>
##     <spage><![CDATA[2213]]></spage>
##     <epage><![CDATA[2216]]></epage>
##     <abstract><![CDATA[In this letter, a highly sensitive curvature sensor arrangement based on an abrupt tapered fiber (ATF) concatenated with an all-fiber micro Fabry-Pe&#x0301;rot interferometer (MFPI) is presented. Here, as the ATF is bent, the MFPI spectral fringes contrast decreases. In addition, the curvature sensitivity is considerably enhanced due to the use of the ATF. Finally, it is shown that with this arrangement, at 1530-nm wavelength, it is possible to detect curvature changes with a sensitivity of 11.27 dB/m<sup>-1</sup> and a curvature resolution of 8.87 &#x00D7; 10<sup>-3</sup> m<sup>-1</sup> within the measurement range of 0 - 3.5 m<sup>-1</sup>.]]></abstract>
##     <issn><![CDATA[1041-1135]]></issn>
##     <htmlFlag><![CDATA[1]]></htmlFlag>
##     <arnumber><![CDATA[6883165]]></arnumber>
##     <doi><![CDATA[10.1109/LPT.2014.2349979]]></doi>
##     <publicationId><![CDATA[6883165]]></publicationId>
##     <mdurl><![CDATA[http://ieeexplore.ieee.org/xpl/articleDetails.jsp?tp=&arnumber=6883165&contentType=Journals+%26+Magazines]]></mdurl>
##     <pdf><![CDATA[http://ieeexplore.ieee.org/stamp/stamp.jsp?arnumber=6883165]]></pdf>
##   </document>
##   <document>
##     <rank>447</rank>
##     <title><![CDATA[Collimation Sensing With Differential Grating and Talbot Interferometry]]></title>
##     <authors><![CDATA[Nan Wang;  Yan Tang;  Wei Jiang;  Wei Yan;  Song Hu]]></authors>
##     <affiliations><![CDATA[State Key Lab. of Opt. Technol. for Microfabrication, Inst. of Opt. & Electron., Chengdu, China]]></affiliations>
##     <controlledterms>
##       <term><![CDATA[Talbot effect]]></term>
##       <term><![CDATA[diffraction gratings]]></term>
##       <term><![CDATA[light interferometry]]></term>
##       <term><![CDATA[moire fringes]]></term>
##       <term><![CDATA[optical collimators]]></term>
##       <term><![CDATA[optical sensors]]></term>
##     </controlledterms>
##     <thesaurusterms>
##       <term><![CDATA[Convergence]]></term>
##       <term><![CDATA[Gratings]]></term>
##       <term><![CDATA[Lenses]]></term>
##       <term><![CDATA[Optical interferometry]]></term>
##       <term><![CDATA[Optical sensors]]></term>
##       <term><![CDATA[Standards]]></term>
##       <term><![CDATA[Testing]]></term>
##     </thesaurusterms>
##     <pubtitle><![CDATA[Photonics Journal, IEEE]]></pubtitle>
##     <punumber><![CDATA[4563994]]></punumber>
##     <pubtype><![CDATA[Journals & Magazines]]></pubtype>
##     <publisher><![CDATA[IEEE]]></publisher>
##     <volume><![CDATA[6]]></volume>
##     <issue><![CDATA[3]]></issue>
##     <py><![CDATA[2014]]></py>
##     <spage><![CDATA[1]]></spage>
##     <epage><![CDATA[10]]></epage>
##     <abstract><![CDATA[The collimation of monochromatic light in optical metrology is of great importance for its determination of the accuracy of physical measurement. This paper proposes a novel collimation testing method with the differential grating (DG) characterized by its high sensitivity. In the process of measurement, the monochromatic light is incident upon a linear grating and is then projected onto a DG to create two sets of linear moire&#x0301; fringes. By analyzing the relationship between the phase of moire&#x0301; fringes and the divergence or convergence angle, the collimation of monochromatic light can be easily measured. Compared with the traditional collimation methods, this method is more accurate with an experimental accuracy of 10<sup>-7</sup> rad , which is in good agreement with the theoretical prediction.]]></abstract>
##     <issn><![CDATA[1943-0655]]></issn>
##     <htmlFlag><![CDATA[1]]></htmlFlag>
##     <arnumber><![CDATA[6799212]]></arnumber>
##     <doi><![CDATA[10.1109/JPHOT.2014.2317678]]></doi>
##     <publicationId><![CDATA[6799212]]></publicationId>
##     <mdurl><![CDATA[http://ieeexplore.ieee.org/xpl/articleDetails.jsp?tp=&arnumber=6799212&contentType=Journals+%26+Magazines]]></mdurl>
##     <pdf><![CDATA[http://ieeexplore.ieee.org/stamp/stamp.jsp?arnumber=6799212]]></pdf>
##   </document>
##   <document>
##     <rank>448</rank>
##     <title><![CDATA[Tunable Microwave Frequency Multiplication by Injection Locking of DFB Laser With a Weakly Phase Modulated Signal]]></title>
##     <authors><![CDATA[Wenrui Wang;  Jinlong Yu;  Bingchen Han;  Ju Wang;  Lingyun Ye;  Enze Yang]]></authors>
##     <affiliations><![CDATA[Sch. of Electr. & Inf. Eng., Tianjin Univ., Tianjin, China]]></affiliations>
##     <controlledterms>
##       <term><![CDATA[distributed feedback lasers]]></term>
##       <term><![CDATA[laser frequency stability]]></term>
##       <term><![CDATA[laser mode locking]]></term>
##       <term><![CDATA[laser noise]]></term>
##       <term><![CDATA[laser tuning]]></term>
##       <term><![CDATA[microwave generation]]></term>
##       <term><![CDATA[microwave photonics]]></term>
##       <term><![CDATA[optical modulation]]></term>
##       <term><![CDATA[phase modulation]]></term>
##       <term><![CDATA[phase noise]]></term>
##       <term><![CDATA[semiconductor lasers]]></term>
##     </controlledterms>
##     <thesaurusterms>
##       <term><![CDATA[Laser mode locking]]></term>
##       <term><![CDATA[Masers]]></term>
##       <term><![CDATA[Microwave filters]]></term>
##       <term><![CDATA[Microwave photonics]]></term>
##       <term><![CDATA[Optical filters]]></term>
##       <term><![CDATA[Phase modulation]]></term>
##       <term><![CDATA[Semiconductor lasers]]></term>
##     </thesaurusterms>
##     <pubtitle><![CDATA[Photonics Journal, IEEE]]></pubtitle>
##     <punumber><![CDATA[4563994]]></punumber>
##     <pubtype><![CDATA[Journals & Magazines]]></pubtype>
##     <publisher><![CDATA[IEEE]]></publisher>
##     <volume><![CDATA[6]]></volume>
##     <issue><![CDATA[2]]></issue>
##     <py><![CDATA[2014]]></py>
##     <spage><![CDATA[1]]></spage>
##     <epage><![CDATA[8]]></epage>
##     <abstract><![CDATA[We have demonstrated in this paper a novel tunable microwave frequency multiplication by injecting a weakly phase-modulated optical signal into a DFB laser diode. Signals with multiple weak sidebands are generated by cross-phase modulation of a continuous wave (CW) with short pulses from mode-locked fiber laser. Then, frequency multiplication is achieved by injection and phase locking a commercially available DFB laser to one of the harmonics of the phase modulated signal. The multiplication factor can be tuned by changing the frequency difference between the CW and the free oscillating wavelength of the DFB laser. The experimental results show that, with an original signal at a repetition rate of 1 GHz, a microwave signal with high spectral purity and stability is generated with a multiplication factor up to 60. The side-mode suppression ratio over 40 dB and phase noise lower than -90 dBc/Hz at 10 kHz are demonstrated over a continuous tuning range from 20 to 40.]]></abstract>
##     <issn><![CDATA[1943-0655]]></issn>
##     <htmlFlag><![CDATA[1]]></htmlFlag>
##     <arnumber><![CDATA[6748869]]></arnumber>
##     <doi><![CDATA[10.1109/JPHOT.2014.2308634]]></doi>
##     <publicationId><![CDATA[6748869]]></publicationId>
##     <mdurl><![CDATA[http://ieeexplore.ieee.org/xpl/articleDetails.jsp?tp=&arnumber=6748869&contentType=Journals+%26+Magazines]]></mdurl>
##     <pdf><![CDATA[http://ieeexplore.ieee.org/stamp/stamp.jsp?arnumber=6748869]]></pdf>
##   </document>
##   <document>
##     <rank>449</rank>
##     <title><![CDATA[Tunable Resonances in the Plasmonic Split-Ring Resonator]]></title>
##     <authors><![CDATA[Jing Chen;  Yudong Li;  Zongqiang Chen;  Jingyang Peng;  Jun Qian;  Jingjun Xu;  Qian Sun]]></authors>
##     <affiliations><![CDATA[MOE Key Lab. of Weak Light Nonlinear Photonics, Nankai Univ., Tianjin, China]]></affiliations>
##     <controlledterms>
##       <term><![CDATA[electric fields]]></term>
##       <term><![CDATA[integrated optics]]></term>
##       <term><![CDATA[nanophotonics]]></term>
##       <term><![CDATA[optical resonators]]></term>
##       <term><![CDATA[optical tuning]]></term>
##       <term><![CDATA[plasmonics]]></term>
##     </controlledterms>
##     <thesaurusterms>
##       <term><![CDATA[Electric fields]]></term>
##       <term><![CDATA[Optical filters]]></term>
##       <term><![CDATA[Optical ring resonators]]></term>
##       <term><![CDATA[Optical waveguides]]></term>
##       <term><![CDATA[Plasmons]]></term>
##       <term><![CDATA[Resonator filters]]></term>
##     </thesaurusterms>
##     <pubtitle><![CDATA[Photonics Journal, IEEE]]></pubtitle>
##     <punumber><![CDATA[4563994]]></punumber>
##     <pubtype><![CDATA[Journals & Magazines]]></pubtype>
##     <publisher><![CDATA[IEEE]]></publisher>
##     <volume><![CDATA[6]]></volume>
##     <issue><![CDATA[3]]></issue>
##     <py><![CDATA[2014]]></py>
##     <spage><![CDATA[1]]></spage>
##     <epage><![CDATA[6]]></epage>
##     <abstract><![CDATA[A nanoscale resonator composed of two metal-insulator-metal (MIM) waveguides and a split ring is investigated numerically. The multipolar plasmonic resonance modes can be excited, weakened, or even cut off by adjusting the split angle. These novel phenomena are due to the electric polarization in the split area. Odd modes exhibit when the electric field is polarized perpendicular to the split. The resonator acts as a LC circuit for the electric field polarized parallel to the split, in which even modes are excited. The capacitance diminished when the split depth is increased, and the resonance wavelengths of even modes exhibit blue shift. Our results imply an extensive potential for tunable multichannel filters and biosensor devices in integrated nano-optics.]]></abstract>
##     <issn><![CDATA[1943-0655]]></issn>
##     <htmlFlag><![CDATA[1]]></htmlFlag>
##     <arnumber><![CDATA[6814796]]></arnumber>
##     <doi><![CDATA[10.1109/JPHOT.2014.2323294]]></doi>
##     <publicationId><![CDATA[6814796]]></publicationId>
##     <mdurl><![CDATA[http://ieeexplore.ieee.org/xpl/articleDetails.jsp?tp=&arnumber=6814796&contentType=Journals+%26+Magazines]]></mdurl>
##     <pdf><![CDATA[http://ieeexplore.ieee.org/stamp/stamp.jsp?arnumber=6814796]]></pdf>
##   </document>
##   <document>
##     <rank>450</rank>
##     <title><![CDATA[Pseudo-Marginal Bayesian Inference for Gaussian Processes]]></title>
##     <authors><![CDATA[Filippone, M.;  Girolami, M.]]></authors>
##     <affiliations><![CDATA[Sch. of Comput. Sci., Univ. of Glasgow, Glasgow, UK]]></affiliations>
##     <controlledterms>
##       <term><![CDATA[Gaussian processes]]></term>
##       <term><![CDATA[Markov processes]]></term>
##       <term><![CDATA[Monte Carlo methods]]></term>
##       <term><![CDATA[belief networks]]></term>
##       <term><![CDATA[inference mechanisms]]></term>
##       <term><![CDATA[pattern classification]]></term>
##     </controlledterms>
##     <thesaurusterms>
##       <term><![CDATA[Approximation methods]]></term>
##       <term><![CDATA[Bayes methods]]></term>
##       <term><![CDATA[Data models]]></term>
##       <term><![CDATA[Gaussian processes]]></term>
##       <term><![CDATA[Monte Carlo methods]]></term>
##       <term><![CDATA[Predictive models]]></term>
##       <term><![CDATA[Uncertainty]]></term>
##     </thesaurusterms>
##     <pubtitle><![CDATA[Pattern Analysis and Machine Intelligence, IEEE Transactions on]]></pubtitle>
##     <punumber><![CDATA[34]]></punumber>
##     <pubtype><![CDATA[Journals & Magazines]]></pubtype>
##     <publisher><![CDATA[IEEE]]></publisher>
##     <volume><![CDATA[36]]></volume>
##     <issue><![CDATA[11]]></issue>
##     <py><![CDATA[2014]]></py>
##     <spage><![CDATA[2214]]></spage>
##     <epage><![CDATA[2226]]></epage>
##     <abstract><![CDATA[The main challenges that arise when adopting Gaussian process priors in probabilistic modeling are how to carry out exact Bayesian inference and how to account for uncertainty on model parameters when making model-based predictions on out-of-sample data. Using probit regression as an illustrative working example, this paper presents a general and effective methodology based on the pseudo-marginal approach to Markov chain Monte Carlo that efficiently addresses both of these issues. The results presented in this paper show improvements over existing sampling methods to simulate from the posterior distribution over the parameters defining the covariance function of the Gaussian Process prior. This is particularly important as it offers a powerful tool to carry out full Bayesian inference of Gaussian Process based hierarchic statistical models in general. The results also demonstrate that Monte Carlo based integration of all model parameters is actually feasible in this class of models providing a superior quantification of uncertainty in predictions. Extensive comparisons with respect to state-of-the-art probabilistic classifiers confirm this assertion.]]></abstract>
##     <issn><![CDATA[0162-8828]]></issn>
##     <htmlFlag><![CDATA[1]]></htmlFlag>
##     <arnumber><![CDATA[6786502]]></arnumber>
##     <doi><![CDATA[10.1109/TPAMI.2014.2316530]]></doi>
##     <publicationId><![CDATA[6786502]]></publicationId>
##     <mdurl><![CDATA[http://ieeexplore.ieee.org/xpl/articleDetails.jsp?tp=&arnumber=6786502&contentType=Journals+%26+Magazines]]></mdurl>
##     <pdf><![CDATA[http://ieeexplore.ieee.org/stamp/stamp.jsp?arnumber=6786502]]></pdf>
##   </document>
##   <document>
##     <rank>451</rank>
##     <title><![CDATA[Optimal Peer-to-Peer Schedulingfor Mobile Wireless Networkswith Redundantly Distributed Data]]></title>
##     <authors><![CDATA[Neely, M.J.]]></authors>
##     <affiliations><![CDATA[Electr. Eng. Dept., Univ. of Southern California, Los Angeles, CA, USA]]></affiliations>
##     <controlledterms>
##       <term><![CDATA[mobile radio]]></term>
##       <term><![CDATA[optimisation]]></term>
##       <term><![CDATA[peer-to-peer computing]]></term>
##       <term><![CDATA[radio networks]]></term>
##       <term><![CDATA[scheduling]]></term>
##     </controlledterms>
##     <thesaurusterms>
##       <term><![CDATA[Algorithm design and analysis]]></term>
##       <term><![CDATA[Heuristic algorithms]]></term>
##       <term><![CDATA[Mobile communication]]></term>
##       <term><![CDATA[Mobile computing]]></term>
##       <term><![CDATA[Optimization]]></term>
##       <term><![CDATA[Topology]]></term>
##       <term><![CDATA[Wireless communication]]></term>
##     </thesaurusterms>
##     <pubtitle><![CDATA[Mobile Computing, IEEE Transactions on]]></pubtitle>
##     <punumber><![CDATA[7755]]></punumber>
##     <pubtype><![CDATA[Journals & Magazines]]></pubtype>
##     <publisher><![CDATA[IEEE]]></publisher>
##     <volume><![CDATA[13]]></volume>
##     <issue><![CDATA[9]]></issue>
##     <py><![CDATA[2014]]></py>
##     <spage><![CDATA[2086]]></spage>
##     <epage><![CDATA[2099]]></epage>
##     <abstract><![CDATA[This paper considers peer-to-peer scheduling for a network with multiple wireless devices. A subset of the devices are mobile users that desire specific files. Each user may already have certain popular files in its cache. The remaining devices are access points that typically have a larger set of files. Users can download packets of their requested file from an access point or from another user. A dynamic algorithm that opportunistically grabs packets from current neighbors is developed. Under a simple model where each user desires a single file with infinite length, the algorithm is shown to optimize utility while incentivizing participation. The algorithm extends as an efficient heuristic in more general cases with finite file sizes and random active and idle periods. Example simulations demonstrate the dramatic throughput gains enabled by wireless peering.]]></abstract>
##     <issn><![CDATA[1536-1233]]></issn>
##     <htmlFlag><![CDATA[1]]></htmlFlag>
##     <arnumber><![CDATA[6427753]]></arnumber>
##     <doi><![CDATA[10.1109/TMC.2013.21]]></doi>
##     <publicationId><![CDATA[6427753]]></publicationId>
##     <mdurl><![CDATA[http://ieeexplore.ieee.org/xpl/articleDetails.jsp?tp=&arnumber=6427753&contentType=Journals+%26+Magazines]]></mdurl>
##     <pdf><![CDATA[http://ieeexplore.ieee.org/stamp/stamp.jsp?arnumber=6427753]]></pdf>
##   </document>
##   <document>
##     <rank>452</rank>
##     <title><![CDATA[Photonic Crystal Fiber Based Wavelength-Tunable Optical Parametric Amplifier and Picosecond Pulse Generation]]></title>
##     <authors><![CDATA[Lei Zhang;  Sigang Yang;  Xiaojian Wang;  Doudou Gou;  Wei Chen;  Wenyong Luo;  Hongwei Chen;  Minghua Chen;  Shizhong Xie]]></authors>
##     <affiliations><![CDATA[Dept. of Electron. Eng., Tsinghua Univ., Beijing, China]]></affiliations>
##     <controlledterms>
##       <term><![CDATA[holey fibres]]></term>
##       <term><![CDATA[laser mode locking]]></term>
##       <term><![CDATA[laser tuning]]></term>
##       <term><![CDATA[optical fibre amplifiers]]></term>
##       <term><![CDATA[optical fibre dispersion]]></term>
##       <term><![CDATA[optical parametric amplifiers]]></term>
##       <term><![CDATA[optical pulse generation]]></term>
##       <term><![CDATA[optical pumping]]></term>
##       <term><![CDATA[photonic crystals]]></term>
##       <term><![CDATA[ytterbium]]></term>
##     </controlledterms>
##     <thesaurusterms>
##       <term><![CDATA[Nonlinear optics]]></term>
##       <term><![CDATA[Optical amplifiers]]></term>
##       <term><![CDATA[Optical fiber amplifiers]]></term>
##       <term><![CDATA[Optical fiber dispersion]]></term>
##     </thesaurusterms>
##     <pubtitle><![CDATA[Photonics Journal, IEEE]]></pubtitle>
##     <punumber><![CDATA[4563994]]></punumber>
##     <pubtype><![CDATA[Journals & Magazines]]></pubtype>
##     <publisher><![CDATA[IEEE]]></publisher>
##     <volume><![CDATA[6]]></volume>
##     <issue><![CDATA[5]]></issue>
##     <py><![CDATA[2014]]></py>
##     <spage><![CDATA[1]]></spage>
##     <epage><![CDATA[8]]></epage>
##     <abstract><![CDATA[We report a fiber optical parametric amplifier (FOPA) based on a photonic crystal fiber (PCF) with optimized dispersion and nonlinear properties pumped by a mode-locked ytterbium-doped fiber laser. Wavelength-tunable parametric gain bands can be obtained by adjusting the pump wavelength from the anomalous to the normal dispersion regime. By utilizing the PCF-based FOPA, weak signals located in the parametric gain bands can be amplified. Wavelength-tunable picosecond pulses can be generated at the signal and idler wavelengths. A 58-dB maximum gain and a wide gain bandwidth from 999 to 1139 nm have been achieved by pumping near the zero-dispersion wavelength. When the FOPA is pumped at 1062.5 nm in the normal dispersion regime of the PCF, the 974-nm weak continuous-wave signal is significantly amplified, and the idler pulse is generated at 1168 nm.]]></abstract>
##     <issn><![CDATA[1943-0655]]></issn>
##     <htmlFlag><![CDATA[1]]></htmlFlag>
##     <arnumber><![CDATA[6902851]]></arnumber>
##     <doi><![CDATA[10.1109/JPHOT.2014.2353616]]></doi>
##     <publicationId><![CDATA[6902851]]></publicationId>
##     <mdurl><![CDATA[http://ieeexplore.ieee.org/xpl/articleDetails.jsp?tp=&arnumber=6902851&contentType=Journals+%26+Magazines]]></mdurl>
##     <pdf><![CDATA[http://ieeexplore.ieee.org/stamp/stamp.jsp?arnumber=6902851]]></pdf>
##   </document>
##   <document>
##     <rank>453</rank>
##     <title><![CDATA[W-Band OFDM for Radio-over-Fiber Direct-Detection Link Enabled by Frequency Nonupling Optical Up-Conversion]]></title>
##     <authors><![CDATA[Alavi, S.E.;  Amiri, I.S.;  Khalily, M.;  Fisal, N.;  Supa'at, A.S.M.;  Ahmad, H.;  Idrus, S.M.]]></authors>
##     <affiliations><![CDATA[Fac. of Electr. Eng., Univ. Teknol. Malaysia (UTM), Skudai, Malaysia]]></affiliations>
##     <controlledterms>
##       <term><![CDATA[OFDM modulation]]></term>
##       <term><![CDATA[error statistics]]></term>
##       <term><![CDATA[optical communication equipment]]></term>
##       <term><![CDATA[optical frequency conversion]]></term>
##       <term><![CDATA[optical modulation]]></term>
##       <term><![CDATA[quadrature amplitude modulation]]></term>
##       <term><![CDATA[radio-over-fibre]]></term>
##     </controlledterms>
##     <thesaurusterms>
##       <term><![CDATA[Bit error rate]]></term>
##       <term><![CDATA[OFDM]]></term>
##       <term><![CDATA[Optical fibers]]></term>
##       <term><![CDATA[Optical modulation]]></term>
##       <term><![CDATA[Optical transmitters]]></term>
##       <term><![CDATA[Wireless communication]]></term>
##     </thesaurusterms>
##     <pubtitle><![CDATA[Photonics Journal, IEEE]]></pubtitle>
##     <punumber><![CDATA[4563994]]></punumber>
##     <pubtype><![CDATA[Journals & Magazines]]></pubtype>
##     <publisher><![CDATA[IEEE]]></publisher>
##     <volume><![CDATA[6]]></volume>
##     <issue><![CDATA[6]]></issue>
##     <py><![CDATA[2014]]></py>
##     <spage><![CDATA[1]]></spage>
##     <epage><![CDATA[7]]></epage>
##     <abstract><![CDATA[A system involving W-band (75-110 GHz) optical millimeter (mm)-wave generation using the external optical modulator (EOM) in a radio-over-fiber (RoF) link is presented for satisfying the requirements for multi-gigabit-per-second data rates. A 90-GHz mm-wave signal was generated by a nonupling (nine times) increase in only a 10-GHz local oscillator by biasing the EOM at its zero level and choosing an appropriate modulation index. To achieve a fast transmission speed wirelessly, high spectral efficiency (SE), and better transmission performance, orthogonal frequency-division multiplexing (OFDM) is used. The bit error rate (BER) and error vector magnitude (EVM) of the system were measured for three different fiber lengths and for a wireless distance of 1-5 m. The results show that the system with the SE of ~4 (b/s)/Hz and 16-ary quadrature amplitude modulation (QAM) 40-GB/s OFDM signals can be received by the end user with BER less than 3.8 &#x00D7; 10<sup>-3</sup> and EVM less than 25% over a 50-km optical fiber and 3-m wireless link.]]></abstract>
##     <issn><![CDATA[1943-0655]]></issn>
##     <htmlFlag><![CDATA[1]]></htmlFlag>
##     <arnumber><![CDATA[6945247]]></arnumber>
##     <doi><![CDATA[10.1109/JPHOT.2014.2366166]]></doi>
##     <publicationId><![CDATA[6945247]]></publicationId>
##     <mdurl><![CDATA[http://ieeexplore.ieee.org/xpl/articleDetails.jsp?tp=&arnumber=6945247&contentType=Journals+%26+Magazines]]></mdurl>
##     <pdf><![CDATA[http://ieeexplore.ieee.org/stamp/stamp.jsp?arnumber=6945247]]></pdf>
##   </document>
##   <document>
##     <rank>454</rank>
##     <title><![CDATA[Integrated Wireless Backhaul Over Optical Access Networks]]></title>
##     <authors><![CDATA[Mitchell, J.E.]]></authors>
##     <affiliations><![CDATA[Dept. of Electron. & Electr. Eng., Univ. Coll. London, London, UK]]></affiliations>
##     <controlledterms>
##       <term><![CDATA[optical fibre networks]]></term>
##       <term><![CDATA[radio access networks]]></term>
##       <term><![CDATA[radio-over-fibre]]></term>
##     </controlledterms>
##     <thesaurusterms>
##       <term><![CDATA[Bandwidth]]></term>
##       <term><![CDATA[Computer architecture]]></term>
##       <term><![CDATA[Optical fiber networks]]></term>
##       <term><![CDATA[Optical fibers]]></term>
##       <term><![CDATA[Radio access networks]]></term>
##       <term><![CDATA[Wireless communication]]></term>
##     </thesaurusterms>
##     <pubtitle><![CDATA[Lightwave Technology, Journal of]]></pubtitle>
##     <punumber><![CDATA[50]]></punumber>
##     <pubtype><![CDATA[Journals & Magazines]]></pubtype>
##     <publisher><![CDATA[IEEE]]></publisher>
##     <volume><![CDATA[32]]></volume>
##     <issue><![CDATA[20]]></issue>
##     <py><![CDATA[2014]]></py>
##     <spage><![CDATA[3373]]></spage>
##     <epage><![CDATA[3382]]></epage>
##     <abstract><![CDATA[Recent technological advances and deployments are creating a new landscape in access networks, with an integration of wireless and fiber technologies a key supporting technology. In the past, a separation between those with fiber in the access networks and those with wireless networks, the relatively low data-rate requirements of backhaul and the relatively large cell sites, have all combined to keep fiber deployment low in wireless backhaul. As fiber has penetrated the access network and the latest wireless standards have demanded smaller, higher bandwidth cells, fiber connectivity has become key. Choices remain as to where the demarcation between key elements should be in the network and whether fiber should be used as just a high data-rate backhaul path or if a transition to radio-over-fiber techniques can afford benefits. This paper will explore the network options available in particular those demonstrated in recent European Union (EU) projects, how they can be integrated with existing access networks and how techniques such as radio-over-fiber can be deployed to offer increased functionality.]]></abstract>
##     <issn><![CDATA[0733-8724]]></issn>
##     <htmlFlag><![CDATA[1]]></htmlFlag>
##     <arnumber><![CDATA[6811153]]></arnumber>
##     <doi><![CDATA[10.1109/JLT.2014.2321774]]></doi>
##     <publicationId><![CDATA[6811153]]></publicationId>
##     <mdurl><![CDATA[http://ieeexplore.ieee.org/xpl/articleDetails.jsp?tp=&arnumber=6811153&contentType=Journals+%26+Magazines]]></mdurl>
##     <pdf><![CDATA[http://ieeexplore.ieee.org/stamp/stamp.jsp?arnumber=6811153]]></pdf>
##   </document>
##   <document>
##     <rank>455</rank>
##     <title><![CDATA[Information Content in the Oxygen <inline-formula> <img src="/images/tex/18899.gif" alt="A"> </inline-formula>-Band for the Retrieval of Macrophysical Cloud Parameters]]></title>
##     <authors><![CDATA[Schuessler, O.;  Rodriguez, D.G.L.;  Doicu, A.;  Spurr, R.]]></authors>
##     <affiliations><![CDATA[Inst. fur Methodik der Fernerkundung (IMF), Deutsches Zentrum fur Luft- und Raumfahrt (DLR), Wessling, Germany]]></affiliations>
##     <controlledterms>
##       <term><![CDATA[aerosols]]></term>
##       <term><![CDATA[atmospheric optics]]></term>
##       <term><![CDATA[atmospheric techniques]]></term>
##       <term><![CDATA[clouds]]></term>
##       <term><![CDATA[radiative transfer]]></term>
##       <term><![CDATA[remote sensing]]></term>
##     </controlledterms>
##     <thesaurusterms>
##       <term><![CDATA[Atmospheric modeling]]></term>
##       <term><![CDATA[Clouds]]></term>
##       <term><![CDATA[Gases]]></term>
##       <term><![CDATA[Optical scattering]]></term>
##       <term><![CDATA[Optical sensors]]></term>
##       <term><![CDATA[Satellites]]></term>
##       <term><![CDATA[Vectors]]></term>
##     </thesaurusterms>
##     <pubtitle><![CDATA[Geoscience and Remote Sensing, IEEE Transactions on]]></pubtitle>
##     <punumber><![CDATA[36]]></punumber>
##     <pubtype><![CDATA[Journals & Magazines]]></pubtype>
##     <publisher><![CDATA[IEEE]]></publisher>
##     <volume><![CDATA[52]]></volume>
##     <issue><![CDATA[6]]></issue>
##     <py><![CDATA[2014]]></py>
##     <spage><![CDATA[3246]]></spage>
##     <epage><![CDATA[3255]]></epage>
##     <abstract><![CDATA[Current and future satellite sensors provide measurements in and around the oxygen A-band on a global basis. These data are commonly used for the determination of cloud and aerosol properties. In this paper, we assess the information content in the oxygen A-band for the retrieval of macrophysical cloud parameters using precise radiative transfer simulations covering a wide range of geophysical conditions in conjunction with advance inversion techniques. The information content of the signal with respect to the retrieved parameters is analyzed in a stochastic framework using two common criteria: the degrees of freedom for a signal and the Shannon information content. It is found that oxygen A-band measurements with moderate spectral resolution (0.2 nm) provide two pieces of independent information that allow the accurate retrieval of cloud-top height together with either cloud optical thickness or cloud fraction. Additionally, our results confirm previous studies indicating that the retrieval of cloud geometrical thickness (CGT) from single-angle measurements is not reliable in this spectral region. Finally, a sensitivity study shows that the retrieval of macrophysical cloud parameters is slightly sensitive to the uncertainty in the CGT and very sensitive to the uncertainty in the surface albedo.]]></abstract>
##     <issn><![CDATA[0196-2892]]></issn>
##     <htmlFlag><![CDATA[1]]></htmlFlag>
##     <arnumber><![CDATA[6563202]]></arnumber>
##     <doi><![CDATA[10.1109/TGRS.2013.2271986]]></doi>
##     <publicationId><![CDATA[6563202]]></publicationId>
##     <mdurl><![CDATA[http://ieeexplore.ieee.org/xpl/articleDetails.jsp?tp=&arnumber=6563202&contentType=Journals+%26+Magazines]]></mdurl>
##     <pdf><![CDATA[http://ieeexplore.ieee.org/stamp/stamp.jsp?arnumber=6563202]]></pdf>
##   </document>
##   <document>
##     <rank>456</rank>
##     <title><![CDATA[Wavelength Dependence of the Sensitivity of All-Fiber Refractometers Based on the Singlemode&#x2013;Multimode&#x2013;Singlemode Structure]]></title>
##     <authors><![CDATA[Yaofei Chen;  Qun Han;  Tiegen Liu;  Hai Xiao]]></authors>
##     <affiliations><![CDATA[Coll. of Precision Instrum. & Opto-Electron. Eng., Tianjin Univ., Tianjin, China]]></affiliations>
##     <controlledterms>
##       <term><![CDATA[optical fibre testing]]></term>
##       <term><![CDATA[refractive index]]></term>
##       <term><![CDATA[refractometers]]></term>
##     </controlledterms>
##     <thesaurusterms>
##       <term><![CDATA[Ethanol]]></term>
##       <term><![CDATA[Numerical simulation]]></term>
##       <term><![CDATA[Optical fiber sensors]]></term>
##       <term><![CDATA[Sensitivity]]></term>
##       <term><![CDATA[Wavelength measurement]]></term>
##     </thesaurusterms>
##     <pubtitle><![CDATA[Photonics Journal, IEEE]]></pubtitle>
##     <punumber><![CDATA[4563994]]></punumber>
##     <pubtype><![CDATA[Journals & Magazines]]></pubtype>
##     <publisher><![CDATA[IEEE]]></publisher>
##     <volume><![CDATA[6]]></volume>
##     <issue><![CDATA[4]]></issue>
##     <py><![CDATA[2014]]></py>
##     <spage><![CDATA[1]]></spage>
##     <epage><![CDATA[7]]></epage>
##     <abstract><![CDATA[In this paper, the wavelength dependence of the sensitivity of all-fiber refractometers based on the singlemode-multimode-singlemode fiber structure, in which a piece of no-core fiber (NCF) is used to serve as the multimode section, is investigated via numerical simulations and experiments. We found that the sensitivity of a refractometer is linearly proportional to the wavelength of the dip/peak in the transmission spectrum that is chosen to measure the refractive index (RI) change. Because the wavelength shift is larger for a dip/peak at a longer wavelength than at a shorter wavelength, a linear spectral chirp will occur in the transmission spectrum when the surrounding RI changes. We also found that the sensitivity significantly increases with the decrease in the diameter of the NCF and is independent of its length. The experimental results agree well with the numerical predictions.]]></abstract>
##     <issn><![CDATA[1943-0655]]></issn>
##     <htmlFlag><![CDATA[1]]></htmlFlag>
##     <arnumber><![CDATA[6867289]]></arnumber>
##     <doi><![CDATA[10.1109/JPHOT.2014.2344004]]></doi>
##     <publicationId><![CDATA[6867289]]></publicationId>
##     <mdurl><![CDATA[http://ieeexplore.ieee.org/xpl/articleDetails.jsp?tp=&arnumber=6867289&contentType=Journals+%26+Magazines]]></mdurl>
##     <pdf><![CDATA[http://ieeexplore.ieee.org/stamp/stamp.jsp?arnumber=6867289]]></pdf>
##   </document>
##   <document>
##     <rank>457</rank>
##     <title><![CDATA[Pixelated Detector With Photon Address Event Driven Time Stamping and Correlation]]></title>
##     <authors><![CDATA[Williams, G.M.;  Rhee, J.;  Lee, A.;  Kevan, S.D.]]></authors>
##     <affiliations><![CDATA[Inc., Voxtel, Beaverton, OR, USA]]></affiliations>
##     <controlledterms>
##       <term><![CDATA[X-ray spectroscopy]]></term>
##       <term><![CDATA[nuclear electronics]]></term>
##       <term><![CDATA[position sensitive particle detectors]]></term>
##       <term><![CDATA[readout electronics]]></term>
##       <term><![CDATA[semiconductor counters]]></term>
##     </controlledterms>
##     <thesaurusterms>
##       <term><![CDATA[Arrays]]></term>
##       <term><![CDATA[Correlation]]></term>
##       <term><![CDATA[Detectors]]></term>
##       <term><![CDATA[Image quality]]></term>
##       <term><![CDATA[Noise]]></term>
##       <term><![CDATA[Photonics]]></term>
##       <term><![CDATA[Scattering]]></term>
##     </thesaurusterms>
##     <pubtitle><![CDATA[Nuclear Science, IEEE Transactions on]]></pubtitle>
##     <punumber><![CDATA[23]]></punumber>
##     <pubtype><![CDATA[Journals & Magazines]]></pubtype>
##     <publisher><![CDATA[IEEE]]></publisher>
##     <volume><![CDATA[61]]></volume>
##     <issue><![CDATA[4]]></issue>
##     <part><![CDATA[3]]></part>
##     <py><![CDATA[2014]]></py>
##     <spage><![CDATA[2323]]></spage>
##     <epage><![CDATA[2332]]></epage>
##     <abstract><![CDATA[We present the design, manufacture, and test results of an asynchronous event-driven address time-stamped (EDATS) pixelated array detector, operational over correlation time spans ranging from less than 10<sup>-6</sup> to greater than 10<sup>4</sup> s. The pixelated EDATS detector was designed to measure the equilibrium density fluctuations on a nanometer-length scale using small angle X-ray scattering in X-ray photon correlation spectroscopy (XPCS) experiments. The detector sensor chip assembly (SCA) includes a custom readout integrated circuit (ROIC), hybridized to a silicon photodiode array, optimized for 500 to 2000 eV X-ray photons. The detector is shown to be capable of handling X-ray photon event rates of 100 million X-ray events per second, with less than 85 ns timing jitter per event.]]></abstract>
##     <issn><![CDATA[0018-9499]]></issn>
##     <htmlFlag><![CDATA[1]]></htmlFlag>
##     <arnumber><![CDATA[6869055]]></arnumber>
##     <doi><![CDATA[10.1109/TNS.2014.2327513]]></doi>
##     <publicationId><![CDATA[6869055]]></publicationId>
##     <mdurl><![CDATA[http://ieeexplore.ieee.org/xpl/articleDetails.jsp?tp=&arnumber=6869055&contentType=Journals+%26+Magazines]]></mdurl>
##     <pdf><![CDATA[http://ieeexplore.ieee.org/stamp/stamp.jsp?arnumber=6869055]]></pdf>
##   </document>
##   <document>
##     <rank>458</rank>
##     <title><![CDATA[Implementation of Chip-Level Optical Interconnect With Laser and Photodetector Using SOI-Based 3-D Guided-Wave Path]]></title>
##     <authors><![CDATA[Po-Kuan Shen;  Chin-Ta Chen;  Chia-Hao Chang;  Chien-Yu Chiu;  Sheng-Long Li;  Chia-Chi Chang;  Mount-Learn Wu]]></authors>
##     <affiliations><![CDATA[Dept. of Opt. & Photonics, Nat. Central Univ., Jhongli, Taiwan]]></affiliations>
##     <controlledterms>
##       <term><![CDATA[amplifiers]]></term>
##       <term><![CDATA[driver circuits]]></term>
##       <term><![CDATA[elemental semiconductors]]></term>
##       <term><![CDATA[integrated optoelectronics]]></term>
##       <term><![CDATA[laser cavity resonators]]></term>
##       <term><![CDATA[optical interconnections]]></term>
##       <term><![CDATA[optical waveguides]]></term>
##       <term><![CDATA[photodetectors]]></term>
##       <term><![CDATA[silicon]]></term>
##       <term><![CDATA[silicon-on-insulator]]></term>
##       <term><![CDATA[surface emitting lasers]]></term>
##     </controlledterms>
##     <thesaurusterms>
##       <term><![CDATA[Laser beams]]></term>
##       <term><![CDATA[Optical coupling]]></term>
##       <term><![CDATA[Optical interconnections]]></term>
##       <term><![CDATA[Optical waveguides]]></term>
##       <term><![CDATA[Silicon]]></term>
##       <term><![CDATA[Substrates]]></term>
##       <term><![CDATA[Waveguide lasers]]></term>
##     </thesaurusterms>
##     <pubtitle><![CDATA[Photonics Journal, IEEE]]></pubtitle>
##     <punumber><![CDATA[4563994]]></punumber>
##     <pubtype><![CDATA[Journals & Magazines]]></pubtype>
##     <publisher><![CDATA[IEEE]]></publisher>
##     <volume><![CDATA[6]]></volume>
##     <issue><![CDATA[6]]></issue>
##     <py><![CDATA[2014]]></py>
##     <spage><![CDATA[1]]></spage>
##     <epage><![CDATA[10]]></epage>
##     <abstract><![CDATA[A chip-level optical interconnect module combined with a vertical-cavity surface-emitting laser (VCSEL) chip, a photodetector (PD) chip, a driver integrated circuit (IC), and an amplifier IC on a silicon-on-insulator (SOI) substrate with 3-D guided-wave paths is experimentally demonstrated. Such an optical interconnect is developed for the signal connection in multicore processors or memory-to-processor interfaces. The 3-D guided-wave path, consisting of silicon-based 45&#x00B0; microreflectors and trapezoidal waveguides, is used to connect the optical signal between transmitter and receiver. In this paper, the VCSEL and PIN PD chips are flip-chip integrated on a SOI substrate to achieve complete chip-level optical interconnects. Due to the unique 3-D guided-wave path design, a higher laser-to-PD optical coupling efficiency of -2.19 dB and a larger alignment tolerance of &#x00B1;10&#x03BC;m for the VCSEL/PD assembly are achieved. The measured laser-to-PD optical transmission efficiency can reach -2.19 dB, and the maximum optical power and threshold current of VCSEL is 3.27 mW and 1 mA, respectively. To verify the data transmission, the commercial driver IC and amplifier IC are assembled upon the silicon chip, and the error-free data transmission of 10 Gbps can be achieved when the VCSEL is operated at the driving current of 9 mA.]]></abstract>
##     <issn><![CDATA[1943-0655]]></issn>
##     <htmlFlag><![CDATA[1]]></htmlFlag>
##     <arnumber><![CDATA[6960117]]></arnumber>
##     <doi><![CDATA[10.1109/JPHOT.2014.2366165]]></doi>
##     <publicationId><![CDATA[6960117]]></publicationId>
##     <mdurl><![CDATA[http://ieeexplore.ieee.org/xpl/articleDetails.jsp?tp=&arnumber=6960117&contentType=Journals+%26+Magazines]]></mdurl>
##     <pdf><![CDATA[http://ieeexplore.ieee.org/stamp/stamp.jsp?arnumber=6960117]]></pdf>
##   </document>
##   <document>
##     <rank>459</rank>
##     <title><![CDATA[A Trust-Management Toolkit for Smart-Grid Protection Systems]]></title>
##     <authors><![CDATA[Fadul, J.E.;  Hopkinson, K.M.;  Andel, T.R.;  Sheffield, C.A.]]></authors>
##     <affiliations><![CDATA[Dept. of Electr. & Comput. Eng., Air Force Inst. of Technol., Wright-Patterson AFB, OH, USA]]></affiliations>
##     <controlledterms>
##       <term><![CDATA[power system protection]]></term>
##       <term><![CDATA[smart power grids]]></term>
##     </controlledterms>
##     <thesaurusterms>
##       <term><![CDATA[Fault detection]]></term>
##       <term><![CDATA[Junctions]]></term>
##       <term><![CDATA[Monitoring]]></term>
##       <term><![CDATA[Reliability]]></term>
##       <term><![CDATA[Security]]></term>
##       <term><![CDATA[Smart grids]]></term>
##       <term><![CDATA[Topology]]></term>
##     </thesaurusterms>
##     <pubtitle><![CDATA[Power Delivery, IEEE Transactions on]]></pubtitle>
##     <punumber><![CDATA[61]]></punumber>
##     <pubtype><![CDATA[Journals & Magazines]]></pubtype>
##     <publisher><![CDATA[IEEE]]></publisher>
##     <volume><![CDATA[29]]></volume>
##     <issue><![CDATA[4]]></issue>
##     <py><![CDATA[2014]]></py>
##     <spage><![CDATA[1768]]></spage>
##     <epage><![CDATA[1779]]></epage>
##     <abstract><![CDATA[This paper discusses the trust-management toolkit, which is a robust and configurable protection system augmentation, which can successfully function in the presence of an untrusted (malfunctioning) smart grid (i.e., communication-based, protection system nodes). The trust-management toolkit combines reputation-based trust with network-flow algorithms to identify and mitigate faulty smart-grid protection nodes. The toolkit assigns trust values to all protection nodes. Faulty nodes, attributed to component or communication system malfunctions (either intentional or unintentional), are assigned a lower trust value, which indicates a higher risk of failure to mitigate detected faults. The utility of the toolkit is demonstrated through simulations comparing &#x201C;enhanced&#x201D; backup and special protection systems to original &#x201C;unenhanced&#x201D; systems via an analysis of variance analysis. The results show promise for the toolkit in the smart-grid protection system.]]></abstract>
##     <issn><![CDATA[0885-8977]]></issn>
##     <htmlFlag><![CDATA[1]]></htmlFlag>
##     <arnumber><![CDATA[6678325]]></arnumber>
##     <doi><![CDATA[10.1109/TPWRD.2013.2289747]]></doi>
##     <publicationId><![CDATA[6678325]]></publicationId>
##     <mdurl><![CDATA[http://ieeexplore.ieee.org/xpl/articleDetails.jsp?tp=&arnumber=6678325&contentType=Journals+%26+Magazines]]></mdurl>
##     <pdf><![CDATA[http://ieeexplore.ieee.org/stamp/stamp.jsp?arnumber=6678325]]></pdf>
##   </document>
##   <document>
##     <rank>460</rank>
##     <title><![CDATA[High-Performance LSPR Fiber Sensor Based on Nanometal Rings]]></title>
##     <authors><![CDATA[Yue Jing He]]></authors>
##     <affiliations><![CDATA[Dept. of Electron. Eng., Nat. Chin-Yi Univ. of Technol., Taichung, Taiwan]]></affiliations>
##     <controlledterms>
##       <term><![CDATA[eigenvalues and eigenfunctions]]></term>
##       <term><![CDATA[etching]]></term>
##       <term><![CDATA[fibre optic sensors]]></term>
##       <term><![CDATA[nanophotonics]]></term>
##       <term><![CDATA[nanosensors]]></term>
##       <term><![CDATA[optical design techniques]]></term>
##       <term><![CDATA[optical fibre cladding]]></term>
##       <term><![CDATA[surface plasmon resonance]]></term>
##     </controlledterms>
##     <thesaurusterms>
##       <term><![CDATA[Metals]]></term>
##       <term><![CDATA[Optical fiber dispersion]]></term>
##       <term><![CDATA[Optical fiber sensors]]></term>
##       <term><![CDATA[Optical fiber theory]]></term>
##       <term><![CDATA[Optical surface waves]]></term>
##       <term><![CDATA[Refractive index]]></term>
##     </thesaurusterms>
##     <pubtitle><![CDATA[Photonics Journal, IEEE]]></pubtitle>
##     <punumber><![CDATA[4563994]]></punumber>
##     <pubtype><![CDATA[Journals & Magazines]]></pubtype>
##     <publisher><![CDATA[IEEE]]></publisher>
##     <volume><![CDATA[6]]></volume>
##     <issue><![CDATA[2]]></issue>
##     <py><![CDATA[2014]]></py>
##     <spage><![CDATA[1]]></spage>
##     <epage><![CDATA[11]]></epage>
##     <abstract><![CDATA[A novel localized surface plasmon resonance (LSPR) fiber sensor was proposed. This LSPR fiber sensor was primarily constructed by conducting etching of the cladding layer and core layer on a single-mode fiber, followed by plating of 1444 nanometal rings. Sensor design and relevant calculations were conducted using a semi-analytical simulation method, which integrated the exact mode solver for the cylindrical coordinate and eigenmode expansion method. It was examined that the current metallic patterns in the fiber sensor can trigger the LSPR by the electric field Er of the core mode HE11, and this is the main reason why this novel fiber sensor can obtain high performance. After performing algorithms, images showed evident excitation of the LSPR. The LSPR fiber sensor designed in this paper possesses excellent attributes of short length (185.173 &#x03BC;m), high resolution (approximately - 70 dB), and high sensitivity (approximately 34 257 nm/RIU).]]></abstract>
##     <issn><![CDATA[1943-0655]]></issn>
##     <htmlFlag><![CDATA[1]]></htmlFlag>
##     <arnumber><![CDATA[6744636]]></arnumber>
##     <doi><![CDATA[10.1109/JPHOT.2014.2306828]]></doi>
##     <publicationId><![CDATA[6744636]]></publicationId>
##     <mdurl><![CDATA[http://ieeexplore.ieee.org/xpl/articleDetails.jsp?tp=&arnumber=6744636&contentType=Journals+%26+Magazines]]></mdurl>
##     <pdf><![CDATA[http://ieeexplore.ieee.org/stamp/stamp.jsp?arnumber=6744636]]></pdf>
##   </document>
##   <document>
##     <rank>461</rank>
##     <title><![CDATA[An Iterative Compensation Approach Without Linearization of Projector Responses for Multiple-Projector System]]></title>
##     <authors><![CDATA[Miyagawa, I.;  Sugaya, Y.;  Arai, H.;  Morimoto, M.]]></authors>
##     <affiliations><![CDATA[NTT Media Intell. Labs., NTT Corp., Yokosuka, Japan]]></affiliations>
##     <controlledterms>
##       <term><![CDATA[brightness]]></term>
##       <term><![CDATA[cameras]]></term>
##       <term><![CDATA[compensation]]></term>
##       <term><![CDATA[iterative methods]]></term>
##       <term><![CDATA[linearisation techniques]]></term>
##       <term><![CDATA[optical projectors]]></term>
##     </controlledterms>
##     <thesaurusterms>
##       <term><![CDATA[Brightness]]></term>
##       <term><![CDATA[Calibration]]></term>
##       <term><![CDATA[Cameras]]></term>
##       <term><![CDATA[Equations]]></term>
##       <term><![CDATA[Feedback control]]></term>
##       <term><![CDATA[Image color analysis]]></term>
##       <term><![CDATA[Iterative methods]]></term>
##     </thesaurusterms>
##     <pubtitle><![CDATA[Image Processing, IEEE Transactions on]]></pubtitle>
##     <punumber><![CDATA[83]]></punumber>
##     <pubtype><![CDATA[Journals & Magazines]]></pubtype>
##     <publisher><![CDATA[IEEE]]></publisher>
##     <volume><![CDATA[23]]></volume>
##     <issue><![CDATA[6]]></issue>
##     <py><![CDATA[2014]]></py>
##     <spage><![CDATA[2676]]></spage>
##     <epage><![CDATA[2687]]></epage>
##     <abstract><![CDATA[We aim to realize a new and simple compensation method that robustly handles multiple-projector systems without recourse to the linearization of projector response functions. We introduce state equations, which distribute arbitrary brightness among the individual projectors, and control the state equations according to the feedback from a camera. By employing the color-mixing matrix with gradient of projector responses, we compensate the controlled brightness input to each projector. Our method dispenses with cooperation among multiple projectors as well as time-consuming photometric calibration. Compared with existing methods, our method is shown to offer superior compensation performance and a more effective way of compensating multiple-projector systems.]]></abstract>
##     <issn><![CDATA[1057-7149]]></issn>
##     <htmlFlag><![CDATA[1]]></htmlFlag>
##     <arnumber><![CDATA[6800035]]></arnumber>
##     <doi><![CDATA[10.1109/TIP.2014.2317979]]></doi>
##     <publicationId><![CDATA[6800035]]></publicationId>
##     <mdurl><![CDATA[http://ieeexplore.ieee.org/xpl/articleDetails.jsp?tp=&arnumber=6800035&contentType=Journals+%26+Magazines]]></mdurl>
##     <pdf><![CDATA[http://ieeexplore.ieee.org/stamp/stamp.jsp?arnumber=6800035]]></pdf>
##   </document>
##   <document>
##     <rank>462</rank>
##     <title><![CDATA[Lessons and Experiences in the Implementation of a Consolidated Transmission Modeling Data System at ERCOT]]></title>
##     <authors><![CDATA[Moseley, J.D.;  Mago, N.V.;  Grady, W.M.;  Santoso, S.]]></authors>
##     <affiliations><![CDATA[, Electric Reliability Council of Texas, Taylor, TX, USA]]></affiliations>
##     <thesaurusterms>
##       <term><![CDATA[Computer integrated manufacturing]]></term>
##       <term><![CDATA[Data models]]></term>
##       <term><![CDATA[Energy management]]></term>
##       <term><![CDATA[Load modeling]]></term>
##       <term><![CDATA[Power system planning]]></term>
##       <term><![CDATA[Power system reliability]]></term>
##       <term><![CDATA[Standards]]></term>
##     </thesaurusterms>
##     <pubtitle><![CDATA[Power and Energy Technology Systems Journal, IEEE]]></pubtitle>
##     <punumber><![CDATA[6687318]]></punumber>
##     <pubtype><![CDATA[Journals & Magazines]]></pubtype>
##     <publisher><![CDATA[IEEE]]></publisher>
##     <volume><![CDATA[1]]></volume>
##     <py><![CDATA[2014]]></py>
##     <spage><![CDATA[12]]></spage>
##     <epage><![CDATA[20]]></epage>
##     <abstract><![CDATA[On 1 September 2009, as part of a new node-based market implementation, the Electric Reliability Council of Texas transitioned from its then-prevalent data modeling processes to a new network model management system (NMMS) for model data management. This change represented the culmination of nearly four and a half years of planning and development. The NMMS provided several improvements over the then-current data maintenance processes. This paper describes some of the existing issues facing the power industry in the area of power system model data management and explains how solutions to these issues were addressed conceptually in the development of and incorporated into the final design of the NMMS. Further, this paper explains how these concepts of design and usability are now gaining wider recognition and acceptance from the industry.]]></abstract>
##     <htmlFlag><![CDATA[1]]></htmlFlag>
##     <arnumber><![CDATA[6963585]]></arnumber>
##     <doi><![CDATA[10.1109/JPETS.2014.2363405]]></doi>
##     <publicationId><![CDATA[6963585]]></publicationId>
##     <mdurl><![CDATA[http://ieeexplore.ieee.org/xpl/articleDetails.jsp?tp=&arnumber=6963585&contentType=Journals+%26+Magazines]]></mdurl>
##     <pdf><![CDATA[http://ieeexplore.ieee.org/stamp/stamp.jsp?arnumber=6963585]]></pdf>
##   </document>
##   <document>
##     <rank>463</rank>
##     <title><![CDATA[Spatial Mutual Information as Similarity Measure for 3-D Brain Image Registration]]></title>
##     <authors><![CDATA[Razlighi, Q.R.;  Kehtarnavaz, N.]]></authors>
##     <affiliations><![CDATA[Dept. of Biomed. Eng. & Neurology, Columbia Univ., New York, NY, USA]]></affiliations>
##     <controlledterms>
##       <term><![CDATA[biomedical MRI]]></term>
##       <term><![CDATA[brain]]></term>
##       <term><![CDATA[image registration]]></term>
##       <term><![CDATA[medical image processing]]></term>
##     </controlledterms>
##     <thesaurusterms>
##       <term><![CDATA[Brain modeling]]></term>
##       <term><![CDATA[Entropy]]></term>
##       <term><![CDATA[Image registration]]></term>
##       <term><![CDATA[Mutual information]]></term>
##       <term><![CDATA[Three-dimensional displays]]></term>
##     </thesaurusterms>
##     <pubtitle><![CDATA[Translational Engineering in Health and Medicine, IEEE Journal of]]></pubtitle>
##     <punumber><![CDATA[6221039]]></punumber>
##     <pubtype><![CDATA[Journals & Magazines]]></pubtype>
##     <publisher><![CDATA[IEEE]]></publisher>
##     <volume><![CDATA[2]]></volume>
##     <py><![CDATA[2014]]></py>
##     <spage><![CDATA[27]]></spage>
##     <epage><![CDATA[34]]></epage>
##     <abstract><![CDATA[Information theoretic-based similarity measures, in particular mutual information, are widely used for intermodal/intersubject 3-D brain image registration. However, conventional mutual information does not consider spatial dependency between adjacent voxels in images, thus reducing its efficacy as a similarity measure in image registration. This paper first presents a review of the existing attempts to incorporate spatial dependency into the computation of mutual information (MI). Then, a recently introduced spatially dependent similarity measure, named spatial MI, is extended to 3-D brain image registration. This extension also eliminates its artifact for translational misregistration. Finally, the effectiveness of the proposed 3-D spatial MI as a similarity measure is compared with three existing MI measures by applying controlled levels of noise degradation to 3-D simulated brain images.]]></abstract>
##     <issn><![CDATA[2168-2372]]></issn>
##     <htmlFlag><![CDATA[1]]></htmlFlag>
##     <arnumber><![CDATA[6708450]]></arnumber>
##     <doi><![CDATA[10.1109/JTEHM.2014.2299280]]></doi>
##     <publicationId><![CDATA[6708450]]></publicationId>
##     <mdurl><![CDATA[http://ieeexplore.ieee.org/xpl/articleDetails.jsp?tp=&arnumber=6708450&contentType=Journals+%26+Magazines]]></mdurl>
##     <pdf><![CDATA[http://ieeexplore.ieee.org/stamp/stamp.jsp?arnumber=6708450]]></pdf>
##   </document>
##   <document>
##     <rank>464</rank>
##     <title><![CDATA[A Neo-Reflective Wrist Pulse Oximeter]]></title>
##     <authors><![CDATA[Pang, G.;  Chao Ma]]></authors>
##     <affiliations><![CDATA[Dept. of Electr. & Electron. Eng., Univ. of Hong Kong, Hong Kong, Chile]]></affiliations>
##     <controlledterms>
##       <term><![CDATA[bioelectric potentials]]></term>
##       <term><![CDATA[blood]]></term>
##       <term><![CDATA[body sensor networks]]></term>
##       <term><![CDATA[calibration]]></term>
##       <term><![CDATA[chemical sensors]]></term>
##       <term><![CDATA[data communication]]></term>
##       <term><![CDATA[optical sensors]]></term>
##       <term><![CDATA[oximetry]]></term>
##       <term><![CDATA[telemedicine]]></term>
##     </controlledterms>
##     <thesaurusterms>
##       <term><![CDATA[Biomedical optical imaging]]></term>
##       <term><![CDATA[Blood pressure]]></term>
##       <term><![CDATA[Optical reflection]]></term>
##       <term><![CDATA[Optical saturation]]></term>
##       <term><![CDATA[Optical sensors]]></term>
##       <term><![CDATA[Oximeters]]></term>
##       <term><![CDATA[Pulse measurement]]></term>
##       <term><![CDATA[Wearable computers]]></term>
##       <term><![CDATA[Wrists]]></term>
##     </thesaurusterms>
##     <pubtitle><![CDATA[Access, IEEE]]></pubtitle>
##     <punumber><![CDATA[6287639]]></punumber>
##     <pubtype><![CDATA[Journals & Magazines]]></pubtype>
##     <publisher><![CDATA[IEEE]]></publisher>
##     <volume><![CDATA[2]]></volume>
##     <py><![CDATA[2014]]></py>
##     <spage><![CDATA[1562]]></spage>
##     <epage><![CDATA[1567]]></epage>
##     <abstract><![CDATA[This paper relates to a genuine wrist pulse oximeter, which is a noninvasive medical device that can measure the pulse rate and oxygen saturation level in a person's blood. The device is novel due to its innovative design. It is a new type of reflective oximeter, which has a concave structure for housing the optical source and sensor. The neo-reflective sensor module of the device is designed to send the sensor data to a nearby intelligent mobile phone using wireless data transmission. The pulse oximeter has been developed and calibrated, and the calibration curve was analyzed. The innovative design of this pulse oximeter would enable the user to wear the low-cost device on one wrist continuously throughout the day, without the inconvenience of a conventional finger pulse oximeter.]]></abstract>
##     <issn><![CDATA[2169-3536]]></issn>
##     <htmlFlag><![CDATA[1]]></htmlFlag>
##     <arnumber><![CDATA[6990728]]></arnumber>
##     <doi><![CDATA[10.1109/ACCESS.2014.2382179]]></doi>
##     <publicationId><![CDATA[6990728]]></publicationId>
##     <mdurl><![CDATA[http://ieeexplore.ieee.org/xpl/articleDetails.jsp?tp=&arnumber=6990728&contentType=Journals+%26+Magazines]]></mdurl>
##     <pdf><![CDATA[http://ieeexplore.ieee.org/stamp/stamp.jsp?arnumber=6990728]]></pdf>
##   </document>
##   <document>
##     <rank>465</rank>
##     <title><![CDATA[Evaluation and Comparison of Current Fetal Ultrasound Image Segmentation Methods for Biometric Measurements: A Grand Challenge]]></title>
##     <authors><![CDATA[Rueda, S.;  Fathima, S.;  Knight, C.L.;  Yaqub, M.;  Papageorghiou, A.T.;  Rahmatullah, B.;  Foi, A.;  Maggioni, M.;  Pepe, A.;  Tohka, J.;  Stebbing, R.V.;  McManigle, J.E.;  Ciurte, A.;  Bresson, X.;  Cuadra, M.B.;  Changming Sun;  Ponomarev, G.V.;  Gelfand, M.S.;  Kazanov, M.D.;  Ching-Wei Wang;  Hsiang-Chou Chen;  Chun-Wei Peng;  Chu-Mei Hung;  Noble, J.A.]]></authors>
##     <affiliations><![CDATA[Dept. of Eng. Sci., Univ. of Oxford, Oxford, UK]]></affiliations>
##     <controlledterms>
##       <term><![CDATA[biomedical ultrasonics]]></term>
##       <term><![CDATA[image segmentation]]></term>
##       <term><![CDATA[medical image processing]]></term>
##       <term><![CDATA[obstetrics]]></term>
##     </controlledterms>
##     <thesaurusterms>
##       <term><![CDATA[Biomedical imaging]]></term>
##       <term><![CDATA[Head]]></term>
##       <term><![CDATA[Image segmentation]]></term>
##       <term><![CDATA[Magnetic heads]]></term>
##       <term><![CDATA[Ultrasonic imaging]]></term>
##       <term><![CDATA[Ultrasonic variables measurement]]></term>
##     </thesaurusterms>
##     <pubtitle><![CDATA[Medical Imaging, IEEE Transactions on]]></pubtitle>
##     <punumber><![CDATA[42]]></punumber>
##     <pubtype><![CDATA[Journals & Magazines]]></pubtype>
##     <publisher><![CDATA[IEEE]]></publisher>
##     <volume><![CDATA[33]]></volume>
##     <issue><![CDATA[4]]></issue>
##     <py><![CDATA[2014]]></py>
##     <spage><![CDATA[797]]></spage>
##     <epage><![CDATA[813]]></epage>
##     <abstract><![CDATA[This paper presents the evaluation results of the methods submitted to Challenge US: Biometric Measurements from Fetal Ultrasound Images, a segmentation challenge held at the IEEE International Symposium on Biomedical Imaging 2012. The challenge was set to compare and evaluate current fetal ultrasound image segmentation methods. It consisted of automatically segmenting fetal anatomical structures to measure standard obstetric biometric parameters, from 2D fetal ultrasound images taken on fetuses at different gestational ages (21 weeks, 28 weeks, and 33 weeks) and with varying image quality to reflect data encountered in real clinical environments. Four independent sub-challenges were proposed, according to the objects of interest measured in clinical practice: abdomen, head, femur, and whole fetus. Five teams participated in the head sub-challenge and two teams in the femur sub-challenge, including one team who tackled both. Nobody attempted the abdomen and whole fetus sub-challenges. The challenge goals were two-fold and the participants were asked to submit the segmentation results as well as the measurements derived from the segmented objects. Extensive quantitative (region-based, distance-based, and Bland-Altman measurements) and qualitative evaluation was performed to compare the results from a representative selection of current methods submitted to the challenge. Several experts (three for the head sub-challenge and two for the femur sub-challenge), with different degrees of expertise, manually delineated the objects of interest to define the ground truth used within the evaluation framework. For the head sub-challenge, several groups produced results that could be potentially used in clinical settings, with comparable performance to manual delineations. The femur sub-challenge had inferior performance to the head sub-challenge due to the fact that it is a harder segmentation problem and that the techniques presented relied more on the femur's appearance.]]></abstract>
##     <issn><![CDATA[0278-0062]]></issn>
##     <htmlFlag><![CDATA[1]]></htmlFlag>
##     <arnumber><![CDATA[6575204]]></arnumber>
##     <doi><![CDATA[10.1109/TMI.2013.2276943]]></doi>
##     <publicationId><![CDATA[6575204]]></publicationId>
##     <mdurl><![CDATA[http://ieeexplore.ieee.org/xpl/articleDetails.jsp?tp=&arnumber=6575204&contentType=Journals+%26+Magazines]]></mdurl>
##     <pdf><![CDATA[http://ieeexplore.ieee.org/stamp/stamp.jsp?arnumber=6575204]]></pdf>
##   </document>
##   <document>
##     <rank>466</rank>
##     <title><![CDATA[Radially Polarized Orbital Angular Momentum Beam Emitter Based on Shallow-Ridge Silicon Microring Cavity]]></title>
##     <authors><![CDATA[Rui Li;  Xue Feng;  Dengke Zhang;  Kaiyu Cui;  Fang Liu;  Yidong Huang]]></authors>
##     <affiliations><![CDATA[Dept. of Electron. Eng., Tsinghua Univ., Beijing, China]]></affiliations>
##     <controlledterms>
##       <term><![CDATA[angular momentum]]></term>
##       <term><![CDATA[diffraction gratings]]></term>
##       <term><![CDATA[elemental semiconductors]]></term>
##       <term><![CDATA[integrated optics]]></term>
##       <term><![CDATA[microcavities]]></term>
##       <term><![CDATA[optical microscopy]]></term>
##       <term><![CDATA[optical waveguides]]></term>
##       <term><![CDATA[silicon]]></term>
##     </controlledterms>
##     <thesaurusterms>
##       <term><![CDATA[Cavity resonators]]></term>
##       <term><![CDATA[Electric fields]]></term>
##       <term><![CDATA[Gratings]]></term>
##       <term><![CDATA[Integrated optics]]></term>
##       <term><![CDATA[Optical waveguides]]></term>
##       <term><![CDATA[Silicon]]></term>
##     </thesaurusterms>
##     <pubtitle><![CDATA[Photonics Journal, IEEE]]></pubtitle>
##     <punumber><![CDATA[4563994]]></punumber>
##     <pubtype><![CDATA[Journals & Magazines]]></pubtype>
##     <publisher><![CDATA[IEEE]]></publisher>
##     <volume><![CDATA[6]]></volume>
##     <issue><![CDATA[3]]></issue>
##     <py><![CDATA[2014]]></py>
##     <spage><![CDATA[1]]></spage>
##     <epage><![CDATA[10]]></epage>
##     <abstract><![CDATA[Radially polarized orbital angular momentum (OAM) beams could be applied in optical manipulation and optical microscopy. An integrated OAM beam emitter with radially polarized radiation is proposed on shallow-ridge silicon microring with azimuthally distributed gratings etched on top of the ring waveguide. Two key structural parameters, the number of grating elements, and the dimension of each grating element are optimized by considering both the OAM order purity (C<sub>p</sub>) and the total radiation efficiency (&#x03B7;<sub>all</sub>). With elaborate design, C<sub>p</sub> = ~98%, &#x03B7;<sub>all</sub> = ~49% are achieved by simulations, and the calculated energy ratio between the far-field radially and azimuthally polarized component is more than 11 dB, which indicates that the radially polarized component is predominant in the radiated electromagnetic field.]]></abstract>
##     <issn><![CDATA[1943-0655]]></issn>
##     <htmlFlag><![CDATA[1]]></htmlFlag>
##     <arnumber><![CDATA[6815730]]></arnumber>
##     <doi><![CDATA[10.1109/JPHOT.2014.2321757]]></doi>
##     <publicationId><![CDATA[6815730]]></publicationId>
##     <mdurl><![CDATA[http://ieeexplore.ieee.org/xpl/articleDetails.jsp?tp=&arnumber=6815730&contentType=Journals+%26+Magazines]]></mdurl>
##     <pdf><![CDATA[http://ieeexplore.ieee.org/stamp/stamp.jsp?arnumber=6815730]]></pdf>
##   </document>
##   <document>
##     <rank>467</rank>
##     <title><![CDATA[Breakthroughs in Photonics 2013: Terahertz Wave Photonics]]></title>
##     <authors><![CDATA[Ito, H.]]></authors>
##     <affiliations><![CDATA[Terahertz-wave Res. Group, RIKEN, Sendai, Japan]]></affiliations>
##     <controlledterms>
##       <term><![CDATA[microwave photonics]]></term>
##     </controlledterms>
##     <thesaurusterms>
##       <term><![CDATA[Nonlinear optics]]></term>
##       <term><![CDATA[Optical imaging]]></term>
##       <term><![CDATA[Optical pulses]]></term>
##       <term><![CDATA[Optical pumping]]></term>
##       <term><![CDATA[Optical sensors]]></term>
##       <term><![CDATA[Quantum cascade lasers]]></term>
##       <term><![CDATA[Surface emitting lasers]]></term>
##     </thesaurusterms>
##     <pubtitle><![CDATA[Photonics Journal, IEEE]]></pubtitle>
##     <punumber><![CDATA[4563994]]></punumber>
##     <pubtype><![CDATA[Journals & Magazines]]></pubtype>
##     <publisher><![CDATA[IEEE]]></publisher>
##     <volume><![CDATA[6]]></volume>
##     <issue><![CDATA[2]]></issue>
##     <py><![CDATA[2014]]></py>
##     <spage><![CDATA[1]]></spage>
##     <epage><![CDATA[5]]></epage>
##     <abstract><![CDATA[This paper presents an overview of recent developments in terahertz science and technology. Important advances have occurred in higher power terahertz sources and in other devices. They are described along with some notable applications.]]></abstract>
##     <issn><![CDATA[1943-0655]]></issn>
##     <htmlFlag><![CDATA[1]]></htmlFlag>
##     <arnumber><![CDATA[6755482]]></arnumber>
##     <doi><![CDATA[10.1109/JPHOT.2014.2309648]]></doi>
##     <publicationId><![CDATA[6755482]]></publicationId>
##     <mdurl><![CDATA[http://ieeexplore.ieee.org/xpl/articleDetails.jsp?tp=&arnumber=6755482&contentType=Journals+%26+Magazines]]></mdurl>
##     <pdf><![CDATA[http://ieeexplore.ieee.org/stamp/stamp.jsp?arnumber=6755482]]></pdf>
##   </document>
##   <document>
##     <rank>468</rank>
##     <title><![CDATA[The Generalized TP Model Transformation for T&#x2013;S Fuzzy Model Manipulation and Generalized Stability Verification]]></title>
##     <authors><![CDATA[Baranyi, P.]]></authors>
##     <affiliations><![CDATA[Comput. & Autom. Res. Inst., Budapest, Hungary]]></affiliations>
##     <controlledterms>
##       <term><![CDATA[control system synthesis]]></term>
##       <term><![CDATA[fuzzy control]]></term>
##       <term><![CDATA[identification]]></term>
##       <term><![CDATA[linear matrix inequalities]]></term>
##       <term><![CDATA[principal component analysis]]></term>
##       <term><![CDATA[stability]]></term>
##       <term><![CDATA[tensors]]></term>
##     </controlledterms>
##     <thesaurusterms>
##       <term><![CDATA[Computational modeling]]></term>
##       <term><![CDATA[Control design]]></term>
##       <term><![CDATA[Mathematical model]]></term>
##       <term><![CDATA[Numerical models]]></term>
##       <term><![CDATA[Stability analysis]]></term>
##       <term><![CDATA[Tensile stress]]></term>
##       <term><![CDATA[Vectors]]></term>
##     </thesaurusterms>
##     <pubtitle><![CDATA[Fuzzy Systems, IEEE Transactions on]]></pubtitle>
##     <punumber><![CDATA[91]]></punumber>
##     <pubtype><![CDATA[Journals & Magazines]]></pubtype>
##     <publisher><![CDATA[IEEE]]></publisher>
##     <volume><![CDATA[22]]></volume>
##     <issue><![CDATA[4]]></issue>
##     <py><![CDATA[2014]]></py>
##     <spage><![CDATA[934]]></spage>
##     <epage><![CDATA[948]]></epage>
##     <abstract><![CDATA[This paper integrates various ideas about the tensor product (TP) model transformation into one conceptual framework and formulates it in terms of the Takagi-Sugeno (T-S) fuzzy model manipulation and control design framework. Several new extensions of the TP model transformation are proposed, such as the quasi and &#x201C;full,&#x201D; compact and rank-reduced higher order singular-value-decomposition-based canonical form of T-S fuzzy models, and the bilinear-, multi, pseudo-, convex-, partial TP model transformations. All of these extensions together form the generalized TP model transformation, which provides an effective tool to freely and readily manipulate the antecedent sets and rules of T-S fuzzy models and also provides main fuzzy rule component analysis, as well as a means for complexity and accuracy tradeoffs. It is demonstrated in this paper that the proposed manipulation forms a new, effective, and necessary optimization step of T-S fuzzy or polytopic models and linear-matrix-inequality-based control design, and can also decrease conservativeness. Identification techniques are typically constructed according to the available data and measurement set, as well as the type of system to be identified. As a result, they may not always provide good representations for control design frameworks. This paper demonstrates that the proposed TP model transformation is unique in that it bridges between various soft-computing-based identification techniques and T-S fuzzy model-based approaches. Finally, this paper proposes the multi-TP model transformation, which is a tractable and nonheuristic framework to verify the stability of the result of fuzzy or various soft-computing-based control designs. The multi-TP model transformation could provide an answer to the frequently emerging criticisms regarding the lack of mathematical stability verification techniques in the soft-computing-based control design. Control examples are provided in this paper.]]></abstract>
##     <issn><![CDATA[1063-6706]]></issn>
##     <htmlFlag><![CDATA[1]]></htmlFlag>
##     <arnumber><![CDATA[6583324]]></arnumber>
##     <doi><![CDATA[10.1109/TFUZZ.2013.2278982]]></doi>
##     <publicationId><![CDATA[6583324]]></publicationId>
##     <mdurl><![CDATA[http://ieeexplore.ieee.org/xpl/articleDetails.jsp?tp=&arnumber=6583324&contentType=Journals+%26+Magazines]]></mdurl>
##     <pdf><![CDATA[http://ieeexplore.ieee.org/stamp/stamp.jsp?arnumber=6583324]]></pdf>
##   </document>
##   <document>
##     <rank>469</rank>
##     <title><![CDATA[Urine Flow Dynamics Through Prostatic Urethra With Tubular Organ Modeling Using Endoscopic Imagery]]></title>
##     <authors><![CDATA[Ishii, T.;  Kambara, Y.;  Yamanishi, T.;  Naya, Y.;  Igarashi, T.]]></authors>
##     <affiliations><![CDATA[Grad. Sch. of Eng., Chiba Univ., Chiba, Japan]]></affiliations>
##     <controlledterms>
##       <term><![CDATA[biological fluid dynamics]]></term>
##       <term><![CDATA[biological organs]]></term>
##       <term><![CDATA[endoscopes]]></term>
##       <term><![CDATA[medical image processing]]></term>
##       <term><![CDATA[video signal processing]]></term>
##     </controlledterms>
##     <thesaurusterms>
##       <term><![CDATA[Bladder]]></term>
##       <term><![CDATA[Computational fluid dynamics]]></term>
##       <term><![CDATA[Endoscopes]]></term>
##       <term><![CDATA[Solid modeling]]></term>
##       <term><![CDATA[Streaming media]]></term>
##       <term><![CDATA[Three-dimensional displays]]></term>
##       <term><![CDATA[Tubular organs]]></term>
##     </thesaurusterms>
##     <pubtitle><![CDATA[Translational Engineering in Health and Medicine, IEEE Journal of]]></pubtitle>
##     <punumber><![CDATA[6221039]]></punumber>
##     <pubtype><![CDATA[Journals & Magazines]]></pubtype>
##     <publisher><![CDATA[IEEE]]></publisher>
##     <volume><![CDATA[2]]></volume>
##     <py><![CDATA[2014]]></py>
##     <spage><![CDATA[1]]></spage>
##     <epage><![CDATA[9]]></epage>
##     <abstract><![CDATA[Voiding dysfunction is common in the aged male population. However, the obstruction mechanism in the lower urinary tract and critical points for obstruction remains uncertain. The aim of this paper was to develop a system to investigate the relationship between voiding dysfunction and alteration of the shape of the prostatic urethra by processing endoscopic video images of the urethra and analyzing the fluid dynamics of the urine stream. A panoramic image of the prostatic urethra was generated from cystourethroscopic video images. A virtual 3-D model of the urethra was constructed using the luminance values in the image. Fluid dynamics using the constructed model was then calculated assuming a static urethra and maximum urine flow rate. Cystourethroscopic videos from 11 patients with benign prostatic hyperplasia were recorded around administration of an alpha-1 adrenoceptor antagonist. The calculated pressure loss through the prostatic urethra in each model corresponded to the prostatic volume, and the improvements of the pressure loss after treatment correlated to the conventional clinical indices. As shown by the proposed method, the shape of the prostatic urethra affects the transporting urine fluid energy, and this paper implies a possible method for detecting critical lesions responsible for voiding dysfunction. The proposed method provides critical information about deformation of the prostatic urethra on voiding function. Detailed differences in the various types of relaxants for the lower urinary tract could be estimated.]]></abstract>
##     <issn><![CDATA[2168-2372]]></issn>
##     <htmlFlag><![CDATA[1]]></htmlFlag>
##     <arnumber><![CDATA[6784320]]></arnumber>
##     <doi><![CDATA[10.1109/JTEHM.2014.2316148]]></doi>
##     <publicationId><![CDATA[6784320]]></publicationId>
##     <mdurl><![CDATA[http://ieeexplore.ieee.org/xpl/articleDetails.jsp?tp=&arnumber=6784320&contentType=Journals+%26+Magazines]]></mdurl>
##     <pdf><![CDATA[http://ieeexplore.ieee.org/stamp/stamp.jsp?arnumber=6784320]]></pdf>
##   </document>
##   <document>
##     <rank>470</rank>
##     <title><![CDATA[Hierarchical Infinite Divisibility for Multiscale Shrinkage]]></title>
##     <authors><![CDATA[Xin Yuan;  Rao, V.;  Shaobo Han;  Carin, L.]]></authors>
##     <affiliations><![CDATA[Dept. of Electr. & Comput. Eng., Duke Univ., Durham, NC, USA]]></affiliations>
##     <controlledterms>
##       <term><![CDATA[Markov processes]]></term>
##       <term><![CDATA[Monte Carlo methods]]></term>
##       <term><![CDATA[approximation theory]]></term>
##       <term><![CDATA[compressed sensing]]></term>
##       <term><![CDATA[data compression]]></term>
##       <term><![CDATA[discrete cosine transforms]]></term>
##       <term><![CDATA[expectation-maximisation algorithm]]></term>
##       <term><![CDATA[gamma distribution]]></term>
##       <term><![CDATA[image coding]]></term>
##       <term><![CDATA[image denoising]]></term>
##       <term><![CDATA[image representation]]></term>
##       <term><![CDATA[inference mechanisms]]></term>
##       <term><![CDATA[tree data structures]]></term>
##       <term><![CDATA[vectors]]></term>
##       <term><![CDATA[wavelet transforms]]></term>
##     </controlledterms>
##     <thesaurusterms>
##       <term><![CDATA[Bayes methods]]></term>
##       <term><![CDATA[Context]]></term>
##       <term><![CDATA[Discrete cosine transforms]]></term>
##       <term><![CDATA[Mathematical model]]></term>
##       <term><![CDATA[Noise]]></term>
##       <term><![CDATA[Random variables]]></term>
##       <term><![CDATA[Wavelet transforms]]></term>
##     </thesaurusterms>
##     <pubtitle><![CDATA[Signal Processing, IEEE Transactions on]]></pubtitle>
##     <punumber><![CDATA[78]]></punumber>
##     <pubtype><![CDATA[Journals & Magazines]]></pubtype>
##     <publisher><![CDATA[IEEE]]></publisher>
##     <volume><![CDATA[62]]></volume>
##     <issue><![CDATA[17]]></issue>
##     <py><![CDATA[2014]]></py>
##     <spage><![CDATA[4363]]></spage>
##     <epage><![CDATA[4374]]></epage>
##     <abstract><![CDATA[A new shrinkage-based construction is developed for a compressible vector mmb x &#x2208; BBR<sup>n</sup>, for cases in which the components of mmb x are naturally associated with a tree structure. Important examples are when mmb x corresponds to the coefficients of a wavelet or block-DCT representation of data. The method we consider in detail, and for which numerical results are presented, is based on the gamma distribution. The gamma distribution is a heavy-tailed distribution that is infinitely divisible, and these characteristics are leveraged within the model. We further demonstrate that the general framework is appropriate for many other types of infinitely divisible heavy-tailed distributions. Bayesian inference is carried out by approximating the posterior with samples from an MCMC algorithm, as well as by constructing a variational approximation to the posterior. We also consider expectation-maximization (EM) for a MAP (point) solution. State-of-the-art results are manifested for compressive sensing and denoising applications, the latter with spiky (non-Gaussian) noise.]]></abstract>
##     <issn><![CDATA[1053-587X]]></issn>
##     <htmlFlag><![CDATA[1]]></htmlFlag>
##     <arnumber><![CDATA[6847180]]></arnumber>
##     <doi><![CDATA[10.1109/TSP.2014.2334557]]></doi>
##     <publicationId><![CDATA[6847180]]></publicationId>
##     <mdurl><![CDATA[http://ieeexplore.ieee.org/xpl/articleDetails.jsp?tp=&arnumber=6847180&contentType=Journals+%26+Magazines]]></mdurl>
##     <pdf><![CDATA[http://ieeexplore.ieee.org/stamp/stamp.jsp?arnumber=6847180]]></pdf>
##   </document>
##   <document>
##     <rank>471</rank>
##     <title><![CDATA[Closed Form Asymptotic Expression of a Random-Access Interference Measure]]></title>
##     <authors><![CDATA[Ferrante, G.C.;  Di Benedetto, M.-G.]]></authors>
##     <affiliations><![CDATA[Dept. of Inf. Eng., Sapienza Univ. of Rome, Rome, Italy]]></affiliations>
##     <controlledterms>
##       <term><![CDATA[interference (signal)]]></term>
##       <term><![CDATA[statistical analysis]]></term>
##       <term><![CDATA[wireless channels]]></term>
##     </controlledterms>
##     <thesaurusterms>
##       <term><![CDATA[Atmospheric measurements]]></term>
##       <term><![CDATA[Closed-form solutions]]></term>
##       <term><![CDATA[Interference]]></term>
##       <term><![CDATA[Lattices]]></term>
##       <term><![CDATA[Particle measurements]]></term>
##       <term><![CDATA[Polynomials]]></term>
##       <term><![CDATA[Zinc]]></term>
##     </thesaurusterms>
##     <pubtitle><![CDATA[Communications Letters, IEEE]]></pubtitle>
##     <punumber><![CDATA[4234]]></punumber>
##     <pubtype><![CDATA[Journals & Magazines]]></pubtype>
##     <publisher><![CDATA[IEEE]]></publisher>
##     <volume><![CDATA[18]]></volume>
##     <issue><![CDATA[7]]></issue>
##     <py><![CDATA[2014]]></py>
##     <spage><![CDATA[1107]]></spage>
##     <epage><![CDATA[1110]]></epage>
##     <abstract><![CDATA[A model describing the cumulative effect of the independent access of K users to a shared resource, which is formed by N elements, is proposed, based on which an integer interference measure &#x03B6; is defined. While traditional cases can be reconducted to reference well-known results, for which &#x03B6; is either Gaussian or Poissonian, the proposed model provides a more general framework that offers the tool for understanding the nature of &#x03B6;. In particular, an asymptotic closed form expression ( K &#x2192; &#x221E;, N &#x2192; &#x221E;, K/N\&#x2192; &#x221E;&#x03B2;&#x2208;(0,&#x221E;)) for &#x03B6; distribution is provided for systems presenting constructive versus destructive interference, and as such is applicable to characterizing statistical properties of interference in a wide range of random multiple access channels.]]></abstract>
##     <issn><![CDATA[1089-7798]]></issn>
##     <htmlFlag><![CDATA[1]]></htmlFlag>
##     <arnumber><![CDATA[6807712]]></arnumber>
##     <doi><![CDATA[10.1109/LCOMM.2014.2320935]]></doi>
##     <publicationId><![CDATA[6807712]]></publicationId>
##     <mdurl><![CDATA[http://ieeexplore.ieee.org/xpl/articleDetails.jsp?tp=&arnumber=6807712&contentType=Journals+%26+Magazines]]></mdurl>
##     <pdf><![CDATA[http://ieeexplore.ieee.org/stamp/stamp.jsp?arnumber=6807712]]></pdf>
##   </document>
##   <document>
##     <rank>472</rank>
##     <title><![CDATA[Asynchronous Binaural Spatial Audition Sensor With 2<formula formulatype="inline"> <img src="/images/tex/353.gif" alt=",\times,"> </formula>64<formula formulatype="inline"> <img src="/images/tex/353.gif" alt=",\times,"> </formula>4 Channel Output]]></title>
##     <authors><![CDATA[Shih-Chii Liu;  van Schaik, A.;  Minch, B.A.;  Delbruck, T.]]></authors>
##     <affiliations><![CDATA[Inst of Neuroinf., Univ. of Zurich, Zurich, Switzerland]]></affiliations>
##     <controlledterms>
##       <term><![CDATA[CMOS integrated circuits]]></term>
##       <term><![CDATA[audio signal processing]]></term>
##       <term><![CDATA[biomedical measurement]]></term>
##       <term><![CDATA[cochlear implants]]></term>
##       <term><![CDATA[ear]]></term>
##       <term><![CDATA[hearing]]></term>
##       <term><![CDATA[medical signal processing]]></term>
##       <term><![CDATA[microphones]]></term>
##       <term><![CDATA[preamplifiers]]></term>
##       <term><![CDATA[pulse frequency modulation]]></term>
##       <term><![CDATA[speech]]></term>
##       <term><![CDATA[timing jitter]]></term>
##     </controlledterms>
##     <thesaurusterms>
##       <term><![CDATA[Computer architecture]]></term>
##       <term><![CDATA[Gain]]></term>
##       <term><![CDATA[Microphones]]></term>
##       <term><![CDATA[Silicon]]></term>
##       <term><![CDATA[System-on-chip]]></term>
##       <term><![CDATA[Transistors]]></term>
##       <term><![CDATA[Universal Serial Bus]]></term>
##     </thesaurusterms>
##     <pubtitle><![CDATA[Biomedical Circuits and Systems, IEEE Transactions on]]></pubtitle>
##     <punumber><![CDATA[4156126]]></punumber>
##     <pubtype><![CDATA[Journals & Magazines]]></pubtype>
##     <publisher><![CDATA[IEEE]]></publisher>
##     <volume><![CDATA[8]]></volume>
##     <issue><![CDATA[4]]></issue>
##     <py><![CDATA[2014]]></py>
##     <spage><![CDATA[453]]></spage>
##     <epage><![CDATA[464]]></epage>
##     <abstract><![CDATA[This paper proposes an integrated event-based binaural silicon cochlea system aimed at efficient spatial audition and auditory scene analysis. The cochlea chip has a matched pair of digitally-calibrated 64-stage cascaded analog second-order filter banks with 512 pulse-frequency modulated (PFM) address-event representation (AER) outputs. The quality factors (Qs) of channels are individually adjusted by local DACs. The 2P4M 0.35 um CMOS chip consumes an average power of 14 mW including its integrated microphone preamplifiers and biasing circuits. Typical speech data rates are 10 k to 100 k events per second (eps) with peak output rates of 10 Meps. The event timing jitter is 2 us for a 250 mVpp input. It is shown that the computational cost of an event-driven source localization application can be up to 40 times lower when compared to a conventional cross-correlation approach.]]></abstract>
##     <issn><![CDATA[1932-4545]]></issn>
##     <htmlFlag><![CDATA[1]]></htmlFlag>
##     <arnumber><![CDATA[6658899]]></arnumber>
##     <doi><![CDATA[10.1109/TBCAS.2013.2281834]]></doi>
##     <publicationId><![CDATA[6658899]]></publicationId>
##     <mdurl><![CDATA[http://ieeexplore.ieee.org/xpl/articleDetails.jsp?tp=&arnumber=6658899&contentType=Journals+%26+Magazines]]></mdurl>
##     <pdf><![CDATA[http://ieeexplore.ieee.org/stamp/stamp.jsp?arnumber=6658899]]></pdf>
##   </document>
##   <document>
##     <rank>473</rank>
##     <title><![CDATA[Low-Cost Dual-Band Circularly Polarized Switched-Beam Array for Global Navigation Satellite System]]></title>
##     <authors><![CDATA[Maqsood, M.;  Gao, S.;  Brown, T.W.C.;  Unwin, M.;  de vos Van Steenwijk, R.;  Xu, J.D.;  Underwood, C.I.]]></authors>
##     <affiliations><![CDATA[Surrey Space Centre, Guildford, UK]]></affiliations>
##     <controlledterms>
##       <term><![CDATA[electromagnetic wave polarisation]]></term>
##       <term><![CDATA[loop antennas]]></term>
##       <term><![CDATA[microstrip antenna arrays]]></term>
##       <term><![CDATA[microstrip lines]]></term>
##       <term><![CDATA[multifrequency antennas]]></term>
##       <term><![CDATA[p-i-n diodes]]></term>
##       <term><![CDATA[satellite navigation]]></term>
##     </controlledterms>
##     <thesaurusterms>
##       <term><![CDATA[Antenna arrays]]></term>
##       <term><![CDATA[Arrays]]></term>
##       <term><![CDATA[Broadband antennas]]></term>
##       <term><![CDATA[Feeds]]></term>
##       <term><![CDATA[PIN photodiodes]]></term>
##       <term><![CDATA[Switches]]></term>
##     </thesaurusterms>
##     <pubtitle><![CDATA[Antennas and Propagation, IEEE Transactions on]]></pubtitle>
##     <punumber><![CDATA[8]]></punumber>
##     <pubtype><![CDATA[Journals & Magazines]]></pubtype>
##     <publisher><![CDATA[IEEE]]></publisher>
##     <volume><![CDATA[62]]></volume>
##     <issue><![CDATA[4]]></issue>
##     <part><![CDATA[2]]></part>
##     <py><![CDATA[2014]]></py>
##     <spage><![CDATA[1975]]></spage>
##     <epage><![CDATA[1982]]></epage>
##     <abstract><![CDATA[This paper presents the design and development of a dual-band switched-beam microstrip array for global navigation satellite system (GNSS) applications such as ocean reflectometry and remote sensing. In contrast to the traditional Butler matrix, a simple, low cost, broadband and low insertion loss beam switching feed network is proposed, designed and integrated with a dual band antenna array to achieve continuous beam coverage of &#x00B1;25<sup>&#x00B0;</sup> around the boresight at the L1 (1.575 GHz) and L2 (1.227 GHz) bands. To reduce the cost, microstrip lines and PIN diode based switches are employed. The proposed switched-beam network is then integrated with dual-band step-shorted annular ring (S-SAR) antenna elements in order to produce a fully integrated compact-sized switched-beam array. Antenna simulation results show that the switched-beam array achieves a maximum gain of 12 dBic at the L1 band and 10 dBic at the L2 band. In order to validate the concept, a scaled down prototype of the simulated design is fabricated and measured. The prototype operates at twice of the original design frequency, i.e., 3.15 GHz and 2.454 GHz and the measured results confirm that the integrated array achieves beam switching and good performance at both bands.]]></abstract>
##     <issn><![CDATA[0018-926X]]></issn>
##     <htmlFlag><![CDATA[1]]></htmlFlag>
##     <arnumber><![CDATA[6717014]]></arnumber>
##     <doi><![CDATA[10.1109/TAP.2014.2301435]]></doi>
##     <publicationId><![CDATA[6717014]]></publicationId>
##     <mdurl><![CDATA[http://ieeexplore.ieee.org/xpl/articleDetails.jsp?tp=&arnumber=6717014&contentType=Journals+%26+Magazines]]></mdurl>
##     <pdf><![CDATA[http://ieeexplore.ieee.org/stamp/stamp.jsp?arnumber=6717014]]></pdf>
##   </document>
##   <document>
##     <rank>474</rank>
##     <title><![CDATA[Strong Linewidth Reduction by Compact Brillouin/Erbium Fiber Laser]]></title>
##     <authors><![CDATA[Mo Chen;  Zhou Meng;  Jianfei Wang;  Wei Chen]]></authors>
##     <affiliations><![CDATA[Coll. of Optoelectron. Sci. & Eng., Nat. Univ. of Defense Technol., Changsha, China]]></affiliations>
##     <controlledterms>
##       <term><![CDATA[erbium]]></term>
##       <term><![CDATA[fibre lasers]]></term>
##       <term><![CDATA[laser cavity resonators]]></term>
##       <term><![CDATA[laser feedback]]></term>
##       <term><![CDATA[optical pumping]]></term>
##     </controlledterms>
##     <thesaurusterms>
##       <term><![CDATA[Cavity resonators]]></term>
##       <term><![CDATA[Gain]]></term>
##       <term><![CDATA[Laser feedback]]></term>
##       <term><![CDATA[Optical pumping]]></term>
##       <term><![CDATA[Optical resonators]]></term>
##       <term><![CDATA[Pump lasers]]></term>
##       <term><![CDATA[Scattering]]></term>
##     </thesaurusterms>
##     <pubtitle><![CDATA[Photonics Journal, IEEE]]></pubtitle>
##     <punumber><![CDATA[4563994]]></punumber>
##     <pubtype><![CDATA[Journals & Magazines]]></pubtype>
##     <publisher><![CDATA[IEEE]]></publisher>
##     <volume><![CDATA[6]]></volume>
##     <issue><![CDATA[5]]></issue>
##     <py><![CDATA[2014]]></py>
##     <spage><![CDATA[1]]></spage>
##     <epage><![CDATA[8]]></epage>
##     <abstract><![CDATA[We demonstrate the strong linewidth reduction by a compact Brillouin/erbium fiber laser (BEFL), in which 4-m erbium-doped fiber (EDF) acts as both the Brillouin and linear gain media. Due to the assistance of the linear gain in the EDF with the acoustic damping and the cavity feedback in the resonator, the proposed BEFL transferred the 20-MHz Brillouin pump (BP) into the 950-Hz Brillouin Stokes laser emission, demonstrating that the laser linewidth is reduced by over 2 &#x00D7; 10<sup>4</sup> times through this BEFL. The linewidth reduction is independent of the 980-nm pump power or the BP wavelength. The BEFL keeps strong linewidth reduction over the whole C-band. This effect of the BEFL provides a simple but effective way to obtain highly coherent light for many applications.]]></abstract>
##     <issn><![CDATA[1943-0655]]></issn>
##     <htmlFlag><![CDATA[1]]></htmlFlag>
##     <arnumber><![CDATA[6912928]]></arnumber>
##     <doi><![CDATA[10.1109/JPHOT.2014.2360290]]></doi>
##     <publicationId><![CDATA[6912928]]></publicationId>
##     <mdurl><![CDATA[http://ieeexplore.ieee.org/xpl/articleDetails.jsp?tp=&arnumber=6912928&contentType=Journals+%26+Magazines]]></mdurl>
##     <pdf><![CDATA[http://ieeexplore.ieee.org/stamp/stamp.jsp?arnumber=6912928]]></pdf>
##   </document>
##   <document>
##     <rank>475</rank>
##     <title><![CDATA[Germanium Gate PhotoMOSFET Integrated to Silicon Photonics]]></title>
##     <authors><![CDATA[Going, R.W.;  Loo, J.;  Tsu-Jae King Liu;  Wu, M.C.]]></authors>
##     <affiliations><![CDATA[Dept. of Electr. Eng. & Comput. Sci., Univ. of California, Berkeley, Berkeley, CA, USA]]></affiliations>
##     <controlledterms>
##       <term><![CDATA[MOSFET]]></term>
##       <term><![CDATA[elemental semiconductors]]></term>
##       <term><![CDATA[germanium]]></term>
##       <term><![CDATA[integrated optoelectronics]]></term>
##       <term><![CDATA[melting]]></term>
##       <term><![CDATA[optical fabrication]]></term>
##       <term><![CDATA[optical waveguide components]]></term>
##       <term><![CDATA[photodetectors]]></term>
##       <term><![CDATA[phototransistors]]></term>
##       <term><![CDATA[recrystallisation annealing]]></term>
##       <term><![CDATA[semiconductor growth]]></term>
##       <term><![CDATA[silicon-on-insulator]]></term>
##     </controlledterms>
##     <thesaurusterms>
##       <term><![CDATA[Capacitance]]></term>
##       <term><![CDATA[Germanium]]></term>
##       <term><![CDATA[Logic gates]]></term>
##       <term><![CDATA[Optical waveguides]]></term>
##       <term><![CDATA[Photodiodes]]></term>
##       <term><![CDATA[Silicon]]></term>
##       <term><![CDATA[Transistors]]></term>
##     </thesaurusterms>
##     <pubtitle><![CDATA[Selected Topics in Quantum Electronics, IEEE Journal of]]></pubtitle>
##     <punumber><![CDATA[2944]]></punumber>
##     <pubtype><![CDATA[Journals & Magazines]]></pubtype>
##     <publisher><![CDATA[IEEE]]></publisher>
##     <volume><![CDATA[20]]></volume>
##     <issue><![CDATA[4]]></issue>
##     <py><![CDATA[2014]]></py>
##     <spage><![CDATA[1]]></spage>
##     <epage><![CDATA[7]]></epage>
##     <abstract><![CDATA[This paper presents a novel germanium gated NMOS phototransistor integrated on a silicon photonics platform on silicon-on-insulator (SOI) substrate. The phototransistor is fabricated with a modified NMOS process flow, with germanium which is recrystallized using rapid melt growth during the source/drain activation anneal step. The resulting device, with 1-&#x03BC;m channel length, and 8-&#x03BC;m channel width, demonstrates a responsivity of over 18 A/W at 1550 nm with 583 nW of incident light. By increasing the incident power to 912 &#x03BC;W, the device operates at 2.5 GHz. Miniaturization is expected to improve both responsivity and speed in future devices.]]></abstract>
##     <issn><![CDATA[1077-260X]]></issn>
##     <htmlFlag><![CDATA[1]]></htmlFlag>
##     <arnumber><![CDATA[6680616]]></arnumber>
##     <doi><![CDATA[10.1109/JSTQE.2013.2294470]]></doi>
##     <publicationId><![CDATA[6680616]]></publicationId>
##     <mdurl><![CDATA[http://ieeexplore.ieee.org/xpl/articleDetails.jsp?tp=&arnumber=6680616&contentType=Journals+%26+Magazines]]></mdurl>
##     <pdf><![CDATA[http://ieeexplore.ieee.org/stamp/stamp.jsp?arnumber=6680616]]></pdf>
##   </document>
##   <document>
##     <rank>476</rank>
##     <title><![CDATA[Cognitive Control: Theory and Application]]></title>
##     <authors><![CDATA[Fatemi, M.;  Haykin, S.]]></authors>
##     <affiliations><![CDATA[Sch. of Comput. Sci. & Eng., McMaster Univ., Hamilton, ON, Canada]]></affiliations>
##     <controlledterms>
##       <term><![CDATA[cognition]]></term>
##       <term><![CDATA[computational complexity]]></term>
##       <term><![CDATA[dynamic programming]]></term>
##     </controlledterms>
##     <thesaurusterms>
##       <term><![CDATA[Brain modeling]]></term>
##       <term><![CDATA[Cognition]]></term>
##       <term><![CDATA[Complexity theory]]></term>
##       <term><![CDATA[Control systems]]></term>
##       <term><![CDATA[Dynamic programming]]></term>
##       <term><![CDATA[Heuristic algorithms]]></term>
##       <term><![CDATA[Perception]]></term>
##       <term><![CDATA[Radar tracking]]></term>
##     </thesaurusterms>
##     <pubtitle><![CDATA[Access, IEEE]]></pubtitle>
##     <punumber><![CDATA[6287639]]></punumber>
##     <pubtype><![CDATA[Journals & Magazines]]></pubtype>
##     <publisher><![CDATA[IEEE]]></publisher>
##     <volume><![CDATA[2]]></volume>
##     <py><![CDATA[2014]]></py>
##     <spage><![CDATA[698]]></spage>
##     <epage><![CDATA[710]]></epage>
##     <abstract><![CDATA[From an engineering point-of-view, cognitive control is inspired by the prefrontal cortex of the human brain; cognitive control may therefore be viewed as the overarching function of a cognitive dynamic system. In this paper, we describe a new way of thinking about cognitive control that embodies two basic components: learning and planning, both of which are based on two notions: 1) two-state model of the environment and the perceptor and 2) perception-action cycle, which is a distinctive characteristic of the cognitive dynamic system. Most importantly, it is shown that the cognitive control learning algorithm is a special form of Bellman's dynamic programming. Distinctive properties of the new algorithm include the following: 1) optimality of performance; 2) algorithmic convergence to optimal policy; and 3) linear law of complexity measured in terms of the number of actions taken by the cognitive controller on the environment. To validate these intrinsic properties of the algorithm, a computational experiment is presented, which involves a cognitive tracking radar that is known to closely mimic the visual brain. The experiment illustrates two different scenarios: 1) the impact of planning on learning curves of the new cognitive controller and 2) comparison of the learning curves of three different controllers, based on dynamic optimization, traditional (Q) -learning, and the new algorithm. The latter two algorithms are based on the two-state model, and they both involve the use of planning.]]></abstract>
##     <issn><![CDATA[2169-3536]]></issn>
##     <htmlFlag><![CDATA[1]]></htmlFlag>
##     <arnumber><![CDATA[6843352]]></arnumber>
##     <doi><![CDATA[10.1109/ACCESS.2014.2332333]]></doi>
##     <publicationId><![CDATA[6843352]]></publicationId>
##     <mdurl><![CDATA[http://ieeexplore.ieee.org/xpl/articleDetails.jsp?tp=&arnumber=6843352&contentType=Journals+%26+Magazines]]></mdurl>
##     <pdf><![CDATA[http://ieeexplore.ieee.org/stamp/stamp.jsp?arnumber=6843352]]></pdf>
##   </document>
##   <document>
##     <rank>477</rank>
##     <title><![CDATA[Real-Time and Simultaneous Control of Artificial Limbs Based on Pattern Recognition Algorithms]]></title>
##     <authors><![CDATA[Ortiz-Catalan, M.;  H&#x00E5; kansson, B.;  Br&#x00E5; nemark, R.]]></authors>
##     <affiliations><![CDATA[Dept. of Signals & Syst., Chalmers Univ. of Technol., Gothenburg, Sweden]]></affiliations>
##     <controlledterms>
##       <term><![CDATA[artificial limbs]]></term>
##       <term><![CDATA[biomechanics]]></term>
##       <term><![CDATA[electromyography]]></term>
##       <term><![CDATA[medical control systems]]></term>
##       <term><![CDATA[medical signal processing]]></term>
##       <term><![CDATA[multilayer perceptrons]]></term>
##       <term><![CDATA[neurophysiology]]></term>
##       <term><![CDATA[pattern recognition]]></term>
##       <term><![CDATA[real-time systems]]></term>
##       <term><![CDATA[signal classification]]></term>
##       <term><![CDATA[topology]]></term>
##     </controlledterms>
##     <thesaurusterms>
##       <term><![CDATA[Accuracy]]></term>
##       <term><![CDATA[Pattern recognition]]></term>
##       <term><![CDATA[Prosthetics]]></term>
##       <term><![CDATA[Real-time systems]]></term>
##       <term><![CDATA[Testing]]></term>
##       <term><![CDATA[Topology]]></term>
##       <term><![CDATA[Training]]></term>
##     </thesaurusterms>
##     <pubtitle><![CDATA[Neural Systems and Rehabilitation Engineering, IEEE Transactions on]]></pubtitle>
##     <punumber><![CDATA[7333]]></punumber>
##     <pubtype><![CDATA[Journals & Magazines]]></pubtype>
##     <publisher><![CDATA[IEEE]]></publisher>
##     <volume><![CDATA[22]]></volume>
##     <issue><![CDATA[4]]></issue>
##     <py><![CDATA[2014]]></py>
##     <spage><![CDATA[756]]></spage>
##     <epage><![CDATA[764]]></epage>
##     <abstract><![CDATA[The prediction of simultaneous limb motions is a highly desirable feature for the control of artificial limbs. In this work, we investigate different classification strategies for individual and simultaneous movements based on pattern recognition of myoelectric signals. Our results suggest that any classifier can be potentially employed in the prediction of simultaneous movements if arranged in a distributed topology. On the other hand, classifiers inherently capable of simultaneous predictions, such as the multi-layer perceptron (MLP), were found to be more cost effective, as they can be successfully employed in their simplest form. In the prediction of individual movements, the one-vs-one (OVO) topology was found to improve classification accuracy across different classifiers and it was therefore used to benchmark the benefits of simultaneous control. As opposed to previous work reporting only offline accuracy, the classification performance and the resulting controllability are evaluated in real time using the motion test and target achievement control (TAC) test, respectively. We propose a simultaneous classification strategy based on MLP that outperformed a top classifier for individual movements (LDA-OVO), thus improving the state-of-the-art classification approach. Furthermore, all the presented classification strategies and data collected in this study are freely available in BioPatRec, an open source platform for the development of advanced prosthetic control strategies.]]></abstract>
##     <issn><![CDATA[1534-4320]]></issn>
##     <htmlFlag><![CDATA[1]]></htmlFlag>
##     <arnumber><![CDATA[6744619]]></arnumber>
##     <doi><![CDATA[10.1109/TNSRE.2014.2305097]]></doi>
##     <publicationId><![CDATA[6744619]]></publicationId>
##     <mdurl><![CDATA[http://ieeexplore.ieee.org/xpl/articleDetails.jsp?tp=&arnumber=6744619&contentType=Journals+%26+Magazines]]></mdurl>
##     <pdf><![CDATA[http://ieeexplore.ieee.org/stamp/stamp.jsp?arnumber=6744619]]></pdf>
##   </document>
##   <document>
##     <rank>478</rank>
##     <title><![CDATA[VMThunder: Fast Provisioning of Large-Scale Virtual Machine Clusters]]></title>
##     <authors><![CDATA[Zhaoning Zhang;  Ziyang Li;  Kui Wu;  Dongsheng Li;  Huiba Li;  Yuxing Peng;  Xicheng Lu]]></authors>
##     <affiliations><![CDATA[Nat. Key Lab. for Parallel & Distrib. Process., Nat. Univ. of Defense Technol., Changsha, China]]></affiliations>
##     <controlledterms>
##       <term><![CDATA[Linux]]></term>
##       <term><![CDATA[cache storage]]></term>
##       <term><![CDATA[cloud computing]]></term>
##       <term><![CDATA[computer bootstrapping]]></term>
##       <term><![CDATA[peer-to-peer computing]]></term>
##       <term><![CDATA[storage area networks]]></term>
##       <term><![CDATA[tree data structures]]></term>
##       <term><![CDATA[virtual machines]]></term>
##     </controlledterms>
##     <thesaurusterms>
##       <term><![CDATA[Clouds]]></term>
##       <term><![CDATA[Image storage]]></term>
##       <term><![CDATA[Peer-to-peer computing]]></term>
##       <term><![CDATA[Relays]]></term>
##       <term><![CDATA[Servers]]></term>
##       <term><![CDATA[Streaming media]]></term>
##       <term><![CDATA[Virtual machining]]></term>
##     </thesaurusterms>
##     <pubtitle><![CDATA[Parallel and Distributed Systems, IEEE Transactions on]]></pubtitle>
##     <punumber><![CDATA[71]]></punumber>
##     <pubtype><![CDATA[Journals & Magazines]]></pubtype>
##     <publisher><![CDATA[IEEE]]></publisher>
##     <volume><![CDATA[25]]></volume>
##     <issue><![CDATA[12]]></issue>
##     <py><![CDATA[2014]]></py>
##     <spage><![CDATA[3328]]></spage>
##     <epage><![CDATA[3338]]></epage>
##     <abstract><![CDATA[Infrastructure as a service (IaaS) allows users to rent resources from the Cloud to meet their various computing requirements. The pay-as-you-use model, however, poses a nontrivial technical challenge to the IaaS cloud service providers: how to fast provision a large number of virtual machines (VMs) to meet users' dynamic computing requests? We address this challenge with VMThunder, a new VM provisioning tool, which downloads data blockson demandduring the VM booting process and speeds up VM image streaming by strategically integrating peer-to-peer (P2P) streaming techniques with enhanced optimization schemes such as transfer on demand, cache on read, snapshot on local, and relay on cache. In particular, VMThunder stores the original images in a share storage and in the meantime it adopts a tree-based P2P streaming scheme so that common image blocks are cached and reused across the nodes in the cluster. We implement VMThunder in CentOS Linux and thoroughly test its performance. Comprehensive experimental results show that VMThunder outperforms the state-of-the-art VM provisioning methods, with respect to scalability, latency, and VM runtime I/O performance.]]></abstract>
##     <issn><![CDATA[1045-9219]]></issn>
##     <htmlFlag><![CDATA[1]]></htmlFlag>
##     <arnumber><![CDATA[6719385]]></arnumber>
##     <doi><![CDATA[10.1109/TPDS.2014.7]]></doi>
##     <publicationId><![CDATA[6719385]]></publicationId>
##     <mdurl><![CDATA[http://ieeexplore.ieee.org/xpl/articleDetails.jsp?tp=&arnumber=6719385&contentType=Journals+%26+Magazines]]></mdurl>
##     <pdf><![CDATA[http://ieeexplore.ieee.org/stamp/stamp.jsp?arnumber=6719385]]></pdf>
##   </document>
##   <document>
##     <rank>479</rank>
##     <title><![CDATA[Edge Couplers With Relaxed Alignment Tolerance for Pick-and-Place Hybrid Integration of III&#x2013;V Lasers With SOI Waveguides]]></title>
##     <authors><![CDATA[Romero-Garcia, S.;  Marzban, B.;  Merget, F.;  Bin Shen;  Witzens, J.]]></authors>
##     <affiliations><![CDATA[Integrated Photonics Lab., RWTH Aachen Univ., Aachen, Germany]]></affiliations>
##     <controlledterms>
##       <term><![CDATA[CMOS integrated circuits]]></term>
##       <term><![CDATA[III-V semiconductors]]></term>
##       <term><![CDATA[integrated optoelectronics]]></term>
##       <term><![CDATA[laser accessories]]></term>
##       <term><![CDATA[laser modes]]></term>
##       <term><![CDATA[optical couplers]]></term>
##       <term><![CDATA[optical fibre couplers]]></term>
##       <term><![CDATA[optical fibre fabrication]]></term>
##       <term><![CDATA[optical fibre losses]]></term>
##       <term><![CDATA[semiconductor lasers]]></term>
##       <term><![CDATA[silicon-on-insulator]]></term>
##       <term><![CDATA[ultraviolet lithography]]></term>
##     </controlledterms>
##     <thesaurusterms>
##       <term><![CDATA[Couplers]]></term>
##       <term><![CDATA[Couplings]]></term>
##       <term><![CDATA[Interference]]></term>
##       <term><![CDATA[Laser beams]]></term>
##       <term><![CDATA[Laser modes]]></term>
##       <term><![CDATA[Optical waveguides]]></term>
##       <term><![CDATA[Waveguide lasers]]></term>
##     </thesaurusterms>
##     <pubtitle><![CDATA[Selected Topics in Quantum Electronics, IEEE Journal of]]></pubtitle>
##     <punumber><![CDATA[2944]]></punumber>
##     <pubtype><![CDATA[Journals & Magazines]]></pubtype>
##     <publisher><![CDATA[IEEE]]></publisher>
##     <volume><![CDATA[20]]></volume>
##     <issue><![CDATA[4]]></issue>
##     <py><![CDATA[2014]]></py>
##     <spage><![CDATA[369]]></spage>
##     <epage><![CDATA[379]]></epage>
##     <abstract><![CDATA[We report on two edge-coupling and power splitting devices for hybrid integration of III-V lasers with sub-micrometric silicon-on-insulator waveguides. The proposed devices relax the horizontal alignment tolerances required to achieve high coupling efficiencies and are suitable for passively aligned assembly with pick-and-place tools. Light is coupled to two on-chip single mode SOI waveguides with almost identical power coupling efficiency, but with a varying relative phase accommodating the lateral misalignment between the laser diode and the coupling devices, and is suitable for the implementation of parallel optics transmitters. Experimental characterization with both a lensed fiber and a Fabry-Pe&#x0301;rot semiconductor laser diode has been performed. Excess insertion losses (in addition to the 3 dB splitting) taken as the worst case over both waveguides of respectively 2 &#x00B1; 0.3 dB and 3.1 &#x00B1; 0.3 dB, as well as excellent 1 dB loss placement tolerance range of respectively 2.8 &#x03BC;m and 3.8 &#x03BC;m (worst case over both in-plane axes) have been measured for the two devices. Back-reflections to the laser are below -20 dB for both devices within the 1 dB misalignment range. Devices were fabricated with 193 nm DUV optical lithography and are compatible with mass-manufacturing with mainstream CMOS technology.]]></abstract>
##     <issn><![CDATA[1077-260X]]></issn>
##     <htmlFlag><![CDATA[1]]></htmlFlag>
##     <arnumber><![CDATA[6684278]]></arnumber>
##     <doi><![CDATA[10.1109/JSTQE.2013.2292523]]></doi>
##     <publicationId><![CDATA[6684278]]></publicationId>
##     <mdurl><![CDATA[http://ieeexplore.ieee.org/xpl/articleDetails.jsp?tp=&arnumber=6684278&contentType=Journals+%26+Magazines]]></mdurl>
##     <pdf><![CDATA[http://ieeexplore.ieee.org/stamp/stamp.jsp?arnumber=6684278]]></pdf>
##   </document>
##   <document>
##     <rank>480</rank>
##     <title><![CDATA[Hybrid-Type Temperature Sensor Using Thin-Film Transistors]]></title>
##     <authors><![CDATA[Kimura, M.;  Kojima, K.;  Mukuda, T.;  Kito, K.;  Hayashi, H.;  Matsuda, T.;  Hiroshima, Y.;  Miyasaka, M.]]></authors>
##     <affiliations><![CDATA[Dept. of Electron. & Inf., Ryukoku Univ., Otsu, Japan]]></affiliations>
##     <controlledterms>
##       <term><![CDATA[leakage currents]]></term>
##       <term><![CDATA[oscillators]]></term>
##       <term><![CDATA[temperature sensors]]></term>
##       <term><![CDATA[thin film transistors]]></term>
##     </controlledterms>
##     <thesaurusterms>
##       <term><![CDATA[Ring oscillators]]></term>
##       <term><![CDATA[Temperature dependence]]></term>
##       <term><![CDATA[Temperature measurement]]></term>
##       <term><![CDATA[Temperature sensors]]></term>
##       <term><![CDATA[Thin film transistors]]></term>
##     </thesaurusterms>
##     <pubtitle><![CDATA[Electron Devices Society, IEEE Journal of the]]></pubtitle>
##     <punumber><![CDATA[6245494]]></punumber>
##     <pubtype><![CDATA[Journals & Magazines]]></pubtype>
##     <publisher><![CDATA[IEEE]]></publisher>
##     <volume><![CDATA[2]]></volume>
##     <issue><![CDATA[6]]></issue>
##     <py><![CDATA[2014]]></py>
##     <spage><![CDATA[182]]></spage>
##     <epage><![CDATA[186]]></epage>
##     <abstract><![CDATA[We have developed a hybrid-type temperature sensor using thin-film transistors. First, we evaluate temperature dependences of transistor characteristics and find that the temperature dependence of the off-leakage current is much larger than that of the on current. Next, we combine a transistor, capacitor, and ring oscillator to develop the hybrid-type temperature sensor and detect the temperature by measuring the oscillation frequency. The large temperature dependences of the off-leakage currents can be utilized, and simultaneously only a digital circuit is required to count the digital pulse.]]></abstract>
##     <issn><![CDATA[2168-6734]]></issn>
##     <htmlFlag><![CDATA[1]]></htmlFlag>
##     <arnumber><![CDATA[6883115]]></arnumber>
##     <doi><![CDATA[10.1109/JEDS.2014.2352218]]></doi>
##     <publicationId><![CDATA[6883115]]></publicationId>
##     <mdurl><![CDATA[http://ieeexplore.ieee.org/xpl/articleDetails.jsp?tp=&arnumber=6883115&contentType=Journals+%26+Magazines]]></mdurl>
##     <pdf><![CDATA[http://ieeexplore.ieee.org/stamp/stamp.jsp?arnumber=6883115]]></pdf>
##   </document>
##   <document>
##     <rank>481</rank>
##     <title><![CDATA[Reconfigurable Temporal Fourier Transformation and Temporal Imaging]]></title>
##     <authors><![CDATA[Zhao Wu;  Lei Lei;  Jianji Dong;  Jie Hou;  Xinliang Zhang]]></authors>
##     <affiliations><![CDATA[Wuhan Nat. Lab. for Optoelectron., Huazhong Univ. of Sci. & Technol., Wuhan, China]]></affiliations>
##     <controlledterms>
##       <term><![CDATA[Fourier transform optics]]></term>
##       <term><![CDATA[lenses]]></term>
##       <term><![CDATA[optical fibre dispersion]]></term>
##     </controlledterms>
##     <thesaurusterms>
##       <term><![CDATA[Band-pass filters]]></term>
##       <term><![CDATA[Imaging]]></term>
##       <term><![CDATA[Lenses]]></term>
##       <term><![CDATA[Optical fiber devices]]></term>
##       <term><![CDATA[Optical fiber dispersion]]></term>
##       <term><![CDATA[Thin film transistors]]></term>
##     </thesaurusterms>
##     <pubtitle><![CDATA[Lightwave Technology, Journal of]]></pubtitle>
##     <punumber><![CDATA[50]]></punumber>
##     <pubtype><![CDATA[Journals & Magazines]]></pubtype>
##     <publisher><![CDATA[IEEE]]></publisher>
##     <volume><![CDATA[32]]></volume>
##     <issue><![CDATA[23]]></issue>
##     <py><![CDATA[2014]]></py>
##     <spage><![CDATA[4565]]></spage>
##     <epage><![CDATA[4570]]></epage>
##     <abstract><![CDATA[We present a multifunctional scheme to implement reconfigurable temporal Fourier transformation (TFT) and temporal imaging (TI) flexibly. This structure is composed of a time lens followed by a section of dispersive fiber with proper length. More specifically, the output waveforms presenting either the spectrum profile or the scaled waveform profile of the input signal are achieved by simply changing the length of the fiber, which is theoretically analyzed and experimentally verified. This proposal requires neither the use of an input dispersive device preceding the time lens nor a second time lens after the dispersion. Hence, it is a simple and practical alternative to the implementation of both TFT and TI.]]></abstract>
##     <issn><![CDATA[0733-8724]]></issn>
##     <htmlFlag><![CDATA[1]]></htmlFlag>
##     <arnumber><![CDATA[6914539]]></arnumber>
##     <doi><![CDATA[10.1109/JLT.2014.2361293]]></doi>
##     <publicationId><![CDATA[6914539]]></publicationId>
##     <mdurl><![CDATA[http://ieeexplore.ieee.org/xpl/articleDetails.jsp?tp=&arnumber=6914539&contentType=Journals+%26+Magazines]]></mdurl>
##     <pdf><![CDATA[http://ieeexplore.ieee.org/stamp/stamp.jsp?arnumber=6914539]]></pdf>
##   </document>
##   <document>
##     <rank>482</rank>
##     <title><![CDATA[Improving the Performance of IndependentTask Assignment Heuristics MinMin,MaxMin and Sufferage]]></title>
##     <authors><![CDATA[Kartal Tabak, E.;  Barla Cambazoglu, B.;  Aykanat, C.]]></authors>
##     <affiliations><![CDATA[HAVELSAN A.S., Ankara, Turkey]]></affiliations>
##     <controlledterms>
##       <term><![CDATA[minimax techniques]]></term>
##       <term><![CDATA[parallel programming]]></term>
##       <term><![CDATA[task analysis]]></term>
##     </controlledterms>
##     <thesaurusterms>
##       <term><![CDATA[Arrays]]></term>
##       <term><![CDATA[Biological cells]]></term>
##       <term><![CDATA[Genetic algorithms]]></term>
##       <term><![CDATA[Heuristic algorithms]]></term>
##       <term><![CDATA[Indexes]]></term>
##       <term><![CDATA[Program processors]]></term>
##       <term><![CDATA[Time complexity]]></term>
##     </thesaurusterms>
##     <pubtitle><![CDATA[Parallel and Distributed Systems, IEEE Transactions on]]></pubtitle>
##     <punumber><![CDATA[71]]></punumber>
##     <pubtype><![CDATA[Journals & Magazines]]></pubtype>
##     <publisher><![CDATA[IEEE]]></publisher>
##     <volume><![CDATA[25]]></volume>
##     <issue><![CDATA[5]]></issue>
##     <py><![CDATA[2014]]></py>
##     <spage><![CDATA[1244]]></spage>
##     <epage><![CDATA[1256]]></epage>
##     <abstract><![CDATA[MinMin, MaxMin, and Sufferage are constructive heuristics that are widely and successfully used in assigning independent tasks to processors in heterogeneous computing systems. All three heuristics are known to run in O(KN<sup>2</sup>) time in assigning N tasks to K processors. In this paper, we propose an algorithmic improvement that asymptotically decreases the running time complexity of MinMin to O(KN log N) without affecting its solution quality. Furthermore, we combine the newly proposed MinMin algorithm with MaxMin as well as Sufferage, obtaining two hybrid algorithms. The motivation behind the former hybrid algorithm is to address the drawback of MaxMin in solving problem instances with highly skewed cost distributions while also improving the running time performance of MaxMin. The latter hybrid algorithm improves the running time performance of Sufferage without degrading its solution quality. The proposed algorithms are easy to implement and we illustrate them through detailed pseudocodes. The experimental results over a large number of real-life data sets show that the proposed fast MinMin algorithm and the proposed hybrid algorithms perform significantly better than their traditional counterparts as well as more recent state-of-the-art assignment heuristics. For the large data sets used in the experiments, MinMin, MaxMin, and Sufferage, as well as recent state-of-the-art heuristics, require days, weeks, or even months to produce a solution, whereas all of the proposed algorithms produce solutions within only two or three minutes.]]></abstract>
##     <issn><![CDATA[1045-9219]]></issn>
##     <htmlFlag><![CDATA[1]]></htmlFlag>
##     <arnumber><![CDATA[6495454]]></arnumber>
##     <doi><![CDATA[10.1109/TPDS.2013.107]]></doi>
##     <publicationId><![CDATA[6495454]]></publicationId>
##     <mdurl><![CDATA[http://ieeexplore.ieee.org/xpl/articleDetails.jsp?tp=&arnumber=6495454&contentType=Journals+%26+Magazines]]></mdurl>
##     <pdf><![CDATA[http://ieeexplore.ieee.org/stamp/stamp.jsp?arnumber=6495454]]></pdf>
##   </document>
##   <document>
##     <rank>483</rank>
##     <title><![CDATA[Multifrequency Electrical Impedance Tomography Using Spectral Constraints]]></title>
##     <authors><![CDATA[Malone, E.;  Sato dos Santos, G.;  Holder, D.;  Arridge, S.]]></authors>
##     <affiliations><![CDATA[Dept. of Med. Phys. & Bioeng., Univ. Coll. London, London, UK]]></affiliations>
##     <controlledterms>
##       <term><![CDATA[bioelectric potentials]]></term>
##       <term><![CDATA[brain]]></term>
##       <term><![CDATA[cancer]]></term>
##       <term><![CDATA[electric impedance]]></term>
##       <term><![CDATA[electric impedance imaging]]></term>
##       <term><![CDATA[electrical conductivity]]></term>
##       <term><![CDATA[image reconstruction]]></term>
##       <term><![CDATA[injuries]]></term>
##       <term><![CDATA[inverse problems]]></term>
##       <term><![CDATA[medical image processing]]></term>
##       <term><![CDATA[phantoms]]></term>
##     </controlledterms>
##     <thesaurusterms>
##       <term><![CDATA[Conductivity]]></term>
##       <term><![CDATA[Frequency measurement]]></term>
##       <term><![CDATA[Image reconstruction]]></term>
##       <term><![CDATA[Impedance]]></term>
##       <term><![CDATA[Tomography]]></term>
##       <term><![CDATA[Voltage measurement]]></term>
##     </thesaurusterms>
##     <pubtitle><![CDATA[Medical Imaging, IEEE Transactions on]]></pubtitle>
##     <punumber><![CDATA[42]]></punumber>
##     <pubtype><![CDATA[Journals & Magazines]]></pubtype>
##     <publisher><![CDATA[IEEE]]></publisher>
##     <volume><![CDATA[33]]></volume>
##     <issue><![CDATA[2]]></issue>
##     <py><![CDATA[2014]]></py>
##     <spage><![CDATA[340]]></spage>
##     <epage><![CDATA[350]]></epage>
##     <abstract><![CDATA[Multifrequency electrical impedance tomography (MFEIT) exploits the dependence of tissue impedance on frequency to recover an image of conductivity. MFEIT could provide emergency diagnosis of pathologies such as acute stroke, brain injury and breast cancer. We present a method for performing MFEIT using spectral constraints. Boundary voltage data is employed directly to reconstruct the volume fraction distribution of component tissues using a nonlinear method. Given that the reconstructed parameter is frequency independent, this approach allows for the simultaneous use of all multifrequency data, thus reducing the degrees of freedom of the reconstruction problem. Furthermore, this method allows for the use of frequency difference data in a nonlinear reconstruction algorithm. Results from empirical phantom measurements suggest that our fraction reconstruction method points to a new direction for the development of multifrequency EIT algorithms in the case that the spectral constraints are known, and may provide a unifying framework for static EIT imaging.]]></abstract>
##     <issn><![CDATA[0278-0062]]></issn>
##     <htmlFlag><![CDATA[1]]></htmlFlag>
##     <arnumber><![CDATA[6626670]]></arnumber>
##     <doi><![CDATA[10.1109/TMI.2013.2284966]]></doi>
##     <publicationId><![CDATA[6626670]]></publicationId>
##     <mdurl><![CDATA[http://ieeexplore.ieee.org/xpl/articleDetails.jsp?tp=&arnumber=6626670&contentType=Journals+%26+Magazines]]></mdurl>
##     <pdf><![CDATA[http://ieeexplore.ieee.org/stamp/stamp.jsp?arnumber=6626670]]></pdf>
##   </document>
##   <document>
##     <rank>484</rank>
##     <title><![CDATA[Electrical Tissue Property Imaging at Low Frequency Using MREIT]]></title>
##     <authors><![CDATA[Jin Keun Seo;  Eung Je Woo]]></authors>
##     <affiliations><![CDATA[Dept. of Comput. Sci. & Eng., Yonsei Univ., Seoul, South Korea]]></affiliations>
##     <controlledterms>
##       <term><![CDATA[Maxwell equations]]></term>
##       <term><![CDATA[bioelectric phenomena]]></term>
##       <term><![CDATA[biological tissues]]></term>
##       <term><![CDATA[biomedical MRI]]></term>
##       <term><![CDATA[data acquisition]]></term>
##       <term><![CDATA[electric impedance imaging]]></term>
##       <term><![CDATA[electrical conductivity]]></term>
##       <term><![CDATA[inverse problems]]></term>
##       <term><![CDATA[medical image processing]]></term>
##       <term><![CDATA[permittivity]]></term>
##     </controlledterms>
##     <thesaurusterms>
##       <term><![CDATA[Conductivity]]></term>
##       <term><![CDATA[Current measurement]]></term>
##       <term><![CDATA[Electrodes]]></term>
##       <term><![CDATA[Magnetic resonance imaging]]></term>
##       <term><![CDATA[Tomography]]></term>
##       <term><![CDATA[Voltage measurement]]></term>
##     </thesaurusterms>
##     <pubtitle><![CDATA[Biomedical Engineering, IEEE Transactions on]]></pubtitle>
##     <punumber><![CDATA[10]]></punumber>
##     <pubtype><![CDATA[Journals & Magazines]]></pubtype>
##     <publisher><![CDATA[IEEE]]></publisher>
##     <volume><![CDATA[61]]></volume>
##     <issue><![CDATA[5]]></issue>
##     <py><![CDATA[2014]]></py>
##     <spage><![CDATA[1390]]></spage>
##     <epage><![CDATA[1399]]></epage>
##     <abstract><![CDATA[The tomographic imaging of tissue's electrical properties (e.g., conductivity and permittivity) has been greatly improved by recent developments in magnetic resonance (MR) imaging techniques, which include MR electrical impedance tomography (MREIT) and electrical property tomography. When the biological material is subjected to an external electric field, local changes in its electrical properties become sources of magnetic field perturbations, which are detectable by the MR signals. Controlling the external excitation and measuring the responses using an MRI scanner, we can formulate the imaging problem as an inverse problem in which unknown tissue properties are recovered from the acquired MR signals. This inverse problem is nonlinear; it involves the incorporation of Maxwell's equations and Bloch equations during data acquisition. Each method for visualizing internal conductivity and permittivity distributions has its own methodological limitations, and is restricted to imaging only a part of the ensemble or mean tissue structures or states. Therefore, imaging methods can be improved by developing complementary methods that can employ the beneficial aspects of various existing techniques. This paper focuses on recent progress in MREIT and discusses its distinct features in comparison with other imaging methods.]]></abstract>
##     <issn><![CDATA[0018-9294]]></issn>
##     <htmlFlag><![CDATA[1]]></htmlFlag>
##     <arnumber><![CDATA[6705619]]></arnumber>
##     <doi><![CDATA[10.1109/TBME.2014.2298859]]></doi>
##     <publicationId><![CDATA[6705619]]></publicationId>
##     <mdurl><![CDATA[http://ieeexplore.ieee.org/xpl/articleDetails.jsp?tp=&arnumber=6705619&contentType=Journals+%26+Magazines]]></mdurl>
##     <pdf><![CDATA[http://ieeexplore.ieee.org/stamp/stamp.jsp?arnumber=6705619]]></pdf>
##   </document>
##   <document>
##     <rank>485</rank>
##     <title><![CDATA[An Extensible Description Language for Video Games]]></title>
##     <authors><![CDATA[Schaul, T.]]></authors>
##     <affiliations><![CDATA[Courant Inst. of Math. Sci., New York Univ., New York, NY, USA]]></affiliations>
##     <controlledterms>
##       <term><![CDATA[computer games]]></term>
##       <term><![CDATA[high level languages]]></term>
##       <term><![CDATA[ontologies (artificial intelligence)]]></term>
##       <term><![CDATA[software libraries]]></term>
##     </controlledterms>
##     <thesaurusterms>
##       <term><![CDATA[Avatars]]></term>
##       <term><![CDATA[Games]]></term>
##       <term><![CDATA[Libraries]]></term>
##       <term><![CDATA[Ontologies]]></term>
##       <term><![CDATA[Physics]]></term>
##       <term><![CDATA[Projectiles]]></term>
##       <term><![CDATA[Visualization]]></term>
##     </thesaurusterms>
##     <pubtitle><![CDATA[Computational Intelligence and AI in Games, IEEE Transactions on]]></pubtitle>
##     <punumber><![CDATA[4804728]]></punumber>
##     <pubtype><![CDATA[Journals & Magazines]]></pubtype>
##     <publisher><![CDATA[IEEE]]></publisher>
##     <volume><![CDATA[6]]></volume>
##     <issue><![CDATA[4]]></issue>
##     <py><![CDATA[2014]]></py>
##     <spage><![CDATA[325]]></spage>
##     <epage><![CDATA[331]]></epage>
##     <abstract><![CDATA[In this short paper, we propose a powerful new tool for conducting research on computational intelligence and games. &#x201C;PyVGDL&#x201D; is a simple, high-level, extensible description language for 2-D video games. It is based on defining locations and dynamics for simple building blocks (objects), together with local interaction effects. A rich ontology defines various controllers, object behaviors, passive effects (physics), and collision effects. It can be used to quickly design games, without having to deal with control structures. We show how the dynamics of many classical games can be generated from a few lines of PyVGDL. Furthermore, the accompanying software library permits parsing and instantly playing those games, visualized from a bird's-eye or first-person viewpoint, and using them as benchmarks for learning algorithms.]]></abstract>
##     <issn><![CDATA[1943-068X]]></issn>
##     <htmlFlag><![CDATA[1]]></htmlFlag>
##     <arnumber><![CDATA[6884801]]></arnumber>
##     <doi><![CDATA[10.1109/TCIAIG.2014.2352795]]></doi>
##     <publicationId><![CDATA[6884801]]></publicationId>
##     <mdurl><![CDATA[http://ieeexplore.ieee.org/xpl/articleDetails.jsp?tp=&arnumber=6884801&contentType=Journals+%26+Magazines]]></mdurl>
##     <pdf><![CDATA[http://ieeexplore.ieee.org/stamp/stamp.jsp?arnumber=6884801]]></pdf>
##   </document>
##   <document>
##     <rank>486</rank>
##     <title><![CDATA[Speed Enhancement of a Single-Mode FP-LD-Based Switch Using the Weak-Lock Technique]]></title>
##     <authors><![CDATA[Tran, Q.H.;  Nakarmi, B.;  Won, Y.H.]]></authors>
##     <affiliations><![CDATA[Dept. of Inf. & Commun. Eng., Korea Adv. Inst. of Sci. & Technol., Daejeon, South Korea]]></affiliations>
##     <controlledterms>
##       <term><![CDATA[laser modes]]></term>
##       <term><![CDATA[optical switches]]></term>
##       <term><![CDATA[semiconductor lasers]]></term>
##     </controlledterms>
##     <thesaurusterms>
##       <term><![CDATA[High-speed optical techniques]]></term>
##       <term><![CDATA[Laser beams]]></term>
##       <term><![CDATA[Logic gates]]></term>
##       <term><![CDATA[Measurement by laser beam]]></term>
##       <term><![CDATA[Optical switches]]></term>
##       <term><![CDATA[Optical wavelength conversion]]></term>
##     </thesaurusterms>
##     <pubtitle><![CDATA[Photonics Journal, IEEE]]></pubtitle>
##     <punumber><![CDATA[4563994]]></punumber>
##     <pubtype><![CDATA[Journals & Magazines]]></pubtype>
##     <publisher><![CDATA[IEEE]]></publisher>
##     <volume><![CDATA[6]]></volume>
##     <issue><![CDATA[4]]></issue>
##     <py><![CDATA[2014]]></py>
##     <spage><![CDATA[1]]></spage>
##     <epage><![CDATA[7]]></epage>
##     <abstract><![CDATA[A technique called weak-lock (WL) has been proposed and applied to a simple optical on/off switch to increase its working speed from 1 to 12 Gb/s with clear output waveforms, whereas the required injection power is decreased. The switch is based on the injection-locking property of a single-mode Fabry-Pe&#x0301;rot laser diode (SMFP-LD). Unlike the case of full-lock (FL) in which self-locked mode and all side modes of the SMFP-LD are fully suppressed due to high injected input power, in the WL case, the injection power is gradually decreased as if the SMFP-LD's self-locked mode is still maintained. It was observed that the output waveform's rising edge, taken at the self-locked mode's wavelength after the filter, in the WL case is clean at speed of 12 Gb/s compared to the blur and noisy rising edge in the FL case even at much lower speed (1 Gb/s). In summary, the WL technique shows advantages compared to the conventional FL technique in terms of lower injection power and higher on/off switching speed.]]></abstract>
##     <issn><![CDATA[1943-0655]]></issn>
##     <htmlFlag><![CDATA[1]]></htmlFlag>
##     <arnumber><![CDATA[6838968]]></arnumber>
##     <doi><![CDATA[10.1109/JPHOT.2014.2331257]]></doi>
##     <publicationId><![CDATA[6838968]]></publicationId>
##     <mdurl><![CDATA[http://ieeexplore.ieee.org/xpl/articleDetails.jsp?tp=&arnumber=6838968&contentType=Journals+%26+Magazines]]></mdurl>
##     <pdf><![CDATA[http://ieeexplore.ieee.org/stamp/stamp.jsp?arnumber=6838968]]></pdf>
##   </document>
##   <document>
##     <rank>487</rank>
##     <title><![CDATA[<italic>In-Situ</italic> Silver Nanoparticle Formation on Surface-Modified Polyetherimide Films]]></title>
##     <authors><![CDATA[Watson, D.E.;  Ng, J.H.-G.;  Aasmundtveit, K.E.;  Desmulliez, M.P.Y.]]></authors>
##     <affiliations><![CDATA[Microsyst. Eng. Centre, Heriot-Watt Univ., Edinburgh, UK]]></affiliations>
##     <controlledterms>
##       <term><![CDATA[X-ray diffraction]]></term>
##       <term><![CDATA[metallisation]]></term>
##       <term><![CDATA[nanofabrication]]></term>
##       <term><![CDATA[nanoparticles]]></term>
##       <term><![CDATA[polymer films]]></term>
##       <term><![CDATA[silver]]></term>
##     </controlledterms>
##     <thesaurusterms>
##       <term><![CDATA[Annealing]]></term>
##       <term><![CDATA[Atmosphere]]></term>
##       <term><![CDATA[Ions]]></term>
##       <term><![CDATA[Nitrogen]]></term>
##       <term><![CDATA[Polyimides]]></term>
##       <term><![CDATA[Silver]]></term>
##       <term><![CDATA[Substrates]]></term>
##     </thesaurusterms>
##     <pubtitle><![CDATA[Nanotechnology, IEEE Transactions on]]></pubtitle>
##     <punumber><![CDATA[7729]]></punumber>
##     <pubtype><![CDATA[Journals & Magazines]]></pubtype>
##     <publisher><![CDATA[IEEE]]></publisher>
##     <volume><![CDATA[13]]></volume>
##     <issue><![CDATA[4]]></issue>
##     <py><![CDATA[2014]]></py>
##     <spage><![CDATA[736]]></spage>
##     <epage><![CDATA[742]]></epage>
##     <abstract><![CDATA[This paper extends the scope of a novel process previously reported by the group for the hydrolysis and subsequent metallization of polyimide substrates to encompass polyetherimide. Silver nanoparticles are grown in-situ by chemical reduction of silver ions implanted in the substrate. Factors affecting the level of hydrolysis are investigated, with temperature of the hydrolyzing solution found to be the key factor. The presence of silver nanoparticles is also confirmed by X-ray diffraction.]]></abstract>
##     <issn><![CDATA[1536-125X]]></issn>
##     <htmlFlag><![CDATA[1]]></htmlFlag>
##     <arnumber><![CDATA[6800043]]></arnumber>
##     <doi><![CDATA[10.1109/TNANO.2014.2318203]]></doi>
##     <publicationId><![CDATA[6800043]]></publicationId>
##     <mdurl><![CDATA[http://ieeexplore.ieee.org/xpl/articleDetails.jsp?tp=&arnumber=6800043&contentType=Journals+%26+Magazines]]></mdurl>
##     <pdf><![CDATA[http://ieeexplore.ieee.org/stamp/stamp.jsp?arnumber=6800043]]></pdf>
##   </document>
##   <document>
##     <rank>488</rank>
##     <title><![CDATA[Developing a Multiobjective Optimization Scheduling System for a Screw Manufacturer: A Refined Genetic Algorithm Approach]]></title>
##     <authors><![CDATA[Tung-Kuan Liu;  Yeh-Peng Chen;  Jyh-Horng Chou]]></authors>
##     <affiliations><![CDATA[Inst. of Eng. Sci. & Technol., Nat. Kaohsiung First Univ. of Sci. & Technol., Kaohsiung, Taiwan]]></affiliations>
##     <controlledterms>
##       <term><![CDATA[fasteners]]></term>
##       <term><![CDATA[flexible manufacturing systems]]></term>
##       <term><![CDATA[genetic algorithms]]></term>
##       <term><![CDATA[job shop scheduling]]></term>
##     </controlledterms>
##     <thesaurusterms>
##       <term><![CDATA[Genetic algorithms]]></term>
##       <term><![CDATA[Job shop scheduling]]></term>
##       <term><![CDATA[Manufacturing processes]]></term>
##       <term><![CDATA[Optimization]]></term>
##       <term><![CDATA[Screws]]></term>
##     </thesaurusterms>
##     <pubtitle><![CDATA[Access, IEEE]]></pubtitle>
##     <punumber><![CDATA[6287639]]></punumber>
##     <pubtype><![CDATA[Journals & Magazines]]></pubtype>
##     <publisher><![CDATA[IEEE]]></publisher>
##     <volume><![CDATA[2]]></volume>
##     <py><![CDATA[2014]]></py>
##     <spage><![CDATA[356]]></spage>
##     <epage><![CDATA[364]]></epage>
##     <abstract><![CDATA[Over time, the traditional single-objective job shop scheduling method has grown increasingly incapable of meeting the requirements of contemporary business models; thus, a multiobjective scheduling solution is required. Because of changing orders, understanding the schedule and output is a constant challenge when using a traditional manual schedule, particularly among manufacturers that produce various products. The multiobjective optimization genetic algorithm (MOGA) is a relatively superior method of solving multiobjective optimization problems; therefore, we used a MOGA to solve flexible job-shop problems for a middle-scale screw manufacturer in Taiwan. For solving the problems of incorrect jobs assign and diversity problem of traditional genetic algorithm (GA) caused by encoding method when applying traditional GA in the flexible manufacturing environment, a refined GA was proposed. Two-phase test has performed for proposed approach, using a classical benchmark of distributed and flexible jobs-shop scheduling problem, and 80 set of work orders, the empirical results indicated that the proposed model yielded substantial savings, regardless of the total order completion time, machine retooling rate, and average machine load rate.]]></abstract>
##     <issn><![CDATA[2169-3536]]></issn>
##     <htmlFlag><![CDATA[1]]></htmlFlag>
##     <arnumber><![CDATA[6804631]]></arnumber>
##     <doi><![CDATA[10.1109/ACCESS.2014.2319351]]></doi>
##     <publicationId><![CDATA[6804631]]></publicationId>
##     <mdurl><![CDATA[http://ieeexplore.ieee.org/xpl/articleDetails.jsp?tp=&arnumber=6804631&contentType=Journals+%26+Magazines]]></mdurl>
##     <pdf><![CDATA[http://ieeexplore.ieee.org/stamp/stamp.jsp?arnumber=6804631]]></pdf>
##   </document>
##   <document>
##     <rank>489</rank>
##     <title><![CDATA[3D Printing for the Rapid Prototyping of Structural Electronics]]></title>
##     <authors><![CDATA[MacDonald, E.;  Salas, R.;  Espalin, D.;  Perez, M.;  Aguilera, E.;  Muse, D.;  Wicker, R.B.]]></authors>
##     <affiliations><![CDATA[W.M. Keck Center for 3D Innovation, Univ. of Texas at El Paso, El Paso, TX, USA]]></affiliations>
##     <controlledterms>
##       <term><![CDATA[accelerometers]]></term>
##       <term><![CDATA[electronics packaging]]></term>
##       <term><![CDATA[light emitting diodes]]></term>
##       <term><![CDATA[microprocessor chips]]></term>
##       <term><![CDATA[rapid prototyping (industrial)]]></term>
##       <term><![CDATA[stereolithography]]></term>
##       <term><![CDATA[three-dimensional integrated circuits]]></term>
##       <term><![CDATA[three-dimensional printing]]></term>
##       <term><![CDATA[time to market]]></term>
##     </controlledterms>
##     <thesaurusterms>
##       <term><![CDATA[Consumer electronics]]></term>
##       <term><![CDATA[Economics]]></term>
##       <term><![CDATA[Packaging]]></term>
##       <term><![CDATA[Printing]]></term>
##       <term><![CDATA[Product development]]></term>
##       <term><![CDATA[Product life cycle management]]></term>
##       <term><![CDATA[Prototypes]]></term>
##       <term><![CDATA[Three dimensional displays]]></term>
##       <term><![CDATA[Time to market]]></term>
##     </thesaurusterms>
##     <pubtitle><![CDATA[Access, IEEE]]></pubtitle>
##     <punumber><![CDATA[6287639]]></punumber>
##     <pubtype><![CDATA[Journals & Magazines]]></pubtype>
##     <publisher><![CDATA[IEEE]]></publisher>
##     <volume><![CDATA[2]]></volume>
##     <py><![CDATA[2014]]></py>
##     <spage><![CDATA[234]]></spage>
##     <epage><![CDATA[242]]></epage>
##     <abstract><![CDATA[In new product development, time to market (TTM) is critical for the success and profitability of next generation products. When these products include sophisticated electronics encased in 3D packaging with complex geometries and intricate detail, TTM can be compromised - resulting in lost opportunity. The use of advanced 3D printing technology enhanced with component placement and electrical interconnect deposition can provide electronic prototypes that now can be rapidly fabricated in comparable time frames as traditional 2D bread-boarded prototypes; however, these 3D prototypes include the advantage of being embedded within more appropriate shapes in order to authentically prototype products earlier in the development cycle. The fabrication freedom offered by 3D printing techniques, such as stereolithography and fused deposition modeling have recently been explored in the context of 3D electronics integration - referred to as 3D structural electronics or 3D printed electronics. Enhanced 3D printing may eventually be employed to manufacture end-use parts and thus offer unit-level customization with local manufacturing; however, until the materials and dimensional accuracies improve (an eventuality), 3D printing technologies can be employed to reduce development times by providing advanced geometrically appropriate electronic prototypes. This paper describes the development process used to design a novelty six-sided gaming die. The die includes a microprocessor and accelerometer, which together detect motion and upon halting, identify the top surface through gravity and illuminate light-emitting diodes for a striking effect. By applying 3D printing of structural electronics to expedite prototyping, the development cycle was reduced from weeks to hours.]]></abstract>
##     <issn><![CDATA[2169-3536]]></issn>
##     <htmlFlag><![CDATA[1]]></htmlFlag>
##     <arnumber><![CDATA[6766751]]></arnumber>
##     <doi><![CDATA[10.1109/ACCESS.2014.2311810]]></doi>
##     <publicationId><![CDATA[6766751]]></publicationId>
##     <mdurl><![CDATA[http://ieeexplore.ieee.org/xpl/articleDetails.jsp?tp=&arnumber=6766751&contentType=Journals+%26+Magazines]]></mdurl>
##     <pdf><![CDATA[http://ieeexplore.ieee.org/stamp/stamp.jsp?arnumber=6766751]]></pdf>
##   </document>
##   <document>
##     <rank>490</rank>
##     <title><![CDATA[Novel Method for Improving the Capacity of Optical MIMO System Using MGDM]]></title>
##     <authors><![CDATA[Baklouti, F.;  Dayoub, I.;  Haxha, S.;  Attia, R.;  Aggoun, A.]]></authors>
##     <affiliations><![CDATA[Polytech. Sch. of Tunisia (EPT), Tunis, Tunisia]]></affiliations>
##     <controlledterms>
##       <term><![CDATA[MIMO communication]]></term>
##       <term><![CDATA[multiplexing]]></term>
##       <term><![CDATA[optical fibre communication]]></term>
##     </controlledterms>
##     <thesaurusterms>
##       <term><![CDATA[Optical fiber LAN]]></term>
##       <term><![CDATA[Optical fiber cables]]></term>
##       <term><![CDATA[Optical fiber dispersion]]></term>
##       <term><![CDATA[Optical fibers]]></term>
##       <term><![CDATA[Optimization]]></term>
##       <term><![CDATA[Receivers]]></term>
##     </thesaurusterms>
##     <pubtitle><![CDATA[Photonics Journal, IEEE]]></pubtitle>
##     <punumber><![CDATA[4563994]]></punumber>
##     <pubtype><![CDATA[Journals & Magazines]]></pubtype>
##     <publisher><![CDATA[IEEE]]></publisher>
##     <volume><![CDATA[6]]></volume>
##     <issue><![CDATA[6]]></issue>
##     <py><![CDATA[2014]]></py>
##     <spage><![CDATA[1]]></spage>
##     <epage><![CDATA[15]]></epage>
##     <abstract><![CDATA[In current local area networks, multimode fibers (MMFs), primarily graded index (GI) MMFs, are the main types of fibers employed for data communications. Due to their enormous bandwidth, it is considered that they are the main channel medium that can offer broadband multiservices using optical multiplexing techniques. Amongst these, mode group diversity multiplexing (MGDM) has been proposed as a way to integrate various services over an MMF network by exciting different groups of modes that can be used as independent and parallel communication channels. In this paper, we study optical multiple-input-multiple-output (O-MIMO) systems using MGDM techniques while also optimizing the launching conditions of light at the fiber inputs and the spot size, radial offset, angular offset, wavelength, and the radii of the segment areas of the detectors. We propose a new approach based on the optimization of launching and detection conditions in order to increase the capacity of an O-MIMO link using the MGDM technique. We propose a (3 &#x00D7; 3) O-MIMO system, where our simulation results show significant improvement in GI MMFs' capacity compared with existing O-MIMO systems.]]></abstract>
##     <issn><![CDATA[1943-0655]]></issn>
##     <htmlFlag><![CDATA[1]]></htmlFlag>
##     <arnumber><![CDATA[6926770]]></arnumber>
##     <doi><![CDATA[10.1109/JPHOT.2014.2363444]]></doi>
##     <publicationId><![CDATA[6926770]]></publicationId>
##     <mdurl><![CDATA[http://ieeexplore.ieee.org/xpl/articleDetails.jsp?tp=&arnumber=6926770&contentType=Journals+%26+Magazines]]></mdurl>
##     <pdf><![CDATA[http://ieeexplore.ieee.org/stamp/stamp.jsp?arnumber=6926770]]></pdf>
##   </document>
##   <document>
##     <rank>491</rank>
##     <title><![CDATA[On Sparse Representation in Fourier and Local Bases]]></title>
##     <authors><![CDATA[Dragotti, P.L.;  Lu, Y.M.]]></authors>
##     <affiliations><![CDATA[Dept. of Electr. & Electron. Eng., Imperial Coll. London, London, UK]]></affiliations>
##     <controlledterms>
##       <term><![CDATA[Fourier transforms]]></term>
##       <term><![CDATA[computational complexity]]></term>
##       <term><![CDATA[discrete cosine transforms]]></term>
##       <term><![CDATA[polynomials]]></term>
##       <term><![CDATA[signal representation]]></term>
##     </controlledterms>
##     <thesaurusterms>
##       <term><![CDATA[Coherence]]></term>
##       <term><![CDATA[Complexity theory]]></term>
##       <term><![CDATA[Dictionaries]]></term>
##       <term><![CDATA[Polynomials]]></term>
##       <term><![CDATA[Signal processing algorithms]]></term>
##       <term><![CDATA[Sparse matrices]]></term>
##       <term><![CDATA[Vectors]]></term>
##     </thesaurusterms>
##     <pubtitle><![CDATA[Information Theory, IEEE Transactions on]]></pubtitle>
##     <punumber><![CDATA[18]]></punumber>
##     <pubtype><![CDATA[Journals & Magazines]]></pubtype>
##     <publisher><![CDATA[IEEE]]></publisher>
##     <volume><![CDATA[60]]></volume>
##     <issue><![CDATA[12]]></issue>
##     <py><![CDATA[2014]]></py>
##     <spage><![CDATA[7888]]></spage>
##     <epage><![CDATA[7899]]></epage>
##     <abstract><![CDATA[We consider the classical problem of finding the sparse representation of a signal in a pair of bases. When both bases are orthogonal, it is known that the sparse representation is unique when the sparsity K of the signal satisfies K &lt;; 1/&#x03BC;(D), where &#x03BC;(D) is the mutual coherence of the dictionary. Furthermore, the sparse representation can be obtained in polynomial time by basis pursuit (BP), when K &lt;; 0.91/&#x03BC;(D). Therefore, there is a gap between the unicity condition and the one required to use the polynomial-complexity BP formulation. For the case of general dictionaries, it is also well known that finding the sparse representation under the only constraint of unicity is NP-hard. In this paper, we introduce, for the case of Fourier and canonical bases, a polynomial complexity algorithm that finds all the possible K-sparse representations of a signal under the weaker condition that K &lt;; &#x221A;2/&#x03BC;(D). Consequently, when K &lt;; 1/&#x03BC;(D), the proposed algorithm solves the unique sparse representation problem for this structured dictionary in polynomial time. We further show that the same method can be extended to many other pairs of bases, one of which must have local atoms. Examples include the union of Fourier and local Fourier bases, the union of discrete cosine transform and canonical bases, and the union of random Gaussian and canonical bases.]]></abstract>
##     <issn><![CDATA[0018-9448]]></issn>
##     <htmlFlag><![CDATA[1]]></htmlFlag>
##     <arnumber><![CDATA[6918471]]></arnumber>
##     <doi><![CDATA[10.1109/TIT.2014.2361858]]></doi>
##     <publicationId><![CDATA[6918471]]></publicationId>
##     <mdurl><![CDATA[http://ieeexplore.ieee.org/xpl/articleDetails.jsp?tp=&arnumber=6918471&contentType=Journals+%26+Magazines]]></mdurl>
##     <pdf><![CDATA[http://ieeexplore.ieee.org/stamp/stamp.jsp?arnumber=6918471]]></pdf>
##   </document>
##   <document>
##     <rank>492</rank>
##     <title><![CDATA[Challenges of System-Level Simulations and Performance Evaluation for 5G Wireless Networks]]></title>
##     <authors><![CDATA[Ying Wang;  Jing Xu;  Lisi Jiang]]></authors>
##     <affiliations><![CDATA[State Key Lab. of Networking & Switching Technol., Beijing Univ. of Posts & Telecommun., Beijing, China]]></affiliations>
##     <controlledterms>
##       <term><![CDATA[5G mobile communication]]></term>
##       <term><![CDATA[performance evaluation]]></term>
##     </controlledterms>
##     <thesaurusterms>
##       <term><![CDATA[5G mobile communication]]></term>
##       <term><![CDATA[Cellular networks]]></term>
##       <term><![CDATA[Long Term Evolution]]></term>
##       <term><![CDATA[Performance evaluation]]></term>
##       <term><![CDATA[System level design and analysis]]></term>
##       <term><![CDATA[Wireless communication]]></term>
##     </thesaurusterms>
##     <pubtitle><![CDATA[Access, IEEE]]></pubtitle>
##     <punumber><![CDATA[6287639]]></punumber>
##     <pubtype><![CDATA[Journals & Magazines]]></pubtype>
##     <publisher><![CDATA[IEEE]]></publisher>
##     <volume><![CDATA[2]]></volume>
##     <py><![CDATA[2014]]></py>
##     <spage><![CDATA[1553]]></spage>
##     <epage><![CDATA[1561]]></epage>
##     <abstract><![CDATA[With the evaluation and simulation of long-term evolution/4G cellular network and hot discussion about new technologies or network architecture for 5G, the appearance of simulation and evaluation guidelines for 5G is in urgent need. This paper analyzes the challenges of building a simulation platform for 5G considering the emerging new technologies and network architectures. Based on the overview of evaluation methodologies issued for 4G candidates, challenges in 5G evaluation are formulated. Additionally, a cloud-based two-level framework of system-level simulator is proposed to validate the candidate technologies and fulfill the promising technology performance identified for 5G.]]></abstract>
##     <issn><![CDATA[2169-3536]]></issn>
##     <htmlFlag><![CDATA[1]]></htmlFlag>
##     <arnumber><![CDATA[6994948]]></arnumber>
##     <doi><![CDATA[10.1109/ACCESS.2014.2383833]]></doi>
##     <publicationId><![CDATA[6994948]]></publicationId>
##     <mdurl><![CDATA[http://ieeexplore.ieee.org/xpl/articleDetails.jsp?tp=&arnumber=6994948&contentType=Journals+%26+Magazines]]></mdurl>
##     <pdf><![CDATA[http://ieeexplore.ieee.org/stamp/stamp.jsp?arnumber=6994948]]></pdf>
##   </document>
##   <document>
##     <rank>493</rank>
##     <title><![CDATA[Multichannel Electrophysiological Spike Sorting via Joint Dictionary Learning and Mixture Modeling]]></title>
##     <authors><![CDATA[Carlson, D.E.;  Vogelstein, J.T.;  Qisong Wu;  Wenzhao Lian;  Mingyuan Zhou;  Stoetzner, C.R.;  Kipke, D.;  Weber, D.;  Dunson, D.B.;  Carin, L.]]></authors>
##     <affiliations><![CDATA[Dept. of Electr. & Comput. Eng., Duke Univ., Durham, NC, USA]]></affiliations>
##     <controlledterms>
##       <term><![CDATA[Kalman filters]]></term>
##       <term><![CDATA[bioelectric potentials]]></term>
##       <term><![CDATA[data acquisition]]></term>
##       <term><![CDATA[learning (artificial intelligence)]]></term>
##       <term><![CDATA[medical signal processing]]></term>
##       <term><![CDATA[mixture models]]></term>
##       <term><![CDATA[neurophysiology]]></term>
##     </controlledterms>
##     <thesaurusterms>
##       <term><![CDATA[Bayes methods]]></term>
##       <term><![CDATA[Computational modeling]]></term>
##       <term><![CDATA[Data models]]></term>
##       <term><![CDATA[Dictionaries]]></term>
##       <term><![CDATA[Mathematical model]]></term>
##       <term><![CDATA[Neurons]]></term>
##       <term><![CDATA[Sorting]]></term>
##     </thesaurusterms>
##     <pubtitle><![CDATA[Biomedical Engineering, IEEE Transactions on]]></pubtitle>
##     <punumber><![CDATA[10]]></punumber>
##     <pubtype><![CDATA[Journals & Magazines]]></pubtype>
##     <publisher><![CDATA[IEEE]]></publisher>
##     <volume><![CDATA[61]]></volume>
##     <issue><![CDATA[1]]></issue>
##     <py><![CDATA[2014]]></py>
##     <spage><![CDATA[41]]></spage>
##     <epage><![CDATA[54]]></epage>
##     <abstract><![CDATA[We propose a methodology for joint feature learning and clustering of multichannel extracellular electrophysiological data, across multiple recording periods for action potential detection and classification (sorting). Our methodology improves over the previous state of the art principally in four ways. First, via sharing information across channels, we can better distinguish between single-unit spikes and artifacts. Second, our proposed &#x201C;focused mixture model&#x201D; (FMM) deals with units appearing, disappearing, or reappearing over multiple recording days, an important consideration for any chronic experiment. Third, by jointly learning features and clusters, we improve performance over previous attempts that proceeded via a two-stage learning process. Fourth, by directly modeling spike rate, we improve the detection of sparsely firing neurons. Moreover, our Bayesian methodology seamlessly handles missing data. We present the state-of-the-art performance without requiring manually tuning hyperparameters, considering both a public dataset with partial ground truth and a new experimental dataset.]]></abstract>
##     <issn><![CDATA[0018-9294]]></issn>
##     <htmlFlag><![CDATA[1]]></htmlFlag>
##     <arnumber><![CDATA[6571240]]></arnumber>
##     <doi><![CDATA[10.1109/TBME.2013.2275751]]></doi>
##     <publicationId><![CDATA[6571240]]></publicationId>
##     <mdurl><![CDATA[http://ieeexplore.ieee.org/xpl/articleDetails.jsp?tp=&arnumber=6571240&contentType=Journals+%26+Magazines]]></mdurl>
##     <pdf><![CDATA[http://ieeexplore.ieee.org/stamp/stamp.jsp?arnumber=6571240]]></pdf>
##   </document>
##   <document>
##     <rank>494</rank>
##     <title><![CDATA[In-Network Aggregation for Vehicular <italic>Ad Hoc</italic> Networks]]></title>
##     <authors><![CDATA[Dietzel, S.;  Petit, J.;  Kargl, F.;  Scheuermann, B.]]></authors>
##     <affiliations><![CDATA[Inst. of Distrib. Syst., Univ. of Ulm, Ulm, Germany]]></affiliations>
##     <controlledterms>
##       <term><![CDATA[vehicular ad hoc networks]]></term>
##     </controlledterms>
##     <thesaurusterms>
##       <term><![CDATA[Bandwidth]]></term>
##       <term><![CDATA[Information systems]]></term>
##       <term><![CDATA[Protocols]]></term>
##       <term><![CDATA[Road traffic]]></term>
##       <term><![CDATA[Safety]]></term>
##       <term><![CDATA[Vehicles]]></term>
##       <term><![CDATA[Vehicular ad hoc networks]]></term>
##     </thesaurusterms>
##     <pubtitle><![CDATA[Communications Surveys & Tutorials, IEEE]]></pubtitle>
##     <punumber><![CDATA[9739]]></punumber>
##     <pubtype><![CDATA[Journals & Magazines]]></pubtype>
##     <publisher><![CDATA[IEEE]]></publisher>
##     <volume><![CDATA[16]]></volume>
##     <issue><![CDATA[4]]></issue>
##     <py><![CDATA[2014]]></py>
##     <spage><![CDATA[1909]]></spage>
##     <epage><![CDATA[1932]]></epage>
##     <abstract><![CDATA[In-network aggregation mechanisms for vehicular ad hoc networks (VANETs) aim at improving communication efficiency by summarizing information that is exchanged between vehicles. Summaries are calculated, while data items are generated in and forwarded through the network. Due to its high bandwidth saving potential, aggregation is a vital building block for many of the applications envisioned in VANETs. At the same time, the specific environment of VANETs calls for novel approaches to aggregation, which address their challenging requirements. In this paper, we survey and structure this active research field. We propose a generic model to describe and classify the proposed approaches, and we identify future research challenges.]]></abstract>
##     <issn><![CDATA[1553-877X]]></issn>
##     <htmlFlag><![CDATA[1]]></htmlFlag>
##     <arnumber><![CDATA[6805128]]></arnumber>
##     <doi><![CDATA[10.1109/COMST.2014.2320091]]></doi>
##     <publicationId><![CDATA[6805128]]></publicationId>
##     <mdurl><![CDATA[http://ieeexplore.ieee.org/xpl/articleDetails.jsp?tp=&arnumber=6805128&contentType=Journals+%26+Magazines]]></mdurl>
##     <pdf><![CDATA[http://ieeexplore.ieee.org/stamp/stamp.jsp?arnumber=6805128]]></pdf>
##   </document>
##   <document>
##     <rank>495</rank>
##     <title><![CDATA[Service-Specific Network Virtualization to Reduce Signaling Processing Loads in EPC/IMS]]></title>
##     <authors><![CDATA[Ito, M.;  Nakauchi, K.;  Shoji, Y.;  Nishinaga, N.;  Kitatsuji, Y.]]></authors>
##     <affiliations><![CDATA[Nat. Inst. of Inf. & Commun. Technol., Koganei, Japan]]></affiliations>
##     <controlledterms>
##       <term><![CDATA[IP networks]]></term>
##       <term><![CDATA[mobile radio]]></term>
##       <term><![CDATA[multimedia communication]]></term>
##       <term><![CDATA[virtual private networks]]></term>
##       <term><![CDATA[virtualisation]]></term>
##     </controlledterms>
##     <thesaurusterms>
##       <term><![CDATA[IP networks]]></term>
##       <term><![CDATA[Inspection]]></term>
##       <term><![CDATA[Mobile communication]]></term>
##       <term><![CDATA[Multimedia communication]]></term>
##       <term><![CDATA[Network virtualization]]></term>
##       <term><![CDATA[Resource management]]></term>
##       <term><![CDATA[Servers]]></term>
##       <term><![CDATA[Virtualization]]></term>
##     </thesaurusterms>
##     <pubtitle><![CDATA[Access, IEEE]]></pubtitle>
##     <punumber><![CDATA[6287639]]></punumber>
##     <pubtype><![CDATA[Journals & Magazines]]></pubtype>
##     <publisher><![CDATA[IEEE]]></publisher>
##     <volume><![CDATA[2]]></volume>
##     <py><![CDATA[2014]]></py>
##     <spage><![CDATA[1076]]></spage>
##     <epage><![CDATA[1084]]></epage>
##     <abstract><![CDATA[This paper proposes a service-specific network virtualization to address the tremendous increase in the signaling processing load in the evolved packet core and IP multimedia subsystem of a fifth-generation mobile communication system. The proposal creates several virtual networks that are composed of functions specialized for particular services on a mobile communication network and efficiently forwards a sequence of signaling messages to the appropriate virtual networks. Using a prototype system, this paper verifies the overheads costs of the proposal that are incurred during the inspection of packet application headers needed to appropriately forward signaling messages as well as the overheads incurred when replicating state information from one virtual network to another. This paper shows that the proposal can reduce the signaling processing load by ~25% under certain assumptions.]]></abstract>
##     <issn><![CDATA[2169-3536]]></issn>
##     <htmlFlag><![CDATA[1]]></htmlFlag>
##     <arnumber><![CDATA[6907942]]></arnumber>
##     <doi><![CDATA[10.1109/ACCESS.2014.2359059]]></doi>
##     <publicationId><![CDATA[6907942]]></publicationId>
##     <mdurl><![CDATA[http://ieeexplore.ieee.org/xpl/articleDetails.jsp?tp=&arnumber=6907942&contentType=Journals+%26+Magazines]]></mdurl>
##     <pdf><![CDATA[http://ieeexplore.ieee.org/stamp/stamp.jsp?arnumber=6907942]]></pdf>
##   </document>
##   <document>
##     <rank>496</rank>
##     <title><![CDATA[Errors in Shortened Far-Field Gain Measurement Due to Mutual Coupling]]></title>
##     <authors><![CDATA[Hirano, T.;  Hirokawa, J.;  Ando, M.]]></authors>
##     <affiliations><![CDATA[Dept. of Int. Dev. Eng., Tokyo Inst. of Technol., Tokyo, Japan]]></affiliations>
##     <controlledterms>
##       <term><![CDATA[dipole antennas]]></term>
##       <term><![CDATA[numerical analysis]]></term>
##     </controlledterms>
##     <thesaurusterms>
##       <term><![CDATA[Analytical models]]></term>
##       <term><![CDATA[Antenna measurements]]></term>
##       <term><![CDATA[Dipole antennas]]></term>
##       <term><![CDATA[Gain]]></term>
##       <term><![CDATA[Gain measurement]]></term>
##       <term><![CDATA[Mutual coupling]]></term>
##     </thesaurusterms>
##     <pubtitle><![CDATA[Antennas and Propagation, IEEE Transactions on]]></pubtitle>
##     <punumber><![CDATA[8]]></punumber>
##     <pubtype><![CDATA[Journals & Magazines]]></pubtype>
##     <publisher><![CDATA[IEEE]]></publisher>
##     <volume><![CDATA[62]]></volume>
##     <issue><![CDATA[10]]></issue>
##     <py><![CDATA[2014]]></py>
##     <spage><![CDATA[5386]]></spage>
##     <epage><![CDATA[5388]]></epage>
##     <abstract><![CDATA[In this study, errors in shortened far-field gain measurement caused by mutual coupling were investigated. In addition, an averaging technique was evaluated to enhance the accuracy of gain measurement. Numerical simulation of an analysis model comprising two half-wavelength dipole antennas was performed for verification. Discrepancies were observed in the calculated distance and gain because of mutual coupling between the two antennas. These differences reduce as the distance between the antennas increases. The proposed averaging technique was shown to enhance the accuracy of gain measurement.]]></abstract>
##     <issn><![CDATA[0018-926X]]></issn>
##     <htmlFlag><![CDATA[1]]></htmlFlag>
##     <arnumber><![CDATA[6863626]]></arnumber>
##     <doi><![CDATA[10.1109/TAP.2014.2342757]]></doi>
##     <publicationId><![CDATA[6863626]]></publicationId>
##     <mdurl><![CDATA[http://ieeexplore.ieee.org/xpl/articleDetails.jsp?tp=&arnumber=6863626&contentType=Journals+%26+Magazines]]></mdurl>
##     <pdf><![CDATA[http://ieeexplore.ieee.org/stamp/stamp.jsp?arnumber=6863626]]></pdf>
##   </document>
##   <document>
##     <rank>497</rank>
##     <title><![CDATA[Multivariable Dynamic Ankle Mechanical Impedance With Relaxed Muscles]]></title>
##     <authors><![CDATA[Hyunglae Lee;  Krebs, H.I.;  Hogan, N.]]></authors>
##     <affiliations><![CDATA[Sensory Motor Performance Program, Rehabilitation Inst. of Chicago, Chicago, IL, USA]]></affiliations>
##     <controlledterms>
##       <term><![CDATA[biomechanics]]></term>
##       <term><![CDATA[bone]]></term>
##       <term><![CDATA[medical disorders]]></term>
##       <term><![CDATA[muscle]]></term>
##       <term><![CDATA[neurophysiology]]></term>
##       <term><![CDATA[orthopaedics]]></term>
##     </controlledterms>
##     <thesaurusterms>
##       <term><![CDATA[Actuators]]></term>
##       <term><![CDATA[Bioimpedance]]></term>
##       <term><![CDATA[Biomechanics]]></term>
##       <term><![CDATA[Biomedical engineering]]></term>
##       <term><![CDATA[Joints]]></term>
##       <term><![CDATA[Muscles]]></term>
##       <term><![CDATA[Patient rehabilitation]]></term>
##     </thesaurusterms>
##     <pubtitle><![CDATA[Neural Systems and Rehabilitation Engineering, IEEE Transactions on]]></pubtitle>
##     <punumber><![CDATA[7333]]></punumber>
##     <pubtype><![CDATA[Journals & Magazines]]></pubtype>
##     <publisher><![CDATA[IEEE]]></publisher>
##     <volume><![CDATA[22]]></volume>
##     <issue><![CDATA[6]]></issue>
##     <py><![CDATA[2014]]></py>
##     <spage><![CDATA[1104]]></spage>
##     <epage><![CDATA[1114]]></epage>
##     <abstract><![CDATA[Neurological or biomechanical disorders may distort ankle mechanical impedance and thereby impair locomotor function. This paper presents a quantitative characterization of multivariable ankle mechanical impedance of young healthy subjects when their muscles were relaxed, to serve as a baseline to compare with pathophysiological ankle properties of biomechanically and/or neurologically impaired patients. Measurements using a highly backdrivable wearable ankle robot combined with multi-input multi-output stochastic system identification methods enabled reliable characterization of ankle mechanical impedance in two degrees-of-freedom (DOFs) simultaneously, the sagittal and frontal planes. The characterization included important ankle properties unavailable from single DOF studies: coupling between DOFs and anisotropy as a function of frequency. Ankle impedance in joint coordinates showed responses largely consistent with a second-order system consisting of inertia, viscosity, and stiffness in both seated (knee flexed) and standing (knee straightened) postures. Stiffness in the sagittal plane was greater than in the frontal plane and furthermore, was greater when standing than when seated, most likely due to the stretch of bi-articular muscles (medial and lateral gastrocnemius). Very low off-diagonal partial coherences implied negligible coupling between dorsiflexion-plantarflexion and inversion-eversion. The directions of principal axes were tilted slightly counterclockwise from the original joint coordinates. The directional variation (anisotropy) of ankle impedance in the 2-D space formed by rotations in the sagittal and frontal planes exhibited a characteristic &#x201C;peanut&#x201D; shape, weak in inversion-eversion over a wide range of frequencies from the stiffness dominated region up to the inertia dominated region. Implications for the assessment of neurological and biomechanical impairments are discussed.]]></abstract>
##     <issn><![CDATA[1534-4320]]></issn>
##     <htmlFlag><![CDATA[1]]></htmlFlag>
##     <arnumber><![CDATA[6778753]]></arnumber>
##     <doi><![CDATA[10.1109/TNSRE.2014.2313838]]></doi>
##     <publicationId><![CDATA[6778753]]></publicationId>
##     <mdurl><![CDATA[http://ieeexplore.ieee.org/xpl/articleDetails.jsp?tp=&arnumber=6778753&contentType=Journals+%26+Magazines]]></mdurl>
##     <pdf><![CDATA[http://ieeexplore.ieee.org/stamp/stamp.jsp?arnumber=6778753]]></pdf>
##   </document>
##   <document>
##     <rank>498</rank>
##     <title><![CDATA[Autonomous Document Cleaning&#x2014;A Generative Approach to Reconstruct Strongly Corrupted Scanned Texts]]></title>
##     <authors><![CDATA[Zhenwen Dai;  Lucke, J.]]></authors>
##     <affiliations><![CDATA[Dept. of Comput. Sci., Univ. of Sheffield, Sheffield, UK]]></affiliations>
##     <controlledterms>
##       <term><![CDATA[document image processing]]></term>
##       <term><![CDATA[expectation-maximisation algorithm]]></term>
##       <term><![CDATA[feature extraction]]></term>
##       <term><![CDATA[image reconstruction]]></term>
##       <term><![CDATA[image representation]]></term>
##       <term><![CDATA[learning (artificial intelligence)]]></term>
##       <term><![CDATA[natural language processing]]></term>
##       <term><![CDATA[probability]]></term>
##       <term><![CDATA[text analysis]]></term>
##       <term><![CDATA[variational techniques]]></term>
##     </controlledterms>
##     <thesaurusterms>
##       <term><![CDATA[Approximation methods]]></term>
##       <term><![CDATA[Computational modeling]]></term>
##       <term><![CDATA[Data models]]></term>
##       <term><![CDATA[Histograms]]></term>
##       <term><![CDATA[Probabilistic logic]]></term>
##       <term><![CDATA[Vectors]]></term>
##       <term><![CDATA[Visualization]]></term>
##     </thesaurusterms>
##     <pubtitle><![CDATA[Pattern Analysis and Machine Intelligence, IEEE Transactions on]]></pubtitle>
##     <punumber><![CDATA[34]]></punumber>
##     <pubtype><![CDATA[Journals & Magazines]]></pubtype>
##     <publisher><![CDATA[IEEE]]></publisher>
##     <volume><![CDATA[36]]></volume>
##     <issue><![CDATA[10]]></issue>
##     <py><![CDATA[2014]]></py>
##     <spage><![CDATA[1950]]></spage>
##     <epage><![CDATA[1962]]></epage>
##     <abstract><![CDATA[We study the task of cleaning scanned text documents that are strongly corrupted by dirt such as manual line strokes, spilled ink, etc. We aim at autonomously removing such corruptions from a single letter-size page based only on the information the page contains. Our approach first learns character representations from document patches without supervision. For learning, we use a probabilistic generative model parameterizing pattern features, their planar arrangements and their variances. The model's latent variables describe pattern position and class, and feature occurrences. Model parameters are efficiently inferred using a truncated variational EM approach. Based on the learned representation, a clean document can be recovered by identifying, for each patch, pattern class and position while a quality measure allows for discrimination between character and non-character patterns. For a full Latin alphabet we found that a single page does not contain sufficiently many character examples. However, even if heavily corrupted by dirt, we show that a page containing a lower number of character types can efficiently and autonomously be cleaned solely based on the structural regularity of the characters it contains. In different example applications with different alphabets, we demonstrate and discuss the effectiveness, efficiency and generality of the approach.]]></abstract>
##     <issn><![CDATA[0162-8828]]></issn>
##     <htmlFlag><![CDATA[1]]></htmlFlag>
##     <arnumber><![CDATA[6777544]]></arnumber>
##     <doi><![CDATA[10.1109/TPAMI.2014.2313126]]></doi>
##     <publicationId><![CDATA[6777544]]></publicationId>
##     <mdurl><![CDATA[http://ieeexplore.ieee.org/xpl/articleDetails.jsp?tp=&arnumber=6777544&contentType=Journals+%26+Magazines]]></mdurl>
##     <pdf><![CDATA[http://ieeexplore.ieee.org/stamp/stamp.jsp?arnumber=6777544]]></pdf>
##   </document>
##   <document>
##     <rank>499</rank>
##     <title><![CDATA[Empirical Emission Eigenmodes of Printed Circuit Boards]]></title>
##     <authors><![CDATA[Arnaut, L.R.;  Obiekezie, C.S.;  Thomas, D.W.P.]]></authors>
##     <affiliations><![CDATA[George Green Inst. of Electromagn. Res., Univ. of Nottingham, Nottingham, UK]]></affiliations>
##     <controlledterms>
##       <term><![CDATA[correlation methods]]></term>
##       <term><![CDATA[eigenvalues and eigenfunctions]]></term>
##       <term><![CDATA[magnetic fields]]></term>
##       <term><![CDATA[principal component analysis]]></term>
##       <term><![CDATA[printed circuit testing]]></term>
##       <term><![CDATA[printed circuits]]></term>
##     </controlledterms>
##     <thesaurusterms>
##       <term><![CDATA[Correlation]]></term>
##       <term><![CDATA[Loading]]></term>
##       <term><![CDATA[Principal component analysis]]></term>
##       <term><![CDATA[Standards]]></term>
##       <term><![CDATA[Time-frequency analysis]]></term>
##       <term><![CDATA[Vectors]]></term>
##     </thesaurusterms>
##     <pubtitle><![CDATA[Electromagnetic Compatibility, IEEE Transactions on]]></pubtitle>
##     <punumber><![CDATA[15]]></punumber>
##     <pubtype><![CDATA[Journals & Magazines]]></pubtype>
##     <publisher><![CDATA[IEEE]]></publisher>
##     <volume><![CDATA[56]]></volume>
##     <issue><![CDATA[3]]></issue>
##     <py><![CDATA[2014]]></py>
##     <spage><![CDATA[715]]></spage>
##     <epage><![CDATA[725]]></epage>
##     <abstract><![CDATA[It is shown that radiated emissions from circuits can be efficiently decomposed and represented using principal component analysis (PCA). A hierarchical set of orthonormal spatial eigenmodes and their associated amplitudes are extracted from data for an emitted field that is scanned and measured across a planar area and wide frequency band. This decomposition is applied to the magnetic tangential field intensity for emissions from an analog transmission-line based circuit. The inhomogeneity of the autocorrelation matrix driving the PCA algorithm is experimentally demonstrated. Eigenmodes, loadings, and scores of the principal components are calculated and interpreted physically. Across the selected frequency band, PCA achieves efficiency savings of one to two orders or magnitude for the representation of emitted radiation.]]></abstract>
##     <issn><![CDATA[0018-9375]]></issn>
##     <htmlFlag><![CDATA[1]]></htmlFlag>
##     <arnumber><![CDATA[6681940]]></arnumber>
##     <doi><![CDATA[10.1109/TEMC.2013.2292548]]></doi>
##     <publicationId><![CDATA[6681940]]></publicationId>
##     <mdurl><![CDATA[http://ieeexplore.ieee.org/xpl/articleDetails.jsp?tp=&arnumber=6681940&contentType=Journals+%26+Magazines]]></mdurl>
##     <pdf><![CDATA[http://ieeexplore.ieee.org/stamp/stamp.jsp?arnumber=6681940]]></pdf>
##   </document>
##   <document>
##     <rank>500</rank>
##     <title><![CDATA[Calibration of a Novel Microstructural Damage Model for Wire Bonds]]></title>
##     <authors><![CDATA[Yang, L.;  Agyakwa, P.A.;  Johnson, C.M.]]></authors>
##     <affiliations><![CDATA[Dept. of Electr. & Electron. Eng., Univ. of Nottingham, Nottingham, UK]]></affiliations>
##     <controlledterms>
##       <term><![CDATA[interconnections]]></term>
##       <term><![CDATA[lead bonding]]></term>
##       <term><![CDATA[power electronics]]></term>
##       <term><![CDATA[time-domain analysis]]></term>
##     </controlledterms>
##     <thesaurusterms>
##       <term><![CDATA[Equations]]></term>
##       <term><![CDATA[Mathematical model]]></term>
##       <term><![CDATA[Plastics]]></term>
##       <term><![CDATA[Strain]]></term>
##       <term><![CDATA[Stress]]></term>
##       <term><![CDATA[Temperature dependence]]></term>
##       <term><![CDATA[Wires]]></term>
##     </thesaurusterms>
##     <pubtitle><![CDATA[Device and Materials Reliability, IEEE Transactions on]]></pubtitle>
##     <punumber><![CDATA[7298]]></punumber>
##     <pubtype><![CDATA[Journals & Magazines]]></pubtype>
##     <publisher><![CDATA[IEEE]]></publisher>
##     <volume><![CDATA[14]]></volume>
##     <issue><![CDATA[4]]></issue>
##     <py><![CDATA[2014]]></py>
##     <spage><![CDATA[989]]></spage>
##     <epage><![CDATA[994]]></epage>
##     <abstract><![CDATA[In a previous paper, a new time-domain damage-based physics model was proposed for the lifetime prediction of wire bond interconnects in power electronic modules. Unlike cycle-dependent life prediction methodologies, this model innovatively incorporates temperature- and time-dependent properties so that rate-sensitive processes associated with the bond degradation can be accurately represented. This paper presents the work on the development and calibration of the damage model by linking its core parameter, i.e., &#x201C;damage,&#x201D; to the strain energy density, which is a physically quantifiable materials property. Isothermal uniaxial tensile data for unbonded pure aluminum wires (99.999%) have been used to develop constitutive functions, and the model has been calibrated by the derived values of the strain energy density.]]></abstract>
##     <issn><![CDATA[1530-4388]]></issn>
##     <htmlFlag><![CDATA[1]]></htmlFlag>
##     <arnumber><![CDATA[6891277]]></arnumber>
##     <doi><![CDATA[10.1109/TDMR.2014.2354739]]></doi>
##     <publicationId><![CDATA[6891277]]></publicationId>
##     <mdurl><![CDATA[http://ieeexplore.ieee.org/xpl/articleDetails.jsp?tp=&arnumber=6891277&contentType=Journals+%26+Magazines]]></mdurl>
##     <pdf><![CDATA[http://ieeexplore.ieee.org/stamp/stamp.jsp?arnumber=6891277]]></pdf>
##   </document>
##   <document>
##     <rank>501</rank>
##     <title><![CDATA[Transitioning to Physics-of-Failure as a Reliability Driver in Power Electronics]]></title>
##     <authors><![CDATA[Huai Wang;  Liserre, M.;  Blaabjerg, F.;  de Place Rimmen, P.;  Jacobsen, J.B.;  Kvisgaard, T.;  Landkildehus, J.]]></authors>
##     <affiliations><![CDATA[Dept. of Energy Technol., Aalborg Univ., Aalborg, Denmark]]></affiliations>
##     <controlledterms>
##       <term><![CDATA[condition monitoring]]></term>
##       <term><![CDATA[failure analysis]]></term>
##       <term><![CDATA[insulated gate bipolar transistors]]></term>
##       <term><![CDATA[power semiconductor devices]]></term>
##       <term><![CDATA[semiconductor device reliability]]></term>
##     </controlledterms>
##     <thesaurusterms>
##       <term><![CDATA[Failure analysis]]></term>
##       <term><![CDATA[Insulated gate bipolar transistors]]></term>
##       <term><![CDATA[Reliability engineering]]></term>
##       <term><![CDATA[Robustness]]></term>
##       <term><![CDATA[Stress]]></term>
##     </thesaurusterms>
##     <pubtitle><![CDATA[Emerging and Selected Topics in Power Electronics, IEEE Journal of]]></pubtitle>
##     <punumber><![CDATA[6245517]]></punumber>
##     <pubtype><![CDATA[Journals & Magazines]]></pubtype>
##     <publisher><![CDATA[IEEE]]></publisher>
##     <volume><![CDATA[2]]></volume>
##     <issue><![CDATA[1]]></issue>
##     <py><![CDATA[2014]]></py>
##     <spage><![CDATA[97]]></spage>
##     <epage><![CDATA[114]]></epage>
##     <abstract><![CDATA[Power electronics has progressively gained an important status in power generation, distribution, and consumption. With more than 70% of electricity processed through power electronics, recent research endeavors to improve the reliability of power electronic systems to comply with more stringent constraints on cost, safety, and availability in various applications. This paper serves to give an overview of the major aspects of reliability in power electronics and to address the future trends in this multidisciplinary research direction. The ongoing paradigm shift in reliability research is presented first. Then, the three major aspects of power electronics reliability are discussed, respectively, which cover physics-of-failure analysis of critical power electronic components, state-of-the-art design for reliability process and robustness validation, and intelligent control and condition monitoring to achieve improved reliability under operation. Finally, the challenges and opportunities for achieving more reliable power electronic systems in the future are discussed.]]></abstract>
##     <issn><![CDATA[2168-6777]]></issn>
##     <htmlFlag><![CDATA[1]]></htmlFlag>
##     <arnumber><![CDATA[6661372]]></arnumber>
##     <doi><![CDATA[10.1109/JESTPE.2013.2290282]]></doi>
##     <publicationId><![CDATA[6661372]]></publicationId>
##     <mdurl><![CDATA[http://ieeexplore.ieee.org/xpl/articleDetails.jsp?tp=&arnumber=6661372&contentType=Journals+%26+Magazines]]></mdurl>
##     <pdf><![CDATA[http://ieeexplore.ieee.org/stamp/stamp.jsp?arnumber=6661372]]></pdf>
##   </document>
##   <document>
##     <rank>502</rank>
##     <title><![CDATA[High-Bandgap Solar Cells for Underwater Photovoltaic Applications]]></title>
##     <authors><![CDATA[Jenkins, P.P.;  Messenger, S.;  Trautz, K.M.;  Maximenko, S.I.;  Goldstein, D.;  Scheiman, D.;  Hoheisel, R.;  Walters, R.J.]]></authors>
##     <affiliations><![CDATA[US Naval Res. Lab., Washington, DC, USA]]></affiliations>
##     <controlledterms>
##       <term><![CDATA[III-V semiconductors]]></term>
##       <term><![CDATA[energy gap]]></term>
##       <term><![CDATA[gallium compounds]]></term>
##       <term><![CDATA[indium compounds]]></term>
##       <term><![CDATA[installation]]></term>
##       <term><![CDATA[secondary cells]]></term>
##       <term><![CDATA[solar cell arrays]]></term>
##     </controlledterms>
##     <thesaurusterms>
##       <term><![CDATA[Absorption]]></term>
##       <term><![CDATA[Photovoltaic cells]]></term>
##       <term><![CDATA[Photovoltaic systems]]></term>
##       <term><![CDATA[Silicon]]></term>
##       <term><![CDATA[Temperature measurement]]></term>
##       <term><![CDATA[Water]]></term>
##     </thesaurusterms>
##     <pubtitle><![CDATA[Photovoltaics, IEEE Journal of]]></pubtitle>
##     <punumber><![CDATA[5503869]]></punumber>
##     <pubtype><![CDATA[Journals & Magazines]]></pubtype>
##     <publisher><![CDATA[IEEE]]></publisher>
##     <volume><![CDATA[4]]></volume>
##     <issue><![CDATA[1]]></issue>
##     <py><![CDATA[2014]]></py>
##     <spage><![CDATA[202]]></spage>
##     <epage><![CDATA[207]]></epage>
##     <abstract><![CDATA[Autonomous systems are increasingly used to provide situational awareness and long-term environment monitoring. Photovoltaics (PV) are favored as a long-endurance power source for many of these applications. To date, the use of PV is limited to space and terrestrial (dry-land) installations. The need for a persistent power source also exists for underwater (UW) systems, which currently rely on surface PV arrays or batteries. In this paper, we demonstrate that high-bandgap-InGaP solar cells can provide useful power UW.]]></abstract>
##     <issn><![CDATA[2156-3381]]></issn>
##     <htmlFlag><![CDATA[1]]></htmlFlag>
##     <arnumber><![CDATA[6631466]]></arnumber>
##     <doi><![CDATA[10.1109/JPHOTOV.2013.2283578]]></doi>
##     <publicationId><![CDATA[6631466]]></publicationId>
##     <mdurl><![CDATA[http://ieeexplore.ieee.org/xpl/articleDetails.jsp?tp=&arnumber=6631466&contentType=Journals+%26+Magazines]]></mdurl>
##     <pdf><![CDATA[http://ieeexplore.ieee.org/stamp/stamp.jsp?arnumber=6631466]]></pdf>
##   </document>
##   <document>
##     <rank>503</rank>
##     <title><![CDATA[Mobile Trusted Computing]]></title>
##     <authors><![CDATA[Asokan, N.;  Ekberg, J.-E.;  Kostiainen, K.;  Rajan, A.;  Rozas, C.;  Sadeghi, A.-R.;  Schulz, S.;  Wachsmann, C.]]></authors>
##     <affiliations><![CDATA[Intel Collaborative Res. Inst. for Secure Comput., Univ. of Helsinki, Aalto, Finland]]></affiliations>
##     <controlledterms>
##       <term><![CDATA[mobile computing]]></term>
##       <term><![CDATA[research and development]]></term>
##       <term><![CDATA[standardisation]]></term>
##       <term><![CDATA[trusted computing]]></term>
##     </controlledterms>
##     <thesaurusterms>
##       <term><![CDATA[Computer architecture]]></term>
##       <term><![CDATA[Cryptography]]></term>
##       <term><![CDATA[Hardware]]></term>
##       <term><![CDATA[Integrated circuit modeling]]></term>
##       <term><![CDATA[Mobile communication]]></term>
##       <term><![CDATA[Mobile handsets]]></term>
##       <term><![CDATA[Software engineering]]></term>
##     </thesaurusterms>
##     <pubtitle><![CDATA[Proceedings of the IEEE]]></pubtitle>
##     <punumber><![CDATA[5]]></punumber>
##     <pubtype><![CDATA[Journals & Magazines]]></pubtype>
##     <publisher><![CDATA[IEEE]]></publisher>
##     <volume><![CDATA[102]]></volume>
##     <issue><![CDATA[8]]></issue>
##     <py><![CDATA[2014]]></py>
##     <spage><![CDATA[1189]]></spage>
##     <epage><![CDATA[1206]]></epage>
##     <abstract><![CDATA[Trusted computing technologies for mobile devices have been researched, developed, and deployed over the past decade. Although their use has been limited so far, ongoing standardization may change this by opening up these technologies for easy access by developers and users. In this survey, we describe the current state of trusted computing solutions for mobile devices from research, standardization, and deployment perspectives.]]></abstract>
##     <issn><![CDATA[0018-9219]]></issn>
##     <htmlFlag><![CDATA[1]]></htmlFlag>
##     <arnumber><![CDATA[6856168]]></arnumber>
##     <doi><![CDATA[10.1109/JPROC.2014.2332007]]></doi>
##     <publicationId><![CDATA[6856168]]></publicationId>
##     <mdurl><![CDATA[http://ieeexplore.ieee.org/xpl/articleDetails.jsp?tp=&arnumber=6856168&contentType=Journals+%26+Magazines]]></mdurl>
##     <pdf><![CDATA[http://ieeexplore.ieee.org/stamp/stamp.jsp?arnumber=6856168]]></pdf>
##   </document>
##   <document>
##     <rank>504</rank>
##     <title><![CDATA[Light Trapping in Thin-Film Cu(InGa)Se<formula formulatype="inline"> <img src="/images/tex/517.gif" alt="_{2}"> </formula> Solar Cells]]></title>
##     <authors><![CDATA[Mutitu, J.G.;  Obahiagbon, U.;  Shouyuan Shi;  Shafarman, W.;  Prather, D.W.]]></authors>
##     <affiliations><![CDATA[Dept. of Electr. & Comput. Eng., Univ. of Delaware, Newark, DE, USA]]></affiliations>
##     <controlledterms>
##       <term><![CDATA[copper compounds]]></term>
##       <term><![CDATA[gallium compounds]]></term>
##       <term><![CDATA[indium compounds]]></term>
##       <term><![CDATA[infrared spectra]]></term>
##       <term><![CDATA[optical constants]]></term>
##       <term><![CDATA[semiconductor device models]]></term>
##       <term><![CDATA[semiconductor thin films]]></term>
##       <term><![CDATA[solar cells]]></term>
##       <term><![CDATA[ternary semiconductors]]></term>
##       <term><![CDATA[thin film devices]]></term>
##       <term><![CDATA[ultraviolet spectra]]></term>
##       <term><![CDATA[visible spectra]]></term>
##     </controlledterms>
##     <thesaurusterms>
##       <term><![CDATA[Indium tin oxide]]></term>
##       <term><![CDATA[Optical materials]]></term>
##       <term><![CDATA[Photovoltaic cells]]></term>
##       <term><![CDATA[Silver]]></term>
##       <term><![CDATA[Substrates]]></term>
##       <term><![CDATA[Zinc oxide]]></term>
##     </thesaurusterms>
##     <pubtitle><![CDATA[Photovoltaics, IEEE Journal of]]></pubtitle>
##     <punumber><![CDATA[5503869]]></punumber>
##     <pubtype><![CDATA[Journals & Magazines]]></pubtype>
##     <publisher><![CDATA[IEEE]]></publisher>
##     <volume><![CDATA[4]]></volume>
##     <issue><![CDATA[3]]></issue>
##     <py><![CDATA[2014]]></py>
##     <spage><![CDATA[948]]></spage>
##     <epage><![CDATA[953]]></epage>
##     <abstract><![CDATA[A fundamental optical analysis of thin-film Cu(InGa)Se<sub>2</sub> solar cell structures is presented, wherein spectroscopic ellipsometry measurements were performed to acquire material optical constants, which were then used as input parameters to perform electromagnetic simulations. The accuracy of the electromagnetic simulation tools, and thus, the validity of the material optical constants, were verified by comparing the values determined from the simulations with experimental measurements obtained using a spectrophotometer. The verified optical modeling tools were then used to analyze thin, &lt;;0.7-&#x03BC;m Cu(InGa)Se<sub>2</sub> solar cell structures, which do not absorb all incident light within a single optical path length, and hence, the need to incorporate light trapping. To this end, a superstrate device configuration was employed in which the metallic back contact is deposited last, giving rise to an opportunity to incorporate photonic engineering device concepts to the back surface layer of the solar cell. Simulations of superstrate Cu(InGa)Se<sub>2</sub> solar cell designs, complete with light trapping structures were then performed and analyzed.]]></abstract>
##     <issn><![CDATA[2156-3381]]></issn>
##     <htmlFlag><![CDATA[1]]></htmlFlag>
##     <arnumber><![CDATA[6767033]]></arnumber>
##     <doi><![CDATA[10.1109/JPHOTOV.2014.2307487]]></doi>
##     <publicationId><![CDATA[6767033]]></publicationId>
##     <mdurl><![CDATA[http://ieeexplore.ieee.org/xpl/articleDetails.jsp?tp=&arnumber=6767033&contentType=Journals+%26+Magazines]]></mdurl>
##     <pdf><![CDATA[http://ieeexplore.ieee.org/stamp/stamp.jsp?arnumber=6767033]]></pdf>
##   </document>
##   <document>
##     <rank>505</rank>
##     <title><![CDATA[Adapting Transmitter Power and Modulation Format to Improve Optical Network Performance Utilizing the Gaussian Noise Model of Nonlinear Impairments]]></title>
##     <authors><![CDATA[Ives, D.J.;  Bayvel, P.;  Savory, S.J.]]></authors>
##     <affiliations><![CDATA[Dept. of Electron. & Electr. Eng., Univ. Coll. London, London, UK]]></affiliations>
##     <controlledterms>
##       <term><![CDATA[Gaussian noise]]></term>
##       <term><![CDATA[channel allocation]]></term>
##       <term><![CDATA[estimation theory]]></term>
##       <term><![CDATA[light interference]]></term>
##       <term><![CDATA[light transmission]]></term>
##       <term><![CDATA[optical fibre networks]]></term>
##       <term><![CDATA[optical modulation]]></term>
##       <term><![CDATA[optical transceivers]]></term>
##       <term><![CDATA[optimisation]]></term>
##       <term><![CDATA[telecommunication network routing]]></term>
##     </controlledterms>
##     <thesaurusterms>
##       <term><![CDATA[Channel allocation]]></term>
##       <term><![CDATA[Interference]]></term>
##       <term><![CDATA[Modulation]]></term>
##       <term><![CDATA[Optical transmitters]]></term>
##       <term><![CDATA[Optimization]]></term>
##       <term><![CDATA[Signal to noise ratio]]></term>
##     </thesaurusterms>
##     <pubtitle><![CDATA[Lightwave Technology, Journal of]]></pubtitle>
##     <punumber><![CDATA[50]]></punumber>
##     <pubtype><![CDATA[Journals & Magazines]]></pubtype>
##     <publisher><![CDATA[IEEE]]></publisher>
##     <volume><![CDATA[32]]></volume>
##     <issue><![CDATA[21]]></issue>
##     <py><![CDATA[2014]]></py>
##     <spage><![CDATA[4087]]></spage>
##     <epage><![CDATA[4096]]></epage>
##     <abstract><![CDATA[This paper serves to highlight the gains in SNR margin and/or data capacity that can be achieved through a proper optimization of the transceiver parameters, for example, launch power, modulation format, and channel allocation. A simple quality of transmission estimator is described that allows a rapid estimation of the signal quality based on ASE noise and nonlinear interference utilizing the Gaussian noise model. The quality of transmission estimator was used to optimize the SNR and maximise the data throughput of transmission signals in a point-to-point link by adjusting the launch power and modulation format. In a three-node network, the launch power and channel allocation were adjusted to minimise the overall effect of nonlinear interference. This paper goes on to show that by optimizing the transceiver modulation format as part of the channel allocation and routing problem gains in network data throughput can be achieved for the 14-node NSF mesh network.]]></abstract>
##     <issn><![CDATA[0733-8724]]></issn>
##     <htmlFlag><![CDATA[1]]></htmlFlag>
##     <arnumber><![CDATA[6874486]]></arnumber>
##     <doi><![CDATA[10.1109/JLT.2014.2346582]]></doi>
##     <publicationId><![CDATA[6874486]]></publicationId>
##     <mdurl><![CDATA[http://ieeexplore.ieee.org/xpl/articleDetails.jsp?tp=&arnumber=6874486&contentType=Journals+%26+Magazines]]></mdurl>
##     <pdf><![CDATA[http://ieeexplore.ieee.org/stamp/stamp.jsp?arnumber=6874486]]></pdf>
##   </document>
##   <document>
##     <rank>506</rank>
##     <title><![CDATA[On-Chip Nanoscale Capacitor Decoupling Architectures for Hardware Security]]></title>
##     <authors><![CDATA[Mayhew, M.;  Muresan, R.]]></authors>
##     <affiliations><![CDATA[Sch. of Eng., Univ. of Guelph, Guelph, ON, Canada]]></affiliations>
##     <controlledterms>
##       <term><![CDATA[CMOS integrated circuits]]></term>
##       <term><![CDATA[capacitors]]></term>
##       <term><![CDATA[cryptography]]></term>
##       <term><![CDATA[nanoelectronics]]></term>
##     </controlledterms>
##     <thesaurusterms>
##       <term><![CDATA[Capacitors]]></term>
##       <term><![CDATA[Computer architecture]]></term>
##       <term><![CDATA[Cryptography]]></term>
##       <term><![CDATA[Encryption]]></term>
##       <term><![CDATA[Nanoscale devices]]></term>
##       <term><![CDATA[Power demand]]></term>
##     </thesaurusterms>
##     <pubtitle><![CDATA[Emerging Topics in Computing, IEEE Transactions on]]></pubtitle>
##     <punumber><![CDATA[6245516]]></punumber>
##     <pubtype><![CDATA[Journals & Magazines]]></pubtype>
##     <publisher><![CDATA[IEEE]]></publisher>
##     <volume><![CDATA[2]]></volume>
##     <issue><![CDATA[1]]></issue>
##     <py><![CDATA[2014]]></py>
##     <spage><![CDATA[4]]></spage>
##     <epage><![CDATA[15]]></epage>
##     <abstract><![CDATA[This paper presents new power analysis attack (PAA) countermeasures for nanoscale cryptographic devices. Specifically, three circuit level architectures called partial decoupling architecture, full decoupling architecture, and randomized switch box architecture are developed and analyzed. The architectures' primary feature is the use of on-chip nMOS gate capacitors as intermediate power storage elements to decouple the power supply from internal low-power modules processing sensitive data. The proposed countermeasures are algorithm independent and allow different tradeoffs between security protection and the incurred overheads. Test benches of the proposed architectures were simulated in 65-nm TSMC CMOS technology. A correlation PAA was performed for each test bench targeting a custom implementation of the advanced encryption standard subbytes operation. All architectures were found to resist the correlation PAA at the power supply, with the more complex architectures also offering protection against invasive attacks. The success value indicator was used to analyze the effectiveness of the countermeasures. It was found that all architectures provided a negative value at the power supply, showing protection against PAAs. We demonstrate that the use of nMOS gate capacitors can help to increase security and present a feasibility analysis focused on the needed decoupling capacitances.]]></abstract>
##     <issn><![CDATA[2168-6750]]></issn>
##     <htmlFlag><![CDATA[1]]></htmlFlag>
##     <arnumber><![CDATA[6730921]]></arnumber>
##     <doi><![CDATA[10.1109/TETC.2014.2303934]]></doi>
##     <publicationId><![CDATA[6730921]]></publicationId>
##     <mdurl><![CDATA[http://ieeexplore.ieee.org/xpl/articleDetails.jsp?tp=&arnumber=6730921&contentType=Journals+%26+Magazines]]></mdurl>
##     <pdf><![CDATA[http://ieeexplore.ieee.org/stamp/stamp.jsp?arnumber=6730921]]></pdf>
##   </document>
##   <document>
##     <rank>507</rank>
##     <title><![CDATA[Soft-Switching Bidirectional Isolated Full-Bridge Converter With Active and Passive Snubbers]]></title>
##     <authors><![CDATA[Tsai-Fu Wu;  Jeng-Gung Yang;  Chia-Ling Kuo;  Yung-Chun Wu]]></authors>
##     <affiliations><![CDATA[Dept. of Electr. Eng., Nat. Tsing Hua Univ., Hsinchu, Taiwan]]></affiliations>
##     <controlledterms>
##       <term><![CDATA[snubbers]]></term>
##       <term><![CDATA[switching convertors]]></term>
##       <term><![CDATA[zero current switching]]></term>
##       <term><![CDATA[zero voltage switching]]></term>
##     </controlledterms>
##     <pubtitle><![CDATA[Industrial Electronics, IEEE Transactions on]]></pubtitle>
##     <punumber><![CDATA[41]]></punumber>
##     <pubtype><![CDATA[Journals & Magazines]]></pubtype>
##     <publisher><![CDATA[IEEE]]></publisher>
##     <volume><![CDATA[61]]></volume>
##     <issue><![CDATA[3]]></issue>
##     <py><![CDATA[2014]]></py>
##     <spage><![CDATA[1368]]></spage>
##     <epage><![CDATA[1376]]></epage>
##     <abstract><![CDATA[A bidirectional isolated full-bridge dc-dc converter with a conversion ratio around nine times, soft start-up, and soft-switching features for battery charging/discharging is proposed in this paper. The converter is equipped with an active flyback and two passive capacitor-diode snubbers, which can reduce voltage and current spikes and reduce voltage and current stresses, while it can achieve near zero-voltage-switching and zero-current-switching soft-switching features. In this paper, the operational principle of the proposed converter is first described, and its analysis and design are then presented. A 1.5-kW prototype with a low-side voltage of 48 V and a high-side voltage of 360 V has been implemented, from which measured results have verified the discussed features.]]></abstract>
##     <issn><![CDATA[0278-0046]]></issn>
##     <htmlFlag><![CDATA[1]]></htmlFlag>
##     <arnumber><![CDATA[6515376]]></arnumber>
##     <doi><![CDATA[10.1109/TIE.2013.2262746]]></doi>
##     <publicationId><![CDATA[6515376]]></publicationId>
##     <mdurl><![CDATA[http://ieeexplore.ieee.org/xpl/articleDetails.jsp?tp=&arnumber=6515376&contentType=Journals+%26+Magazines]]></mdurl>
##     <pdf><![CDATA[http://ieeexplore.ieee.org/stamp/stamp.jsp?arnumber=6515376]]></pdf>
##   </document>
##   <document>
##     <rank>508</rank>
##     <title><![CDATA[Computational Implications of Lognormally Distributed Synaptic Weights]]></title>
##     <authors><![CDATA[Teramae, J.-N.;  Fukai, T.]]></authors>
##     <affiliations><![CDATA[Grad. Sch. of Inf. Sci. & Technol., Osaka Univ., Suita, Japan]]></affiliations>
##     <controlledterms>
##       <term><![CDATA[bioelectric potentials]]></term>
##       <term><![CDATA[brain]]></term>
##       <term><![CDATA[log normal distribution]]></term>
##       <term><![CDATA[medical computing]]></term>
##       <term><![CDATA[neural nets]]></term>
##       <term><![CDATA[neurophysiology]]></term>
##       <term><![CDATA[stochastic processes]]></term>
##     </controlledterms>
##     <thesaurusterms>
##       <term><![CDATA[Biological neural networks]]></term>
##       <term><![CDATA[Integrated circuit modeling]]></term>
##       <term><![CDATA[Memory]]></term>
##       <term><![CDATA[Neurons]]></term>
##       <term><![CDATA[Noise measurement]]></term>
##       <term><![CDATA[Principal component analysis]]></term>
##       <term><![CDATA[Stability analysis]]></term>
##       <term><![CDATA[Stochastic resonance]]></term>
##       <term><![CDATA[Stochastic systems]]></term>
##     </thesaurusterms>
##     <pubtitle><![CDATA[Proceedings of the IEEE]]></pubtitle>
##     <punumber><![CDATA[5]]></punumber>
##     <pubtype><![CDATA[Journals & Magazines]]></pubtype>
##     <publisher><![CDATA[IEEE]]></publisher>
##     <volume><![CDATA[102]]></volume>
##     <issue><![CDATA[4]]></issue>
##     <py><![CDATA[2014]]></py>
##     <spage><![CDATA[500]]></spage>
##     <epage><![CDATA[512]]></epage>
##     <abstract><![CDATA[The connectivity structure of neural networks has significant implications for neural information processing, and much experimental effort has been made to clarify the structure of neural networks in the brain, i.e., both graph structure and weight structure of synaptic connections. A traditional view of neural information processing suggests that neurons compute in a highly parallel and distributed manner, in which the cooperation of many weak synaptic inputs is necessary to activate a single neuron. Recent experiments, however, have shown that not all synapses are weak in cortical circuits, but some synapses are extremely strong (several tens of times larger than the average weight). In fact, the weights of excitatory synapses between cortical excitatory neurons often obey a lognormal distribution with a long tail of strong synapses. Here, we review some of our important and recent works on computation with sparsely distributed synaptic weights and discuss the possible implications of this synaptic principle for neural computation by spiking neurons. We demonstrate that internal noise emerges from long-tailed distributions of synaptic weights to produce stochastic resonance effect in the reverberating synaptic pathways constituted by strong synapses. We show a spike-timing-dependent plasticity rule and other mechanisms that produce such weight distributions. A possible hardware realization of lognormally connected networks is also shown.]]></abstract>
##     <issn><![CDATA[0018-9219]]></issn>
##     <htmlFlag><![CDATA[1]]></htmlFlag>
##     <arnumber><![CDATA[6767040]]></arnumber>
##     <doi><![CDATA[10.1109/JPROC.2014.2306254]]></doi>
##     <publicationId><![CDATA[6767040]]></publicationId>
##     <mdurl><![CDATA[http://ieeexplore.ieee.org/xpl/articleDetails.jsp?tp=&arnumber=6767040&contentType=Journals+%26+Magazines]]></mdurl>
##     <pdf><![CDATA[http://ieeexplore.ieee.org/stamp/stamp.jsp?arnumber=6767040]]></pdf>
##   </document>
##   <document>
##     <rank>509</rank>
##     <title><![CDATA[Performance Optimization of Organic Solar Cells]]></title>
##     <authors><![CDATA[Wong, Y.Q.;  Wong, H.Y.;  Tan, C.S.;  Meng, H.F.]]></authors>
##     <affiliations><![CDATA[Fac. of Eng., Multimedia Univ., Cyberjaya, Malaysia]]></affiliations>
##     <controlledterms>
##       <term><![CDATA[fullerene devices]]></term>
##       <term><![CDATA[nanophotonics]]></term>
##       <term><![CDATA[optical design techniques]]></term>
##       <term><![CDATA[optical polymers]]></term>
##       <term><![CDATA[optimisation]]></term>
##       <term><![CDATA[power conversion]]></term>
##       <term><![CDATA[solar cells]]></term>
##     </controlledterms>
##     <thesaurusterms>
##       <term><![CDATA[Absorption]]></term>
##       <term><![CDATA[Energy states]]></term>
##       <term><![CDATA[Excitons]]></term>
##       <term><![CDATA[Optimization]]></term>
##       <term><![CDATA[Photovoltaic cells]]></term>
##       <term><![CDATA[Plastics]]></term>
##     </thesaurusterms>
##     <pubtitle><![CDATA[Photonics Journal, IEEE]]></pubtitle>
##     <punumber><![CDATA[4563994]]></punumber>
##     <pubtype><![CDATA[Journals & Magazines]]></pubtype>
##     <publisher><![CDATA[IEEE]]></publisher>
##     <volume><![CDATA[6]]></volume>
##     <issue><![CDATA[4]]></issue>
##     <py><![CDATA[2014]]></py>
##     <spage><![CDATA[1]]></spage>
##     <epage><![CDATA[26]]></epage>
##     <abstract><![CDATA[Organic solar cells (OSCs) have been gaining great popularity in recent years due to their potentials to be low cost, lightweight, and flexible. The performance of OSCs is growing steadily, and they have achieved a power conversion efficiency close to 10% (for single-junction polymer-fullerene OSC). Although there are still limitations and challenges faced by the development of OSCs, in view of the potentials, recent studies have been focusing on the design optimization of OSC layer structure through material engineering, interfacial layer insertion, layer thickness optimization, and morphological control. In this paper, we provide a comprehensive review and detailed discussion on the optimization works and development on OSCs, with particular focus on the bulk-heterojunction (BHJ) polymer-fullerene OSCs. We also provide a summary of the performance, in a chronological order, and the future outlook of OSC.]]></abstract>
##     <issn><![CDATA[1943-0655]]></issn>
##     <htmlFlag><![CDATA[1]]></htmlFlag>
##     <arnumber><![CDATA[6851853]]></arnumber>
##     <doi><![CDATA[10.1109/JPHOT.2014.2337896]]></doi>
##     <publicationId><![CDATA[6851853]]></publicationId>
##     <mdurl><![CDATA[http://ieeexplore.ieee.org/xpl/articleDetails.jsp?tp=&arnumber=6851853&contentType=Journals+%26+Magazines]]></mdurl>
##     <pdf><![CDATA[http://ieeexplore.ieee.org/stamp/stamp.jsp?arnumber=6851853]]></pdf>
##   </document>
##   <document>
##     <rank>510</rank>
##     <title><![CDATA[Overvoltages due to Synchronous Tripping of Plug-in Electric-Vehicle Chargers Following Voltage Dips]]></title>
##     <authors><![CDATA[Kundu, S.;  Hiskens, I.A.]]></authors>
##     <affiliations><![CDATA[Dept. of Electr. Eng. & Comput. Sci., Univ. of Michigan, Ann Arbor, MI, USA]]></affiliations>
##     <controlledterms>
##       <term><![CDATA[electric vehicles]]></term>
##       <term><![CDATA[overvoltage protection]]></term>
##       <term><![CDATA[power grids]]></term>
##       <term><![CDATA[power supply quality]]></term>
##     </controlledterms>
##     <thesaurusterms>
##       <term><![CDATA[Capacitors]]></term>
##       <term><![CDATA[Load modeling]]></term>
##       <term><![CDATA[Loading]]></term>
##       <term><![CDATA[Mathematical model]]></term>
##       <term><![CDATA[Standards]]></term>
##       <term><![CDATA[Surges]]></term>
##       <term><![CDATA[Voltage fluctuations]]></term>
##     </thesaurusterms>
##     <pubtitle><![CDATA[Power Delivery, IEEE Transactions on]]></pubtitle>
##     <punumber><![CDATA[61]]></punumber>
##     <pubtype><![CDATA[Journals & Magazines]]></pubtype>
##     <publisher><![CDATA[IEEE]]></publisher>
##     <volume><![CDATA[29]]></volume>
##     <issue><![CDATA[3]]></issue>
##     <py><![CDATA[2014]]></py>
##     <spage><![CDATA[1147]]></spage>
##     <epage><![CDATA[1156]]></epage>
##     <abstract><![CDATA[Plug-in electric vehicle (PEV) charging equipment incorporates protection that ensures that grid disturbances do not damage the charger or vehicle. When the grid voltage sags below 80% of nominal, undervoltage protection is likely to disconnect the charging load from the grid. Most PEV charging will occur overnight, when non-PEV load is at a minimum. This paper argues that PEV voltage-sag response, when synchronized across large numbers of PEVs, could result in the loss of a significant proportion of the total load. It is shown that this load loss can lead to unacceptably high voltages once the initiating event has been cleared. This paper explores the nature of this voltage-rise phenomenon. Analysis tools are developed to assist in determining PEV loading conditions that demarcate acceptable postdisturbance voltage response from unacceptable outcomes. Two examples, based on standard distribution test systems, are used to illustrate PEV-induced overvoltage behavior, and demonstrate applications of the analysis tools.]]></abstract>
##     <issn><![CDATA[0885-8977]]></issn>
##     <htmlFlag><![CDATA[1]]></htmlFlag>
##     <arnumber><![CDATA[6798774]]></arnumber>
##     <doi><![CDATA[10.1109/TPWRD.2014.2311112]]></doi>
##     <publicationId><![CDATA[6798774]]></publicationId>
##     <mdurl><![CDATA[http://ieeexplore.ieee.org/xpl/articleDetails.jsp?tp=&arnumber=6798774&contentType=Journals+%26+Magazines]]></mdurl>
##     <pdf><![CDATA[http://ieeexplore.ieee.org/stamp/stamp.jsp?arnumber=6798774]]></pdf>
##   </document>
##   <document>
##     <rank>511</rank>
##     <title><![CDATA[Temperature-Independent Ultrasensitive Fabry&#x2013;Perot All-Fiber Strain Sensor Based on a Bubble-Expanded Microcavity]]></title>
##     <authors><![CDATA[Chenchen Yin;  Zhigang Cao;  Zhao Zhang;  Tao Shui;  Rui Wang;  Jian Wang;  Liang Lu;  Shenglai Zhen;  Benli Yu]]></authors>
##     <affiliations><![CDATA[Key Lab. of Opto-Electron. Inf. Acquisition & Manipulation of Minist. of Educ., Anhui Univ., Hefei, China]]></affiliations>
##     <controlledterms>
##       <term><![CDATA[Fabry-Perot interferometers]]></term>
##       <term><![CDATA[fibre optic sensors]]></term>
##       <term><![CDATA[micro-optics]]></term>
##       <term><![CDATA[microcavities]]></term>
##       <term><![CDATA[optical resonators]]></term>
##       <term><![CDATA[strain sensors]]></term>
##     </controlledterms>
##     <thesaurusterms>
##       <term><![CDATA[Cavity resonators]]></term>
##       <term><![CDATA[Microcavities]]></term>
##       <term><![CDATA[Optical fiber sensors]]></term>
##       <term><![CDATA[Optical fibers]]></term>
##       <term><![CDATA[Sensitivity]]></term>
##       <term><![CDATA[Splicing]]></term>
##       <term><![CDATA[Strain]]></term>
##     </thesaurusterms>
##     <pubtitle><![CDATA[Photonics Journal, IEEE]]></pubtitle>
##     <punumber><![CDATA[4563994]]></punumber>
##     <pubtype><![CDATA[Journals & Magazines]]></pubtype>
##     <publisher><![CDATA[IEEE]]></publisher>
##     <volume><![CDATA[6]]></volume>
##     <issue><![CDATA[4]]></issue>
##     <py><![CDATA[2014]]></py>
##     <spage><![CDATA[1]]></spage>
##     <epage><![CDATA[9]]></epage>
##     <abstract><![CDATA[A fiber Fabry-Perot interferometer (FPI) based on a bubble-expanded microcavity is investigated in this paper. Due to the expanded bubble microcavity, the FPI exhibits high strain sensitivity of 30.66 pm/&#x03BC;&#x03B5;. To the best of our knowledge, such strain sensitivity is three to ten times higher than that of the fiber FPIs based on the microcavities previously reported. The strain response of the FPI with different shapes is analyzed, and the experimental results are consistent with the theoretical analysis.]]></abstract>
##     <issn><![CDATA[1943-0655]]></issn>
##     <htmlFlag><![CDATA[1]]></htmlFlag>
##     <arnumber><![CDATA[6872790]]></arnumber>
##     <doi><![CDATA[10.1109/JPHOT.2014.2345883]]></doi>
##     <publicationId><![CDATA[6872790]]></publicationId>
##     <mdurl><![CDATA[http://ieeexplore.ieee.org/xpl/articleDetails.jsp?tp=&arnumber=6872790&contentType=Journals+%26+Magazines]]></mdurl>
##     <pdf><![CDATA[http://ieeexplore.ieee.org/stamp/stamp.jsp?arnumber=6872790]]></pdf>
##   </document>
##   <document>
##     <rank>512</rank>
##     <title><![CDATA[Seamless Fusion of LiDAR and Aerial Imagery for Building Extraction]]></title>
##     <authors><![CDATA[Guoqing Zhou;  Xiang Zhou]]></authors>
##     <affiliations><![CDATA[GuangXi Key Lab. for Geospatial Inf. & Geomatics, Guilin Univ. of Technol., Guilin, China]]></affiliations>
##     <controlledterms>
##       <term><![CDATA[geophysical image processing]]></term>
##       <term><![CDATA[graph theory]]></term>
##       <term><![CDATA[image coding]]></term>
##       <term><![CDATA[image fusion]]></term>
##       <term><![CDATA[image registration]]></term>
##       <term><![CDATA[image representation]]></term>
##       <term><![CDATA[optical radar]]></term>
##       <term><![CDATA[radar imaging]]></term>
##     </controlledterms>
##     <thesaurusterms>
##       <term><![CDATA[Buildings]]></term>
##       <term><![CDATA[Encoding]]></term>
##       <term><![CDATA[Face]]></term>
##       <term><![CDATA[Feature extraction]]></term>
##       <term><![CDATA[Laser radar]]></term>
##       <term><![CDATA[Merging]]></term>
##       <term><![CDATA[Optical imaging]]></term>
##     </thesaurusterms>
##     <pubtitle><![CDATA[Geoscience and Remote Sensing, IEEE Transactions on]]></pubtitle>
##     <punumber><![CDATA[36]]></punumber>
##     <pubtype><![CDATA[Journals & Magazines]]></pubtype>
##     <publisher><![CDATA[IEEE]]></publisher>
##     <volume><![CDATA[52]]></volume>
##     <issue><![CDATA[11]]></issue>
##     <py><![CDATA[2014]]></py>
##     <spage><![CDATA[7393]]></spage>
##     <epage><![CDATA[7407]]></epage>
##     <abstract><![CDATA[Although many efforts have been made on the fusion of Light Detection and Ranging (LiDAR) and aerial imagery for the extraction of houses, little research on taking advantage of a building's geometric features, properties, and structures for assisting the further fusion of the two types of data has been made. For this reason, this paper develops a seamless fusion between LiDAR and aerial imagery on the basis of aspect graphs, which utilize the features of houses, such as geometry, structures, and shapes. First, 3-D primitives, standing for houses, are chosen, and their projections are represented by the aspects. A hierarchical aspect graph is then constructed using aerial image processing in combination with the results of LiDAR data processing. In the aspect graph, the note represents the face aspect and the arc is described by attributes obtained by the formulated coding regulations, and the coregistration between the aspect and LiDAR data is implemented. As a consequence, the aspects and/or the aspect graph are interpreted for the extraction of houses, and then the houses are fitted using a planar equation for creating a digital building model (DBM). The experimental field, which is located in Wytheville, VA, is used to evaluate the proposed method. The experimental results demonstrated that the proposed method is capable of effectively extracting houses at a successful rate of 93%, as compared with another method, which is 82% effective when LiDAR spacing is approximately 7.3 by 7.3 ft<sup>2</sup>. The accuracy of 3-D DBM is higher than the method using only single LiDAR data.]]></abstract>
##     <issn><![CDATA[0196-2892]]></issn>
##     <htmlFlag><![CDATA[1]]></htmlFlag>
##     <arnumber><![CDATA[6804760]]></arnumber>
##     <doi><![CDATA[10.1109/TGRS.2014.2311991]]></doi>
##     <publicationId><![CDATA[6804760]]></publicationId>
##     <mdurl><![CDATA[http://ieeexplore.ieee.org/xpl/articleDetails.jsp?tp=&arnumber=6804760&contentType=Journals+%26+Magazines]]></mdurl>
##     <pdf><![CDATA[http://ieeexplore.ieee.org/stamp/stamp.jsp?arnumber=6804760]]></pdf>
##   </document>
##   <document>
##     <rank>513</rank>
##     <title><![CDATA[Progress on Developing Adaptive Optics&#x2013;Optical Coherence Tomography for In Vivo Retinal Imaging: Monitoring and Correction of Eye Motion Artifacts]]></title>
##     <authors><![CDATA[Zawadzki, R.J.;  Capps, A.G.;  Dae Yu Kim;  Panorgias, A.;  Stevenson, S.B.;  Hamann, B.;  Werner, J.S.]]></authors>
##     <affiliations><![CDATA[Dept. of Ophthalmology & Vision Sci., Univ. of California Davis, Sacramento, CA, USA]]></affiliations>
##     <controlledterms>
##       <term><![CDATA[adaptive optics]]></term>
##       <term><![CDATA[bio-optics]]></term>
##       <term><![CDATA[biomechanics]]></term>
##       <term><![CDATA[biomedical optical imaging]]></term>
##       <term><![CDATA[cellular biophysics]]></term>
##       <term><![CDATA[data acquisition]]></term>
##       <term><![CDATA[eye]]></term>
##       <term><![CDATA[image registration]]></term>
##       <term><![CDATA[image resolution]]></term>
##       <term><![CDATA[laser applications in medicine]]></term>
##       <term><![CDATA[medical image processing]]></term>
##       <term><![CDATA[optical tomography]]></term>
##       <term><![CDATA[patient monitoring]]></term>
##       <term><![CDATA[vision]]></term>
##     </controlledterms>
##     <thesaurusterms>
##       <term><![CDATA[Adaptive optics]]></term>
##       <term><![CDATA[Image resolution]]></term>
##       <term><![CDATA[Instruments]]></term>
##       <term><![CDATA[Mirrors]]></term>
##       <term><![CDATA[Optical imaging]]></term>
##       <term><![CDATA[Retina]]></term>
##     </thesaurusterms>
##     <pubtitle><![CDATA[Selected Topics in Quantum Electronics, IEEE Journal of]]></pubtitle>
##     <punumber><![CDATA[2944]]></punumber>
##     <pubtype><![CDATA[Journals & Magazines]]></pubtype>
##     <publisher><![CDATA[IEEE]]></publisher>
##     <volume><![CDATA[20]]></volume>
##     <issue><![CDATA[2]]></issue>
##     <py><![CDATA[2014]]></py>
##     <spage><![CDATA[322]]></spage>
##     <epage><![CDATA[333]]></epage>
##     <abstract><![CDATA[Recent progress in retinal image acquisition techniques, including optical coherence tomography (OCT) and scanning laser ophthalmoscopy (SLO), combined with improved performance of adaptive optics (AO) instrumentation, has resulted in improvement in the quality of in vivo images of cellular structures in the human retina. Here, we present a short review of progress on developing AO-OCT instruments. Despite significant progress in imaging speed and resolution, eye movements present during acquisition of a retinal image with OCT introduce motion artifacts into the image, complicating analysis and registration. This effect is especially pronounced in high-resolution datasets acquired with AO-OCT instruments. Several retinal tracking systems have been introduced to correct retinal motion during data acquisition. We present a method for correcting motion artifacts in AO-OCT volume data after acquisition using simultaneously captured adaptive optics-scanning laser ophthalmoscope (AO-SLO) images. We extract transverse eye motion data from the AO-SLO images, assign a motion adjustment vector to each AO-OCT A-scan, and re-sample from the scattered data back onto a regular grid. The corrected volume data improve the accuracy of quantitative analyses of microscopic structures.]]></abstract>
##     <issn><![CDATA[1077-260X]]></issn>
##     <htmlFlag><![CDATA[1]]></htmlFlag>
##     <arnumber><![CDATA[6683119]]></arnumber>
##     <doi><![CDATA[10.1109/JSTQE.2013.2288302]]></doi>
##     <publicationId><![CDATA[6683119]]></publicationId>
##     <mdurl><![CDATA[http://ieeexplore.ieee.org/xpl/articleDetails.jsp?tp=&arnumber=6683119&contentType=Journals+%26+Magazines]]></mdurl>
##     <pdf><![CDATA[http://ieeexplore.ieee.org/stamp/stamp.jsp?arnumber=6683119]]></pdf>
##   </document>
##   <document>
##     <rank>514</rank>
##     <title><![CDATA[Cloud-RAN Architecture for Indoor DAS]]></title>
##     <authors><![CDATA[Beyene, Y.D.;  Jantti, R.;  Ruttik, K.]]></authors>
##     <affiliations><![CDATA[Dept. of Commun. & Networking, Aalto Univ., Espoo, Finland]]></affiliations>
##     <controlledterms>
##       <term><![CDATA[cellular radio]]></term>
##       <term><![CDATA[cloud computing]]></term>
##       <term><![CDATA[radio access networks]]></term>
##       <term><![CDATA[software radio]]></term>
##       <term><![CDATA[transmitting antennas]]></term>
##     </controlledterms>
##     <thesaurusterms>
##       <term><![CDATA[Algorithm design and analysis]]></term>
##       <term><![CDATA[Base stations]]></term>
##       <term><![CDATA[Baseband]]></term>
##       <term><![CDATA[Cellular networks]]></term>
##       <term><![CDATA[Cloud computing]]></term>
##       <term><![CDATA[Computer architecture]]></term>
##       <term><![CDATA[Microprocessors]]></term>
##       <term><![CDATA[Open loop systems]]></term>
##       <term><![CDATA[Radio access networks]]></term>
##       <term><![CDATA[Servers]]></term>
##     </thesaurusterms>
##     <pubtitle><![CDATA[Access, IEEE]]></pubtitle>
##     <punumber><![CDATA[6287639]]></punumber>
##     <pubtype><![CDATA[Journals & Magazines]]></pubtype>
##     <publisher><![CDATA[IEEE]]></publisher>
##     <volume><![CDATA[2]]></volume>
##     <py><![CDATA[2014]]></py>
##     <spage><![CDATA[1205]]></spage>
##     <epage><![CDATA[1212]]></epage>
##     <abstract><![CDATA[A cloud radio access network (Cloud-RAN) is a new cellular technology that brings baseband processing units for a set of base stations into a central server retaining only the radio front-ends at the cell sites. This new architecture opens up opportunities for algorithms that require centralized processing. However, efficient implementation of the algorithms presents a number of challenges the most critical being latency, fronthaul capacity, and resource control. In this paper, we propose a software-defined radio-based architecture that addresses these problems and can be implemented on a cloud of general purpose computing platforms. We also present the practical implementation of Cloud-RAN running on an off-the-shelf server to validate the flexibility of the architecture. The implementation is able to realize various cellular networks, including heterogeneous networks, distribute-antenna systems, and transmission schemes, such as transmit antenna selection and open-loop transmit diversity.]]></abstract>
##     <issn><![CDATA[2169-3536]]></issn>
##     <htmlFlag><![CDATA[1]]></htmlFlag>
##     <arnumber><![CDATA[6914529]]></arnumber>
##     <doi><![CDATA[10.1109/ACCESS.2014.2361259]]></doi>
##     <publicationId><![CDATA[6914529]]></publicationId>
##     <mdurl><![CDATA[http://ieeexplore.ieee.org/xpl/articleDetails.jsp?tp=&arnumber=6914529&contentType=Journals+%26+Magazines]]></mdurl>
##     <pdf><![CDATA[http://ieeexplore.ieee.org/stamp/stamp.jsp?arnumber=6914529]]></pdf>
##   </document>
##   <document>
##     <rank>515</rank>
##     <title><![CDATA[SCoRS&#x2014;A Method Based on Stability for Feature Selection and Mapping in Neuroimaging]]></title>
##     <authors><![CDATA[Rondina, J.M.;  Hahn, T.;  de Oliveira, L.;  Marquand, A.F.;  Dresler, T.;  Leitner, T.;  Fallgatter, A.J.;  Shawe-Taylor, J.;  Mourao-Miranda, J.]]></authors>
##     <affiliations><![CDATA[Centre for Neuroimaging Sci., King's Coll. London, London, UK]]></affiliations>
##     <controlledterms>
##       <term><![CDATA[biomedical MRI]]></term>
##       <term><![CDATA[face recognition]]></term>
##       <term><![CDATA[feature selection]]></term>
##       <term><![CDATA[image classification]]></term>
##       <term><![CDATA[iterative methods]]></term>
##       <term><![CDATA[medical image processing]]></term>
##       <term><![CDATA[neurophysiology]]></term>
##       <term><![CDATA[random processes]]></term>
##     </controlledterms>
##     <thesaurusterms>
##       <term><![CDATA[Accuracy]]></term>
##       <term><![CDATA[Context]]></term>
##       <term><![CDATA[Educational institutions]]></term>
##       <term><![CDATA[Neuroimaging]]></term>
##       <term><![CDATA[Support vector machines]]></term>
##       <term><![CDATA[Training]]></term>
##       <term><![CDATA[Vegetation]]></term>
##     </thesaurusterms>
##     <pubtitle><![CDATA[Medical Imaging, IEEE Transactions on]]></pubtitle>
##     <punumber><![CDATA[42]]></punumber>
##     <pubtype><![CDATA[Journals & Magazines]]></pubtype>
##     <publisher><![CDATA[IEEE]]></publisher>
##     <volume><![CDATA[33]]></volume>
##     <issue><![CDATA[1]]></issue>
##     <py><![CDATA[2014]]></py>
##     <spage><![CDATA[85]]></spage>
##     <epage><![CDATA[98]]></epage>
##     <abstract><![CDATA[Feature selection (FS) methods play two important roles in the context of neuroimaging based classification: potentially increase classification accuracy by eliminating irrelevant features from the model and facilitate interpretation by identifying sets of meaningful features that best discriminate the classes. Although the development of FS techniques specifically tuned for neuroimaging data is an active area of research, up to date most of the studies have focused on finding a subset of features that maximizes accuracy. However, maximizing accuracy does not guarantee reliable interpretation as similar accuracies can be obtained from distinct sets of features. In the current paper we propose a new approach for selecting features: SCoRS (survival count on random subsamples) based on a recently proposed Stability Selection theory. SCoRS relies on the idea of choosing relevant features that are stable under data perturbation. Data are perturbed by iteratively sub-sampling both features (subspaces) and examples. We demonstrate the potential of the proposed method in a clinical application to classify depressed patients versus healthy individuals based on functional magnetic resonance imaging data acquired during visualization of happy faces.]]></abstract>
##     <issn><![CDATA[0278-0062]]></issn>
##     <htmlFlag><![CDATA[1]]></htmlFlag>
##     <arnumber><![CDATA[6595571]]></arnumber>
##     <doi><![CDATA[10.1109/TMI.2013.2281398]]></doi>
##     <publicationId><![CDATA[6595571]]></publicationId>
##     <mdurl><![CDATA[http://ieeexplore.ieee.org/xpl/articleDetails.jsp?tp=&arnumber=6595571&contentType=Journals+%26+Magazines]]></mdurl>
##     <pdf><![CDATA[http://ieeexplore.ieee.org/stamp/stamp.jsp?arnumber=6595571]]></pdf>
##   </document>
##   <document>
##     <rank>516</rank>
##     <title><![CDATA[Multimode Fiber Coupled Superconductor Nanowire Single-Photon Detector]]></title>
##     <authors><![CDATA[Labao Zhang;  Ming Gu;  Tao Jia;  Ruiyin Xu;  Chao Wan;  Lin Kang;  Jian Chen;  Peiheng Wu]]></authors>
##     <affiliations><![CDATA[Sch. of Electron. Sci. & Eng., Univ. of Nanjing, Nanjing, China]]></affiliations>
##     <controlledterms>
##       <term><![CDATA[critical currents]]></term>
##       <term><![CDATA[nanophotonics]]></term>
##       <term><![CDATA[nanowires]]></term>
##       <term><![CDATA[optical communication equipment]]></term>
##       <term><![CDATA[optical design techniques]]></term>
##       <term><![CDATA[optical fabrication]]></term>
##       <term><![CDATA[optical fibre communication]]></term>
##       <term><![CDATA[optical fibre couplers]]></term>
##       <term><![CDATA[photodetectors]]></term>
##       <term><![CDATA[superconducting materials]]></term>
##       <term><![CDATA[timing jitter]]></term>
##     </controlledterms>
##     <thesaurusterms>
##       <term><![CDATA[Couplings]]></term>
##       <term><![CDATA[Detectors]]></term>
##       <term><![CDATA[Optical fiber amplifiers]]></term>
##       <term><![CDATA[Optical fiber communication]]></term>
##       <term><![CDATA[Photonics]]></term>
##     </thesaurusterms>
##     <pubtitle><![CDATA[Photonics Journal, IEEE]]></pubtitle>
##     <punumber><![CDATA[4563994]]></punumber>
##     <pubtype><![CDATA[Journals & Magazines]]></pubtype>
##     <publisher><![CDATA[IEEE]]></publisher>
##     <volume><![CDATA[6]]></volume>
##     <issue><![CDATA[5]]></issue>
##     <py><![CDATA[2014]]></py>
##     <spage><![CDATA[1]]></spage>
##     <epage><![CDATA[8]]></epage>
##     <abstract><![CDATA[High-performance superconductor nanowire single-photon detectors (SNSPDs) coupled by 50-&#x03BC;m multimode fiber were designed and fabricated for the purpose of space applications. Microlens coupling between multimode fiber and detector chip was proposed, resulting in coupling efficiency of 86% at the chips with detection area of 15 x 15 &#x03BC;m<sup>2</sup>. An optical structure for enhancing the photons absorption and superconductor nanowires with low critical current of 3 &#x03BC;A were fabricated on the chips. A cryogenic amplifier was adopted to increase the signal-to-noise ratio and time jitter of output pulse. The SNSPD exhibited system efficiency (1550 nm) of 56% at a dark count rate of 100 cps, 50% at a dark count rate of 5 cps, and 40% at a dark count rate of 1 cps. The time jitter was 46 ps at full-width at half-maximum. The multimode fiber coupling system with high performance will promote its applications in free space.]]></abstract>
##     <issn><![CDATA[1943-0655]]></issn>
##     <htmlFlag><![CDATA[1]]></htmlFlag>
##     <arnumber><![CDATA[6910208]]></arnumber>
##     <doi><![CDATA[10.1109/JPHOT.2014.2360285]]></doi>
##     <publicationId><![CDATA[6910208]]></publicationId>
##     <mdurl><![CDATA[http://ieeexplore.ieee.org/xpl/articleDetails.jsp?tp=&arnumber=6910208&contentType=Journals+%26+Magazines]]></mdurl>
##     <pdf><![CDATA[http://ieeexplore.ieee.org/stamp/stamp.jsp?arnumber=6910208]]></pdf>
##   </document>
##   <document>
##     <rank>517</rank>
##     <title><![CDATA[Geodesic Paths for Time-Dependent Covariance Matrices in a Riemannian Manifold]]></title>
##     <authors><![CDATA[Ben-David, A.;  Marks, J.]]></authors>
##     <affiliations><![CDATA[RDECOM, Edgewood Chem. Biol. Center, Aberdeen Proving Ground, MD, USA]]></affiliations>
##     <controlledterms>
##       <term><![CDATA[geodesy]]></term>
##       <term><![CDATA[geophysical techniques]]></term>
##       <term><![CDATA[remote sensing]]></term>
##     </controlledterms>
##     <thesaurusterms>
##       <term><![CDATA[Covariance matrices]]></term>
##       <term><![CDATA[Eigenvalues and eigenfunctions]]></term>
##       <term><![CDATA[Hyperspectral imaging]]></term>
##       <term><![CDATA[Manifolds]]></term>
##       <term><![CDATA[Signal to noise ratio]]></term>
##       <term><![CDATA[Vectors]]></term>
##     </thesaurusterms>
##     <pubtitle><![CDATA[Geoscience and Remote Sensing Letters, IEEE]]></pubtitle>
##     <punumber><![CDATA[8859]]></punumber>
##     <pubtype><![CDATA[Journals & Magazines]]></pubtype>
##     <publisher><![CDATA[IEEE]]></publisher>
##     <volume><![CDATA[11]]></volume>
##     <issue><![CDATA[9]]></issue>
##     <py><![CDATA[2014]]></py>
##     <spage><![CDATA[1499]]></spage>
##     <epage><![CDATA[1503]]></epage>
##     <abstract><![CDATA[Time-dependent covariance matrices are important in remote sensing and hyperspectral detection theory. The difficulty is that C(t) is usually available only at two endpoints C(t<sub>0</sub>) = A and C(t<sub>1</sub>) = B where is needed. We present the Riemannian manifold of positive definite symmetric matrices as a framework for predicting a geodesic time-dependent covariance matrix. The geodesic path A&#x2192;B is the shortest and most efficient path (minimum energy). Although there is no guarantee that data will necessarily follow a geodesic path, the predicted geodesic C(t) is of value as a concept. The path for the inverse covariance is also geodesic and is easily computed. We present an interpretation of C(t) with coloring and whitening operators to be a sum of scaled, stretched, contracted, and rotated ellipses.]]></abstract>
##     <issn><![CDATA[1545-598X]]></issn>
##     <htmlFlag><![CDATA[1]]></htmlFlag>
##     <arnumber><![CDATA[6727392]]></arnumber>
##     <doi><![CDATA[10.1109/LGRS.2013.2296833]]></doi>
##     <publicationId><![CDATA[6727392]]></publicationId>
##     <mdurl><![CDATA[http://ieeexplore.ieee.org/xpl/articleDetails.jsp?tp=&arnumber=6727392&contentType=Journals+%26+Magazines]]></mdurl>
##     <pdf><![CDATA[http://ieeexplore.ieee.org/stamp/stamp.jsp?arnumber=6727392]]></pdf>
##   </document>
##   <document>
##     <rank>518</rank>
##     <title><![CDATA[Detection performance of spatial-frequency diversity MIMO radar]]></title>
##     <authors><![CDATA[Hongwei Liu;  Shenghua Zhou;  Hongtao Su;  Yao Yu]]></authors>
##     <affiliations><![CDATA[Xidian Univ., Xi'an, China]]></affiliations>
##     <controlledterms>
##       <term><![CDATA[MIMO radar]]></term>
##       <term><![CDATA[correlation methods]]></term>
##       <term><![CDATA[probability]]></term>
##       <term><![CDATA[radar detection]]></term>
##       <term><![CDATA[radar target recognition]]></term>
##     </controlledterms>
##     <thesaurusterms>
##       <term><![CDATA[Correlation]]></term>
##       <term><![CDATA[Detection algorithms]]></term>
##       <term><![CDATA[Frequency diversity]]></term>
##       <term><![CDATA[MIMO radar]]></term>
##       <term><![CDATA[Radar cross-sections]]></term>
##       <term><![CDATA[Radar detection]]></term>
##     </thesaurusterms>
##     <pubtitle><![CDATA[Aerospace and Electronic Systems, IEEE Transactions on]]></pubtitle>
##     <punumber><![CDATA[7]]></punumber>
##     <pubtype><![CDATA[Journals & Magazines]]></pubtype>
##     <publisher><![CDATA[IEEE]]></publisher>
##     <volume><![CDATA[50]]></volume>
##     <issue><![CDATA[4]]></issue>
##     <py><![CDATA[2014]]></py>
##     <spage><![CDATA[3137]]></spage>
##     <epage><![CDATA[3155]]></epage>
##     <abstract><![CDATA[For spatial-frequency diversity radar, diversity channels may receive partially correlated target returns and channel signal-to-noise ratios (SNRs) may be different. For six typical scenarios of diversity radar, we design detection algorithms and analyze their detection performances in theory and via numerical results. It is shown that the detectors considering target correlation and channel SNR distribution can improve the detection performance of diversity radar. Whether certain channel SNRs can achieve a higher optimal detection probability than another depends on total channel SNR and false alarm rate.]]></abstract>
##     <issn><![CDATA[0018-9251]]></issn>
##     <htmlFlag><![CDATA[1]]></htmlFlag>
##     <arnumber><![CDATA[6978904]]></arnumber>
##     <doi><![CDATA[10.1109/TAES.2013.120040]]></doi>
##     <publicationId><![CDATA[6978904]]></publicationId>
##     <mdurl><![CDATA[http://ieeexplore.ieee.org/xpl/articleDetails.jsp?tp=&arnumber=6978904&contentType=Journals+%26+Magazines]]></mdurl>
##     <pdf><![CDATA[http://ieeexplore.ieee.org/stamp/stamp.jsp?arnumber=6978904]]></pdf>
##   </document>
##   <document>
##     <rank>519</rank>
##     <title><![CDATA[Bioinspired Polarization Imaging Sensors: From Circuits and Optics to Signal Processing Algorithms and Biomedical Applications]]></title>
##     <authors><![CDATA[York, T.;  Powell, S.B.;  Gao, S.;  Kahan, L.;  Charanya, T.;  Saha, D.;  Roberts, N.W.;  Cronin, T.W.;  Marshall, J.;  Achilefu, S.;  Lake, S.P.;  Raman, B.;  Gruev, V.]]></authors>
##     <affiliations><![CDATA[Dept. of Comput. Sci. & Eng., Washington Univ. in St. Louis, St. Louis, MO, USA]]></affiliations>
##     <controlledterms>
##       <term><![CDATA[CMOS image sensors]]></term>
##       <term><![CDATA[biomedical electronics]]></term>
##       <term><![CDATA[biomedical transducers]]></term>
##       <term><![CDATA[electro-optical devices]]></term>
##       <term><![CDATA[light polarisation]]></term>
##       <term><![CDATA[medical signal processing]]></term>
##       <term><![CDATA[optical information processing]]></term>
##       <term><![CDATA[optical sensors]]></term>
##     </controlledterms>
##     <thesaurusterms>
##       <term><![CDATA[Biomedical imaging]]></term>
##       <term><![CDATA[Biosensors]]></term>
##       <term><![CDATA[CMOS integrated circuits]]></term>
##       <term><![CDATA[Image sensors]]></term>
##       <term><![CDATA[Neural networks]]></term>
##       <term><![CDATA[Optical filters]]></term>
##       <term><![CDATA[Optical sensors]]></term>
##     </thesaurusterms>
##     <pubtitle><![CDATA[Proceedings of the IEEE]]></pubtitle>
##     <punumber><![CDATA[5]]></punumber>
##     <pubtype><![CDATA[Journals & Magazines]]></pubtype>
##     <publisher><![CDATA[IEEE]]></publisher>
##     <volume><![CDATA[102]]></volume>
##     <issue><![CDATA[10]]></issue>
##     <py><![CDATA[2014]]></py>
##     <spage><![CDATA[1450]]></spage>
##     <epage><![CDATA[1469]]></epage>
##     <abstract><![CDATA[In this paper, we present recent work on bioinspired polarization imaging sensors and their applications in biomedicine. In particular, we focus on three different aspects of these sensors. First, we describe the electro-optical challenges in realizing a bioinspired polarization imager, and in particular, we provide a detailed description of a recent low-power complementary metal-oxide-semiconductor (CMOS) polarization imager. Second, we focus on signal processing algorithms tailored for this new class of bioinspired polarization imaging sensors, such as calibration and interpolation. Third, the emergence of these sensors has enabled rapid progress in characterizing polarization signals and environmental parameters in nature, as well as several biomedical areas, such as label-free optical neural recording, dynamic tissue strength analysis, and early diagnosis of flat cancerous lesions in a murine colorectal tumor model. We highlight results obtained from these three areas and discuss future applications for these sensors.]]></abstract>
##     <issn><![CDATA[0018-9219]]></issn>
##     <htmlFlag><![CDATA[1]]></htmlFlag>
##     <arnumber><![CDATA[6880796]]></arnumber>
##     <doi><![CDATA[10.1109/JPROC.2014.2342537]]></doi>
##     <publicationId><![CDATA[6880796]]></publicationId>
##     <mdurl><![CDATA[http://ieeexplore.ieee.org/xpl/articleDetails.jsp?tp=&arnumber=6880796&contentType=Journals+%26+Magazines]]></mdurl>
##     <pdf><![CDATA[http://ieeexplore.ieee.org/stamp/stamp.jsp?arnumber=6880796]]></pdf>
##   </document>
##   <document>
##     <rank>520</rank>
##     <title><![CDATA[Recent Progress on C-RAN Centralization and Cloudification]]></title>
##     <authors><![CDATA[Chih-Lin I;  Jinri Huang;  Ran Duan;  Chunfeng Cui;  Jiang, J.X.;  Lei Li]]></authors>
##     <affiliations><![CDATA[Green Commun. Res. Center, China Mobile Res. Inst., Beijing, China]]></affiliations>
##     <controlledterms>
##       <term><![CDATA[Long Term Evolution]]></term>
##       <term><![CDATA[cellular radio]]></term>
##       <term><![CDATA[cloud computing]]></term>
##       <term><![CDATA[frequency division multiplexing]]></term>
##       <term><![CDATA[radio access networks]]></term>
##       <term><![CDATA[telecommunication computing]]></term>
##       <term><![CDATA[time division multiplexing]]></term>
##       <term><![CDATA[virtualisation]]></term>
##       <term><![CDATA[wavelength division multiplexing]]></term>
##     </controlledterms>
##     <thesaurusterms>
##       <term><![CDATA[Mobile communication]]></term>
##       <term><![CDATA[Optical fiber devices]]></term>
##       <term><![CDATA[Optical fiber networks]]></term>
##       <term><![CDATA[Optical fiber testing]]></term>
##       <term><![CDATA[Radio access networks]]></term>
##       <term><![CDATA[Virtualization]]></term>
##       <term><![CDATA[Wavelength division multiplexing]]></term>
##     </thesaurusterms>
##     <pubtitle><![CDATA[Access, IEEE]]></pubtitle>
##     <punumber><![CDATA[6287639]]></punumber>
##     <pubtype><![CDATA[Journals & Magazines]]></pubtype>
##     <publisher><![CDATA[IEEE]]></publisher>
##     <volume><![CDATA[2]]></volume>
##     <py><![CDATA[2014]]></py>
##     <spage><![CDATA[1030]]></spage>
##     <epage><![CDATA[1039]]></epage>
##     <abstract><![CDATA[This paper presents the latest progress on cloud RAN (C-RAN) in the areas of centralization and virtualization. A C-RAN system centralizes the baseband processing resources into a pool and virtualizes soft base-band units on demand. The major challenges for C-RAN including front-haul and virtualization are analyzed with potential solutions proposed. Extensive field trials verify the viability of various front-haul solutions, including common public radio interface compression, single fiber bidirection and wavelength-division multiplexing. In addition, C-RANs facilitation of coordinated multipoint (CoMP) implementation is demonstrated with 50%-100% uplink CoMP gain observed in field trials. Finally, a test bed is established based on general purpose platform with assisted accelerators. It is demonstrated that this test bed can support multi-RAT, i.e., Time-Division Duplexing Long Term Evolution, Frequency-Division Duplexing Long Term Evolution, and Global System for Mobile Communications efficiently and presents similar performance to traditional systems.]]></abstract>
##     <issn><![CDATA[2169-3536]]></issn>
##     <htmlFlag><![CDATA[1]]></htmlFlag>
##     <arnumber><![CDATA[6882182]]></arnumber>
##     <doi><![CDATA[10.1109/ACCESS.2014.2351411]]></doi>
##     <publicationId><![CDATA[6882182]]></publicationId>
##     <mdurl><![CDATA[http://ieeexplore.ieee.org/xpl/articleDetails.jsp?tp=&arnumber=6882182&contentType=Journals+%26+Magazines]]></mdurl>
##     <pdf><![CDATA[http://ieeexplore.ieee.org/stamp/stamp.jsp?arnumber=6882182]]></pdf>
##   </document>
##   <document>
##     <rank>521</rank>
##     <title><![CDATA[A Rule-Based Dynamic Decision-Making Stock Trading System Based on Quantum-Inspired Tabu Search Algorithm]]></title>
##     <authors><![CDATA[Yao-Hsin Chou;  Shu-Yu Kuo;  Chi-Yuan Chen;  Han-Chieh Chao]]></authors>
##     <affiliations><![CDATA[Dept. of Comput. Sci. & Inf. Eng., Nat. Chi Nan Univ., Nantou, Taiwan]]></affiliations>
##     <controlledterms>
##       <term><![CDATA[combinatorial mathematics]]></term>
##       <term><![CDATA[decision making]]></term>
##       <term><![CDATA[econophysics]]></term>
##       <term><![CDATA[evolutionary computation]]></term>
##       <term><![CDATA[search problems]]></term>
##       <term><![CDATA[stock markets]]></term>
##     </controlledterms>
##     <thesaurusterms>
##       <term><![CDATA[Decision making]]></term>
##       <term><![CDATA[Encoding]]></term>
##       <term><![CDATA[Evolutionary computation]]></term>
##       <term><![CDATA[Genetic algorithms]]></term>
##       <term><![CDATA[Heuristic algorithms]]></term>
##       <term><![CDATA[Information analysis]]></term>
##       <term><![CDATA[Information retrieval]]></term>
##       <term><![CDATA[Search methods]]></term>
##       <term><![CDATA[Stock markets]]></term>
##       <term><![CDATA[Training]]></term>
##     </thesaurusterms>
##     <pubtitle><![CDATA[Access, IEEE]]></pubtitle>
##     <punumber><![CDATA[6287639]]></punumber>
##     <pubtype><![CDATA[Journals & Magazines]]></pubtype>
##     <publisher><![CDATA[IEEE]]></publisher>
##     <volume><![CDATA[2]]></volume>
##     <py><![CDATA[2014]]></py>
##     <spage><![CDATA[883]]></spage>
##     <epage><![CDATA[896]]></epage>
##     <abstract><![CDATA[Heuristic methods or evolutionary algorithms (such as genetic algorithms and genetic programs) are common approaches applied in financial applications, such as trading systems. Determining the best time to buy or sell stocks in a stock market, and thereby maximizing profit with low risks, is an important issue in financial research. Recent studies have used trading rules based on technique analysis to address this problem. This method can determine trading times by analyzing the value of technical indicators. In other words, we can make trading rules by finding the trading value of technique indicators. An example of a trading rule would be, if one technical indicator's value achieves the setting value, then either buy or sell. A combination of trading rules would become a trading strategy. The process of making trading strategies can be formulated as a combinational optimization problem. In this paper, we propose a novel method for applying a trading system. First, the proposed method uses the quantum-inspired Tabu search algorithm to find the optimal composition and combination of trading strategies. Second, this method uses a sliding window to avoid the major problem of over-fitting. The experiment results of earning money show much better performance than other approaches, and the proposed method outperforms the buy and hold method (which is a benchmark in this field).]]></abstract>
##     <issn><![CDATA[2169-3536]]></issn>
##     <htmlFlag><![CDATA[1]]></htmlFlag>
##     <arnumber><![CDATA[6883114]]></arnumber>
##     <doi><![CDATA[10.1109/ACCESS.2014.2352261]]></doi>
##     <publicationId><![CDATA[6883114]]></publicationId>
##     <mdurl><![CDATA[http://ieeexplore.ieee.org/xpl/articleDetails.jsp?tp=&arnumber=6883114&contentType=Journals+%26+Magazines]]></mdurl>
##     <pdf><![CDATA[http://ieeexplore.ieee.org/stamp/stamp.jsp?arnumber=6883114]]></pdf>
##   </document>
##   <document>
##     <rank>522</rank>
##     <title><![CDATA[A 56.4-to-63.4 GHz Multi-Rate All-Digital Fractional-N PLL for FMCW Radar Applications in 65 nm CMOS]]></title>
##     <authors><![CDATA[Wanghua Wu;  Staszewski, R.B.;  Long, J.R.]]></authors>
##     <affiliations><![CDATA[Marvell Semicond. Inc., Santa Clara, CA, USA]]></affiliations>
##     <controlledterms>
##       <term><![CDATA[CMOS digital integrated circuits]]></term>
##       <term><![CDATA[CW radar]]></term>
##       <term><![CDATA[FM radar]]></term>
##       <term><![CDATA[UHF integrated circuits]]></term>
##       <term><![CDATA[UHF oscillators]]></term>
##       <term><![CDATA[UHF power amplifiers]]></term>
##       <term><![CDATA[circuit tuning]]></term>
##       <term><![CDATA[digital phase locked loops]]></term>
##       <term><![CDATA[millimetre wave integrated circuits]]></term>
##       <term><![CDATA[millimetre wave oscillators]]></term>
##       <term><![CDATA[millimetre wave power amplifiers]]></term>
##       <term><![CDATA[millimetre wave radar]]></term>
##       <term><![CDATA[radar transmitters]]></term>
##       <term><![CDATA[transformers]]></term>
##     </controlledterms>
##     <thesaurusterms>
##       <term><![CDATA[Bandwidth]]></term>
##       <term><![CDATA[CMOS integrated circuits]]></term>
##       <term><![CDATA[Chirp]]></term>
##       <term><![CDATA[Frequency modulation]]></term>
##       <term><![CDATA[Phase locked loops]]></term>
##       <term><![CDATA[Tuning]]></term>
##     </thesaurusterms>
##     <pubtitle><![CDATA[Solid-State Circuits, IEEE Journal of]]></pubtitle>
##     <punumber><![CDATA[4]]></punumber>
##     <pubtype><![CDATA[Journals & Magazines]]></pubtype>
##     <publisher><![CDATA[IEEE]]></publisher>
##     <volume><![CDATA[49]]></volume>
##     <issue><![CDATA[5]]></issue>
##     <py><![CDATA[2014]]></py>
##     <spage><![CDATA[1081]]></spage>
##     <epage><![CDATA[1096]]></epage>
##     <abstract><![CDATA[A mm-wave digital transmitter based on a 60 GHz all-digital phase-locked loop (ADPLL) with wideband frequency modulation (FM) for FMCW radar applications is proposed. The fractional-N ADPLL employs a high-resolution 60 GHz digitally-controlled oscillator (DCO) and is capable of multi-rate two-point FM. It achieves a measured rms jitter of 590.2 fs, while the loop settles within 3 &#x03BC;s. The measured reference spur is only -74 dBc, the fractional spurs are below -62 dBc, with no other significant spurs. A closed-loop DCO gain linearization scheme realizes a GHz-level triangular chirp across multiple DCO tuning banks with a measured frequency error (i.e., nonlinearity) in the FMCW ramp of only 117 kHz rms for a 62 GHz carrier with 1.22 GHz bandwidth. The synthesizer is transformer-coupled to a 3-stage neutralized power amplifier (PA) that delivers +5 dBm to a 50 &#x03A9; load. Implemented in 65 nm CMOS, the transmitter prototype (including PA) consumes 89 mW from a 1.2 V supply.]]></abstract>
##     <issn><![CDATA[0018-9200]]></issn>
##     <htmlFlag><![CDATA[1]]></htmlFlag>
##     <arnumber><![CDATA[6736142]]></arnumber>
##     <doi><![CDATA[10.1109/JSSC.2014.2301764]]></doi>
##     <publicationId><![CDATA[6736142]]></publicationId>
##     <mdurl><![CDATA[http://ieeexplore.ieee.org/xpl/articleDetails.jsp?tp=&arnumber=6736142&contentType=Journals+%26+Magazines]]></mdurl>
##     <pdf><![CDATA[http://ieeexplore.ieee.org/stamp/stamp.jsp?arnumber=6736142]]></pdf>
##   </document>
##   <document>
##     <rank>523</rank>
##     <title><![CDATA[Efficient- and Broadband-Coupled Selective Spot-Size Converters With Damage-Free Graphene Integration Process]]></title>
##     <authors><![CDATA[Kou, R.;  Kobayashi, Y.;  Warabi, K.;  Nishi, H.;  Tsuchizawa, T.;  Yamamoto, T.;  Nakajima, H.;  Yamada, K.]]></authors>
##     <affiliations><![CDATA[NTT Nanophotonics Center, Atsugi, Japan]]></affiliations>
##     <controlledterms>
##       <term><![CDATA[elemental semiconductors]]></term>
##       <term><![CDATA[graphene]]></term>
##       <term><![CDATA[integrated optics]]></term>
##       <term><![CDATA[optical design techniques]]></term>
##       <term><![CDATA[optical fibre couplers]]></term>
##       <term><![CDATA[optical fibre losses]]></term>
##       <term><![CDATA[optical fibre polarisation]]></term>
##       <term><![CDATA[optical resonators]]></term>
##       <term><![CDATA[optimisation]]></term>
##       <term><![CDATA[silicon]]></term>
##     </controlledterms>
##     <thesaurusterms>
##       <term><![CDATA[Couplings]]></term>
##       <term><![CDATA[Graphene]]></term>
##       <term><![CDATA[Optical device fabrication]]></term>
##       <term><![CDATA[Optical fibers]]></term>
##       <term><![CDATA[Optical polarization]]></term>
##       <term><![CDATA[Silicon]]></term>
##     </thesaurusterms>
##     <pubtitle><![CDATA[Photonics Journal, IEEE]]></pubtitle>
##     <punumber><![CDATA[4563994]]></punumber>
##     <pubtype><![CDATA[Journals & Magazines]]></pubtype>
##     <publisher><![CDATA[IEEE]]></publisher>
##     <volume><![CDATA[6]]></volume>
##     <issue><![CDATA[3]]></issue>
##     <py><![CDATA[2014]]></py>
##     <spage><![CDATA[1]]></spage>
##     <epage><![CDATA[9]]></epage>
##     <abstract><![CDATA[We report an enhancement of coupling efficiency between an optical fiber and graphene photonic device integrated on a silicon photonics platform. The design and optimization of selective spot-size converters (SSSCs) enables low coupling losses of 1.4 dB/facet for TE and 2.4 dB/facet for TM modes without causing damage to the graphene. In order to maximize the potential of the perfect wavelength independent response in the graphene, ultra-broad-bandwidth coupling from O- to U-bands (1300-1650 nm) is characterized. The bandwidth is over 300 nm (at 3-dB roll off) in both polarization modes. We integrated the SSSCs with a partially graphene integrated silicon-ring resonator to substantiate the wavelength response improvement by a pico-meter-order high-resolution spectroscopy.]]></abstract>
##     <issn><![CDATA[1943-0655]]></issn>
##     <htmlFlag><![CDATA[1]]></htmlFlag>
##     <arnumber><![CDATA[6814819]]></arnumber>
##     <doi><![CDATA[10.1109/JPHOT.2014.2323297]]></doi>
##     <publicationId><![CDATA[6814819]]></publicationId>
##     <mdurl><![CDATA[http://ieeexplore.ieee.org/xpl/articleDetails.jsp?tp=&arnumber=6814819&contentType=Journals+%26+Magazines]]></mdurl>
##     <pdf><![CDATA[http://ieeexplore.ieee.org/stamp/stamp.jsp?arnumber=6814819]]></pdf>
##   </document>
##   <document>
##     <rank>524</rank>
##     <title><![CDATA[Polarization-Insensitive Phase-Transmultiplexing and Multicasting of CSRZ-OOK and 4 <formula formulatype="inline"> <img src="/images/tex/326.gif" alt="\times"> </formula> RZ-BPSK to 4 <formula formulatype="inline"> <img src="/images/tex/21445.gif" alt="\times"> </formula> RZ-QPSK via XPM in a Birefringent PCF]]></title>
##     <authors><![CDATA[Cannon, B.M.;  Mahmood, T.;  Astar, W.;  Boudra, P.;  Mohsenin, T.;  Carter, G.M.]]></authors>
##     <affiliations><![CDATA[Lab. for Phys. Sci., College Park, MD, USA]]></affiliations>
##     <controlledterms>
##       <term><![CDATA[amplitude shift keying]]></term>
##       <term><![CDATA[error statistics]]></term>
##       <term><![CDATA[holey fibres]]></term>
##       <term><![CDATA[multicast communication]]></term>
##       <term><![CDATA[optical fibre communication]]></term>
##       <term><![CDATA[optical receivers]]></term>
##       <term><![CDATA[photonic crystals]]></term>
##       <term><![CDATA[quadrature phase shift keying]]></term>
##       <term><![CDATA[transmultiplexing]]></term>
##     </controlledterms>
##     <thesaurusterms>
##       <term><![CDATA[Binary phase shift keying]]></term>
##       <term><![CDATA[Erbium-doped fiber amplifiers]]></term>
##       <term><![CDATA[Optical attenuators]]></term>
##       <term><![CDATA[Probes]]></term>
##       <term><![CDATA[Receivers]]></term>
##     </thesaurusterms>
##     <pubtitle><![CDATA[Photonics Journal, IEEE]]></pubtitle>
##     <punumber><![CDATA[4563994]]></punumber>
##     <pubtype><![CDATA[Journals & Magazines]]></pubtype>
##     <publisher><![CDATA[IEEE]]></publisher>
##     <volume><![CDATA[6]]></volume>
##     <issue><![CDATA[2]]></issue>
##     <py><![CDATA[2014]]></py>
##     <spage><![CDATA[1]]></spage>
##     <epage><![CDATA[11]]></epage>
##     <abstract><![CDATA[By utilizing cross-phase modulation and birefringence, we demonstrated simultaneous polarization-insensitive phase-transmultiplexing and multicasting of a single 10-Gbaud CSRZ-OOK signal and 4 &#x00D7; 10-Gbaud RZ-BPSK signals to generate 4 &#x00D7; 10-Gbaud RZ-QPSK signals in a photonic crystal fiber. The measured receiver sensitivity OSNR penalty at BER of 10<sup>-9</sup> was &#x2248;1.8 dB, relative to the FPGA-precoded RZ-DQPSK baseline, for a randomly polarized CSRZ-OOK signal, when the launch angles of all RZ-BPSK signals were fixed at &#x2248; 45<sup>&#x00B0;</sup>. In addition, the OSNR of the remotely generated CSRZ-OOK signal could be degraded down to &#x2248;34 dB/0.1 nm, before the BER performance of the phase-transmultiplexing and multicasting operation began to degrade.]]></abstract>
##     <issn><![CDATA[1943-0655]]></issn>
##     <htmlFlag><![CDATA[1]]></htmlFlag>
##     <arnumber><![CDATA[6755461]]></arnumber>
##     <doi><![CDATA[10.1109/JPHOT.2014.2309642]]></doi>
##     <publicationId><![CDATA[6755461]]></publicationId>
##     <mdurl><![CDATA[http://ieeexplore.ieee.org/xpl/articleDetails.jsp?tp=&arnumber=6755461&contentType=Journals+%26+Magazines]]></mdurl>
##     <pdf><![CDATA[http://ieeexplore.ieee.org/stamp/stamp.jsp?arnumber=6755461]]></pdf>
##   </document>
##   <document>
##     <rank>525</rank>
##     <title><![CDATA[DC Microgrid for Wind and Solar Power Integration]]></title>
##     <authors><![CDATA[Strunz, K.;  Abbasi, E.;  Duc Nguyen Huu]]></authors>
##     <affiliations><![CDATA[Dept. of Electr. Eng., Tech. Univ. of Berlin, Berlin, Germany]]></affiliations>
##     <controlledterms>
##       <term><![CDATA[distributed power generation]]></term>
##       <term><![CDATA[power generation control]]></term>
##       <term><![CDATA[power generation scheduling]]></term>
##       <term><![CDATA[solar power stations]]></term>
##       <term><![CDATA[wind power plants]]></term>
##     </controlledterms>
##     <thesaurusterms>
##       <term><![CDATA[Batteries]]></term>
##       <term><![CDATA[Microgrids]]></term>
##       <term><![CDATA[Optimization]]></term>
##       <term><![CDATA[Predictive models]]></term>
##       <term><![CDATA[Wind forecasting]]></term>
##       <term><![CDATA[Wind power generation]]></term>
##     </thesaurusterms>
##     <pubtitle><![CDATA[Emerging and Selected Topics in Power Electronics, IEEE Journal of]]></pubtitle>
##     <punumber><![CDATA[6245517]]></punumber>
##     <pubtype><![CDATA[Journals & Magazines]]></pubtype>
##     <publisher><![CDATA[IEEE]]></publisher>
##     <volume><![CDATA[2]]></volume>
##     <issue><![CDATA[1]]></issue>
##     <py><![CDATA[2014]]></py>
##     <spage><![CDATA[115]]></spage>
##     <epage><![CDATA[126]]></epage>
##     <abstract><![CDATA[Operational controls are designed to support the integration of wind and solar power within microgrids. An aggregated model of renewable wind and solar power generation forecast is proposed to support the quantification of the operational reserve for day-ahead and real-time scheduling. Then, a droop control for power electronic converters connected to battery storage is developed and tested. Compared with the existing droop controls, it is distinguished in that the droop curves are set as a function of the storage state-of-charge (SOC) and can become asymmetric. The adaptation of the slopes ensures that the power output supports the terminal voltage while at the same keeping the SOC within a target range of desired operational reserve. This is shown to maintain the equilibrium of the microgrid's real-time supply and demand. The controls are implemented for the special case of a dc microgrid that is vertically integrated within a high-rise host building of an urban area. Previously untapped wind and solar power are harvested on the roof and sides of a tower, thereby supporting delivery to electric vehicles on the ground. The microgrid vertically integrates with the host building without creating a large footprint.]]></abstract>
##     <issn><![CDATA[2168-6777]]></issn>
##     <htmlFlag><![CDATA[1]]></htmlFlag>
##     <arnumber><![CDATA[6681907]]></arnumber>
##     <doi><![CDATA[10.1109/JESTPE.2013.2294738]]></doi>
##     <publicationId><![CDATA[6681907]]></publicationId>
##     <mdurl><![CDATA[http://ieeexplore.ieee.org/xpl/articleDetails.jsp?tp=&arnumber=6681907&contentType=Journals+%26+Magazines]]></mdurl>
##     <pdf><![CDATA[http://ieeexplore.ieee.org/stamp/stamp.jsp?arnumber=6681907]]></pdf>
##   </document>
##   <document>
##     <rank>526</rank>
##     <title><![CDATA[Music Genre and Emotion Recognition Using Gaussian Processes]]></title>
##     <authors><![CDATA[Markov, K.;  Matsui, T.]]></authors>
##     <affiliations><![CDATA[Div. of Inf. Syst., Univ. of Aizu, Aizu-Wakamatsu, Japan]]></affiliations>
##     <controlledterms>
##       <term><![CDATA[Bayes methods]]></term>
##       <term><![CDATA[Gaussian processes]]></term>
##       <term><![CDATA[audio signal processing]]></term>
##       <term><![CDATA[emotion recognition]]></term>
##       <term><![CDATA[feature extraction]]></term>
##       <term><![CDATA[information retrieval]]></term>
##       <term><![CDATA[learning (artificial intelligence)]]></term>
##       <term><![CDATA[music]]></term>
##       <term><![CDATA[nonparametric statistics]]></term>
##       <term><![CDATA[pattern classification]]></term>
##       <term><![CDATA[support vector machines]]></term>
##     </controlledterms>
##     <thesaurusterms>
##       <term><![CDATA[Acoustic measurement]]></term>
##       <term><![CDATA[Analytical models]]></term>
##       <term><![CDATA[Bayes methods]]></term>
##       <term><![CDATA[Data models]]></term>
##       <term><![CDATA[Emotion recognition]]></term>
##       <term><![CDATA[Gaussian processes]]></term>
##       <term><![CDATA[Information retrieval]]></term>
##       <term><![CDATA[Music]]></term>
##       <term><![CDATA[Regression analysis]]></term>
##       <term><![CDATA[Support vector machines]]></term>
##       <term><![CDATA[Time series analysis]]></term>
##     </thesaurusterms>
##     <pubtitle><![CDATA[Access, IEEE]]></pubtitle>
##     <punumber><![CDATA[6287639]]></punumber>
##     <pubtype><![CDATA[Journals & Magazines]]></pubtype>
##     <publisher><![CDATA[IEEE]]></publisher>
##     <volume><![CDATA[2]]></volume>
##     <py><![CDATA[2014]]></py>
##     <spage><![CDATA[688]]></spage>
##     <epage><![CDATA[697]]></epage>
##     <abstract><![CDATA[Gaussian Processes (GPs) are Bayesian nonparametric models that are becoming more and more popular for their superior capabilities to capture highly nonlinear data relationships in various tasks, such as dimensionality reduction, time series analysis, novelty detection, as well as classical regression and classification tasks. In this paper, we investigate the feasibility and applicability of GP models for music genre classification and music emotion estimation. These are two of the main tasks in the music information retrieval (MIR) field. So far, the support vector machine (SVM) has been the dominant model used in MIR systems. Like SVM, GP models are based on kernel functions and Gram matrices; but, in contrast, they produce truly probabilistic outputs with an explicit degree of prediction uncertainty. In addition, there exist algorithms for GP hyperparameter learning-something the SVM framework lacks. In this paper, we built two systems, one for music genre classification and another for music emotion estimation using both SVM and GP models, and compared their performances on two databases of similar size. In all cases, the music audio signal was processed in the same way, and the effects of different feature extraction methods and their various combinations were also investigated. The evaluation experiments clearly showed that in both music genre classification and music emotion estimation tasks the GP performed consistently better than the SVM. The GP achieved a 13.6% relative genre classification error reduction and up to an 11% absolute increase of the coefficient of determination in the emotion estimation task.]]></abstract>
##     <issn><![CDATA[2169-3536]]></issn>
##     <htmlFlag><![CDATA[1]]></htmlFlag>
##     <arnumber><![CDATA[6843353]]></arnumber>
##     <doi><![CDATA[10.1109/ACCESS.2014.2333095]]></doi>
##     <publicationId><![CDATA[6843353]]></publicationId>
##     <mdurl><![CDATA[http://ieeexplore.ieee.org/xpl/articleDetails.jsp?tp=&arnumber=6843353&contentType=Journals+%26+Magazines]]></mdurl>
##     <pdf><![CDATA[http://ieeexplore.ieee.org/stamp/stamp.jsp?arnumber=6843353]]></pdf>
##   </document>
##   <document>
##     <rank>527</rank>
##     <title><![CDATA[Sensorless Predictive Peak Current Control for Boost Converter Using Comprehensive Compensation Strategy]]></title>
##     <authors><![CDATA[Qiaoling Tong;  Qiao Zhang;  Run Min;  Xuecheng Zou;  Zhenglin Liu;  Zhiqian Chen]]></authors>
##     <affiliations><![CDATA[Sch. of Opt. & Electron. Inf., Huazhong Univ. of Sci. & Technol., Wuhan, China]]></affiliations>
##     <controlledterms>
##       <term><![CDATA[PI control]]></term>
##       <term><![CDATA[electric current control]]></term>
##       <term><![CDATA[observers]]></term>
##       <term><![CDATA[power convertors]]></term>
##       <term><![CDATA[predictive control]]></term>
##       <term><![CDATA[transient response]]></term>
##     </controlledterms>
##     <thesaurusterms>
##       <term><![CDATA[Current control]]></term>
##       <term><![CDATA[Inductors]]></term>
##       <term><![CDATA[Mathematical model]]></term>
##       <term><![CDATA[Observers]]></term>
##       <term><![CDATA[Steady-state]]></term>
##       <term><![CDATA[Switches]]></term>
##       <term><![CDATA[Voltage control]]></term>
##     </thesaurusterms>
##     <pubtitle><![CDATA[Industrial Electronics, IEEE Transactions on]]></pubtitle>
##     <punumber><![CDATA[41]]></punumber>
##     <pubtype><![CDATA[Journals & Magazines]]></pubtype>
##     <publisher><![CDATA[IEEE]]></publisher>
##     <volume><![CDATA[61]]></volume>
##     <issue><![CDATA[6]]></issue>
##     <py><![CDATA[2014]]></py>
##     <spage><![CDATA[2754]]></spage>
##     <epage><![CDATA[2766]]></epage>
##     <abstract><![CDATA[For a sensorless predictive-peak-current-controlled boost converter, the output voltage steady-state error cannot be eliminated by voltage loop PI controller. The basic cause for this is investigated through analysis and theoretical approaches. To eliminate the voltage steady-state error and achieve high-accuracy current estimation, a comprehensive compensation strategy is proposed. First, a compensation algorithm for output voltage sampling is introduced. It can not only effectively eliminate the output voltage steady-state error but also guarantee current observer convergence. The compensation schemes for component parasitic parameter effects and switching delay are also investigated. With this comprehensive compensation strategy, both the system transient response and current estimation accuracy are greatly improved. Finally, the effectiveness of the proposed algorithm is verified by experimental results.]]></abstract>
##     <issn><![CDATA[0278-0046]]></issn>
##     <htmlFlag><![CDATA[1]]></htmlFlag>
##     <arnumber><![CDATA[6567904]]></arnumber>
##     <doi><![CDATA[10.1109/TIE.2013.2274428]]></doi>
##     <publicationId><![CDATA[6567904]]></publicationId>
##     <mdurl><![CDATA[http://ieeexplore.ieee.org/xpl/articleDetails.jsp?tp=&arnumber=6567904&contentType=Journals+%26+Magazines]]></mdurl>
##     <pdf><![CDATA[http://ieeexplore.ieee.org/stamp/stamp.jsp?arnumber=6567904]]></pdf>
##   </document>
##   <document>
##     <rank>528</rank>
##     <title><![CDATA[A 101 dB PSRR, 0.0027% THD + N and 94% Power-Efficiency Filterless Class D Amplifier]]></title>
##     <authors><![CDATA[Linfei Guo;  Tong Ge;  Chang, J.S.]]></authors>
##     <affiliations><![CDATA[Sch. of Electr. & Electron. Eng., Nanyang Technol. Univ., Singapore, Singapore]]></affiliations>
##     <controlledterms>
##       <term><![CDATA[CMOS integrated circuits]]></term>
##       <term><![CDATA[audio-frequency amplifiers]]></term>
##       <term><![CDATA[harmonic distortion]]></term>
##       <term><![CDATA[power amplifiers]]></term>
##       <term><![CDATA[pulse width modulation]]></term>
##     </controlledterms>
##     <thesaurusterms>
##       <term><![CDATA[Distortion]]></term>
##       <term><![CDATA[Generators]]></term>
##       <term><![CDATA[Noise]]></term>
##       <term><![CDATA[Phase modulation]]></term>
##       <term><![CDATA[Pulse width modulation]]></term>
##       <term><![CDATA[Switches]]></term>
##       <term><![CDATA[Switching frequency]]></term>
##     </thesaurusterms>
##     <pubtitle><![CDATA[Solid-State Circuits, IEEE Journal of]]></pubtitle>
##     <punumber><![CDATA[4]]></punumber>
##     <pubtype><![CDATA[Journals & Magazines]]></pubtype>
##     <publisher><![CDATA[IEEE]]></publisher>
##     <volume><![CDATA[49]]></volume>
##     <issue><![CDATA[11]]></issue>
##     <py><![CDATA[2014]]></py>
##     <spage><![CDATA[2608]]></spage>
##     <epage><![CDATA[2617]]></epage>
##     <abstract><![CDATA[Present-day smartphones and tablets demand high audio fidelity (e.g., total harmonic distortion + noise, THD + N &#x226A; 0.01%), and high noise immunity (e.g., power supply rejection ratio, PSRR &#x226B; 80 dB) to allow high integration in an SoC. The design of conventional closed-loop pulse width modulation (PWM) Class-D amplifiers (CDAs) typically involves undesirable trade-offs between fidelity (qualified by THD + N), PSRR and switching frequency. In this paper, we propose a fully integrated CMOS CDA that embodies a novel input-modulated carrier generator and a novel phase-error-free PWM modulator, collectively allowing the employment of high loop-gain to achieve high PSRR, yet without compromising linearity/dynamic-range or resorting to high switching frequency. The prototype CDA, realized in 65 nm CMOS, achieves a THD + N of 0.0027% and a power efficiency of 94% when delivering 500 mW to an 8 &#x03A9; load from V<sub>DD</sub> = 3.6 V. The PSRR of the prototype CDA is very high, -101 dB @217 Hz and 90 dB @1 kHz, arguably the highest to-date. Furthermore, the switching frequency of the prototype CDA varies from ~320 to 420 kHz, potentially reducing the EMI due to spread-spectrum. In addition, the prototype CDA is versatile with a large operating-voltage range, with V ranging from rechargeable 1.2 V single battery to standard 3.6 V smart-device supply voltages.]]></abstract>
##     <issn><![CDATA[0018-9200]]></issn>
##     <htmlFlag><![CDATA[1]]></htmlFlag>
##     <arnumber><![CDATA[6919339]]></arnumber>
##     <doi><![CDATA[10.1109/JSSC.2014.2359913]]></doi>
##     <publicationId><![CDATA[6919339]]></publicationId>
##     <mdurl><![CDATA[http://ieeexplore.ieee.org/xpl/articleDetails.jsp?tp=&arnumber=6919339&contentType=Journals+%26+Magazines]]></mdurl>
##     <pdf><![CDATA[http://ieeexplore.ieee.org/stamp/stamp.jsp?arnumber=6919339]]></pdf>
##   </document>
##   <document>
##     <rank>529</rank>
##     <title><![CDATA[Demonstration of <inline-formula> <img src="/images/tex/21809.gif" alt="V_{\pi}"> </inline-formula> Reduction in Electrooptic Modulators Using Modulation Instability]]></title>
##     <authors><![CDATA[Borlaug, D.;  DeVore, P.T.S.;  Rostami, A.;  Boyraz, O.;  Jalali, B.]]></authors>
##     <affiliations><![CDATA[Dept. of Electr. Eng., Univ. of California, Los Angeles, Los Angeles, CA, USA]]></affiliations>
##     <controlledterms>
##       <term><![CDATA[electro-optical modulation]]></term>
##       <term><![CDATA[optical fibre communication]]></term>
##     </controlledterms>
##     <thesaurusterms>
##       <term><![CDATA[Amplitude modulation]]></term>
##       <term><![CDATA[Electrooptic modulators]]></term>
##       <term><![CDATA[Noise]]></term>
##       <term><![CDATA[Optical fiber amplifiers]]></term>
##     </thesaurusterms>
##     <pubtitle><![CDATA[Photonics Journal, IEEE]]></pubtitle>
##     <punumber><![CDATA[4563994]]></punumber>
##     <pubtype><![CDATA[Journals & Magazines]]></pubtype>
##     <publisher><![CDATA[IEEE]]></publisher>
##     <volume><![CDATA[6]]></volume>
##     <issue><![CDATA[5]]></issue>
##     <py><![CDATA[2014]]></py>
##     <spage><![CDATA[1]]></spage>
##     <epage><![CDATA[9]]></epage>
##     <abstract><![CDATA[Reduction in the operating voltage of electrooptic modulators is needed, in order to remove the bottleneck in the flow of data between electronics and optical interconnects. We report an experimental demonstration of a tenfold reduction in an electrooptic modulator's half-wave voltage up to 50 GHz. This was achieved by employing optical sideband-only amplification.]]></abstract>
##     <issn><![CDATA[1943-0655]]></issn>
##     <htmlFlag><![CDATA[1]]></htmlFlag>
##     <arnumber><![CDATA[6884807]]></arnumber>
##     <doi><![CDATA[10.1109/JPHOT.2014.2352626]]></doi>
##     <publicationId><![CDATA[6884807]]></publicationId>
##     <mdurl><![CDATA[http://ieeexplore.ieee.org/xpl/articleDetails.jsp?tp=&arnumber=6884807&contentType=Journals+%26+Magazines]]></mdurl>
##     <pdf><![CDATA[http://ieeexplore.ieee.org/stamp/stamp.jsp?arnumber=6884807]]></pdf>
##   </document>
##   <document>
##     <rank>530</rank>
##     <title><![CDATA[On the Number of Ranked Species Trees Producing Anomalous Ranked Gene Trees]]></title>
##     <authors><![CDATA[Disanto, F.;  Rosenberg, N.A.]]></authors>
##     <affiliations><![CDATA[Dept. of Biol., Stanford Univ., Stanford, CA, USA]]></affiliations>
##     <controlledterms>
##       <term><![CDATA[evolution (biological)]]></term>
##       <term><![CDATA[genetics]]></term>
##       <term><![CDATA[probability]]></term>
##       <term><![CDATA[trees (mathematics)]]></term>
##     </controlledterms>
##     <thesaurusterms>
##       <term><![CDATA[Bioinformatics]]></term>
##       <term><![CDATA[Computational biology]]></term>
##       <term><![CDATA[Genetics]]></term>
##       <term><![CDATA[Labeling]]></term>
##       <term><![CDATA[Phylogeny]]></term>
##       <term><![CDATA[Vegetation]]></term>
##     </thesaurusterms>
##     <pubtitle><![CDATA[Computational Biology and Bioinformatics, IEEE/ACM Transactions on]]></pubtitle>
##     <punumber><![CDATA[8857]]></punumber>
##     <pubtype><![CDATA[Journals & Magazines]]></pubtype>
##     <publisher><![CDATA[IEEE]]></publisher>
##     <volume><![CDATA[11]]></volume>
##     <issue><![CDATA[6]]></issue>
##     <py><![CDATA[2014]]></py>
##     <spage><![CDATA[1229]]></spage>
##     <epage><![CDATA[1238]]></epage>
##     <abstract><![CDATA[Analysis of probability distributions conditional on species trees has demonstrated the existence of anomalous ranked gene trees (ARGTs), ranked gene trees that are more probable than the ranked gene tree that accords with the ranked species tree. Here, to improve the characterization of ARGTs, we study enumerative and probabilistic properties of two classes of ranked labeled species trees, focusing on the presence or avoidance of certain subtree patterns associated with the production of ARGTs. We provide exact enumerations and asymptotic estimates for cardinalities of these sets of trees, showing that as the number of species increases without bound, the fraction of all ranked labeled species trees that are ARGT-producing approaches 1. This result extends beyond earlier existence results to provide a probabilistic claim about the frequency of ARGTs.]]></abstract>
##     <issn><![CDATA[1545-5963]]></issn>
##     <htmlFlag><![CDATA[1]]></htmlFlag>
##     <arnumber><![CDATA[6867321]]></arnumber>
##     <doi><![CDATA[10.1109/TCBB.2014.2343977]]></doi>
##     <publicationId><![CDATA[6867321]]></publicationId>
##     <mdurl><![CDATA[http://ieeexplore.ieee.org/xpl/articleDetails.jsp?tp=&arnumber=6867321&contentType=Journals+%26+Magazines]]></mdurl>
##     <pdf><![CDATA[http://ieeexplore.ieee.org/stamp/stamp.jsp?arnumber=6867321]]></pdf>
##   </document>
##   <document>
##     <rank>531</rank>
##     <title><![CDATA[Light Trapping Textures Designed by Electromagnetic Optimization for Subwavelength Thick Solar Cells]]></title>
##     <authors><![CDATA[Ganapati, V.;  Miller, O.D.;  Yablonovitch, E.]]></authors>
##     <affiliations><![CDATA[Mater. Sci. Div., Univ. of California, Berkeley, Berkeley, CA, USA]]></affiliations>
##     <controlledterms>
##       <term><![CDATA[optimisation]]></term>
##       <term><![CDATA[ray tracing]]></term>
##       <term><![CDATA[refractive index]]></term>
##       <term><![CDATA[solar cells]]></term>
##       <term><![CDATA[surface texture]]></term>
##     </controlledterms>
##     <thesaurusterms>
##       <term><![CDATA[Absorption]]></term>
##       <term><![CDATA[Charge carrier processes]]></term>
##       <term><![CDATA[Materials]]></term>
##       <term><![CDATA[Optical surface waves]]></term>
##       <term><![CDATA[Optimization]]></term>
##       <term><![CDATA[Photovoltaic cells]]></term>
##       <term><![CDATA[Surface texture]]></term>
##     </thesaurusterms>
##     <pubtitle><![CDATA[Photovoltaics, IEEE Journal of]]></pubtitle>
##     <punumber><![CDATA[5503869]]></punumber>
##     <pubtype><![CDATA[Journals & Magazines]]></pubtype>
##     <publisher><![CDATA[IEEE]]></publisher>
##     <volume><![CDATA[4]]></volume>
##     <issue><![CDATA[1]]></issue>
##     <py><![CDATA[2014]]></py>
##     <spage><![CDATA[175]]></spage>
##     <epage><![CDATA[182]]></epage>
##     <abstract><![CDATA[Light trapping in solar cells allows for increased current and voltage, as well as reduced materials cost. It is known that in geometrical optics, a maximum 4 n<sup>2</sup> absorption enhancement factor can be achieved by randomly texturing the surface of the solar cell, where n is the material refractive index. This ray-optics absorption enhancement (AE) limit only holds when the thickness of the solar cell is much greater than the optical wavelength. In subwavelength thin films, the fundamental questions remain unanswered: 1) what is the subwavelength AE limit and 2) what surface texture realizes this optimal AE? We turn to computational electromagnetic optimization in order to design nanoscale textures for light trapping in subwavelength thin films. For high-index thin films, in the weakly absorbing limit, our optimized surface textures yield an angle- and frequency-averaged enhancement factor ~39. They perform roughly 30% better than randomly textured structures, but they fall short of the ray optics enhancement limit of 4 n<sup>2</sup> ~ 50.]]></abstract>
##     <issn><![CDATA[2156-3381]]></issn>
##     <htmlFlag><![CDATA[1]]></htmlFlag>
##     <arnumber><![CDATA[6601630]]></arnumber>
##     <doi><![CDATA[10.1109/JPHOTOV.2013.2280340]]></doi>
##     <publicationId><![CDATA[6601630]]></publicationId>
##     <mdurl><![CDATA[http://ieeexplore.ieee.org/xpl/articleDetails.jsp?tp=&arnumber=6601630&contentType=Journals+%26+Magazines]]></mdurl>
##     <pdf><![CDATA[http://ieeexplore.ieee.org/stamp/stamp.jsp?arnumber=6601630]]></pdf>
##   </document>
##   <document>
##     <rank>532</rank>
##     <title><![CDATA[Multilayer Periodic and Random Metamaterial Structures: Analysis and Applications]]></title>
##     <authors><![CDATA[Aylo, R.;  Nehmetallah, G.;  Han Li;  Banerjee, P.P.]]></authors>
##     <affiliations><![CDATA[Dept. of Electr. & Eng. Comput. Sci., Catholic Univ. of America, Washington, DC, USA]]></affiliations>
##     <controlledterms>
##       <term><![CDATA[light propagation]]></term>
##       <term><![CDATA[light reflection]]></term>
##       <term><![CDATA[light transmission]]></term>
##       <term><![CDATA[optical losses]]></term>
##       <term><![CDATA[optical metamaterials]]></term>
##       <term><![CDATA[optical multilayers]]></term>
##       <term><![CDATA[optical sensors]]></term>
##       <term><![CDATA[periodic structures]]></term>
##       <term><![CDATA[photonic band gap]]></term>
##       <term><![CDATA[random media]]></term>
##     </controlledterms>
##     <thesaurusterms>
##       <term><![CDATA[Image sensors]]></term>
##       <term><![CDATA[Indexes]]></term>
##       <term><![CDATA[Metamaterials]]></term>
##       <term><![CDATA[Nonhomogeneous media]]></term>
##       <term><![CDATA[Photonic band gap]]></term>
##       <term><![CDATA[Sensors]]></term>
##     </thesaurusterms>
##     <pubtitle><![CDATA[Access, IEEE]]></pubtitle>
##     <punumber><![CDATA[6287639]]></punumber>
##     <pubtype><![CDATA[Journals & Magazines]]></pubtype>
##     <publisher><![CDATA[IEEE]]></publisher>
##     <volume><![CDATA[2]]></volume>
##     <py><![CDATA[2014]]></py>
##     <spage><![CDATA[437]]></spage>
##     <epage><![CDATA[450]]></epage>
##     <abstract><![CDATA[In recent years, multilayer photonic bandgap structures comprising stacks of alternating layers of positive and negative index have been proposed for a variety of applications, such as perfect imaging, filters, sensors, coatings for tailored emittance, absorptance, etc. Following a brief review of the history of negative index materials, the performance of such stacks is reviewed, with emphasis on analysis of plane wave and beam propagation, and possible applications in sensing. First, the use of the transfer matrix method to analyze plane wave propagation in such structures to determine the transmittance and reflectance is developed. Examples of cases where the Bragg bandgap and the so-called zero &lt;;(n ) &gt; gap can be used for possible applications in sensing are illustrated. Next, the transfer matrix approach is extended to simulate the spatial evolution of a collection of propagating and nonpropagating TE and TM plane waves (or plane wave spectra) incident on such multilayer structures. The use of the complex Poynting theorem in checking the computations, as well as monitoring powers and the stored electric or magnetic energy in any section of the multilayer stack, is illustrated, along with its use in designing alternating positive and negative index structures with optimal gain to compensate for losses in the negative index material. Finally, the robustness of PIM-NIM stacks with respect to randomness in the dimensions of the PIM-NIM structure is examined. This should be useful in determining the performance of such structures when they are physically fabricated.]]></abstract>
##     <issn><![CDATA[2169-3536]]></issn>
##     <htmlFlag><![CDATA[1]]></htmlFlag>
##     <arnumber><![CDATA[6813606]]></arnumber>
##     <doi><![CDATA[10.1109/ACCESS.2014.2321661]]></doi>
##     <publicationId><![CDATA[6813606]]></publicationId>
##     <mdurl><![CDATA[http://ieeexplore.ieee.org/xpl/articleDetails.jsp?tp=&arnumber=6813606&contentType=Journals+%26+Magazines]]></mdurl>
##     <pdf><![CDATA[http://ieeexplore.ieee.org/stamp/stamp.jsp?arnumber=6813606]]></pdf>
##   </document>
##   <document>
##     <rank>533</rank>
##     <title><![CDATA[A Circuit Synthesis Flow for Controllable-Polarity Transistors]]></title>
##     <authors><![CDATA[Amaru, L.;  Gaillardon, P.-E.;  De Micheli, G.]]></authors>
##     <affiliations><![CDATA[Integrated Syst. Lab., Ecole Polytech. Fed. de Lausanne, Lausanne, Switzerland]]></affiliations>
##     <controlledterms>
##       <term><![CDATA[digital arithmetic]]></term>
##       <term><![CDATA[digital circuits]]></term>
##       <term><![CDATA[elemental semiconductors]]></term>
##       <term><![CDATA[field effect transistors]]></term>
##       <term><![CDATA[nanowires]]></term>
##       <term><![CDATA[network synthesis]]></term>
##       <term><![CDATA[silicon]]></term>
##     </controlledterms>
##     <thesaurusterms>
##       <term><![CDATA[Data structures]]></term>
##       <term><![CDATA[Design automation]]></term>
##       <term><![CDATA[Field effect transistors]]></term>
##       <term><![CDATA[Layout]]></term>
##       <term><![CDATA[Logic gates]]></term>
##       <term><![CDATA[Nanowires]]></term>
##     </thesaurusterms>
##     <pubtitle><![CDATA[Nanotechnology, IEEE Transactions on]]></pubtitle>
##     <punumber><![CDATA[7729]]></punumber>
##     <pubtype><![CDATA[Journals & Magazines]]></pubtype>
##     <publisher><![CDATA[IEEE]]></publisher>
##     <volume><![CDATA[13]]></volume>
##     <issue><![CDATA[6]]></issue>
##     <py><![CDATA[2014]]></py>
##     <spage><![CDATA[1074]]></spage>
##     <epage><![CDATA[1083]]></epage>
##     <abstract><![CDATA[Double-gate (DG) controllable-polarity field-effect transistors (FETs) are devices whose n- or p- polarity is online configurable by adjusting the second gate voltage. Such emerging transistors have been fabricated in silicon nanowires, carbon nanotubes, and graphene technologies. Thanks to their enhanced functionality, DG controllable-polarity FETs implement arithmetic functions, such as XOR and MAJ, with limited physical resources enabling compact and high-performance datapaths. In order to design digital circuits with this technology, automated design techniques are of paramount importance. In this paper, we describe a design automation framework for DG controllable-polarity transistors. First, we present a novel dedicated logic representation form capable to exploit the polarity control during logic synthesis. Then, we tackle challenges at the physical level, presenting a regular layout technique that alleviates the interconnection issue deriving from the second gate routing. We use logic and physical synthesis tools to form a complete design automation flow. Experimental results show that the proposed flow is able to reduce the area and delay of digital circuits, based on 22-nm DG controllable-polarity Silicon nanowire (SiNW) FETs, by 22% and 42%, respectively, as compared to a commercial synthesis tool. With respect to a 22-nm FinFET technology, the proposed flow produces circuits, based on 22-nm DG controllable-polarity SiNWFETs, with 2.9 &#x00D7; smaller area-delay product.]]></abstract>
##     <issn><![CDATA[1536-125X]]></issn>
##     <htmlFlag><![CDATA[1]]></htmlFlag>
##     <arnumber><![CDATA[6914607]]></arnumber>
##     <doi><![CDATA[10.1109/TNANO.2014.2361059]]></doi>
##     <publicationId><![CDATA[6914607]]></publicationId>
##     <mdurl><![CDATA[http://ieeexplore.ieee.org/xpl/articleDetails.jsp?tp=&arnumber=6914607&contentType=Journals+%26+Magazines]]></mdurl>
##     <pdf><![CDATA[http://ieeexplore.ieee.org/stamp/stamp.jsp?arnumber=6914607]]></pdf>
##   </document>
##   <document>
##     <rank>534</rank>
##     <title><![CDATA[Complex Locations of Equivalent Dipoles for Improved Characterization of Radiated Emissions]]></title>
##     <authors><![CDATA[Obiekezie, C.S.;  Thomas, D.W.P.;  Nothofer, A.;  Greedy, S.;  Arnaut, L.R.;  Sewell, P.]]></authors>
##     <affiliations><![CDATA[Electr. Syst. & Opt. Res. Div., Univ. of Nottingham, Nottingham, UK]]></affiliations>
##     <controlledterms>
##       <term><![CDATA[electromagnetic wave propagation]]></term>
##       <term><![CDATA[mathematical analysis]]></term>
##       <term><![CDATA[particle swarm optimisation]]></term>
##     </controlledterms>
##     <thesaurusterms>
##       <term><![CDATA[Accuracy]]></term>
##       <term><![CDATA[Atmospheric measurements]]></term>
##       <term><![CDATA[Computational modeling]]></term>
##       <term><![CDATA[Image edge detection]]></term>
##       <term><![CDATA[Magnetic moments]]></term>
##       <term><![CDATA[Mathematical model]]></term>
##       <term><![CDATA[Vectors]]></term>
##     </thesaurusterms>
##     <pubtitle><![CDATA[Electromagnetic Compatibility, IEEE Transactions on]]></pubtitle>
##     <punumber><![CDATA[15]]></punumber>
##     <pubtype><![CDATA[Journals & Magazines]]></pubtype>
##     <publisher><![CDATA[IEEE]]></publisher>
##     <volume><![CDATA[56]]></volume>
##     <issue><![CDATA[5]]></issue>
##     <py><![CDATA[2014]]></py>
##     <spage><![CDATA[1087]]></spage>
##     <epage><![CDATA[1094]]></epage>
##     <abstract><![CDATA[The positioning of electromagnetic (EM) sources on the complex plane, though a mathematical construct, is often applied in solving EM problems with directive confined (collimated) propagation characteristics. Equivalent dipole modeling, which finds its application in characterizing various current sources can be computationally expensive for large structures. Here, the complex localization of equivalent source points combined with the particle swarm optimization is used to improve the performance of the equivalent dipole modeling.]]></abstract>
##     <issn><![CDATA[0018-9375]]></issn>
##     <htmlFlag><![CDATA[1]]></htmlFlag>
##     <arnumber><![CDATA[6784500]]></arnumber>
##     <doi><![CDATA[10.1109/TEMC.2014.2313406]]></doi>
##     <publicationId><![CDATA[6784500]]></publicationId>
##     <mdurl><![CDATA[http://ieeexplore.ieee.org/xpl/articleDetails.jsp?tp=&arnumber=6784500&contentType=Journals+%26+Magazines]]></mdurl>
##     <pdf><![CDATA[http://ieeexplore.ieee.org/stamp/stamp.jsp?arnumber=6784500]]></pdf>
##   </document>
##   <document>
##     <rank>535</rank>
##     <title><![CDATA[Efficient Coupling Between Photonic and Dielectric-Loaded Surface Plasmon Polariton Waveguides With the Same Core Material]]></title>
##     <authors><![CDATA[Hyun-Jun Lim;  Min-Suk Kwon]]></authors>
##     <affiliations><![CDATA[Sch. of Electr. & Comput. Eng., Ulsan Nat. Inst. of Sci. & Technol., Ulsan, South Korea]]></affiliations>
##     <controlledterms>
##       <term><![CDATA[integrated optics]]></term>
##       <term><![CDATA[optical couplers]]></term>
##       <term><![CDATA[optical losses]]></term>
##       <term><![CDATA[optical waveguides]]></term>
##       <term><![CDATA[polaritons]]></term>
##       <term><![CDATA[refractive index]]></term>
##       <term><![CDATA[surface plasmons]]></term>
##     </controlledterms>
##     <thesaurusterms>
##       <term><![CDATA[Couplings]]></term>
##       <term><![CDATA[Dielectrics]]></term>
##       <term><![CDATA[Optical losses]]></term>
##       <term><![CDATA[Optical waveguides]]></term>
##       <term><![CDATA[Photonics]]></term>
##       <term><![CDATA[Refractive index]]></term>
##     </thesaurusterms>
##     <pubtitle><![CDATA[Photonics Journal, IEEE]]></pubtitle>
##     <punumber><![CDATA[4563994]]></punumber>
##     <pubtype><![CDATA[Journals & Magazines]]></pubtype>
##     <publisher><![CDATA[IEEE]]></publisher>
##     <volume><![CDATA[6]]></volume>
##     <issue><![CDATA[3]]></issue>
##     <py><![CDATA[2014]]></py>
##     <spage><![CDATA[1]]></spage>
##     <epage><![CDATA[9]]></epage>
##     <abstract><![CDATA[We theoretically investigate how to efficiently couple a photonic waveguide to a dielectric-loaded surface plasmon polariton (DLSPP) waveguide when they are based on a common core material. The DLSPP waveguide is tapered and butt coupled to the photonic waveguide. First, we propose the use of a dielectric with a higher refractive index than the dielectrics of previous DLSPP waveguides. The photonic and DLSPP waveguides are designed to reduce the loss and tapering region length of the coupling, and the tapering region is optimized. We achieve the coupling between the photonic and DLSPP waveguides based on a dielectric of refractive index of 1.57 with a coupling loss of 2.3 dB through a 3-&#x03BC;m-long coupling region. The coupling loss is further reduced by modifying the DLSPP waveguide into a double-DLSPP ( D<sup>2</sup>LSPP) waveguide. The D<sup>2</sup>LSPP waveguide has an additional low-index dielectric between its high-index dielectric and metal layer. Designed appropriately, the D<sup>2</sup>LSPP waveguide can be coupled to the photonic waveguide with a coupling loss of 1.1 dB through a 4-&#x03BC;m-long coupling region. Since the photonic and DLSPP or D<sup>2</sup>LSPP waveguides investigated in this paper can be simultaneously fabricated, they may constitute an easily realizable hybrid planar lightwave circuit with a relatively low loss.]]></abstract>
##     <issn><![CDATA[1943-0655]]></issn>
##     <htmlFlag><![CDATA[1]]></htmlFlag>
##     <arnumber><![CDATA[6820730]]></arnumber>
##     <doi><![CDATA[10.1109/JPHOT.2014.2326657]]></doi>
##     <publicationId><![CDATA[6820730]]></publicationId>
##     <mdurl><![CDATA[http://ieeexplore.ieee.org/xpl/articleDetails.jsp?tp=&arnumber=6820730&contentType=Journals+%26+Magazines]]></mdurl>
##     <pdf><![CDATA[http://ieeexplore.ieee.org/stamp/stamp.jsp?arnumber=6820730]]></pdf>
##   </document>
##   <document>
##     <rank>536</rank>
##     <title><![CDATA[Ultrabroadband Silicon-on-Insulator Polarization Beam Splitter Based on Cascaded Mode-Sorting Asymmetric Y-Junctions]]></title>
##     <authors><![CDATA[Jing Wang;  Minghao Qi;  Yi Xuan;  Haiyang Huang;  You Li;  Ming Li;  Xin Chen;  Qi Jia;  Zhen Sheng;  Aimin Wu;  Wei Li;  Xi Wang;  Shichang Zou;  Fuwan Gan]]></authors>
##     <affiliations><![CDATA[State Key Lab. of Functional Mater. for Inf., Shanghai Inst. of Microsyst. & Inf. Technol., Shanghai, China]]></affiliations>
##     <controlledterms>
##       <term><![CDATA[CMOS integrated circuits]]></term>
##       <term><![CDATA[light polarisation]]></term>
##       <term><![CDATA[optical beam splitters]]></term>
##       <term><![CDATA[optical communication]]></term>
##       <term><![CDATA[optical losses]]></term>
##       <term><![CDATA[silicon-on-insulator]]></term>
##       <term><![CDATA[tolerance analysis]]></term>
##     </controlledterms>
##     <thesaurusterms>
##       <term><![CDATA[Bandwidth]]></term>
##       <term><![CDATA[Optical device fabrication]]></term>
##       <term><![CDATA[Optical fiber communication]]></term>
##       <term><![CDATA[Optical waveguides]]></term>
##       <term><![CDATA[Silicon photonics]]></term>
##       <term><![CDATA[Sorting]]></term>
##     </thesaurusterms>
##     <pubtitle><![CDATA[Photonics Journal, IEEE]]></pubtitle>
##     <punumber><![CDATA[4563994]]></punumber>
##     <pubtype><![CDATA[Journals & Magazines]]></pubtype>
##     <publisher><![CDATA[IEEE]]></publisher>
##     <volume><![CDATA[6]]></volume>
##     <issue><![CDATA[6]]></issue>
##     <py><![CDATA[2014]]></py>
##     <spage><![CDATA[1]]></spage>
##     <epage><![CDATA[8]]></epage>
##     <abstract><![CDATA[A very novel silicon-on-insulator polarization beam splitter is proposed based on cascaded mode-sorting asymmetric Y-junctions. The width and length of each Y-junction are optimized to achieve correct mode sorting with high conversion efficiency. The numerical simulation results show that the mode conversion efficiency increases with the length of the Y-junction for the waveguide widths varying in a large range. This proposed device has &lt;; 0.7 dB insertion loss with &gt; 22 dB polarization extinction ratio in an ultrabroad wavelength range from 1450 to 1750 nm. Fabrication tolerance analysis is also performed with respect to the deviation in device width and height. With such a broad operating bandwidth and robust fabrication tolerance, this device offers potential applications for complementary metal oxide semiconductor (CMOS)-compatible polarization diversity operating across the S, C, L, and U optical communication bands, particularly in the booming 100-Gb/s coherent optical communications based on silicon photonics technology.]]></abstract>
##     <issn><![CDATA[1943-0655]]></issn>
##     <htmlFlag><![CDATA[1]]></htmlFlag>
##     <arnumber><![CDATA[6949597]]></arnumber>
##     <doi><![CDATA[10.1109/JPHOT.2014.2368780]]></doi>
##     <publicationId><![CDATA[6949597]]></publicationId>
##     <mdurl><![CDATA[http://ieeexplore.ieee.org/xpl/articleDetails.jsp?tp=&arnumber=6949597&contentType=Journals+%26+Magazines]]></mdurl>
##     <pdf><![CDATA[http://ieeexplore.ieee.org/stamp/stamp.jsp?arnumber=6949597]]></pdf>
##   </document>
##   <document>
##     <rank>537</rank>
##     <title><![CDATA[Package Optimization of the Cilium-Type MEMS Bionic Vector Hydrophone]]></title>
##     <authors><![CDATA[Liu Linxian;  Zhang Wendong;  Zhang Guojun;  Xue Chenyang]]></authors>
##     <affiliations><![CDATA[Key Lab. of Instrum. Sci. & Dynamic Meas., North Univ. of China, Taiyuan, China]]></affiliations>
##     <controlledterms>
##       <term><![CDATA[biocybernetics]]></term>
##       <term><![CDATA[calibration]]></term>
##       <term><![CDATA[electronics packaging]]></term>
##       <term><![CDATA[hydrophones]]></term>
##       <term><![CDATA[microsensors]]></term>
##       <term><![CDATA[optimisation]]></term>
##     </controlledterms>
##     <thesaurusterms>
##       <term><![CDATA[Acoustics]]></term>
##       <term><![CDATA[Frequency response]]></term>
##       <term><![CDATA[Materials]]></term>
##       <term><![CDATA[Resonant frequency]]></term>
##       <term><![CDATA[Sensitivity]]></term>
##       <term><![CDATA[Sonar equipment]]></term>
##       <term><![CDATA[Vectors]]></term>
##     </thesaurusterms>
##     <pubtitle><![CDATA[Sensors Journal, IEEE]]></pubtitle>
##     <punumber><![CDATA[7361]]></punumber>
##     <pubtype><![CDATA[Journals & Magazines]]></pubtype>
##     <publisher><![CDATA[IEEE]]></publisher>
##     <volume><![CDATA[14]]></volume>
##     <issue><![CDATA[4]]></issue>
##     <py><![CDATA[2014]]></py>
##     <spage><![CDATA[1185]]></spage>
##     <epage><![CDATA[1192]]></epage>
##     <abstract><![CDATA[We optimize the package of the cilium-type MEMS bionic vector hydrophone introduced by Chenyang and Wendong in 2007, which has the disadvantages of a low receiving sensitivity, narrow frequency band, and fluctuating frequency response curve. Initially, a full parametric analysis of frequency response and sensitivity with different material sound transparent cap were made. Then, we propose and realize an umbrella-type packaged structure with high receiving sensitivity and wide frequency band. The theoretical analysis and simulation analysis are conducted by ANSYS software and LMS virtual. Lab acoustic software. Finally, to verify the practicability of the package, the umbrella-type packaged hydrophone calibration was carried out in the National Defense Underwater Acoustics Calibration Laboratory of China. The test results show that the performance of umbrella-type hydrophone has been greatly improved compared with the previous packaged hydrophone: exhibiting a receiving sensitivity of -178 dB (increasing by 20 dB, 0-dB reference 1 V/&#x03BC;Pa), the frequency response ranging from 20 Hz to 2 kHz (broaden one times), the fluctuation of frequency response curve within &#x00B1;2 dB, and a good dipole directivity.]]></abstract>
##     <issn><![CDATA[1530-437X]]></issn>
##     <htmlFlag><![CDATA[1]]></htmlFlag>
##     <arnumber><![CDATA[6678570]]></arnumber>
##     <doi><![CDATA[10.1109/JSEN.2013.2293669]]></doi>
##     <publicationId><![CDATA[6678570]]></publicationId>
##     <mdurl><![CDATA[http://ieeexplore.ieee.org/xpl/articleDetails.jsp?tp=&arnumber=6678570&contentType=Journals+%26+Magazines]]></mdurl>
##     <pdf><![CDATA[http://ieeexplore.ieee.org/stamp/stamp.jsp?arnumber=6678570]]></pdf>
##   </document>
##   <document>
##     <rank>538</rank>
##     <title><![CDATA[Full Extraction of 2D Photonic Crystal Assisted <formula formulatype="inline"> <img src="/images/tex/21384.gif" alt=" \hbox {Y}_{3}\hbox {Al}_{5}\hbox {O}_{12}"> </formula>:Ce Ceramic Plate Phosphor for Highly Efficient White LEDs]]></title>
##     <authors><![CDATA[Seong Woong Yoon;  Hoo Keun Park;  Ji Hye Oh;  Young Rag Do]]></authors>
##     <affiliations><![CDATA[Dept. of Chem., Kookmin Univ., Seoul, South Korea]]></affiliations>
##     <controlledterms>
##       <term><![CDATA[ceramics]]></term>
##       <term><![CDATA[cerium]]></term>
##       <term><![CDATA[electronics packaging]]></term>
##       <term><![CDATA[light emitting diodes]]></term>
##       <term><![CDATA[phosphors]]></term>
##       <term><![CDATA[photonic crystals]]></term>
##       <term><![CDATA[yttrium compounds]]></term>
##     </controlledterms>
##     <thesaurusterms>
##       <term><![CDATA[Ceramics]]></term>
##       <term><![CDATA[Light emitting diodes]]></term>
##       <term><![CDATA[Nanostructures]]></term>
##       <term><![CDATA[Phosphors]]></term>
##       <term><![CDATA[Powders]]></term>
##       <term><![CDATA[Resins]]></term>
##       <term><![CDATA[Temperature measurement]]></term>
##     </thesaurusterms>
##     <pubtitle><![CDATA[Photonics Journal, IEEE]]></pubtitle>
##     <punumber><![CDATA[4563994]]></punumber>
##     <pubtype><![CDATA[Journals & Magazines]]></pubtype>
##     <publisher><![CDATA[IEEE]]></publisher>
##     <volume><![CDATA[6]]></volume>
##     <issue><![CDATA[1]]></issue>
##     <py><![CDATA[2014]]></py>
##     <spage><![CDATA[1]]></spage>
##     <epage><![CDATA[10]]></epage>
##     <abstract><![CDATA[In this paper, we report the optical and thermal effects of filling the air gap between a ceramic plate phosphor (CPP) and a cup-type light-emitting diode (LED) in a white phosphor-converted LED (pc-LED) package. The aim of our work was to improve extraction efficiency fully, as well as to enhance heat dissipation; and in this case, the CPP was a two-dimensional (2D) TiO<sub>2</sub> nanobowl photonic crystal layer (PCL)-assisted Y<sub>3</sub>Al<sub>5</sub>O<sub>12</sub>: Ce<sup>3 +</sup>(YAG: Ce). By adding a 2D TiO<sub>2</sub> PCL and eliminating the air gap between the CPP and the LED cup, our package exhibited better luminous efficacy, more uniform optical properties, and higher thermal conductivity than those of a 2D TiO<sub>2</sub> PCL-assisted CPP-capped LED with an air gap and a conventional phosphor powder dispersed in an LED package.]]></abstract>
##     <issn><![CDATA[1943-0655]]></issn>
##     <htmlFlag><![CDATA[1]]></htmlFlag>
##     <arnumber><![CDATA[6725599]]></arnumber>
##     <doi><![CDATA[10.1109/JPHOT.2014.2302813]]></doi>
##     <publicationId><![CDATA[6725599]]></publicationId>
##     <mdurl><![CDATA[http://ieeexplore.ieee.org/xpl/articleDetails.jsp?tp=&arnumber=6725599&contentType=Journals+%26+Magazines]]></mdurl>
##     <pdf><![CDATA[http://ieeexplore.ieee.org/stamp/stamp.jsp?arnumber=6725599]]></pdf>
##   </document>
##   <document>
##     <rank>539</rank>
##     <title><![CDATA[Generalized Boundaries from Multiple Image Interpretations]]></title>
##     <authors><![CDATA[Leordeanu, M.;  Sukthankar, R.;  Sminchisescu, C.]]></authors>
##     <affiliations><![CDATA[Inst. of Math., Bucharest, Romania]]></affiliations>
##     <controlledterms>
##       <term><![CDATA[image recognition]]></term>
##       <term><![CDATA[image representation]]></term>
##       <term><![CDATA[image segmentation]]></term>
##       <term><![CDATA[object detection]]></term>
##       <term><![CDATA[object recognition]]></term>
##     </controlledterms>
##     <thesaurusterms>
##       <term><![CDATA[Computational modeling]]></term>
##       <term><![CDATA[Image color analysis]]></term>
##       <term><![CDATA[Image edge detection]]></term>
##       <term><![CDATA[Image segmentation]]></term>
##       <term><![CDATA[Lead]]></term>
##       <term><![CDATA[Mathematical model]]></term>
##       <term><![CDATA[Optical imaging]]></term>
##     </thesaurusterms>
##     <pubtitle><![CDATA[Pattern Analysis and Machine Intelligence, IEEE Transactions on]]></pubtitle>
##     <punumber><![CDATA[34]]></punumber>
##     <pubtype><![CDATA[Journals & Magazines]]></pubtype>
##     <publisher><![CDATA[IEEE]]></publisher>
##     <volume><![CDATA[36]]></volume>
##     <issue><![CDATA[7]]></issue>
##     <py><![CDATA[2014]]></py>
##     <spage><![CDATA[1312]]></spage>
##     <epage><![CDATA[1324]]></epage>
##     <abstract><![CDATA[Boundary detection is a fundamental computer vision problem that is essential for a variety of tasks, such as contour and region segmentation, symmetry detection and object recognition and categorization. We propose a generalized formulation for boundary detection, with closed-form solution, applicable to the localization of different types of boundaries, such as object edges in natural images and occlusion boundaries from video. Our generalized boundary detection method (Gb) simultaneously combines low-level and mid-level image representations in a single eigenvalue problem and solves for the optimal continuous boundary orientation and strength. The closed-form solution to boundary detection enables our algorithm to achieve state-of-the-art results at a significantly lower computational cost than current methods. We also propose two complementary novel components that can seamlessly be combined with Gb: first, we introduce a soft-segmentation procedure that provides region input layers to our boundary detection algorithm for a significant improvement in accuracy, at negligible computational cost; second, we present an efficient method for contour grouping and reasoning, which when applied as a final post-processing stage, further increases the boundary detection performance.]]></abstract>
##     <issn><![CDATA[0162-8828]]></issn>
##     <htmlFlag><![CDATA[1]]></htmlFlag>
##     <arnumber><![CDATA[6702414]]></arnumber>
##     <doi><![CDATA[10.1109/TPAMI.2014.17]]></doi>
##     <publicationId><![CDATA[6702414]]></publicationId>
##     <mdurl><![CDATA[http://ieeexplore.ieee.org/xpl/articleDetails.jsp?tp=&arnumber=6702414&contentType=Journals+%26+Magazines]]></mdurl>
##     <pdf><![CDATA[http://ieeexplore.ieee.org/stamp/stamp.jsp?arnumber=6702414]]></pdf>
##   </document>
##   <document>
##     <rank>540</rank>
##     <title><![CDATA[Evaluation of Precipitation Estimates by at-Launch Codes of GPM/DPR Algorithms Using Synthetic Data from TRMM/PR Observations]]></title>
##     <authors><![CDATA[Kubota, T.;  Yoshida, N.;  Urita, S.;  Iguchi, T.;  Seto, S.;  Meneghini, R.;  Awaka, J.;  Hanado, H.;  Kida, S.;  Oki, R.]]></authors>
##     <affiliations><![CDATA[Earth Obs. Res. Center, Japan Aerosp. Exploration Agency, Tsukuba, Japan]]></affiliations>
##     <controlledterms>
##       <term><![CDATA[atmospheric precipitation]]></term>
##       <term><![CDATA[remote sensing by radar]]></term>
##     </controlledterms>
##     <thesaurusterms>
##       <term><![CDATA[Attenuation]]></term>
##       <term><![CDATA[Extraterrestrial measurements]]></term>
##       <term><![CDATA[Geometry]]></term>
##       <term><![CDATA[Radar measurements]]></term>
##       <term><![CDATA[Sea surface]]></term>
##       <term><![CDATA[Spaceborne radar]]></term>
##     </thesaurusterms>
##     <pubtitle><![CDATA[Selected Topics in Applied Earth Observations and Remote Sensing, IEEE Journal of]]></pubtitle>
##     <punumber><![CDATA[4609443]]></punumber>
##     <pubtype><![CDATA[Journals & Magazines]]></pubtype>
##     <publisher><![CDATA[IEEE]]></publisher>
##     <volume><![CDATA[7]]></volume>
##     <issue><![CDATA[9]]></issue>
##     <py><![CDATA[2014]]></py>
##     <spage><![CDATA[3931]]></spage>
##     <epage><![CDATA[3944]]></epage>
##     <abstract><![CDATA[The Global Precipitation Measurement (GPM) Core Observatory will carry a Dual-frequency Precipitation Radar (DPR) consisting of a Ku-band precipitation radar (KuPR) and a Ka-band precipitation radar (KaPR). In this study, &#x201C;at-launch&#x201D; codes of DPR precipitation algorithms, which will be used in GPM ground systems at launch, were evaluated using synthetic data based upon the Tropical Rainfall Measuring Mission (TRMM) Precipitation Radar (PR) data. Results from the codes (Version 4.20131010) of the KuPR-only, KaPR-only, and DPR algorithms were compared with &#x201C;true values&#x201D; calculated based upon drop size distributions assumed in the synthetic data and standard results from the TRMM algorithms at an altitude of 2 km over the ocean. The results indicate that the total precipitation amounts during April 2011 from the KuPR and DPR algorithms are similar to the true values, whereas the estimates from the KaPR data are underestimated. Moreover, the DPR estimates yielded smaller precipitation rates for rates less than about 10 mm/h and greater precipitation rates above 10 mm/h. Underestimation of the KaPR estimates was analyzed in terms of measured radar reflectivity (Z<sub>m</sub>) of the KaPR at an altitude of 2 km. The underestimation of the KaPR data was most pronounced during strong precipitation events of Z<sub>m</sub> &lt;; 18 dBZ (high attenuation cases) over heavy precipitation areas in the Tropics, whereas the underestimation was less pronounced when the Z<sub>m</sub> &gt; 26 (moderate attenuation cases). The results suggest that the underestimation is caused by a problem in the attenuation correction method, which was verified by the improved codes.]]></abstract>
##     <issn><![CDATA[1939-1404]]></issn>
##     <htmlFlag><![CDATA[1]]></htmlFlag>
##     <arnumber><![CDATA[6820746]]></arnumber>
##     <doi><![CDATA[10.1109/JSTARS.2014.2320960]]></doi>
##     <publicationId><![CDATA[6820746]]></publicationId>
##     <mdurl><![CDATA[http://ieeexplore.ieee.org/xpl/articleDetails.jsp?tp=&arnumber=6820746&contentType=Journals+%26+Magazines]]></mdurl>
##     <pdf><![CDATA[http://ieeexplore.ieee.org/stamp/stamp.jsp?arnumber=6820746]]></pdf>
##   </document>
##   <document>
##     <rank>541</rank>
##     <title><![CDATA[A Bayesian Algorithm for Anxiety Index Prediction Based on Cerebral Blood Oxygenation in the Prefrontal Cortex Measured by Near Infrared Spectroscopy]]></title>
##     <authors><![CDATA[Fukuda, Y.;  Ida, Y.;  Matsumoto, T.;  Takemura, N.;  Sakatani, K.]]></authors>
##     <affiliations><![CDATA[Dept. of Electr. Eng. & Biosci., Waseda Univ., Tokyo, Japan]]></affiliations>
##     <controlledterms>
##       <term><![CDATA[Bayes methods]]></term>
##       <term><![CDATA[Markov processes]]></term>
##       <term><![CDATA[Monte Carlo methods]]></term>
##       <term><![CDATA[bio-optics]]></term>
##       <term><![CDATA[biochemistry]]></term>
##       <term><![CDATA[blood]]></term>
##       <term><![CDATA[brain]]></term>
##       <term><![CDATA[diseases]]></term>
##       <term><![CDATA[geriatrics]]></term>
##       <term><![CDATA[infrared spectroscopy]]></term>
##       <term><![CDATA[learning (artificial intelligence)]]></term>
##       <term><![CDATA[medical signal processing]]></term>
##       <term><![CDATA[molecular biophysics]]></term>
##       <term><![CDATA[neurophysiology]]></term>
##       <term><![CDATA[proteins]]></term>
##     </controlledterms>
##     <thesaurusterms>
##       <term><![CDATA[Bayes methods]]></term>
##       <term><![CDATA[Educational institutions]]></term>
##       <term><![CDATA[Feature extraction]]></term>
##       <term><![CDATA[Indexes]]></term>
##       <term><![CDATA[Machine learning algorithms]]></term>
##       <term><![CDATA[Prediction algorithms]]></term>
##       <term><![CDATA[Stress]]></term>
##     </thesaurusterms>
##     <pubtitle><![CDATA[Translational Engineering in Health and Medicine, IEEE Journal of]]></pubtitle>
##     <punumber><![CDATA[6221039]]></punumber>
##     <pubtype><![CDATA[Journals & Magazines]]></pubtype>
##     <publisher><![CDATA[IEEE]]></publisher>
##     <volume><![CDATA[2]]></volume>
##     <py><![CDATA[2014]]></py>
##     <spage><![CDATA[1]]></spage>
##     <epage><![CDATA[10]]></epage>
##     <abstract><![CDATA[Stress-induced psychological and somatic diseases are virtually endemic nowadays. Written self-report anxiety measures are available; however, these indices tend to be time consuming to acquire. For medical patients, completing written reports can be burdensome if they are weak, in pain, or in acute anxiety states. Consequently, simple and fast non-invasive methods for assessing stress response from neurophysiological data are essential. In this paper, we report on a study that makes predictions of the state-trait anxiety inventory (STAI) index from oxyhemoglobin and deoxyhemoglobin concentration changes of the prefrontal cortex using a two-channel portable near-infrared spectroscopy device. Predictions are achieved by constructing machine learning algorithms within a Bayesian framework with nonlinear basis function together with Markov Chain Monte Carlo implementation. In this paper, prediction experiments were performed against four different data sets, i.e., two comprising young subjects, and the remaining two comprising elderly subjects. The number of subjects in each data set varied between 17 and 20 and each subject participated only once. They were not asked to perform any task; instead, they were at rest. The root mean square errors for the four groups were 6.20, 6.62, 4.50, and 6.38, respectively. There appeared to be no significant distinctions of prediction accuracies between age groups and since the STAI are defined between 20 and 80, the predictions appeared reasonably accurate. The results indicate potential applications to practical situations such as stress management and medical practice.]]></abstract>
##     <issn><![CDATA[2168-2372]]></issn>
##     <htmlFlag><![CDATA[1]]></htmlFlag>
##     <arnumber><![CDATA[6917199]]></arnumber>
##     <doi><![CDATA[10.1109/JTEHM.2014.2361757]]></doi>
##     <publicationId><![CDATA[6917199]]></publicationId>
##     <mdurl><![CDATA[http://ieeexplore.ieee.org/xpl/articleDetails.jsp?tp=&arnumber=6917199&contentType=Journals+%26+Magazines]]></mdurl>
##     <pdf><![CDATA[http://ieeexplore.ieee.org/stamp/stamp.jsp?arnumber=6917199]]></pdf>
##   </document>
##   <document>
##     <rank>542</rank>
##     <title><![CDATA[Receiver-Based Recovery of Clipped OFDM Signals for PAPR Reduction: A Bayesian Approach]]></title>
##     <authors><![CDATA[Ali, A.;  Al-Rabah, A.;  Masood, M.;  Al-Naffouri, T.Y.]]></authors>
##     <affiliations><![CDATA[Dept. of Electr. Eng., King Abdullah Univ. of Sci. & Technol., Thuwal, Saudi Arabia]]></affiliations>
##     <controlledterms>
##       <term><![CDATA[Bayes methods]]></term>
##       <term><![CDATA[OFDM modulation]]></term>
##       <term><![CDATA[antenna arrays]]></term>
##       <term><![CDATA[channel estimation]]></term>
##       <term><![CDATA[multiuser channels]]></term>
##       <term><![CDATA[radio receivers]]></term>
##       <term><![CDATA[receiving antennas]]></term>
##       <term><![CDATA[signal restoration]]></term>
##     </controlledterms>
##     <thesaurusterms>
##       <term><![CDATA[Bayes methods]]></term>
##       <term><![CDATA[OFDM]]></term>
##       <term><![CDATA[Peak to average power ratio]]></term>
##       <term><![CDATA[Receivers]]></term>
##       <term><![CDATA[Time-domain analysis]]></term>
##     </thesaurusterms>
##     <pubtitle><![CDATA[Access, IEEE]]></pubtitle>
##     <punumber><![CDATA[6287639]]></punumber>
##     <pubtype><![CDATA[Journals & Magazines]]></pubtype>
##     <publisher><![CDATA[IEEE]]></publisher>
##     <volume><![CDATA[2]]></volume>
##     <py><![CDATA[2014]]></py>
##     <spage><![CDATA[1213]]></spage>
##     <epage><![CDATA[1224]]></epage>
##     <abstract><![CDATA[Clipping is one of the simplest peak-to-average power ratio reduction schemes for orthogonal frequency division multiplexing (OFDM). Deliberately clipping the transmission signal degrades system performance, and clipping mitigation is required at the receiver for information restoration. In this paper, we acknowledge the sparse nature of the clipping signal and propose a low-complexity Bayesian clipping estimation scheme. The proposed scheme utilizes a priori information about the sparsity rate and noise variance for enhanced recovery. At the same time, the proposed scheme is robust against inaccurate estimates of the clipping signal statistics. The undistorted phase property of the clipped signal, as well as the clipping likelihood, is utilized for enhanced reconstruction. Furthermore, motivated by the nature of modern OFDM-based communication systems, we extend our clipping reconstruction approach to multiple antenna receivers and multi-user OFDM.We also address the problem of channel estimation from pilots contaminated by the clipping distortion. Numerical findings are presented that depict favorable results for the proposed scheme compared to the established sparse reconstruction schemes.]]></abstract>
##     <issn><![CDATA[2169-3536]]></issn>
##     <htmlFlag><![CDATA[1]]></htmlFlag>
##     <arnumber><![CDATA[6919993]]></arnumber>
##     <doi><![CDATA[10.1109/ACCESS.2014.2362772]]></doi>
##     <publicationId><![CDATA[6919993]]></publicationId>
##     <mdurl><![CDATA[http://ieeexplore.ieee.org/xpl/articleDetails.jsp?tp=&arnumber=6919993&contentType=Journals+%26+Magazines]]></mdurl>
##     <pdf><![CDATA[http://ieeexplore.ieee.org/stamp/stamp.jsp?arnumber=6919993]]></pdf>
##   </document>
##   <document>
##     <rank>543</rank>
##     <title><![CDATA[Development of the ALMA Band-3 and Band-6 Sideband-Separating SIS Mixers]]></title>
##     <authors><![CDATA[Kerr, A.R.;  Shing-Kuo Pan;  Claude, S.M.X.;  Dindo, P.;  Lichtenberger, A.W.;  Effland, J.E.;  Lauria, E.F.]]></authors>
##     <affiliations><![CDATA[Nat. Radio Astron. Obs., Charlottesville, VA, USA]]></affiliations>
##     <controlledterms>
##       <term><![CDATA[aluminium]]></term>
##       <term><![CDATA[aluminium compounds]]></term>
##       <term><![CDATA[interference suppression]]></term>
##       <term><![CDATA[millimetre wave mixers]]></term>
##       <term><![CDATA[niobium]]></term>
##       <term><![CDATA[submillimetre wave mixers]]></term>
##       <term><![CDATA[superconducting microwave devices]]></term>
##       <term><![CDATA[superconductor-insulator-superconductor mixers]]></term>
##       <term><![CDATA[tunnelling]]></term>
##     </controlledterms>
##     <thesaurusterms>
##       <term><![CDATA[Amplitude modulation]]></term>
##       <term><![CDATA[Arrays]]></term>
##       <term><![CDATA[Impedance]]></term>
##       <term><![CDATA[Junctions]]></term>
##       <term><![CDATA[Mixers]]></term>
##       <term><![CDATA[Radio frequency]]></term>
##       <term><![CDATA[Receivers]]></term>
##     </thesaurusterms>
##     <pubtitle><![CDATA[Terahertz Science and Technology, IEEE Transactions on]]></pubtitle>
##     <punumber><![CDATA[5503871]]></punumber>
##     <pubtype><![CDATA[Journals & Magazines]]></pubtype>
##     <publisher><![CDATA[IEEE]]></publisher>
##     <volume><![CDATA[4]]></volume>
##     <issue><![CDATA[2]]></issue>
##     <py><![CDATA[2014]]></py>
##     <spage><![CDATA[201]]></spage>
##     <epage><![CDATA[212]]></epage>
##     <abstract><![CDATA[As the Atacama Large Millimeter/submillimeter Array (ALMA) nears completion, 73 dual-polarization receivers have been delivered for each of Bands 3 (84-116 GHz) and 6 (211-275 GHz). The receivers use sideband-separating superconducting Nb/Al-AlOx/Nb tunnel-junction (SIS) mixers, developed for ALMA to suppress atmospheric noise in the image band. The mixers were designed taking into account dynamic range, input return loss, and signal-to-image conversion (which can be significant in SIS mixers). Typical SSB receiver noise temperatures in Bands 3 and 6 are 30 and 60 K, respectively, and the image rejection is typically 15 dB.]]></abstract>
##     <issn><![CDATA[2156-342X]]></issn>
##     <htmlFlag><![CDATA[1]]></htmlFlag>
##     <arnumber><![CDATA[6740089]]></arnumber>
##     <doi><![CDATA[10.1109/TTHZ.2014.2302537]]></doi>
##     <publicationId><![CDATA[6740089]]></publicationId>
##     <mdurl><![CDATA[http://ieeexplore.ieee.org/xpl/articleDetails.jsp?tp=&arnumber=6740089&contentType=Journals+%26+Magazines]]></mdurl>
##     <pdf><![CDATA[http://ieeexplore.ieee.org/stamp/stamp.jsp?arnumber=6740089]]></pdf>
##   </document>
##   <document>
##     <rank>544</rank>
##     <title><![CDATA[A New Mass Production Technology for High-Efficiency Thin-Film CIS-Absorber Formation]]></title>
##     <authors><![CDATA[Probst, V.;  Koetschau, I.;  Novak, E.;  Jasenek, A.;  Eschrich, H.;  Hergert, F.;  Hahn, T.;  Feichtinger, J.;  Maier, M.;  Walther, B.;  Nadenau, V.]]></authors>
##     <affiliations><![CDATA[R&D, Bosch Solar CISTech GmbH, Brandenburg, Germany]]></affiliations>
##     <controlledterms>
##       <term><![CDATA[batch processing (industrial)]]></term>
##       <term><![CDATA[cooling]]></term>
##       <term><![CDATA[corrosion]]></term>
##       <term><![CDATA[durability]]></term>
##       <term><![CDATA[heating]]></term>
##       <term><![CDATA[mass production]]></term>
##       <term><![CDATA[selenium]]></term>
##       <term><![CDATA[solar cells]]></term>
##       <term><![CDATA[solar energy concentrators]]></term>
##       <term><![CDATA[sulphur]]></term>
##       <term><![CDATA[thin film devices]]></term>
##     </controlledterms>
##     <thesaurusterms>
##       <term><![CDATA[Gallium]]></term>
##       <term><![CDATA[Heating]]></term>
##       <term><![CDATA[Mass production]]></term>
##       <term><![CDATA[Substrates]]></term>
##       <term><![CDATA[Surface morphology]]></term>
##       <term><![CDATA[X-ray diffraction]]></term>
##     </thesaurusterms>
##     <pubtitle><![CDATA[Photovoltaics, IEEE Journal of]]></pubtitle>
##     <punumber><![CDATA[5503869]]></punumber>
##     <pubtype><![CDATA[Journals & Magazines]]></pubtype>
##     <publisher><![CDATA[IEEE]]></publisher>
##     <volume><![CDATA[4]]></volume>
##     <issue><![CDATA[2]]></issue>
##     <py><![CDATA[2014]]></py>
##     <spage><![CDATA[687]]></spage>
##     <epage><![CDATA[692]]></epage>
##     <abstract><![CDATA[A new mass production technology for CIS-absorber formation yielding high-average module efficiencies is introduced. A novel custom-designed oven very successfully exploits the principle of forced convection during heating, CIS formation reaction, and cooling. Cu(In,Ga)(Se,S) 2 absorbers are formed by metal precursor deposition on soda lime glass followed by reaction in selenium/sulfur atmosphere. Processing is performed in a multiple-chamber equipment which handles corrosive, flammable, and toxic process gases from atmospheric pressure to vacuum at high durability. The substrates (size: 50 cm &#x00D7; 120 cm) are processed in batches up to 102 substrates, applying forced convection for very homogenous heat transfer and high heating and cooling rates. Multiple-chamber design and batch size yield high throughput at cycle times above 1 h. This approach combines the specific advantages of batch type and inline processing. An excellent average efficiency of 14.3% with a narrow distribution (+/-0.31%) and a peak efficiency of 15.1% is shown with this technology. Module characteristic distributions during pilot production are presented. Detailed layer analytics is discussed. This straightforward reliable mass production technology is a key for highest module performance and for upscaling. Module efficiencies of 17% can be reached, enabling production costs below 0.38 US$/Wp in a projected GWp plant.]]></abstract>
##     <issn><![CDATA[2156-3381]]></issn>
##     <htmlFlag><![CDATA[1]]></htmlFlag>
##     <arnumber><![CDATA[6737216]]></arnumber>
##     <doi><![CDATA[10.1109/JPHOTOV.2014.2302235]]></doi>
##     <publicationId><![CDATA[6737216]]></publicationId>
##     <mdurl><![CDATA[http://ieeexplore.ieee.org/xpl/articleDetails.jsp?tp=&arnumber=6737216&contentType=Journals+%26+Magazines]]></mdurl>
##     <pdf><![CDATA[http://ieeexplore.ieee.org/stamp/stamp.jsp?arnumber=6737216]]></pdf>
##   </document>
##   <document>
##     <rank>545</rank>
##     <title><![CDATA[Constructive Models of Discrete and Continuous Physical Phenomena]]></title>
##     <authors><![CDATA[Lee, E.A.]]></authors>
##     <affiliations><![CDATA[Department of Electrical and Engineering Computer Sciences, University of California at Berkeley, Berkeley, CA, USA]]></affiliations>
##     <thesaurusterms>
##       <term><![CDATA[Collision avoidance]]></term>
##       <term><![CDATA[Computational modeling]]></term>
##       <term><![CDATA[Discrete-time systems]]></term>
##       <term><![CDATA[Electronic circuits]]></term>
##       <term><![CDATA[Integrated circuit modeling]]></term>
##       <term><![CDATA[Semantics]]></term>
##       <term><![CDATA[Switching circuits]]></term>
##     </thesaurusterms>
##     <pubtitle><![CDATA[Access, IEEE]]></pubtitle>
##     <punumber><![CDATA[6287639]]></punumber>
##     <pubtype><![CDATA[Journals & Magazines]]></pubtype>
##     <publisher><![CDATA[IEEE]]></publisher>
##     <volume><![CDATA[2]]></volume>
##     <py><![CDATA[2014]]></py>
##     <spage><![CDATA[797]]></spage>
##     <epage><![CDATA[821]]></epage>
##     <abstract><![CDATA[This paper studies the semantics of models for discrete physical phenomena, such as rigid body collisions and switching in electronic circuits. This paper combines generalized functions (specifically the Dirac delta function), superdense time, modal models, and constructive semantics to get a rich, flexible, efficient, and rigorous approach to modeling such systems. It shows that many physical scenarios that have been problematic for modeling techniques manifest as nonconstructive models, and that constructive versions of some of the models properly reflect uncertainty in the behavior of the physical systems that plausibly arise from the principles of the underlying physics. This paper argues that these modeling difficulties are not reasonably solved by more detailed continuous models of the underlying physical phenomena. Such more detailed models simply shift the uncertainty to other aspects of the model. Since such detailed models come with a high computational cost, there is little justification in using them unless the goal of modeling is specifically to understand these more detailed physical processes. All models in this paper are implemented in the Ptolemy II modeling and simulation environment and made available online.]]></abstract>
##     <issn><![CDATA[2169-3536]]></issn>
##     <htmlFlag><![CDATA[1]]></htmlFlag>
##     <arnumber><![CDATA[6873221]]></arnumber>
##     <doi><![CDATA[10.1109/ACCESS.2014.2345759]]></doi>
##     <publicationId><![CDATA[6873221]]></publicationId>
##     <mdurl><![CDATA[http://ieeexplore.ieee.org/xpl/articleDetails.jsp?tp=&arnumber=6873221&contentType=Journals+%26+Magazines]]></mdurl>
##     <pdf><![CDATA[http://ieeexplore.ieee.org/stamp/stamp.jsp?arnumber=6873221]]></pdf>
##   </document>
##   <document>
##     <rank>546</rank>
##     <title><![CDATA[Damping Microvibration Measurement Using Laser Diode Self-Mixing Interference]]></title>
##     <authors><![CDATA[Minliang Chen;  Yongbing Zhang;  Chenxi Chen;  Lu Wang;  Wencai Huang]]></authors>
##     <affiliations><![CDATA[Dept. of Electron. Eng., Xiamen Univ., Xiamen, China]]></affiliations>
##     <controlledterms>
##       <term><![CDATA[damping]]></term>
##       <term><![CDATA[light interference]]></term>
##       <term><![CDATA[measurement by laser beam]]></term>
##       <term><![CDATA[semiconductor lasers]]></term>
##       <term><![CDATA[vibration measurement]]></term>
##     </controlledterms>
##     <thesaurusterms>
##       <term><![CDATA[Damping]]></term>
##       <term><![CDATA[Equations]]></term>
##       <term><![CDATA[Interference]]></term>
##       <term><![CDATA[Mathematical model]]></term>
##       <term><![CDATA[Optical feedback]]></term>
##       <term><![CDATA[Optical interferometry]]></term>
##       <term><![CDATA[Vibrations]]></term>
##     </thesaurusterms>
##     <pubtitle><![CDATA[Photonics Journal, IEEE]]></pubtitle>
##     <punumber><![CDATA[4563994]]></punumber>
##     <pubtype><![CDATA[Journals & Magazines]]></pubtype>
##     <publisher><![CDATA[IEEE]]></publisher>
##     <volume><![CDATA[6]]></volume>
##     <issue><![CDATA[3]]></issue>
##     <py><![CDATA[2014]]></py>
##     <spage><![CDATA[1]]></spage>
##     <epage><![CDATA[8]]></epage>
##     <abstract><![CDATA[Based on the study of laser diode self-mixing interference effects, a simple damping microvibration measuring method that can accurately obtain the damping factor is presented. The damping factor is solved by recording the period and counting the fringe number of the self-mixing signal. The damping factor of 0.0483 s<sup>-1</sup> with a standard deviation of 0.0013 and the coefficient of variation of 2.69% was experimentally achieved. Theoretical simulation of the feedback strength on the measuring accuracy shows that the measuring method of the damping factor has high accuracy in the case of weak feedback.]]></abstract>
##     <issn><![CDATA[1943-0655]]></issn>
##     <htmlFlag><![CDATA[1]]></htmlFlag>
##     <arnumber><![CDATA[6814844]]></arnumber>
##     <doi><![CDATA[10.1109/JPHOT.2014.2323314]]></doi>
##     <publicationId><![CDATA[6814844]]></publicationId>
##     <mdurl><![CDATA[http://ieeexplore.ieee.org/xpl/articleDetails.jsp?tp=&arnumber=6814844&contentType=Journals+%26+Magazines]]></mdurl>
##     <pdf><![CDATA[http://ieeexplore.ieee.org/stamp/stamp.jsp?arnumber=6814844]]></pdf>
##   </document>
##   <document>
##     <rank>547</rank>
##     <title><![CDATA[Evolutionary Scheduling of Dynamic Multitasking Workloads for Big-Data Analytics in Elastic Cloud]]></title>
##     <authors><![CDATA[Fan Zhang;  Junwei Cao;  Wei Tan;  Khan, S.U.;  Keqin Li;  Zomaya, A.Y.]]></authors>
##     <affiliations><![CDATA[Kavli Inst. for Astrophys. & Space Res., Massachusetts Inst. of Technol., Cambridge, MA, USA]]></affiliations>
##     <controlledterms>
##       <term><![CDATA[Big Data]]></term>
##       <term><![CDATA[Monte Carlo methods]]></term>
##       <term><![CDATA[cloud computing]]></term>
##       <term><![CDATA[evolutionary computation]]></term>
##       <term><![CDATA[iterative methods]]></term>
##       <term><![CDATA[optimisation]]></term>
##       <term><![CDATA[scheduling]]></term>
##     </controlledterms>
##     <thesaurusterms>
##       <term><![CDATA[Dynamic scheduling]]></term>
##       <term><![CDATA[Monte Carlo methods]]></term>
##       <term><![CDATA[Optimization]]></term>
##       <term><![CDATA[Processor scheduling]]></term>
##       <term><![CDATA[Schedules]]></term>
##       <term><![CDATA[Throughput]]></term>
##     </thesaurusterms>
##     <pubtitle><![CDATA[Emerging Topics in Computing, IEEE Transactions on]]></pubtitle>
##     <punumber><![CDATA[6245516]]></punumber>
##     <pubtype><![CDATA[Journals & Magazines]]></pubtype>
##     <publisher><![CDATA[IEEE]]></publisher>
##     <volume><![CDATA[2]]></volume>
##     <issue><![CDATA[3]]></issue>
##     <py><![CDATA[2014]]></py>
##     <spage><![CDATA[338]]></spage>
##     <epage><![CDATA[351]]></epage>
##     <abstract><![CDATA[Scheduling of dynamic and multitasking workloads for big-data analytics is a challenging issue, as it requires a significant amount of parameter sweeping and iterations. Therefore, real-time scheduling becomes essential to increase the throughput of many-task computing. The difficulty lies in obtaining a series of optimal yet responsive schedules. In dynamic scenarios, such as virtual clusters in cloud, scheduling must be processed fast enough to keep pace with the unpredictable fluctuations in the workloads to optimize the overall system performance. In this paper, ordinal optimization using rough models and fast simulation is introduced to obtain suboptimal solutions in a much shorter timeframe. While the scheduling solution for each period may not be the best, ordinal optimization can be processed fast in an iterative and evolutionary way to capture the details of big-data workload dynamism. Experimental results show that our evolutionary approach compared with existing methods, such as Monte Carlo and Blind Pick, can achieve higher overall average scheduling performance, such as throughput, in real-world applications with dynamic workloads. Furthermore, performance improvement is seen by implementing an optimal computing budget allocating method that smartly allocates computing cycles to the most promising schedules.]]></abstract>
##     <issn><![CDATA[2168-6750]]></issn>
##     <htmlFlag><![CDATA[1]]></htmlFlag>
##     <arnumber><![CDATA[6879452]]></arnumber>
##     <doi><![CDATA[10.1109/TETC.2014.2348196]]></doi>
##     <publicationId><![CDATA[6879452]]></publicationId>
##     <mdurl><![CDATA[http://ieeexplore.ieee.org/xpl/articleDetails.jsp?tp=&arnumber=6879452&contentType=Journals+%26+Magazines]]></mdurl>
##     <pdf><![CDATA[http://ieeexplore.ieee.org/stamp/stamp.jsp?arnumber=6879452]]></pdf>
##   </document>
##   <document>
##     <rank>548</rank>
##     <title><![CDATA[GaN Schottky Barrier Diode With TiN Electrode for Microwave Rectification]]></title>
##     <authors><![CDATA[Liuan Li;  Kishi, A.;  Qiang Liu;  Itai, Y.;  Fujihara, R.;  Ohno, Y.;  Jin-Ping Ao]]></authors>
##     <affiliations><![CDATA[Inst. of Technol. & Sci., Univ. of Tokushima, Tokushima, Japan]]></affiliations>
##     <controlledterms>
##       <term><![CDATA[III-V semiconductors]]></term>
##       <term><![CDATA[Schottky diodes]]></term>
##       <term><![CDATA[electrodes]]></term>
##       <term><![CDATA[gallium compounds]]></term>
##       <term><![CDATA[leakage currents]]></term>
##       <term><![CDATA[microwave diodes]]></term>
##       <term><![CDATA[rectification]]></term>
##       <term><![CDATA[semiconductor device breakdown]]></term>
##       <term><![CDATA[titanium compounds]]></term>
##       <term><![CDATA[wide band gap semiconductors]]></term>
##     </controlledterms>
##     <thesaurusterms>
##       <term><![CDATA[Electrodes]]></term>
##       <term><![CDATA[Gallium nitride]]></term>
##       <term><![CDATA[Rectennas]]></term>
##       <term><![CDATA[Schottky barriers]]></term>
##       <term><![CDATA[Schottky diodes]]></term>
##       <term><![CDATA[Titanium compounds]]></term>
##     </thesaurusterms>
##     <pubtitle><![CDATA[Electron Devices Society, IEEE Journal of the]]></pubtitle>
##     <punumber><![CDATA[6245494]]></punumber>
##     <pubtype><![CDATA[Journals & Magazines]]></pubtype>
##     <publisher><![CDATA[IEEE]]></publisher>
##     <volume><![CDATA[2]]></volume>
##     <issue><![CDATA[6]]></issue>
##     <py><![CDATA[2014]]></py>
##     <spage><![CDATA[168]]></spage>
##     <epage><![CDATA[173]]></epage>
##     <abstract><![CDATA[GaN Schottky barrier diodes (SBDs) with low turn-on voltage are developed for microwave rectification. The diodes with reactively-sputtered TiN electrodes have a lower turn-on voltage compared with the diodes with Ni electrode, while the on-resistance, the reverse leakage current, and the reverse breakdown characteristics are comparable to each other. Theoretically, the SBDs with TiN electrodes can enhance the efficiency of a rectenna circuit at 2.45 GHz from 84% to 89% when the turn-on voltage decreases from 1.0 to 0.5 V.]]></abstract>
##     <issn><![CDATA[2168-6734]]></issn>
##     <htmlFlag><![CDATA[1]]></htmlFlag>
##     <arnumber><![CDATA[6874492]]></arnumber>
##     <doi><![CDATA[10.1109/JEDS.2014.2346395]]></doi>
##     <publicationId><![CDATA[6874492]]></publicationId>
##     <mdurl><![CDATA[http://ieeexplore.ieee.org/xpl/articleDetails.jsp?tp=&arnumber=6874492&contentType=Journals+%26+Magazines]]></mdurl>
##     <pdf><![CDATA[http://ieeexplore.ieee.org/stamp/stamp.jsp?arnumber=6874492]]></pdf>
##   </document>
##   <document>
##     <rank>549</rank>
##     <title><![CDATA[Pre to Intraoperative Data Fusion Framework for Multimodal Characterization of Myocardial Scar Tissue]]></title>
##     <authors><![CDATA[Porras, A.R.;  Piella, G.;  Berruezo, A.;  Fernandez-Armenta, J.;  Frangi, A.F.]]></authors>
##     <affiliations><![CDATA[Dept. of Inf. & Commun. Technol., Univ. Pompeu Fabra, Barcelona, Spain]]></affiliations>
##     <controlledterms>
##       <term><![CDATA[bioelectric phenomena]]></term>
##       <term><![CDATA[biological tissues]]></term>
##       <term><![CDATA[biomechanics]]></term>
##       <term><![CDATA[biomedical MRI]]></term>
##       <term><![CDATA[cardiology]]></term>
##       <term><![CDATA[catheters]]></term>
##       <term><![CDATA[deformation]]></term>
##       <term><![CDATA[image fusion]]></term>
##       <term><![CDATA[image segmentation]]></term>
##       <term><![CDATA[medical image processing]]></term>
##       <term><![CDATA[motion estimation]]></term>
##       <term><![CDATA[patient treatment]]></term>
##     </controlledterms>
##     <thesaurusterms>
##       <term><![CDATA[Biomedical imaging]]></term>
##       <term><![CDATA[Catheters]]></term>
##       <term><![CDATA[Heart]]></term>
##       <term><![CDATA[Image segmentation]]></term>
##       <term><![CDATA[Magnetic resonance imaging]]></term>
##       <term><![CDATA[Myocardium]]></term>
##       <term><![CDATA[Three-dimensional displays]]></term>
##     </thesaurusterms>
##     <pubtitle><![CDATA[Translational Engineering in Health and Medicine, IEEE Journal of]]></pubtitle>
##     <punumber><![CDATA[6221039]]></punumber>
##     <pubtype><![CDATA[Journals & Magazines]]></pubtype>
##     <publisher><![CDATA[IEEE]]></publisher>
##     <volume><![CDATA[2]]></volume>
##     <py><![CDATA[2014]]></py>
##     <spage><![CDATA[1]]></spage>
##     <epage><![CDATA[11]]></epage>
##     <abstract><![CDATA[Merging multimodal information about myocardial scar tissue can help electrophysiologists to find the most appropriate target during catheter ablation of ventricular arrhythmias. A framework is presented to analyze and combine information from delayed enhancement magnetic resonance imaging (DE-MRI) and electro-anatomical mapping data. Using this information, electrical, mechanical, and image-based characterization of the myocardium are performed. The presented framework allows the left ventricle to be segmented by DE-MRI and the scar to be characterized prior to the intervention based on image information. It allows the electro-anatomical maps obtained during the intervention from a navigation system to be merged together with the anatomy and scar information extracted from DE-MRI. It also allows for the estimation of endocardial motion and deformation to assess cardiac mechanics. Therefore, electrical, mechanical, and image-based characterization of the myocardium can be performed. The feasibility of this approach was demonstrated on three patients with ventricular tachycardia associated to ischemic cardiomyopathy by integrating images from DE-MRI and electro-anatomical maps data in a common framework for intraoperative myocardial tissue characterization. The proposed framework has the potential to guide and monitor delivery of radio frequency ablation of ventricular tachycardia. It is also helpful for research purposes, facilitating the study of the relationship between electrical and mechanical properties of the tissue, as well as with tissue viability from DE-MRI.]]></abstract>
##     <issn><![CDATA[2168-2372]]></issn>
##     <htmlFlag><![CDATA[1]]></htmlFlag>
##     <arnumber><![CDATA[6891115]]></arnumber>
##     <doi><![CDATA[10.1109/JTEHM.2014.2354332]]></doi>
##     <publicationId><![CDATA[6891115]]></publicationId>
##     <mdurl><![CDATA[http://ieeexplore.ieee.org/xpl/articleDetails.jsp?tp=&arnumber=6891115&contentType=Journals+%26+Magazines]]></mdurl>
##     <pdf><![CDATA[http://ieeexplore.ieee.org/stamp/stamp.jsp?arnumber=6891115]]></pdf>
##   </document>
##   <document>
##     <rank>550</rank>
##     <title><![CDATA[On the Device Design and Drive-Current Capability of SOI Lateral Bipolar Transistors]]></title>
##     <authors><![CDATA[Jin Cai;  Ning, T.H.;  D'Emic, C.P.;  Jeng-Bang Yau;  Chan, K.K.;  Joonah Yoon;  Muralidhar, R.;  Dae-Gyu Park]]></authors>
##     <affiliations><![CDATA[IBM Thomas J. Watson Res. Center, Yorktown Heights, NY, USA]]></affiliations>
##     <controlledterms>
##       <term><![CDATA[bipolar transistors]]></term>
##       <term><![CDATA[carrier density]]></term>
##       <term><![CDATA[semiconductor doping]]></term>
##       <term><![CDATA[silicon-on-insulator]]></term>
##     </controlledterms>
##     <thesaurusterms>
##       <term><![CDATA[Bipolar transistors]]></term>
##       <term><![CDATA[Doping]]></term>
##       <term><![CDATA[Equations]]></term>
##       <term><![CDATA[Implants]]></term>
##       <term><![CDATA[Integrated circuits]]></term>
##       <term><![CDATA[Niobium]]></term>
##       <term><![CDATA[Transistors]]></term>
##     </thesaurusterms>
##     <pubtitle><![CDATA[Electron Devices Society, IEEE Journal of the]]></pubtitle>
##     <punumber><![CDATA[6245494]]></punumber>
##     <pubtype><![CDATA[Journals & Magazines]]></pubtype>
##     <publisher><![CDATA[IEEE]]></publisher>
##     <volume><![CDATA[2]]></volume>
##     <issue><![CDATA[5]]></issue>
##     <py><![CDATA[2014]]></py>
##     <spage><![CDATA[105]]></spage>
##     <epage><![CDATA[113]]></epage>
##     <abstract><![CDATA[The SOI symmetric lateral bipolar transistor is uniquely suitable for operation at high injection currents where the injected minority carrier density in the base region is larger than the base doping concentration. Transistors operating in high-injection can achieve record-high drive currents on the order of 3-5 mA/&#x03BC;m. The commonly used Shockley diode and bipolar current equations are modified to be applicable for all injection levels. Excellent agreement is shown between measured and modeled currents for data at V<sub>BC</sub> = 0. A novel partially depleted-base design can further increase the drive current and the current gain, especially at low V<sub>BE</sub>.]]></abstract>
##     <issn><![CDATA[2168-6734]]></issn>
##     <htmlFlag><![CDATA[1]]></htmlFlag>
##     <arnumber><![CDATA[6837451]]></arnumber>
##     <doi><![CDATA[10.1109/JEDS.2014.2331053]]></doi>
##     <publicationId><![CDATA[6837451]]></publicationId>
##     <mdurl><![CDATA[http://ieeexplore.ieee.org/xpl/articleDetails.jsp?tp=&arnumber=6837451&contentType=Journals+%26+Magazines]]></mdurl>
##     <pdf><![CDATA[http://ieeexplore.ieee.org/stamp/stamp.jsp?arnumber=6837451]]></pdf>
##   </document>
##   <document>
##     <rank>551</rank>
##     <title><![CDATA[Microwave Photonic Integrated Circuits for Millimeter-Wave Wireless Communications]]></title>
##     <authors><![CDATA[Carpintero, G.;  Balakier, K.;  Yang, Z.;  Guzman, R.C.;  Corradi, A.;  Jimenez, A.;  Kervella, G.;  Fice, M.J.;  Lamponi, M.;  Chitoui, M.;  van Dijk, F.;  Renaud, C.C.;  Wonfor, A.;  Bente, E.A.J.M.;  Penty, R.V.;  White, I.H.;  Seeds, A.J.]]></authors>
##     <affiliations><![CDATA[Univ. Carlos III de Madrid, Legane&#x0301;s, Spain]]></affiliations>
##     <controlledterms>
##       <term><![CDATA[integrated circuits]]></term>
##       <term><![CDATA[microwave photonics]]></term>
##       <term><![CDATA[millimetre waves]]></term>
##       <term><![CDATA[radio links]]></term>
##       <term><![CDATA[radio networks]]></term>
##     </controlledterms>
##     <thesaurusterms>
##       <term><![CDATA[Arrayed waveguide gratings]]></term>
##       <term><![CDATA[Optical device fabrication]]></term>
##       <term><![CDATA[Optical feedback]]></term>
##       <term><![CDATA[Optical reflection]]></term>
##       <term><![CDATA[Optical variables measurement]]></term>
##       <term><![CDATA[Photodiodes]]></term>
##       <term><![CDATA[Semiconductor optical amplifiers]]></term>
##     </thesaurusterms>
##     <pubtitle><![CDATA[Lightwave Technology, Journal of]]></pubtitle>
##     <punumber><![CDATA[50]]></punumber>
##     <pubtype><![CDATA[Journals & Magazines]]></pubtype>
##     <publisher><![CDATA[IEEE]]></publisher>
##     <volume><![CDATA[32]]></volume>
##     <issue><![CDATA[20]]></issue>
##     <py><![CDATA[2014]]></py>
##     <spage><![CDATA[3495]]></spage>
##     <epage><![CDATA[3501]]></epage>
##     <abstract><![CDATA[This paper describes the advantages that the introduction of photonic integration technologies can bring to the development of photonic-enabled wireless communications systems operating in the millimeter wave frequency range. We present two approaches for the development of dual wavelength sources for heterodyne-based millimeter wave generation realized using active/passive photonic integration technology. One approach integrates monolithically two distributed feedback semiconductor lasers along with semiconductor optical amplifiers, wavelength combiners, electro-optic modulators and broad bandwidth photodiodes. The other uses a generic photonic integration platform, developing narrow linewidth dual wavelength lasers based on arrayed waveguide gratings. Moreover, data transmission over a wireless link at a carrier wave frequency above 100 GHz is presented, in which the two lasers are free-running, and the modulation is directly applied to the single photonic chip without the requirement of any additional component.]]></abstract>
##     <issn><![CDATA[0733-8724]]></issn>
##     <htmlFlag><![CDATA[1]]></htmlFlag>
##     <arnumber><![CDATA[6810774]]></arnumber>
##     <doi><![CDATA[10.1109/JLT.2014.2321573]]></doi>
##     <publicationId><![CDATA[6810774]]></publicationId>
##     <mdurl><![CDATA[http://ieeexplore.ieee.org/xpl/articleDetails.jsp?tp=&arnumber=6810774&contentType=Journals+%26+Magazines]]></mdurl>
##     <pdf><![CDATA[http://ieeexplore.ieee.org/stamp/stamp.jsp?arnumber=6810774]]></pdf>
##   </document>
##   <document>
##     <rank>552</rank>
##     <title><![CDATA[Predicting and Evaluating the Effect of Bivalirudin in Cardiac Surgical Patients]]></title>
##     <authors><![CDATA[Qi Zhao;  Edrich, T.;  Paschalidis, I.C.]]></authors>
##     <affiliations><![CDATA[Div. of Syst. Eng., Boston Univ., Boston, MA, USA]]></affiliations>
##     <controlledterms>
##       <term><![CDATA[cardiology]]></term>
##       <term><![CDATA[drugs]]></term>
##       <term><![CDATA[surgery]]></term>
##     </controlledterms>
##     <thesaurusterms>
##       <term><![CDATA[Adaptation models]]></term>
##       <term><![CDATA[Coagulation]]></term>
##       <term><![CDATA[Kernel]]></term>
##       <term><![CDATA[Liver]]></term>
##       <term><![CDATA[Mathematical model]]></term>
##       <term><![CDATA[Training]]></term>
##       <term><![CDATA[Vectors]]></term>
##     </thesaurusterms>
##     <pubtitle><![CDATA[Biomedical Engineering, IEEE Transactions on]]></pubtitle>
##     <punumber><![CDATA[10]]></punumber>
##     <pubtype><![CDATA[Journals & Magazines]]></pubtype>
##     <publisher><![CDATA[IEEE]]></publisher>
##     <volume><![CDATA[61]]></volume>
##     <issue><![CDATA[2]]></issue>
##     <py><![CDATA[2014]]></py>
##     <spage><![CDATA[435]]></spage>
##     <epage><![CDATA[443]]></epage>
##     <abstract><![CDATA[Bivalirudin, used in patients with heparin-induced thrombocytopenia, is a direct thrombin inhibitor. Since it is a rarely used drug, clinical experience with its dosing is sparse. We develop two approaches to predict the Partial Thromboplastin Time (PTT) based on bivalirudin infusion rates. The first approach is model free and utilizes regularized regression. It is flexible enough to be used as predictors bivalirudin infusion rates measured over several time instances before the time at which a PTT prediction is sought. The second approach is model based and proposes a specific model for obtaining PTT which uses a shorter history of the past measurements. We learn population-wide model parameters by solving a nonlinear optimization problem. We also devise an adaptive algorithm based on the extended Kalman filter that can adapt model parameters to individual patients. The latter adaptive model emerges as the most promising as it yields reduced mean error compared to the model-free approach. The model accuracy we demonstrate on actual patient measurements is sufficient to be useful in guiding the optimal therapy.]]></abstract>
##     <issn><![CDATA[0018-9294]]></issn>
##     <htmlFlag><![CDATA[1]]></htmlFlag>
##     <arnumber><![CDATA[6588904]]></arnumber>
##     <doi><![CDATA[10.1109/TBME.2013.2280636]]></doi>
##     <publicationId><![CDATA[6588904]]></publicationId>
##     <mdurl><![CDATA[http://ieeexplore.ieee.org/xpl/articleDetails.jsp?tp=&arnumber=6588904&contentType=Journals+%26+Magazines]]></mdurl>
##     <pdf><![CDATA[http://ieeexplore.ieee.org/stamp/stamp.jsp?arnumber=6588904]]></pdf>
##   </document>
##   <document>
##     <rank>553</rank>
##     <title><![CDATA[Frequency Response of a Ferroelectric Material in Double-Negative Region]]></title>
##     <authors><![CDATA[Meng-Ru Wu;  Heng-Tung Hsu;  Chien-Jang Wu;  Shoou-Jinn Chang]]></authors>
##     <affiliations><![CDATA[Dept. of Electr. Eng., Nat. Cheng Kung Univ., Tainan, Taiwan]]></affiliations>
##     <controlledterms>
##       <term><![CDATA[ferroelectric thin films]]></term>
##       <term><![CDATA[infrared spectra]]></term>
##       <term><![CDATA[lithium compounds]]></term>
##       <term><![CDATA[permeability]]></term>
##       <term><![CDATA[permittivity]]></term>
##       <term><![CDATA[surface impedance]]></term>
##     </controlledterms>
##     <thesaurusterms>
##       <term><![CDATA[Films]]></term>
##       <term><![CDATA[Impedance]]></term>
##       <term><![CDATA[Permeability]]></term>
##       <term><![CDATA[Permittivity]]></term>
##       <term><![CDATA[Surface impedance]]></term>
##       <term><![CDATA[Surface resistance]]></term>
##     </thesaurusterms>
##     <pubtitle><![CDATA[Photonics Journal, IEEE]]></pubtitle>
##     <punumber><![CDATA[4563994]]></punumber>
##     <pubtype><![CDATA[Journals & Magazines]]></pubtype>
##     <publisher><![CDATA[IEEE]]></publisher>
##     <volume><![CDATA[6]]></volume>
##     <issue><![CDATA[5]]></issue>
##     <py><![CDATA[2014]]></py>
##     <spage><![CDATA[1]]></spage>
##     <epage><![CDATA[11]]></epage>
##     <abstract><![CDATA[In this work, we investigate the far-infrared wave properties for a ferroelectric material, i.e., lithium tantalate (LiTaO<sub>3</sub>), in the double-negative region, i.e., both the real parts of permittivity and permeability are negative. The analysis has been done based on the calculated surface impedances for three model structures, i.e., material occupying semi-infinite space (structure I), material of a slab immersed in free space (structure II), and a layered structure made of film on a dielectric substrate (structure III). It is found that the surface impedance spectrum exhibits resonant behavior. In structure I, there are only two resonant points, which arise mainly from the pole of permeability and the zero of permittivity. In structure II, multiple resonances can be found, and they strongly rely on the film thickness. Finally, we specifically investigate the substrate resonant phenomenon in structure III.]]></abstract>
##     <issn><![CDATA[1943-0655]]></issn>
##     <htmlFlag><![CDATA[1]]></htmlFlag>
##     <arnumber><![CDATA[6910251]]></arnumber>
##     <doi><![CDATA[10.1109/JPHOT.2014.2360295]]></doi>
##     <publicationId><![CDATA[6910251]]></publicationId>
##     <mdurl><![CDATA[http://ieeexplore.ieee.org/xpl/articleDetails.jsp?tp=&arnumber=6910251&contentType=Journals+%26+Magazines]]></mdurl>
##     <pdf><![CDATA[http://ieeexplore.ieee.org/stamp/stamp.jsp?arnumber=6910251]]></pdf>
##   </document>
##   <document>
##     <rank>554</rank>
##     <title><![CDATA[High-Power CMOS Current Driver With Accurate Transconductance for Electrical Impedance Tomography]]></title>
##     <authors><![CDATA[Constantinou, L.;  Triantis, I.F.;  Bayford, R.;  Demosthenous, A.]]></authors>
##     <affiliations><![CDATA[Dept. of Electron. & Electr. Eng., Univ. Coll. London, London, UK]]></affiliations>
##     <controlledterms>
##       <term><![CDATA[CMOS integrated circuits]]></term>
##       <term><![CDATA[bioelectric phenomena]]></term>
##       <term><![CDATA[biomedical electronics]]></term>
##       <term><![CDATA[electric impedance imaging]]></term>
##       <term><![CDATA[harmonic distortion]]></term>
##       <term><![CDATA[resistors]]></term>
##     </controlledterms>
##     <thesaurusterms>
##       <term><![CDATA[Current measurement]]></term>
##       <term><![CDATA[Impedance]]></term>
##       <term><![CDATA[Resistors]]></term>
##       <term><![CDATA[Topology]]></term>
##       <term><![CDATA[Transconductance]]></term>
##       <term><![CDATA[Transistors]]></term>
##       <term><![CDATA[Voltage measurement]]></term>
##     </thesaurusterms>
##     <pubtitle><![CDATA[Biomedical Circuits and Systems, IEEE Transactions on]]></pubtitle>
##     <punumber><![CDATA[4156126]]></punumber>
##     <pubtype><![CDATA[Journals & Magazines]]></pubtype>
##     <publisher><![CDATA[IEEE]]></publisher>
##     <volume><![CDATA[8]]></volume>
##     <issue><![CDATA[4]]></issue>
##     <py><![CDATA[2014]]></py>
##     <spage><![CDATA[575]]></spage>
##     <epage><![CDATA[583]]></epage>
##     <abstract><![CDATA[Current drivers are fundamental circuits in bioimpedance measurements including electrical impedance tomography (EIT). In the case of EIT, the current driver is required to have a large output impedance to guarantee high current accuracy over a wide range of load impedance values. This paper presents an integrated current driver which meets these requirements and is capable of delivering large sinusoidal currents to the load. The current driver employs a differential architecture and negative feedback, the latter allowing the output current to be accurately set by the ratio of the input voltage to a resistor value. The circuit was fabricated in a 0.6-&#x03BC;m high-voltage CMOS process technology and its core occupies a silicon area of 0.64 mm<sup>2</sup>. It operates from a &#x00B1; 9 V power supply and can deliver output currents up to 5 mA p-p. The accuracy of the maximum output current is within 0.41% up to 500 kHz, reducing to 0.47% at 1 MHz with a total harmonic distortion of 0.69%. The output impedance is 665 k&#x03A9; at 100 kHz and 372 k&#x03A9; at 500 kHz.]]></abstract>
##     <issn><![CDATA[1932-4545]]></issn>
##     <htmlFlag><![CDATA[1]]></htmlFlag>
##     <arnumber><![CDATA[6693752]]></arnumber>
##     <doi><![CDATA[10.1109/TBCAS.2013.2285481]]></doi>
##     <publicationId><![CDATA[6693752]]></publicationId>
##     <mdurl><![CDATA[http://ieeexplore.ieee.org/xpl/articleDetails.jsp?tp=&arnumber=6693752&contentType=Journals+%26+Magazines]]></mdurl>
##     <pdf><![CDATA[http://ieeexplore.ieee.org/stamp/stamp.jsp?arnumber=6693752]]></pdf>
##   </document>
##   <document>
##     <rank>555</rank>
##     <title><![CDATA[Broadband Microwave Photonic Splitter With Arbitrary Amplitude Ratio and Phase Shift]]></title>
##     <authors><![CDATA[Wei Li;  Wen Ting Wang;  Ning Hua Zhu]]></authors>
##     <affiliations><![CDATA[State Key Lab. on Integrated Optoelectron., Inst. of Semicond., Beijing, China]]></affiliations>
##     <controlledterms>
##       <term><![CDATA[intensity modulation]]></term>
##       <term><![CDATA[microwave photonics]]></term>
##       <term><![CDATA[optical beam splitters]]></term>
##       <term><![CDATA[optical modulation]]></term>
##       <term><![CDATA[optical polarisers]]></term>
##       <term><![CDATA[optical tuning]]></term>
##     </controlledterms>
##     <thesaurusterms>
##       <term><![CDATA[Microwave amplifiers]]></term>
##       <term><![CDATA[Microwave filters]]></term>
##       <term><![CDATA[Microwave photonics]]></term>
##       <term><![CDATA[Optical filters]]></term>
##       <term><![CDATA[Optical modulation]]></term>
##       <term><![CDATA[Optical polarization]]></term>
##     </thesaurusterms>
##     <pubtitle><![CDATA[Photonics Journal, IEEE]]></pubtitle>
##     <punumber><![CDATA[4563994]]></punumber>
##     <pubtype><![CDATA[Journals & Magazines]]></pubtype>
##     <publisher><![CDATA[IEEE]]></publisher>
##     <volume><![CDATA[6]]></volume>
##     <issue><![CDATA[6]]></issue>
##     <py><![CDATA[2014]]></py>
##     <spage><![CDATA[1]]></spage>
##     <epage><![CDATA[7]]></epage>
##     <abstract><![CDATA[We report a broadband 1&#x00D7; N microwave photonic splitter (MPS) with arbitrary amplitude ratio and phase shift. The key devices used in the MPS are a polarization modulator (PolM), an optical bandpass filter (OBPF), polarization controllers (PCs), and polarizers (Pols). The cascaded PolM and OBPF generate an orthogonally polarized single-sideband modulated optical signal. The phase difference between the two orthogonal components and the state of polarization of the optical signal can be adjusted by the PC independently. After the polarization-to-intensity modulation conversion by the Pol, the microwave is recovered in the PD. The amplitude and phase of the microwave signal can be independently tuned by adjusting the three plates of the PC. The proposed MPS is theoretically analyzed and experimentally verified. Broadband MPS with bandwidth from 8 to 40 GHz is successfully achieved. In addition, we apply the proposed MPS to a two-tap microwave photonic filter (MPF). The notch position and depth of the MPF are tunable by adjusting the phase shift and amplitude ratio between the two optical branches, respectively.]]></abstract>
##     <issn><![CDATA[1943-0655]]></issn>
##     <htmlFlag><![CDATA[1]]></htmlFlag>
##     <arnumber><![CDATA[6951420]]></arnumber>
##     <doi><![CDATA[10.1109/JPHOT.2014.2368784]]></doi>
##     <publicationId><![CDATA[6951420]]></publicationId>
##     <mdurl><![CDATA[http://ieeexplore.ieee.org/xpl/articleDetails.jsp?tp=&arnumber=6951420&contentType=Journals+%26+Magazines]]></mdurl>
##     <pdf><![CDATA[http://ieeexplore.ieee.org/stamp/stamp.jsp?arnumber=6951420]]></pdf>
##   </document>
##   <document>
##     <rank>556</rank>
##     <title><![CDATA[Energy-Aware Data Allocation and Task Scheduling on Heterogeneous Multiprocessor Systems With Time Constraints]]></title>
##     <authors><![CDATA[Yan Wang;  Kenli Li;  Hao Chen;  Ligang He;  Keqin Li]]></authors>
##     <affiliations><![CDATA[Coll. of Inf. Sci. & Eng., Hunan Univ., Changsha, China]]></affiliations>
##     <controlledterms>
##       <term><![CDATA[computational complexity]]></term>
##       <term><![CDATA[distributed shared memory systems]]></term>
##       <term><![CDATA[energy consumption]]></term>
##       <term><![CDATA[greedy algorithms]]></term>
##       <term><![CDATA[integer programming]]></term>
##       <term><![CDATA[linear programming]]></term>
##       <term><![CDATA[processor scheduling]]></term>
##     </controlledterms>
##     <thesaurusterms>
##       <term><![CDATA[Energy consumption]]></term>
##       <term><![CDATA[Memory management]]></term>
##       <term><![CDATA[Processor scheduling]]></term>
##       <term><![CDATA[Program processors]]></term>
##       <term><![CDATA[Resource management]]></term>
##       <term><![CDATA[Scheduling]]></term>
##     </thesaurusterms>
##     <pubtitle><![CDATA[Emerging Topics in Computing, IEEE Transactions on]]></pubtitle>
##     <punumber><![CDATA[6245516]]></punumber>
##     <pubtype><![CDATA[Journals & Magazines]]></pubtype>
##     <publisher><![CDATA[IEEE]]></publisher>
##     <volume><![CDATA[2]]></volume>
##     <issue><![CDATA[2]]></issue>
##     <py><![CDATA[2014]]></py>
##     <spage><![CDATA[134]]></spage>
##     <epage><![CDATA[148]]></epage>
##     <abstract><![CDATA[In this paper, we address the problem of energy-aware heterogeneous data allocation and task scheduling on heterogeneous multiprocessor systems for real-time applications. In a heterogeneous distributed shared-memory multiprocessor system, an important problem is how to assign processors to real-time application tasks, allocate data to local memories, and generate an efficient schedule in such a way that a time constraint can be met and the total system energy consumption can be minimized. We propose an optimal approach, i.e., an integer linear programming method, to solve this problem. As the problem has been conclusively shown to be computationally very complicated, we also present two heuristic algorithms, i.e., task assignment considering data allocation (TAC-DA) and task ratio greedy scheduling (TRGS), to generate near-optimal solutions for real-time applications in polynomial time. We evaluate the performance of our algorithms by comparing them with a greedy algorithm that is commonly used to solve heterogeneous task scheduling problems. Based on our extensive simulation study, we observe that our algorithms exhibit excellent performance. We conducted experimental performance evaluation on two heterogeneous multiprocessor systems. The average reduction rates of the total energy consumption of the TAC-DA and TRGS algorithms to that of the greedy algorithm are 13.72% and 15.76%, respectively, on the first system, and 19.76% and 24.67%, respectively, on the second system. To the best of our knowledge, this is the first study to solve the problem of task scheduling incorporated with data allocation and energy consumption on heterogeneous distributed shared-memory multiprocessor systems.]]></abstract>
##     <issn><![CDATA[2168-6750]]></issn>
##     <htmlFlag><![CDATA[1]]></htmlFlag>
##     <arnumber><![CDATA[6714422]]></arnumber>
##     <doi><![CDATA[10.1109/TETC.2014.2300632]]></doi>
##     <publicationId><![CDATA[6714422]]></publicationId>
##     <mdurl><![CDATA[http://ieeexplore.ieee.org/xpl/articleDetails.jsp?tp=&arnumber=6714422&contentType=Journals+%26+Magazines]]></mdurl>
##     <pdf><![CDATA[http://ieeexplore.ieee.org/stamp/stamp.jsp?arnumber=6714422]]></pdf>
##   </document>
##   <document>
##     <rank>557</rank>
##     <title><![CDATA[Preference Learning for Move Prediction and Evaluation Function Approximation in Othello]]></title>
##     <authors><![CDATA[Runarsson, T.P.;  Lucas, S.M.]]></authors>
##     <affiliations><![CDATA[Sch. of Eng. & Natural Sci., Univ. of Iceland, Reykjavik, Iceland]]></affiliations>
##     <controlledterms>
##       <term><![CDATA[computer games]]></term>
##       <term><![CDATA[function approximation]]></term>
##       <term><![CDATA[learning (artificial intelligence)]]></term>
##       <term><![CDATA[least squares approximations]]></term>
##       <term><![CDATA[pattern classification]]></term>
##     </controlledterms>
##     <thesaurusterms>
##       <term><![CDATA[Games]]></term>
##       <term><![CDATA[Monte Carlo methods]]></term>
##       <term><![CDATA[Radiation detectors]]></term>
##       <term><![CDATA[Standards]]></term>
##       <term><![CDATA[Training]]></term>
##       <term><![CDATA[Trajectory]]></term>
##       <term><![CDATA[Vectors]]></term>
##     </thesaurusterms>
##     <pubtitle><![CDATA[Computational Intelligence and AI in Games, IEEE Transactions on]]></pubtitle>
##     <punumber><![CDATA[4804728]]></punumber>
##     <pubtype><![CDATA[Journals & Magazines]]></pubtype>
##     <publisher><![CDATA[IEEE]]></publisher>
##     <volume><![CDATA[6]]></volume>
##     <issue><![CDATA[3]]></issue>
##     <py><![CDATA[2014]]></py>
##     <spage><![CDATA[300]]></spage>
##     <epage><![CDATA[313]]></epage>
##     <abstract><![CDATA[This paper investigates the use of preference learning as an approach to move prediction and evaluation function approximation, using the game of Othello as a test domain. Using the same sets of features, we compare our approach with least squares temporal difference learning, direct classification, and with the Bradley-Terry model, fitted using minorization-maximization (MM). The results show that the exact way in which preference learning is applied is critical to achieving high performance. Best results were obtained using a combination of board inversion and pair-wise preference learning. This combination significantly outperformed the others under test, both in terms of move prediction accuracy, and in the level of play achieved when using the learned evaluation function as a move selector during game play.]]></abstract>
##     <issn><![CDATA[1943-068X]]></issn>
##     <htmlFlag><![CDATA[1]]></htmlFlag>
##     <arnumber><![CDATA[6762937]]></arnumber>
##     <doi><![CDATA[10.1109/TCIAIG.2014.2307272]]></doi>
##     <publicationId><![CDATA[6762937]]></publicationId>
##     <mdurl><![CDATA[http://ieeexplore.ieee.org/xpl/articleDetails.jsp?tp=&arnumber=6762937&contentType=Journals+%26+Magazines]]></mdurl>
##     <pdf><![CDATA[http://ieeexplore.ieee.org/stamp/stamp.jsp?arnumber=6762937]]></pdf>
##   </document>
##   <document>
##     <rank>558</rank>
##     <title><![CDATA[A Trend-Oriented Power System Security Analysis Method Based on Load Profile]]></title>
##     <authors><![CDATA[Anjia Mao;  Iravani, M.R.]]></authors>
##     <affiliations><![CDATA[Sch. of Electr. & Electron. Eng., North China Electr. Power Univ., Beijing, China]]></affiliations>
##     <controlledterms>
##       <term><![CDATA[power system security]]></term>
##       <term><![CDATA[probability]]></term>
##     </controlledterms>
##     <thesaurusterms>
##       <term><![CDATA[Equations]]></term>
##       <term><![CDATA[Market research]]></term>
##       <term><![CDATA[Mathematical model]]></term>
##       <term><![CDATA[Power system security]]></term>
##       <term><![CDATA[Power system stability]]></term>
##     </thesaurusterms>
##     <pubtitle><![CDATA[Power Systems, IEEE Transactions on]]></pubtitle>
##     <punumber><![CDATA[59]]></punumber>
##     <pubtype><![CDATA[Journals & Magazines]]></pubtype>
##     <publisher><![CDATA[IEEE]]></publisher>
##     <volume><![CDATA[29]]></volume>
##     <issue><![CDATA[3]]></issue>
##     <py><![CDATA[2014]]></py>
##     <spage><![CDATA[1279]]></spage>
##     <epage><![CDATA[1286]]></epage>
##     <abstract><![CDATA[Conventional power system security analysis based on multiple case studies cannot predict all operational states and hence neither guarantees an adequate security margin nor an economical operation for the power system. Based on the trend analysis method which is used in the field of economics, this paper introduces the concept and a methodology for &#x201C;trend security analysis&#x201D; of power systems. This method utilizes the load profile forecast and the contingency occurrence probability and determines the system security trend in the subsequent time window. Based on a recursive algorithm which is developed by utilizing the higher order derivatives of power flow equations, this paper also presents a method for fast determination of the trend variations of the system states and security indices. Three IEEE test systems are used to demonstrate the applications of the proposed concepts, evaluate their performance and verify their accuracy.]]></abstract>
##     <issn><![CDATA[0885-8950]]></issn>
##     <htmlFlag><![CDATA[1]]></htmlFlag>
##     <arnumber><![CDATA[6675082]]></arnumber>
##     <doi><![CDATA[10.1109/TPWRS.2013.2291400]]></doi>
##     <publicationId><![CDATA[6675082]]></publicationId>
##     <mdurl><![CDATA[http://ieeexplore.ieee.org/xpl/articleDetails.jsp?tp=&arnumber=6675082&contentType=Journals+%26+Magazines]]></mdurl>
##     <pdf><![CDATA[http://ieeexplore.ieee.org/stamp/stamp.jsp?arnumber=6675082]]></pdf>
##   </document>
##   <document>
##     <rank>559</rank>
##     <title><![CDATA[Effect of Using an In-Vehicle Smart Driving Aid on Real-World Driver Performance]]></title>
##     <authors><![CDATA[Birrell, S.A.;  Fowkes, M.;  Jennings, P.A.]]></authors>
##     <affiliations><![CDATA[MIRA Ltd., Nuneaton, UK]]></affiliations>
##     <controlledterms>
##       <term><![CDATA[driver information systems]]></term>
##       <term><![CDATA[feedback]]></term>
##       <term><![CDATA[intelligent transportation systems]]></term>
##       <term><![CDATA[smart phones]]></term>
##     </controlledterms>
##     <thesaurusterms>
##       <term><![CDATA[Acceleration]]></term>
##       <term><![CDATA[Fuels]]></term>
##       <term><![CDATA[Gears]]></term>
##       <term><![CDATA[Global Positioning System]]></term>
##       <term><![CDATA[Real-time systems]]></term>
##       <term><![CDATA[Safety]]></term>
##       <term><![CDATA[Vehicles]]></term>
##     </thesaurusterms>
##     <pubtitle><![CDATA[Intelligent Transportation Systems, IEEE Transactions on]]></pubtitle>
##     <punumber><![CDATA[6979]]></punumber>
##     <pubtype><![CDATA[Journals & Magazines]]></pubtype>
##     <publisher><![CDATA[IEEE]]></publisher>
##     <volume><![CDATA[15]]></volume>
##     <issue><![CDATA[4]]></issue>
##     <py><![CDATA[2014]]></py>
##     <spage><![CDATA[1801]]></spage>
##     <epage><![CDATA[1810]]></epage>
##     <abstract><![CDATA[A smart driving system (providing both safety and fuel-efficient driving advice in real time in the vehicle) was evaluated in real-world on-road driving trials to see if any measurable beneficial changes in driving performance would be observed. Forty participants drove an instrumented vehicle over a 50-min mixed-route driving scenario. Two conditions were adopted: one is a control with no smart driving feedback offered and the other is with advice being presented to the driver via a smartphone in the vehicle. Key findings from the study showed a 4.1% improvement in fuel efficiency when using the smart driving aid, importantly with no increase in journey time or reduction in average speed. Primarily, these efficiency savings were enabled by limiting the use of lower gears (facilitated by planning ahead to avoid unnecessary stops) and an increase in the use of the fifth gear (as advised by the in-vehicle system). Significant and important changes in driving safety behaviors were also observed, with an increase in mean headway to 2.3 s and an almost threefold reduction in time spent traveling closer than 1.5 s to the vehicle in front. This paper has shown that an in-vehicle smart driving system specifically developed and designed with the drivers' information requirements in mind can lead to significant improvements in driving behaviors in the real world on real roads with real users.]]></abstract>
##     <issn><![CDATA[1524-9050]]></issn>
##     <htmlFlag><![CDATA[1]]></htmlFlag>
##     <arnumber><![CDATA[6841028]]></arnumber>
##     <doi><![CDATA[10.1109/TITS.2014.2328357]]></doi>
##     <publicationId><![CDATA[6841028]]></publicationId>
##     <mdurl><![CDATA[http://ieeexplore.ieee.org/xpl/articleDetails.jsp?tp=&arnumber=6841028&contentType=Journals+%26+Magazines]]></mdurl>
##     <pdf><![CDATA[http://ieeexplore.ieee.org/stamp/stamp.jsp?arnumber=6841028]]></pdf>
##   </document>
##   <document>
##     <rank>560</rank>
##     <title><![CDATA[ConTour: Data-Driven Exploration of Multi-Relational Datasets for Drug Discovery]]></title>
##     <authors><![CDATA[Partl, C.;  Lex, A.;  Streit, M.;  Strobelt, H.;  Wassermann, A.-M.;  Pfister, H.;  Schmalstieg, D.]]></authors>
##     <affiliations><![CDATA[Graz Univ. of Technol., Graz, Austria]]></affiliations>
##     <controlledterms>
##       <term><![CDATA[chemistry computing]]></term>
##       <term><![CDATA[data analysis]]></term>
##       <term><![CDATA[data visualisation]]></term>
##       <term><![CDATA[sorting]]></term>
##     </controlledterms>
##     <thesaurusterms>
##       <term><![CDATA[Biomedical informatics]]></term>
##       <term><![CDATA[Data visualization]]></term>
##       <term><![CDATA[Drugs]]></term>
##       <term><![CDATA[Large-scale systems]]></term>
##       <term><![CDATA[Proteins]]></term>
##       <term><![CDATA[Visual analytics]]></term>
##     </thesaurusterms>
##     <pubtitle><![CDATA[Visualization and Computer Graphics, IEEE Transactions on]]></pubtitle>
##     <punumber><![CDATA[2945]]></punumber>
##     <pubtype><![CDATA[Journals & Magazines]]></pubtype>
##     <publisher><![CDATA[IEEE]]></publisher>
##     <volume><![CDATA[20]]></volume>
##     <issue><![CDATA[12]]></issue>
##     <py><![CDATA[2014]]></py>
##     <spage><![CDATA[1883]]></spage>
##     <epage><![CDATA[1892]]></epage>
##     <abstract><![CDATA[Large scale data analysis is nowadays a crucial part of drug discovery. Biologists and chemists need to quickly explore and evaluate potentially effective yet safe compounds based on many datasets that are in relationship with each other. However, there is a lack of tools that support them in these processes. To remedy this, we developed ConTour, an interactive visual analytics technique that enables the exploration of these complex, multi-relational datasets. At its core ConTour lists all items of each dataset in a column. Relationships between the columns are revealed through interaction: selecting one or multiple items in one column highlights and re-sorts the items in other columns. Filters based on relationships enable drilling down into the large data space. To identify interesting items in the first place, ConTour employs advanced sorting strategies, including strategies based on connectivity strength and uniqueness, as well as sorting based on item attributes. ConTour also introduces interactive nesting of columns, a powerful method to show the related items of a child column for each item in the parent column. Within the columns, ConTour shows rich attribute data about the items as well as information about the connection strengths to other datasets. Finally, ConTour provides a number of detail views, which can show items from multiple datasets and their associated data at the same time. We demonstrate the utility of our system in case studies conducted with a team of chemical biologists, who investigate the effects of chemical compounds on cells and need to understand the underlying mechanisms.]]></abstract>
##     <issn><![CDATA[1077-2626]]></issn>
##     <htmlFlag><![CDATA[1]]></htmlFlag>
##     <arnumber><![CDATA[6875994]]></arnumber>
##     <doi><![CDATA[10.1109/TVCG.2014.2346752]]></doi>
##     <publicationId><![CDATA[6875994]]></publicationId>
##     <mdurl><![CDATA[http://ieeexplore.ieee.org/xpl/articleDetails.jsp?tp=&arnumber=6875994&contentType=Journals+%26+Magazines]]></mdurl>
##     <pdf><![CDATA[http://ieeexplore.ieee.org/stamp/stamp.jsp?arnumber=6875994]]></pdf>
##   </document>
##   <document>
##     <rank>561</rank>
##     <title><![CDATA[Channel Estimation for Two-Way Relay Networks in the Presence of Synchronization Errors]]></title>
##     <authors><![CDATA[Xinqian Xie;  Mugen Peng;  Yonghui Li;  Wenbo Wang;  Poor, H.V.]]></authors>
##     <affiliations><![CDATA[Key Lab. of Universal Wireless Commun. for Minist. of Educ., Beijing Univ. of Posts & Telecommun., Beijing, China]]></affiliations>
##     <controlledterms>
##       <term><![CDATA[channel estimation]]></term>
##       <term><![CDATA[error detection]]></term>
##       <term><![CDATA[error statistics]]></term>
##       <term><![CDATA[least mean squares methods]]></term>
##       <term><![CDATA[relay networks (telecommunication)]]></term>
##       <term><![CDATA[synchronisation]]></term>
##       <term><![CDATA[wireless channels]]></term>
##     </controlledterms>
##     <thesaurusterms>
##       <term><![CDATA[Channel estimation]]></term>
##       <term><![CDATA[Estimation]]></term>
##       <term><![CDATA[Relays]]></term>
##       <term><![CDATA[Signal processing algorithms]]></term>
##       <term><![CDATA[Silicon]]></term>
##       <term><![CDATA[Synchronization]]></term>
##       <term><![CDATA[Training]]></term>
##     </thesaurusterms>
##     <pubtitle><![CDATA[Signal Processing, IEEE Transactions on]]></pubtitle>
##     <punumber><![CDATA[78]]></punumber>
##     <pubtype><![CDATA[Journals & Magazines]]></pubtype>
##     <publisher><![CDATA[IEEE]]></publisher>
##     <volume><![CDATA[62]]></volume>
##     <issue><![CDATA[23]]></issue>
##     <py><![CDATA[2014]]></py>
##     <spage><![CDATA[6235]]></spage>
##     <epage><![CDATA[6248]]></epage>
##     <abstract><![CDATA[This paper investigates pilot-aided channel estimation for two-way relay networks (TWRNs) in the presence of synchronization errors between the two sources. The unpredictable synchronization error leads to time domain offset and signal arriving order (SAO) ambiguity when two signals sent from two sources are superimposed at the relay. A two-step channel estimation algorithm is first proposed, in which the linear minimum mean-square-error (LMMSE) estimator is used to obtain initial channel estimates based on pilot symbols and a linear minimum error probability (LMEP) estimator is then developed to update these estimates. Optimal training sequences and power allocation at the relay are designed to further improve the performance for LMMSE based initial channel estimation. To tackle the SAO ambiguity problem, the generalized likelihood ratio testing method is applied and an upper bound on the SAO detection error probability is derived. By using the SAO information, a scaled LMEP estimation algorithm is proposed to compensate for the performance degradation caused by SAO detection error. Simulation results show that the proposed estimation algorithms can effectively mitigate the negative effects caused by asynchronous transmissions in TWRNs, thus significantly outperforming the existing channel estimation algorithms.]]></abstract>
##     <issn><![CDATA[1053-587X]]></issn>
##     <htmlFlag><![CDATA[1]]></htmlFlag>
##     <arnumber><![CDATA[6907998]]></arnumber>
##     <doi><![CDATA[10.1109/TSP.2014.2360146]]></doi>
##     <publicationId><![CDATA[6907998]]></publicationId>
##     <mdurl><![CDATA[http://ieeexplore.ieee.org/xpl/articleDetails.jsp?tp=&arnumber=6907998&contentType=Journals+%26+Magazines]]></mdurl>
##     <pdf><![CDATA[http://ieeexplore.ieee.org/stamp/stamp.jsp?arnumber=6907998]]></pdf>
##   </document>
##   <document>
##     <rank>562</rank>
##     <title><![CDATA[Effect of Humidity on Scintillation Performance in Na and Tl Activated CsI Crystals]]></title>
##     <authors><![CDATA[Pin Yang;  Harmon, C.D.;  Doty, F.P.;  Ohlhausen, J.A.]]></authors>
##     <affiliations><![CDATA[Sandia Nat. Labs., Albuquerque, NM, USA]]></affiliations>
##     <controlledterms>
##       <term><![CDATA[scintillation counters]]></term>
##       <term><![CDATA[time of flight mass spectroscopy]]></term>
##     </controlledterms>
##     <thesaurusterms>
##       <term><![CDATA[Crystals]]></term>
##       <term><![CDATA[Degradation]]></term>
##       <term><![CDATA[Detectors]]></term>
##       <term><![CDATA[Humidity]]></term>
##       <term><![CDATA[Surface treatment]]></term>
##       <term><![CDATA[Temperature measurement]]></term>
##     </thesaurusterms>
##     <pubtitle><![CDATA[Nuclear Science, IEEE Transactions on]]></pubtitle>
##     <punumber><![CDATA[23]]></punumber>
##     <pubtype><![CDATA[Journals & Magazines]]></pubtype>
##     <publisher><![CDATA[IEEE]]></publisher>
##     <volume><![CDATA[61]]></volume>
##     <issue><![CDATA[2]]></issue>
##     <py><![CDATA[2014]]></py>
##     <spage><![CDATA[1024]]></spage>
##     <epage><![CDATA[1031]]></epage>
##     <abstract><![CDATA[Time dependent photoluminescence and radioluminescence for sodium (Na) and thallium (Tl) activated cesium iodide (CsI) single crystals exposed to 50% and 75% relative humidity (RH) has been investigated. These results indicate that Tl activated crystals are more robust than the Na activated crystals against humidity induced scintillation degradation. The development of &#x201C;etching pits&#x201D; and &#x201C;inactive&#x201D; domains are the characteristics of deteriorated Na activated CsI crystals. These &#x201C;inactive&#x201D; domains, bearing a resemblance to a polycrystalline appearance beneath the crystal surface, can be readily detected by a 250 nm light emitting diode. These features are commonly observed at the corners and deep scratched areas where moisture condensation is more likely to occur. Mechanisms contributing to the scintillation degradation in Na activated CsI crystals were investigated by Time-of-Flight Secondary Ion Mass Spectrometry (ToF-SIMS). ToF-SIMS depth profiles indicate that Na has been preferentially diffused out of CsI crystal, leaving the Na concentration in these &#x201C;inactive&#x201D; domains below its scintillation threshold.]]></abstract>
##     <issn><![CDATA[0018-9499]]></issn>
##     <htmlFlag><![CDATA[1]]></htmlFlag>
##     <arnumber><![CDATA[6767156]]></arnumber>
##     <doi><![CDATA[10.1109/TNS.2014.2300471]]></doi>
##     <publicationId><![CDATA[6767156]]></publicationId>
##     <mdurl><![CDATA[http://ieeexplore.ieee.org/xpl/articleDetails.jsp?tp=&arnumber=6767156&contentType=Journals+%26+Magazines]]></mdurl>
##     <pdf><![CDATA[http://ieeexplore.ieee.org/stamp/stamp.jsp?arnumber=6767156]]></pdf>
##   </document>
##   <document>
##     <rank>563</rank>
##     <title><![CDATA[Multivariable Dynamic Ankle Mechanical Impedance With Active Muscles]]></title>
##     <authors><![CDATA[Hyunglae Lee;  Krebs, H.I.;  Hogan, N.]]></authors>
##     <affiliations><![CDATA[Dept. of Mech. Eng., Massachusetts Inst. of Technol., Cambridge, MA, USA]]></affiliations>
##     <controlledterms>
##       <term><![CDATA[biomechanics]]></term>
##       <term><![CDATA[biomedical measurement]]></term>
##       <term><![CDATA[elastic constants]]></term>
##       <term><![CDATA[impedance matrix]]></term>
##       <term><![CDATA[medical disorders]]></term>
##       <term><![CDATA[muscle]]></term>
##     </controlledterms>
##     <thesaurusterms>
##       <term><![CDATA[Electromyography]]></term>
##       <term><![CDATA[Impedance]]></term>
##       <term><![CDATA[Impedance measurement]]></term>
##       <term><![CDATA[Joints]]></term>
##       <term><![CDATA[Muscles]]></term>
##       <term><![CDATA[Robots]]></term>
##       <term><![CDATA[Torque]]></term>
##     </thesaurusterms>
##     <pubtitle><![CDATA[Neural Systems and Rehabilitation Engineering, IEEE Transactions on]]></pubtitle>
##     <punumber><![CDATA[7333]]></punumber>
##     <pubtype><![CDATA[Journals & Magazines]]></pubtype>
##     <publisher><![CDATA[IEEE]]></publisher>
##     <volume><![CDATA[22]]></volume>
##     <issue><![CDATA[5]]></issue>
##     <py><![CDATA[2014]]></py>
##     <spage><![CDATA[971]]></spage>
##     <epage><![CDATA[981]]></epage>
##     <abstract><![CDATA[Multivariable dynamic ankle mechanical impedance in two coupled degrees-of-freedom (DOFs) was quantified when muscles were active. Measurements were performed at five different target activation levels of tibialis anterior and soleus, from 10% to 30% of maximum voluntary contraction (MVC) with increments of 5% MVC. Interestingly, several ankle behaviors characterized in our previous study of the relaxed ankle were observed with muscles active: ankle mechanical impedance in joint coordinates showed responses largely consistent with a second-order system consisting of inertia, viscosity, and stiffness; stiffness was greater in the sagittal plane than in the frontal plane at all activation conditions for all subjects; and the coupling between dorsiflexion-plantarflexion and inversion-eversion was small - the two DOF measurements were well explained by a strictly diagonal impedance matrix. In general, ankle stiffness increased linearly with muscle activation in all directions in the 2-D space formed by the sagittal and frontal planes, but more in the sagittal than in the frontal plane, resulting in an accentuated &#x201C;peanut shape.&#x201D; This characterization of young healthy subjects' ankle mechanical impedance with active muscles will serve as a baseline to investigate pathophysiological ankle behaviors of biomechanically and/or neurologically impaired patients.]]></abstract>
##     <issn><![CDATA[1534-4320]]></issn>
##     <htmlFlag><![CDATA[1]]></htmlFlag>
##     <arnumber><![CDATA[6825865]]></arnumber>
##     <doi><![CDATA[10.1109/TNSRE.2014.2328235]]></doi>
##     <publicationId><![CDATA[6825865]]></publicationId>
##     <mdurl><![CDATA[http://ieeexplore.ieee.org/xpl/articleDetails.jsp?tp=&arnumber=6825865&contentType=Journals+%26+Magazines]]></mdurl>
##     <pdf><![CDATA[http://ieeexplore.ieee.org/stamp/stamp.jsp?arnumber=6825865]]></pdf>
##   </document>
##   <document>
##     <rank>564</rank>
##     <title><![CDATA[A Fair Comparison Should Be Based on the Same Protocol--Comments on &#x0022;Trainable Convolution Filters and Their Application to Face Recognition&#x0022;]]></title>
##     <authors><![CDATA[Liang Chen]]></authors>
##     <affiliations><![CDATA[Coll. of Math. & Inf. Sci., Wenzhou Univ. (adjunct), Wenzhou, China]]></affiliations>
##     <controlledterms>
##       <term><![CDATA[convolution]]></term>
##       <term><![CDATA[face recognition]]></term>
##       <term><![CDATA[filtering theory]]></term>
##       <term><![CDATA[image classification]]></term>
##       <term><![CDATA[visual databases]]></term>
##     </controlledterms>
##     <thesaurusterms>
##       <term><![CDATA[Computer vision]]></term>
##       <term><![CDATA[Face recognition]]></term>
##       <term><![CDATA[Kernel]]></term>
##       <term><![CDATA[Protocols]]></term>
##       <term><![CDATA[Standards]]></term>
##       <term><![CDATA[Training]]></term>
##     </thesaurusterms>
##     <pubtitle><![CDATA[Pattern Analysis and Machine Intelligence, IEEE Transactions on]]></pubtitle>
##     <punumber><![CDATA[34]]></punumber>
##     <pubtype><![CDATA[Journals & Magazines]]></pubtype>
##     <publisher><![CDATA[IEEE]]></publisher>
##     <volume><![CDATA[36]]></volume>
##     <issue><![CDATA[3]]></issue>
##     <py><![CDATA[2014]]></py>
##     <spage><![CDATA[622]]></spage>
##     <epage><![CDATA[623]]></epage>
##     <abstract><![CDATA[We comment on a paper describing an image classification approach called Volterra kernel classifier, which was called Volterrafaces when applied to face recognition. The performances were evaluated by the experiments on face recognition databases. We find that their comparisons with the state of the art of three databases were indeed based on unfair settings. The results with the settings of the standard protocol on three data sets are generated, which show that Volterrafaces achieves the state-of-the-art performance only in one database.]]></abstract>
##     <issn><![CDATA[0162-8828]]></issn>
##     <arnumber><![CDATA[6616538]]></arnumber>
##     <doi><![CDATA[10.1109/TPAMI.2013.187]]></doi>
##     <publicationId><![CDATA[6616538]]></publicationId>
##     <mdurl><![CDATA[http://ieeexplore.ieee.org/xpl/articleDetails.jsp?tp=&arnumber=6616538&contentType=Journals+%26+Magazines]]></mdurl>
##     <pdf><![CDATA[http://ieeexplore.ieee.org/stamp/stamp.jsp?arnumber=6616538]]></pdf>
##   </document>
##   <document>
##     <rank>565</rank>
##     <title><![CDATA[Predicting Hurricane Power Outages to Support Storm Response Planning]]></title>
##     <authors><![CDATA[Guikema, S.D.;  Nateghi, R.;  Quiring, S.M.;  Staid, A.;  Reilly, A.C.;  Gao, M.]]></authors>
##     <affiliations><![CDATA[Johns Hopkins Univ., Baltimore, MD, USA]]></affiliations>
##     <controlledterms>
##       <term><![CDATA[emergency services]]></term>
##       <term><![CDATA[power system faults]]></term>
##       <term><![CDATA[power system planning]]></term>
##       <term><![CDATA[storms]]></term>
##     </controlledterms>
##     <thesaurusterms>
##       <term><![CDATA[Contingency planning]]></term>
##       <term><![CDATA[Emergency services]]></term>
##       <term><![CDATA[Hurricanes]]></term>
##       <term><![CDATA[Power outages]]></term>
##       <term><![CDATA[Power system restoration]]></term>
##       <term><![CDATA[Predictive models]]></term>
##       <term><![CDATA[Storms]]></term>
##     </thesaurusterms>
##     <pubtitle><![CDATA[Access, IEEE]]></pubtitle>
##     <punumber><![CDATA[6287639]]></punumber>
##     <pubtype><![CDATA[Journals & Magazines]]></pubtype>
##     <publisher><![CDATA[IEEE]]></publisher>
##     <volume><![CDATA[2]]></volume>
##     <py><![CDATA[2014]]></py>
##     <spage><![CDATA[1364]]></spage>
##     <epage><![CDATA[1373]]></epage>
##     <abstract><![CDATA[Hurricanes regularly cause widespread and prolonged power outages along the U.S. coastline. These power outages have significant impacts on other infrastructure dependent on electric power and on the population living in the impacted area. Efficient and effective emergency response planning within power utilities, other utilities dependent on electric power, private companies, and local, state, and federal government agencies benefit from accurate estimates of the extent and spatial distribution of power outages in advance of an approaching hurricane. A number of models have been developed for predicting power outages in advance of a hurricane, but these have been specific to a given utility service area, limiting their use to support wider emergency response planning. In this paper, we describe the development of a hurricane power outage prediction model applicable along the full U.S. coastline using only publicly available data, we demonstrate the use of the model for Hurricane Sandy, and we use the model to estimate what the impacts of a number of historic storms, including Typhoon Haiyan, would be on current U.S. energy infrastructure.]]></abstract>
##     <issn><![CDATA[2169-3536]]></issn>
##     <htmlFlag><![CDATA[1]]></htmlFlag>
##     <arnumber><![CDATA[6949604]]></arnumber>
##     <doi><![CDATA[10.1109/ACCESS.2014.2365716]]></doi>
##     <publicationId><![CDATA[6949604]]></publicationId>
##     <mdurl><![CDATA[http://ieeexplore.ieee.org/xpl/articleDetails.jsp?tp=&arnumber=6949604&contentType=Journals+%26+Magazines]]></mdurl>
##     <pdf><![CDATA[http://ieeexplore.ieee.org/stamp/stamp.jsp?arnumber=6949604]]></pdf>
##   </document>
##   <document>
##     <rank>566</rank>
##     <title><![CDATA[Catching Objects in Flight]]></title>
##     <authors><![CDATA[Seungsu Kim;  Shukla, A.;  Billard, A.]]></authors>
##     <affiliations><![CDATA[Swiss Fed. Inst. of Technol. Lausanne, Lausanne, Switzerland]]></affiliations>
##     <controlledterms>
##       <term><![CDATA[Gaussian processes]]></term>
##       <term><![CDATA[automatic programming]]></term>
##       <term><![CDATA[control engineering computing]]></term>
##       <term><![CDATA[humanoid robots]]></term>
##       <term><![CDATA[learning (artificial intelligence)]]></term>
##       <term><![CDATA[manipulator dynamics]]></term>
##       <term><![CDATA[motion control]]></term>
##       <term><![CDATA[redundant manipulators]]></term>
##       <term><![CDATA[robot programming]]></term>
##       <term><![CDATA[support vector machines]]></term>
##       <term><![CDATA[trajectory control]]></term>
##     </controlledterms>
##     <thesaurusterms>
##       <term><![CDATA[Aerospace electronics]]></term>
##       <term><![CDATA[Dynamics]]></term>
##       <term><![CDATA[Grasping]]></term>
##       <term><![CDATA[Robot kinematics]]></term>
##       <term><![CDATA[Robot sensing systems]]></term>
##       <term><![CDATA[Trajectory]]></term>
##     </thesaurusterms>
##     <pubtitle><![CDATA[Robotics, IEEE Transactions on]]></pubtitle>
##     <punumber><![CDATA[8860]]></punumber>
##     <pubtype><![CDATA[Journals & Magazines]]></pubtype>
##     <publisher><![CDATA[IEEE]]></publisher>
##     <volume><![CDATA[30]]></volume>
##     <issue><![CDATA[5]]></issue>
##     <py><![CDATA[2014]]></py>
##     <spage><![CDATA[1049]]></spage>
##     <epage><![CDATA[1065]]></epage>
##     <abstract><![CDATA[We address the difficult problem of catching in-flight objects with uneven shapes. This requires the solution of three complex problems: accurate prediction of the trajectory of fastmoving objects, predicting the feasible catching configuration, and planning the arm motion, and all within milliseconds. We follow a programming-by-demonstration approach in order to learn, from throwing examples, models of the object dynamics and arm movement. We propose a new methodology to find a feasible catching configuration in a probabilistic manner. We use the dynamical systems approach to encode motion from several demonstrations. This enables a rapid and reactive adaptation of the arm motion in the presence of sensor uncertainty. We validate the approach in simulation with the iCub humanoid robot and in real-world experiments with the KUKA LWR 4+ (7-degree-of-freedom arm robot) to catch a hammer, a tennis racket, an empty bottle, a partially filled bottle, and a cardboard box.]]></abstract>
##     <issn><![CDATA[1552-3098]]></issn>
##     <htmlFlag><![CDATA[1]]></htmlFlag>
##     <arnumber><![CDATA[6810147]]></arnumber>
##     <doi><![CDATA[10.1109/TRO.2014.2316022]]></doi>
##     <publicationId><![CDATA[6810147]]></publicationId>
##     <mdurl><![CDATA[http://ieeexplore.ieee.org/xpl/articleDetails.jsp?tp=&arnumber=6810147&contentType=Journals+%26+Magazines]]></mdurl>
##     <pdf><![CDATA[http://ieeexplore.ieee.org/stamp/stamp.jsp?arnumber=6810147]]></pdf>
##   </document>
##   <document>
##     <rank>567</rank>
##     <title><![CDATA[Toward QoI and Energy-Efficiency in Internet-of-Things Sensory Environments]]></title>
##     <authors><![CDATA[Liu, C.H.;  Fan, J.;  Branch, J.W.;  Leung, K.K.]]></authors>
##     <affiliations><![CDATA[Sch. of Software, Beijing Inst. of Technol., Beijing, China]]></affiliations>
##     <controlledterms>
##       <term><![CDATA[Internet of Things]]></term>
##       <term><![CDATA[energy conservation]]></term>
##       <term><![CDATA[telecommunication power management]]></term>
##       <term><![CDATA[wireless sensor networks]]></term>
##     </controlledterms>
##     <thesaurusterms>
##       <term><![CDATA[Energy consumption]]></term>
##       <term><![CDATA[Energy management]]></term>
##       <term><![CDATA[Internet of Things]]></term>
##       <term><![CDATA[Logic gates]]></term>
##       <term><![CDATA[Sensor fusion]]></term>
##       <term><![CDATA[Sensors]]></term>
##       <term><![CDATA[Wireless sensor networks]]></term>
##     </thesaurusterms>
##     <pubtitle><![CDATA[Emerging Topics in Computing, IEEE Transactions on]]></pubtitle>
##     <punumber><![CDATA[6245516]]></punumber>
##     <pubtype><![CDATA[Journals & Magazines]]></pubtype>
##     <publisher><![CDATA[IEEE]]></publisher>
##     <volume><![CDATA[2]]></volume>
##     <issue><![CDATA[4]]></issue>
##     <py><![CDATA[2014]]></py>
##     <spage><![CDATA[473]]></spage>
##     <epage><![CDATA[487]]></epage>
##     <abstract><![CDATA[Considering physical sensors with certain sensing capabilities in an Internet-of-Things (IoTs) sensory environment, in this paper, we propose an efficient energy management framework to control the duty cycles of these sensors under quality-of-information (QoI) expectations in a multitask-oriented environment. Contrary to past research efforts, our proposal is transparent and compatible both with the underlying low-layer protocols and diverse applications, and preserving energy-efficiency in the long run without sacrificing the QoI levels attained. In particular, we first introduce the novel concept of QoI-aware sensor-to-task relevancy to explicitly consider the sensing capabilities offered by a sensor to the IoT sensory environments, and QoI requirements required by a task. Second, we propose a novel concept of the critical covering set of any given task in selecting the sensors to service a task over time. Third, energy management decision is made dynamically at runtime, to reach the optimum for long-term application arrivals and departures under the constraint of their service delay. We show a case study to utilize sensors to perform environmental monitoring with a complete set of performance analysis. We further consider the signal propagation and processing latency into the proposal, and provide a thorough analysis on its impact on average measured delay probability.]]></abstract>
##     <issn><![CDATA[2168-6750]]></issn>
##     <htmlFlag><![CDATA[1]]></htmlFlag>
##     <arnumber><![CDATA[6935003]]></arnumber>
##     <doi><![CDATA[10.1109/TETC.2014.2364915]]></doi>
##     <publicationId><![CDATA[6935003]]></publicationId>
##     <mdurl><![CDATA[http://ieeexplore.ieee.org/xpl/articleDetails.jsp?tp=&arnumber=6935003&contentType=Journals+%26+Magazines]]></mdurl>
##     <pdf><![CDATA[http://ieeexplore.ieee.org/stamp/stamp.jsp?arnumber=6935003]]></pdf>
##   </document>
##   <document>
##     <rank>568</rank>
##     <title><![CDATA[Future Distribution Feeder Protection Using Directional Overcurrent Elements]]></title>
##     <authors><![CDATA[Jones, D.;  Kumm, J.J.]]></authors>
##     <affiliations><![CDATA[SCADA & Anal. Services, POWER Eng., Inc., Lakewood, CO, USA]]></affiliations>
##     <controlledterms>
##       <term><![CDATA[distributed power generation]]></term>
##       <term><![CDATA[overcurrent protection]]></term>
##       <term><![CDATA[power generation faults]]></term>
##       <term><![CDATA[power generation protection]]></term>
##       <term><![CDATA[power generation reliability]]></term>
##       <term><![CDATA[risk management]]></term>
##       <term><![CDATA[wind power plants]]></term>
##     </controlledterms>
##     <thesaurusterms>
##       <term><![CDATA[Circuit faults]]></term>
##       <term><![CDATA[Distributed power generation]]></term>
##       <term><![CDATA[Generators]]></term>
##       <term><![CDATA[Reactive power]]></term>
##       <term><![CDATA[Relays]]></term>
##       <term><![CDATA[Substations]]></term>
##       <term><![CDATA[Wind turbines]]></term>
##     </thesaurusterms>
##     <pubtitle><![CDATA[Industry Applications, IEEE Transactions on]]></pubtitle>
##     <punumber><![CDATA[28]]></punumber>
##     <pubtype><![CDATA[Journals & Magazines]]></pubtype>
##     <publisher><![CDATA[IEEE]]></publisher>
##     <volume><![CDATA[50]]></volume>
##     <issue><![CDATA[2]]></issue>
##     <py><![CDATA[2014]]></py>
##     <spage><![CDATA[1385]]></spage>
##     <epage><![CDATA[1390]]></epage>
##     <abstract><![CDATA[Distribution feeder protection could soon be complicated by nonradial flows of real and reactive power available from high penetration distributed generation and potentially from microgrids. Nondirectional overcurrent protection may not provide necessary security and sensitivity for faults on remote points of the circuit. Directional supervision is necessary to set overcurrent pickups with adequate sensitivity for remote faults. Setting the directional element by traditional means provides a reliability risk at varying VAR flows within reach of specific types of distributed generation. This paper will demonstrate the limitations of nondirectional overcurrent protection and the pitfalls of an improperly configured directional element. A unique solution using directional overcurrent elements further secured by a load encroachment function can solve these problems. This approach has been validated in renewable plant collector circuit protection applications over a wide range of operating conditions.]]></abstract>
##     <issn><![CDATA[0093-9994]]></issn>
##     <htmlFlag><![CDATA[1]]></htmlFlag>
##     <arnumber><![CDATA[6607207]]></arnumber>
##     <doi><![CDATA[10.1109/TIA.2013.2283237]]></doi>
##     <publicationId><![CDATA[6607207]]></publicationId>
##     <mdurl><![CDATA[http://ieeexplore.ieee.org/xpl/articleDetails.jsp?tp=&arnumber=6607207&contentType=Journals+%26+Magazines]]></mdurl>
##     <pdf><![CDATA[http://ieeexplore.ieee.org/stamp/stamp.jsp?arnumber=6607207]]></pdf>
##   </document>
##   <document>
##     <rank>569</rank>
##     <title><![CDATA[Enhancing Carrier Injection Using Graded Superlattice Electron Blocking Layer for UVB Light-Emitting Diodes]]></title>
##     <authors><![CDATA[Janjua, B.;  Tien Khee Ng;  Alyamani, A.Y.;  El-Desouki, M.M.;  Ooi, B.S.]]></authors>
##     <affiliations><![CDATA[Comput., Electr., & Math. Sci. & Eng. Div., King Abdullah Univ. of Sci. & Technol., Thuwal, Saudi Arabia]]></affiliations>
##     <controlledterms>
##       <term><![CDATA[III-V semiconductors]]></term>
##       <term><![CDATA[aluminium compounds]]></term>
##       <term><![CDATA[energy gap]]></term>
##       <term><![CDATA[gallium compounds]]></term>
##       <term><![CDATA[interface states]]></term>
##       <term><![CDATA[k.p calculations]]></term>
##       <term><![CDATA[light emitting diodes]]></term>
##       <term><![CDATA[semiconductor superlattices]]></term>
##       <term><![CDATA[wide band gap semiconductors]]></term>
##     </controlledterms>
##     <thesaurusterms>
##       <term><![CDATA[Aluminum gallium nitride]]></term>
##       <term><![CDATA[Charge carrier processes]]></term>
##       <term><![CDATA[Light emitting diodes]]></term>
##       <term><![CDATA[Materials]]></term>
##       <term><![CDATA[Photonic band gap]]></term>
##       <term><![CDATA[Radiative recombination]]></term>
##     </thesaurusterms>
##     <pubtitle><![CDATA[Photonics Journal, IEEE]]></pubtitle>
##     <punumber><![CDATA[4563994]]></punumber>
##     <pubtype><![CDATA[Journals & Magazines]]></pubtype>
##     <publisher><![CDATA[IEEE]]></publisher>
##     <volume><![CDATA[6]]></volume>
##     <issue><![CDATA[6]]></issue>
##     <py><![CDATA[2014]]></py>
##     <spage><![CDATA[1]]></spage>
##     <epage><![CDATA[12]]></epage>
##     <abstract><![CDATA[We have studied enhanced carrier injection by having an electron blocking layer (EBL) based on a graded superlattice (SL) design. Here, we examine, using a selfconsistent 6 &#x00D7; 6 k.p method, the energy band alignment diagrams under equilibrium and forward bias conditions while also considering carrier distribution and recombination rates (Shockley-Read-Hall, Auger, and radiative recombination rates). The graded SL is based on Al<sub>x</sub>Ga<sub>1-x</sub>N (larger bandgap) Al<sub>0:5</sub>Ga<sub>0:5</sub>N (smaller bandgap) SL, where x is changed from 0.8 to 0.56 in steps of 0.06. Graded SL was found to be effective in reducing electron leakage and enhancing hole injection into the active region. Due to our band engineering scheme for EBL, four orders-of-magnitude enhancement were observed in the direct recombination rate, as compared with the conventional bulk EBL consisting of Al<sub>0:8</sub>Ga<sub>0:2</sub>N. An increase in the spatial overlap of carrier wavefunction was obtained due to polarization-induced band bending in the active region. An efficient single quantum-well ultraviolet-B light-emitting diode was designed, which emits at 280 nm. This is the effective wavelength for water disinfection application, among others.]]></abstract>
##     <issn><![CDATA[1943-0655]]></issn>
##     <htmlFlag><![CDATA[1]]></htmlFlag>
##     <arnumber><![CDATA[6967698]]></arnumber>
##     <doi><![CDATA[10.1109/JPHOT.2014.2374596]]></doi>
##     <publicationId><![CDATA[6967698]]></publicationId>
##     <mdurl><![CDATA[http://ieeexplore.ieee.org/xpl/articleDetails.jsp?tp=&arnumber=6967698&contentType=Journals+%26+Magazines]]></mdurl>
##     <pdf><![CDATA[http://ieeexplore.ieee.org/stamp/stamp.jsp?arnumber=6967698]]></pdf>
##   </document>
##   <document>
##     <rank>570</rank>
##     <title><![CDATA[A SPAD-Based Photon Detecting System for Optical Communications]]></title>
##     <authors><![CDATA[Chitnis, D.;  Collins, S.]]></authors>
##     <affiliations><![CDATA[Dept. of Eng. Sci., Univ. of Oxford, Oxford, UK]]></affiliations>
##     <controlledterms>
##       <term><![CDATA[CMOS integrated circuits]]></term>
##       <term><![CDATA[avalanche photodiodes]]></term>
##       <term><![CDATA[error statistics]]></term>
##       <term><![CDATA[integrated optics]]></term>
##       <term><![CDATA[integrated optoelectronics]]></term>
##       <term><![CDATA[optical arrays]]></term>
##       <term><![CDATA[optical design techniques]]></term>
##       <term><![CDATA[optical fabrication]]></term>
##       <term><![CDATA[optical receivers]]></term>
##     </controlledterms>
##     <thesaurusterms>
##       <term><![CDATA[Arrays]]></term>
##       <term><![CDATA[Bit error rate]]></term>
##       <term><![CDATA[Histograms]]></term>
##       <term><![CDATA[Optical receivers]]></term>
##       <term><![CDATA[Optical transmitters]]></term>
##       <term><![CDATA[Photonics]]></term>
##     </thesaurusterms>
##     <pubtitle><![CDATA[Lightwave Technology, Journal of]]></pubtitle>
##     <punumber><![CDATA[50]]></punumber>
##     <pubtype><![CDATA[Journals & Magazines]]></pubtype>
##     <publisher><![CDATA[IEEE]]></publisher>
##     <volume><![CDATA[32]]></volume>
##     <issue><![CDATA[10]]></issue>
##     <py><![CDATA[2014]]></py>
##     <spage><![CDATA[2028]]></spage>
##     <epage><![CDATA[2034]]></epage>
##     <abstract><![CDATA[A small array of single photon avalanche detectors (SPADs) has been designed and fabricated in a standard 0.18 &#x03BC;m CMOS process to test a new photon detecting system for optical communications. First numerical results are presented which show that using arrays of SPADs reduces the optical power density required at the receiver. Experimental results then show that the new system preserves the photon counting ability of the SPADs. Finally a simple method is presented which can be used to estimate the size of array needed to achieve a particular target bit error rate at a specific optical power density. Together these results indicate that by replacing the avalanche photodiode in a receiver with the new system it will be possible to count the received photons.]]></abstract>
##     <issn><![CDATA[0733-8724]]></issn>
##     <htmlFlag><![CDATA[1]]></htmlFlag>
##     <arnumber><![CDATA[6786995]]></arnumber>
##     <doi><![CDATA[10.1109/JLT.2014.2316972]]></doi>
##     <publicationId><![CDATA[6786995]]></publicationId>
##     <mdurl><![CDATA[http://ieeexplore.ieee.org/xpl/articleDetails.jsp?tp=&arnumber=6786995&contentType=Journals+%26+Magazines]]></mdurl>
##     <pdf><![CDATA[http://ieeexplore.ieee.org/stamp/stamp.jsp?arnumber=6786995]]></pdf>
##   </document>
##   <document>
##     <rank>571</rank>
##     <title><![CDATA[Artificial Graphene and Related Photonic Lattices Generated With a Simple Method]]></title>
##     <authors><![CDATA[Yuanmei Gao;  Daohong Song;  Shanshan Chu;  Zhigang Chen]]></authors>
##     <affiliations><![CDATA[Coll. of Phys. & Electron., Shandong Normal Univ., Jinan, China]]></affiliations>
##     <controlledterms>
##       <term><![CDATA[Brillouin spectra]]></term>
##       <term><![CDATA[Fourier transform optics]]></term>
##       <term><![CDATA[graphene]]></term>
##       <term><![CDATA[molybdenum compounds]]></term>
##       <term><![CDATA[optical fabrication]]></term>
##       <term><![CDATA[optical lattices]]></term>
##       <term><![CDATA[optical vortices]]></term>
##       <term><![CDATA[photorefractive materials]]></term>
##     </controlledterms>
##     <thesaurusterms>
##       <term><![CDATA[Glass]]></term>
##       <term><![CDATA[Lattices]]></term>
##       <term><![CDATA[Lenses]]></term>
##       <term><![CDATA[Nonlinear optics]]></term>
##       <term><![CDATA[Optical diffraction]]></term>
##       <term><![CDATA[Optical vortices]]></term>
##       <term><![CDATA[Photonics]]></term>
##     </thesaurusterms>
##     <pubtitle><![CDATA[Photonics Journal, IEEE]]></pubtitle>
##     <punumber><![CDATA[4563994]]></punumber>
##     <pubtype><![CDATA[Journals & Magazines]]></pubtype>
##     <publisher><![CDATA[IEEE]]></publisher>
##     <volume><![CDATA[6]]></volume>
##     <issue><![CDATA[6]]></issue>
##     <py><![CDATA[2014]]></py>
##     <spage><![CDATA[1]]></spage>
##     <epage><![CDATA[6]]></epage>
##     <abstract><![CDATA[We fabricate honeycomb and related photonic lattices in a nonlinear crystal with a simple method that is based on the optical Fourier transformation through an amplitude mask (six-hole aperture) superimposed with a phase mask (three tilted glass plates). Compared with using the spatial light modulator, our method is cost-effective and easy to control for almost every one. Numerically, we use the transmittance function to describe the amplitude mask instead of treating each hole as a simple point source and give out the field distribution function of the honeycomb lattice beam. Experimentally, the induced lattice structure is examined by the Brillouin zone spectroscopy and the far-field diffraction pattern, as well as by monitoring the linear and nonlinear propagation of a probe beam. In addition to the honeycomb, vortex, and Kagome lattices, we illustrate the phase conditions for optical induction of molybdenum disulfide-like photonic lattices for the first time. Our approach can be easily extended to generate more complex microstructures by designing the amplitude and phase mask properly, promising a convenient way to establish a photonic platform for various applications.]]></abstract>
##     <issn><![CDATA[1943-0655]]></issn>
##     <htmlFlag><![CDATA[1]]></htmlFlag>
##     <arnumber><![CDATA[6923444]]></arnumber>
##     <doi><![CDATA[10.1109/JPHOT.2014.2363436]]></doi>
##     <publicationId><![CDATA[6923444]]></publicationId>
##     <mdurl><![CDATA[http://ieeexplore.ieee.org/xpl/articleDetails.jsp?tp=&arnumber=6923444&contentType=Journals+%26+Magazines]]></mdurl>
##     <pdf><![CDATA[http://ieeexplore.ieee.org/stamp/stamp.jsp?arnumber=6923444]]></pdf>
##   </document>
##   <document>
##     <rank>572</rank>
##     <title><![CDATA[Picosecond Resolution Time-to-Digital Converter Using <formula formulatype="inline"> <img src="/images/tex/21450.gif" alt="{{\rm G}_{\rm m}} \hbox {-C}"> </formula> Integrator and SAR-ADC]]></title>
##     <authors><![CDATA[Zule Xu;  Miyahara, M.;  Matsuzawa, A.]]></authors>
##     <affiliations><![CDATA[Tokyo Inst. of Technol., Tokyo, Japan]]></affiliations>
##     <controlledterms>
##       <term><![CDATA[CMOS digital integrated circuits]]></term>
##       <term><![CDATA[integrating circuits]]></term>
##       <term><![CDATA[time-digital conversion]]></term>
##     </controlledterms>
##     <thesaurusterms>
##       <term><![CDATA[Calibration]]></term>
##       <term><![CDATA[Capacitors]]></term>
##       <term><![CDATA[Computer architecture]]></term>
##       <term><![CDATA[Delays]]></term>
##       <term><![CDATA[Noise]]></term>
##       <term><![CDATA[Switches]]></term>
##     </thesaurusterms>
##     <pubtitle><![CDATA[Nuclear Science, IEEE Transactions on]]></pubtitle>
##     <punumber><![CDATA[23]]></punumber>
##     <pubtype><![CDATA[Journals & Magazines]]></pubtype>
##     <publisher><![CDATA[IEEE]]></publisher>
##     <volume><![CDATA[61]]></volume>
##     <issue><![CDATA[2]]></issue>
##     <py><![CDATA[2014]]></py>
##     <spage><![CDATA[852]]></spage>
##     <epage><![CDATA[859]]></epage>
##     <abstract><![CDATA[A picosecond resolution time-to-digital converter (TDC) is presented. The resolution of a conventional delay chain TDC is limited by the delay of a logic buffer. Various types of recent TDCs are successful in breaking this limitation, but they require a significant calibration effort to achieve picosecond resolution with a sufficient linear range. To address these issues, we propose a simple method to break the resolution limitation without any calibration: a G<sub>m</sub>-C integrator followed by a successive approximation register analog-to-digital converter (SAR-ADC). This translates the time interval into charge, and then the charge is quantized. A prototype chip was fabricated in 90 nm CMOS. The measurement results reveal a 1 ps resolution, a -0.6/0.7 LSB differential nonlinearity (DNL), a -1.1/2.3 LSB integral nonlinearity (INL), and a 9-bit range. The measured 11.74 ps single-shot precision is caused by the noise of the integrator. We analyze the noise of the integrator and propose an improved front-end circuit to reduce this noise. The proposal is verified by simulations showing the maximum single-shot precision is less than 1 ps. The proposed front-end circuit can also diminish the mismatch effects.]]></abstract>
##     <issn><![CDATA[0018-9499]]></issn>
##     <htmlFlag><![CDATA[1]]></htmlFlag>
##     <arnumber><![CDATA[6782441]]></arnumber>
##     <doi><![CDATA[10.1109/TNS.2014.2309652]]></doi>
##     <publicationId><![CDATA[6782441]]></publicationId>
##     <mdurl><![CDATA[http://ieeexplore.ieee.org/xpl/articleDetails.jsp?tp=&arnumber=6782441&contentType=Journals+%26+Magazines]]></mdurl>
##     <pdf><![CDATA[http://ieeexplore.ieee.org/stamp/stamp.jsp?arnumber=6782441]]></pdf>
##   </document>
##   <document>
##     <rank>573</rank>
##     <title><![CDATA[Long-Range High Spatial Resolution Distributed Temperature and Strain Sensing Based on Optical Frequency-Domain Reflectometry]]></title>
##     <authors><![CDATA[Jia Song;  Wenhai Li;  Ping Lu;  Yanping Xu;  Liang Chen;  Xiaoyi Bao]]></authors>
##     <affiliations><![CDATA[Dept. of Phys., Univ. of Ottawa, Ottawa, ON, Canada]]></affiliations>
##     <controlledterms>
##       <term><![CDATA[fibre optic sensors]]></term>
##       <term><![CDATA[optical tuning]]></term>
##       <term><![CDATA[reflectometry]]></term>
##       <term><![CDATA[strain sensors]]></term>
##       <term><![CDATA[temperature sensors]]></term>
##     </controlledterms>
##     <thesaurusterms>
##       <term><![CDATA[Interpolation]]></term>
##       <term><![CDATA[Optical fibers]]></term>
##       <term><![CDATA[Optical interferometry]]></term>
##       <term><![CDATA[Sensors]]></term>
##       <term><![CDATA[Spatial resolution]]></term>
##       <term><![CDATA[Strain]]></term>
##       <term><![CDATA[Temperature measurement]]></term>
##     </thesaurusterms>
##     <pubtitle><![CDATA[Photonics Journal, IEEE]]></pubtitle>
##     <punumber><![CDATA[4563994]]></punumber>
##     <pubtype><![CDATA[Journals & Magazines]]></pubtype>
##     <publisher><![CDATA[IEEE]]></publisher>
##     <volume><![CDATA[6]]></volume>
##     <issue><![CDATA[3]]></issue>
##     <py><![CDATA[2014]]></py>
##     <spage><![CDATA[1]]></spage>
##     <epage><![CDATA[8]]></epage>
##     <abstract><![CDATA[A novel approach to realize long-range distributed temperature and strain measurement with high spatial resolution, as well as high temperature and strain resolution, is proposed based on optical frequency-domain reflectometry (OFDR). To maintain the high measurement resolution and accuracy while increasing the sensing length, an optimized nonlinearity compensation algorithm is implemented to ensure a large wavelength tuning range. The compensated OFDR trace exhibits improved sensing resolution at a short distance, and the spatial resolution gradually deteriorates at the far end due to accumulated phase noise induced by fast tuning of the laser wavelength. We demonstrated the spatial resolution of 0.3 mm over a single-mode fiber sensing length of over 300 m, and temperature and strain resolution of 0.7 <sup>&#x00B0;</sup>C and 2.3 &#x03BC;&#x03B5; with spatial resolution of up to 7 cm, respectively.]]></abstract>
##     <issn><![CDATA[1943-0655]]></issn>
##     <htmlFlag><![CDATA[1]]></htmlFlag>
##     <arnumber><![CDATA[6807679]]></arnumber>
##     <doi><![CDATA[10.1109/JPHOT.2014.2320742]]></doi>
##     <publicationId><![CDATA[6807679]]></publicationId>
##     <mdurl><![CDATA[http://ieeexplore.ieee.org/xpl/articleDetails.jsp?tp=&arnumber=6807679&contentType=Journals+%26+Magazines]]></mdurl>
##     <pdf><![CDATA[http://ieeexplore.ieee.org/stamp/stamp.jsp?arnumber=6807679]]></pdf>
##   </document>
##   <document>
##     <rank>574</rank>
##     <title><![CDATA[A Wideband 2<formula formulatype="inline"> <img src="/images/tex/326.gif" alt="\times"> </formula>13-bit All-Digital I/Q RF-DAC]]></title>
##     <authors><![CDATA[Alavi, M.S.;  Staszewski, R.B.;  de Vreede, L.C.N.;  Long, J.R.]]></authors>
##     <affiliations><![CDATA[Dept. of Microelectron., Delft Univ. of Technol., Delft, Netherlands]]></affiliations>
##     <controlledterms>
##       <term><![CDATA[CMOS digital integrated circuits]]></term>
##       <term><![CDATA[OFDM modulation]]></term>
##       <term><![CDATA[UHF integrated circuits]]></term>
##       <term><![CDATA[digital-analogue conversion]]></term>
##       <term><![CDATA[quadrature amplitude modulation]]></term>
##     </controlledterms>
##     <thesaurusterms>
##       <term><![CDATA[Baseband]]></term>
##       <term><![CDATA[Clocks]]></term>
##       <term><![CDATA[Modulation]]></term>
##       <term><![CDATA[Noise]]></term>
##       <term><![CDATA[Power generation]]></term>
##       <term><![CDATA[Radio frequency]]></term>
##       <term><![CDATA[Switches]]></term>
##     </thesaurusterms>
##     <pubtitle><![CDATA[Microwave Theory and Techniques, IEEE Transactions on]]></pubtitle>
##     <punumber><![CDATA[22]]></punumber>
##     <pubtype><![CDATA[Journals & Magazines]]></pubtype>
##     <publisher><![CDATA[IEEE]]></publisher>
##     <volume><![CDATA[62]]></volume>
##     <issue><![CDATA[4]]></issue>
##     <part><![CDATA[1]]></part>
##     <py><![CDATA[2014]]></py>
##     <spage><![CDATA[732]]></spage>
##     <epage><![CDATA[752]]></epage>
##     <abstract><![CDATA[This paper presents a wideband 2 &#x00D7;13-bit in-phase/quadrature-phase (I/Q) RF digital-to-analog converter-based all-digital modulator realized in 65-nm CMOS. The isolation between I and Q paths is guaranteed employing 25% duty-cycle differential quadrature clocks. With a 1.3-V supply and an on-chip power combiner, the digital I/Q transmitter provides more than 21-dBm RF output power within a frequency range of 1.36-2.51 GHz. The peak RF output power, overall system, and drain efficiencies of the modulator are 22.8 dBm, 34%, and 42%, respectively. The measured static noise floor is below -160 dBc/Hz. The digital I/Q RF modulator demonstrates an IQ image rejection and local oscillator leakage of -65 and -68 dBc, respectively. It could be linearized using either of the two digital predistortion (DPD) approaches: a memoryless polynomial or a lookup table. Its linearity is examined using single-carrier 4/16/64/256/1024 quadrature amplitude modulation (QAM), as well as multi-carrier 256-QAM orthogonal frequency-division multiplexing baseband signals while their related modulation bandwidth can be as high as 154 MHz. Employing DPD improves the third-order intermodulation product (IM<sub>3</sub>) by more than 25 dB, while the measured error vector magnitude for a &#x201C;single-carrier 22-MHz 64-QAM&#x201D; signal is better than -28 dB.]]></abstract>
##     <issn><![CDATA[0018-9480]]></issn>
##     <htmlFlag><![CDATA[1]]></htmlFlag>
##     <arnumber><![CDATA[6762810]]></arnumber>
##     <doi><![CDATA[10.1109/TMTT.2014.2307876]]></doi>
##     <publicationId><![CDATA[6762810]]></publicationId>
##     <mdurl><![CDATA[http://ieeexplore.ieee.org/xpl/articleDetails.jsp?tp=&arnumber=6762810&contentType=Journals+%26+Magazines]]></mdurl>
##     <pdf><![CDATA[http://ieeexplore.ieee.org/stamp/stamp.jsp?arnumber=6762810]]></pdf>
##   </document>
##   <document>
##     <rank>575</rank>
##     <title><![CDATA[M/M/1 Multiple Vacation Queueing Systems With Differentiated Vacations and Vacation Interruptions]]></title>
##     <authors><![CDATA[Isijola-Adakeja, O.A.;  Ibe, O.C.]]></authors>
##     <affiliations><![CDATA[Dept. of Electr. & Comput. Eng., Univ. of Massachusetts Lowell, Lowell, MA, USA]]></affiliations>
##     <controlledterms>
##       <term><![CDATA[queueing theory]]></term>
##     </controlledterms>
##     <thesaurusterms>
##       <term><![CDATA[Maintenance engineering]]></term>
##       <term><![CDATA[Power system reliability]]></term>
##       <term><![CDATA[Queueing analysis]]></term>
##       <term><![CDATA[Servers]]></term>
##     </thesaurusterms>
##     <pubtitle><![CDATA[Access, IEEE]]></pubtitle>
##     <punumber><![CDATA[6287639]]></punumber>
##     <pubtype><![CDATA[Journals & Magazines]]></pubtype>
##     <publisher><![CDATA[IEEE]]></publisher>
##     <volume><![CDATA[2]]></volume>
##     <py><![CDATA[2014]]></py>
##     <spage><![CDATA[1384]]></spage>
##     <epage><![CDATA[1395]]></epage>
##     <abstract><![CDATA[We consider an M/M/1 multiple vacation queueing system with two types of server vacations. Type 1 vacation is taken after the server has exhaustively served all the customers in the system, where the number of customers served is at least one. Type 2 vacation is taken when the server returns from a vacation and finds no customer waiting. Each type of vacation can be interrupted when the number of customers in the system reaches two predefined thresholds, where each vacation type has a different threshold. It is assumed that service times and vacation durations are exponentially distributed with different means. We present a steady-state solution of the system under two vacation interruption policies.]]></abstract>
##     <issn><![CDATA[2169-3536]]></issn>
##     <htmlFlag><![CDATA[1]]></htmlFlag>
##     <arnumber><![CDATA[6960843]]></arnumber>
##     <doi><![CDATA[10.1109/ACCESS.2014.2372671]]></doi>
##     <publicationId><![CDATA[6960843]]></publicationId>
##     <mdurl><![CDATA[http://ieeexplore.ieee.org/xpl/articleDetails.jsp?tp=&arnumber=6960843&contentType=Journals+%26+Magazines]]></mdurl>
##     <pdf><![CDATA[http://ieeexplore.ieee.org/stamp/stamp.jsp?arnumber=6960843]]></pdf>
##   </document>
##   <document>
##     <rank>576</rank>
##     <title><![CDATA[Bright and Dark Square Pulses Generated From a Graphene-Oxide Mode-Locked Ytterbium-Doped Fiber Laser]]></title>
##     <authors><![CDATA[Rong-yong Lin;  Yong-gang Wang;  Pei-guang Yan;  Ge-lin Zhang;  Jun-qing Zhao;  Hui-quan Li;  Shi-sheng Huang;  Guang-Zhong Cao;  Ji-An Duan]]></authors>
##     <affiliations><![CDATA[Shenzhen Key Lab. of Laser Eng., Shenzhen Univ., Shenzhen, China]]></affiliations>
##     <controlledterms>
##       <term><![CDATA[fibre lasers]]></term>
##       <term><![CDATA[graphene]]></term>
##       <term><![CDATA[laser cavity resonators]]></term>
##       <term><![CDATA[laser mode locking]]></term>
##       <term><![CDATA[optical pulse generation]]></term>
##       <term><![CDATA[optical saturable absorption]]></term>
##       <term><![CDATA[ytterbium]]></term>
##     </controlledterms>
##     <thesaurusterms>
##       <term><![CDATA[Cavity resonators]]></term>
##       <term><![CDATA[Fiber lasers]]></term>
##       <term><![CDATA[Laser mode locking]]></term>
##       <term><![CDATA[Laser theory]]></term>
##       <term><![CDATA[Optical fiber communication]]></term>
##       <term><![CDATA[Optical fiber dispersion]]></term>
##       <term><![CDATA[Optical fiber polarization]]></term>
##     </thesaurusterms>
##     <pubtitle><![CDATA[Photonics Journal, IEEE]]></pubtitle>
##     <punumber><![CDATA[4563994]]></punumber>
##     <pubtype><![CDATA[Journals & Magazines]]></pubtype>
##     <publisher><![CDATA[IEEE]]></publisher>
##     <volume><![CDATA[6]]></volume>
##     <issue><![CDATA[3]]></issue>
##     <py><![CDATA[2014]]></py>
##     <spage><![CDATA[1]]></spage>
##     <epage><![CDATA[8]]></epage>
##     <abstract><![CDATA[The observation of the bright pulses and dark square pulses in a graphene-oxide saturable absorber (GOSA) passively mode-locked ytterbium-doped fiber laser has been investigated experimentally. Bright pulses are achieved at a pump power of ~ 200 mW. However, the dark-square-pulse generation starts at a much higher pump power of ~ 450 mW. At the maximum pump power of 600 mW, the dark-square-pulse bunches and harmonic mode locking (HML) can be also obtained by tuning the polarization controller (PC) to different orientations. It is the first demonstration of the bunches and HML of dark square pulses in a GOSA passively mode-locked ytterbium-doped fiber laser with large normal dispersion cavity.]]></abstract>
##     <issn><![CDATA[1943-0655]]></issn>
##     <htmlFlag><![CDATA[1]]></htmlFlag>
##     <arnumber><![CDATA[6807644]]></arnumber>
##     <doi><![CDATA[10.1109/JPHOT.2014.2319099]]></doi>
##     <publicationId><![CDATA[6807644]]></publicationId>
##     <mdurl><![CDATA[http://ieeexplore.ieee.org/xpl/articleDetails.jsp?tp=&arnumber=6807644&contentType=Journals+%26+Magazines]]></mdurl>
##     <pdf><![CDATA[http://ieeexplore.ieee.org/stamp/stamp.jsp?arnumber=6807644]]></pdf>
##   </document>
##   <document>
##     <rank>577</rank>
##     <title><![CDATA[Stereo, Shading, and Surfaces: Curvature Constraints Couple Neural Computations]]></title>
##     <authors><![CDATA[Zucker, S.W.]]></authors>
##     <affiliations><![CDATA[Depts. of Comput. Sci. & Biomed. Eng., Yale Univ., New Haven, CT, USA]]></affiliations>
##     <controlledterms>
##       <term><![CDATA[biomedical optical imaging]]></term>
##       <term><![CDATA[brain]]></term>
##       <term><![CDATA[edge detection]]></term>
##       <term><![CDATA[learning (artificial intelligence)]]></term>
##       <term><![CDATA[medical image processing]]></term>
##       <term><![CDATA[statistical analysis]]></term>
##       <term><![CDATA[stereo image processing]]></term>
##       <term><![CDATA[vision defects]]></term>
##     </controlledterms>
##     <thesaurusterms>
##       <term><![CDATA[Boundary conditions]]></term>
##       <term><![CDATA[Computational modeling]]></term>
##       <term><![CDATA[Feedforward neural networks]]></term>
##       <term><![CDATA[Image edge detection]]></term>
##       <term><![CDATA[Neurons]]></term>
##       <term><![CDATA[Neuroscience]]></term>
##       <term><![CDATA[Probability]]></term>
##       <term><![CDATA[Visualization]]></term>
##     </thesaurusterms>
##     <pubtitle><![CDATA[Proceedings of the IEEE]]></pubtitle>
##     <punumber><![CDATA[5]]></punumber>
##     <pubtype><![CDATA[Journals & Magazines]]></pubtype>
##     <publisher><![CDATA[IEEE]]></publisher>
##     <volume><![CDATA[102]]></volume>
##     <issue><![CDATA[5]]></issue>
##     <py><![CDATA[2014]]></py>
##     <spage><![CDATA[812]]></spage>
##     <epage><![CDATA[829]]></epage>
##     <abstract><![CDATA[Vision problems are inherently ambiguous: Do abrupt brightness changes correspond to object boundaries? Are smooth intensity changes due to shading or material properties? For stereo: Which point in the left image corresponds to which point in the right one? What is the role of color in visual information processing? To answer these (seemingly different) questions we develop an analogy between the role of orientation in organizing visual cortex and tangents in differential geometry. Machine learning experiments suggest using geometry as a surrogate for high-order statistical interactions. The cortical columnar architecture becomes a bundle structure in geometry. Connection forms within these bundles suggest answers to the above questions, and curvatures emerge in key roles. More generally, our path through these questions suggests an overall strategy for solving the inverse problems of vision: decompose the global problems into networks of smaller ones and then seek constraints from these coupled problems to reduce ambiguity. Neural computations thus amount to satisfying constraints rather than seeking uniform approximations. Even when no global formulation exists one may be able to find localized structures on which ambiguity is minimal; these can then anchor an overall approximation.]]></abstract>
##     <issn><![CDATA[0018-9219]]></issn>
##     <arnumber><![CDATA[6804761]]></arnumber>
##     <doi><![CDATA[10.1109/JPROC.2014.2314723]]></doi>
##     <publicationId><![CDATA[6804761]]></publicationId>
##     <mdurl><![CDATA[http://ieeexplore.ieee.org/xpl/articleDetails.jsp?tp=&arnumber=6804761&contentType=Journals+%26+Magazines]]></mdurl>
##     <pdf><![CDATA[http://ieeexplore.ieee.org/stamp/stamp.jsp?arnumber=6804761]]></pdf>
##   </document>
##   <document>
##     <rank>578</rank>
##     <title><![CDATA[Optical Modeling-Assisted Characterization of Polymer:Fullerene Photodiodes]]></title>
##     <authors><![CDATA[Jun Hyung Jo;  Yong-Sik Chun;  Ilgu Kim]]></authors>
##     <affiliations><![CDATA[Sch. of Inf. & Commun. Technol., Griffith Univ., Griffith, QLD, Australia]]></affiliations>
##     <controlledterms>
##       <term><![CDATA[dark conductivity]]></term>
##       <term><![CDATA[fullerene devices]]></term>
##       <term><![CDATA[integrated optoelectronics]]></term>
##       <term><![CDATA[optical polymers]]></term>
##       <term><![CDATA[organic semiconductors]]></term>
##       <term><![CDATA[photodiodes]]></term>
##       <term><![CDATA[semiconductor heterojunctions]]></term>
##     </controlledterms>
##     <thesaurusterms>
##       <term><![CDATA[Current measurement]]></term>
##       <term><![CDATA[Dark current]]></term>
##       <term><![CDATA[Optical imaging]]></term>
##       <term><![CDATA[Optical polymers]]></term>
##       <term><![CDATA[Photodiodes]]></term>
##     </thesaurusterms>
##     <pubtitle><![CDATA[Photonics Journal, IEEE]]></pubtitle>
##     <punumber><![CDATA[4563994]]></punumber>
##     <pubtype><![CDATA[Journals & Magazines]]></pubtype>
##     <publisher><![CDATA[IEEE]]></publisher>
##     <volume><![CDATA[6]]></volume>
##     <issue><![CDATA[4]]></issue>
##     <py><![CDATA[2014]]></py>
##     <spage><![CDATA[1]]></spage>
##     <epage><![CDATA[7]]></epage>
##     <abstract><![CDATA[In this paper, the practical viability of an organic bulk heterojunction (BHJ)based photodiode is studied, including the analysis of dark current density (J<sub>d</sub> in A/cm<sup>2</sup>), external quantum efficiency (EQE in percent), responsivity (R in A/W), noise-equivalent power (in W/Hz<sup>1/2</sup>), and specific detectivity (Jones in cmHz<sup>1/2</sup>/W). The dark current was minimized down to 90 pA/cm<sup>2</sup> in P3HT : PC<sub>60</sub>BM BHJ photodiodes by increasing the thickness, whereas the EQE maintained high values. The measured noise current was 2:20 &#x03C7; 10<sup>-13</sup>, 1:13 &#x03C7; 10<sup>-13</sup>, and 1:94 &#x03C7; 10<sup>-14</sup> A/Hz<sup>1/2</sup> for 45-, 55-, and 65-nm BHJ photodiodes, respectively. With those values, the calculated detectivity obtained, given light of a 550 nm wavelength, was 2:55 &#x03C7; 10<sup>11</sup>, 5:93 &#x03C7; 10<sup>11</sup>, and 4:16 &#x03C7; 10<sup>12</sup> Jones, respectively. The results demonstrated a performance of polymer:fullerence photodiode near equivalent to that of Si-based photodiodes.]]></abstract>
##     <issn><![CDATA[1943-0655]]></issn>
##     <htmlFlag><![CDATA[1]]></htmlFlag>
##     <arnumber><![CDATA[6872783]]></arnumber>
##     <doi><![CDATA[10.1109/JPHOT.2014.2345881]]></doi>
##     <publicationId><![CDATA[6872783]]></publicationId>
##     <mdurl><![CDATA[http://ieeexplore.ieee.org/xpl/articleDetails.jsp?tp=&arnumber=6872783&contentType=Journals+%26+Magazines]]></mdurl>
##     <pdf><![CDATA[http://ieeexplore.ieee.org/stamp/stamp.jsp?arnumber=6872783]]></pdf>
##   </document>
##   <document>
##     <rank>579</rank>
##     <title><![CDATA[Pixel Level Characterization of Pinned Photodiode and Transfer Gate Physical Parameters in CMOS Image Sensors]]></title>
##     <authors><![CDATA[Goiffon, V.;  Estribeau, M.;  Michelot, J.;  Cervantes, P.;  Pelamatti, A.;  Marcelot, O.;  Magnan, P.]]></authors>
##     <affiliations><![CDATA[Inst. Super. de l'Aeronautique et de l'Espace (ISAE), Univ. de Toulouse, Toulouse, France]]></affiliations>
##     <controlledterms>
##       <term><![CDATA[CMOS image sensors]]></term>
##       <term><![CDATA[photodetectors]]></term>
##       <term><![CDATA[photodiodes]]></term>
##       <term><![CDATA[sensor arrays]]></term>
##     </controlledterms>
##     <thesaurusterms>
##       <term><![CDATA[Capacitance]]></term>
##       <term><![CDATA[Electric potential]]></term>
##       <term><![CDATA[Logic gates]]></term>
##       <term><![CDATA[Photodiodes]]></term>
##       <term><![CDATA[Thermionic emission]]></term>
##       <term><![CDATA[Threshold voltage]]></term>
##       <term><![CDATA[Voltage measurement]]></term>
##     </thesaurusterms>
##     <pubtitle><![CDATA[Electron Devices Society, IEEE Journal of the]]></pubtitle>
##     <punumber><![CDATA[6245494]]></punumber>
##     <pubtype><![CDATA[Journals & Magazines]]></pubtype>
##     <publisher><![CDATA[IEEE]]></publisher>
##     <volume><![CDATA[2]]></volume>
##     <issue><![CDATA[4]]></issue>
##     <py><![CDATA[2014]]></py>
##     <spage><![CDATA[65]]></spage>
##     <epage><![CDATA[76]]></epage>
##     <abstract><![CDATA[A method to extract the pinned photodiode (PPD) physical parameters inside a CMOS image sensor pixel array is presented. The proposed technique is based on the Tan et al. pinning voltage characteristic. This pixel device characterization can be performed directly at the solid-state circuit output without the need of any external test structure. The presented study analyzes the different injection mechanisms involved in the different regimes of the characteristic. It is demonstrated that in addition to the pinning voltage, this fast measurement can be used to retrieve the PPD capacitance, the pixel equilibrium full well capacity, and both the transfer gate threshold voltage and its channel potential at a given gate voltage. An alternative approach is also proposed to extract an objective pinning voltage value from this measurement.]]></abstract>
##     <issn><![CDATA[2168-6734]]></issn>
##     <htmlFlag><![CDATA[1]]></htmlFlag>
##     <arnumber><![CDATA[6819767]]></arnumber>
##     <doi><![CDATA[10.1109/JEDS.2014.2326299]]></doi>
##     <publicationId><![CDATA[6819767]]></publicationId>
##     <mdurl><![CDATA[http://ieeexplore.ieee.org/xpl/articleDetails.jsp?tp=&arnumber=6819767&contentType=Journals+%26+Magazines]]></mdurl>
##     <pdf><![CDATA[http://ieeexplore.ieee.org/stamp/stamp.jsp?arnumber=6819767]]></pdf>
##   </document>
##   <document>
##     <rank>580</rank>
##     <title><![CDATA[A Geometric Approach to Head/Eye Control]]></title>
##     <authors><![CDATA[Ghosh, B.K.;  Wijayasinghe, I.B.;  Kahagalage, S.D.]]></authors>
##     <affiliations><![CDATA[Dept. of Math. & Stat., Texas Tech Univ., Lubbock, TX, USA]]></affiliations>
##     <controlledterms>
##       <term><![CDATA[angular velocity control]]></term>
##       <term><![CDATA[eye]]></term>
##       <term><![CDATA[geometry]]></term>
##       <term><![CDATA[motion control]]></term>
##       <term><![CDATA[optimal control]]></term>
##       <term><![CDATA[torque control]]></term>
##       <term><![CDATA[vectors]]></term>
##     </controlledterms>
##     <thesaurusterms>
##       <term><![CDATA[Equations]]></term>
##       <term><![CDATA[Eyes]]></term>
##       <term><![CDATA[Head control]]></term>
##       <term><![CDATA[Mathematical model]]></term>
##       <term><![CDATA[Motion analysis]]></term>
##       <term><![CDATA[Optimal control]]></term>
##       <term><![CDATA[Quaterernions]]></term>
##     </thesaurusterms>
##     <pubtitle><![CDATA[Access, IEEE]]></pubtitle>
##     <punumber><![CDATA[6287639]]></punumber>
##     <pubtype><![CDATA[Journals & Magazines]]></pubtype>
##     <publisher><![CDATA[IEEE]]></publisher>
##     <volume><![CDATA[2]]></volume>
##     <py><![CDATA[2014]]></py>
##     <spage><![CDATA[316]]></spage>
##     <epage><![CDATA[332]]></epage>
##     <abstract><![CDATA[In this paper, we study control problems that can be directly applied to controlling the rotational motion of eye and head. We model eye and head as a sphere, or ellipsoid, rotating about its center, or about its south pole, where the axes of rotation are physiologically constrained, as was proposed originally by Listing and Donders. The Donders' constraint is either derived from Fick gimbals or from observed rotation data of adult human head. The movement dynamics is derived on SO(3) or on a suitable submanifold of SO(3) after describing a Lagrangian. Using two forms of parametrization, the axis-angle and Tait-Bryan, the motion dynamics is described as an Euler-Lagrange's equation, which is written together with an externally applied control torque. Using the control system, so obtained, we propose a class of optimal control problem that minimizes the norm of the applied external torque vector. Our control objective is to point the eye or head, toward a stationary point target, also called the regulation problem. The optimal control problem has also been analyzed by writing the dynamical system as a Newton-Euler's equation using angular velocity as part of the state variables. In this approach, explicit parametrization of SO(3) is not required. Finally, in the appendix, we describe a recently introduced potential control problem to address the regulation problem.]]></abstract>
##     <issn><![CDATA[2169-3536]]></issn>
##     <htmlFlag><![CDATA[1]]></htmlFlag>
##     <arnumber><![CDATA[6782433]]></arnumber>
##     <doi><![CDATA[10.1109/ACCESS.2014.2315523]]></doi>
##     <publicationId><![CDATA[6782433]]></publicationId>
##     <mdurl><![CDATA[http://ieeexplore.ieee.org/xpl/articleDetails.jsp?tp=&arnumber=6782433&contentType=Journals+%26+Magazines]]></mdurl>
##     <pdf><![CDATA[http://ieeexplore.ieee.org/stamp/stamp.jsp?arnumber=6782433]]></pdf>
##   </document>
##   <document>
##     <rank>581</rank>
##     <title><![CDATA[Sart-Type Half-Threshold Filtering Approach for CT Reconstruction]]></title>
##     <authors><![CDATA[Hengyong Yu;  Ge Wang]]></authors>
##     <affiliations><![CDATA[Dept. of Biomed. Eng., Wake Forest Univ. Health Sci., Winston-Salem, NC, USA]]></affiliations>
##     <controlledterms>
##       <term><![CDATA[algebra]]></term>
##       <term><![CDATA[computerised tomography]]></term>
##       <term><![CDATA[filtering theory]]></term>
##       <term><![CDATA[image enhancement]]></term>
##       <term><![CDATA[image reconstruction]]></term>
##       <term><![CDATA[image segmentation]]></term>
##       <term><![CDATA[inverse transforms]]></term>
##       <term><![CDATA[iterative methods]]></term>
##       <term><![CDATA[medical image processing]]></term>
##       <term><![CDATA[minimisation]]></term>
##     </controlledterms>
##     <thesaurusterms>
##       <term><![CDATA[Biomedical measurement]]></term>
##       <term><![CDATA[Computed tomography]]></term>
##       <term><![CDATA[Filtering]]></term>
##       <term><![CDATA[Image reconstruction]]></term>
##       <term><![CDATA[Minization]]></term>
##       <term><![CDATA[Noise measurement]]></term>
##       <term><![CDATA[Threshold analysis]]></term>
##       <term><![CDATA[Transforms]]></term>
##     </thesaurusterms>
##     <pubtitle><![CDATA[Access, IEEE]]></pubtitle>
##     <punumber><![CDATA[6287639]]></punumber>
##     <pubtype><![CDATA[Journals & Magazines]]></pubtype>
##     <publisher><![CDATA[IEEE]]></publisher>
##     <volume><![CDATA[2]]></volume>
##     <py><![CDATA[2014]]></py>
##     <spage><![CDATA[602]]></spage>
##     <epage><![CDATA[613]]></epage>
##     <abstract><![CDATA[The &#x2113;<sub>1</sub> regularization problem has been widely used to solve the sparsity constrained problems. To enhance the sparsity constraint for better imaging performance, a promising direction is to use the &#x2113;<sub>p</sub> norm (0 &lt;; p &lt;; 1) and solve the &#x2113;<sub>p</sub> minimization problem. Very recently, Xu et al. developed an analytic solution for the &#x2113;<sub>1/2</sub> regularization via an iterative thresholding operation, which is also referred to as half-threshold filtering. In this paper, we design a simultaneous algebraic reconstruction technique (SART)-type half-threshold filtering framework to solve the computed tomography (CT) reconstruction problem. In the medical imaging filed, the discrete gradient transform (DGT) is widely used to define the sparsity. However, the DGT is noninvertible and it cannot be applied to half-threshold filtering for CT reconstruction. To demonstrate the utility of the proposed SART-type half-threshold filtering framework, an emphasis of this paper is to construct a pseudoinverse transforms for DGT. The proposed algorithms are evaluated with numerical and physical phantom data sets. Our results show that the SART-type half-threshold filtering algorithms have great potential to improve the reconstructed image quality from few and noisy projections. They are complementary to the counterparts of the state-of-the-art soft-threshold filtering and hard-threshold filtering.]]></abstract>
##     <issn><![CDATA[2169-3536]]></issn>
##     <htmlFlag><![CDATA[1]]></htmlFlag>
##     <arnumber><![CDATA[6819396]]></arnumber>
##     <doi><![CDATA[10.1109/ACCESS.2014.2326165]]></doi>
##     <publicationId><![CDATA[6819396]]></publicationId>
##     <mdurl><![CDATA[http://ieeexplore.ieee.org/xpl/articleDetails.jsp?tp=&arnumber=6819396&contentType=Journals+%26+Magazines]]></mdurl>
##     <pdf><![CDATA[http://ieeexplore.ieee.org/stamp/stamp.jsp?arnumber=6819396]]></pdf>
##   </document>
##   <document>
##     <rank>582</rank>
##     <title><![CDATA[Comparison of Long-Wave Infrared Imaging and Visible/Near-Infrared Imaging of Vegetation for Detecting Leaking <formula formulatype="inline"> <img src="/images/tex/21418.gif" alt="{\rm CO}_2"> </formula> Gas]]></title>
##     <authors><![CDATA[Johnson, J.E.;  Shaw, J.A.;  Lawrence, R.L.;  Nugent, P.W.;  Hogan, J.A.;  Dobeck, L.M.;  Spangler, L.H.]]></authors>
##     <affiliations><![CDATA[Electr. & Comput. Eng. Dept., Montana State Univ., Bozeman, MT, USA]]></affiliations>
##     <controlledterms>
##       <term><![CDATA[carbon capture and storage]]></term>
##       <term><![CDATA[carbon compounds]]></term>
##       <term><![CDATA[environmental monitoring (geophysics)]]></term>
##       <term><![CDATA[infrared imaging]]></term>
##       <term><![CDATA[leak detection]]></term>
##       <term><![CDATA[reflectivity]]></term>
##       <term><![CDATA[regression analysis]]></term>
##       <term><![CDATA[sunlight]]></term>
##       <term><![CDATA[vegetation mapping]]></term>
##     </controlledterms>
##     <thesaurusterms>
##       <term><![CDATA[Brightness temperature]]></term>
##       <term><![CDATA[Imaging]]></term>
##       <term><![CDATA[Monitoring]]></term>
##       <term><![CDATA[Soil]]></term>
##       <term><![CDATA[Soil measurements]]></term>
##       <term><![CDATA[Stress]]></term>
##       <term><![CDATA[Vegetation mapping]]></term>
##     </thesaurusterms>
##     <pubtitle><![CDATA[Selected Topics in Applied Earth Observations and Remote Sensing, IEEE Journal of]]></pubtitle>
##     <punumber><![CDATA[4609443]]></punumber>
##     <pubtype><![CDATA[Journals & Magazines]]></pubtype>
##     <publisher><![CDATA[IEEE]]></publisher>
##     <volume><![CDATA[7]]></volume>
##     <issue><![CDATA[5]]></issue>
##     <py><![CDATA[2014]]></py>
##     <spage><![CDATA[1651]]></spage>
##     <epage><![CDATA[1657]]></epage>
##     <abstract><![CDATA[Recent research demonstrated that CO<sub>2</sub> gas leaking from underground can be identified by observing increased stress in overlying vegetation using spectral imaging. This has been accomplished with both visible/near-infrared (Vis/NIR) sunlight reflection and long-wave infrared (LWIR) thermal emission. During a 4-week period in summer 2011, a controlled CO<sub>2</sub>release experiment was conducted in Bozeman, Montana, as part of a study of methods for monitoring carbon sequestration facilities. As part of this experiment, reflective and emissive imagers were deployed together to enable a comparison of these two types of imaging systems for vegetation-based CO<sub>2</sub> leak detection. A linear regression was performed using time as the response variable with red and NIR reflectances, Normalized Difference Vegetation Index (NDVI), and LWIR brightness temperature as predictors. The regression study showed that the reflectance and LWIR brightness temperature data together explained the most variability in the data (96%), equal to the performance of the Vis/NIR reflectance data alone, followed by NDVI alone (90%), and LWIR data alone (44%). Therefore, the two types of imagers contributed in a synergistic fashion, while either method alone was capable of gas detection with increased statistical variability.]]></abstract>
##     <issn><![CDATA[1939-1404]]></issn>
##     <htmlFlag><![CDATA[1]]></htmlFlag>
##     <arnumber><![CDATA[6754147]]></arnumber>
##     <doi><![CDATA[10.1109/JSTARS.2013.2295760]]></doi>
##     <publicationId><![CDATA[6754147]]></publicationId>
##     <mdurl><![CDATA[http://ieeexplore.ieee.org/xpl/articleDetails.jsp?tp=&arnumber=6754147&contentType=Journals+%26+Magazines]]></mdurl>
##     <pdf><![CDATA[http://ieeexplore.ieee.org/stamp/stamp.jsp?arnumber=6754147]]></pdf>
##   </document>
##   <document>
##     <rank>583</rank>
##     <title><![CDATA[Detection of Psychological Stress Using a Hyperspectral Imaging Technique]]></title>
##     <authors><![CDATA[Tong Chen;  Yuen, P.;  Richardson, M.;  Guangyuan Liu;  Zhishun She]]></authors>
##     <affiliations><![CDATA[Sch. of Electron. & Inf. Eng., Southwest Univ., Chongqing, China]]></affiliations>
##     <controlledterms>
##       <term><![CDATA[face recognition]]></term>
##       <term><![CDATA[hyperspectral imaging]]></term>
##       <term><![CDATA[image classification]]></term>
##       <term><![CDATA[medical image processing]]></term>
##       <term><![CDATA[psychology]]></term>
##     </controlledterms>
##     <thesaurusterms>
##       <term><![CDATA[Biomedical image processing]]></term>
##       <term><![CDATA[Hyperspectral imaging]]></term>
##       <term><![CDATA[Physiology]]></term>
##       <term><![CDATA[Psychology]]></term>
##       <term><![CDATA[Stress]]></term>
##       <term><![CDATA[Temperature measurement]]></term>
##       <term><![CDATA[Transient analysis]]></term>
##     </thesaurusterms>
##     <pubtitle><![CDATA[Affective Computing, IEEE Transactions on]]></pubtitle>
##     <punumber><![CDATA[5165369]]></punumber>
##     <pubtype><![CDATA[Journals & Magazines]]></pubtype>
##     <publisher><![CDATA[IEEE]]></publisher>
##     <volume><![CDATA[5]]></volume>
##     <issue><![CDATA[4]]></issue>
##     <py><![CDATA[2014]]></py>
##     <spage><![CDATA[391]]></spage>
##     <epage><![CDATA[405]]></epage>
##     <abstract><![CDATA[The detection of stress at early stages is beneficial to both individuals and communities. However, traditional stress detection methods that use physiological signals are contact-based and require sensors to be in contact with test subjects for measurement. In this paper, we present a method to detect psychological stress in a non-contact manner using a human physiological response. In particular, we utilize a hyperspectral imaging (HSI) technique to extract the tissue oxygen saturation (StO2) value as a physiological feature for stress detection. Our experimental results indicate that this new feature may be independent from perspiration and ambient temperature. Trier Social Stress Tests (TSSTs) on 21 volunteers demonstrated a significant difference $p\&lt; 0.005$ and a large practical discrimination (d 1/4 1.37) between normalized baseline and stress StO2 levels. The accuracy for stress recognition from baseline using a binary classifier was 76.19 and 88.1 percent for the automatic and manual selections of the classifier threshold, respectively. These results suggest that the StO2 level could serve as a new modality to recognize stress at standoff distances.]]></abstract>
##     <issn><![CDATA[1949-3045]]></issn>
##     <htmlFlag><![CDATA[1]]></htmlFlag>
##     <arnumber><![CDATA[6919328]]></arnumber>
##     <doi><![CDATA[10.1109/TAFFC.2014.2362513]]></doi>
##     <publicationId><![CDATA[6919328]]></publicationId>
##     <mdurl><![CDATA[http://ieeexplore.ieee.org/xpl/articleDetails.jsp?tp=&arnumber=6919328&contentType=Journals+%26+Magazines]]></mdurl>
##     <pdf><![CDATA[http://ieeexplore.ieee.org/stamp/stamp.jsp?arnumber=6919328]]></pdf>
##   </document>
##   <document>
##     <rank>584</rank>
##     <title><![CDATA[Input Balun Embedded Low-Noise Amplifier With a Differential Structure]]></title>
##     <authors><![CDATA[Jaehyuk Yoon;  Changkun Park]]></authors>
##     <affiliations><![CDATA[Sch. of Electron. Eng., Soongsil Univ., Seoul, South Korea]]></affiliations>
##     <controlledterms>
##       <term><![CDATA[CMOS integrated circuits]]></term>
##       <term><![CDATA[baluns]]></term>
##       <term><![CDATA[low noise amplifiers]]></term>
##     </controlledterms>
##     <thesaurusterms>
##       <term><![CDATA[Gain]]></term>
##       <term><![CDATA[Impedance matching]]></term>
##       <term><![CDATA[Logic gates]]></term>
##       <term><![CDATA[Noise]]></term>
##       <term><![CDATA[Noise measurement]]></term>
##       <term><![CDATA[Periodic structures]]></term>
##       <term><![CDATA[Wireless communication]]></term>
##     </thesaurusterms>
##     <pubtitle><![CDATA[Microwave and Wireless Components Letters, IEEE]]></pubtitle>
##     <punumber><![CDATA[7260]]></punumber>
##     <pubtype><![CDATA[Journals & Magazines]]></pubtype>
##     <publisher><![CDATA[IEEE]]></publisher>
##     <volume><![CDATA[24]]></volume>
##     <issue><![CDATA[6]]></issue>
##     <py><![CDATA[2014]]></py>
##     <spage><![CDATA[403]]></spage>
##     <epage><![CDATA[405]]></epage>
##     <abstract><![CDATA[In this study, we designed a 5 GHz low-noise amplifier (LNA) with a differential structure using 0.18 &#x03BC;m RFCMOS technology. An input balun is embedded into the LNA to enhance the gain, minimize the noise figure (NF), and miniaturize the overall chip size. The NF is minimized because the loss induced by the passive balun is removed. The first stage of the designed LNA performs the activities of the input balun and serves as the gain stage. To verify the feasibility of the proposed input-balun-embedded amplifier, we designed a typical LNA and the proposed LNA. We obtained a 29.4 dB gain with a NF of 1.85 dB. The measured dc power consumption is approximately 27 mW. The chip size is 1.0&#x00D7;0.74 mm<sup>2</sup>. From the measured results of the typical and proposed LNAs, we successfully prove the feasibility of the proposed method to minimize the NF and enhance the gain.]]></abstract>
##     <issn><![CDATA[1531-1309]]></issn>
##     <htmlFlag><![CDATA[1]]></htmlFlag>
##     <arnumber><![CDATA[6784509]]></arnumber>
##     <doi><![CDATA[10.1109/LMWC.2014.2313472]]></doi>
##     <publicationId><![CDATA[6784509]]></publicationId>
##     <mdurl><![CDATA[http://ieeexplore.ieee.org/xpl/articleDetails.jsp?tp=&arnumber=6784509&contentType=Journals+%26+Magazines]]></mdurl>
##     <pdf><![CDATA[http://ieeexplore.ieee.org/stamp/stamp.jsp?arnumber=6784509]]></pdf>
##   </document>
##   <document>
##     <rank>585</rank>
##     <title><![CDATA[Domino: Extracting, Comparing, and Manipulating Subsets Across Multiple Tabular Datasets]]></title>
##     <authors><![CDATA[Gratzl, S.;  Gehlenborg, N.;  Lex, A.;  Pfister, H.;  Streit, M.]]></authors>
##     <affiliations><![CDATA[Johannes Kepler Univ. Linz, Linz, Austria]]></affiliations>
##     <controlledterms>
##       <term><![CDATA[cancer]]></term>
##       <term><![CDATA[data visualisation]]></term>
##       <term><![CDATA[distributed databases]]></term>
##       <term><![CDATA[genomics]]></term>
##       <term><![CDATA[interactive systems]]></term>
##       <term><![CDATA[set theory]]></term>
##     </controlledterms>
##     <thesaurusterms>
##       <term><![CDATA[Biomedical measurements]]></term>
##       <term><![CDATA[Cancer]]></term>
##       <term><![CDATA[Data visualization]]></term>
##       <term><![CDATA[Genomics]]></term>
##     </thesaurusterms>
##     <pubtitle><![CDATA[Visualization and Computer Graphics, IEEE Transactions on]]></pubtitle>
##     <punumber><![CDATA[2945]]></punumber>
##     <pubtype><![CDATA[Journals & Magazines]]></pubtype>
##     <publisher><![CDATA[IEEE]]></publisher>
##     <volume><![CDATA[20]]></volume>
##     <issue><![CDATA[12]]></issue>
##     <py><![CDATA[2014]]></py>
##     <spage><![CDATA[2023]]></spage>
##     <epage><![CDATA[2032]]></epage>
##     <abstract><![CDATA[Answering questions about complex issues often requires analysts to take into account information contained in multiple interconnected datasets. A common strategy in analyzing and visualizing large and heterogeneous data is dividing it into meaningful subsets. Interesting subsets can then be selected and the associated data and the relationships between the subsets visualized. However, neither the extraction and manipulation nor the comparison of subsets is well supported by state-of-the-art techniques. In this paper we present Domino, a novel multiform visualization technique for effectively representing subsets and the relationships between them. By providing comprehensive tools to arrange, combine, and extract subsets, Domino allows users to create both common visualization techniques and advanced visualizations tailored to specific use cases. In addition to the novel technique, we present an implementation that enables analysts to manage the wide range of options that our approach offers. Innovative interactive features such as placeholders and live previews support rapid creation of complex analysis setups. We introduce the technique and the implementation using a simple example and demonstrate scalability and effectiveness in a use case from the field of cancer genomics.]]></abstract>
##     <issn><![CDATA[1077-2626]]></issn>
##     <htmlFlag><![CDATA[1]]></htmlFlag>
##     <arnumber><![CDATA[6875920]]></arnumber>
##     <doi><![CDATA[10.1109/TVCG.2014.2346260]]></doi>
##     <publicationId><![CDATA[6875920]]></publicationId>
##     <mdurl><![CDATA[http://ieeexplore.ieee.org/xpl/articleDetails.jsp?tp=&arnumber=6875920&contentType=Journals+%26+Magazines]]></mdurl>
##     <pdf><![CDATA[http://ieeexplore.ieee.org/stamp/stamp.jsp?arnumber=6875920]]></pdf>
##   </document>
##   <document>
##     <rank>586</rank>
##     <title><![CDATA[Cost Minimization for Big Data Processing in Geo-Distributed Data Centers]]></title>
##     <authors><![CDATA[Lin Gu;  Deze Zeng;  Peng Li;  Song Guo]]></authors>
##     <affiliations><![CDATA[Univ. of Aizu, Aizu-Wakamatsu, Japan]]></affiliations>
##     <controlledterms>
##       <term><![CDATA[Big Data]]></term>
##       <term><![CDATA[Markov processes]]></term>
##       <term><![CDATA[computer centres]]></term>
##       <term><![CDATA[integer programming]]></term>
##       <term><![CDATA[minimisation]]></term>
##       <term><![CDATA[nonlinear programming]]></term>
##     </controlledterms>
##     <thesaurusterms>
##       <term><![CDATA[Big data]]></term>
##       <term><![CDATA[Data handling]]></term>
##       <term><![CDATA[Data storage systems]]></term>
##       <term><![CDATA[Distributed databases]]></term>
##       <term><![CDATA[Information management]]></term>
##       <term><![CDATA[Minimization]]></term>
##       <term><![CDATA[Routing protocols]]></term>
##     </thesaurusterms>
##     <pubtitle><![CDATA[Emerging Topics in Computing, IEEE Transactions on]]></pubtitle>
##     <punumber><![CDATA[6245516]]></punumber>
##     <pubtype><![CDATA[Journals & Magazines]]></pubtype>
##     <publisher><![CDATA[IEEE]]></publisher>
##     <volume><![CDATA[2]]></volume>
##     <issue><![CDATA[3]]></issue>
##     <py><![CDATA[2014]]></py>
##     <spage><![CDATA[314]]></spage>
##     <epage><![CDATA[323]]></epage>
##     <abstract><![CDATA[The explosive growth of demands on big data processing imposes a heavy burden on computation, storage, and communication in data centers, which hence incurs considerable operational expenditure to data center providers. Therefore, cost minimization has become an emergent issue for the upcoming big data era. Different from conventional cloud services, one of the main features of big data services is the tight coupling between data and computation as computation tasks can be conducted only when the corresponding data are available. As a result, three factors, i.e., task assignment, data placement, and data movement, deeply influence the operational expenditure of data centers. In this paper, we are motivated to study the cost minimization problem via a joint optimization of these three factors for big data services in geo-distributed data centers. To describe the task completion time with the consideration of both data transmission and computation, we propose a 2-D Markov chain and derive the average task completion time in closed-form. Furthermore, we model the problem as a mixed-integer nonlinear programming and propose an efficient solution to linearize it. The high efficiency of our proposal is validated by extensive simulation-based studies.]]></abstract>
##     <issn><![CDATA[2168-6750]]></issn>
##     <htmlFlag><![CDATA[1]]></htmlFlag>
##     <arnumber><![CDATA[6762920]]></arnumber>
##     <doi><![CDATA[10.1109/TETC.2014.2310456]]></doi>
##     <publicationId><![CDATA[6762920]]></publicationId>
##     <mdurl><![CDATA[http://ieeexplore.ieee.org/xpl/articleDetails.jsp?tp=&arnumber=6762920&contentType=Journals+%26+Magazines]]></mdurl>
##     <pdf><![CDATA[http://ieeexplore.ieee.org/stamp/stamp.jsp?arnumber=6762920]]></pdf>
##   </document>
##   <document>
##     <rank>587</rank>
##     <title><![CDATA[Wound Rotor Machine Fed by a Single-Phase Grid and Controlled by an Isolated Inverter]]></title>
##     <authors><![CDATA[Yongsu Han;  Jung-Ik Ha]]></authors>
##     <affiliations><![CDATA[Dept. of Electr. & Comput. Eng., Seoul Nat. Univ., Seoul, South Korea]]></affiliations>
##     <controlledterms>
##       <term><![CDATA[direct energy conversion]]></term>
##       <term><![CDATA[invertors]]></term>
##       <term><![CDATA[machine control]]></term>
##       <term><![CDATA[power factor]]></term>
##       <term><![CDATA[rotors]]></term>
##       <term><![CDATA[stators]]></term>
##       <term><![CDATA[variable speed drives]]></term>
##     </controlledterms>
##     <thesaurusterms>
##       <term><![CDATA[Energy conversion]]></term>
##       <term><![CDATA[Inverters]]></term>
##       <term><![CDATA[Rotors]]></term>
##       <term><![CDATA[Stator windings]]></term>
##       <term><![CDATA[Torque]]></term>
##       <term><![CDATA[Voltage control]]></term>
##     </thesaurusterms>
##     <pubtitle><![CDATA[Power Electronics, IEEE Transactions on]]></pubtitle>
##     <punumber><![CDATA[63]]></punumber>
##     <pubtype><![CDATA[Journals & Magazines]]></pubtype>
##     <publisher><![CDATA[IEEE]]></publisher>
##     <volume><![CDATA[29]]></volume>
##     <issue><![CDATA[9]]></issue>
##     <py><![CDATA[2014]]></py>
##     <spage><![CDATA[4843]]></spage>
##     <epage><![CDATA[4854]]></epage>
##     <abstract><![CDATA[This paper proposes a novel energy conversion system for variable speed drives. It consists of a wound rotor machine and an inverter without any rectifier and input filter. In the proposed system, the stator of the machine is directly connected to a single-phase grid and the rotor is connected to three-phase inverter isolated with any external power source. The inverter can not only be connected through slip rings but also integrated on the rotor due to the structure. In this paper, based on the positive and negative sequence model, the rotor, stator powers, and the torque capability in the rotor energy balance and unity grid power factor are analyzed. From these analyses, the vector control methods of the torque, speed, the dc-link voltage of the isolated inverter, and the grid power factor are proposed. Since the power supplied from the single-phase grid is pulsating and the machine is directly connected to the grid, the controlled torque and power inevitably pulsates at twice the grid frequency. Nevertheless, the machine can start, accelerate, and decelerate in the wide range. The experimental results present the performance and the feasibility of the proposed system.]]></abstract>
##     <issn><![CDATA[0885-8993]]></issn>
##     <htmlFlag><![CDATA[1]]></htmlFlag>
##     <arnumber><![CDATA[6615970]]></arnumber>
##     <doi><![CDATA[10.1109/TPEL.2013.2283738]]></doi>
##     <publicationId><![CDATA[6615970]]></publicationId>
##     <mdurl><![CDATA[http://ieeexplore.ieee.org/xpl/articleDetails.jsp?tp=&arnumber=6615970&contentType=Journals+%26+Magazines]]></mdurl>
##     <pdf><![CDATA[http://ieeexplore.ieee.org/stamp/stamp.jsp?arnumber=6615970]]></pdf>
##   </document>
##   <document>
##     <rank>588</rank>
##     <title><![CDATA[Minitaur, an Event-Driven FPGA-Based Spiking Network Accelerator]]></title>
##     <authors><![CDATA[Neil, D.;  Shih-Chii Liu]]></authors>
##     <affiliations><![CDATA[Inst. of Neuroinf., Univ. of Zurich, Zu&#x0308;rich, Switzerland]]></affiliations>
##     <controlledterms>
##       <term><![CDATA[field programmable gate arrays]]></term>
##       <term><![CDATA[neural nets]]></term>
##     </controlledterms>
##     <thesaurusterms>
##       <term><![CDATA[Biological neural networks]]></term>
##       <term><![CDATA[Clocks]]></term>
##       <term><![CDATA[Computer architecture]]></term>
##       <term><![CDATA[Field programmable gate arrays]]></term>
##       <term><![CDATA[Mathematical model]]></term>
##       <term><![CDATA[Neurons]]></term>
##       <term><![CDATA[Performance evaluation]]></term>
##     </thesaurusterms>
##     <pubtitle><![CDATA[Very Large Scale Integration (VLSI) Systems, IEEE Transactions on]]></pubtitle>
##     <punumber><![CDATA[92]]></punumber>
##     <pubtype><![CDATA[Journals & Magazines]]></pubtype>
##     <publisher><![CDATA[IEEE]]></publisher>
##     <volume><![CDATA[22]]></volume>
##     <issue><![CDATA[12]]></issue>
##     <py><![CDATA[2014]]></py>
##     <spage><![CDATA[2621]]></spage>
##     <epage><![CDATA[2628]]></epage>
##     <abstract><![CDATA[Current neural networks are accumulating accolades for their performance on a variety of real-world computational tasks including recognition, classification, regression, and prediction, yet there are few scalable architectures that have emerged to address the challenges posed by their computation. This paper introduces Minitaur, an event-driven neural network accelerator, which is designed for low power and high performance. As an field-programmable gate array-based system, it can be integrated into existing robotics or it can offload computationally expensive neural network tasks from the CPU. The version presented here implements a spiking deep network which achieves 19 million postsynaptic currents per second on 1.5 W of power and supports up to 65 K neurons per board. The system records 92% accuracy on the MNIST handwritten digit classification and 71% accuracy on the 20 newsgroups classification data set. Due to its event-driven nature, it allows for trading off between accuracy and latency.]]></abstract>
##     <issn><![CDATA[1063-8210]]></issn>
##     <htmlFlag><![CDATA[1]]></htmlFlag>
##     <arnumber><![CDATA[6701396]]></arnumber>
##     <doi><![CDATA[10.1109/TVLSI.2013.2294916]]></doi>
##     <publicationId><![CDATA[6701396]]></publicationId>
##     <mdurl><![CDATA[http://ieeexplore.ieee.org/xpl/articleDetails.jsp?tp=&arnumber=6701396&contentType=Journals+%26+Magazines]]></mdurl>
##     <pdf><![CDATA[http://ieeexplore.ieee.org/stamp/stamp.jsp?arnumber=6701396]]></pdf>
##   </document>
##   <document>
##     <rank>589</rank>
##     <title><![CDATA[Dynamic Uncertain Causality Graph for Knowledge Representation and Probabilistic Reasoning: Statistics Base, Matrix, and Application]]></title>
##     <authors><![CDATA[Qin Zhang;  Chunling Dong;  Yan Cui;  Zhihui Yang]]></authors>
##     <affiliations><![CDATA[Sch. of Comput. Sci. & Eng., Beihang Univ., Beijing, China]]></affiliations>
##     <controlledterms>
##       <term><![CDATA[belief networks]]></term>
##       <term><![CDATA[causality]]></term>
##       <term><![CDATA[fault diagnosis]]></term>
##       <term><![CDATA[inference mechanisms]]></term>
##       <term><![CDATA[laptop computers]]></term>
##       <term><![CDATA[matrix algebra]]></term>
##       <term><![CDATA[nuclear power stations]]></term>
##       <term><![CDATA[power engineering computing]]></term>
##       <term><![CDATA[statistics]]></term>
##     </controlledterms>
##     <thesaurusterms>
##       <term><![CDATA[Cognition]]></term>
##       <term><![CDATA[Heuristic algorithms]]></term>
##       <term><![CDATA[Hidden Markov models]]></term>
##       <term><![CDATA[Inference algorithms]]></term>
##       <term><![CDATA[Logic gates]]></term>
##       <term><![CDATA[Probabilistic logic]]></term>
##       <term><![CDATA[Probability]]></term>
##     </thesaurusterms>
##     <pubtitle><![CDATA[Neural Networks and Learning Systems, IEEE Transactions on]]></pubtitle>
##     <punumber><![CDATA[5962385]]></punumber>
##     <pubtype><![CDATA[Journals & Magazines]]></pubtype>
##     <publisher><![CDATA[IEEE]]></publisher>
##     <volume><![CDATA[25]]></volume>
##     <issue><![CDATA[4]]></issue>
##     <py><![CDATA[2014]]></py>
##     <spage><![CDATA[645]]></spage>
##     <epage><![CDATA[663]]></epage>
##     <abstract><![CDATA[Graphical models for probabilistic reasoning are now in widespread use. Many approaches have been developed such as Bayesian network. A newly developed approach named as dynamic uncertain causality graph (DUCG) is initially presented in a previous paper, in which only the inference algorithm in terms of individual events and probabilities is addressed. In this paper, we first explain the statistic basis of DUCG. Then, we extend the algorithm to the form of matrices of events and probabilities. It is revealed that the representation of DUCG can be incomplete and the exact probabilistic inference may still be made. A real application of DUCG for fault diagnoses of a generator system of a nuclear power plant is demonstrated, which involves variables. Most inferences take with a laptop computer. The causal logic between inference result and observations is graphically displayed to users so that they know not only the result, but also why the result obtained.]]></abstract>
##     <issn><![CDATA[2162-237X]]></issn>
##     <htmlFlag><![CDATA[1]]></htmlFlag>
##     <arnumber><![CDATA[6600881]]></arnumber>
##     <doi><![CDATA[10.1109/TNNLS.2013.2279320]]></doi>
##     <publicationId><![CDATA[6600881]]></publicationId>
##     <mdurl><![CDATA[http://ieeexplore.ieee.org/xpl/articleDetails.jsp?tp=&arnumber=6600881&contentType=Journals+%26+Magazines]]></mdurl>
##     <pdf><![CDATA[http://ieeexplore.ieee.org/stamp/stamp.jsp?arnumber=6600881]]></pdf>
##   </document>
##   <document>
##     <rank>590</rank>
##     <title><![CDATA[Temperature Sensing Using Photonic Crystal Fiber Filled With Silver Nanowires and Liquid]]></title>
##     <authors><![CDATA[Lu, Y.;  Wang, M.T.;  Hao, C.J.;  Zhao, Z.Q.;  Yao, J.Q.]]></authors>
##     <affiliations><![CDATA[Key Lab. of Opto-Electron. Inf. Technol. (Minist. of Educ.), Tianjin Univ., Tianjin, China]]></affiliations>
##     <controlledterms>
##       <term><![CDATA[fibre optic sensors]]></term>
##       <term><![CDATA[holey fibres]]></term>
##       <term><![CDATA[nanophotonics]]></term>
##       <term><![CDATA[nanowires]]></term>
##       <term><![CDATA[photonic crystals]]></term>
##       <term><![CDATA[silver]]></term>
##       <term><![CDATA[spectral line shift]]></term>
##       <term><![CDATA[surface plasmon resonance]]></term>
##       <term><![CDATA[temperature sensors]]></term>
##     </controlledterms>
##     <thesaurusterms>
##       <term><![CDATA[Nanowires]]></term>
##       <term><![CDATA[Optical fiber sensors]]></term>
##       <term><![CDATA[Plasmons]]></term>
##       <term><![CDATA[Sensitivity]]></term>
##       <term><![CDATA[Silver]]></term>
##       <term><![CDATA[Temperature sensors]]></term>
##     </thesaurusterms>
##     <pubtitle><![CDATA[Photonics Journal, IEEE]]></pubtitle>
##     <punumber><![CDATA[4563994]]></punumber>
##     <pubtype><![CDATA[Journals & Magazines]]></pubtype>
##     <publisher><![CDATA[IEEE]]></publisher>
##     <volume><![CDATA[6]]></volume>
##     <issue><![CDATA[3]]></issue>
##     <py><![CDATA[2014]]></py>
##     <spage><![CDATA[1]]></spage>
##     <epage><![CDATA[7]]></epage>
##     <abstract><![CDATA[A temperature sensor based on photonic crystal fiber (PCF) surface plasmon resonance (SPR) is proposed in this paper. We use the dual function of the PCF filled with different concentrations of analyte and silver nanowires to realize temperature sensing. The proposed sensor has been analyzed through numerical simulations and demonstrated by experiments. The results of the simulations and experiments show that a blue shift will be obtained with the temperature increase, and different concentrations will change the resonance wavelength and confinement loss. Temperature sensitivity is as high as 2.7 nm/&#x00B0;C with the experiment, which can provide a reference for the implementation and application of a PCF-based SPR temperature sensor or other PCF-based SPR sensing.]]></abstract>
##     <issn><![CDATA[1943-0655]]></issn>
##     <htmlFlag><![CDATA[1]]></htmlFlag>
##     <arnumber><![CDATA[6803883]]></arnumber>
##     <doi><![CDATA[10.1109/JPHOT.2014.2319086]]></doi>
##     <publicationId><![CDATA[6803883]]></publicationId>
##     <mdurl><![CDATA[http://ieeexplore.ieee.org/xpl/articleDetails.jsp?tp=&arnumber=6803883&contentType=Journals+%26+Magazines]]></mdurl>
##     <pdf><![CDATA[http://ieeexplore.ieee.org/stamp/stamp.jsp?arnumber=6803883]]></pdf>
##   </document>
##   <document>
##     <rank>591</rank>
##     <title><![CDATA[UV-Curable Polymer Microhemisphere-Based Fiber-Optic Fabry&#x2013;Perot Interferometer for Simultaneous Measurement of Refractive Index and Temperature]]></title>
##     <authors><![CDATA[Tan, X.L.;  Geng, Y.F.;  Li, X.J.;  Deng, Y.L.;  Yin, Z.;  Gao, R.]]></authors>
##     <affiliations><![CDATA[Shenzhen Key Lab. of Sensor Technol., Shenzhen Univ., Shenzhen, China]]></affiliations>
##     <controlledterms>
##       <term><![CDATA[Fabry-Perot interferometers]]></term>
##       <term><![CDATA[curing]]></term>
##       <term><![CDATA[fibre optic sensors]]></term>
##       <term><![CDATA[optical polymers]]></term>
##       <term><![CDATA[reflectivity]]></term>
##       <term><![CDATA[refractive index measurement]]></term>
##       <term><![CDATA[temperature measurement]]></term>
##     </controlledterms>
##     <thesaurusterms>
##       <term><![CDATA[Liquids]]></term>
##       <term><![CDATA[Optical fiber sensors]]></term>
##       <term><![CDATA[Polymers]]></term>
##       <term><![CDATA[Refractive index]]></term>
##       <term><![CDATA[Sensitivity]]></term>
##       <term><![CDATA[Temperature measurement]]></term>
##       <term><![CDATA[Temperature sensors]]></term>
##     </thesaurusterms>
##     <pubtitle><![CDATA[Photonics Journal, IEEE]]></pubtitle>
##     <punumber><![CDATA[4563994]]></punumber>
##     <pubtype><![CDATA[Journals & Magazines]]></pubtype>
##     <publisher><![CDATA[IEEE]]></publisher>
##     <volume><![CDATA[6]]></volume>
##     <issue><![CDATA[4]]></issue>
##     <py><![CDATA[2014]]></py>
##     <spage><![CDATA[1]]></spage>
##     <epage><![CDATA[8]]></epage>
##     <abstract><![CDATA[A fiber-optic Fabry-Perot interferometer based on UV-curable polymer microhemisphere is proposed and demonstrated. The polymer microhemisphere is formed by adhering and solidifying a liquid microdroplet of UV-curable adhesive to the end face of a cleaved single-mode fiber. The height of polymer microhemisphere could be flexibly controlled by adjusting the diameter of a single-mode fiber. The theoretical and experimental results demonstrate that the refractive index (RI) and the temperature of external environment can be simultaneously measured by the fringe contrast variation and the wavelength shift of reflection spectra separately, alleviating the cross sensitivity effectively. The obtained temperature and RI sensitivities are about 0.19 nm/&#x00B0;C and 260 dB/RIU in the RI range of 1.38-1.42.]]></abstract>
##     <issn><![CDATA[1943-0655]]></issn>
##     <htmlFlag><![CDATA[1]]></htmlFlag>
##     <arnumber><![CDATA[6842650]]></arnumber>
##     <doi><![CDATA[10.1109/JPHOT.2014.2332460]]></doi>
##     <publicationId><![CDATA[6842650]]></publicationId>
##     <mdurl><![CDATA[http://ieeexplore.ieee.org/xpl/articleDetails.jsp?tp=&arnumber=6842650&contentType=Journals+%26+Magazines]]></mdurl>
##     <pdf><![CDATA[http://ieeexplore.ieee.org/stamp/stamp.jsp?arnumber=6842650]]></pdf>
##   </document>
##   <document>
##     <rank>592</rank>
##     <title><![CDATA[Adaptive Dynamic Programming for Optimal Tracking Control of Unknown Nonlinear Systems With Application to Coal Gasification]]></title>
##     <authors><![CDATA[Qinglai Wei;  Derong Liu]]></authors>
##     <affiliations><![CDATA[State Key Lab. of Manage. & Control for Complex Syst., Inst. of Autom.many, Beijing, China]]></affiliations>
##     <controlledterms>
##       <term><![CDATA[coal gasification]]></term>
##       <term><![CDATA[control engineering computing]]></term>
##       <term><![CDATA[dynamic programming]]></term>
##       <term><![CDATA[iterative methods]]></term>
##       <term><![CDATA[neural nets]]></term>
##       <term><![CDATA[nonlinear control systems]]></term>
##       <term><![CDATA[optimal control]]></term>
##     </controlledterms>
##     <thesaurusterms>
##       <term><![CDATA[Coal gas]]></term>
##       <term><![CDATA[Dynamic programming]]></term>
##       <term><![CDATA[Neural networks]]></term>
##       <term><![CDATA[Optimal control]]></term>
##       <term><![CDATA[Tracking]]></term>
##     </thesaurusterms>
##     <pubtitle><![CDATA[Automation Science and Engineering, IEEE Transactions on]]></pubtitle>
##     <punumber><![CDATA[8856]]></punumber>
##     <pubtype><![CDATA[Journals & Magazines]]></pubtype>
##     <publisher><![CDATA[IEEE]]></publisher>
##     <volume><![CDATA[11]]></volume>
##     <issue><![CDATA[4]]></issue>
##     <py><![CDATA[2014]]></py>
##     <spage><![CDATA[1020]]></spage>
##     <epage><![CDATA[1036]]></epage>
##     <abstract><![CDATA[In this paper, we establish a new data-based iterative optimal learning control scheme for discrete-time nonlinear systems using iterative adaptive dynamic programming (ADP) approach and apply the developed control scheme to solve a coal gasification optimal tracking control problem. According to the system data, neural networks (NNs) are used to construct the dynamics of coal gasification process, coal quality and reference control, respectively, where the mathematical model of the system is unnecessary. The approximation errors from neural network construction of the disturbance and the controls are both considered. Via system transformation, the optimal tracking control problem with approximation errors and disturbances is effectively transformed into a two-person zero-sum optimal control problem. A new iterative ADP algorithm is then developed to obtain the optimal control laws for the transformed system. Convergence property is developed to guarantee that the performance index function converges to a finite neighborhood of the optimal performance index function, and the convergence criterion is also obtained. Finally, numerical results are given to illustrate the performance of the present method.]]></abstract>
##     <issn><![CDATA[1545-5955]]></issn>
##     <htmlFlag><![CDATA[1]]></htmlFlag>
##     <arnumber><![CDATA[6656960]]></arnumber>
##     <doi><![CDATA[10.1109/TASE.2013.2284545]]></doi>
##     <publicationId><![CDATA[6656960]]></publicationId>
##     <mdurl><![CDATA[http://ieeexplore.ieee.org/xpl/articleDetails.jsp?tp=&arnumber=6656960&contentType=Journals+%26+Magazines]]></mdurl>
##     <pdf><![CDATA[http://ieeexplore.ieee.org/stamp/stamp.jsp?arnumber=6656960]]></pdf>
##   </document>
##   <document>
##     <rank>593</rank>
##     <title><![CDATA[Scalable and Efficient Diagnosis for 5G Data Center Network Traffic]]></title>
##     <authors><![CDATA[Liu, C.H.;  Jun Fan]]></authors>
##     <affiliations><![CDATA[Sch. of Software, Beijing Inst. of Technol., Beijing, China]]></affiliations>
##     <controlledterms>
##       <term><![CDATA[Internet]]></term>
##       <term><![CDATA[computational complexity]]></term>
##       <term><![CDATA[computer centres]]></term>
##       <term><![CDATA[computer network reliability]]></term>
##       <term><![CDATA[mobile communication]]></term>
##       <term><![CDATA[telecommunication congestion control]]></term>
##       <term><![CDATA[telecommunication traffic]]></term>
##     </controlledterms>
##     <thesaurusterms>
##       <term><![CDATA[Algorithm design and analysis]]></term>
##       <term><![CDATA[Data centers]]></term>
##       <term><![CDATA[IP networks]]></term>
##       <term><![CDATA[Next generation networking]]></term>
##       <term><![CDATA[Ports (Computers)]]></term>
##       <term><![CDATA[Radiation detectors]]></term>
##       <term><![CDATA[Scability]]></term>
##       <term><![CDATA[Telecommunication traffic]]></term>
##     </thesaurusterms>
##     <pubtitle><![CDATA[Access, IEEE]]></pubtitle>
##     <punumber><![CDATA[6287639]]></punumber>
##     <pubtype><![CDATA[Journals & Magazines]]></pubtype>
##     <publisher><![CDATA[IEEE]]></publisher>
##     <volume><![CDATA[2]]></volume>
##     <py><![CDATA[2014]]></py>
##     <spage><![CDATA[841]]></spage>
##     <epage><![CDATA[855]]></epage>
##     <abstract><![CDATA[Data center networks (DCNs) for 5G are expected to support a large number of different bandwidth-hungry applications with exploding data, such as real-time search and data analysis. As a result, significant challenges are imposed to identify the cause of link congestion between any pair of switch ports that may severely damage the overall network performance. Generally, it is expected that the granularity of the flow monitoring to diagnose network congestion in 5G DCNs needs to be down to the flow level on a physical port of a switch in real time with high-estimation accuracy, low-computational complexity, and good scalability. In this paper, motivated by a comprehensive study of a real DCN trace, we propose two sketch-based algorithms, called &#x03B1;-conservative update (CU) and P(d)-CU, based on the existing CU approach. &#x03B1;-CU adds no extra implementation cost to the traditional CU, but successfully trades off the achieved error with time complexity. P(d)-CU fully considers the amount of skew for different types of network services to aggregate traffic statistics of each type of network traffic at an individual, horizontally partitioned sketch. We also introduce a way to produce the real-time moving average of the reported results. By theoretical analysis and sufficient experimental results on a real DCN trace, we extensively evaluate the proposed and existing algorithms on their error performance, recall, space cost, and time complexity.]]></abstract>
##     <issn><![CDATA[2169-3536]]></issn>
##     <htmlFlag><![CDATA[1]]></htmlFlag>
##     <arnumber><![CDATA[6879489]]></arnumber>
##     <doi><![CDATA[10.1109/ACCESS.2014.2349000]]></doi>
##     <publicationId><![CDATA[6879489]]></publicationId>
##     <mdurl><![CDATA[http://ieeexplore.ieee.org/xpl/articleDetails.jsp?tp=&arnumber=6879489&contentType=Journals+%26+Magazines]]></mdurl>
##     <pdf><![CDATA[http://ieeexplore.ieee.org/stamp/stamp.jsp?arnumber=6879489]]></pdf>
##   </document>
##   <document>
##     <rank>594</rank>
##     <title><![CDATA[On the Outage Performance of Selection Amplify-and-Forward Relaying Scheme]]></title>
##     <authors><![CDATA[Jeehoon Lee;  Minjoong Rim;  Kiseon Kim]]></authors>
##     <affiliations><![CDATA[Sch. of Inf. & Commun., GIST, Gwangju, South Korea]]></affiliations>
##     <controlledterms>
##       <term><![CDATA[amplify and forward communication]]></term>
##       <term><![CDATA[probability]]></term>
##       <term><![CDATA[relay networks (telecommunication)]]></term>
##     </controlledterms>
##     <thesaurusterms>
##       <term><![CDATA[Antenna arrays]]></term>
##       <term><![CDATA[Correlation]]></term>
##       <term><![CDATA[Diversity methods]]></term>
##       <term><![CDATA[Probability density function]]></term>
##       <term><![CDATA[Random variables]]></term>
##       <term><![CDATA[Relays]]></term>
##       <term><![CDATA[Signal to noise ratio]]></term>
##     </thesaurusterms>
##     <pubtitle><![CDATA[Communications Letters, IEEE]]></pubtitle>
##     <punumber><![CDATA[4234]]></punumber>
##     <pubtype><![CDATA[Journals & Magazines]]></pubtype>
##     <publisher><![CDATA[IEEE]]></publisher>
##     <volume><![CDATA[18]]></volume>
##     <issue><![CDATA[3]]></issue>
##     <py><![CDATA[2014]]></py>
##     <spage><![CDATA[423]]></spage>
##     <epage><![CDATA[426]]></epage>
##     <abstract><![CDATA[In this letter, we propose a selection amplify-and-forward (AF) relaying scheme which has the lower outage probability than that of a conventional AF relaying scheme in cooperative relay networks. In real wireless environments, as the channel of source-to-destination (SD) link varies with an increase in time, we can also obtain a diversity gain through the SD link by retransmission in common with a conventional AF relaying scheme. Thus, we can expect a performance enhancement by adaptively determining the transmitting node between the relaying and source nodes. We propose a method for adaptively determining between AF relaying and retransmission schemes, and show that the additional diversity gain can be obtained by the proposed scheme.]]></abstract>
##     <issn><![CDATA[1089-7798]]></issn>
##     <htmlFlag><![CDATA[1]]></htmlFlag>
##     <arnumber><![CDATA[6784537]]></arnumber>
##     <doi><![CDATA[10.1109/LCOMM.2014.011214.132477]]></doi>
##     <publicationId><![CDATA[6784537]]></publicationId>
##     <mdurl><![CDATA[http://ieeexplore.ieee.org/xpl/articleDetails.jsp?tp=&arnumber=6784537&contentType=Journals+%26+Magazines]]></mdurl>
##     <pdf><![CDATA[http://ieeexplore.ieee.org/stamp/stamp.jsp?arnumber=6784537]]></pdf>
##   </document>
##   <document>
##     <rank>595</rank>
##     <title><![CDATA[Human&#x2013;Manipulator Interface Based on Multisensory Process via Kalman Filters]]></title>
##     <authors><![CDATA[Guanglong Du;  Ping Zhang;  Di Li]]></authors>
##     <affiliations><![CDATA[Sch. of Mech. & Automotive Eng., South China Univ. of Technol., Guangzhou, China]]></affiliations>
##     <controlledterms>
##       <term><![CDATA[Kalman filters]]></term>
##       <term><![CDATA[human-robot interaction]]></term>
##       <term><![CDATA[image sensors]]></term>
##       <term><![CDATA[manipulators]]></term>
##       <term><![CDATA[position control]]></term>
##     </controlledterms>
##     <thesaurusterms>
##       <term><![CDATA[Damping]]></term>
##       <term><![CDATA[Joints]]></term>
##       <term><![CDATA[Licenses]]></term>
##       <term><![CDATA[Robot sensing systems]]></term>
##       <term><![CDATA[Tracking]]></term>
##       <term><![CDATA[Vectors]]></term>
##     </thesaurusterms>
##     <pubtitle><![CDATA[Industrial Electronics, IEEE Transactions on]]></pubtitle>
##     <punumber><![CDATA[41]]></punumber>
##     <pubtype><![CDATA[Journals & Magazines]]></pubtype>
##     <publisher><![CDATA[IEEE]]></publisher>
##     <volume><![CDATA[61]]></volume>
##     <issue><![CDATA[10]]></issue>
##     <py><![CDATA[2014]]></py>
##     <spage><![CDATA[5411]]></spage>
##     <epage><![CDATA[5418]]></epage>
##     <abstract><![CDATA[This paper presents a human-robot interface, which incorporates Kalman filters (KFs) and adaptive multispace transformation (AMT), to track movements of the human hand and control the robot manipulator. This system employs one inertial measurement unit and a 3-D camera (Kinect) to determine the orientation and translation of the human hand, and uses KFs to estimate these. Although KFs can estimate the translation, the translation error increases in a short period of time when the sensor fails to sense hand movement, including handshaking. Therefore, a method to correct the translation error is required. In this paper, the change rate of the human hand is used to determine the posture of the robot. An overdamping strategy is also employed to eliminate the effect of movement sensing failure. Given that a human operator has difficulty operating with high precision due to perceptive and motor limitations, an AMT method is proposed to assist the operator in improving the accuracy and reliability of determining the movement of the robot. The human-manipulator interface is then experimentally tested in a laboratory environment. The results indicate that the system based on the human-manipulator interface can successfully control the robot manipulator.]]></abstract>
##     <issn><![CDATA[0278-0046]]></issn>
##     <arnumber><![CDATA[6718048]]></arnumber>
##     <doi><![CDATA[10.1109/TIE.2014.2301728]]></doi>
##     <publicationId><![CDATA[6718048]]></publicationId>
##     <mdurl><![CDATA[http://ieeexplore.ieee.org/xpl/articleDetails.jsp?tp=&arnumber=6718048&contentType=Journals+%26+Magazines]]></mdurl>
##     <pdf><![CDATA[http://ieeexplore.ieee.org/stamp/stamp.jsp?arnumber=6718048]]></pdf>
##   </document>
##   <document>
##     <rank>596</rank>
##     <title><![CDATA[Lightweight mobile core networks for machine type communications]]></title>
##     <authors><![CDATA[Taleb, T.;  Ksentini, A.;  Kobbane, A.]]></authors>
##     <affiliations><![CDATA[Aalto Univ., Espoo, Finland]]></affiliations>
##     <controlledterms>
##       <term><![CDATA[3G mobile communication]]></term>
##       <term><![CDATA[cloud computing]]></term>
##       <term><![CDATA[mobile computing]]></term>
##       <term><![CDATA[telecommunication traffic]]></term>
##       <term><![CDATA[virtualisation]]></term>
##     </controlledterms>
##     <thesaurusterms>
##       <term><![CDATA[Logic gates]]></term>
##       <term><![CDATA[Machine-to-machine communications]]></term>
##       <term><![CDATA[Mobile communication]]></term>
##       <term><![CDATA[Mobile computing]]></term>
##       <term><![CDATA[Object recognition]]></term>
##       <term><![CDATA[Protocols]]></term>
##       <term><![CDATA[Radio access networks]]></term>
##       <term><![CDATA[Servers]]></term>
##     </thesaurusterms>
##     <pubtitle><![CDATA[Access, IEEE]]></pubtitle>
##     <punumber><![CDATA[6287639]]></punumber>
##     <pubtype><![CDATA[Journals & Magazines]]></pubtype>
##     <publisher><![CDATA[IEEE]]></publisher>
##     <volume><![CDATA[2]]></volume>
##     <py><![CDATA[2014]]></py>
##     <spage><![CDATA[1128]]></spage>
##     <epage><![CDATA[1137]]></epage>
##     <abstract><![CDATA[Machine type communications (MTCs) enable the communications of machines (devices) to machines over mobile networks. Besides simplifying our daily lives, the MTC business represents a promising market for mobile operators to increase their revenues. However, before a complete deployment of MTC over mobile networks, there is need to update the specifications of mobile networks in order to cope with the expected high number (massive deployment) of MTC devices. Indeed, large scale deployment of MTC devices represents an important challenge as a high number of MTC devices, simultaneously connecting to the mobile network, may cause congestion and system overload, which can degrade the network performance and even result in network node failures. Several activities have been led by 3GPP to alleviate system overload introduced by MTC. Most of the devised approaches represent only incremental solutions. Unlike these solutions, we devise a complete new architectural vision to support MTC in mobile networks. This vision relies on the marriage of mobile networks and cloud computing, specifically exploiting recent advances in network function virtualization (NFV). The aim of the proposed vision, namely LightEPC, is: 1) to orchestrate the on-demand creation of cloud-based lightweight mobile core networks dedicated for MTC and 2) to simplify the network attach procedure for MTC devices by creating only one NFV MTC function that groups all the usual procedures. By doing so, LightEPC is able to create and scale instances of NFV MTC functions on demand and in an elastic manner to cope with any sudden increase in traffic generated by MTC devices. To evaluate LightEPC, some preliminary analysis were conducted and the obtained analytical results indicate the ability of LightEPC in alleviating congestion and scaling up fast with massive numbers of MTC devices in mobile networks. Finally, a real-life implementation of LightEPC on top of cloud platform is discussed.]]></abstract>
##     <issn><![CDATA[2169-3536]]></issn>
##     <htmlFlag><![CDATA[1]]></htmlFlag>
##     <arnumber><![CDATA[6906244]]></arnumber>
##     <doi><![CDATA[10.1109/ACCESS.2014.2359649]]></doi>
##     <publicationId><![CDATA[6906244]]></publicationId>
##     <mdurl><![CDATA[http://ieeexplore.ieee.org/xpl/articleDetails.jsp?tp=&arnumber=6906244&contentType=Journals+%26+Magazines]]></mdurl>
##     <pdf><![CDATA[http://ieeexplore.ieee.org/stamp/stamp.jsp?arnumber=6906244]]></pdf>
##   </document>
##   <document>
##     <rank>597</rank>
##     <title><![CDATA[A Comparison of Ballistic Resistance Testing Techniques in the Department of Defense]]></title>
##     <authors><![CDATA[Johnson, T.H.;  Freeman, L.;  Hester, J.;  Bell, J.L.]]></authors>
##     <affiliations><![CDATA[Inst. for Defense Anal., Alexandria, VA, USA]]></affiliations>
##     <controlledterms>
##       <term><![CDATA[Monte Carlo methods]]></term>
##       <term><![CDATA[ballistics]]></term>
##       <term><![CDATA[defence industry]]></term>
##       <term><![CDATA[mechanical testing]]></term>
##       <term><![CDATA[probability]]></term>
##       <term><![CDATA[projectiles]]></term>
##       <term><![CDATA[sensitivity analysis]]></term>
##     </controlledterms>
##     <thesaurusterms>
##       <term><![CDATA[Ballistics]]></term>
##       <term><![CDATA[Data models]]></term>
##       <term><![CDATA[Design of experiments]]></term>
##       <term><![CDATA[Maximum likelihood estimation]]></term>
##       <term><![CDATA[Projectiles]]></term>
##       <term><![CDATA[Sequential analysis]]></term>
##       <term><![CDATA[Testing]]></term>
##       <term><![CDATA[US Department of Defense]]></term>
##     </thesaurusterms>
##     <pubtitle><![CDATA[Access, IEEE]]></pubtitle>
##     <punumber><![CDATA[6287639]]></punumber>
##     <pubtype><![CDATA[Journals & Magazines]]></pubtype>
##     <publisher><![CDATA[IEEE]]></publisher>
##     <volume><![CDATA[2]]></volume>
##     <py><![CDATA[2014]]></py>
##     <spage><![CDATA[1442]]></spage>
##     <epage><![CDATA[1455]]></epage>
##     <abstract><![CDATA[Ballistic resistance testing is conducted in the Department of Defense (DoD) to estimate the probability that a projectile will perforate the armor of a system under test. Ballistic resistance testing routinely employs sensitivity experiment techniques where sequential test designs are used to estimate a particular quantile of the probability of perforation. Statistical procedures used to estimate the ballistic resistance of armor in the DoD have remained relatively unchanged for decades. In the current fiscal atmosphere of sequestration and budget deficits, efficiency is critical for test and evaluation. In this paper, we review and compare sequential methods, estimators, and stopping criteria used in the DoD to those found in literature. Using Monte Carlo simulation, we find that the three-phase optimal design, a probit model, and a break separation stopping criteria are most accurate and efficient at estimating V<sub>50</sub>, while the three-phase optimal design or Robbins-Monroe-Joseph method should be used to estimate V<sub>10</sub>.]]></abstract>
##     <issn><![CDATA[2169-3536]]></issn>
##     <htmlFlag><![CDATA[1]]></htmlFlag>
##     <arnumber><![CDATA[6974993]]></arnumber>
##     <doi><![CDATA[10.1109/ACCESS.2014.2377633]]></doi>
##     <publicationId><![CDATA[6974993]]></publicationId>
##     <mdurl><![CDATA[http://ieeexplore.ieee.org/xpl/articleDetails.jsp?tp=&arnumber=6974993&contentType=Journals+%26+Magazines]]></mdurl>
##     <pdf><![CDATA[http://ieeexplore.ieee.org/stamp/stamp.jsp?arnumber=6974993]]></pdf>
##   </document>
##   <document>
##     <rank>598</rank>
##     <title><![CDATA[VCSEL Based Coherent PONs]]></title>
##     <authors><![CDATA[Jensen, J.B.;  Rodes, R.;  Caballero, A.;  Ning Cheng;  Zibar, D.;  Monroy, I.T.]]></authors>
##     <affiliations><![CDATA[Dept. of Photonics Eng., Tech. Univ. of Denmark, Lyngby, Denmark]]></affiliations>
##     <controlledterms>
##       <term><![CDATA[laser cavity resonators]]></term>
##       <term><![CDATA[optical fibre couplers]]></term>
##       <term><![CDATA[optical links]]></term>
##       <term><![CDATA[optical modulation]]></term>
##       <term><![CDATA[optical receivers]]></term>
##       <term><![CDATA[optical transmitters]]></term>
##       <term><![CDATA[passive optical networks]]></term>
##       <term><![CDATA[semiconductor lasers]]></term>
##       <term><![CDATA[surface emitting lasers]]></term>
##     </controlledterms>
##     <thesaurusterms>
##       <term><![CDATA[Coherence]]></term>
##       <term><![CDATA[Optical modulation]]></term>
##       <term><![CDATA[Optical receivers]]></term>
##       <term><![CDATA[Optical signal processing]]></term>
##       <term><![CDATA[Optical transmitters]]></term>
##       <term><![CDATA[Passive optical networks]]></term>
##       <term><![CDATA[Vertical cavity surface emitting lasers]]></term>
##     </thesaurusterms>
##     <pubtitle><![CDATA[Lightwave Technology, Journal of]]></pubtitle>
##     <punumber><![CDATA[50]]></punumber>
##     <pubtype><![CDATA[Journals & Magazines]]></pubtype>
##     <publisher><![CDATA[IEEE]]></publisher>
##     <volume><![CDATA[32]]></volume>
##     <issue><![CDATA[8]]></issue>
##     <py><![CDATA[2014]]></py>
##     <spage><![CDATA[1423]]></spage>
##     <epage><![CDATA[1433]]></epage>
##     <abstract><![CDATA[We present a review of research performed in the area of coherent access technologies employing vertical cavity surface emitting lasers (VCSELs). Experimental demonstrations of optical transmission over a passive fiber link with coherent detection using VCSEL local oscillators and directly modulated VCSEL transmitters at bit rates up to 10 Gbps in the C-band as well as in the O-band are presented. The broad linewidth and frequency chirp associated with directly modulated VCSELs are utilized in an envelope detection receiver scheme which is demonstrated digitally (off-line) as well as analog (real-time). Additionally, it is shown that in the optical front-end of a coherent receiver for access networks, the 90 <sup>&#x00B0;</sup> hybrid can be replaced by a 3-dB coupler. The achieved results show that VCSELs are attractive light source candidates for transmitter as well as local oscillator for coherent detection PONs.]]></abstract>
##     <issn><![CDATA[0733-8724]]></issn>
##     <htmlFlag><![CDATA[1]]></htmlFlag>
##     <arnumber><![CDATA[6737252]]></arnumber>
##     <doi><![CDATA[10.1109/JLT.2014.2305572]]></doi>
##     <publicationId><![CDATA[6737252]]></publicationId>
##     <mdurl><![CDATA[http://ieeexplore.ieee.org/xpl/articleDetails.jsp?tp=&arnumber=6737252&contentType=Journals+%26+Magazines]]></mdurl>
##     <pdf><![CDATA[http://ieeexplore.ieee.org/stamp/stamp.jsp?arnumber=6737252]]></pdf>
##   </document>
##   <document>
##     <rank>599</rank>
##     <title><![CDATA[<sc>SymbexNet</sc>: Testing Network Protocol Implementations with Symbolic Execution and Rule-Based Specifications]]></title>
##     <authors><![CDATA[JaeSeung Song;  Cadar, C.;  Pietzuch, P.]]></authors>
##     <affiliations><![CDATA[Dept. of Comput. & Inf. Security, Sejong Univ., Seoul, South Korea]]></affiliations>
##     <controlledterms>
##       <term><![CDATA[formal specification]]></term>
##       <term><![CDATA[open systems]]></term>
##       <term><![CDATA[program debugging]]></term>
##       <term><![CDATA[program testing]]></term>
##     </controlledterms>
##     <thesaurusterms>
##       <term><![CDATA[Computer bugs]]></term>
##       <term><![CDATA[Concrete]]></term>
##       <term><![CDATA[IP networks]]></term>
##       <term><![CDATA[Interoperability]]></term>
##       <term><![CDATA[Protocols]]></term>
##       <term><![CDATA[Servers]]></term>
##       <term><![CDATA[Testing]]></term>
##     </thesaurusterms>
##     <pubtitle><![CDATA[Software Engineering, IEEE Transactions on]]></pubtitle>
##     <punumber><![CDATA[32]]></punumber>
##     <pubtype><![CDATA[Journals & Magazines]]></pubtype>
##     <publisher><![CDATA[IEEE]]></publisher>
##     <volume><![CDATA[40]]></volume>
##     <issue><![CDATA[7]]></issue>
##     <py><![CDATA[2014]]></py>
##     <spage><![CDATA[695]]></spage>
##     <epage><![CDATA[709]]></epage>
##     <abstract><![CDATA[Implementations of network protocols, such as DNS, DHCP and Zeroconf, are prone to flaws, security vulnerabilities and interoperability issues caused by developer mistakes and ambiguous requirements in protocol specifications. Detecting such problems is not easy because (i) many bugs manifest themselves only after prolonged operation; (ii) reasoning about semantic errors requires a machine-readable specification; and (iii) the state space of complex protocol implementations is large. This article presents a novel approach that combines symbolic execution and rule-based specifications to detect various types of flaws in network protocol implementations. The core idea behind our approach is to (1) automatically generate high-coverage test input packets for a network protocol implementation using single- and multi-packet exchange symbolic execution (targeting stateless and stateful protocols, respectively) and then (2) use these packets to detect potential violations of manual rules derived from the protocol specification, and check the interoperability of different implementations of the same network protocol. We present a system based on these techniques, SymbexNet, and evaluate it on multiple implementations of two network protocols: Zeroconf, a service discovery protocol, and DHCP, a network configuration protocol. SymbexNet is able to discover non-trivial bugs as well as interoperability problems, most of which have been confirmed by the developers.]]></abstract>
##     <issn><![CDATA[0098-5589]]></issn>
##     <htmlFlag><![CDATA[1]]></htmlFlag>
##     <arnumber><![CDATA[6815719]]></arnumber>
##     <doi><![CDATA[10.1109/TSE.2014.2323977]]></doi>
##     <publicationId><![CDATA[6815719]]></publicationId>
##     <mdurl><![CDATA[http://ieeexplore.ieee.org/xpl/articleDetails.jsp?tp=&arnumber=6815719&contentType=Journals+%26+Magazines]]></mdurl>
##     <pdf><![CDATA[http://ieeexplore.ieee.org/stamp/stamp.jsp?arnumber=6815719]]></pdf>
##   </document>
##   <document>
##     <rank>600</rank>
##     <title><![CDATA[Box-particle probability hypothesis density filtering]]></title>
##     <authors><![CDATA[Schikora, M.;  Gning, A.;  Mihaylova, L.;  Cremers, D.;  Koch, W.]]></authors>
##     <affiliations><![CDATA[Fraunhofer FKIE, Wachtberg, Germany]]></affiliations>
##     <controlledterms>
##       <term><![CDATA[Monte Carlo methods]]></term>
##       <term><![CDATA[particle filtering (numerical methods)]]></term>
##       <term><![CDATA[probability]]></term>
##       <term><![CDATA[sensor fusion]]></term>
##       <term><![CDATA[set theory]]></term>
##       <term><![CDATA[target tracking]]></term>
##     </controlledterms>
##     <thesaurusterms>
##       <term><![CDATA[Approximation methods]]></term>
##       <term><![CDATA[Atmospheric measurements]]></term>
##       <term><![CDATA[Noise measurement]]></term>
##       <term><![CDATA[Particle measurements]]></term>
##       <term><![CDATA[Target tracking]]></term>
##       <term><![CDATA[Uncertainty]]></term>
##     </thesaurusterms>
##     <pubtitle><![CDATA[Aerospace and Electronic Systems, IEEE Transactions on]]></pubtitle>
##     <punumber><![CDATA[7]]></punumber>
##     <pubtype><![CDATA[Journals & Magazines]]></pubtype>
##     <publisher><![CDATA[IEEE]]></publisher>
##     <volume><![CDATA[50]]></volume>
##     <issue><![CDATA[3]]></issue>
##     <py><![CDATA[2014]]></py>
##     <spage><![CDATA[1660]]></spage>
##     <epage><![CDATA[1672]]></epage>
##     <abstract><![CDATA[This paper develops a novel approach for multitarget tracking, called box-particle probability hypothesis density filter (box-PHD filter). The approach is able to track multiple targets and estimates the unknown number of targets. Furthermore, it is capable of dealing with three sources of uncertainty: stochastic, set-theoretic, and data association uncertainty. The box-PHD filter reduces the number of particles significantly, which improves the runtime considerably. The small number of box-particles makes this approach attractive for distributed inference, especially when particles have to be shared over networks. A box-particle is a random sample that occupies a small and controllable rectangular region of non-zero volume. Manipulation of boxes utilizes methods from the field of interval analysis. The theoretical derivation of the box-PHD filter is presented followed by a comparative analysis with a standard sequential Monte Carlo (SMC) version of the PHD filter. To measure the performance objectively three measures are used: inclusion, volume, and the optimum subpattern assignment (OSPA) metric. Our studies suggest that the box-PHD filter reaches similar accuracy results, like an SMC-PHD filter but with considerably less computational costs. Furthermore, we can show that in the presence of strongly biased measurement the box-PHD filter even outperforms the classical SMC-PHD filter.]]></abstract>
##     <issn><![CDATA[0018-9251]]></issn>
##     <htmlFlag><![CDATA[1]]></htmlFlag>
##     <arnumber><![CDATA[6965728]]></arnumber>
##     <doi><![CDATA[10.1109/TAES.2014.120238]]></doi>
##     <publicationId><![CDATA[6965728]]></publicationId>
##     <mdurl><![CDATA[http://ieeexplore.ieee.org/xpl/articleDetails.jsp?tp=&arnumber=6965728&contentType=Journals+%26+Magazines]]></mdurl>
##     <pdf><![CDATA[http://ieeexplore.ieee.org/stamp/stamp.jsp?arnumber=6965728]]></pdf>
##   </document>
## </root> 
## 
## [[2]]
## <root>
##   <totalfound>1195</totalfound>
##   <totalsearched>3953943</totalsearched>
##   <document>
##     <rank>601</rank>
##     <title><![CDATA[A 16 Kb Spin-Transfer Torque Random Access Memory With Self-Enable Switching and Precharge Sensing Schemes]]></title>
##     <authors><![CDATA[Li Zhang;  Weisheng Zhao;  Yiqi Zhuang;  Junlin Bao;  Gefei Wang;  Hualian Tang;  Cong Li;  Beilei Xu]]></authors>
##     <affiliations><![CDATA[Sch. of Microelectron., Xi'dian Univ., Xi'an, China]]></affiliations>
##     <controlledterms>
##       <term><![CDATA[MRAM devices]]></term>
##       <term><![CDATA[amplifiers]]></term>
##       <term><![CDATA[energy consumption]]></term>
##       <term><![CDATA[magnetic sensors]]></term>
##       <term><![CDATA[magnetic switching]]></term>
##       <term><![CDATA[reliability]]></term>
##     </controlledterms>
##     <thesaurusterms>
##       <term><![CDATA[Arrays]]></term>
##       <term><![CDATA[Reliability]]></term>
##       <term><![CDATA[Sensors]]></term>
##       <term><![CDATA[Switches]]></term>
##       <term><![CDATA[Switching circuits]]></term>
##       <term><![CDATA[Transistors]]></term>
##       <term><![CDATA[Writing]]></term>
##     </thesaurusterms>
##     <pubtitle><![CDATA[Magnetics, IEEE Transactions on]]></pubtitle>
##     <punumber><![CDATA[20]]></punumber>
##     <pubtype><![CDATA[Journals & Magazines]]></pubtype>
##     <publisher><![CDATA[IEEE]]></publisher>
##     <volume><![CDATA[50]]></volume>
##     <issue><![CDATA[4]]></issue>
##     <part><![CDATA[2]]></part>
##     <py><![CDATA[2014]]></py>
##     <spage><![CDATA[1]]></spage>
##     <epage><![CDATA[7]]></epage>
##     <abstract><![CDATA[Spin-transfer torque magnetic random access memory (STT-MRAM) is considered one of the most promising non-volatile memory candidates thanks to its excellent performance in terms of access speed, endurance, and compatibility to CMOS. However, high power supply voltage is required in the conventional STT-MRAM writing circuit, which results in high power consumption (e.g., ~10 pJ/bit). In addition, it suffers from stochastic switching behavior and process voltage temperature variations. These make power-efficient and reliable write/read circuits become critical challenges. In this paper, we present novel circuits and architectures to build a 16 kb STT-MRAM design with low power and high reliability. For example, the self-enable switching scheme reduces the power consumption effectively and the fore-placed sense amplifier improves the robustness to process variation. Using an accurate compact model of 65 nm STT-MRAM and a commercial CMOS design kit, mixed transient and statistical simulations have been performed to validate this design.]]></abstract>
##     <issn><![CDATA[0018-9464]]></issn>
##     <arnumber><![CDATA[6665113]]></arnumber>
##     <doi><![CDATA[10.1109/TMAG.2013.2291222]]></doi>
##     <publicationId><![CDATA[6665113]]></publicationId>
##     <mdurl><![CDATA[http://ieeexplore.ieee.org/xpl/articleDetails.jsp?tp=&arnumber=6665113&contentType=Journals+%26+Magazines]]></mdurl>
##     <pdf><![CDATA[http://ieeexplore.ieee.org/stamp/stamp.jsp?arnumber=6665113]]></pdf>
##   </document>
##   <document>
##     <rank>602</rank>
##     <title><![CDATA[All-Optical Logic Gate for XOR Operation Between 40-Gbaud QPSK Tributaries in an Ultra-Short Silicon Nanowire]]></title>
##     <authors><![CDATA[Zuoshan Yin;  Jian Wu;  Jizhao Zang;  Deming Kong;  Jifang Qiu;  Jindan Shi;  Wuyi Li;  Shile Wei;  Jintong Lin]]></authors>
##     <affiliations><![CDATA[State Key Lab. of Inf. Photonics & Opt. Commun., Beijing Univ. of Posts & Telecommun., Beijing, China]]></affiliations>
##     <controlledterms>
##       <term><![CDATA[elemental semiconductors]]></term>
##       <term><![CDATA[integrated optics]]></term>
##       <term><![CDATA[multiwave mixing]]></term>
##       <term><![CDATA[nanophotonics]]></term>
##       <term><![CDATA[nanowires]]></term>
##       <term><![CDATA[optical logic]]></term>
##       <term><![CDATA[optical losses]]></term>
##       <term><![CDATA[optical modulation]]></term>
##       <term><![CDATA[optical noise]]></term>
##       <term><![CDATA[phase modulation]]></term>
##       <term><![CDATA[quadrature phase shift keying]]></term>
##       <term><![CDATA[semiconductor quantum wires]]></term>
##       <term><![CDATA[silicon]]></term>
##       <term><![CDATA[two-photon processes]]></term>
##     </controlledterms>
##     <thesaurusterms>
##       <term><![CDATA[Bit error rate]]></term>
##       <term><![CDATA[Logic gates]]></term>
##       <term><![CDATA[Nonlinear optics]]></term>
##       <term><![CDATA[Optical fiber amplifiers]]></term>
##       <term><![CDATA[Optical filters]]></term>
##       <term><![CDATA[Phase shift keying]]></term>
##       <term><![CDATA[Silicon]]></term>
##     </thesaurusterms>
##     <pubtitle><![CDATA[Photonics Journal, IEEE]]></pubtitle>
##     <punumber><![CDATA[4563994]]></punumber>
##     <pubtype><![CDATA[Journals & Magazines]]></pubtype>
##     <publisher><![CDATA[IEEE]]></publisher>
##     <volume><![CDATA[6]]></volume>
##     <issue><![CDATA[3]]></issue>
##     <py><![CDATA[2014]]></py>
##     <spage><![CDATA[1]]></spage>
##     <epage><![CDATA[7]]></epage>
##     <abstract><![CDATA[We demonstrate an all-optical XOR logic gate for 40-Gbaud quadrature phase-shift keying (QPSK) tributaries in a silicon nanowire for the first time. The XOR logic operation is realized based on four-wave mixing (FWM) in the C-band. Experimental results show negligible power penalty at a bit error ratio (BER) of 10<sup>-6</sup> compared with the back-to-back signals. The BER floor appears at BER of 10<sup>-6</sup> for both I and Q tributaries of the logic operation, and the main reasons are the optical signal-to-noise ratio (OSNR) deterioration caused by nonlinear loss due to free carrier absorption (FCA) and two-photon absorption (TPA) and the cross-phase modulation (XPM)-induced phase impairments.]]></abstract>
##     <issn><![CDATA[1943-0655]]></issn>
##     <htmlFlag><![CDATA[1]]></htmlFlag>
##     <arnumber><![CDATA[6811147]]></arnumber>
##     <doi><![CDATA[10.1109/JPHOT.2014.2319101]]></doi>
##     <publicationId><![CDATA[6811147]]></publicationId>
##     <mdurl><![CDATA[http://ieeexplore.ieee.org/xpl/articleDetails.jsp?tp=&arnumber=6811147&contentType=Journals+%26+Magazines]]></mdurl>
##     <pdf><![CDATA[http://ieeexplore.ieee.org/stamp/stamp.jsp?arnumber=6811147]]></pdf>
##   </document>
##   <document>
##     <rank>603</rank>
##     <title><![CDATA[Discovering and Profiling Overlapping Communities in Location-Based Social Networks]]></title>
##     <authors><![CDATA[Zhu Wang;  Daqing Zhang;  Xingshe Zhou;  Dingqi Yang;  Zhiyong Yu;  Zhiwen Yu]]></authors>
##     <affiliations><![CDATA[Sch. of Comput. Sci., Northwestern Polytech. Univ., Xi'an, China]]></affiliations>
##     <controlledterms>
##       <term><![CDATA[mobile computing]]></term>
##       <term><![CDATA[pattern clustering]]></term>
##       <term><![CDATA[social networking (online)]]></term>
##     </controlledterms>
##     <pubtitle><![CDATA[Systems, Man, and Cybernetics: Systems, IEEE Transactions on]]></pubtitle>
##     <punumber><![CDATA[6221021]]></punumber>
##     <pubtype><![CDATA[Journals & Magazines]]></pubtype>
##     <publisher><![CDATA[IEEE]]></publisher>
##     <volume><![CDATA[44]]></volume>
##     <issue><![CDATA[4]]></issue>
##     <py><![CDATA[2014]]></py>
##     <spage><![CDATA[499]]></spage>
##     <epage><![CDATA[509]]></epage>
##     <abstract><![CDATA[With the recent surge of location-based social networks (LBSNs), such as Foursquare and Facebook Places, huge digital footprints of people's locations, profiles, and online social connections become accessible to service providers. Unlike social networks (e.g., Flickr, Facebook) that have explicit groups for users to subscribe to or join, LBSNs usually have no explicit community structure. In order to capitalize on the large number of potential users, quality community detection and profiling approaches are needed. In the meantime, the diversity of people's interests and behaviors when using LBSNs suggests that their community structures overlap. In this paper, based on the user check-in traces at venues and user/venue attributes, we come out with a novel multimode multi-attribute edge-centric coclustering framework to discover the overlapping and hierarchical communities of LBSNs users. By employing both intermode and intramode features, the proposed framework is not only able to group like-minded users from different social perspectives but also discover communities with explicit profiles indicating the interests of community members. The efficacy of our approach is validated by intensive empirical evaluations using the collected Foursquare dataset.]]></abstract>
##     <issn><![CDATA[2168-2216]]></issn>
##     <htmlFlag><![CDATA[1]]></htmlFlag>
##     <arnumber><![CDATA[6522477]]></arnumber>
##     <doi><![CDATA[10.1109/TSMC.2013.2256890]]></doi>
##     <publicationId><![CDATA[6522477]]></publicationId>
##     <mdurl><![CDATA[http://ieeexplore.ieee.org/xpl/articleDetails.jsp?tp=&arnumber=6522477&contentType=Journals+%26+Magazines]]></mdurl>
##     <pdf><![CDATA[http://ieeexplore.ieee.org/stamp/stamp.jsp?arnumber=6522477]]></pdf>
##   </document>
##   <document>
##     <rank>604</rank>
##     <title><![CDATA[Radiometer Calibration Using Colocated GPS Radio Occultation Measurements]]></title>
##     <authors><![CDATA[Blackwell, W.J.;  Bishop, R.;  Cahoy, K.;  Cohen, B.;  Crail, C.;  Cucurull, L.;  Dave, P.;  DiLiberto, M.;  Erickson, N.;  Fish, C.;  Shu-peng Ho;  Leslie, R.V.;  Milstein, A.B.;  Osaretin, I.A.]]></authors>
##     <affiliations><![CDATA[Lincoln Lab., Massachusetts Inst. of Technol., Lexington, MA, USA]]></affiliations>
##     <controlledterms>
##       <term><![CDATA[Global Positioning System]]></term>
##       <term><![CDATA[antennas]]></term>
##       <term><![CDATA[atmospheric techniques]]></term>
##       <term><![CDATA[calibration]]></term>
##       <term><![CDATA[diodes]]></term>
##       <term><![CDATA[microwave measurement]]></term>
##       <term><![CDATA[occultations]]></term>
##       <term><![CDATA[radio receivers]]></term>
##       <term><![CDATA[radiometers]]></term>
##       <term><![CDATA[radiotelemetry]]></term>
##       <term><![CDATA[space vehicles]]></term>
##       <term><![CDATA[stratosphere]]></term>
##     </controlledterms>
##     <thesaurusterms>
##       <term><![CDATA[Atmospheric modeling]]></term>
##       <term><![CDATA[Brightness temperature]]></term>
##       <term><![CDATA[Calibration]]></term>
##       <term><![CDATA[Microwave radiometry]]></term>
##       <term><![CDATA[Microwave theory and techniques]]></term>
##       <term><![CDATA[Refractive index]]></term>
##       <term><![CDATA[Terrestrial atmosphere]]></term>
##     </thesaurusterms>
##     <pubtitle><![CDATA[Geoscience and Remote Sensing, IEEE Transactions on]]></pubtitle>
##     <punumber><![CDATA[36]]></punumber>
##     <pubtype><![CDATA[Journals & Magazines]]></pubtype>
##     <publisher><![CDATA[IEEE]]></publisher>
##     <volume><![CDATA[52]]></volume>
##     <issue><![CDATA[10]]></issue>
##     <py><![CDATA[2014]]></py>
##     <spage><![CDATA[6423]]></spage>
##     <epage><![CDATA[6433]]></epage>
##     <abstract><![CDATA[We present a new high-fidelity method of calibrating a cross-track scanning microwave radiometer using Global Positioning System (GPS) radio occultation (GPSRO) measurements. The radiometer and GPSRO receiver periodically observe the same volume of atmosphere near the Earth's limb, and these overlapping measurements are used to calibrate the radiometer. Performance analyses show that absolute calibration accuracy better than 0.25 K is achievable for temperature sounding channels in the 50-60-GHz band for a total-power radiometer using a weakly coupled noise diode for frequent calibration and proximal GPSRO measurements for infrequent (approximately daily) calibration. The method requires GPSRO penetration depth only down to the stratosphere, thus permitting the use of a relatively small GPS antenna. Furthermore, only coarse spacecraft angular knowledge (approximately one degree rms) is required for the technique, as more precise angular knowledge can be retrieved directly from the combined radiometer and GPSRO data, assuming that the radiometer angular sampling is uniform. These features make the technique particularly well suited for implementation on a low-cost CubeSat hosting both radiometer and GPSRO receiver systems on the same spacecraft. We describe a validation platform for this calibration method, the Microwave Radiometer Technology Acceleration (MiRaTA) CubeSat, currently in development for the National Aeronautics and Space Administration (NASA) Earth Science Technology Office. MiRaTA will fly a multiband radiometer and the Compact TEC/Atmosphere GPS Sensor in 2015.]]></abstract>
##     <issn><![CDATA[0196-2892]]></issn>
##     <htmlFlag><![CDATA[1]]></htmlFlag>
##     <arnumber><![CDATA[6740017]]></arnumber>
##     <doi><![CDATA[10.1109/TGRS.2013.2296558]]></doi>
##     <publicationId><![CDATA[6740017]]></publicationId>
##     <mdurl><![CDATA[http://ieeexplore.ieee.org/xpl/articleDetails.jsp?tp=&arnumber=6740017&contentType=Journals+%26+Magazines]]></mdurl>
##     <pdf><![CDATA[http://ieeexplore.ieee.org/stamp/stamp.jsp?arnumber=6740017]]></pdf>
##   </document>
##   <document>
##     <rank>605</rank>
##     <title><![CDATA[Fiber Bragg Grating Cryosensors for Superconducting Accelerator Magnets]]></title>
##     <authors><![CDATA[Chiuchiolo, A.;  Bajko, M.;  Perez, J.C.;  Bajas, H.;  Consales, M.;  Giordano, M.;  Breglio, G.;  Cusano, A.]]></authors>
##     <affiliations><![CDATA[Dept. of Eng., Univ. of Sannio, Benevento, Italy]]></affiliations>
##     <controlledterms>
##       <term><![CDATA[Bragg gratings]]></term>
##       <term><![CDATA[cryogenics]]></term>
##       <term><![CDATA[fibre optic sensors]]></term>
##       <term><![CDATA[power transmission lines]]></term>
##       <term><![CDATA[strain measurement]]></term>
##       <term><![CDATA[superconducting cables]]></term>
##       <term><![CDATA[superconducting magnets]]></term>
##       <term><![CDATA[temperature measurement]]></term>
##     </controlledterms>
##     <thesaurusterms>
##       <term><![CDATA[Bragg gratings]]></term>
##       <term><![CDATA[Cryogenics]]></term>
##       <term><![CDATA[Fiber gratings]]></term>
##       <term><![CDATA[Optical fiber sensors]]></term>
##       <term><![CDATA[Strain]]></term>
##       <term><![CDATA[Superconducting magnets]]></term>
##       <term><![CDATA[Superconducting transmission lines]]></term>
##     </thesaurusterms>
##     <pubtitle><![CDATA[Photonics Journal, IEEE]]></pubtitle>
##     <punumber><![CDATA[4563994]]></punumber>
##     <pubtype><![CDATA[Journals & Magazines]]></pubtype>
##     <publisher><![CDATA[IEEE]]></publisher>
##     <volume><![CDATA[6]]></volume>
##     <issue><![CDATA[6]]></issue>
##     <py><![CDATA[2014]]></py>
##     <spage><![CDATA[1]]></spage>
##     <epage><![CDATA[10]]></epage>
##     <abstract><![CDATA[The design, fabrication, and tests of the new generation of superconducting magnets for the High Luminosity upgrade of the Large Hadron Collider (HL-LHC) require the support of an adequate sensing technology able to assure the integrity of the strain-sensitive and brittle superconducting cables through the whole service life of the magnet: assembly up to 150 MPa, cool down to 1.9 K, and powering up to about 16 kA. A precise temperature monitoring is also needed, in order to guarantee the safe working condition of the superconducting cables in the power transmission lines (SC-Link) designed to feed the magnet over long distance. Fiber Bragg Grating-based temperature and strain monitoring systems have been implemented in the first SC-Link prototype and in two subscale dipole magnets and tested in the cryogenic test facility at CERN, at 30 K, 77 K, and 1.9 K.]]></abstract>
##     <issn><![CDATA[1943-0655]]></issn>
##     <htmlFlag><![CDATA[1]]></htmlFlag>
##     <arnumber><![CDATA[6867322]]></arnumber>
##     <doi><![CDATA[10.1109/JPHOT.2014.2343994]]></doi>
##     <publicationId><![CDATA[6867322]]></publicationId>
##     <mdurl><![CDATA[http://ieeexplore.ieee.org/xpl/articleDetails.jsp?tp=&arnumber=6867322&contentType=Journals+%26+Magazines]]></mdurl>
##     <pdf><![CDATA[http://ieeexplore.ieee.org/stamp/stamp.jsp?arnumber=6867322]]></pdf>
##   </document>
##   <document>
##     <rank>606</rank>
##     <title><![CDATA[All-Fiber Low-Pedestal Spectral Compression in a Novel Architecture Based on an SMF Cascading an HNLF-NOLM]]></title>
##     <authors><![CDATA[Ying Chen;  Fan Yang;  Zhiyao Zhang;  Liang Pei;  Xiaojun Zhou;  Xiangning Chen;  Yong Liu]]></authors>
##     <affiliations><![CDATA[Dept. of Optoelectron., Acad. of Equip., Beijing, China]]></affiliations>
##     <controlledterms>
##       <term><![CDATA[mirrors]]></term>
##       <term><![CDATA[nonlinear optics]]></term>
##       <term><![CDATA[optical fibre communication]]></term>
##     </controlledterms>
##     <thesaurusterms>
##       <term><![CDATA[Chirp]]></term>
##       <term><![CDATA[Clocks]]></term>
##       <term><![CDATA[Optical fiber communication]]></term>
##       <term><![CDATA[Optical fiber devices]]></term>
##       <term><![CDATA[Optical fiber dispersion]]></term>
##       <term><![CDATA[Optical pulses]]></term>
##     </thesaurusterms>
##     <pubtitle><![CDATA[Photonics Journal, IEEE]]></pubtitle>
##     <punumber><![CDATA[4563994]]></punumber>
##     <pubtype><![CDATA[Journals & Magazines]]></pubtype>
##     <publisher><![CDATA[IEEE]]></publisher>
##     <volume><![CDATA[6]]></volume>
##     <issue><![CDATA[5]]></issue>
##     <py><![CDATA[2014]]></py>
##     <spage><![CDATA[1]]></spage>
##     <epage><![CDATA[9]]></epage>
##     <abstract><![CDATA[A novel all-fiber low-pedestal spectral compression scheme is proposed and demonstrated. The scheme is based on an anomalously dispersive single-mode fiber (SMF) cascading a nonlinear optical loop mirror with a highly nonlinear fiber (HNLF) in the loop. Both numerical and experimental results show that the spectral pedestal after spectral compression in the HNLF can be efficiently suppressed by the nonlinear optical loop mirror through the chirp-related intensity filtering effect. The measured spectral pedestal energy ratio is 9.59% using the proposed scheme, which is nearly a quarter of that using the conventional alternative based on an anomalously dispersive SMF followed by a feedthrough HNLF.]]></abstract>
##     <issn><![CDATA[1943-0655]]></issn>
##     <htmlFlag><![CDATA[1]]></htmlFlag>
##     <arnumber><![CDATA[6887338]]></arnumber>
##     <doi><![CDATA[10.1109/JPHOT.2014.2352642]]></doi>
##     <publicationId><![CDATA[6887338]]></publicationId>
##     <mdurl><![CDATA[http://ieeexplore.ieee.org/xpl/articleDetails.jsp?tp=&arnumber=6887338&contentType=Journals+%26+Magazines]]></mdurl>
##     <pdf><![CDATA[http://ieeexplore.ieee.org/stamp/stamp.jsp?arnumber=6887338]]></pdf>
##   </document>
##   <document>
##     <rank>607</rank>
##     <title><![CDATA[Large Iterative Multitier Ensemble Classifiers for Security of Big Data]]></title>
##     <authors><![CDATA[Abawajy, J.H.;  Kelarev, A.;  Chowdhury, M.]]></authors>
##     <affiliations><![CDATA[Sch. of Inf. Technol., Deakin Univ., Geelong, VIC, Australia]]></affiliations>
##     <controlledterms>
##       <term><![CDATA[Big Data]]></term>
##       <term><![CDATA[iterative methods]]></term>
##       <term><![CDATA[learning (artificial intelligence)]]></term>
##       <term><![CDATA[pattern classification]]></term>
##       <term><![CDATA[security of data]]></term>
##     </controlledterms>
##     <thesaurusterms>
##       <term><![CDATA[Big data]]></term>
##       <term><![CDATA[Data handling]]></term>
##       <term><![CDATA[Data mining]]></term>
##       <term><![CDATA[Data storage systems]]></term>
##       <term><![CDATA[Information management]]></term>
##       <term><![CDATA[Iterative methods]]></term>
##       <term><![CDATA[Malware]]></term>
##     </thesaurusterms>
##     <pubtitle><![CDATA[Emerging Topics in Computing, IEEE Transactions on]]></pubtitle>
##     <punumber><![CDATA[6245516]]></punumber>
##     <pubtype><![CDATA[Journals & Magazines]]></pubtype>
##     <publisher><![CDATA[IEEE]]></publisher>
##     <volume><![CDATA[2]]></volume>
##     <issue><![CDATA[3]]></issue>
##     <py><![CDATA[2014]]></py>
##     <spage><![CDATA[352]]></spage>
##     <epage><![CDATA[363]]></epage>
##     <abstract><![CDATA[This paper introduces and investigates large iterative multitier ensemble (LIME) classifiers specifically tailored for big data. These classifiers are very large, but are quite easy to generate and use. They can be so large that it makes sense to use them only for big data. They are generated automatically as a result of several iterations in applying ensemble meta classifiers. They incorporate diverse ensemble meta classifiers into several tiers simultaneously and combine them into one automatically generated iterative system so that many ensemble meta classifiers function as integral parts of other ensemble meta classifiers at higher tiers. In this paper, we carry out a comprehensive investigation of the performance of LIME classifiers for a problem concerning security of big data. Our experiments compare LIME classifiers with various base classifiers and standard ordinary ensemble meta classifiers. The results obtained demonstrate that LIME classifiers can significantly increase the accuracy of classifications. LIME classifiers performed better than the base classifiers and standard ensemble meta classifiers.]]></abstract>
##     <issn><![CDATA[2168-6750]]></issn>
##     <htmlFlag><![CDATA[1]]></htmlFlag>
##     <arnumber><![CDATA[6808522]]></arnumber>
##     <doi><![CDATA[10.1109/TETC.2014.2316510]]></doi>
##     <publicationId><![CDATA[6808522]]></publicationId>
##     <mdurl><![CDATA[http://ieeexplore.ieee.org/xpl/articleDetails.jsp?tp=&arnumber=6808522&contentType=Journals+%26+Magazines]]></mdurl>
##     <pdf><![CDATA[http://ieeexplore.ieee.org/stamp/stamp.jsp?arnumber=6808522]]></pdf>
##   </document>
##   <document>
##     <rank>608</rank>
##     <title><![CDATA[A Comprehensive Review of Stability Analysis of Continuous-Time Recurrent Neural Networks]]></title>
##     <authors><![CDATA[Huaguang Zhang;  Zhanshan Wang;  Derong Liu]]></authors>
##     <affiliations><![CDATA[Sch. of Inf. Sci. & Eng., Northeastern Univ., Shenyang, China]]></affiliations>
##     <controlledterms>
##       <term><![CDATA[delays]]></term>
##       <term><![CDATA[linear matrix inequalities]]></term>
##       <term><![CDATA[recurrent neural nets]]></term>
##       <term><![CDATA[stability]]></term>
##     </controlledterms>
##     <thesaurusterms>
##       <term><![CDATA[Biological neural networks]]></term>
##       <term><![CDATA[Delays]]></term>
##       <term><![CDATA[Neurons]]></term>
##       <term><![CDATA[Recurrent neural networks]]></term>
##       <term><![CDATA[Stability criteria]]></term>
##     </thesaurusterms>
##     <pubtitle><![CDATA[Neural Networks and Learning Systems, IEEE Transactions on]]></pubtitle>
##     <punumber><![CDATA[5962385]]></punumber>
##     <pubtype><![CDATA[Journals & Magazines]]></pubtype>
##     <publisher><![CDATA[IEEE]]></publisher>
##     <volume><![CDATA[25]]></volume>
##     <issue><![CDATA[7]]></issue>
##     <py><![CDATA[2014]]></py>
##     <spage><![CDATA[1229]]></spage>
##     <epage><![CDATA[1262]]></epage>
##     <abstract><![CDATA[Stability problems of continuous-time recurrent neural networks have been extensively studied, and many papers have been published in the literature. The purpose of this paper is to provide a comprehensive review of the research on stability of continuous-time recurrent neural networks, including Hopfield neural networks, Cohen-Grossberg neural networks, and related models. Since time delay is inevitable in practice, stability results of recurrent neural networks with different classes of time delays are reviewed in detail. For the case of delay-dependent stability, the results on how to deal with the constant/variable delay in recurrent neural networks are summarized. The relationship among stability results in different forms, such as algebraic inequality forms, M-matrix forms, linear matrix inequality forms, and Lyapunov diagonal stability forms, is discussed and compared. Some necessary and sufficient stability conditions for recurrent neural networks without time delays are also discussed. Concluding remarks and future directions of stability analysis of recurrent neural networks are given.]]></abstract>
##     <issn><![CDATA[2162-237X]]></issn>
##     <htmlFlag><![CDATA[1]]></htmlFlag>
##     <arnumber><![CDATA[6814892]]></arnumber>
##     <doi><![CDATA[10.1109/TNNLS.2014.2317880]]></doi>
##     <publicationId><![CDATA[6814892]]></publicationId>
##     <mdurl><![CDATA[http://ieeexplore.ieee.org/xpl/articleDetails.jsp?tp=&arnumber=6814892&contentType=Journals+%26+Magazines]]></mdurl>
##     <pdf><![CDATA[http://ieeexplore.ieee.org/stamp/stamp.jsp?arnumber=6814892]]></pdf>
##   </document>
##   <document>
##     <rank>609</rank>
##     <title><![CDATA[Performance Analysis of Relay-Based Cooperative Spectrum Sensing in Cognitive Radio Networks Over Non-Identical Nakagami-<named-content content-type="math" xlink:type="simple"> <inline-formula> <img src="/images/tex/254.gif" alt="m"> </inline-formula></named-content> Channels]]></title>
##     <authors><![CDATA[Hussain, S.;  Fernando, X.N.]]></authors>
##     <affiliations><![CDATA[Dept. of Electr. & Comput. Eng., Ryerson Univ., Toronto, ON, Canada]]></affiliations>
##     <controlledterms>
##       <term><![CDATA[Nakagami channels]]></term>
##       <term><![CDATA[amplify and forward communication]]></term>
##       <term><![CDATA[cognitive radio]]></term>
##       <term><![CDATA[computational complexity]]></term>
##       <term><![CDATA[fading channels]]></term>
##       <term><![CDATA[radio spectrum management]]></term>
##       <term><![CDATA[relay networks (telecommunication)]]></term>
##       <term><![CDATA[signal detection]]></term>
##       <term><![CDATA[telecommunication network reliability]]></term>
##     </controlledterms>
##     <thesaurusterms>
##       <term><![CDATA[Cascading style sheets]]></term>
##       <term><![CDATA[Fading]]></term>
##       <term><![CDATA[Noise]]></term>
##       <term><![CDATA[Performance analysis]]></term>
##       <term><![CDATA[Protocols]]></term>
##       <term><![CDATA[Relays]]></term>
##       <term><![CDATA[Sensors]]></term>
##     </thesaurusterms>
##     <pubtitle><![CDATA[Communications, IEEE Transactions on]]></pubtitle>
##     <punumber><![CDATA[26]]></punumber>
##     <pubtype><![CDATA[Journals & Magazines]]></pubtype>
##     <publisher><![CDATA[IEEE]]></publisher>
##     <volume><![CDATA[62]]></volume>
##     <issue><![CDATA[8]]></issue>
##     <py><![CDATA[2014]]></py>
##     <spage><![CDATA[2733]]></spage>
##     <epage><![CDATA[2746]]></epage>
##     <abstract><![CDATA[This paper provides performance analysis of relay-based cognitive radio (CR) networks and presents a detect-amplify-and-forward (DAF) relaying strategy for cooperative spectrum sensing over non-identical Nakagami-m fading channels. An advanced statistical approach is introduced to derive new exact closed-form expressions for average false alarm probability and average detection probability. We also introduce a novel approximation to alleviate the computational complexity of the proposed models. This paper points out the inconsistency of several assumptions that are typically used for performance analysis of CR networks and reveals that channel fading on the relaying links yields similar performance degradations as on the sensing channel. The study also shows that it is not necessary to incorporate all CRs in the cooperative process and that a small number of reliable radios are enough to achieve practical detection level. Compared with the amplify-and-forward strategy, refraining the heavily faded relays in the DAF strategy improves the detection accuracy and reduces the bandwidth requirement of the relaying links. The presented analysis could lead to intuitive system design guidelines for CR networks impaired with non-identical faded channels.]]></abstract>
##     <issn><![CDATA[0090-6778]]></issn>
##     <htmlFlag><![CDATA[1]]></htmlFlag>
##     <arnumber><![CDATA[6855340]]></arnumber>
##     <doi><![CDATA[10.1109/TCOMM.2014.2338856]]></doi>
##     <publicationId><![CDATA[6855340]]></publicationId>
##     <mdurl><![CDATA[http://ieeexplore.ieee.org/xpl/articleDetails.jsp?tp=&arnumber=6855340&contentType=Journals+%26+Magazines]]></mdurl>
##     <pdf><![CDATA[http://ieeexplore.ieee.org/stamp/stamp.jsp?arnumber=6855340]]></pdf>
##   </document>
##   <document>
##     <rank>610</rank>
##     <title><![CDATA[Spectral Calibration of Hyperspectral Data Observed From a Hyperspectrometer Loaded on an Unmanned Aerial Vehicle Platform]]></title>
##     <authors><![CDATA[Yaokai Liu;  Tianxing Wang;  Lingling Ma;  Ning Wang]]></authors>
##     <affiliations><![CDATA[Key Lab. of Quantitative Remote Sensing Inf. Technol., Acad. of Opto-Electron., Beijing, China]]></affiliations>
##     <controlledterms>
##       <term><![CDATA[geophysical image processing]]></term>
##       <term><![CDATA[hyperspectral imaging]]></term>
##       <term><![CDATA[remote sensing]]></term>
##     </controlledterms>
##     <thesaurusterms>
##       <term><![CDATA[Atmospheric measurements]]></term>
##       <term><![CDATA[Calibration]]></term>
##       <term><![CDATA[Hyperspectral imaging]]></term>
##       <term><![CDATA[Optical surface waves]]></term>
##       <term><![CDATA[Radiometry]]></term>
##       <term><![CDATA[Wavelength measurement]]></term>
##     </thesaurusterms>
##     <pubtitle><![CDATA[Selected Topics in Applied Earth Observations and Remote Sensing, IEEE Journal of]]></pubtitle>
##     <punumber><![CDATA[4609443]]></punumber>
##     <pubtype><![CDATA[Journals & Magazines]]></pubtype>
##     <publisher><![CDATA[IEEE]]></publisher>
##     <volume><![CDATA[7]]></volume>
##     <issue><![CDATA[6]]></issue>
##     <py><![CDATA[2014]]></py>
##     <spage><![CDATA[2630]]></spage>
##     <epage><![CDATA[2638]]></epage>
##     <abstract><![CDATA[Hyperspectral imaging has been widely applied in remote sensing scientific fields. For this study, hyperspectral imaging data covering the spectral region from 400 to 1000 nm were collected from an unmanned aerial vehicle visible/near-infrared imaging hyperspectrometer (UAV-VNIRIS). Theoretically, the spectral calibration parameters of the UAV-VNIRIS measured in the laboratory should be refined when applied to the hyperspectral data obtained from the UAV platform due to variations between the laboratory and actual flight environments. Therefore, accurate spectral calibration of the UAV-VNIRIS is essential to further applications of the hyperspectral data. Shifts in both the spectral center wavelength position and the full-width at half-maximum (FWHM) were retrieved using two different methods (Methods I and II) based on spectrum matching of atmospheric absorption features at oxygen bands near 760 nm and water vapor bands near 820 and 940 nm. Comparison of the spectral calibration results of these two methods over the calibration targets showed that the derived center wavelength and FWHM shifts are similar. For the UAV-VNIRIS observed data used here, the shifts in center wavelength derived from both Methods I and II over the three absorption bands are less than 0.13 nm, and less than 0.22 nm in terms of FWHM. The findings of this paper revealed: 1) the UAV-VNIRIS payload on the UAV platform performed well in terms of spectral calibration; and 2) the applied methods are effective for on-orbit spectral calibration of the hyper spectrometer.]]></abstract>
##     <issn><![CDATA[1939-1404]]></issn>
##     <htmlFlag><![CDATA[1]]></htmlFlag>
##     <arnumber><![CDATA[6849968]]></arnumber>
##     <doi><![CDATA[10.1109/JSTARS.2014.2329891]]></doi>
##     <publicationId><![CDATA[6849968]]></publicationId>
##     <mdurl><![CDATA[http://ieeexplore.ieee.org/xpl/articleDetails.jsp?tp=&arnumber=6849968&contentType=Journals+%26+Magazines]]></mdurl>
##     <pdf><![CDATA[http://ieeexplore.ieee.org/stamp/stamp.jsp?arnumber=6849968]]></pdf>
##   </document>
##   <document>
##     <rank>611</rank>
##     <title><![CDATA[The Ventilation Effect on Stator Convective Heat Transfer of an Axial-Flux Permanent-Magnet Machine]]></title>
##     <authors><![CDATA[Yew Chuan Chong;  Echenique Subiabre, E.J.P.;  Mueller, M.A.;  Chick, J.;  Staton, D.A.;  McDonald, A.S.]]></authors>
##     <affiliations><![CDATA[Inst. for Energy Syst., Univ. of Edinburgh, Edinburgh, UK]]></affiliations>
##     <controlledterms>
##       <term><![CDATA[computational fluid dynamics]]></term>
##       <term><![CDATA[heat transfer]]></term>
##       <term><![CDATA[permanent magnet machines]]></term>
##       <term><![CDATA[stators]]></term>
##       <term><![CDATA[ventilation]]></term>
##     </controlledterms>
##     <thesaurusterms>
##       <term><![CDATA[Coils]]></term>
##       <term><![CDATA[Iron]]></term>
##       <term><![CDATA[Magnetic cores]]></term>
##       <term><![CDATA[Magnetic flux]]></term>
##       <term><![CDATA[Rotors]]></term>
##       <term><![CDATA[Stators]]></term>
##       <term><![CDATA[Temperature measurement]]></term>
##     </thesaurusterms>
##     <pubtitle><![CDATA[Industrial Electronics, IEEE Transactions on]]></pubtitle>
##     <punumber><![CDATA[41]]></punumber>
##     <pubtype><![CDATA[Journals & Magazines]]></pubtype>
##     <publisher><![CDATA[IEEE]]></publisher>
##     <volume><![CDATA[61]]></volume>
##     <issue><![CDATA[8]]></issue>
##     <py><![CDATA[2014]]></py>
##     <spage><![CDATA[4392]]></spage>
##     <epage><![CDATA[4403]]></epage>
##     <abstract><![CDATA[This paper investigates the effect of the inlet configuration on cooling for an air-cooled axial-flux permanent-magnet (AFPM) machine. Temperature rises in the stator were measured and compared with results predicted using computational fluid dynamic (CFD) methods linked to a detailed machine loss characterization. It is found that an improved inlet design can significantly reduce the stator temperature rises. Comparison between the validated CFD model results and the values obtained from heat transfer correlations addresses the suitability of those correlations proposed specifically for AFPM machines.]]></abstract>
##     <issn><![CDATA[0278-0046]]></issn>
##     <htmlFlag><![CDATA[1]]></htmlFlag>
##     <arnumber><![CDATA[6616617]]></arnumber>
##     <doi><![CDATA[10.1109/TIE.2013.2284151]]></doi>
##     <publicationId><![CDATA[6616617]]></publicationId>
##     <mdurl><![CDATA[http://ieeexplore.ieee.org/xpl/articleDetails.jsp?tp=&arnumber=6616617&contentType=Journals+%26+Magazines]]></mdurl>
##     <pdf><![CDATA[http://ieeexplore.ieee.org/stamp/stamp.jsp?arnumber=6616617]]></pdf>
##   </document>
##   <document>
##     <rank>612</rank>
##     <title><![CDATA[A Low-Wear Onload Tap Changer Diverter Switch for Frequent Voltage Control on Distribution Networks]]></title>
##     <authors><![CDATA[Rogers, D.J.;  Green, T.C.;  Silversides, R.W.]]></authors>
##     <affiliations><![CDATA[Sch. of Eng., Cardiff Univ., Cardiff, UK]]></affiliations>
##     <controlledterms>
##       <term><![CDATA[actuators]]></term>
##       <term><![CDATA[air insulation]]></term>
##       <term><![CDATA[arcs (electric)]]></term>
##       <term><![CDATA[cantilevers]]></term>
##       <term><![CDATA[divertors]]></term>
##       <term><![CDATA[electric drives]]></term>
##       <term><![CDATA[electrical contacts]]></term>
##       <term><![CDATA[mechanical contact]]></term>
##       <term><![CDATA[on load tap changers]]></term>
##       <term><![CDATA[permanent magnets]]></term>
##       <term><![CDATA[power distribution control]]></term>
##       <term><![CDATA[power semiconductor switches]]></term>
##       <term><![CDATA[springs (mechanical)]]></term>
##       <term><![CDATA[voltage control]]></term>
##     </controlledterms>
##     <thesaurusterms>
##       <term><![CDATA[Coils]]></term>
##       <term><![CDATA[Contacts]]></term>
##       <term><![CDATA[Force]]></term>
##       <term><![CDATA[Magnetic separation]]></term>
##       <term><![CDATA[Springs]]></term>
##       <term><![CDATA[Switches]]></term>
##       <term><![CDATA[Torque]]></term>
##     </thesaurusterms>
##     <pubtitle><![CDATA[Power Delivery, IEEE Transactions on]]></pubtitle>
##     <punumber><![CDATA[61]]></punumber>
##     <pubtype><![CDATA[Journals & Magazines]]></pubtype>
##     <publisher><![CDATA[IEEE]]></publisher>
##     <volume><![CDATA[29]]></volume>
##     <issue><![CDATA[2]]></issue>
##     <py><![CDATA[2014]]></py>
##     <spage><![CDATA[860]]></spage>
##     <epage><![CDATA[869]]></epage>
##     <abstract><![CDATA[This paper presents a fast mechanical diverter switch design suitable for new &#x201C;arcless&#x201D; hybrid onload tap-changing systems. In such systems, arcing at contact separation and contact closure is almost completely eliminated by the inclusion of alternate current paths incorporating semiconductor devices. This allows the use of compact, air-insulated mechanical contacts that do not need to withstand significant arc erosion or provide arc quenching. As a result, the moving mass and the drive system for the switch may be dramatically reduced in size, leading to low inertia of the moving parts and resulting in very rapid operation times. An integrated, high-torque, low-mass permanent-magnet actuator is presented that provides detent (unpowered) contact force coupled with a cantilever spring contact system sized for an 11-kV 2-MVA onload tap changer. The design delivers operation times of under 20 ms and is capable of sustaining more than 10<sup>6</sup> operations. The complete design is experimentally verified under representative electrical conditions, and contact wear levels comparable to pure mechanical (zero current) operation are demonstrated.]]></abstract>
##     <issn><![CDATA[0885-8977]]></issn>
##     <htmlFlag><![CDATA[1]]></htmlFlag>
##     <arnumber><![CDATA[6573413]]></arnumber>
##     <doi><![CDATA[10.1109/TPWRD.2013.2272335]]></doi>
##     <publicationId><![CDATA[6573413]]></publicationId>
##     <mdurl><![CDATA[http://ieeexplore.ieee.org/xpl/articleDetails.jsp?tp=&arnumber=6573413&contentType=Journals+%26+Magazines]]></mdurl>
##     <pdf><![CDATA[http://ieeexplore.ieee.org/stamp/stamp.jsp?arnumber=6573413]]></pdf>
##   </document>
##   <document>
##     <rank>613</rank>
##     <title><![CDATA[A Survey of Agent-Based Modeling of Hospital Environments]]></title>
##     <authors><![CDATA[Friesen, M.R.;  McLeod, R.D.]]></authors>
##     <affiliations><![CDATA[Dept. of Electr. & Comput. Eng., Univ. of Manitoba, Winnipeg, MB, Canada]]></affiliations>
##     <controlledterms>
##       <term><![CDATA[emergency services]]></term>
##       <term><![CDATA[health care]]></term>
##       <term><![CDATA[hospitals]]></term>
##       <term><![CDATA[multi-agent systems]]></term>
##       <term><![CDATA[social sciences computing]]></term>
##     </controlledterms>
##     <thesaurusterms>
##       <term><![CDATA[Agent based modeling]]></term>
##       <term><![CDATA[Flow management]]></term>
##       <term><![CDATA[Hospitals]]></term>
##       <term><![CDATA[Modeling]]></term>
##       <term><![CDATA[Patient monitoring]]></term>
##       <term><![CDATA[Social factors]]></term>
##       <term><![CDATA[Statistical analysis]]></term>
##     </thesaurusterms>
##     <pubtitle><![CDATA[Access, IEEE]]></pubtitle>
##     <punumber><![CDATA[6287639]]></punumber>
##     <pubtype><![CDATA[Journals & Magazines]]></pubtype>
##     <publisher><![CDATA[IEEE]]></publisher>
##     <volume><![CDATA[2]]></volume>
##     <py><![CDATA[2014]]></py>
##     <spage><![CDATA[227]]></spage>
##     <epage><![CDATA[233]]></epage>
##     <abstract><![CDATA[Agent-based modeling has become a viable alternative and complement-to-traditional analysis methods for studying complex social environments. In this paper, we survey the role of agent-based modeling within hospital settings, where agent-based models investigate patient flow and other operational issues as well as the dynamics of infection spread within hospitals or hospital units. While there is a rich history of simulation and modeling of hospitals and hospital units, relatively little work exists, which applies agent-based models to this context.]]></abstract>
##     <issn><![CDATA[2169-3536]]></issn>
##     <htmlFlag><![CDATA[1]]></htmlFlag>
##     <arnumber><![CDATA[6778747]]></arnumber>
##     <doi><![CDATA[10.1109/ACCESS.2014.2313957]]></doi>
##     <publicationId><![CDATA[6778747]]></publicationId>
##     <mdurl><![CDATA[http://ieeexplore.ieee.org/xpl/articleDetails.jsp?tp=&arnumber=6778747&contentType=Journals+%26+Magazines]]></mdurl>
##     <pdf><![CDATA[http://ieeexplore.ieee.org/stamp/stamp.jsp?arnumber=6778747]]></pdf>
##   </document>
##   <document>
##     <rank>614</rank>
##     <title><![CDATA[A Nonlinear Mapping Approach to Stain Normalization in Digital Histopathology Images Using Image-Specific Color Deconvolution]]></title>
##     <authors><![CDATA[Khan, A.M.;  Rajpoot, N.;  Treanor, D.;  Magee, D.]]></authors>
##     <affiliations><![CDATA[Dept. of Comput. Sci., Univ. of Warwick, Coventry, UK]]></affiliations>
##     <controlledterms>
##       <term><![CDATA[decision support systems]]></term>
##       <term><![CDATA[deconvolution]]></term>
##       <term><![CDATA[image colour analysis]]></term>
##       <term><![CDATA[medical image processing]]></term>
##       <term><![CDATA[patient diagnosis]]></term>
##       <term><![CDATA[tumours]]></term>
##     </controlledterms>
##     <thesaurusterms>
##       <term><![CDATA[Deconvolution]]></term>
##       <term><![CDATA[Estimation]]></term>
##       <term><![CDATA[Histograms]]></term>
##       <term><![CDATA[Image analysis]]></term>
##       <term><![CDATA[Image color analysis]]></term>
##       <term><![CDATA[Licenses]]></term>
##       <term><![CDATA[Training]]></term>
##     </thesaurusterms>
##     <pubtitle><![CDATA[Biomedical Engineering, IEEE Transactions on]]></pubtitle>
##     <punumber><![CDATA[10]]></punumber>
##     <pubtype><![CDATA[Journals & Magazines]]></pubtype>
##     <publisher><![CDATA[IEEE]]></publisher>
##     <volume><![CDATA[61]]></volume>
##     <issue><![CDATA[6]]></issue>
##     <py><![CDATA[2014]]></py>
##     <spage><![CDATA[1729]]></spage>
##     <epage><![CDATA[1738]]></epage>
##     <abstract><![CDATA[Histopathology diagnosis is based on visual examination of the morphology of histological sections under a microscope. With the increasing popularity of digital slide scanners, decision support systems based on the analysis of digital pathology images are in high demand. However, computerized decision support systems are fraught with problems that stem from color variations in tissue appearance due to variation in tissue preparation, variation in stain reactivity from different manufacturers/batches, user or protocol variation, and the use of scanners from different manufacturers. In this paper, we present a novel approach to stain normalization in histopathology images. The method is based on nonlinear mapping of a source image to a target image using a representation derived from color deconvolution. Color deconvolution is a method to obtain stain concentration values when the stain matrix, describing how the color is affected by the stain concentration, is given. Rather than relying on standard stain matrices, which may be inappropriate for a given image, we propose the use of a color-based classifier that incorporates a novel stain color descriptor to calculate image-specific stain matrix. In order to demonstrate the efficacy of the proposed stain matrix estimation and stain normalization methods, they are applied to the problem of tumor segmentation in breast histopathology images. The experimental results suggest that the paradigm of color normalization, as a preprocessing step, can significantly help histological image analysis algorithms to demonstrate stable performance which is insensitive to imaging conditions in general and scanner variations in particular.]]></abstract>
##     <issn><![CDATA[0018-9294]]></issn>
##     <htmlFlag><![CDATA[1]]></htmlFlag>
##     <arnumber><![CDATA[6727397]]></arnumber>
##     <doi><![CDATA[10.1109/TBME.2014.2303294]]></doi>
##     <publicationId><![CDATA[6727397]]></publicationId>
##     <mdurl><![CDATA[http://ieeexplore.ieee.org/xpl/articleDetails.jsp?tp=&arnumber=6727397&contentType=Journals+%26+Magazines]]></mdurl>
##     <pdf><![CDATA[http://ieeexplore.ieee.org/stamp/stamp.jsp?arnumber=6727397]]></pdf>
##   </document>
##   <document>
##     <rank>615</rank>
##     <title><![CDATA[On-Wafer Noise Measurement at 300 GHz Using UTC-PD as Noise Source]]></title>
##     <authors><![CDATA[Ho-Jin Song;  Yaita, M.]]></authors>
##     <affiliations><![CDATA[Microsyst. Integration Labs., NTT Corp., Atsugi, Japan]]></affiliations>
##     <controlledterms>
##       <term><![CDATA[MMIC amplifiers]]></term>
##       <term><![CDATA[electric noise measurement]]></term>
##       <term><![CDATA[integrated circuit noise]]></term>
##       <term><![CDATA[millimetre wave amplifiers]]></term>
##       <term><![CDATA[millimetre wave diodes]]></term>
##       <term><![CDATA[millimetre wave measurement]]></term>
##       <term><![CDATA[millimetre wave mixers]]></term>
##       <term><![CDATA[photodiodes]]></term>
##       <term><![CDATA[submillimetre wave amplifiers]]></term>
##       <term><![CDATA[submillimetre wave diodes]]></term>
##       <term><![CDATA[submillimetre wave measurement]]></term>
##       <term><![CDATA[submillimetre wave mixers]]></term>
##       <term><![CDATA[waveguides]]></term>
##     </controlledterms>
##     <thesaurusterms>
##       <term><![CDATA[Noise]]></term>
##       <term><![CDATA[Noise measurement]]></term>
##       <term><![CDATA[Optical attenuators]]></term>
##       <term><![CDATA[Optical waveguides]]></term>
##       <term><![CDATA[Probes]]></term>
##       <term><![CDATA[Receivers]]></term>
##       <term><![CDATA[Temperature measurement]]></term>
##     </thesaurusterms>
##     <pubtitle><![CDATA[Microwave and Wireless Components Letters, IEEE]]></pubtitle>
##     <punumber><![CDATA[7260]]></punumber>
##     <pubtype><![CDATA[Journals & Magazines]]></pubtype>
##     <publisher><![CDATA[IEEE]]></publisher>
##     <volume><![CDATA[24]]></volume>
##     <issue><![CDATA[8]]></issue>
##     <py><![CDATA[2014]]></py>
##     <spage><![CDATA[578]]></spage>
##     <epage><![CDATA[580]]></epage>
##     <abstract><![CDATA[We demonstrate on-wafer noise measurement based on the Y-factor technique at 300 GHz utilizing a uni-traveling-carrier photodiode (UTC-PD) as a noise source. Since the UTC-PD can provide high and scalable noise power for the hot state and has a waveguide output port, easy and reliable measurement is available even at terahertz frequencies, where the loss from the experiment setup is not ignorable. We describe the experimental procedure with the UTC-PD as a noise source and a diode mixer as a noise receiver and present a noise figure measurement for a 300 GHz amplifier MMIC in an on-wafer environment. In addition, we discuss the uncertainty due to using the UTC-PD.]]></abstract>
##     <issn><![CDATA[1531-1309]]></issn>
##     <htmlFlag><![CDATA[1]]></htmlFlag>
##     <arnumber><![CDATA[6832643]]></arnumber>
##     <doi><![CDATA[10.1109/LMWC.2014.2324762]]></doi>
##     <publicationId><![CDATA[6832643]]></publicationId>
##     <mdurl><![CDATA[http://ieeexplore.ieee.org/xpl/articleDetails.jsp?tp=&arnumber=6832643&contentType=Journals+%26+Magazines]]></mdurl>
##     <pdf><![CDATA[http://ieeexplore.ieee.org/stamp/stamp.jsp?arnumber=6832643]]></pdf>
##   </document>
##   <document>
##     <rank>616</rank>
##     <title><![CDATA[A Single-Photon Avalanche Camera for Fluorescence Lifetime Imaging Microscopy and Correlation Spectroscopy]]></title>
##     <authors><![CDATA[Vitali, M.;  Bronzi, D.;  Krmpot, A.J.;  Nikolic, S.N.;  Schmitt, F.-J.;  Junghans, C.;  Tisa, S.;  Friedrich, T.;  Vukojevic, V.;  Terenius, L.;  Zappa, F.;  Rigler, R.]]></authors>
##     <affiliations><![CDATA[Dept. of Chem., Berlin Inst. of Technol., Berlin, Germany]]></affiliations>
##     <controlledterms>
##       <term><![CDATA[avalanche diodes]]></term>
##       <term><![CDATA[biochemistry]]></term>
##       <term><![CDATA[biodiffusion]]></term>
##       <term><![CDATA[cameras]]></term>
##       <term><![CDATA[cellular biophysics]]></term>
##       <term><![CDATA[fluorescence]]></term>
##       <term><![CDATA[high-speed optical techniques]]></term>
##       <term><![CDATA[molecular biophysics]]></term>
##       <term><![CDATA[optical microscopy]]></term>
##       <term><![CDATA[proteins]]></term>
##       <term><![CDATA[quantum dots]]></term>
##     </controlledterms>
##     <thesaurusterms>
##       <term><![CDATA[Arrays]]></term>
##       <term><![CDATA[Cameras]]></term>
##       <term><![CDATA[Laser beams]]></term>
##       <term><![CDATA[Logic gates]]></term>
##       <term><![CDATA[Microscopy]]></term>
##       <term><![CDATA[Optical imaging]]></term>
##       <term><![CDATA[Photonics]]></term>
##     </thesaurusterms>
##     <pubtitle><![CDATA[Selected Topics in Quantum Electronics, IEEE Journal of]]></pubtitle>
##     <punumber><![CDATA[2944]]></punumber>
##     <pubtype><![CDATA[Journals & Magazines]]></pubtype>
##     <publisher><![CDATA[IEEE]]></publisher>
##     <volume><![CDATA[20]]></volume>
##     <issue><![CDATA[6]]></issue>
##     <py><![CDATA[2014]]></py>
##     <spage><![CDATA[344]]></spage>
##     <epage><![CDATA[353]]></epage>
##     <abstract><![CDATA[Confocal laser scanning microscopy (CLSM) is commonly used to observe molecules of biological relevance in their native environment, the live cell, and study their spatial distribution and interactions nondestructively. CLSM can be easily extended to measure the lifetime of the excited state, the concentration and the diffusion properties of fluorescently labeled molecules, using fluorescence lifetime imaging microscopy (FLIM) and fluorescence correlation spectroscopy (FCS), respectively, in order to provide information about the local environment and the kinetics of molecular interaction in live cells. However, these parameters cannot be measured simultaneously using conventional CLSM due to damaging effects that are associated with strong illumination, including phototoxicity, photobleaching, and saturation of the fluorescence signal. To overcome these limitations, we have developed a new camera consisting of 1024 single-photon avalanche diodes that is optimized for multifocal microscopy, FLIM and FCS. We show proof-of-principle measurements of fluorescence intensity distribution and lifetime of the enhanced green fluorescent protein expressed in live cells and measurement of quantum dot diffusion in solution by FCS using the same detector.]]></abstract>
##     <issn><![CDATA[1077-260X]]></issn>
##     <htmlFlag><![CDATA[1]]></htmlFlag>
##     <arnumber><![CDATA[6843866]]></arnumber>
##     <doi><![CDATA[10.1109/JSTQE.2014.2333238]]></doi>
##     <publicationId><![CDATA[6843866]]></publicationId>
##     <mdurl><![CDATA[http://ieeexplore.ieee.org/xpl/articleDetails.jsp?tp=&arnumber=6843866&contentType=Journals+%26+Magazines]]></mdurl>
##     <pdf><![CDATA[http://ieeexplore.ieee.org/stamp/stamp.jsp?arnumber=6843866]]></pdf>
##   </document>
##   <document>
##     <rank>617</rank>
##     <title><![CDATA[Comparison of AWGs and Echelle Gratings for Wavelength Division Multiplexing on Silicon-on-Insulator]]></title>
##     <authors><![CDATA[Pathak, S.;  Dumon, P.;  Van Thourhout, D.;  Bogaerts, W.]]></authors>
##     <affiliations><![CDATA[Dept. of Inf. Technol., IMEC, Ghent Univ., Ghent, Belgium]]></affiliations>
##     <controlledterms>
##       <term><![CDATA[arrayed waveguide gratings]]></term>
##       <term><![CDATA[demultiplexing]]></term>
##       <term><![CDATA[diffraction gratings]]></term>
##       <term><![CDATA[optical crosstalk]]></term>
##       <term><![CDATA[optical losses]]></term>
##       <term><![CDATA[silicon-on-insulator]]></term>
##       <term><![CDATA[wavelength division multiplexing]]></term>
##     </controlledterms>
##     <thesaurusterms>
##       <term><![CDATA[Arrayed waveguide gratings]]></term>
##       <term><![CDATA[Arrays]]></term>
##       <term><![CDATA[Channel spacing]]></term>
##       <term><![CDATA[Crosstalk]]></term>
##       <term><![CDATA[Delays]]></term>
##       <term><![CDATA[Gratings]]></term>
##     </thesaurusterms>
##     <pubtitle><![CDATA[Photonics Journal, IEEE]]></pubtitle>
##     <punumber><![CDATA[4563994]]></punumber>
##     <pubtype><![CDATA[Journals & Magazines]]></pubtype>
##     <publisher><![CDATA[IEEE]]></publisher>
##     <volume><![CDATA[6]]></volume>
##     <issue><![CDATA[5]]></issue>
##     <py><![CDATA[2014]]></py>
##     <spage><![CDATA[1]]></spage>
##     <epage><![CDATA[9]]></epage>
##     <abstract><![CDATA[We compare the performance (insertion loss and crosstalk) of silicon-based arrayed waveguide gratings (AWGs) and echelle gratings for different channel spacings. For high-resolution de/multiplexer (DWDM) applications, AWGs are the better choice, whereas echelle gratings perform well for low-resolution de/multiplexer (CWDM) applications. Alternatively, for low-resolution de/multiplexer applications, the conventional box-shaped silicon AWG can be modified by an S-shaped AWG. We report crosstalk as low as -27 dB for regular AWGs, whereas in the S-shaped AWGs, the crosstalk is better than -19 dB, with an insertion loss below -2 dB. The crosstalk of the echelle gratings varies between -19 and -23 dB, with insertion loss below -2 dB.]]></abstract>
##     <issn><![CDATA[1943-0655]]></issn>
##     <htmlFlag><![CDATA[1]]></htmlFlag>
##     <arnumber><![CDATA[6916984]]></arnumber>
##     <doi><![CDATA[10.1109/JPHOT.2014.2361658]]></doi>
##     <publicationId><![CDATA[6916984]]></publicationId>
##     <mdurl><![CDATA[http://ieeexplore.ieee.org/xpl/articleDetails.jsp?tp=&arnumber=6916984&contentType=Journals+%26+Magazines]]></mdurl>
##     <pdf><![CDATA[http://ieeexplore.ieee.org/stamp/stamp.jsp?arnumber=6916984]]></pdf>
##   </document>
##   <document>
##     <rank>618</rank>
##     <title><![CDATA[A New Approach for Benefit Evaluation of Multiterminal VSC&#x2013;HVDC Using A Proposed Mixed AC/DC Optimal Power Flow]]></title>
##     <authors><![CDATA[Wang Feng;  Anh Le Tuan;  Tjernberg, L.B.;  Mannikoff, A.;  Bergman, A.]]></authors>
##     <affiliations><![CDATA[Electr. Power Eng., Chalmers Univ. of Technol., Gothenburg, Sweden]]></affiliations>
##     <controlledterms>
##       <term><![CDATA[HVDC power convertors]]></term>
##       <term><![CDATA[cost-benefit analysis]]></term>
##       <term><![CDATA[load flow]]></term>
##       <term><![CDATA[power transmission economics]]></term>
##     </controlledterms>
##     <thesaurusterms>
##       <term><![CDATA[Economics]]></term>
##       <term><![CDATA[HVDC transmission]]></term>
##       <term><![CDATA[Mathematical model]]></term>
##       <term><![CDATA[Planning]]></term>
##       <term><![CDATA[Power conversion]]></term>
##       <term><![CDATA[Reactive power]]></term>
##     </thesaurusterms>
##     <pubtitle><![CDATA[Power Delivery, IEEE Transactions on]]></pubtitle>
##     <punumber><![CDATA[61]]></punumber>
##     <pubtype><![CDATA[Journals & Magazines]]></pubtype>
##     <publisher><![CDATA[IEEE]]></publisher>
##     <volume><![CDATA[29]]></volume>
##     <issue><![CDATA[1]]></issue>
##     <py><![CDATA[2014]]></py>
##     <spage><![CDATA[432]]></spage>
##     <epage><![CDATA[443]]></epage>
##     <abstract><![CDATA[In this paper, an extended optimal power-flow (OPF) model incorporating a detailed model of a voltage-source converter-based-multiterminal high-voltage direct current system (VSC-MTDC) is proposed, hereafter referred to as the mixed ac/dc OPF (M-OPF) model. A cost-benefit analysis approach using the M-OPF model as the calculation engine is proposed to determine the preferred VSC-MTDC alternative to be installed in an existing ac transmission system. In this approach, the operational benefits of VSC-MTDC systems are evaluated against their investment costs to derive the benefit-to-cost ratios (BCR) which reflect the cost-effectiveness of the alternatives. A case study has been carried out using a modified Nordic 32-bus system. The results of the study show that VSC-MTDC systems might lead to a reduction in total operation cost, and the reduction of the total system transmission loss depends to a large extent on the VSC-MTDC configuration. The results from sensitivity analyses show that if the VSC loss could be reduced to a third of the original level, the total benefit from the system would be increased by about 70%. A suggestion for the placement and configuration of a VSC-MTDC system is made based on calculated BCRs.]]></abstract>
##     <issn><![CDATA[0885-8977]]></issn>
##     <htmlFlag><![CDATA[1]]></htmlFlag>
##     <arnumber><![CDATA[6571278]]></arnumber>
##     <doi><![CDATA[10.1109/TPWRD.2013.2267056]]></doi>
##     <publicationId><![CDATA[6571278]]></publicationId>
##     <mdurl><![CDATA[http://ieeexplore.ieee.org/xpl/articleDetails.jsp?tp=&arnumber=6571278&contentType=Journals+%26+Magazines]]></mdurl>
##     <pdf><![CDATA[http://ieeexplore.ieee.org/stamp/stamp.jsp?arnumber=6571278]]></pdf>
##   </document>
##   <document>
##     <rank>619</rank>
##     <title><![CDATA[Group Delay Dispersion Measurement From a Spectral Interferogram Based on the Cubic Phase Function]]></title>
##     <authors><![CDATA[Chongxiang Zeng;  Hongxia Zhang;  Dagong Jia;  Tiegen Liu;  Yimo Zhang]]></authors>
##     <affiliations><![CDATA[Coll. of Precision Instrum. & Optoelectron. Eng., Tianjin Univ., Tianjin, China]]></affiliations>
##     <controlledterms>
##       <term><![CDATA[light interference]]></term>
##       <term><![CDATA[light interferometry]]></term>
##       <term><![CDATA[optical fibre polarisation]]></term>
##       <term><![CDATA[optical information processing]]></term>
##       <term><![CDATA[optical variables measurement]]></term>
##     </controlledterms>
##     <thesaurusterms>
##       <term><![CDATA[Delays]]></term>
##       <term><![CDATA[Interference]]></term>
##       <term><![CDATA[Optical fiber dispersion]]></term>
##       <term><![CDATA[Optical fiber polarization]]></term>
##       <term><![CDATA[Optical interferometry]]></term>
##     </thesaurusterms>
##     <pubtitle><![CDATA[Photonics Journal, IEEE]]></pubtitle>
##     <punumber><![CDATA[4563994]]></punumber>
##     <pubtype><![CDATA[Journals & Magazines]]></pubtype>
##     <publisher><![CDATA[IEEE]]></publisher>
##     <volume><![CDATA[6]]></volume>
##     <issue><![CDATA[6]]></issue>
##     <py><![CDATA[2014]]></py>
##     <spage><![CDATA[1]]></spage>
##     <epage><![CDATA[9]]></epage>
##     <abstract><![CDATA[A novel group delay dispersion (GDD) measurement from a spectral interferogram is presented. Based on the cubic phase function (CPF) of the interference term, the GDD can be directly read from the ridge of CPF without phase retrieval and numerical differentiation operation. The proposed method is applied to measure the GDD and chromatic dispersion difference (CDD) of a polarization-maintaining fiber. The experimental results are unambiguous and insensitive to the filter choice of interference term calculation. GDD measurement can reach a resolution of 1 fs<sup>2</sup>, with a processing time of 25 s for 1001 wavelength points, and CDD measuring deviation is less than 0.5 fs/(km&#x00B7;nm) around the center wavelength. The method is expected to be suitable for processing all spectral interferograms with polynomial phases.]]></abstract>
##     <issn><![CDATA[1943-0655]]></issn>
##     <htmlFlag><![CDATA[1]]></htmlFlag>
##     <arnumber><![CDATA[6942169]]></arnumber>
##     <doi><![CDATA[10.1109/JPHOT.2014.2366117]]></doi>
##     <publicationId><![CDATA[6942169]]></publicationId>
##     <mdurl><![CDATA[http://ieeexplore.ieee.org/xpl/articleDetails.jsp?tp=&arnumber=6942169&contentType=Journals+%26+Magazines]]></mdurl>
##     <pdf><![CDATA[http://ieeexplore.ieee.org/stamp/stamp.jsp?arnumber=6942169]]></pdf>
##   </document>
##   <document>
##     <rank>620</rank>
##     <title><![CDATA[Coherent Time-Stretch Transform for Near-Field Spectroscopy]]></title>
##     <authors><![CDATA[DeVore, P.T.S.;  Buckley, B.W.;  Asghari, M.H.;  Solli, D.R.;  Jalali, B.]]></authors>
##     <affiliations><![CDATA[Dept. of Electr. Eng., Univ. of California, Los Angeles, Los Angeles, CA, USA]]></affiliations>
##     <controlledterms>
##       <term><![CDATA[infrared spectroscopy]]></term>
##       <term><![CDATA[light interferometry]]></term>
##       <term><![CDATA[transforms]]></term>
##       <term><![CDATA[ultraviolet spectroscopy]]></term>
##       <term><![CDATA[visible spectroscopy]]></term>
##     </controlledterms>
##     <thesaurusterms>
##       <term><![CDATA[Absorption]]></term>
##       <term><![CDATA[Bandwidth]]></term>
##       <term><![CDATA[Dispersion]]></term>
##       <term><![CDATA[Optical interferometry]]></term>
##       <term><![CDATA[Optical pulses]]></term>
##       <term><![CDATA[Phase measurement]]></term>
##       <term><![CDATA[Transforms]]></term>
##     </thesaurusterms>
##     <pubtitle><![CDATA[Photonics Journal, IEEE]]></pubtitle>
##     <punumber><![CDATA[4563994]]></punumber>
##     <pubtype><![CDATA[Journals & Magazines]]></pubtype>
##     <publisher><![CDATA[IEEE]]></publisher>
##     <volume><![CDATA[6]]></volume>
##     <issue><![CDATA[2]]></issue>
##     <py><![CDATA[2014]]></py>
##     <spage><![CDATA[1]]></spage>
##     <epage><![CDATA[7]]></epage>
##     <abstract><![CDATA[The coherent time-stretch transform enables high-throughput acquisition of complex optical fields in single-shot measurements. Full-field spectra are recovered via temporal interferometry on waveforms dispersed in the temporal near field. Real-time absorption spectra, including both amplitude and phase information, are acquired at 37 MHz.]]></abstract>
##     <issn><![CDATA[1943-0655]]></issn>
##     <htmlFlag><![CDATA[1]]></htmlFlag>
##     <arnumber><![CDATA[6777357]]></arnumber>
##     <doi><![CDATA[10.1109/JPHOT.2014.2312949]]></doi>
##     <publicationId><![CDATA[6777357]]></publicationId>
##     <mdurl><![CDATA[http://ieeexplore.ieee.org/xpl/articleDetails.jsp?tp=&arnumber=6777357&contentType=Journals+%26+Magazines]]></mdurl>
##     <pdf><![CDATA[http://ieeexplore.ieee.org/stamp/stamp.jsp?arnumber=6777357]]></pdf>
##   </document>
##   <document>
##     <rank>621</rank>
##     <title><![CDATA[A Comparative Study of Radiation-Tolerant Fiber Optic Sensors for Relative Humidity Monitoring in High-Radiation Environments at CERN]]></title>
##     <authors><![CDATA[Berruti, G.;  Consales, M.;  Borriello, A.;  Giordano, M.;  Buontempo, S.;  Makovec, A.;  Breglio, G.;  Petagna, P.;  Cusano, A.]]></authors>
##     <affiliations><![CDATA[Dept. of Eng., Univ. of Sannio, Benevento, Italy]]></affiliations>
##     <controlledterms>
##       <term><![CDATA[Bragg gratings]]></term>
##       <term><![CDATA[antireflection coatings]]></term>
##       <term><![CDATA[fibre optic sensors]]></term>
##       <term><![CDATA[humidity measurement]]></term>
##       <term><![CDATA[hygrometers]]></term>
##       <term><![CDATA[optical polymers]]></term>
##     </controlledterms>
##     <thesaurusterms>
##       <term><![CDATA[Bragg gratings]]></term>
##       <term><![CDATA[Fiber gratings]]></term>
##       <term><![CDATA[Humidity]]></term>
##       <term><![CDATA[Monitoring]]></term>
##       <term><![CDATA[Optical fiber sensors]]></term>
##       <term><![CDATA[Titanium compounds]]></term>
##     </thesaurusterms>
##     <pubtitle><![CDATA[Photonics Journal, IEEE]]></pubtitle>
##     <punumber><![CDATA[4563994]]></punumber>
##     <pubtype><![CDATA[Journals & Magazines]]></pubtype>
##     <publisher><![CDATA[IEEE]]></publisher>
##     <volume><![CDATA[6]]></volume>
##     <issue><![CDATA[6]]></issue>
##     <py><![CDATA[2014]]></py>
##     <spage><![CDATA[1]]></spage>
##     <epage><![CDATA[15]]></epage>
##     <abstract><![CDATA[In this paper, we report a comparative study of fiber optic sensors for applications of relative humidity (RH) monitoring in high-radiation environments. In particular, we present investigations carried out since 2011 by our multidisciplinary research group, in collaboration with the European Organization for Nuclear Research (CERN) in Geneva. Our research has been first focused on the development of polyimide-coated fiber Bragg grating (FBG) sensors, and recently, it has been extended to nanoscale metal oxide-coated long-period gratings (LPGs). Experimental tests in the [0-70] %RH range at different temperatures, before and after &#x03B3;-ionizing radiation exposures, have been carried out to assess the sensors' performances in conditions required in experiments running at CERN. The advantages and disadvantages of the two proposed technologies are discussed in this paper in light of their possible application in high-energy physics environments. In particular, reported results suggest that LPG-based sensors can be preferred in some applications (particularly in presence of very low humidity levels) mainly because they are able to provide very high RH sensitivity (up to 1.4 nm/% RH), which is up to three orders of magnitude higher than that exhibited by FBG-based hygrometers. On the other side, compared with FBGs, LPGs are more difficult to multiplex due to limitations in terms of available bandwidth.]]></abstract>
##     <issn><![CDATA[1943-0655]]></issn>
##     <htmlFlag><![CDATA[1]]></htmlFlag>
##     <arnumber><![CDATA[6909002]]></arnumber>
##     <doi><![CDATA[10.1109/JPHOT.2014.2357433]]></doi>
##     <publicationId><![CDATA[6909002]]></publicationId>
##     <mdurl><![CDATA[http://ieeexplore.ieee.org/xpl/articleDetails.jsp?tp=&arnumber=6909002&contentType=Journals+%26+Magazines]]></mdurl>
##     <pdf><![CDATA[http://ieeexplore.ieee.org/stamp/stamp.jsp?arnumber=6909002]]></pdf>
##   </document>
##   <document>
##     <rank>622</rank>
##     <title><![CDATA[Power Loss Reduction for MMSE-THP With Multidimensional Symbol Scaling]]></title>
##     <authors><![CDATA[Garcia-Rodriguez, A.;  Masouros, C.]]></authors>
##     <affiliations><![CDATA[Dept. of Electron. & Electr. Eng., Univ. Coll. London, London, UK]]></affiliations>
##     <controlledterms>
##       <term><![CDATA[least mean squares methods]]></term>
##       <term><![CDATA[multi-access systems]]></term>
##       <term><![CDATA[power consumption]]></term>
##       <term><![CDATA[precoding]]></term>
##       <term><![CDATA[radiofrequency interference]]></term>
##     </controlledterms>
##     <thesaurusterms>
##       <term><![CDATA[Complexity theory]]></term>
##       <term><![CDATA[Interference]]></term>
##       <term><![CDATA[Optimization]]></term>
##       <term><![CDATA[Power demand]]></term>
##       <term><![CDATA[Receivers]]></term>
##       <term><![CDATA[Signal to noise ratio]]></term>
##       <term><![CDATA[Transmitting antennas]]></term>
##     </thesaurusterms>
##     <pubtitle><![CDATA[Communications Letters, IEEE]]></pubtitle>
##     <punumber><![CDATA[4234]]></punumber>
##     <pubtype><![CDATA[Journals & Magazines]]></pubtype>
##     <publisher><![CDATA[IEEE]]></publisher>
##     <volume><![CDATA[18]]></volume>
##     <issue><![CDATA[7]]></issue>
##     <py><![CDATA[2014]]></py>
##     <spage><![CDATA[1147]]></spage>
##     <epage><![CDATA[1150]]></epage>
##     <abstract><![CDATA[This letter presents a strategy to reduce the power consumption of the Tomlinson-Harashima precoder (THP) based on the minimum mean square error (MMSE) criterion for the multi-user transmission. We show that a significant power loss reduction can be obtained by optimizing the interference to be cancelled by THP, using appropriate scaling of the interfering symbols. We advance the state of the art, by adopting a multidimensional optimization across a number of users and further modifying this optimization to apply to MMSE-THP, where it was previously inapplicable. By use of these improvements, the proposed approach is able to maintain or increase the error performance of MMSE-THP while providing up to 50% reduction in the power consumption.]]></abstract>
##     <issn><![CDATA[1089-7798]]></issn>
##     <htmlFlag><![CDATA[1]]></htmlFlag>
##     <arnumber><![CDATA[6817518]]></arnumber>
##     <doi><![CDATA[10.1109/LCOMM.2014.2325023]]></doi>
##     <publicationId><![CDATA[6817518]]></publicationId>
##     <mdurl><![CDATA[http://ieeexplore.ieee.org/xpl/articleDetails.jsp?tp=&arnumber=6817518&contentType=Journals+%26+Magazines]]></mdurl>
##     <pdf><![CDATA[http://ieeexplore.ieee.org/stamp/stamp.jsp?arnumber=6817518]]></pdf>
##   </document>
##   <document>
##     <rank>623</rank>
##     <title><![CDATA[Comparison of Two Two-Photon Dressed Rules in Multi-Wave Mixing]]></title>
##     <authors><![CDATA[Jia Sun;  Peiying Li;  Gaoping Huang;  Yunzhe Zhang;  Dan Zhang;  Hao Tian;  Heqing Huang;  Yanpeng Zhang]]></authors>
##     <affiliations><![CDATA[Key Lab. for Phys. Electron. & Devices, Xi'an Jiaotong Univ., Xi'an, China]]></affiliations>
##     <controlledterms>
##       <term><![CDATA[fluorescence]]></term>
##       <term><![CDATA[multiwave mixing]]></term>
##       <term><![CDATA[rubidium]]></term>
##       <term><![CDATA[transparency]]></term>
##     </controlledterms>
##     <thesaurusterms>
##       <term><![CDATA[Couplings]]></term>
##       <term><![CDATA[Energy states]]></term>
##       <term><![CDATA[Fluorescence]]></term>
##       <term><![CDATA[Laser beams]]></term>
##       <term><![CDATA[Laser excitation]]></term>
##       <term><![CDATA[Licenses]]></term>
##       <term><![CDATA[Probes]]></term>
##     </thesaurusterms>
##     <pubtitle><![CDATA[Photonics Journal, IEEE]]></pubtitle>
##     <punumber><![CDATA[4563994]]></punumber>
##     <pubtype><![CDATA[Journals & Magazines]]></pubtype>
##     <publisher><![CDATA[IEEE]]></publisher>
##     <volume><![CDATA[6]]></volume>
##     <issue><![CDATA[1]]></issue>
##     <py><![CDATA[2014]]></py>
##     <spage><![CDATA[1]]></spage>
##     <epage><![CDATA[9]]></epage>
##     <abstract><![CDATA[For the first time, we report a novel two-photon dressed rule in a five-level <sup>85</sup>Rb atomic system. In comparison with a traditional two-photon dressed rule where the signals move with probe field frequency detuning changing, the novel one indicates that the signals do not move with coupling field frequency detuning changing. The finding is verified by the experimentally detected transmitted probe, multi-wave mixing, and fluorescence signals. We also find that the fluorescence signals show Autler-Townes splitting, which are observed experimentally and explained theoretically. By scanning one coupling field frequency detuning, variation of the dressing strength of another coupling field is revealed. Finally, we demonstrate that the alternation can be phase controlled if one of the incident beams deviates an angle (i.e., change from a normal state to an abnormal state).]]></abstract>
##     <issn><![CDATA[1943-0655]]></issn>
##     <htmlFlag><![CDATA[1]]></htmlFlag>
##     <arnumber><![CDATA[6714394]]></arnumber>
##     <doi><![CDATA[10.1109/JPHOT.2014.2300498]]></doi>
##     <publicationId><![CDATA[6714394]]></publicationId>
##     <mdurl><![CDATA[http://ieeexplore.ieee.org/xpl/articleDetails.jsp?tp=&arnumber=6714394&contentType=Journals+%26+Magazines]]></mdurl>
##     <pdf><![CDATA[http://ieeexplore.ieee.org/stamp/stamp.jsp?arnumber=6714394]]></pdf>
##   </document>
##   <document>
##     <rank>624</rank>
##     <title><![CDATA[Quick Hypervolume]]></title>
##     <authors><![CDATA[Russo, L.M.S.;  Francisco, A.P.]]></authors>
##     <affiliations><![CDATA[Dept. de Eng. Inf., Univ. Tec. de Lisboa, Lisbon, Portugal]]></affiliations>
##     <controlledterms>
##       <term><![CDATA[evolutionary computation]]></term>
##     </controlledterms>
##     <thesaurusterms>
##       <term><![CDATA[Algorithm design and analysis]]></term>
##       <term><![CDATA[Arrays]]></term>
##       <term><![CDATA[Complexity theory]]></term>
##       <term><![CDATA[Indexes]]></term>
##       <term><![CDATA[Optimization]]></term>
##       <term><![CDATA[Prototypes]]></term>
##       <term><![CDATA[Radiation detectors]]></term>
##     </thesaurusterms>
##     <pubtitle><![CDATA[Evolutionary Computation, IEEE Transactions on]]></pubtitle>
##     <punumber><![CDATA[4235]]></punumber>
##     <pubtype><![CDATA[Journals & Magazines]]></pubtype>
##     <publisher><![CDATA[IEEE]]></publisher>
##     <volume><![CDATA[18]]></volume>
##     <issue><![CDATA[4]]></issue>
##     <py><![CDATA[2014]]></py>
##     <spage><![CDATA[481]]></spage>
##     <epage><![CDATA[502]]></epage>
##     <abstract><![CDATA[In this paper, we present a new algorithm for calculating exact hypervolumes. Given a set of d -dimensional points, it computes the hypervolume of the dominated space. Determining this value is an important subroutine of multiobjective evolutionary algorithms. We analyze the quick hypervolume (QHV) algorithm theoretically and experimentally. The theoretical results are a significant contribution to the current state of the art. Moreover, the experimental performance is also very competitive, compared with existing exact hypervolume algorithms.]]></abstract>
##     <issn><![CDATA[1089-778X]]></issn>
##     <htmlFlag><![CDATA[1]]></htmlFlag>
##     <arnumber><![CDATA[6595628]]></arnumber>
##     <doi><![CDATA[10.1109/TEVC.2013.2281525]]></doi>
##     <publicationId><![CDATA[6595628]]></publicationId>
##     <mdurl><![CDATA[http://ieeexplore.ieee.org/xpl/articleDetails.jsp?tp=&arnumber=6595628&contentType=Journals+%26+Magazines]]></mdurl>
##     <pdf><![CDATA[http://ieeexplore.ieee.org/stamp/stamp.jsp?arnumber=6595628]]></pdf>
##   </document>
##   <document>
##     <rank>625</rank>
##     <title><![CDATA[Compressed Vision Information Restoration Based on Cloud Prior and Local Prior]]></title>
##     <authors><![CDATA[Jiang, F.;  Ji, X.;  Hu, C.;  Liu, S.;  Zhao, D.]]></authors>
##     <affiliations><![CDATA[School of Computer Science and Technology, Harbin Institute of Technology, Harbin, China]]></affiliations>
##     <thesaurusterms>
##       <term><![CDATA[Degradation]]></term>
##       <term><![CDATA[Image coding]]></term>
##       <term><![CDATA[Image registration]]></term>
##       <term><![CDATA[Standards]]></term>
##       <term><![CDATA[Video coding]]></term>
##       <term><![CDATA[Visualization]]></term>
##       <term><![CDATA[Wireless communication]]></term>
##     </thesaurusterms>
##     <pubtitle><![CDATA[Access, IEEE]]></pubtitle>
##     <punumber><![CDATA[6287639]]></punumber>
##     <pubtype><![CDATA[Journals & Magazines]]></pubtype>
##     <publisher><![CDATA[IEEE]]></publisher>
##     <volume><![CDATA[2]]></volume>
##     <py><![CDATA[2014]]></py>
##     <spage><![CDATA[1117]]></spage>
##     <epage><![CDATA[1127]]></epage>
##     <abstract><![CDATA[In wireless communication, compressed vision information may suffer from kinds of degradation, which dramatically influences the final visual quality. In this paper, a compressed vision information restoration method is proposed based on two explored vision priors: 1) the cloud prior and 2) the local prior. The cloud prior can be obtained from the nature images set in the cloud, and fields of experts is used to formulate the statistical character of the nature image contents as a high order Markov random field. The local prior is achieved from the degraded image itself, and K-SVD is adopted to model the sparse and redundant representation characters of nature images. These priors are effectively comprised in the proposed vision information restoration method. The relation between the quantization parameter and the optimal configuration of the prior models is further analyzed. In addition, an enhanced quantization constrained projection algorithm is proposed to refine the high frequency components. We extend this paper to compressed video restoration for H.264/AVC and the experiment results demonstrate that the proposed scheme can reproduce higher quality images compared with conventional H.264/AVC.]]></abstract>
##     <issn><![CDATA[2169-3536]]></issn>
##     <htmlFlag><![CDATA[1]]></htmlFlag>
##     <arnumber><![CDATA[6887333]]></arnumber>
##     <doi><![CDATA[10.1109/ACCESS.2014.2353056]]></doi>
##     <publicationId><![CDATA[6887333]]></publicationId>
##     <mdurl><![CDATA[http://ieeexplore.ieee.org/xpl/articleDetails.jsp?tp=&arnumber=6887333&contentType=Journals+%26+Magazines]]></mdurl>
##     <pdf><![CDATA[http://ieeexplore.ieee.org/stamp/stamp.jsp?arnumber=6887333]]></pdf>
##   </document>
##   <document>
##     <rank>626</rank>
##     <title><![CDATA[1.2 V 10-bit 75 MS/s Pipelined ADC With Phase-Dependent Gain-Transition CDS]]></title>
##     <authors><![CDATA[Jong-Kwan Woo;  Hyunjoong Lee;  Hwi-Cheol Kim;  Deog-Kyoon Jeong;  Suhwan Kim]]></authors>
##     <affiliations><![CDATA[Dept. of Electr. Eng., Seoul Nat. Univ., Seoul, South Korea]]></affiliations>
##     <controlledterms>
##       <term><![CDATA[CMOS integrated circuits]]></term>
##       <term><![CDATA[analogue-digital conversion]]></term>
##       <term><![CDATA[digital-analogue conversion]]></term>
##       <term><![CDATA[operational amplifiers]]></term>
##       <term><![CDATA[sampling methods]]></term>
##     </controlledterms>
##     <pubtitle><![CDATA[Very Large Scale Integration (VLSI) Systems, IEEE Transactions on]]></pubtitle>
##     <punumber><![CDATA[92]]></punumber>
##     <pubtype><![CDATA[Journals & Magazines]]></pubtype>
##     <publisher><![CDATA[IEEE]]></publisher>
##     <volume><![CDATA[22]]></volume>
##     <issue><![CDATA[3]]></issue>
##     <py><![CDATA[2014]]></py>
##     <spage><![CDATA[585]]></spage>
##     <epage><![CDATA[592]]></epage>
##     <abstract><![CDATA[A phase-dependent gain-transition correlated double-sampling technique is proposed and applied to a 10-bit 75-MS/s pipelined analog-to-digital converter. This reduces the accumulation of predictive error of each multiplying digital-to-analog converter stage due to the finite gain of the operational amplifiers, without the need for additional capacitors and switches at the input. With a 10-MHz sinusoidal input, a prototype fabricated in a 0.13- &#x03BC;m CMOS process has a 56.90 dB signal-to-noise plus distortion ratio (SNDR) and a 64.57 dB spurious-free dynamic range (SFDR) at 75 MS/s. For a 37 MHz input at full sampling rate, the SNDR and SFDR are 55.01 and 60.77 dB, respectively. The IC has an active area of 0.65 mm<sup>2</sup> and consumes 32 mW with a 1.2 V supply.]]></abstract>
##     <issn><![CDATA[1063-8210]]></issn>
##     <htmlFlag><![CDATA[1]]></htmlFlag>
##     <arnumber><![CDATA[6494330]]></arnumber>
##     <doi><![CDATA[10.1109/TVLSI.2013.2251019]]></doi>
##     <publicationId><![CDATA[6494330]]></publicationId>
##     <mdurl><![CDATA[http://ieeexplore.ieee.org/xpl/articleDetails.jsp?tp=&arnumber=6494330&contentType=Journals+%26+Magazines]]></mdurl>
##     <pdf><![CDATA[http://ieeexplore.ieee.org/stamp/stamp.jsp?arnumber=6494330]]></pdf>
##   </document>
##   <document>
##     <rank>627</rank>
##     <title><![CDATA[On the Static Loop Modes in the Marching-on-in-Time Solution of the Time-Domain Electric Field Integral Equation]]></title>
##     <authors><![CDATA[Yifei Shi;  Bagci, H.;  Mingyu Lu]]></authors>
##     <affiliations><![CDATA[Div. of Comput., Electr., & Math. Sci. & Eng. (CEMSE), King Abdullah Univ. of Sci. & Eng. (KAUST), Thuwal, Saudi Arabia]]></affiliations>
##     <controlledterms>
##       <term><![CDATA[electric field integral equations]]></term>
##       <term><![CDATA[interpolation]]></term>
##       <term><![CDATA[matrix algebra]]></term>
##       <term><![CDATA[time-domain analysis]]></term>
##     </controlledterms>
##     <thesaurusterms>
##       <term><![CDATA[Antennas]]></term>
##       <term><![CDATA[Electric fields]]></term>
##       <term><![CDATA[Integral equations]]></term>
##       <term><![CDATA[Interpolation]]></term>
##       <term><![CDATA[Matrix decomposition]]></term>
##       <term><![CDATA[Null space]]></term>
##       <term><![CDATA[Time-domain analysis]]></term>
##     </thesaurusterms>
##     <pubtitle><![CDATA[Antennas and Wireless Propagation Letters, IEEE]]></pubtitle>
##     <punumber><![CDATA[7727]]></punumber>
##     <pubtype><![CDATA[Journals & Magazines]]></pubtype>
##     <publisher><![CDATA[IEEE]]></publisher>
##     <volume><![CDATA[13]]></volume>
##     <py><![CDATA[2014]]></py>
##     <spage><![CDATA[317]]></spage>
##     <epage><![CDATA[320]]></epage>
##     <abstract><![CDATA[When marching-on-in-time (MOT) method is applied to solve the time-domain electric field integral equation, spurious internal resonant and static loop modes are always observed in the solution. The internal resonant modes have recently been studied by the authors; this letter investigates the static loop modes. Like internal resonant modes, static loop modes, in theory, should not be observed in the MOT solution since they do not satisfy the zero initial conditions; their appearance is attributed to numerical errors. It is discussed in this letter that the dependence of spurious static loop modes on numerical errors is substantially different from that of spurious internal resonant modes. More specifically, when Rao-Wilton-Glisson functions and Lagrange interpolation functions are used as spatial and temporal basis functions, respectively, errors due to space-time discretization have no discernible impact on spurious static loop modes. Numerical experiments indeed support this discussion and demonstrate that the numerical errors due to the approximate solution of the MOT matrix system have dominant impact on spurious static loop modes in the MOT solution.]]></abstract>
##     <issn><![CDATA[1536-1225]]></issn>
##     <htmlFlag><![CDATA[1]]></htmlFlag>
##     <arnumber><![CDATA[6737267]]></arnumber>
##     <doi><![CDATA[10.1109/LAWP.2014.2305716]]></doi>
##     <publicationId><![CDATA[6737267]]></publicationId>
##     <mdurl><![CDATA[http://ieeexplore.ieee.org/xpl/articleDetails.jsp?tp=&arnumber=6737267&contentType=Journals+%26+Magazines]]></mdurl>
##     <pdf><![CDATA[http://ieeexplore.ieee.org/stamp/stamp.jsp?arnumber=6737267]]></pdf>
##   </document>
##   <document>
##     <rank>628</rank>
##     <title><![CDATA[Differential Evolution With Dynamic Parameters Selection for Optimization Problems]]></title>
##     <authors><![CDATA[Sarker, R.A.;  Elsayed, S.M.;  Ray, T.]]></authors>
##     <affiliations><![CDATA[Sch. of Eng. & Inf. Technol., Univ. of New South Wales, Canberra, ACT, Australia]]></affiliations>
##     <controlledterms>
##       <term><![CDATA[computational complexity]]></term>
##       <term><![CDATA[evolutionary computation]]></term>
##       <term><![CDATA[optimisation]]></term>
##       <term><![CDATA[parameter estimation]]></term>
##     </controlledterms>
##     <thesaurusterms>
##       <term><![CDATA[Algorithm design and analysis]]></term>
##       <term><![CDATA[Equations]]></term>
##       <term><![CDATA[Heuristic algorithms]]></term>
##       <term><![CDATA[Optimization]]></term>
##       <term><![CDATA[Sociology]]></term>
##       <term><![CDATA[Statistics]]></term>
##       <term><![CDATA[Vectors]]></term>
##     </thesaurusterms>
##     <pubtitle><![CDATA[Evolutionary Computation, IEEE Transactions on]]></pubtitle>
##     <punumber><![CDATA[4235]]></punumber>
##     <pubtype><![CDATA[Journals & Magazines]]></pubtype>
##     <publisher><![CDATA[IEEE]]></publisher>
##     <volume><![CDATA[18]]></volume>
##     <issue><![CDATA[5]]></issue>
##     <py><![CDATA[2014]]></py>
##     <spage><![CDATA[689]]></spage>
##     <epage><![CDATA[707]]></epage>
##     <abstract><![CDATA[Over the last few decades, a number of differential evolution (DE) algorithms have been proposed with excellent performance on mathematical benchmarks. However, like any other optimization algorithm, the success of DE is highly dependent on the search operators and control parameters that are often decided a priori. The selection of the parameter values is itself a combinatorial optimization problem. Although a considerable number of investigations have been conducted with regards to parameter selection, it is known to be a tedious task. In this paper, a DE algorithm is proposed that uses a new mechanism to dynamically select the best performing combinations of parameters (amplification factor, crossover rate, and the population size) for a problem during the course of a single run. The performance of the algorithm is judged by solving three well known sets of optimization test problems (two constrained and one unconstrained). The results demonstrate that the proposed algorithm not only saves the computational time, but also shows better performance over the state-of-the-art algorithms. The proposed mechanism can easily be applied to other population-based algorithms.]]></abstract>
##     <issn><![CDATA[1089-778X]]></issn>
##     <htmlFlag><![CDATA[1]]></htmlFlag>
##     <arnumber><![CDATA[6600821]]></arnumber>
##     <doi><![CDATA[10.1109/TEVC.2013.2281528]]></doi>
##     <publicationId><![CDATA[6600821]]></publicationId>
##     <mdurl><![CDATA[http://ieeexplore.ieee.org/xpl/articleDetails.jsp?tp=&arnumber=6600821&contentType=Journals+%26+Magazines]]></mdurl>
##     <pdf><![CDATA[http://ieeexplore.ieee.org/stamp/stamp.jsp?arnumber=6600821]]></pdf>
##   </document>
##   <document>
##     <rank>629</rank>
##     <title><![CDATA[A Novel Polarization Splitter Based on Dual-Core Photonic Crystal Fiber With a Liquid Crystal Modulation Core]]></title>
##     <authors><![CDATA[Chen, H.L.;  Li, S.G.;  Fan, Z.K.;  An, G.W.;  Li, J.S.;  Han, Y.]]></authors>
##     <affiliations><![CDATA[Key Lab. of Metastable Mater. Sci. & Technol., Yanshan Univ., Qinhuangdao, China]]></affiliations>
##     <controlledterms>
##       <term><![CDATA[birefringence]]></term>
##       <term><![CDATA[finite element analysis]]></term>
##       <term><![CDATA[holey fibres]]></term>
##       <term><![CDATA[liquid crystals]]></term>
##       <term><![CDATA[optical beam splitters]]></term>
##       <term><![CDATA[optical communication equipment]]></term>
##       <term><![CDATA[optical fibre communication]]></term>
##       <term><![CDATA[optical fibre fabrication]]></term>
##       <term><![CDATA[optical fibre polarisation]]></term>
##       <term><![CDATA[optical modulation]]></term>
##       <term><![CDATA[optical polarisers]]></term>
##       <term><![CDATA[photonic crystals]]></term>
##     </controlledterms>
##     <thesaurusterms>
##       <term><![CDATA[Bandwidth]]></term>
##       <term><![CDATA[Couplings]]></term>
##       <term><![CDATA[Erbium]]></term>
##       <term><![CDATA[Extinction ratio]]></term>
##       <term><![CDATA[Fabrication]]></term>
##       <term><![CDATA[Modulation]]></term>
##       <term><![CDATA[Photonic crystal fibers]]></term>
##     </thesaurusterms>
##     <pubtitle><![CDATA[Photonics Journal, IEEE]]></pubtitle>
##     <punumber><![CDATA[4563994]]></punumber>
##     <pubtype><![CDATA[Journals & Magazines]]></pubtype>
##     <publisher><![CDATA[IEEE]]></publisher>
##     <volume><![CDATA[6]]></volume>
##     <issue><![CDATA[4]]></issue>
##     <py><![CDATA[2014]]></py>
##     <spage><![CDATA[1]]></spage>
##     <epage><![CDATA[9]]></epage>
##     <abstract><![CDATA[A novel polarization splitter based on dual-core silica glass photonic crystal fiber with a liquid crystal modulation core is studied by the finite-element method. The mode birefringence is enlarged greatly with the infilling of nematic liquid crystal of E7. The simulation results demonstrate that the polarization splitter has an ultrabroad bandwidth of 250 nm, covering the E S C L optical communication bands, with the extinction ratio better than -20 dB. The separate length is 0.175 mm, and the extinction ratio is -80.7 dB at the communication wavelength of 1550 nm. The polarization splitter exhibits satisfactory splitter performance as the fabrication deviation reaches to 1%. The extinction ratio maintains better than -20 dB, at the C L optical communication bands, as the temperature increases from 15 &#x00B0;C to 50 &#x00B0;C.]]></abstract>
##     <issn><![CDATA[1943-0655]]></issn>
##     <htmlFlag><![CDATA[1]]></htmlFlag>
##     <arnumber><![CDATA[6851859]]></arnumber>
##     <doi><![CDATA[10.1109/JPHOT.2014.2337874]]></doi>
##     <publicationId><![CDATA[6851859]]></publicationId>
##     <mdurl><![CDATA[http://ieeexplore.ieee.org/xpl/articleDetails.jsp?tp=&arnumber=6851859&contentType=Journals+%26+Magazines]]></mdurl>
##     <pdf><![CDATA[http://ieeexplore.ieee.org/stamp/stamp.jsp?arnumber=6851859]]></pdf>
##   </document>
##   <document>
##     <rank>630</rank>
##     <title><![CDATA[A Survey on Internet of Things From Industrial Market Perspective]]></title>
##     <authors><![CDATA[Perera, C.;  Liu, C.H.;  Jayawardena, S.;  Min Chen]]></authors>
##     <affiliations><![CDATA[Res. Sch. of Comput. Sci., Australian Nat. Univ., Canberra, ACT, Australia]]></affiliations>
##     <controlledterms>
##       <term><![CDATA[Internet of Things]]></term>
##       <term><![CDATA[industrial economics]]></term>
##       <term><![CDATA[product development]]></term>
##       <term><![CDATA[production engineering computing]]></term>
##     </controlledterms>
##     <thesaurusterms>
##       <term><![CDATA[Context awareness]]></term>
##       <term><![CDATA[Futures research]]></term>
##       <term><![CDATA[Globalization]]></term>
##       <term><![CDATA[Information networks]]></term>
##       <term><![CDATA[Interconnections]]></term>
##       <term><![CDATA[Internet]]></term>
##       <term><![CDATA[Internet of things]]></term>
##       <term><![CDATA[Market research]]></term>
##       <term><![CDATA[Product development]]></term>
##     </thesaurusterms>
##     <pubtitle><![CDATA[Access, IEEE]]></pubtitle>
##     <punumber><![CDATA[6287639]]></punumber>
##     <pubtype><![CDATA[Journals & Magazines]]></pubtype>
##     <publisher><![CDATA[IEEE]]></publisher>
##     <volume><![CDATA[2]]></volume>
##     <py><![CDATA[2014]]></py>
##     <spage><![CDATA[1660]]></spage>
##     <epage><![CDATA[1679]]></epage>
##     <abstract><![CDATA[The Internet of Things (IoT) is a dynamic global information network consisting of Internet-connected objects, such as radio frequency identifications, sensors, and actuators, as well as other instruments and smart appliances that are becoming an integral component of the Internet. Over the last few years, we have seen a plethora of IoT solutions making their way into the industry marketplace. Context-aware communications and computing have played a critical role throughout the last few years of ubiquitous computing and are expected to play a significant role in the IoT paradigm as well. In this paper, we examine a variety of popular and innovative IoT solutions in terms of context-aware technology perspectives. More importantly, we evaluate these IoT solutions using a framework that we built around well-known context-aware computing theories. This survey is intended to serve as a guideline and a conceptual framework for context-aware product development and research in the IoT paradigm. It also provides a systematic exploration of existing IoT products in the marketplace and highlights a number of potentially significant research directions and trends.]]></abstract>
##     <issn><![CDATA[2169-3536]]></issn>
##     <htmlFlag><![CDATA[1]]></htmlFlag>
##     <arnumber><![CDATA[7004894]]></arnumber>
##     <doi><![CDATA[10.1109/ACCESS.2015.2389854]]></doi>
##     <publicationId><![CDATA[7004894]]></publicationId>
##     <mdurl><![CDATA[http://ieeexplore.ieee.org/xpl/articleDetails.jsp?tp=&arnumber=7004894&contentType=Journals+%26+Magazines]]></mdurl>
##     <pdf><![CDATA[http://ieeexplore.ieee.org/stamp/stamp.jsp?arnumber=7004894]]></pdf>
##   </document>
##   <document>
##     <rank>631</rank>
##     <title><![CDATA[Controlling Knee Swing Initiation and Ankle Plantarflexion With an Active Prosthesis on Level and Inclined Surfaces at Variable Walking Speeds]]></title>
##     <authors><![CDATA[Fey, N.P.;  Simon, A.M.;  Young, A.J.;  Hargrove, L.J.]]></authors>
##     <affiliations><![CDATA[Center for Bionic Med., Rehabilitation Inst. of Chicago, Chicago, IL, USA]]></affiliations>
##     <controlledterms>
##       <term><![CDATA[bone]]></term>
##       <term><![CDATA[gait analysis]]></term>
##       <term><![CDATA[kinematics]]></term>
##       <term><![CDATA[prosthetics]]></term>
##     </controlledterms>
##     <thesaurusterms>
##       <term><![CDATA[Impedance]]></term>
##       <term><![CDATA[Joints]]></term>
##       <term><![CDATA[Knee]]></term>
##       <term><![CDATA[Legged locomotion]]></term>
##       <term><![CDATA[Prosthetics]]></term>
##       <term><![CDATA[Surface impedance]]></term>
##       <term><![CDATA[Tuning]]></term>
##     </thesaurusterms>
##     <pubtitle><![CDATA[Translational Engineering in Health and Medicine, IEEE Journal of]]></pubtitle>
##     <punumber><![CDATA[6221039]]></punumber>
##     <pubtype><![CDATA[Journals & Magazines]]></pubtype>
##     <publisher><![CDATA[IEEE]]></publisher>
##     <volume><![CDATA[2]]></volume>
##     <py><![CDATA[2014]]></py>
##     <spage><![CDATA[1]]></spage>
##     <epage><![CDATA[12]]></epage>
##     <abstract><![CDATA[Improving lower-limb prostheses is important to enhance the mobility of amputees. The purpose of this paper is to introduce an impedance-based control strategy (consisting of four novel algorithms) for an active knee and ankle prosthesis and test its generalizability across multiple walking speeds, walking surfaces, and users. The four algorithms increased ankle stiffness throughout stance, decreased knee stiffness during terminal stance, as well as provided powered ankle plantarflexion and knee swing initiation through modifications of equilibrium positions of the ankle and knee, respectively. Seven amputees (knee disarticulation and transfemoral levels) walked at slow, comfortable, and hurried speeds on level and inclined (10&#x00B0;) surfaces. The prosthesis was tuned at their comfortable level ground walking speed. We further quantified trends in prosthetic knee and ankle kinematics, and kinetics across conditions. Subjects modulated their walking speed by &#x00B1;25% (average) from their comfortable speeds. As speed increased, increasing ankle angles and velocities as well as stance phase ankle power and plantarflexion torque were observed. At slow and comfortable speeds, plantarflexion torque was increased on the incline. At slow and comfortable speeds, stance phase positive knee power was increased and knee torque more flexor on the incline. As speed increased, knee torque became less flexor on the incline. These algorithms were shown to generalize well across speed, produce gait mechanics that compare favorably with non-amputee data, and display evidence of scalable device function. They have the potential to reduce the challenge of clinically configuring such devices and increase their viability during daily use.]]></abstract>
##     <issn><![CDATA[2168-2372]]></issn>
##     <htmlFlag><![CDATA[1]]></htmlFlag>
##     <arnumber><![CDATA[6866125]]></arnumber>
##     <doi><![CDATA[10.1109/JTEHM.2014.2343228]]></doi>
##     <publicationId><![CDATA[6866125]]></publicationId>
##     <mdurl><![CDATA[http://ieeexplore.ieee.org/xpl/articleDetails.jsp?tp=&arnumber=6866125&contentType=Journals+%26+Magazines]]></mdurl>
##     <pdf><![CDATA[http://ieeexplore.ieee.org/stamp/stamp.jsp?arnumber=6866125]]></pdf>
##   </document>
##   <document>
##     <rank>632</rank>
##     <title><![CDATA[A Delaunay Quadrangle-Based Fingerprint Authentication System With Template Protection Using Topology Code for Local Registration and Security Enhancement]]></title>
##     <authors><![CDATA[Wencheng Yang;  Jiankun Hu;  Song Wang]]></authors>
##     <affiliations><![CDATA[Sch. of Eng. & Inf. Technol., Univ. of New South Wales, Canberra, ACT, Australia]]></affiliations>
##     <controlledterms>
##       <term><![CDATA[authorisation]]></term>
##       <term><![CDATA[fingerprint identification]]></term>
##       <term><![CDATA[mesh generation]]></term>
##       <term><![CDATA[nonlinear distortion]]></term>
##       <term><![CDATA[set theory]]></term>
##     </controlledterms>
##     <thesaurusterms>
##       <term><![CDATA[Authentication]]></term>
##       <term><![CDATA[Cryptography]]></term>
##       <term><![CDATA[Feature extraction]]></term>
##       <term><![CDATA[Fingerprint recognition]]></term>
##       <term><![CDATA[Nonlinear distortion]]></term>
##       <term><![CDATA[Topology]]></term>
##     </thesaurusterms>
##     <pubtitle><![CDATA[Information Forensics and Security, IEEE Transactions on]]></pubtitle>
##     <punumber><![CDATA[10206]]></punumber>
##     <pubtype><![CDATA[Journals & Magazines]]></pubtype>
##     <publisher><![CDATA[IEEE]]></publisher>
##     <volume><![CDATA[9]]></volume>
##     <issue><![CDATA[7]]></issue>
##     <py><![CDATA[2014]]></py>
##     <spage><![CDATA[1179]]></spage>
##     <epage><![CDATA[1192]]></epage>
##     <abstract><![CDATA[Although some nice properties of the Delaunay triangle-based structure have been exploited in many fingerprint authentication systems and satisfactory outcomes have been reported, most of these systems operate without template protection. In addition, the feature sets and similarity measures utilized in these systems are not suitable for existing template protection techniques. Moreover, local structural change caused by nonlinear distortion is often not considered adequately in these systems. In this paper, we propose a Delaunay quadrangle-based fingerprint authentication system to deal with nonlinear distortion-induced local structural change that the Delaunay triangle-based structure suffers. Fixed-length and alignment-free feature vectors extracted from Delaunay quadrangles are less sensitive to nonlinear distortion and more discriminative than those from Delaunay triangles and can be applied to existing template protection directly. Furthermore, we propose to construct a unique topology code from each Delaunay quadrangle. Not only can this unique topology code help to carry out accurate local registration under distortion, but it also enhances the security of template data. Experimental results on public databases and security analysis show that the Delaunay quadrangle-based system with topology code can achieve better performance and higher security level than the Delaunay triangle-based system, the Delaunay quadrangle-based system without topology code, and some other similar systems.]]></abstract>
##     <issn><![CDATA[1556-6013]]></issn>
##     <htmlFlag><![CDATA[1]]></htmlFlag>
##     <arnumber><![CDATA[6824822]]></arnumber>
##     <doi><![CDATA[10.1109/TIFS.2014.2328095]]></doi>
##     <publicationId><![CDATA[6824822]]></publicationId>
##     <mdurl><![CDATA[http://ieeexplore.ieee.org/xpl/articleDetails.jsp?tp=&arnumber=6824822&contentType=Journals+%26+Magazines]]></mdurl>
##     <pdf><![CDATA[http://ieeexplore.ieee.org/stamp/stamp.jsp?arnumber=6824822]]></pdf>
##   </document>
##   <document>
##     <rank>633</rank>
##     <title><![CDATA[Impact of the Band Offset for n-Zn(O,S)/p-Cu(In,Ga)Se<formula formulatype="inline"> <img src="/images/tex/517.gif" alt="_{2}"> </formula> Solar Cells]]></title>
##     <authors><![CDATA[Sharbati, S.;  Sites, J.R.]]></authors>
##     <affiliations><![CDATA[Phys. Dept., Colorado State Univ., Fort Collins, CO, USA]]></affiliations>
##     <controlledterms>
##       <term><![CDATA[II-VI semiconductors]]></term>
##       <term><![CDATA[conduction bands]]></term>
##       <term><![CDATA[copper compounds]]></term>
##       <term><![CDATA[electron affinity]]></term>
##       <term><![CDATA[energy gap]]></term>
##       <term><![CDATA[gallium compounds]]></term>
##       <term><![CDATA[indium compounds]]></term>
##       <term><![CDATA[p-n heterojunctions]]></term>
##       <term><![CDATA[solar cells]]></term>
##       <term><![CDATA[ternary semiconductors]]></term>
##       <term><![CDATA[wide band gap semiconductors]]></term>
##       <term><![CDATA[zinc compounds]]></term>
##     </controlledterms>
##     <thesaurusterms>
##       <term><![CDATA[Buffer layers]]></term>
##       <term><![CDATA[Gallium]]></term>
##       <term><![CDATA[Photonic band gap]]></term>
##       <term><![CDATA[Photovoltaic cells]]></term>
##       <term><![CDATA[Photovoltaic systems]]></term>
##       <term><![CDATA[Temperature]]></term>
##     </thesaurusterms>
##     <pubtitle><![CDATA[Photovoltaics, IEEE Journal of]]></pubtitle>
##     <punumber><![CDATA[5503869]]></punumber>
##     <pubtype><![CDATA[Journals & Magazines]]></pubtype>
##     <publisher><![CDATA[IEEE]]></publisher>
##     <volume><![CDATA[4]]></volume>
##     <issue><![CDATA[2]]></issue>
##     <py><![CDATA[2014]]></py>
##     <spage><![CDATA[697]]></spage>
##     <epage><![CDATA[702]]></epage>
##     <abstract><![CDATA[The conduction-band offset (CBO) of the Zn(O, S)/Cu(In,Ga)Se<sub>2</sub> heterojunction can play a significant role in the performance of solar cells. The individual electron affinities and bandgaps are controlled by the oxygen-to-sulfur and gallium-to-indium ratios, and the resulting offsets can range from +1.3 eV in the &#x201C;spike&#x201D; direction to -0.7 eV in the &#x201C;cliff&#x201D; direction if the full range of the two ratios is considered. The optimal CBO of near +0.3 eV can be achieved with various combinations of the two ratios. The traditional CdS emitter is near optimal for the commonly used 1.15-eV Cu(In,Ga)Se 2 (CIGS) but less optimal for higher Ga. The flexibility with Zn(O,S) emitters ranging from above 90% oxygen for CIS down to 50% oxygen for CGS allows an optimal CBO over the full gallium range. Assuming that other factors remain constant, the optimal offset should also be able to reduce the loss in cell efficiency between room temperature and temperatures more typical of field conditions by about 1% absolute.]]></abstract>
##     <issn><![CDATA[2156-3381]]></issn>
##     <htmlFlag><![CDATA[1]]></htmlFlag>
##     <arnumber><![CDATA[6722905]]></arnumber>
##     <doi><![CDATA[10.1109/JPHOTOV.2014.2298093]]></doi>
##     <publicationId><![CDATA[6722905]]></publicationId>
##     <mdurl><![CDATA[http://ieeexplore.ieee.org/xpl/articleDetails.jsp?tp=&arnumber=6722905&contentType=Journals+%26+Magazines]]></mdurl>
##     <pdf><![CDATA[http://ieeexplore.ieee.org/stamp/stamp.jsp?arnumber=6722905]]></pdf>
##   </document>
##   <document>
##     <rank>634</rank>
##     <title><![CDATA[An MDP Model-Based Reinforcement Learning Approach for Production Station Ramp-Up Optimization: Q-Learning Analysis]]></title>
##     <authors><![CDATA[Doltsinis, S.;  Ferreira, P.;  Lohse, N.]]></authors>
##     <affiliations><![CDATA[Manuf. Div., Univ. of Nottingham, Nottingham, UK]]></affiliations>
##     <controlledterms>
##       <term><![CDATA[Markov processes]]></term>
##       <term><![CDATA[decision making]]></term>
##       <term><![CDATA[learning (artificial intelligence)]]></term>
##       <term><![CDATA[manufacturing systems]]></term>
##       <term><![CDATA[production engineering computing]]></term>
##     </controlledterms>
##     <thesaurusterms>
##       <term><![CDATA[Algorithm design and analysis]]></term>
##       <term><![CDATA[Decision making]]></term>
##       <term><![CDATA[Learning (artificial intelligence)]]></term>
##       <term><![CDATA[Manufacturing systems]]></term>
##       <term><![CDATA[Personnel]]></term>
##     </thesaurusterms>
##     <pubtitle><![CDATA[Systems, Man, and Cybernetics: Systems, IEEE Transactions on]]></pubtitle>
##     <punumber><![CDATA[6221021]]></punumber>
##     <pubtype><![CDATA[Journals & Magazines]]></pubtype>
##     <publisher><![CDATA[IEEE]]></publisher>
##     <volume><![CDATA[44]]></volume>
##     <issue><![CDATA[9]]></issue>
##     <py><![CDATA[2014]]></py>
##     <spage><![CDATA[1125]]></spage>
##     <epage><![CDATA[1138]]></epage>
##     <abstract><![CDATA[Ramp-up is a significant bottleneck for the introduction of new or adapted manufacturing systems. The effort and time required to ramp-up a system is largely dependent on the effectiveness of the human decision making process to select the most promising sequence of actions to improve the system to the required level of performance. Although existing work has identified significant factors influencing the effectiveness of ramp-up, little has been done to support the decision making during the process. This paper approaches ramp-up as a sequential adjustment and tuning process that aims to get a manufacturing system to a desirable performance in the fastest possible time. Production stations and machines are the key resources in a manufacturing system. They are often functionally decoupled and can be treated in the first instance as independent ramp-up problems. Hence, this paper focuses on developing a Markov decision process (MDP) model to formalize ramp-up of production stations and enable their formal analysis. The aim is to capture the cause-and-effect relationships between an operator's adaptation or adjustment of a station and the station's response to improve the effectiveness of the process. Reinforcement learning has been identified as a promising approach to learn from ramp-up experience and discover more successful decision-making policies. Batch learning in particular can perform well with little data. This paper investigates the application of a Q-batch learning algorithm combined with an MDP model of the ramp-up process. The approach has been applied to a highly automated production station where several ramp-up processes are carried out. The convergence of the Q-learning algorithm has been analyzed along with the variation of its parameters. Finally, the learned policy has been applied and compared against previous ramp-up cases.]]></abstract>
##     <issn><![CDATA[2168-2216]]></issn>
##     <htmlFlag><![CDATA[1]]></htmlFlag>
##     <arnumber><![CDATA[6702489]]></arnumber>
##     <doi><![CDATA[10.1109/TSMC.2013.2294155]]></doi>
##     <publicationId><![CDATA[6702489]]></publicationId>
##     <mdurl><![CDATA[http://ieeexplore.ieee.org/xpl/articleDetails.jsp?tp=&arnumber=6702489&contentType=Journals+%26+Magazines]]></mdurl>
##     <pdf><![CDATA[http://ieeexplore.ieee.org/stamp/stamp.jsp?arnumber=6702489]]></pdf>
##   </document>
##   <document>
##     <rank>635</rank>
##     <title><![CDATA[Breakthroughs in Photonics 2013: A Microwatt-Threshold Raman Silicon Laser]]></title>
##     <authors><![CDATA[Takahashi, Y.;  Noda, S.]]></authors>
##     <affiliations><![CDATA[Nanosci. & Nanotechnol. Res. Center, Osaka Prefecture Univ., Sakai, Japan]]></affiliations>
##     <controlledterms>
##       <term><![CDATA[Raman lasers]]></term>
##       <term><![CDATA[laser cavity resonators]]></term>
##       <term><![CDATA[nanophotonics]]></term>
##       <term><![CDATA[photonic crystals]]></term>
##       <term><![CDATA[semiconductor lasers]]></term>
##       <term><![CDATA[silicon]]></term>
##     </controlledterms>
##     <thesaurusterms>
##       <term><![CDATA[Laser excitation]]></term>
##       <term><![CDATA[Optical waveguides]]></term>
##       <term><![CDATA[Photonic crystals]]></term>
##       <term><![CDATA[Photonics]]></term>
##       <term><![CDATA[Pump lasers]]></term>
##       <term><![CDATA[Silicon]]></term>
##       <term><![CDATA[Waveguide lasers]]></term>
##     </thesaurusterms>
##     <pubtitle><![CDATA[Photonics Journal, IEEE]]></pubtitle>
##     <punumber><![CDATA[4563994]]></punumber>
##     <pubtype><![CDATA[Journals & Magazines]]></pubtype>
##     <publisher><![CDATA[IEEE]]></publisher>
##     <volume><![CDATA[6]]></volume>
##     <issue><![CDATA[2]]></issue>
##     <py><![CDATA[2014]]></py>
##     <spage><![CDATA[1]]></spage>
##     <epage><![CDATA[5]]></epage>
##     <abstract><![CDATA[A continuous-wave Raman silicon laser using a photonic-crystal high- Q nanocavity was developed in 2013, with a cavity size of less than 10 &#x03BC;m and an unprecedented ultralow threshold of 1 &#x03BC;W. This report describes how this breakthrough was accomplished by adopting a unique nanocavity design.]]></abstract>
##     <issn><![CDATA[1943-0655]]></issn>
##     <htmlFlag><![CDATA[1]]></htmlFlag>
##     <arnumber><![CDATA[6762863]]></arnumber>
##     <doi><![CDATA[10.1109/JPHOT.2014.2310218]]></doi>
##     <publicationId><![CDATA[6762863]]></publicationId>
##     <mdurl><![CDATA[http://ieeexplore.ieee.org/xpl/articleDetails.jsp?tp=&arnumber=6762863&contentType=Journals+%26+Magazines]]></mdurl>
##     <pdf><![CDATA[http://ieeexplore.ieee.org/stamp/stamp.jsp?arnumber=6762863]]></pdf>
##   </document>
##   <document>
##     <rank>636</rank>
##     <title><![CDATA[Characterization of Negative-Bias Temperature Instability of Ge MOSFETs With <inline-formula> <img src="/images/tex/21640.gif" alt="{\rm GeO}_{2}/{\rm Al}_{2}{\rm O}_{3}"> </inline-formula> Stack]]></title>
##     <authors><![CDATA[Jigang Ma;  Jian Fu Zhang;  Zhigang Ji;  Benbakhti, B.;  Wei Dong Zhang;  Xue Feng Zheng;  Mitard, J.;  Kaczer, B.;  Groeseneken, G.;  Hall, S.;  Robertson, J.;  Chalker, P.R.]]></authors>
##     <affiliations><![CDATA[Sch. of Eng., John Moores Univ., Liverpool, UK]]></affiliations>
##     <controlledterms>
##       <term><![CDATA[MOSFET]]></term>
##       <term><![CDATA[aluminium compounds]]></term>
##       <term><![CDATA[annealing]]></term>
##       <term><![CDATA[germanium]]></term>
##       <term><![CDATA[germanium compounds]]></term>
##       <term><![CDATA[hole mobility]]></term>
##       <term><![CDATA[hole traps]]></term>
##       <term><![CDATA[interface states]]></term>
##       <term><![CDATA[negative bias temperature instability]]></term>
##       <term><![CDATA[semiconductor device models]]></term>
##     </controlledterms>
##     <thesaurusterms>
##       <term><![CDATA[Aluminum oxide]]></term>
##       <term><![CDATA[Annealing]]></term>
##       <term><![CDATA[Interface states]]></term>
##       <term><![CDATA[Silicon]]></term>
##       <term><![CDATA[Stress]]></term>
##       <term><![CDATA[Temperature measurement]]></term>
##     </thesaurusterms>
##     <pubtitle><![CDATA[Electron Devices, IEEE Transactions on]]></pubtitle>
##     <punumber><![CDATA[16]]></punumber>
##     <pubtype><![CDATA[Journals & Magazines]]></pubtype>
##     <publisher><![CDATA[IEEE]]></publisher>
##     <volume><![CDATA[61]]></volume>
##     <issue><![CDATA[5]]></issue>
##     <py><![CDATA[2014]]></py>
##     <spage><![CDATA[1307]]></spage>
##     <epage><![CDATA[1315]]></epage>
##     <abstract><![CDATA[Ge is a candidate for replacing Si, especially for pMOSFETs, because of its high hole mobility. For Si-pMOSFETs, negative-bias temperature instabilities (NBTI) limit their lifetime. There is little information available for the NBTI of Ge-pMOSFETs with Ge/GeO<sub>2</sub>/Al<sub>2</sub>O<sub>3</sub> stack. The objective of this paper is to provide this information and compare the NBTI of Ge- and Si-pMOSFETs. New findings include: 1) the time exponent varies with stress biases/field when measured by either the conventional slow dc or pulse I-V technique, making the conventional V<sub>g</sub>-accelerated method for predicting the lifetime of Si-pMOSFETs inapplicable to Ge-pMOSFETs used in this paper; 2) the NBTI is dominated by positive charges (PCs) in dielectric, rather than generated interface states; 3) the PC in Ge/GeO<sub>2</sub>/Al<sub>2</sub>O<sub>3</sub> can be fully annealed at 150 &#x00B0;C; and 4) the defect losses reported for Si sample were not observed. For the first time, we report that the PCs in oxides on Ge and Si behave differently, and to explain the difference, an energy-switching model is proposed for hole traps in Ge-MOSFETs: their energy levels have a spread below the edge of valence band, i.e., E<sub>v</sub>, when neutral, lift well above E<sub>v</sub> after charging, and return below E<sub>v</sub> following neutralization.]]></abstract>
##     <issn><![CDATA[0018-9383]]></issn>
##     <arnumber><![CDATA[6786032]]></arnumber>
##     <doi><![CDATA[10.1109/TED.2014.2314178]]></doi>
##     <publicationId><![CDATA[6786032]]></publicationId>
##     <mdurl><![CDATA[http://ieeexplore.ieee.org/xpl/articleDetails.jsp?tp=&arnumber=6786032&contentType=Journals+%26+Magazines]]></mdurl>
##     <pdf><![CDATA[http://ieeexplore.ieee.org/stamp/stamp.jsp?arnumber=6786032]]></pdf>
##   </document>
##   <document>
##     <rank>637</rank>
##     <title><![CDATA[Chip-Level 1 <formula formulatype="inline"> <img src="/images/tex/326.gif" alt="\times"> </formula> 2 Optical Interconnects Using Polymer Vertical Splitter on Silicon Substrate]]></title>
##     <authors><![CDATA[Chin-Ta Chen;  Po-Kuan Shen;  Teng-Zhang Zhu;  Chia-Chi Chang;  Shu-Shuan Lin;  Mao-Yuan Zeng;  Chien-Yu Chiu;  Hsu-Liang Hsiao;  Hsiao-Chin Lan;  Yun-Chih Lee;  Yo-Shen Lin;  Mount-Learn Wu]]></authors>
##     <affiliations><![CDATA[Dept. of Opt. & Photonics, Nat. Central Univ., Jhongli, Taiwan]]></affiliations>
##     <controlledterms>
##       <term><![CDATA[elemental semiconductors]]></term>
##       <term><![CDATA[error statistics]]></term>
##       <term><![CDATA[integrated optics]]></term>
##       <term><![CDATA[integrated optoelectronics]]></term>
##       <term><![CDATA[laser cavity resonators]]></term>
##       <term><![CDATA[optical beam splitters]]></term>
##       <term><![CDATA[optical communication equipment]]></term>
##       <term><![CDATA[optical interconnections]]></term>
##       <term><![CDATA[optical losses]]></term>
##       <term><![CDATA[optical polymers]]></term>
##       <term><![CDATA[optical waveguides]]></term>
##       <term><![CDATA[photodetectors]]></term>
##       <term><![CDATA[silicon]]></term>
##       <term><![CDATA[surface emitting lasers]]></term>
##     </controlledterms>
##     <thesaurusterms>
##       <term><![CDATA[Optical interconnections]]></term>
##       <term><![CDATA[Optical polymers]]></term>
##       <term><![CDATA[Optical reflection]]></term>
##       <term><![CDATA[Optical waveguides]]></term>
##       <term><![CDATA[Silicon]]></term>
##       <term><![CDATA[Vertical cavity surface emitting lasers]]></term>
##     </thesaurusterms>
##     <pubtitle><![CDATA[Photonics Journal, IEEE]]></pubtitle>
##     <punumber><![CDATA[4563994]]></punumber>
##     <pubtype><![CDATA[Journals & Magazines]]></pubtype>
##     <publisher><![CDATA[IEEE]]></publisher>
##     <volume><![CDATA[6]]></volume>
##     <issue><![CDATA[2]]></issue>
##     <py><![CDATA[2014]]></py>
##     <spage><![CDATA[1]]></spage>
##     <epage><![CDATA[10]]></epage>
##     <abstract><![CDATA[The chip-level 1 &#x00D7; 2 optical interconnects using the polymer vertical splitter developed on a silicon substrate are demonstrated. The 1 &#x00D7; 2 vertical-splitting configuration is realized using a polymer waveguide terminated at three silicon 45 <sup>&#x00B0;</sup> reflectors. The high-frequency transmission lines combined with the indium solder bumps are developed to flip-chip assemble a vertical-cavity surface-emitting laser chip at the input port and two photodetector chips at two output ports. Total transmission loss of -3.26 dB with a splitting ratio of 1 : 1 for the proposed splitter is experimentally obtained. A 10-Gbit/s data transmission with bit error rates better than 10<sup>-12</sup> for two output ports is achieved. It reveals that such chip-level 1 &#x00D7; 2 optical interconnects using the polymer vertical splitter are suitable for high-speed data transmission with multiple output ports.]]></abstract>
##     <issn><![CDATA[1943-0655]]></issn>
##     <htmlFlag><![CDATA[1]]></htmlFlag>
##     <arnumber><![CDATA[6746668]]></arnumber>
##     <doi><![CDATA[10.1109/JPHOT.2014.2306843]]></doi>
##     <publicationId><![CDATA[6746668]]></publicationId>
##     <mdurl><![CDATA[http://ieeexplore.ieee.org/xpl/articleDetails.jsp?tp=&arnumber=6746668&contentType=Journals+%26+Magazines]]></mdurl>
##     <pdf><![CDATA[http://ieeexplore.ieee.org/stamp/stamp.jsp?arnumber=6746668]]></pdf>
##   </document>
##   <document>
##     <rank>638</rank>
##     <title><![CDATA[SIM&#x002F;SM-Aided Free-Space Optical Communication With Receiver Diversity]]></title>
##     <authors><![CDATA[Seung-Hoon Hwang;  Yan Cheng]]></authors>
##     <affiliations><![CDATA[Div. of Electron. & Electr. Eng., Dongguk Univ. - Seoul, Seoul, South Korea]]></affiliations>
##     <controlledterms>
##       <term><![CDATA[AWGN channels]]></term>
##       <term><![CDATA[optical fibre networks]]></term>
##       <term><![CDATA[optical receivers]]></term>
##     </controlledterms>
##     <thesaurusterms>
##       <term><![CDATA[AWGN channels]]></term>
##       <term><![CDATA[Bit error rate]]></term>
##       <term><![CDATA[Fading]]></term>
##       <term><![CDATA[Optical modulation]]></term>
##       <term><![CDATA[Optical receivers]]></term>
##       <term><![CDATA[Optical transmitters]]></term>
##     </thesaurusterms>
##     <pubtitle><![CDATA[Lightwave Technology, Journal of]]></pubtitle>
##     <punumber><![CDATA[50]]></punumber>
##     <pubtype><![CDATA[Journals & Magazines]]></pubtype>
##     <publisher><![CDATA[IEEE]]></publisher>
##     <volume><![CDATA[32]]></volume>
##     <issue><![CDATA[14]]></issue>
##     <py><![CDATA[2014]]></py>
##     <spage><![CDATA[2443]]></spage>
##     <epage><![CDATA[2450]]></epage>
##     <abstract><![CDATA[In this paper, we propose subcarrier intensity-modulation (SIM)/spatial modulation (SM)-aided free space optical communications with receiver diversity. Using computer simulation, the performances of the proposed SIM/SM scheme are investigated and compared with those of the conventional SIM scheme in an additive white Gaussian noise (AWGN) channel and in outdoor applications with lognormal atmospheric channels. The error performances are also mathematically analyzed in the AWGN channel for spectral efficiency of 2 bits/s/Hz in order to validate the simulation results. Numerical results show that the proposed SIM/SM scheme can outperform the conventional SIM for both channels with various spectral efficiencies. We also demonstrate that, by employing multiple receivers, the deleterious effect of scintillation on the proposed SIM/SM scheme has been further ameliorated. Furthermore, our proposed SIM/SM scheme with the receiver diversity gets more resistant to turbulence-induced fading, when the spectral efficiency and the number of receivers are increased.]]></abstract>
##     <issn><![CDATA[0733-8724]]></issn>
##     <htmlFlag><![CDATA[1]]></htmlFlag>
##     <arnumber><![CDATA[6823084]]></arnumber>
##     <doi><![CDATA[10.1109/JLT.2014.2327078]]></doi>
##     <publicationId><![CDATA[6823084]]></publicationId>
##     <mdurl><![CDATA[http://ieeexplore.ieee.org/xpl/articleDetails.jsp?tp=&arnumber=6823084&contentType=Journals+%26+Magazines]]></mdurl>
##     <pdf><![CDATA[http://ieeexplore.ieee.org/stamp/stamp.jsp?arnumber=6823084]]></pdf>
##   </document>
##   <document>
##     <rank>639</rank>
##     <title><![CDATA[Improved Evolutionary Algorithm Design for the Project Scheduling Problem Based on Runtime Analysis]]></title>
##     <authors><![CDATA[Minku, L.L.;  Sudholt, D.;  Xin Yao]]></authors>
##     <affiliations><![CDATA[CERCIA, Univ. of Birmingham, Birmingham, UK]]></affiliations>
##     <controlledterms>
##       <term><![CDATA[evolutionary computation]]></term>
##       <term><![CDATA[project management]]></term>
##       <term><![CDATA[scheduling]]></term>
##       <term><![CDATA[software development management]]></term>
##     </controlledterms>
##     <thesaurusterms>
##       <term><![CDATA[Algorithm design and analysis]]></term>
##       <term><![CDATA[Resource management]]></term>
##       <term><![CDATA[Schedules]]></term>
##       <term><![CDATA[Scheduling]]></term>
##       <term><![CDATA[Software]]></term>
##       <term><![CDATA[Software algorithms]]></term>
##       <term><![CDATA[Software engineering]]></term>
##     </thesaurusterms>
##     <pubtitle><![CDATA[Software Engineering, IEEE Transactions on]]></pubtitle>
##     <punumber><![CDATA[32]]></punumber>
##     <pubtype><![CDATA[Journals & Magazines]]></pubtype>
##     <publisher><![CDATA[IEEE]]></publisher>
##     <volume><![CDATA[40]]></volume>
##     <issue><![CDATA[1]]></issue>
##     <py><![CDATA[2014]]></py>
##     <spage><![CDATA[83]]></spage>
##     <epage><![CDATA[102]]></epage>
##     <abstract><![CDATA[Several variants of evolutionary algorithms (EAs) have been applied to solve the project scheduling problem (PSP), yet their performance highly depends on design choices for the EA. It is still unclear how and why different EAs perform differently. We present the first runtime analysis for the PSP, gaining insights into the performance of EAs on the PSP in general, and on specific instance classes that are easy or hard. Our theoretical analysis has practical implications-based on it, we derive an improved EA design. This includes normalizing employees' dedication for different tasks to ensure they are not working overtime; a fitness function that requires fewer pre-defined parameters and provides a clear gradient towards feasible solutions; and an improved representation and mutation operator. Both our theoretical and empirical results show that our design is very effective. Combining the use of normalization to a population gave the best results in our experiments, and normalization was a key component for the practical effectiveness of the new design. Not only does our paper offer a new and effective algorithm for the PSP, it also provides a rigorous theoretical analysis to explain the efficiency of the algorithm, especially for increasingly large projects.]]></abstract>
##     <issn><![CDATA[0098-5589]]></issn>
##     <htmlFlag><![CDATA[1]]></htmlFlag>
##     <arnumber><![CDATA[6648326]]></arnumber>
##     <doi><![CDATA[10.1109/TSE.2013.52]]></doi>
##     <publicationId><![CDATA[6648326]]></publicationId>
##     <mdurl><![CDATA[http://ieeexplore.ieee.org/xpl/articleDetails.jsp?tp=&arnumber=6648326&contentType=Journals+%26+Magazines]]></mdurl>
##     <pdf><![CDATA[http://ieeexplore.ieee.org/stamp/stamp.jsp?arnumber=6648326]]></pdf>
##   </document>
##   <document>
##     <rank>640</rank>
##     <title><![CDATA[Linear and Nonlinear Regression Techniques for Simultaneous and Proportional Myoelectric Control]]></title>
##     <authors><![CDATA[Hahne, J.M.;  BieBmann, F.;  Jiang, N.;  Rehbaum, H.;  Farina, D.;  Meinecke, F.C.;  Muller, K.-R.;  Parra, L.C.]]></authors>
##     <affiliations><![CDATA[Machine Learning Lab., Berlin Inst. of Technol., Berlin, Germany]]></affiliations>
##     <controlledterms>
##       <term><![CDATA[data acquisition]]></term>
##       <term><![CDATA[electrochemical electrodes]]></term>
##       <term><![CDATA[electromyography]]></term>
##       <term><![CDATA[feature extraction]]></term>
##       <term><![CDATA[medical control systems]]></term>
##       <term><![CDATA[medical disorders]]></term>
##       <term><![CDATA[medical signal processing]]></term>
##       <term><![CDATA[multilayer perceptrons]]></term>
##       <term><![CDATA[prosthetics]]></term>
##       <term><![CDATA[regression analysis]]></term>
##     </controlledterms>
##     <thesaurusterms>
##       <term><![CDATA[Electrodes]]></term>
##       <term><![CDATA[Electromyography]]></term>
##       <term><![CDATA[Kernel]]></term>
##       <term><![CDATA[Training]]></term>
##       <term><![CDATA[Training data]]></term>
##       <term><![CDATA[Trajectory]]></term>
##       <term><![CDATA[Wrist]]></term>
##     </thesaurusterms>
##     <pubtitle><![CDATA[Neural Systems and Rehabilitation Engineering, IEEE Transactions on]]></pubtitle>
##     <punumber><![CDATA[7333]]></punumber>
##     <pubtype><![CDATA[Journals & Magazines]]></pubtype>
##     <publisher><![CDATA[IEEE]]></publisher>
##     <volume><![CDATA[22]]></volume>
##     <issue><![CDATA[2]]></issue>
##     <py><![CDATA[2014]]></py>
##     <spage><![CDATA[269]]></spage>
##     <epage><![CDATA[279]]></epage>
##     <abstract><![CDATA[In recent years the number of active controllable joints in electrically powered hand-prostheses has increased significantly. However, the control strategies for these devices in current clinical use are inadequate as they require separate and sequential control of each degree-of-freedom (DoF). In this study we systematically compare linear and nonlinear regression techniques for an independent, simultaneous and proportional myoelectric control of wrist movements with two DoF. These techniques include linear regression, mixture of linear experts (ME), multilayer-perceptron, and kernel ridge regression (KRR). They are investigated offline with electro-myographic signals acquired from ten able-bodied subjects and one person with congenital upper limb deficiency. The control accuracy is reported as a function of the number of electrodes and the amount and diversity of training data providing guidance for the requirements in clinical practice. The results showed that KRR, a nonparametric statistical learning method, outperformed the other methods. However, simple transformations in the feature space could linearize the problem, so that linear models could achieve similar performance as KRR at much lower computational costs. Especially ME, a physiologically inspired extension of linear regression represents a promising candidate for the next generation of prosthetic devices.]]></abstract>
##     <issn><![CDATA[1534-4320]]></issn>
##     <htmlFlag><![CDATA[1]]></htmlFlag>
##     <arnumber><![CDATA[6742730]]></arnumber>
##     <doi><![CDATA[10.1109/TNSRE.2014.2305520]]></doi>
##     <publicationId><![CDATA[6742730]]></publicationId>
##     <mdurl><![CDATA[http://ieeexplore.ieee.org/xpl/articleDetails.jsp?tp=&arnumber=6742730&contentType=Journals+%26+Magazines]]></mdurl>
##     <pdf><![CDATA[http://ieeexplore.ieee.org/stamp/stamp.jsp?arnumber=6742730]]></pdf>
##   </document>
##   <document>
##     <rank>641</rank>
##     <title><![CDATA[Optically Powered Energy Source in a Standard CMOS Process for Integration in Smart Dust Applications]]></title>
##     <authors><![CDATA[Jing Jing Liu;  Faulkner, G.;  Choubey, B.;  Jia Liu;  Ri Qing Chen;  O'Brien, D.C.;  Collins, S.]]></authors>
##     <affiliations><![CDATA[Fujian Agric. & Forestry Univ., Fuzhou, China]]></affiliations>
##     <controlledterms>
##       <term><![CDATA[CMOS integrated circuits]]></term>
##       <term><![CDATA[low-power electronics]]></term>
##       <term><![CDATA[microsensors]]></term>
##       <term><![CDATA[photodiodes]]></term>
##       <term><![CDATA[telecommunication power management]]></term>
##       <term><![CDATA[wireless sensor networks]]></term>
##     </controlledterms>
##     <thesaurusterms>
##       <term><![CDATA[CMOS process]]></term>
##       <term><![CDATA[Junctions]]></term>
##       <term><![CDATA[Photodiodes]]></term>
##       <term><![CDATA[Wireless sensor networks]]></term>
##     </thesaurusterms>
##     <pubtitle><![CDATA[Electron Devices Society, IEEE Journal of the]]></pubtitle>
##     <punumber><![CDATA[6245494]]></punumber>
##     <pubtype><![CDATA[Journals & Magazines]]></pubtype>
##     <publisher><![CDATA[IEEE]]></publisher>
##     <volume><![CDATA[2]]></volume>
##     <issue><![CDATA[6]]></issue>
##     <py><![CDATA[2014]]></py>
##     <spage><![CDATA[158]]></spage>
##     <epage><![CDATA[163]]></epage>
##     <abstract><![CDATA[In order to miniaturize nano-power sensor nodes or smart dust, an optically powered energy source is developed to replace traditional batteries or solar cells. This energy source consists of two photodiodes, which are P-well/DN-well and N-well/P-sub. The two photodiodes with an area of 1.5mm2 were fabricated using the UMC 0.25&#x03BC;m CMOS process and tested using an 830nm laser. Measurement results show that the energy source is able to generate a voltage from 0.5V to 0.8V with a 3.5% conversion efficiency. The proposed energy source was made using a standard CMOS process and therefore can to be integrated with the smart dust circuit on a single chip.]]></abstract>
##     <issn><![CDATA[2168-6734]]></issn>
##     <htmlFlag><![CDATA[1]]></htmlFlag>
##     <arnumber><![CDATA[6861949]]></arnumber>
##     <doi><![CDATA[10.1109/JEDS.2014.2341602]]></doi>
##     <publicationId><![CDATA[6861949]]></publicationId>
##     <mdurl><![CDATA[http://ieeexplore.ieee.org/xpl/articleDetails.jsp?tp=&arnumber=6861949&contentType=Journals+%26+Magazines]]></mdurl>
##     <pdf><![CDATA[http://ieeexplore.ieee.org/stamp/stamp.jsp?arnumber=6861949]]></pdf>
##   </document>
##   <document>
##     <rank>642</rank>
##     <title><![CDATA[A Monte Carlo Simulation Platform for Studying Low Voltage Residential Networks]]></title>
##     <authors><![CDATA[Torquato, R.;  Qingxin Shi;  Wilsun Xu;  Freitas, W.]]></authors>
##     <affiliations><![CDATA[Dept. of Electr. & Comput. Eng., Univ. of Alberta, Edmonton, AB, Canada]]></affiliations>
##     <controlledterms>
##       <term><![CDATA[Monte Carlo methods]]></term>
##       <term><![CDATA[distributed power generation]]></term>
##       <term><![CDATA[load flow]]></term>
##       <term><![CDATA[power system harmonics]]></term>
##       <term><![CDATA[smart power grids]]></term>
##     </controlledterms>
##     <thesaurusterms>
##       <term><![CDATA[Home appliances]]></term>
##       <term><![CDATA[Load modeling]]></term>
##       <term><![CDATA[Low voltage]]></term>
##       <term><![CDATA[Microgrids]]></term>
##       <term><![CDATA[Monte Carlo methods]]></term>
##       <term><![CDATA[Power quality]]></term>
##     </thesaurusterms>
##     <pubtitle><![CDATA[Smart Grid, IEEE Transactions on]]></pubtitle>
##     <punumber><![CDATA[5165411]]></punumber>
##     <pubtype><![CDATA[Journals & Magazines]]></pubtype>
##     <publisher><![CDATA[IEEE]]></publisher>
##     <volume><![CDATA[5]]></volume>
##     <issue><![CDATA[6]]></issue>
##     <py><![CDATA[2014]]></py>
##     <spage><![CDATA[2766]]></spage>
##     <epage><![CDATA[2776]]></epage>
##     <abstract><![CDATA[The smart grid vision has resulted in many demand side innovations such as nonintrusive load monitoring techniques, residential micro-grids, and demand response programs. Many of these techniques need a detailed residential network model for their research, evaluation, and validation. In response to such a need, this paper presents a sequential Monte Carlo (SMC) simulation platform for modeling and simulating low voltage residential networks. This platform targets the simulation of the quasi-steady-state network condition over an extended period such as 24 h. It consists of two main components. The first is a multiphase network model with power flow, harmonic, and motor starting study capabilities. The second is a load/generation behavior model that establishes the operating characteristics of various loads and generators based on time-of-use probability curves. These two components are combined together through an SMC simulation scheme. Four case studies are presented to demonstrate the applications of the proposed platform.]]></abstract>
##     <issn><![CDATA[1949-3053]]></issn>
##     <htmlFlag><![CDATA[1]]></htmlFlag>
##     <arnumber><![CDATA[6853399]]></arnumber>
##     <doi><![CDATA[10.1109/TSG.2014.2331175]]></doi>
##     <publicationId><![CDATA[6853399]]></publicationId>
##     <mdurl><![CDATA[http://ieeexplore.ieee.org/xpl/articleDetails.jsp?tp=&arnumber=6853399&contentType=Journals+%26+Magazines]]></mdurl>
##     <pdf><![CDATA[http://ieeexplore.ieee.org/stamp/stamp.jsp?arnumber=6853399]]></pdf>
##   </document>
##   <document>
##     <rank>643</rank>
##     <title><![CDATA[Solving Distributed and Flexible Job-Shop Scheduling Problems for a Real-World Fastener Manufacturer]]></title>
##     <authors><![CDATA[Tung-Kuan Liu;  Yeh-Peng Chen;  Jyh-Horng Chou]]></authors>
##     <affiliations><![CDATA[Inst. of Eng. Sci. & Technol., Nat. Kaohsiung First Univ. of Sci. & Technol., Kaohsiung, Taiwan]]></affiliations>
##     <controlledterms>
##       <term><![CDATA[computational complexity]]></term>
##       <term><![CDATA[fasteners]]></term>
##       <term><![CDATA[flexible manufacturing systems]]></term>
##       <term><![CDATA[genetic algorithms]]></term>
##       <term><![CDATA[job shop scheduling]]></term>
##       <term><![CDATA[probability]]></term>
##     </controlledterms>
##     <thesaurusterms>
##       <term><![CDATA[Biological cells]]></term>
##       <term><![CDATA[Encoding]]></term>
##       <term><![CDATA[Genetic algorithms]]></term>
##       <term><![CDATA[Job shop scheduling]]></term>
##       <term><![CDATA[Manufacturing]]></term>
##       <term><![CDATA[Production facilities]]></term>
##     </thesaurusterms>
##     <pubtitle><![CDATA[Access, IEEE]]></pubtitle>
##     <punumber><![CDATA[6287639]]></punumber>
##     <pubtype><![CDATA[Journals & Magazines]]></pubtype>
##     <publisher><![CDATA[IEEE]]></publisher>
##     <volume><![CDATA[2]]></volume>
##     <py><![CDATA[2014]]></py>
##     <spage><![CDATA[1598]]></spage>
##     <epage><![CDATA[1606]]></epage>
##     <abstract><![CDATA[Over the last few decades, there has been considerable concern over the multifactory manufacturing environments owing to globalization. Numerous studies have indicated that flexible job-shop scheduling problems (FJSPs) and the distributed and FJSPs (DFJSPs) belong to NP-hard puzzle. The allocation of jobs to appropriate factories or flexible manufacturing units is an essential task in multifactory optimization scheduling, which involves the consideration of equipment performance, technology, capacity, and utilization level for each factory or manufacturing unit. Several variables and constraints should be considered in the encoding problem of DFJSPs when using genetic algorithms (GAs). In particular, it has been reported in the literature that the traditional GA encoding method may generate infeasible solutions or illegal solutions; thus, a specially designed evolution process is required. However, in such a process, the diversity of chromosomes is lost. To overcome this drawback, this paper proposes a refined encoding operator that integrates probability concepts into a real-parameter encoding method. In addition, the length of chromosomes can be substantially reduced using the proposed algorithm, thereby, saving computation space. The proposed refined GA algorithm was evaluated with satisfactory results through two-stage validation; in the first stage, a classical DFJSP was adopted to show the effectiveness of the algorithm, and in the second stage, the algorithm was used to solve a real-world case. The real-world case involved the use of historical data with 100 and 200 sets of work orders of a fastener manufacturer in Taiwan. The results were satisfactory and indicated that the proposed refined GA algorithm could effectively overcome the conflicts caused by GA encoding algorithms.]]></abstract>
##     <issn><![CDATA[2169-3536]]></issn>
##     <htmlFlag><![CDATA[1]]></htmlFlag>
##     <arnumber><![CDATA[7004149]]></arnumber>
##     <doi><![CDATA[10.1109/ACCESS.2015.2388486]]></doi>
##     <publicationId><![CDATA[7004149]]></publicationId>
##     <mdurl><![CDATA[http://ieeexplore.ieee.org/xpl/articleDetails.jsp?tp=&arnumber=7004149&contentType=Journals+%26+Magazines]]></mdurl>
##     <pdf><![CDATA[http://ieeexplore.ieee.org/stamp/stamp.jsp?arnumber=7004149]]></pdf>
##   </document>
##   <document>
##     <rank>644</rank>
##     <title><![CDATA[Inter-Comparison and Validation of the FY-3A/MERSI LAI Product Over Mainland China]]></title>
##     <authors><![CDATA[Lin Zhu;  Chen, J.M.;  Shihao Tang;  Guicai Li;  Zhaodi Guo]]></authors>
##     <affiliations><![CDATA[Nat. Satellite Meteorol. Center, China Meteorol. Adm., Beijing, China]]></affiliations>
##     <controlledterms>
##       <term><![CDATA[geophysical techniques]]></term>
##       <term><![CDATA[remote sensing]]></term>
##       <term><![CDATA[vegetation]]></term>
##     </controlledterms>
##     <thesaurusterms>
##       <term><![CDATA[Earth]]></term>
##       <term><![CDATA[Indexes]]></term>
##       <term><![CDATA[Land surface]]></term>
##       <term><![CDATA[MODIS]]></term>
##       <term><![CDATA[Remote sensing]]></term>
##       <term><![CDATA[Satellites]]></term>
##       <term><![CDATA[Surface topography]]></term>
##     </thesaurusterms>
##     <pubtitle><![CDATA[Selected Topics in Applied Earth Observations and Remote Sensing, IEEE Journal of]]></pubtitle>
##     <punumber><![CDATA[4609443]]></punumber>
##     <pubtype><![CDATA[Journals & Magazines]]></pubtype>
##     <publisher><![CDATA[IEEE]]></publisher>
##     <volume><![CDATA[7]]></volume>
##     <issue><![CDATA[2]]></issue>
##     <py><![CDATA[2014]]></py>
##     <spage><![CDATA[458]]></spage>
##     <epage><![CDATA[468]]></epage>
##     <abstract><![CDATA[Leaf area index (LAI) is a key surface parameter that describes the structure of vegetation and plays an important role in Earth system process modeling. In this paper, a new set of LAI products (MERSI GLOBCARBON LAI) has been developed based on the GLOBCARBON LAI algorithm and one year of FY-3A/MERSI land surface reflectance data. MERSI GLOBCARBON LAI has been inter-compared and validated over mainland China against MODIS land surface reflectance (LSR) derived LAI (using the same algorithm) and field LAI measurements. MERSI GLOBCARBON LAI and MODIS GLOBCARBON LAI show continuous and smooth LAI distributions at the start and end of the growing season. For most areas in China, the two LAI products agree well. The temporal variation in MERSI GLOBCARBON LAI and MODIS GLOBCARBON LAI consistently follows the growing season. The largest LAI difference occurs during July, when MERSI shows a much higher frequency of retrievals than does MODIS. Through validation of LAI retrievals with field measurements, our study demonstrates that LAI derived from MERSI and MODIS land surface reflectance products have comparable accuracy. MODIS top-of-atmosphere simple ratio (MODIS TOA SR) is related to MERSI TOA SR with linear correlation coefficients greater than 0.6. After atmospheric correction, the correlation coefficient increases from 0.69 to 0.75 over cropland and from 0.82 to 0.93 over grassland. However, atmospheric correction can still give rise to substantial differences in the reflectance data between the two sensors. Furthermore, different land cover types and different terrain relief have contrasting influences on the atmospheric correction, and these influences reduce the agreement between the two LAI products. This study shows the great potential of FY-3A/MERSI data for global LAI retrieval.]]></abstract>
##     <issn><![CDATA[1939-1404]]></issn>
##     <htmlFlag><![CDATA[1]]></htmlFlag>
##     <arnumber><![CDATA[6616017]]></arnumber>
##     <doi><![CDATA[10.1109/JSTARS.2013.2280466]]></doi>
##     <publicationId><![CDATA[6616017]]></publicationId>
##     <mdurl><![CDATA[http://ieeexplore.ieee.org/xpl/articleDetails.jsp?tp=&arnumber=6616017&contentType=Journals+%26+Magazines]]></mdurl>
##     <pdf><![CDATA[http://ieeexplore.ieee.org/stamp/stamp.jsp?arnumber=6616017]]></pdf>
##   </document>
##   <document>
##     <rank>645</rank>
##     <title><![CDATA[RF-Arbitrary Waveform Generation Based on Microwave Photonic Filtering]]></title>
##     <authors><![CDATA[Adams, R.;  Ashrafi, R.;  Junjia Wang;  Dizaji, M.R.;  Chen, L.R.]]></authors>
##     <affiliations><![CDATA[Dept. of Electr. & Comput. Eng., McGill Univ., Montreal, QC, Canada]]></affiliations>
##     <controlledterms>
##       <term><![CDATA[microwave filters]]></term>
##       <term><![CDATA[microwave photonics]]></term>
##       <term><![CDATA[multiwave mixing]]></term>
##       <term><![CDATA[nanowires]]></term>
##       <term><![CDATA[optical filters]]></term>
##       <term><![CDATA[silicon]]></term>
##       <term><![CDATA[waveform generators]]></term>
##     </controlledterms>
##     <thesaurusterms>
##       <term><![CDATA[Microwave filters]]></term>
##       <term><![CDATA[Nonlinear optics]]></term>
##       <term><![CDATA[Optical fibers]]></term>
##       <term><![CDATA[Optical filters]]></term>
##       <term><![CDATA[Optical pulses]]></term>
##       <term><![CDATA[Radio frequency]]></term>
##     </thesaurusterms>
##     <pubtitle><![CDATA[Photonics Journal, IEEE]]></pubtitle>
##     <punumber><![CDATA[4563994]]></punumber>
##     <pubtype><![CDATA[Journals & Magazines]]></pubtype>
##     <publisher><![CDATA[IEEE]]></publisher>
##     <volume><![CDATA[6]]></volume>
##     <issue><![CDATA[5]]></issue>
##     <py><![CDATA[2014]]></py>
##     <spage><![CDATA[1]]></spage>
##     <epage><![CDATA[8]]></epage>
##     <abstract><![CDATA[We demonstrate RF arbitrary waveform generation based on microwave photonic filtering. We use four-wave mixing in a silicon nanowire to increase the number of taps in an N-tap microwave photonic filter (MPF). Using a programmable optical filter, we can control the tap weights and, hence, the MPF response and corresponding generated waveform. We show uniform and apodized waveforms with four taps and seven taps with tunable central frequencies.]]></abstract>
##     <issn><![CDATA[1943-0655]]></issn>
##     <htmlFlag><![CDATA[1]]></htmlFlag>
##     <arnumber><![CDATA[6917197]]></arnumber>
##     <doi><![CDATA[10.1109/JPHOT.2014.2361637]]></doi>
##     <publicationId><![CDATA[6917197]]></publicationId>
##     <mdurl><![CDATA[http://ieeexplore.ieee.org/xpl/articleDetails.jsp?tp=&arnumber=6917197&contentType=Journals+%26+Magazines]]></mdurl>
##     <pdf><![CDATA[http://ieeexplore.ieee.org/stamp/stamp.jsp?arnumber=6917197]]></pdf>
##   </document>
##   <document>
##     <rank>646</rank>
##     <title><![CDATA[Feature Extraction Using Extrema Sampling of Discrete Derivatives for Spike Sorting in Implantable Upper-Limb Neural Prostheses]]></title>
##     <authors><![CDATA[Zamani, M.;  Demosthenous, A.]]></authors>
##     <affiliations><![CDATA[Dept. of Electron. & Electr. Eng., Univ. Coll. London, London, UK]]></affiliations>
##     <controlledterms>
##       <term><![CDATA[bioelectric phenomena]]></term>
##       <term><![CDATA[computational complexity]]></term>
##       <term><![CDATA[feature extraction]]></term>
##       <term><![CDATA[medical signal processing]]></term>
##       <term><![CDATA[neurophysiology]]></term>
##       <term><![CDATA[pattern clustering]]></term>
##       <term><![CDATA[prosthetics]]></term>
##       <term><![CDATA[signal classification]]></term>
##     </controlledterms>
##     <thesaurusterms>
##       <term><![CDATA[Accuracy]]></term>
##       <term><![CDATA[Clustering algorithms]]></term>
##       <term><![CDATA[Feature extraction]]></term>
##       <term><![CDATA[Microchannel]]></term>
##       <term><![CDATA[Principal component analysis]]></term>
##       <term><![CDATA[Shape]]></term>
##       <term><![CDATA[Sorting]]></term>
##     </thesaurusterms>
##     <pubtitle><![CDATA[Neural Systems and Rehabilitation Engineering, IEEE Transactions on]]></pubtitle>
##     <punumber><![CDATA[7333]]></punumber>
##     <pubtype><![CDATA[Journals & Magazines]]></pubtype>
##     <publisher><![CDATA[IEEE]]></publisher>
##     <volume><![CDATA[22]]></volume>
##     <issue><![CDATA[4]]></issue>
##     <py><![CDATA[2014]]></py>
##     <spage><![CDATA[716]]></spage>
##     <epage><![CDATA[726]]></epage>
##     <abstract><![CDATA[Next generation neural interfaces for upper-limb (and other) prostheses aim to develop implantable interfaces for one or more nerves, each interface having many neural signal channels that work reliably in the stump without harming the nerves. To achieve real-time multi-channel processing it is important to integrate spike sorting on-chip to overcome limitations in transmission bandwidth. This requires computationally efficient algorithms for feature extraction and clustering suitable for low-power hardware implementation. This paper describes a new feature extraction method for real-time spike sorting based on extrema analysis (namely positive peaks and negative peaks) of spike shapes and their discrete derivatives at different frequency bands. Employing simulation across different datasets, the accuracy and computational complexity of the proposed method are assessed and compared with other methods. The average classification accuracy of the proposed method in conjunction with online sorting (O-Sort) is 91.6%, outperforming all the other methods tested with the O-Sort clustering algorithm. The proposed method offers a better tradeoff between classification error and computational complexity, making it a particularly strong choice for on-chip spike sorting.]]></abstract>
##     <issn><![CDATA[1534-4320]]></issn>
##     <htmlFlag><![CDATA[1]]></htmlFlag>
##     <arnumber><![CDATA[6757022]]></arnumber>
##     <doi><![CDATA[10.1109/TNSRE.2014.2309678]]></doi>
##     <publicationId><![CDATA[6757022]]></publicationId>
##     <mdurl><![CDATA[http://ieeexplore.ieee.org/xpl/articleDetails.jsp?tp=&arnumber=6757022&contentType=Journals+%26+Magazines]]></mdurl>
##     <pdf><![CDATA[http://ieeexplore.ieee.org/stamp/stamp.jsp?arnumber=6757022]]></pdf>
##   </document>
##   <document>
##     <rank>647</rank>
##     <title><![CDATA[Reconfigurable Parallel Coupled Band Notch Resonator With Wide Tuning Range]]></title>
##     <authors><![CDATA[Zhengpeng Wang;  Kelly, J.R.;  Hall, P.S.;  Borja, A.L.;  Gardner, P.]]></authors>
##     <affiliations><![CDATA[Sch. of Autom. & Electr. Eng., Univ. of Sci. & Technol. Beijing, Beijing, China]]></affiliations>
##     <controlledterms>
##       <term><![CDATA[band-stop filters]]></term>
##       <term><![CDATA[equivalent circuits]]></term>
##       <term><![CDATA[resonators]]></term>
##       <term><![CDATA[varactors]]></term>
##     </controlledterms>
##     <thesaurusterms>
##       <term><![CDATA[Capacitance]]></term>
##       <term><![CDATA[Equivalent circuits]]></term>
##       <term><![CDATA[Resonant frequency]]></term>
##       <term><![CDATA[Resonator filters]]></term>
##       <term><![CDATA[Tuning]]></term>
##       <term><![CDATA[Varactors]]></term>
##     </thesaurusterms>
##     <pubtitle><![CDATA[Industrial Electronics, IEEE Transactions on]]></pubtitle>
##     <punumber><![CDATA[41]]></punumber>
##     <pubtype><![CDATA[Journals & Magazines]]></pubtype>
##     <publisher><![CDATA[IEEE]]></publisher>
##     <volume><![CDATA[61]]></volume>
##     <issue><![CDATA[11]]></issue>
##     <py><![CDATA[2014]]></py>
##     <spage><![CDATA[6316]]></spage>
##     <epage><![CDATA[6326]]></epage>
##     <abstract><![CDATA[This paper presents a new form of reconfigurable band notch resonator with a wide tuning range. The whole structure consists of a through line and two parallel coupled resonators. This paper presents an accurate lumped element equivalent circuit for the parallel coupled resonator. The mechanism giving rise to the wide tuning range is also investigated. The first harmonic of the fundamental stop-band is analyzed and compared with that of the varactor-coupled resonator. Measurement results show that the stop-band of the filter can be tuned from 0.6 to 4.0 GHz. This is achieved by varying the capacitance of two pairs of varactor diodes. The stop-band tuning range is restricted by the capacitance variation which can be achieved using the varactor diode. The potential tuning range of a filter based on this resonator is greater than one decade.]]></abstract>
##     <issn><![CDATA[0278-0046]]></issn>
##     <htmlFlag><![CDATA[1]]></htmlFlag>
##     <arnumber><![CDATA[6766257]]></arnumber>
##     <doi><![CDATA[10.1109/TIE.2014.2311386]]></doi>
##     <publicationId><![CDATA[6766257]]></publicationId>
##     <mdurl><![CDATA[http://ieeexplore.ieee.org/xpl/articleDetails.jsp?tp=&arnumber=6766257&contentType=Journals+%26+Magazines]]></mdurl>
##     <pdf><![CDATA[http://ieeexplore.ieee.org/stamp/stamp.jsp?arnumber=6766257]]></pdf>
##   </document>
##   <document>
##     <rank>648</rank>
##     <title><![CDATA[All-Optical OFDM Generation for IEEE802.11a Based on Soliton Carriers Using Microring Resonators]]></title>
##     <authors><![CDATA[Alavi, S.E.;  Amiri, I.S.;  Idrus, S.M.;  Supa'at, A.S.M.;  Ali, J.;  Yupapin, P.P.]]></authors>
##     <affiliations><![CDATA[Lightwave Commun. Res. Group, Univ. Teknol. Malaysia (UTM), Skudai, Malaysia]]></affiliations>
##     <controlledterms>
##       <term><![CDATA[OFDM modulation]]></term>
##       <term><![CDATA[micromechanical resonators]]></term>
##       <term><![CDATA[optical communication]]></term>
##       <term><![CDATA[optical modulation]]></term>
##       <term><![CDATA[optical solitons]]></term>
##       <term><![CDATA[wireless LAN]]></term>
##     </controlledterms>
##     <thesaurusterms>
##       <term><![CDATA[Nonlinear optics]]></term>
##       <term><![CDATA[OFDM]]></term>
##       <term><![CDATA[Optical fibers]]></term>
##       <term><![CDATA[Optical filters]]></term>
##       <term><![CDATA[Optical ring resonators]]></term>
##       <term><![CDATA[Optical solitons]]></term>
##       <term><![CDATA[Optical transmitters]]></term>
##     </thesaurusterms>
##     <pubtitle><![CDATA[Photonics Journal, IEEE]]></pubtitle>
##     <punumber><![CDATA[4563994]]></punumber>
##     <pubtype><![CDATA[Journals & Magazines]]></pubtype>
##     <publisher><![CDATA[IEEE]]></publisher>
##     <volume><![CDATA[6]]></volume>
##     <issue><![CDATA[1]]></issue>
##     <py><![CDATA[2014]]></py>
##     <spage><![CDATA[1]]></spage>
##     <epage><![CDATA[9]]></epage>
##     <abstract><![CDATA[The optical carrier generation is the basic building block to implement all-optical orthogonal frequency-division multiplexing (OFDM) transmission. One method to optically generate single and multicarriers is to use the microring resonator (MRR). The MRRs can be used as filter devices, where generation of high-frequency (GHz) soliton signals as single and multicarriers can be performed using suitable system parameters. Here, the optical soliton in a nonlinear fiber MRR system is analyzed, using a modified add/drop system known as a Panda ring resonator connected to an add/drop system. In order to set up a transmission system, i.e., IEEE802.11a, first, 64 uniform optical carriers were generated and separated by a splitter and modulated; afterward, the spectra of the modulated optical subcarriers are overlapped, which results one optical OFDM channel band. The quadrature amplitude modulation (QAM) and 16-QAM are used for modulating the subcarriers. The generated OFDM signal is multiplexed with a single-carrier soliton and transmitted through the single-mode fiber (SMF). After photodetection, the radio frequency (RF) signal was propagated. On the receiver side, the RF signal was optically modulated and processed. The results show the generation of 64 multicarriers evenly spaced in the range from 54.09 to 55.01 GHz, where demodulation of these signals is performed, and the performance of the system is analyzed.]]></abstract>
##     <issn><![CDATA[1943-0655]]></issn>
##     <htmlFlag><![CDATA[1]]></htmlFlag>
##     <arnumber><![CDATA[6725601]]></arnumber>
##     <doi><![CDATA[10.1109/JPHOT.2014.2302791]]></doi>
##     <publicationId><![CDATA[6725601]]></publicationId>
##     <mdurl><![CDATA[http://ieeexplore.ieee.org/xpl/articleDetails.jsp?tp=&arnumber=6725601&contentType=Journals+%26+Magazines]]></mdurl>
##     <pdf><![CDATA[http://ieeexplore.ieee.org/stamp/stamp.jsp?arnumber=6725601]]></pdf>
##   </document>
##   <document>
##     <rank>649</rank>
##     <title><![CDATA[EPPP4SMS: Efficient Privacy-Preserving Protocol for Smart Metering Systems and Its Simulation Using Real-World Data]]></title>
##     <authors><![CDATA[Borges, F.;  Muhlhauser, M.]]></authors>
##     <affiliations><![CDATA[Tech. Univ. Darmstadt, Darmstadt, Germany]]></affiliations>
##     <controlledterms>
##       <term><![CDATA[cryptographic protocols]]></term>
##       <term><![CDATA[data privacy]]></term>
##       <term><![CDATA[power system security]]></term>
##       <term><![CDATA[smart meters]]></term>
##       <term><![CDATA[smart power grids]]></term>
##       <term><![CDATA[substations]]></term>
##       <term><![CDATA[time series]]></term>
##     </controlledterms>
##     <thesaurusterms>
##       <term><![CDATA[Encryption]]></term>
##       <term><![CDATA[Power system security]]></term>
##       <term><![CDATA[Privacy]]></term>
##       <term><![CDATA[Protocols]]></term>
##       <term><![CDATA[Smart meters]]></term>
##       <term><![CDATA[Time series analysis]]></term>
##     </thesaurusterms>
##     <pubtitle><![CDATA[Smart Grid, IEEE Transactions on]]></pubtitle>
##     <punumber><![CDATA[5165411]]></punumber>
##     <pubtype><![CDATA[Journals & Magazines]]></pubtype>
##     <publisher><![CDATA[IEEE]]></publisher>
##     <volume><![CDATA[5]]></volume>
##     <issue><![CDATA[6]]></issue>
##     <py><![CDATA[2014]]></py>
##     <spage><![CDATA[2701]]></spage>
##     <epage><![CDATA[2708]]></epage>
##     <abstract><![CDATA[The main contribution of this paper is the construction of the efficient privacy-preserving protocol for smart metering systems (EPPP4SMS), which brings together features of the best privacy-preserving protocols in the literature for smart grids. In addition, EPPP4SMS is faster on the meter side-and in the whole round (encryption, aggregation, and decryption)-than other protocols based on homomorphic encryption and it is still scalable. Moreover, EPPP4SMS enables energy suppliers and customers to verify the billing information and measurements without leaking private information. Since the energy supplier knows the amount of generated electricity and its flow throughout electrical substations, the energy supplier can use this verification to detect energy loss and fraud. Different from verification based on digital signature, our verification enables new features; for example, smart meters and their energy supplier can compute the verification without storing the respective encrypted measurements. Furthermore, EPPP4SMS may be suitable to many other scenarios that need aggregation of time-series data keeping privacy protected, including electronic voting, reputation systems, and sensor networks. In this paper, we present theoretical results of EPPP4SMS and their validation by implementation of algorithms and simulation using real-world measurement data.]]></abstract>
##     <issn><![CDATA[1949-3053]]></issn>
##     <htmlFlag><![CDATA[1]]></htmlFlag>
##     <arnumber><![CDATA[6862920]]></arnumber>
##     <doi><![CDATA[10.1109/TSG.2014.2336265]]></doi>
##     <publicationId><![CDATA[6862920]]></publicationId>
##     <mdurl><![CDATA[http://ieeexplore.ieee.org/xpl/articleDetails.jsp?tp=&arnumber=6862920&contentType=Journals+%26+Magazines]]></mdurl>
##     <pdf><![CDATA[http://ieeexplore.ieee.org/stamp/stamp.jsp?arnumber=6862920]]></pdf>
##   </document>
##   <document>
##     <rank>650</rank>
##     <title><![CDATA[Integrated Magneto-Optical Materials and Isolators: A Review]]></title>
##     <authors><![CDATA[Stadler, B.J.H.;  Mizumoto, T.]]></authors>
##     <affiliations><![CDATA[Electr. & Comput. Eng., Univ. of Minnesota, Minneapolis, MN, USA]]></affiliations>
##     <controlledterms>
##       <term><![CDATA[birefringence]]></term>
##       <term><![CDATA[integrated optics]]></term>
##       <term><![CDATA[magneto-optical devices]]></term>
##       <term><![CDATA[optical circulators]]></term>
##     </controlledterms>
##     <thesaurusterms>
##       <term><![CDATA[Faraday effect]]></term>
##       <term><![CDATA[Garnets]]></term>
##       <term><![CDATA[Isolators]]></term>
##       <term><![CDATA[Magnetooptic devices]]></term>
##       <term><![CDATA[Optical waveguides]]></term>
##     </thesaurusterms>
##     <pubtitle><![CDATA[Photonics Journal, IEEE]]></pubtitle>
##     <punumber><![CDATA[4563994]]></punumber>
##     <pubtype><![CDATA[Journals & Magazines]]></pubtype>
##     <publisher><![CDATA[IEEE]]></publisher>
##     <volume><![CDATA[6]]></volume>
##     <issue><![CDATA[1]]></issue>
##     <py><![CDATA[2014]]></py>
##     <spage><![CDATA[1]]></spage>
##     <epage><![CDATA[15]]></epage>
##     <abstract><![CDATA[Many novel materials and device designs have been proposed as photonic analogs to electrical diodes over the last four decades. This paper seeks to revisit these materials and designs as advanced technologies may enable experimental realization that was not possible upon conception of several of these designs. The background behind integration challenges, including waveguide birefringence, fabrication tolerances, garnet/semiconductor mismatch, and optimized interfaces will hopefully spark new ideas that will finally enable the realization of integrated optical isolators and circulators.]]></abstract>
##     <issn><![CDATA[1943-0655]]></issn>
##     <htmlFlag><![CDATA[1]]></htmlFlag>
##     <arnumber><![CDATA[6678206]]></arnumber>
##     <doi><![CDATA[10.1109/JPHOT.2013.2293618]]></doi>
##     <publicationId><![CDATA[6678206]]></publicationId>
##     <mdurl><![CDATA[http://ieeexplore.ieee.org/xpl/articleDetails.jsp?tp=&arnumber=6678206&contentType=Journals+%26+Magazines]]></mdurl>
##     <pdf><![CDATA[http://ieeexplore.ieee.org/stamp/stamp.jsp?arnumber=6678206]]></pdf>
##   </document>
##   <document>
##     <rank>651</rank>
##     <title><![CDATA[Managing Wheat From Space: Linking MODIS NDVI and Crop Models for Predicting Australian Dryland Wheat Biomass]]></title>
##     <authors><![CDATA[Perry, E.M.;  Morse-McNabb, E.M.;  Nuttall, J.G.;  O'Leary, G.J.;  Clark, R.]]></authors>
##     <affiliations><![CDATA[Dept. of Environ. & Primary Ind., Agric. Res., Epsom, VIC, Australia]]></affiliations>
##     <controlledterms>
##       <term><![CDATA[remote sensing]]></term>
##       <term><![CDATA[vegetation]]></term>
##     </controlledterms>
##     <thesaurusterms>
##       <term><![CDATA[Agriculture]]></term>
##       <term><![CDATA[Biological system modeling]]></term>
##       <term><![CDATA[Biomass]]></term>
##       <term><![CDATA[MODIS]]></term>
##       <term><![CDATA[Mathematical model]]></term>
##       <term><![CDATA[Remote sensing]]></term>
##       <term><![CDATA[Vegetation mapping]]></term>
##     </thesaurusterms>
##     <pubtitle><![CDATA[Selected Topics in Applied Earth Observations and Remote Sensing, IEEE Journal of]]></pubtitle>
##     <punumber><![CDATA[4609443]]></punumber>
##     <pubtype><![CDATA[Journals & Magazines]]></pubtype>
##     <publisher><![CDATA[IEEE]]></publisher>
##     <volume><![CDATA[7]]></volume>
##     <issue><![CDATA[9]]></issue>
##     <py><![CDATA[2014]]></py>
##     <spage><![CDATA[3724]]></spage>
##     <epage><![CDATA[3731]]></epage>
##     <abstract><![CDATA[This study explored the relationships between moderate resolution imaging spectroradiometer (MODIS) NDVI observations with both measured and simulated fractional green cover (FGrC), leaf area index (LAI), and above ground biomass (AGB) for dryland wheat in Australia. A total of 37 paddocks in north-western Victoria, Australia, were sampled during 2003-2006 for AGB at anthesis, and for FGrC, NDVI (from an active optical sensor), and AGB during 2012. The 2012 FGrC and NDVI measurements were fitted to MODIS NDVI, resulting in positive, linear relationships when the MODIS NDVI values were &#x2264; 0.80. Measured AGB was also positively, linearly related to MODIS summed NDVI, resulting in an overall R<sup>2</sup> of 0.81 and root mean square error (RMSE) of 1397 kg/ha. Crop simulations were run for the fourteen paddocks from 2003 to 2006, and six paddocks from 2012. Four crop phenological points were selected to extract corresponding NDVI and simulated crop parameters: emergence, peak LAI, the mid-point between emergence and peak LAI, and anthesis. Linear models were fit between the MODIS NDVI and simulated values of FGrC, LAI, and AGB. Overall, the highest R<sup>2</sup> values corresponded to using all of the dates for FGrC (R<sup>2</sup> = 0.82) and AGB (R<sup>2</sup> = 0.92), and anthesis dates for LAI (R<sup>2</sup> = 0.74). For FGrC and AGB, the RMSE with simulated parameters were comparable or better than the equivalent results from the in situ measurements (note that there were no LAI in situ measurements to compare with). The results support the notion for extending the value of the MODIS NDVI using crop simulation models. The combination of remotely sensed and simulation data might offer regional maps of spatial AGB and ultimately grain yield, which would have high value for research, resource management, policy, and potentially, crop management.]]></abstract>
##     <issn><![CDATA[1939-1404]]></issn>
##     <htmlFlag><![CDATA[1]]></htmlFlag>
##     <arnumber><![CDATA[6823087]]></arnumber>
##     <doi><![CDATA[10.1109/JSTARS.2014.2323705]]></doi>
##     <publicationId><![CDATA[6823087]]></publicationId>
##     <mdurl><![CDATA[http://ieeexplore.ieee.org/xpl/articleDetails.jsp?tp=&arnumber=6823087&contentType=Journals+%26+Magazines]]></mdurl>
##     <pdf><![CDATA[http://ieeexplore.ieee.org/stamp/stamp.jsp?arnumber=6823087]]></pdf>
##   </document>
##   <document>
##     <rank>652</rank>
##     <title><![CDATA[Polytope Codes Against Adversaries in Networks]]></title>
##     <authors><![CDATA[Kosut, O.;  Lang Tong;  Tse, D.N.C.]]></authors>
##     <affiliations><![CDATA[Sch. of Electr., Comput. & Energy Eng., Arizona State Univ., Tempe, AZ, USA]]></affiliations>
##     <controlledterms>
##       <term><![CDATA[linear codes]]></term>
##       <term><![CDATA[network coding]]></term>
##     </controlledterms>
##     <thesaurusterms>
##       <term><![CDATA[Educational institutions]]></term>
##       <term><![CDATA[Linear codes]]></term>
##       <term><![CDATA[Network coding]]></term>
##       <term><![CDATA[Upper bound]]></term>
##       <term><![CDATA[Vectors]]></term>
##       <term><![CDATA[Xenon]]></term>
##     </thesaurusterms>
##     <pubtitle><![CDATA[Information Theory, IEEE Transactions on]]></pubtitle>
##     <punumber><![CDATA[18]]></punumber>
##     <pubtype><![CDATA[Journals & Magazines]]></pubtype>
##     <publisher><![CDATA[IEEE]]></publisher>
##     <volume><![CDATA[60]]></volume>
##     <issue><![CDATA[6]]></issue>
##     <py><![CDATA[2014]]></py>
##     <spage><![CDATA[3308]]></spage>
##     <epage><![CDATA[3344]]></epage>
##     <abstract><![CDATA[This paper investigates a network coding problem wherein an adversary controls a subset of nodes in the network of limited quantity but unknown location. This problem is shown to be more difficult than that of an adversary controlling a given number of edges in the network, in that linear codes are insufficient. To solve the node problem, the class of polytope codes is introduced. Polytope codes are constant composition codes operating over bounded polytopes in integer vector fields. The polytope structure creates additional complexity, but it induces properties on marginal distributions of code vectors so that validities of codewords can be checked by internal nodes of the network. It is shown that polytope codes achieve a cut-set bound for a class of planar networks. It is also shown that this cut-set bound is not always tight, and a tighter bound is given for an example network.]]></abstract>
##     <issn><![CDATA[0018-9448]]></issn>
##     <arnumber><![CDATA[6781646]]></arnumber>
##     <doi><![CDATA[10.1109/TIT.2014.2314642]]></doi>
##     <publicationId><![CDATA[6781646]]></publicationId>
##     <mdurl><![CDATA[http://ieeexplore.ieee.org/xpl/articleDetails.jsp?tp=&arnumber=6781646&contentType=Journals+%26+Magazines]]></mdurl>
##     <pdf><![CDATA[http://ieeexplore.ieee.org/stamp/stamp.jsp?arnumber=6781646]]></pdf>
##   </document>
##   <document>
##     <rank>653</rank>
##     <title><![CDATA[Nonlinear Fiber Loop Mirror Optimization to Enhance the Performance of Multiwavelength Brillouin/Erbium-Doped Fiber Laser]]></title>
##     <authors><![CDATA[Al-Alimi, A.W.;  Yaacob, M.H.;  Abas, A.F.]]></authors>
##     <affiliations><![CDATA[Wireless & Photonic Networks Res. Centre, Univ. Putra Malaysia, Serdang, Malaysia]]></affiliations>
##     <controlledterms>
##       <term><![CDATA[Brillouin spectra]]></term>
##       <term><![CDATA[erbium]]></term>
##       <term><![CDATA[fibre lasers]]></term>
##       <term><![CDATA[laser mirrors]]></term>
##       <term><![CDATA[nonlinear optics]]></term>
##       <term><![CDATA[optical pumping]]></term>
##     </controlledterms>
##     <thesaurusterms>
##       <term><![CDATA[Cavity resonators]]></term>
##       <term><![CDATA[Couplers]]></term>
##       <term><![CDATA[Erbium-doped fiber lasers]]></term>
##       <term><![CDATA[Mirrors]]></term>
##       <term><![CDATA[Optical fibers]]></term>
##       <term><![CDATA[Pump lasers]]></term>
##       <term><![CDATA[Scattering]]></term>
##     </thesaurusterms>
##     <pubtitle><![CDATA[Photonics Journal, IEEE]]></pubtitle>
##     <punumber><![CDATA[4563994]]></punumber>
##     <pubtype><![CDATA[Journals & Magazines]]></pubtype>
##     <publisher><![CDATA[IEEE]]></publisher>
##     <volume><![CDATA[6]]></volume>
##     <issue><![CDATA[6]]></issue>
##     <py><![CDATA[2014]]></py>
##     <spage><![CDATA[1]]></spage>
##     <epage><![CDATA[10]]></epage>
##     <abstract><![CDATA[The performance of multiwavelength BEFL based on nonlinear fiber loop mirror (NOLM) optimization has been experimentally investigated. The NOLM is optimized to act as highly reflective mirror, which results in reducing the threshold power and enhancing the number of the generated Stokes lines at low EDF pump power as well. This design presents a low threshold power of 2 mW and generates up to 26 Stokes lines at EDF pump power of 25 mW.]]></abstract>
##     <issn><![CDATA[1943-0655]]></issn>
##     <htmlFlag><![CDATA[1]]></htmlFlag>
##     <arnumber><![CDATA[6966726]]></arnumber>
##     <doi><![CDATA[10.1109/JPHOT.2014.2366171]]></doi>
##     <publicationId><![CDATA[6966726]]></publicationId>
##     <mdurl><![CDATA[http://ieeexplore.ieee.org/xpl/articleDetails.jsp?tp=&arnumber=6966726&contentType=Journals+%26+Magazines]]></mdurl>
##     <pdf><![CDATA[http://ieeexplore.ieee.org/stamp/stamp.jsp?arnumber=6966726]]></pdf>
##   </document>
##   <document>
##     <rank>654</rank>
##     <title><![CDATA[Joint Congestion Control and Scheduling in Wireless Networks With Network Coding]]></title>
##     <authors><![CDATA[Ronghui Hou;  King-Shan Lui;  Jiandong Li]]></authors>
##     <affiliations><![CDATA[State Key Lab. of Integrated Service Networks, Xidian Univ., Xi'an, China]]></affiliations>
##     <controlledterms>
##       <term><![CDATA[channel allocation]]></term>
##       <term><![CDATA[network coding]]></term>
##       <term><![CDATA[radio networks]]></term>
##       <term><![CDATA[resource allocation]]></term>
##       <term><![CDATA[telecommunication congestion control]]></term>
##     </controlledterms>
##     <thesaurusterms>
##       <term><![CDATA[Encoding]]></term>
##       <term><![CDATA[Interference]]></term>
##       <term><![CDATA[Network coding]]></term>
##       <term><![CDATA[Schedules]]></term>
##       <term><![CDATA[Scheduling]]></term>
##       <term><![CDATA[Throughput]]></term>
##       <term><![CDATA[Wireless networks]]></term>
##     </thesaurusterms>
##     <pubtitle><![CDATA[Vehicular Technology, IEEE Transactions on]]></pubtitle>
##     <punumber><![CDATA[25]]></punumber>
##     <pubtype><![CDATA[Journals & Magazines]]></pubtype>
##     <publisher><![CDATA[IEEE]]></publisher>
##     <volume><![CDATA[63]]></volume>
##     <issue><![CDATA[7]]></issue>
##     <py><![CDATA[2014]]></py>
##     <spage><![CDATA[3304]]></spage>
##     <epage><![CDATA[3317]]></epage>
##     <abstract><![CDATA[This paper studies how to perform joint congestion control and scheduling with network coding in wireless networks. Under network coding, a node may need to buffer a sent packet for decoding a packet to be received later. If sent packets are not forgotten smartly, much buffer space will be taken up, leading to dropping of new incoming packets. This unexpected packet dropping harms the final throughput obtained, although optimal scheduling has been used. To solve the problem, we introduce a new node model incorporating a transmission-mode preassignment procedure and a scheduling procedure. The introduced transmission-mode preassignment avoids memorizing several sent packets to reduce buffer overhead. We develop a new scheduling policy based on our node model and analyze formally the stability property of a network system using the proposed policy. We finally evaluate the efficiency of our algorithm through simulations from the perspectives of throughput and packet loss ratio.]]></abstract>
##     <issn><![CDATA[0018-9545]]></issn>
##     <htmlFlag><![CDATA[1]]></htmlFlag>
##     <arnumber><![CDATA[6736091]]></arnumber>
##     <doi><![CDATA[10.1109/TVT.2014.2298404]]></doi>
##     <publicationId><![CDATA[6736091]]></publicationId>
##     <mdurl><![CDATA[http://ieeexplore.ieee.org/xpl/articleDetails.jsp?tp=&arnumber=6736091&contentType=Journals+%26+Magazines]]></mdurl>
##     <pdf><![CDATA[http://ieeexplore.ieee.org/stamp/stamp.jsp?arnumber=6736091]]></pdf>
##   </document>
##   <document>
##     <rank>655</rank>
##     <title><![CDATA[Fuzzy Cognitive Network Process: Comparisons With Fuzzy Analytic Hierarchy Process in New Product Development Strategy]]></title>
##     <authors><![CDATA[Yuen, K.K.F.]]></authors>
##     <affiliations><![CDATA[Dept. of Comput. Sci. & Software Eng., Xi'an Jiaotong-Liverpool Univ., Suzhou, China]]></affiliations>
##     <controlledterms>
##       <term><![CDATA[analytic hierarchy process]]></term>
##       <term><![CDATA[cognition]]></term>
##       <term><![CDATA[fuzzy set theory]]></term>
##       <term><![CDATA[product development]]></term>
##     </controlledterms>
##     <pubtitle><![CDATA[Fuzzy Systems, IEEE Transactions on]]></pubtitle>
##     <punumber><![CDATA[91]]></punumber>
##     <pubtype><![CDATA[Journals & Magazines]]></pubtype>
##     <publisher><![CDATA[IEEE]]></publisher>
##     <volume><![CDATA[22]]></volume>
##     <issue><![CDATA[3]]></issue>
##     <py><![CDATA[2014]]></py>
##     <spage><![CDATA[597]]></spage>
##     <epage><![CDATA[610]]></epage>
##     <abstract><![CDATA[Fuzzy analytic hierarchy process (F-AHP) has increasingly been applied in many areas. However, as the perception and cognition toward the semantic representation for the linguistic rating scale used by the fuzzy pairwise comparison in F-AHP are still open to discuss, F-AHP very likely produces misapplications. This research proposes the fuzzy cognitive network process (F-CNP) as an ideal alternative to F-AHP. A new product development application using F-AHP is revised using F-CNP. This study shows that the proposed F-CNP yields better results due to the appropriate mathematical definition of the fuzzy paired interval scale for the human perception of paired difference.]]></abstract>
##     <issn><![CDATA[1063-6706]]></issn>
##     <htmlFlag><![CDATA[1]]></htmlFlag>
##     <arnumber><![CDATA[6542700]]></arnumber>
##     <doi><![CDATA[10.1109/TFUZZ.2013.2269150]]></doi>
##     <publicationId><![CDATA[6542700]]></publicationId>
##     <mdurl><![CDATA[http://ieeexplore.ieee.org/xpl/articleDetails.jsp?tp=&arnumber=6542700&contentType=Journals+%26+Magazines]]></mdurl>
##     <pdf><![CDATA[http://ieeexplore.ieee.org/stamp/stamp.jsp?arnumber=6542700]]></pdf>
##   </document>
##   <document>
##     <rank>656</rank>
##     <title><![CDATA[Amplitude Noise and Timing Jitter Characterization of a High-Power Mode-Locked Integrated External-Cavity Surface Emitting Laser]]></title>
##     <authors><![CDATA[Mangold, M.;  Link, S.M.;  Klenner, A.;  Zaugg, C.A.;  Golling, M.;  Tilma, B.W.;  Keller, U.]]></authors>
##     <affiliations><![CDATA[Dept. of Phys., ETH Zurich, Zurich, Switzerland]]></affiliations>
##     <controlledterms>
##       <term><![CDATA[integrated optoelectronics]]></term>
##       <term><![CDATA[laser beams]]></term>
##       <term><![CDATA[laser cavity resonators]]></term>
##       <term><![CDATA[laser feedback]]></term>
##       <term><![CDATA[laser mode locking]]></term>
##       <term><![CDATA[laser noise]]></term>
##       <term><![CDATA[laser stability]]></term>
##       <term><![CDATA[optical pulse generation]]></term>
##       <term><![CDATA[optical saturable absorption]]></term>
##       <term><![CDATA[piezoelectric actuators]]></term>
##       <term><![CDATA[semiconductor lasers]]></term>
##       <term><![CDATA[surface emitting lasers]]></term>
##       <term><![CDATA[timing jitter]]></term>
##     </controlledterms>
##     <thesaurusterms>
##       <term><![CDATA[Cavity resonators]]></term>
##       <term><![CDATA[Laser mode locking]]></term>
##       <term><![CDATA[Laser noise]]></term>
##       <term><![CDATA[Vertical cavity surface emitting lasers]]></term>
##     </thesaurusterms>
##     <pubtitle><![CDATA[Photonics Journal, IEEE]]></pubtitle>
##     <punumber><![CDATA[4563994]]></punumber>
##     <pubtype><![CDATA[Journals & Magazines]]></pubtype>
##     <publisher><![CDATA[IEEE]]></publisher>
##     <volume><![CDATA[6]]></volume>
##     <issue><![CDATA[1]]></issue>
##     <py><![CDATA[2014]]></py>
##     <spage><![CDATA[1]]></spage>
##     <epage><![CDATA[9]]></epage>
##     <abstract><![CDATA[We present a timing jitter and amplitude noise characterization of a high-power mode-locked integrated external-cavity surface emitting laser (MIXSEL). In the MIXSEL, the semiconductor saturable absorber of a SESAM is integrated into the structure of a VECSEL to start and stabilize passive mode-locking. In comparison to previous noise characterization of SESAM-mode-locked VECSELs, this first noise characterization of a MIXSEL is performed at a much higher average output power. In a free-running operation, the laser generates 14.3-ps pulses at an average output power of 645 mW at a 2-GHz pulse repetition rate and an RMS amplitude noise of 0.15% [1 Hz, 10 MHz]. We measured an RMS timing jitter of 129 fs [100 Hz, 10 MHz], which represents the lowest value for a free-running passively mode-locked semiconductor disk laser to date. Additionally, we stabilized the pulse repetition rate with a piezo actuator to control the cavity length. With the laser generating 16.7-ps pulses at an average output power of 701 mW, the repetition frequency was phase-locked to a low-noise electronic reference using a feedback loop. In actively stabilized operation, the RMS timing jitter was reduced to less than 70 fs [1 Hz, 100 MHz]. In the 100-Hz to 10-MHz bandwidth, we report the lowest timing jitter measured from a passively mode-locked semiconductor disk laser to date with a value of 31 fs. These results show that the MIXSEL technology provides compact ultrafast laser sources combining high-power and low-noise performance similar to diode-pumped solid-state lasers, which enable world-record optical communication rates and low-noise frequency combs.]]></abstract>
##     <issn><![CDATA[1943-0655]]></issn>
##     <htmlFlag><![CDATA[1]]></htmlFlag>
##     <arnumber><![CDATA[6690115]]></arnumber>
##     <doi><![CDATA[10.1109/JPHOT.2013.2295464]]></doi>
##     <publicationId><![CDATA[6690115]]></publicationId>
##     <mdurl><![CDATA[http://ieeexplore.ieee.org/xpl/articleDetails.jsp?tp=&arnumber=6690115&contentType=Journals+%26+Magazines]]></mdurl>
##     <pdf><![CDATA[http://ieeexplore.ieee.org/stamp/stamp.jsp?arnumber=6690115]]></pdf>
##   </document>
##   <document>
##     <rank>657</rank>
##     <title><![CDATA[Top-<formula formulatype="inline"> <img src="/images/tex/348.gif" alt="k"> </formula> Query Result Completeness Verification in Tiered Sensor Networks]]></title>
##     <authors><![CDATA[Chia-Mu Yu;  Guo-Kai Ni;  Ing-Yi Chen;  Gelenbe, E.;  Sy-Yen Kuo]]></authors>
##     <affiliations><![CDATA[Dept. of Comput. Sci. & Eng., Yuan Ze Univ., Taoyuan, Taiwan]]></affiliations>
##     <controlledterms>
##       <term><![CDATA[communication complexity]]></term>
##       <term><![CDATA[query processing]]></term>
##       <term><![CDATA[telecommunication computing]]></term>
##       <term><![CDATA[wireless sensor networks]]></term>
##     </controlledterms>
##     <thesaurusterms>
##       <term><![CDATA[Computer architecture]]></term>
##       <term><![CDATA[Encryption]]></term>
##       <term><![CDATA[Microprocessors]]></term>
##       <term><![CDATA[Privacy]]></term>
##       <term><![CDATA[Topology]]></term>
##     </thesaurusterms>
##     <pubtitle><![CDATA[Information Forensics and Security, IEEE Transactions on]]></pubtitle>
##     <punumber><![CDATA[10206]]></punumber>
##     <pubtype><![CDATA[Journals & Magazines]]></pubtype>
##     <publisher><![CDATA[IEEE]]></publisher>
##     <volume><![CDATA[9]]></volume>
##     <issue><![CDATA[1]]></issue>
##     <py><![CDATA[2014]]></py>
##     <spage><![CDATA[109]]></spage>
##     <epage><![CDATA[124]]></epage>
##     <abstract><![CDATA[Storage nodes are expected to be placed as an intermediate tier of large scale sensor networks for caching the collected sensor readings and responding to queries with benefits of power and storage saving for ordinary sensors. Nevertheless, an important issue is that the compromised storage node may not only cause the privacy problem, but also return fake/incomplete query results. We propose a simple yet effective dummy reading-based anonymization framework, under which the query result integrity can be guaranteed by our proposed verifiable top- k query (VQ) schemes. Compared with existing works, the VQ schemes have a fundamentally different design philosophy and achieve the lower communication complexity at the cost of slight detection capability degradation. Analytical studies, numerical simulations, and prototype implementations are conducted to demonstrate the practicality of our proposed methods.]]></abstract>
##     <issn><![CDATA[1556-6013]]></issn>
##     <htmlFlag><![CDATA[1]]></htmlFlag>
##     <arnumber><![CDATA[6665142]]></arnumber>
##     <doi><![CDATA[10.1109/TIFS.2013.2291326]]></doi>
##     <publicationId><![CDATA[6665142]]></publicationId>
##     <mdurl><![CDATA[http://ieeexplore.ieee.org/xpl/articleDetails.jsp?tp=&arnumber=6665142&contentType=Journals+%26+Magazines]]></mdurl>
##     <pdf><![CDATA[http://ieeexplore.ieee.org/stamp/stamp.jsp?arnumber=6665142]]></pdf>
##   </document>
##   <document>
##     <rank>658</rank>
##     <title><![CDATA[Electroluminescence Devices Based on Si Quantum Dots/SiC Multilayers Embedded in PN Junction]]></title>
##     <authors><![CDATA[Xu, X.;  Cao, Y.Q.;  Lu, P.;  Xu, J.;  Li, W.;  Chen, K.J.]]></authors>
##     <affiliations><![CDATA[Nat. Lab. of Solid State Microstructures, Nanjing Univ., Nanjing, China]]></affiliations>
##     <controlledterms>
##       <term><![CDATA[Raman spectra]]></term>
##       <term><![CDATA[amorphous semiconductors]]></term>
##       <term><![CDATA[crystallisation]]></term>
##       <term><![CDATA[electroluminescence]]></term>
##       <term><![CDATA[electroluminescent devices]]></term>
##       <term><![CDATA[electron-hole recombination]]></term>
##       <term><![CDATA[elemental semiconductors]]></term>
##       <term><![CDATA[laser materials processing]]></term>
##       <term><![CDATA[multilayers]]></term>
##       <term><![CDATA[p-n junctions]]></term>
##       <term><![CDATA[plasma CVD]]></term>
##       <term><![CDATA[semiconductor quantum dots]]></term>
##       <term><![CDATA[silicon]]></term>
##       <term><![CDATA[silicon compounds]]></term>
##       <term><![CDATA[wide band gap semiconductors]]></term>
##     </controlledterms>
##     <thesaurusterms>
##       <term><![CDATA[Delays]]></term>
##       <term><![CDATA[Integrated circuit modeling]]></term>
##       <term><![CDATA[Logic gates]]></term>
##       <term><![CDATA[Noise]]></term>
##       <term><![CDATA[Power supplies]]></term>
##       <term><![CDATA[SPICE]]></term>
##       <term><![CDATA[Switches]]></term>
##     </thesaurusterms>
##     <pubtitle><![CDATA[Photonics Journal, IEEE]]></pubtitle>
##     <punumber><![CDATA[4563994]]></punumber>
##     <pubtype><![CDATA[Journals & Magazines]]></pubtype>
##     <publisher><![CDATA[IEEE]]></publisher>
##     <volume><![CDATA[6]]></volume>
##     <issue><![CDATA[1]]></issue>
##     <py><![CDATA[2014]]></py>
##     <spage><![CDATA[1]]></spage>
##     <epage><![CDATA[7]]></epage>
##     <abstract><![CDATA[We deposited a p-i-n structure device with alternative amorphous Si (a-Si) and amorphous SiC (a-SiC) multilayers as an intrinsic layer in a plasma-enhanced chemical vapor deposition (PECVD) system. A KrF pulsed excimer laser-induced crystallization of a-Si/a-SiC stacked structures was used to prepare Si quantum dots (Si QDs)/SiC multilayers. The formation of Si QDs with an average size of 4 nm was confirmed by Raman spectra, whereas the layered structures were revealed by cross-sectional transmission electron microscopy. Electroluminescence (EL) devices containing Si QDs/SiC multilayers embedded in a p-n junction were fabricated, and the device performance was studied and compared with the reference device without the p-i-n structure. It was found that the turn-on voltage was reduced and that luminescence efficiency was significantly enhanced by using the p-i-n device structure. The recombination mechanism of carriers in a Si-QD-based EL device was also discussed, and the improved device performance can be attributed to the enhanced radiative recombination probability in a p-i-n EL device.]]></abstract>
##     <issn><![CDATA[1943-0655]]></issn>
##     <htmlFlag><![CDATA[1]]></htmlFlag>
##     <arnumber><![CDATA[6690105]]></arnumber>
##     <doi><![CDATA[10.1109/JPHOT.2013.2295467]]></doi>
##     <publicationId><![CDATA[6690105]]></publicationId>
##     <mdurl><![CDATA[http://ieeexplore.ieee.org/xpl/articleDetails.jsp?tp=&arnumber=6690105&contentType=Journals+%26+Magazines]]></mdurl>
##     <pdf><![CDATA[http://ieeexplore.ieee.org/stamp/stamp.jsp?arnumber=6690105]]></pdf>
##   </document>
##   <document>
##     <rank>659</rank>
##     <title><![CDATA[Visualizing Statistical Mix Effects and Simpson&#x0027;s Paradox]]></title>
##     <authors><![CDATA[Armstrong, Z.;  Wattenberg, M.]]></authors>
##     <controlledterms>
##       <term><![CDATA[data visualisation]]></term>
##       <term><![CDATA[statistical analysis]]></term>
##     </controlledterms>
##     <thesaurusterms>
##       <term><![CDATA[Data visualization]]></term>
##       <term><![CDATA[Image color analysis]]></term>
##       <term><![CDATA[Image segmentation]]></term>
##       <term><![CDATA[Statistics]]></term>
##     </thesaurusterms>
##     <pubtitle><![CDATA[Visualization and Computer Graphics, IEEE Transactions on]]></pubtitle>
##     <punumber><![CDATA[2945]]></punumber>
##     <pubtype><![CDATA[Journals & Magazines]]></pubtype>
##     <publisher><![CDATA[IEEE]]></publisher>
##     <volume><![CDATA[20]]></volume>
##     <issue><![CDATA[12]]></issue>
##     <py><![CDATA[2014]]></py>
##     <spage><![CDATA[2132]]></spage>
##     <epage><![CDATA[2141]]></epage>
##     <abstract><![CDATA[We discuss how &#x201C;mix effects&#x201D; can surprise users of visualizations and potentially lead them to incorrect conclusions. This statistical issue (also known as &#x201C;omitted variable bias&#x201D; or, in extreme cases, as &#x201C;Simpson's paradox&#x201D;) is widespread and can affect any visualization in which the quantity of interest is an aggregated value such as a weighted sum or average. Our first contribution is to document how mix effects can be a serious issue for visualizations, and we analyze how mix effects can cause problems in a variety of popular visualization techniques, from bar charts to treemaps. Our second contribution is a new technique, the &#x201C;comet chart,&#x201D; that is meant to ameliorate some of these issues.]]></abstract>
##     <issn><![CDATA[1077-2626]]></issn>
##     <htmlFlag><![CDATA[1]]></htmlFlag>
##     <arnumber><![CDATA[6875927]]></arnumber>
##     <doi><![CDATA[10.1109/TVCG.2014.2346297]]></doi>
##     <publicationId><![CDATA[6875927]]></publicationId>
##     <mdurl><![CDATA[http://ieeexplore.ieee.org/xpl/articleDetails.jsp?tp=&arnumber=6875927&contentType=Journals+%26+Magazines]]></mdurl>
##     <pdf><![CDATA[http://ieeexplore.ieee.org/stamp/stamp.jsp?arnumber=6875927]]></pdf>
##   </document>
##   <document>
##     <rank>660</rank>
##     <title><![CDATA[A Novel Crowding Genetic Algorithm and Its Applications to Manufacturing Robots]]></title>
##     <authors><![CDATA[Chiu-Hung Chen;  Tung-Kuan Liu;  Jyh-Horng Chou]]></authors>
##     <affiliations><![CDATA[Dept. of Inf. Technol., Kao Yuan Univ., Kaohsiung, Taiwan]]></affiliations>
##     <controlledterms>
##       <term><![CDATA[genetic algorithms]]></term>
##       <term><![CDATA[industrial manipulators]]></term>
##       <term><![CDATA[path planning]]></term>
##     </controlledterms>
##     <thesaurusterms>
##       <term><![CDATA[Benchmark testing]]></term>
##       <term><![CDATA[Genetic algorithms]]></term>
##       <term><![CDATA[Genetics]]></term>
##       <term><![CDATA[Optimization]]></term>
##       <term><![CDATA[Robots]]></term>
##       <term><![CDATA[Sociology]]></term>
##       <term><![CDATA[Statistics]]></term>
##     </thesaurusterms>
##     <pubtitle><![CDATA[Industrial Informatics, IEEE Transactions on]]></pubtitle>
##     <punumber><![CDATA[9424]]></punumber>
##     <pubtype><![CDATA[Journals & Magazines]]></pubtype>
##     <publisher><![CDATA[IEEE]]></publisher>
##     <volume><![CDATA[10]]></volume>
##     <issue><![CDATA[3]]></issue>
##     <py><![CDATA[2014]]></py>
##     <spage><![CDATA[1705]]></spage>
##     <epage><![CDATA[1716]]></epage>
##     <abstract><![CDATA[A niche genetic algorithm (GA) based on a novel twin-space crowding (TC) approach is proposed for solving multimodal manufacturing optimization problems. The proposed TC method is designed in a parameter-free paradigm. That is, when cooperatively exploring solutions with GAs, it does not require prior knowledge related to the solution space to design additional problem-dependent parameters in the evolutionary process. This feature makes the proposed TC method suitable for assisting GAs in solving real-world engineering optimization problems involving intractable solution landscapes. A set of numerical benchmark functions is used to compare effectiveness and efficiency in the proposed TCGA, in different niche GAs, and in several evolutionary computation methods. The TCGA is then used to solve multimodal joint-space inverse problems in serial-link robots to compare its convergence performance with that of conventional methods that apply the sharing function. Finally, the TCGA is used to solve iterative collision-free design problems for linkage-bar robotic hands to demonstrate its effectiveness for generating diverse solutions during the design process.]]></abstract>
##     <issn><![CDATA[1551-3203]]></issn>
##     <htmlFlag><![CDATA[1]]></htmlFlag>
##     <arnumber><![CDATA[6786983]]></arnumber>
##     <doi><![CDATA[10.1109/TII.2014.2316638]]></doi>
##     <publicationId><![CDATA[6786983]]></publicationId>
##     <mdurl><![CDATA[http://ieeexplore.ieee.org/xpl/articleDetails.jsp?tp=&arnumber=6786983&contentType=Journals+%26+Magazines]]></mdurl>
##     <pdf><![CDATA[http://ieeexplore.ieee.org/stamp/stamp.jsp?arnumber=6786983]]></pdf>
##   </document>
##   <document>
##     <rank>661</rank>
##     <title><![CDATA[Adaptive Multi-Channel Transmission Power Control for Industrial Wireless Instrumentation]]></title>
##     <authors><![CDATA[Ikram, W.;  Petersen, S.;  Orten, P.;  Thornhill, N.F.]]></authors>
##     <affiliations><![CDATA[Imperial Coll. London, London, UK]]></affiliations>
##     <controlledterms>
##       <term><![CDATA[energy consumption]]></term>
##       <term><![CDATA[fading channels]]></term>
##       <term><![CDATA[power control]]></term>
##       <term><![CDATA[radio networks]]></term>
##       <term><![CDATA[radiofrequency interference]]></term>
##       <term><![CDATA[telecommunication power management]]></term>
##     </controlledterms>
##     <thesaurusterms>
##       <term><![CDATA[IEEE 802.11 Standards]]></term>
##       <term><![CDATA[Interference]]></term>
##       <term><![CDATA[Receivers]]></term>
##       <term><![CDATA[Signal to noise ratio]]></term>
##       <term><![CDATA[Wireless communication]]></term>
##       <term><![CDATA[Wireless sensor networks]]></term>
##     </thesaurusterms>
##     <pubtitle><![CDATA[Industrial Informatics, IEEE Transactions on]]></pubtitle>
##     <punumber><![CDATA[9424]]></punumber>
##     <pubtype><![CDATA[Journals & Magazines]]></pubtype>
##     <publisher><![CDATA[IEEE]]></publisher>
##     <volume><![CDATA[10]]></volume>
##     <issue><![CDATA[2]]></issue>
##     <py><![CDATA[2014]]></py>
##     <spage><![CDATA[978]]></spage>
##     <epage><![CDATA[990]]></epage>
##     <abstract><![CDATA[The adoption of wireless technology for industrial wireless instrumentation requires high-quality communication performance. The use of transmission power control (TPC) can help address industrial issues concerning energy consumption, interference, and fading. This paper presents a TPC algorithm designed for industrial applications based on theoretical and empirical studies. It is shown that the proposed algorithm adapts to variations in link quality, and is hardware-independent and practical.]]></abstract>
##     <issn><![CDATA[1551-3203]]></issn>
##     <htmlFlag><![CDATA[1]]></htmlFlag>
##     <arnumber><![CDATA[6763067]]></arnumber>
##     <doi><![CDATA[10.1109/TII.2014.2310594]]></doi>
##     <publicationId><![CDATA[6763067]]></publicationId>
##     <mdurl><![CDATA[http://ieeexplore.ieee.org/xpl/articleDetails.jsp?tp=&arnumber=6763067&contentType=Journals+%26+Magazines]]></mdurl>
##     <pdf><![CDATA[http://ieeexplore.ieee.org/stamp/stamp.jsp?arnumber=6763067]]></pdf>
##   </document>
##   <document>
##     <rank>662</rank>
##     <title><![CDATA[Smooth Max-Information as One-Shot Generalization for Mutual Information]]></title>
##     <authors><![CDATA[Ciganovic, N.;  Beaudry, N.J.;  Renner, R.]]></authors>
##     <affiliations><![CDATA[Dept. of Bioeng., Imperial Coll. London, London, UK]]></affiliations>
##     <controlledterms>
##       <term><![CDATA[channel coding]]></term>
##       <term><![CDATA[information theory]]></term>
##       <term><![CDATA[smoothing methods]]></term>
##     </controlledterms>
##     <thesaurusterms>
##       <term><![CDATA[Entropy]]></term>
##       <term><![CDATA[Hilbert space]]></term>
##       <term><![CDATA[Mutual information]]></term>
##       <term><![CDATA[Quantum mechanics]]></term>
##       <term><![CDATA[Smoothing methods]]></term>
##     </thesaurusterms>
##     <pubtitle><![CDATA[Information Theory, IEEE Transactions on]]></pubtitle>
##     <punumber><![CDATA[18]]></punumber>
##     <pubtype><![CDATA[Journals & Magazines]]></pubtype>
##     <publisher><![CDATA[IEEE]]></publisher>
##     <volume><![CDATA[60]]></volume>
##     <issue><![CDATA[3]]></issue>
##     <py><![CDATA[2014]]></py>
##     <spage><![CDATA[1573]]></spage>
##     <epage><![CDATA[1581]]></epage>
##     <abstract><![CDATA[We study formal properties of smooth max-information, a generalization of von Neumann mutual information derived from the max-relative entropy. Recent work suggests that it is a useful quantity in one-shot channel coding, quantum rate distortion theory, and the physics of quantum many-body systems. Max-information can be defined in multiple ways. We demonstrate that different smoothed definitions are essentially equivalent (up to logarithmic terms in the smoothing parameters). These equivalence relations allow us to derive new chain rules for the max-information in terms of min- and max-entropies, thus extending the smooth entropy formalism to mutual information.]]></abstract>
##     <issn><![CDATA[0018-9448]]></issn>
##     <htmlFlag><![CDATA[1]]></htmlFlag>
##     <arnumber><![CDATA[6687244]]></arnumber>
##     <doi><![CDATA[10.1109/TIT.2013.2295314]]></doi>
##     <publicationId><![CDATA[6687244]]></publicationId>
##     <mdurl><![CDATA[http://ieeexplore.ieee.org/xpl/articleDetails.jsp?tp=&arnumber=6687244&contentType=Journals+%26+Magazines]]></mdurl>
##     <pdf><![CDATA[http://ieeexplore.ieee.org/stamp/stamp.jsp?arnumber=6687244]]></pdf>
##   </document>
##   <document>
##     <rank>663</rank>
##     <title><![CDATA[Initial Skill Acquisition of Handrim Wheelchair Propulsion: A New Perspective]]></title>
##     <authors><![CDATA[Vegter, R.J.K.;  de Groot, S.;  Lamoth, C.J.;  Veeger, D.H.;  van der Woude, L.H.V.]]></authors>
##     <affiliations><![CDATA[Med. Center, Center for Human Movement Sci., Univ. of Groningen, Groningen, Netherlands]]></affiliations>
##     <controlledterms>
##       <term><![CDATA[force control]]></term>
##       <term><![CDATA[gait analysis]]></term>
##       <term><![CDATA[handicapped aids]]></term>
##       <term><![CDATA[patient rehabilitation]]></term>
##       <term><![CDATA[power consumption]]></term>
##       <term><![CDATA[wheelchairs]]></term>
##     </controlledterms>
##     <thesaurusterms>
##       <term><![CDATA[Atmospheric measurements]]></term>
##       <term><![CDATA[Educational institutions]]></term>
##       <term><![CDATA[Particle measurements]]></term>
##       <term><![CDATA[Propulsion]]></term>
##       <term><![CDATA[Torque]]></term>
##       <term><![CDATA[Wheelchairs]]></term>
##       <term><![CDATA[Wheels]]></term>
##     </thesaurusterms>
##     <pubtitle><![CDATA[Neural Systems and Rehabilitation Engineering, IEEE Transactions on]]></pubtitle>
##     <punumber><![CDATA[7333]]></punumber>
##     <pubtype><![CDATA[Journals & Magazines]]></pubtype>
##     <publisher><![CDATA[IEEE]]></publisher>
##     <volume><![CDATA[22]]></volume>
##     <issue><![CDATA[1]]></issue>
##     <py><![CDATA[2014]]></py>
##     <spage><![CDATA[104]]></spage>
##     <epage><![CDATA[113]]></epage>
##     <abstract><![CDATA[To gain insight into cyclic motor learning processes, hand rim wheelchair propulsion is a suitable cyclic task, to be learned during early rehabilitation and novel to almost every individual. To propel in an energy efficient manner, wheelchair users must learn to control bimanually applied forces onto the rims, preserving both speed and direction of locomotion. The purpose of this study was to evaluate mechanical efficiency and propulsion technique during the initial stage of motor learning. Therefore, 70 naive able-bodied men received 12-min uninstructed wheelchair practice, consisting of three 4-min blocks separated by 2 min rest. Practice was performed on a motor-driven treadmill at a fixed belt speed and constant power output relative to body mass. Energy consumption and the kinetics of propulsion technique were continuously measured. Participants significantly increased their mechanical efficiency and changed their propulsion technique from a high frequency mode with a lot of negative work to a longer-slower movement pattern with less power losses. Furthermore a multi-level model showed propulsion technique to relate to mechanical efficiency. Finally improvers and non-improvers were identified. The non-improving group was already more efficient and had a better propulsion technique in the first block of practice (i.e., the fourth minute). These findings link propulsion technique to mechanical efficiency, support the importance of a correct propulsion technique for wheelchair users and show motor learning differences.]]></abstract>
##     <issn><![CDATA[1534-4320]]></issn>
##     <htmlFlag><![CDATA[1]]></htmlFlag>
##     <arnumber><![CDATA[6626622]]></arnumber>
##     <doi><![CDATA[10.1109/TNSRE.2013.2280301]]></doi>
##     <publicationId><![CDATA[6626622]]></publicationId>
##     <mdurl><![CDATA[http://ieeexplore.ieee.org/xpl/articleDetails.jsp?tp=&arnumber=6626622&contentType=Journals+%26+Magazines]]></mdurl>
##     <pdf><![CDATA[http://ieeexplore.ieee.org/stamp/stamp.jsp?arnumber=6626622]]></pdf>
##   </document>
##   <document>
##     <rank>664</rank>
##     <title><![CDATA[Progress in Hyperspectral Remote Sensing Science and Technology in China Over the Past Three Decades]]></title>
##     <authors><![CDATA[Qingxi Tong;  Yongqi Xue;  Lifu Zhang]]></authors>
##     <affiliations><![CDATA[Inst. of Remote Sensing & Digital Earth, Beijing, China]]></affiliations>
##     <controlledterms>
##       <term><![CDATA[geophysical image processing]]></term>
##       <term><![CDATA[geophysical prospecting]]></term>
##       <term><![CDATA[hyperspectral imaging]]></term>
##       <term><![CDATA[minerals]]></term>
##       <term><![CDATA[remote sensing]]></term>
##     </controlledterms>
##     <thesaurusterms>
##       <term><![CDATA[Earth]]></term>
##       <term><![CDATA[Hyperspectral imaging]]></term>
##       <term><![CDATA[Image sensors]]></term>
##       <term><![CDATA[Sensors]]></term>
##     </thesaurusterms>
##     <pubtitle><![CDATA[Selected Topics in Applied Earth Observations and Remote Sensing, IEEE Journal of]]></pubtitle>
##     <punumber><![CDATA[4609443]]></punumber>
##     <pubtype><![CDATA[Journals & Magazines]]></pubtype>
##     <publisher><![CDATA[IEEE]]></publisher>
##     <volume><![CDATA[7]]></volume>
##     <issue><![CDATA[1]]></issue>
##     <py><![CDATA[2014]]></py>
##     <spage><![CDATA[70]]></spage>
##     <epage><![CDATA[91]]></epage>
##     <abstract><![CDATA[This paper reviews progress in hyperspectral remote sensing (HRS) in China, focusing on the past three decades. China has made great achievements since starting in this promising field in the early 1980s. A series of advanced hyperspectral imaging systems ranging from ground to airborne and satellite platforms have been designed, built, and operated. These include the field imaging spectrometer system (FISS), the Modular Airborne Imaging Spectrometer (MAIS), and the Chang'E-I Interferometer Spectrometer (IIM). In addition to developing sensors, Chinese scientists have proposed various novel image processing techniques. Applications of hyperspectral imaging in China have been also performed including mineral exploration in the Qilian Mountains and oil exploration in Xinjiang province. To promote the development of HRS, many generic and professional software tools have been developed. These tools such as the Hyperspectral Image Processing and Analysis System (HIPAS) incorporate a number of special algorithms and features designed to take advantage of the wealth of information contained in HRS data, allowing them to meet the demands of both common users and researchers in the scientific community.]]></abstract>
##     <issn><![CDATA[1939-1404]]></issn>
##     <htmlFlag><![CDATA[1]]></htmlFlag>
##     <arnumber><![CDATA[6565416]]></arnumber>
##     <doi><![CDATA[10.1109/JSTARS.2013.2267204]]></doi>
##     <publicationId><![CDATA[6565416]]></publicationId>
##     <mdurl><![CDATA[http://ieeexplore.ieee.org/xpl/articleDetails.jsp?tp=&arnumber=6565416&contentType=Journals+%26+Magazines]]></mdurl>
##     <pdf><![CDATA[http://ieeexplore.ieee.org/stamp/stamp.jsp?arnumber=6565416]]></pdf>
##   </document>
##   <document>
##     <rank>665</rank>
##     <title><![CDATA[Selectively Liquid-Infiltrated Microstructured Optical Fiber for Simultaneous Temperature and Force Measurement]]></title>
##     <authors><![CDATA[Chunxue Yang;  Hao Zhang;  Hu Liang;  Yinping Miao;  Bo Liu;  Zhi Wang;  Yange Liu]]></authors>
##     <affiliations><![CDATA[Key Lab. of Opt. Inf. Sci. & Technol., Nankai Univ., Tianjin, China]]></affiliations>
##     <controlledterms>
##       <term><![CDATA[fibre optic sensors]]></term>
##       <term><![CDATA[force measurement]]></term>
##       <term><![CDATA[micro-optics]]></term>
##       <term><![CDATA[microsensors]]></term>
##       <term><![CDATA[optical fibre cladding]]></term>
##       <term><![CDATA[optical phase matching]]></term>
##       <term><![CDATA[temperature measurement]]></term>
##     </controlledterms>
##     <thesaurusterms>
##       <term><![CDATA[Liquid waveguides]]></term>
##       <term><![CDATA[Liquids]]></term>
##       <term><![CDATA[Optical fiber sensors]]></term>
##       <term><![CDATA[Optical fibers]]></term>
##       <term><![CDATA[Sensitivity]]></term>
##       <term><![CDATA[Temperature sensors]]></term>
##     </thesaurusterms>
##     <pubtitle><![CDATA[Photonics Journal, IEEE]]></pubtitle>
##     <punumber><![CDATA[4563994]]></punumber>
##     <pubtype><![CDATA[Journals & Magazines]]></pubtype>
##     <publisher><![CDATA[IEEE]]></publisher>
##     <volume><![CDATA[6]]></volume>
##     <issue><![CDATA[2]]></issue>
##     <py><![CDATA[2014]]></py>
##     <spage><![CDATA[1]]></spage>
##     <epage><![CDATA[8]]></epage>
##     <abstract><![CDATA[A dual-parameter sensor based on a liquid-infiltrated microstructured optical fiber (MOF) is constructed by selective liquid infiltration of one particular cladding air hole in the MOF. The fundamental core mode could be simultaneously coupled to several higher order modes of the adjacent liquid rod waveguide when the phase-matching condition is satisfied, resulting in the emergence of multiple resonance peaks with different sensing characteristics. Our proposed liquid-infiltrated MOF sensor has several advantages, such as high sensitivity and a large measurement range. Moreover, it is able to resolve the temperature-force cross-sensitivity issue that is commonly encountered in most practically deployed sensing components.]]></abstract>
##     <issn><![CDATA[1943-0655]]></issn>
##     <htmlFlag><![CDATA[1]]></htmlFlag>
##     <arnumber><![CDATA[6744591]]></arnumber>
##     <doi><![CDATA[10.1109/JPHOT.2014.2306813]]></doi>
##     <publicationId><![CDATA[6744591]]></publicationId>
##     <mdurl><![CDATA[http://ieeexplore.ieee.org/xpl/articleDetails.jsp?tp=&arnumber=6744591&contentType=Journals+%26+Magazines]]></mdurl>
##     <pdf><![CDATA[http://ieeexplore.ieee.org/stamp/stamp.jsp?arnumber=6744591]]></pdf>
##   </document>
##   <document>
##     <rank>666</rank>
##     <title><![CDATA[Communicating Power Supplies: Bringing the Internet to the Ubiquitous Energy Gateways of Electronic Devices]]></title>
##     <authors><![CDATA[Lanzisera, S.;  Weber, A.R.;  Liao, A.;  Pajak, D.;  Meier, A.K.]]></authors>
##     <affiliations><![CDATA[Environ. Energy Technol. Div., Lawrence Berkeley Nat. Lab., Berkeley, CA, USA]]></affiliations>
##     <controlledterms>
##       <term><![CDATA[Internet]]></term>
##       <term><![CDATA[building management systems]]></term>
##       <term><![CDATA[energy conservation]]></term>
##       <term><![CDATA[power engineering computing]]></term>
##     </controlledterms>
##     <thesaurusterms>
##       <term><![CDATA[Electricity supply industry]]></term>
##       <term><![CDATA[Energy efficiency]]></term>
##       <term><![CDATA[Energy management]]></term>
##       <term><![CDATA[Internet of Things]]></term>
##       <term><![CDATA[Monitoring]]></term>
##       <term><![CDATA[Power supplies]]></term>
##       <term><![CDATA[Standards]]></term>
##       <term><![CDATA[Switches]]></term>
##       <term><![CDATA[Ubiquitous computing]]></term>
##       <term><![CDATA[Voltage measurement]]></term>
##     </thesaurusterms>
##     <pubtitle><![CDATA[Internet of Things Journal, IEEE]]></pubtitle>
##     <punumber><![CDATA[6488907]]></punumber>
##     <pubtype><![CDATA[Journals & Magazines]]></pubtype>
##     <publisher><![CDATA[IEEE]]></publisher>
##     <volume><![CDATA[1]]></volume>
##     <issue><![CDATA[2]]></issue>
##     <py><![CDATA[2014]]></py>
##     <spage><![CDATA[153]]></spage>
##     <epage><![CDATA[160]]></epage>
##     <abstract><![CDATA[Saving energy in buildings is often hampered by the lack of detailed information about what is using the energy, how much it is using, and how to automatically and remotely control devices. The problem is especially acute for the large number of small, energy-using devices that are present in both residential and commercial buildings. Most of these products use a switching ac to dc power supply to operate electronic and other internal components. We describe a &#x201C;communicating power supply&#x201D; (CPS) to enable the communication of energy and control information between the device and a building management system or other central entities. We developed a proof-of-concept system of Internet-connected CPSs and demonstrated both energy reporting and control utilizing a custom, cloud-based information clearing house. If CPS technology became widespread in devices, a combination of automated and human interactive solutions would enable high levels of energy savings.]]></abstract>
##     <issn><![CDATA[2327-4662]]></issn>
##     <htmlFlag><![CDATA[1]]></htmlFlag>
##     <arnumber><![CDATA[6778069]]></arnumber>
##     <doi><![CDATA[10.1109/JIOT.2014.2307077]]></doi>
##     <publicationId><![CDATA[6778069]]></publicationId>
##     <mdurl><![CDATA[http://ieeexplore.ieee.org/xpl/articleDetails.jsp?tp=&arnumber=6778069&contentType=Journals+%26+Magazines]]></mdurl>
##     <pdf><![CDATA[http://ieeexplore.ieee.org/stamp/stamp.jsp?arnumber=6778069]]></pdf>
##   </document>
##   <document>
##     <rank>667</rank>
##     <title><![CDATA[Quantification of Thermal Lensing Using an Artificial Eye]]></title>
##     <authors><![CDATA[Towle, E.L.;  Rickman, M.;  Dunn, A.K.;  Welch, A.J.;  Thomas, R.J.]]></authors>
##     <affiliations><![CDATA[Air Force Res. Lab., Fort Sam Houston, TX, USA]]></affiliations>
##     <controlledterms>
##       <term><![CDATA[artificial organs]]></term>
##       <term><![CDATA[bio-optics]]></term>
##       <term><![CDATA[eye]]></term>
##       <term><![CDATA[infrared imaging]]></term>
##       <term><![CDATA[thermal lensing]]></term>
##       <term><![CDATA[vision]]></term>
##     </controlledterms>
##     <thesaurusterms>
##       <term><![CDATA[Biological tissues]]></term>
##       <term><![CDATA[Infrared detectors]]></term>
##       <term><![CDATA[Lasers]]></term>
##       <term><![CDATA[Nonlinear optics]]></term>
##       <term><![CDATA[Thermal analysis]]></term>
##     </thesaurusterms>
##     <pubtitle><![CDATA[Selected Topics in Quantum Electronics, IEEE Journal of]]></pubtitle>
##     <punumber><![CDATA[2944]]></punumber>
##     <pubtype><![CDATA[Journals & Magazines]]></pubtype>
##     <publisher><![CDATA[IEEE]]></publisher>
##     <volume><![CDATA[20]]></volume>
##     <issue><![CDATA[2]]></issue>
##     <py><![CDATA[2014]]></py>
##     <spage><![CDATA[158]]></spage>
##     <epage><![CDATA[165]]></epage>
##     <abstract><![CDATA[Recent experiments have concluded that it is possible to interrupt the vision of human subjects using infrared (IR) light through an effect known as thermal lensing. While these experiments successfully demonstrated the influence of thermal lensing on an Amsler grid target, little has been done to quantify the amount of visual disruption resulting from this phenomenon. Therefore, an artificial eye system was configured to better quantify the refractive power of the thermal lens generated within the human eye. The influence of 1319 nm energy with power levels from 220 to 630 mW and exposure durations between 0.25 and 1.00 s was evaluated based on changes induced within a visible probe beam (542 nm). Results showed up to a -2.0 D blur could be induced in human subjects using these energy levels. Results also established a relationship between the peak IR power and exposure durations used to determine the strength of the thermal lens.]]></abstract>
##     <issn><![CDATA[1077-260X]]></issn>
##     <htmlFlag><![CDATA[1]]></htmlFlag>
##     <arnumber><![CDATA[6516575]]></arnumber>
##     <doi><![CDATA[10.1109/JSTQE.2013.2260728]]></doi>
##     <publicationId><![CDATA[6516575]]></publicationId>
##     <mdurl><![CDATA[http://ieeexplore.ieee.org/xpl/articleDetails.jsp?tp=&arnumber=6516575&contentType=Journals+%26+Magazines]]></mdurl>
##     <pdf><![CDATA[http://ieeexplore.ieee.org/stamp/stamp.jsp?arnumber=6516575]]></pdf>
##   </document>
##   <document>
##     <rank>668</rank>
##     <title><![CDATA[Adaptive Sliding Mode Observer for Engine Cylinder Pressure Imbalance Under Different Parameter Uncertainties]]></title>
##     <authors><![CDATA[Al-Durra, A.]]></authors>
##     <affiliations><![CDATA[Dept. of Electr. Eng., Pet. Inst., Abu Dhabi, United Arab Emirates]]></affiliations>
##     <controlledterms>
##       <term><![CDATA[adaptive control]]></term>
##       <term><![CDATA[closed loop systems]]></term>
##       <term><![CDATA[combustion]]></term>
##       <term><![CDATA[control system synthesis]]></term>
##       <term><![CDATA[diesel engines]]></term>
##       <term><![CDATA[observers]]></term>
##       <term><![CDATA[pressure transducers]]></term>
##       <term><![CDATA[sensors]]></term>
##       <term><![CDATA[shafts]]></term>
##       <term><![CDATA[torque]]></term>
##       <term><![CDATA[variable structure systems]]></term>
##       <term><![CDATA[velocity measurement]]></term>
##     </controlledterms>
##     <thesaurusterms>
##       <term><![CDATA[Combustion]]></term>
##       <term><![CDATA[Compensation]]></term>
##       <term><![CDATA[Engine cylinders]]></term>
##       <term><![CDATA[Engines]]></term>
##       <term><![CDATA[Pressure control]]></term>
##       <term><![CDATA[Sliding mode control]]></term>
##       <term><![CDATA[Temperature control]]></term>
##       <term><![CDATA[Torque control]]></term>
##     </thesaurusterms>
##     <pubtitle><![CDATA[Access, IEEE]]></pubtitle>
##     <punumber><![CDATA[6287639]]></punumber>
##     <pubtype><![CDATA[Journals & Magazines]]></pubtype>
##     <publisher><![CDATA[IEEE]]></publisher>
##     <volume><![CDATA[2]]></volume>
##     <py><![CDATA[2014]]></py>
##     <spage><![CDATA[1085]]></spage>
##     <epage><![CDATA[1091]]></epage>
##     <abstract><![CDATA[One of the principal issues of alternative combustion modes for diesel engines (such as HCCI, PCCI, and LTC) is caused by the imbalances in the distribution of air and EGR across the cylinders, which affects the combustion process and ultimately cause significant differences in the pressure trace and indicated torque for each cylinder. In principle, a cylinder-by-cylinder control approach could compensate for air, residuals, and temperature imbalance. However, in order to fully benefit from closed-loop combustion control, it is necessary to obtain feedback signals from each engine cylinder to reconstruct the pressure trace. Therefore, cylinder imbalance is an issue that can be detected in a laboratory environment, wherein each engine cylinder is instrumented with a dedicated pressure transducer. This paper describes the framework and preliminary results of a model-based estimation approach to predict the individual pressure traces in a multicylinder engine relying on a very restricted sensor set, namely, a crankshaft speed sensor, a single production-grade pressure sensor. The objective of the estimator is to reconstruct the complete pressure trace during an engine cycle with sufficient accuracy to allow for detection of cylinder to cylinder imbalances. Starting from a model of the engine crankshaft dynamics, an adaptive sliding mode observer is designed to estimate the cylinder pressure from the crankshaft speed fluctuation measurement. The results obtained by the estimator are compared with experimental data obtained on a four-cylinder diesel engine.]]></abstract>
##     <issn><![CDATA[2169-3536]]></issn>
##     <htmlFlag><![CDATA[1]]></htmlFlag>
##     <arnumber><![CDATA[6901187]]></arnumber>
##     <doi><![CDATA[10.1109/ACCESS.2014.2358954]]></doi>
##     <publicationId><![CDATA[6901187]]></publicationId>
##     <mdurl><![CDATA[http://ieeexplore.ieee.org/xpl/articleDetails.jsp?tp=&arnumber=6901187&contentType=Journals+%26+Magazines]]></mdurl>
##     <pdf><![CDATA[http://ieeexplore.ieee.org/stamp/stamp.jsp?arnumber=6901187]]></pdf>
##   </document>
##   <document>
##     <rank>669</rank>
##     <title><![CDATA[Wavelength-Tunable and Bandwidth-Variable Ultra-Flat Optical Frequency Comb Block Generation From a Bismuth-Based Actively Mode-Locked Fiber Laser]]></title>
##     <authors><![CDATA[Fukuchi, Y.;  Hirata, K.;  Ikeoka, H.]]></authors>
##     <affiliations><![CDATA[Dept. of Electr. Eng., Tokyo Univ. of Sci., Tokyo, Japan]]></affiliations>
##     <controlledterms>
##       <term><![CDATA[bismuth]]></term>
##       <term><![CDATA[fibre lasers]]></term>
##       <term><![CDATA[laser mode locking]]></term>
##       <term><![CDATA[laser tuning]]></term>
##     </controlledterms>
##     <thesaurusterms>
##       <term><![CDATA[Erbium-doped fiber lasers]]></term>
##       <term><![CDATA[Laser mode locking]]></term>
##       <term><![CDATA[Optical fiber amplifiers]]></term>
##       <term><![CDATA[Optical fiber dispersion]]></term>
##       <term><![CDATA[Optical filters]]></term>
##     </thesaurusterms>
##     <pubtitle><![CDATA[Photonics Journal, IEEE]]></pubtitle>
##     <punumber><![CDATA[4563994]]></punumber>
##     <pubtype><![CDATA[Journals & Magazines]]></pubtype>
##     <publisher><![CDATA[IEEE]]></publisher>
##     <volume><![CDATA[6]]></volume>
##     <issue><![CDATA[1]]></issue>
##     <py><![CDATA[2014]]></py>
##     <spage><![CDATA[1]]></spage>
##     <epage><![CDATA[9]]></epage>
##     <abstract><![CDATA[We generate, for the first time and to the best of our knowledge, a tunable ultra-flat optical frequency comb (OFC) block from a 10-GHz bismuth-based actively mode-locked fiber laser. A bismuth-based erbium-doped fiber (Bi-EDF) with a length of 1.5 m and a bismuth-based highly nonlinear fiber (Bi-HNLF) with a length of 4.1 m are used as a gain medium and a spectral broadening device, respectively. We also employ a wavelength-tunable and bandwidth-variable optical filter for an intracavity filter. The center wavelength of the OFC output can be widely tuned from 1535 to 1585 nm because the Bi-EDF has a broadband gain profile. A spectrally broadened ultra-flat OFC is successfully generated by the self-phase modulation in the Bi-HNLF. By tuning the filter bandwidth, the 10-dB spectral width and the pulsewidth of the generated OFC can be widely varied from 0.23 to 2.4 nm and from 3 to 20.1 ps, respectively. The fiber ring cavity is as short as 17.8 m. For the entire tuning ranges, the proposed bismuth-based tunable ultra-flat OFC generator also realizes stable bit-error-free mode-locking operation.]]></abstract>
##     <issn><![CDATA[1943-0655]]></issn>
##     <htmlFlag><![CDATA[1]]></htmlFlag>
##     <arnumber><![CDATA[6698374]]></arnumber>
##     <doi><![CDATA[10.1109/JPHOT.2013.2295469]]></doi>
##     <publicationId><![CDATA[6698374]]></publicationId>
##     <mdurl><![CDATA[http://ieeexplore.ieee.org/xpl/articleDetails.jsp?tp=&arnumber=6698374&contentType=Journals+%26+Magazines]]></mdurl>
##     <pdf><![CDATA[http://ieeexplore.ieee.org/stamp/stamp.jsp?arnumber=6698374]]></pdf>
##   </document>
##   <document>
##     <rank>670</rank>
##     <title><![CDATA[Periodic Training Sequence Aided In-Band OSNR Monitoring in Digital Coherent Receiver]]></title>
##     <authors><![CDATA[Donghe Zhao;  Lixia Xi;  Xianfeng Tang;  Wenbo Zhang;  Yaojun Qiao;  Xiaoguang Zhang]]></authors>
##     <affiliations><![CDATA[State Key Lab. of Inf. Photonics & Opt. Commun., Beijing Univ. of Posts & Telecommun., Beijing, China]]></affiliations>
##     <controlledterms>
##       <term><![CDATA[light coherence]]></term>
##       <term><![CDATA[multiplexing]]></term>
##       <term><![CDATA[optical fibre communication]]></term>
##       <term><![CDATA[optical fibre dispersion]]></term>
##       <term><![CDATA[optical fibre polarisation]]></term>
##       <term><![CDATA[optical receivers]]></term>
##       <term><![CDATA[quadrature amplitude modulation]]></term>
##       <term><![CDATA[quadrature phase shift keying]]></term>
##     </controlledterms>
##     <thesaurusterms>
##       <term><![CDATA[Correlation]]></term>
##       <term><![CDATA[Monitoring]]></term>
##       <term><![CDATA[Optical fibers]]></term>
##       <term><![CDATA[Optical noise]]></term>
##       <term><![CDATA[Optical polarization]]></term>
##       <term><![CDATA[Signal to noise ratio]]></term>
##     </thesaurusterms>
##     <pubtitle><![CDATA[Photonics Journal, IEEE]]></pubtitle>
##     <punumber><![CDATA[4563994]]></punumber>
##     <pubtype><![CDATA[Journals & Magazines]]></pubtype>
##     <publisher><![CDATA[IEEE]]></publisher>
##     <volume><![CDATA[6]]></volume>
##     <issue><![CDATA[4]]></issue>
##     <py><![CDATA[2014]]></py>
##     <spage><![CDATA[1]]></spage>
##     <epage><![CDATA[8]]></epage>
##     <abstract><![CDATA[We propose and demonstrate a new data-aided in-band optical-signal-to-noise ratio (OSNR) monitoring method, which employs the periodic property of a specially designed short training sequence to differentiate the signal power from noise and is proved to be insensitive to the first-order polarization mode dispersion (PMD). This method is verified in a typical 112-Gb/s polarization multiplexed-quadrature phase shift keying and a 224-Gb/s polarization multiplexed-16 quadrature amplitude modulation coherent optical transmission system. The results show that the method is highly accurate, with the monitoring error less than 1 dB in the OSNR range of 7-30 dB under different linewidths and has good robustness to a large amount of residual accumulated CD and first-order PMD.]]></abstract>
##     <issn><![CDATA[1943-0655]]></issn>
##     <htmlFlag><![CDATA[1]]></htmlFlag>
##     <arnumber><![CDATA[6857332]]></arnumber>
##     <doi><![CDATA[10.1109/JPHOT.2014.2339319]]></doi>
##     <publicationId><![CDATA[6857332]]></publicationId>
##     <mdurl><![CDATA[http://ieeexplore.ieee.org/xpl/articleDetails.jsp?tp=&arnumber=6857332&contentType=Journals+%26+Magazines]]></mdurl>
##     <pdf><![CDATA[http://ieeexplore.ieee.org/stamp/stamp.jsp?arnumber=6857332]]></pdf>
##   </document>
##   <document>
##     <rank>671</rank>
##     <title><![CDATA[Mapping Tree Species in Coastal Portugal Using Statistically Segmented Principal Component Analysis and Other Methods]]></title>
##     <authors><![CDATA[Pandey, P.C.;  Tate, N.J.;  Balzter, H.]]></authors>
##     <affiliations><![CDATA[Dept. of Geogr., Univ. of Leicester, Leicester, UK]]></affiliations>
##     <controlledterms>
##       <term><![CDATA[geophysical image processing]]></term>
##       <term><![CDATA[geophysical techniques]]></term>
##       <term><![CDATA[hyperspectral imaging]]></term>
##       <term><![CDATA[image classification]]></term>
##       <term><![CDATA[image segmentation]]></term>
##       <term><![CDATA[remote sensing]]></term>
##       <term><![CDATA[vegetation]]></term>
##     </controlledterms>
##     <thesaurusterms>
##       <term><![CDATA[Atmospheric modeling]]></term>
##       <term><![CDATA[Hyperspectral imaging]]></term>
##       <term><![CDATA[Image segmentation]]></term>
##       <term><![CDATA[Principal component analysis]]></term>
##       <term><![CDATA[Sensors]]></term>
##       <term><![CDATA[Vegetation]]></term>
##     </thesaurusterms>
##     <pubtitle><![CDATA[Sensors Journal, IEEE]]></pubtitle>
##     <punumber><![CDATA[7361]]></punumber>
##     <pubtype><![CDATA[Journals & Magazines]]></pubtype>
##     <publisher><![CDATA[IEEE]]></publisher>
##     <volume><![CDATA[14]]></volume>
##     <issue><![CDATA[12]]></issue>
##     <py><![CDATA[2014]]></py>
##     <spage><![CDATA[4434]]></spage>
##     <epage><![CDATA[4441]]></epage>
##     <abstract><![CDATA[Hyperspectral sensors record radiances in a large number of wavelengths of the electromagnetic spectrum and can be used to distinguish different tree species based on their characteristic reflectance signatures. Reflectance spectra were measured from airborne hyperspectral AISA Eagle/Hawk imagery in order to identify different Mediterranean tree species at a coastal test site in Portugal. A spectral range from 400 to 2450 nm was recorded at 2-m spatial resolution. The hyperspectral data are divided into five spectral data ranges. The chosen ranges for segmentation are based on statistical properties as well as on their wavelengths, as radiances of a particular wavelength may overlap with neighboring wavelengths. Principal component analysis (PCA) is applied individually to each spectral range. The first three principal components (PCs) of each range are chosen and are fused into a new data segment of reduced dimensionality. The resulting 15 PCs contain 99.42% of the information content of the original hyperspectral image. These PCs were used for a maximum likelihood classification (MLC). Spectral signatures were also analyzed for the hyperspectral data, and were validated with ground data collected in the field by a handheld spectro-radiometer. Different RGB combinations of PC bands of segmented PC image provide distinct feature identification. A comparison with other classification approaches (spectral angle mapper and MLC of the original hyperspectral imagery) shows that the MLC of the segmented PCA achieves the highest accuracy, due to its ability to reduce the Hughes phenomenon.]]></abstract>
##     <issn><![CDATA[1530-437X]]></issn>
##     <htmlFlag><![CDATA[1]]></htmlFlag>
##     <arnumber><![CDATA[6848757]]></arnumber>
##     <doi><![CDATA[10.1109/JSEN.2014.2335612]]></doi>
##     <publicationId><![CDATA[6848757]]></publicationId>
##     <mdurl><![CDATA[http://ieeexplore.ieee.org/xpl/articleDetails.jsp?tp=&arnumber=6848757&contentType=Journals+%26+Magazines]]></mdurl>
##     <pdf><![CDATA[http://ieeexplore.ieee.org/stamp/stamp.jsp?arnumber=6848757]]></pdf>
##   </document>
##   <document>
##     <rank>672</rank>
##     <title><![CDATA[Blind Multilinear Identification]]></title>
##     <authors><![CDATA[Lek-Heng Lim;  Comon, P.]]></authors>
##     <affiliations><![CDATA[Dept. of Stat., Univ. of Chicago, Chicago, IL, USA]]></affiliations>
##     <controlledterms>
##       <term><![CDATA[antenna arrays]]></term>
##       <term><![CDATA[approximation theory]]></term>
##       <term><![CDATA[blind source separation]]></term>
##       <term><![CDATA[channel estimation]]></term>
##       <term><![CDATA[code division multiple access]]></term>
##       <term><![CDATA[fluorescence spectroscopy]]></term>
##       <term><![CDATA[inverse problems]]></term>
##       <term><![CDATA[radiocommunication]]></term>
##     </controlledterms>
##     <thesaurusterms>
##       <term><![CDATA[Approximation methods]]></term>
##       <term><![CDATA[Hilbert space]]></term>
##       <term><![CDATA[Inverse problems]]></term>
##       <term><![CDATA[Matrix decomposition]]></term>
##       <term><![CDATA[Singular value decomposition]]></term>
##       <term><![CDATA[Tensile stress]]></term>
##       <term><![CDATA[Vectors]]></term>
##     </thesaurusterms>
##     <pubtitle><![CDATA[Information Theory, IEEE Transactions on]]></pubtitle>
##     <punumber><![CDATA[18]]></punumber>
##     <pubtype><![CDATA[Journals & Magazines]]></pubtype>
##     <publisher><![CDATA[IEEE]]></publisher>
##     <volume><![CDATA[60]]></volume>
##     <issue><![CDATA[2]]></issue>
##     <py><![CDATA[2014]]></py>
##     <spage><![CDATA[1260]]></spage>
##     <epage><![CDATA[1280]]></epage>
##     <abstract><![CDATA[We discuss a technique that allows blind recovery of signals or blind identification of mixtures in instances where such recovery or identification were previously thought to be impossible. These instances include: 1) closely located or highly correlated sources in antenna array processing; 2) highly correlated spreading codes in code division multiple access (CDMA) radio communication; and 3) nearly dependent spectra in fluorescence spectroscopy. These have important implications. In the case of antenna array processing, it allows for joint localization and extraction of multiple sources from the measurement of a noisy mixture recorded on multiple sensors in an entirely deterministic manner. In the case of CDMA, it allows the possibility of having a number of users larger than the spreading gain. In the case of fluorescence spectroscopy, it allows for detection of nearly identical chemical constituents. The proposed technique involves the solution of a bounded coherence low-rank multilinear approximation problem. We show that bounded coherence allows us to establish existence and uniqueness of the recovered solution. We will provide some statistical motivation for the approximation problem and discuss greedy approximation bounds. To provide the theoretical underpinnings for this technique, we develop a corresponding theory of sparse separable decompositions of functions, including notions of rank and nuclear norm that can be specialized to the usual ones for matrices and operators and also be applied to hypermatrices and tensors.]]></abstract>
##     <issn><![CDATA[0018-9448]]></issn>
##     <htmlFlag><![CDATA[1]]></htmlFlag>
##     <arnumber><![CDATA[6671475]]></arnumber>
##     <doi><![CDATA[10.1109/TIT.2013.2291876]]></doi>
##     <publicationId><![CDATA[6671475]]></publicationId>
##     <mdurl><![CDATA[http://ieeexplore.ieee.org/xpl/articleDetails.jsp?tp=&arnumber=6671475&contentType=Journals+%26+Magazines]]></mdurl>
##     <pdf><![CDATA[http://ieeexplore.ieee.org/stamp/stamp.jsp?arnumber=6671475]]></pdf>
##   </document>
##   <document>
##     <rank>673</rank>
##     <title><![CDATA[Ladder-Type Soil Model for Dynamic Thermal Rating of Underground Power Cables]]></title>
##     <authors><![CDATA[Diaz-Aguilo, M.;  De Leon, F.;  Jazebi, S.;  Terracciano, M.]]></authors>
##     <affiliations><![CDATA[Department of Electrical and Computer EngineeringNYU Polytechnic School of Engineering, New York University, Brooklyn, NY, USA]]></affiliations>
##     <thesaurusterms>
##       <term><![CDATA[Computational modeling]]></term>
##       <term><![CDATA[Integrated circuit modeling]]></term>
##       <term><![CDATA[Mathematical model]]></term>
##       <term><![CDATA[Power cables]]></term>
##       <term><![CDATA[Soil measurements]]></term>
##       <term><![CDATA[Thermal resistance]]></term>
##       <term><![CDATA[Transient analysis]]></term>
##       <term><![CDATA[Underground cables]]></term>
##     </thesaurusterms>
##     <pubtitle><![CDATA[Power and Energy Technology Systems Journal, IEEE]]></pubtitle>
##     <punumber><![CDATA[6687318]]></punumber>
##     <pubtype><![CDATA[Journals & Magazines]]></pubtype>
##     <publisher><![CDATA[IEEE]]></publisher>
##     <volume><![CDATA[1]]></volume>
##     <py><![CDATA[2014]]></py>
##     <spage><![CDATA[21]]></spage>
##     <epage><![CDATA[30]]></epage>
##     <abstract><![CDATA[This paper presents an optimal <italic>RC</italic> ladder-type equivalent circuit for the representation of the soil for dynamic thermal rating of underground cable installations. This is useful and necessary for their optimal and accurate real-time operation. The model stems from a nonuniform discretization of the soil into layers. The resistive and capacitive circuit elements are computed from the dimensions and physical parameters of the layers. The model is perfectly compatible with the International Electrotechnical Commission thermal&#x2013;electric analog circuits for cables. The optimum model order is determined, for fast and slow thermal transients, from a comprehensive parametric study. It is shown that an exponential distribution of the soil layers leads to accurate results with differences of less than 0.5 &#x00B0;C with respect to transient finite-element simulations. An optimal model with only five layers that delivers accurate results for all practical installations and for all time scenarios is presented. The model of this paper is a simple-to-use and accurate tool to design and analyze transient operation of underground cables. It represents a relevant improvement to the available operation and monitoring tools. For illustration purposes, a step-by-step model construction example is given. The model has been validated against numerous dynamic finite-element simulations.]]></abstract>
##     <htmlFlag><![CDATA[1]]></htmlFlag>
##     <arnumber><![CDATA[6967756]]></arnumber>
##     <doi><![CDATA[10.1109/JPETS.2014.2365017]]></doi>
##     <publicationId><![CDATA[6967756]]></publicationId>
##     <mdurl><![CDATA[http://ieeexplore.ieee.org/xpl/articleDetails.jsp?tp=&arnumber=6967756&contentType=Journals+%26+Magazines]]></mdurl>
##     <pdf><![CDATA[http://ieeexplore.ieee.org/stamp/stamp.jsp?arnumber=6967756]]></pdf>
##   </document>
##   <document>
##     <rank>674</rank>
##     <title><![CDATA[Temperature Droop Characteristics of Internal Efficiency in <inline-formula> <img src="/images/tex/21818.gif" alt="\hbox {In}_{x}\hbox {Ga}_{1-x}\hbox {N/GaN}"> </inline-formula> Quantum Well Light-Emitting Diodes]]></title>
##     <authors><![CDATA[Seoung-Hwan Park;  Yong-Tae Moon]]></authors>
##     <affiliations><![CDATA[Dept. of Electr. Eng., Catholic Univ. of Daegu, Kyungsan, South Korea]]></affiliations>
##     <controlledterms>
##       <term><![CDATA[Auger effect]]></term>
##       <term><![CDATA[III-V semiconductors]]></term>
##       <term><![CDATA[effective mass]]></term>
##       <term><![CDATA[gallium compounds]]></term>
##       <term><![CDATA[indium compounds]]></term>
##       <term><![CDATA[light emitting diodes]]></term>
##       <term><![CDATA[quantum well devices]]></term>
##       <term><![CDATA[semiconductor quantum wells]]></term>
##     </controlledterms>
##     <thesaurusterms>
##       <term><![CDATA[Charge carrier density]]></term>
##       <term><![CDATA[Gallium nitride]]></term>
##       <term><![CDATA[Light emitting diodes]]></term>
##       <term><![CDATA[Radiative recombination]]></term>
##       <term><![CDATA[Stimulated emission]]></term>
##       <term><![CDATA[Temperature]]></term>
##     </thesaurusterms>
##     <pubtitle><![CDATA[Photonics Journal, IEEE]]></pubtitle>
##     <punumber><![CDATA[4563994]]></punumber>
##     <pubtype><![CDATA[Journals & Magazines]]></pubtype>
##     <publisher><![CDATA[IEEE]]></publisher>
##     <volume><![CDATA[6]]></volume>
##     <issue><![CDATA[5]]></issue>
##     <py><![CDATA[2014]]></py>
##     <spage><![CDATA[1]]></spage>
##     <epage><![CDATA[9]]></epage>
##     <abstract><![CDATA[The temperature droop characteristics of internal efficiency (IE) in InGaN/GaN quantum well (QW) structures were investigated using the multiband effective mass theory. In the case of a relatively small Auger recombination (&lt;; C<sub>A</sub> = 5 &#x00D7; 10<sup>-30</sup> cm<sup>6</sup>s), the QW structure with a smaller In composition (x = 0.1) shows a larger hot/cold factor (IE<sub>T1</sub>/IE<sub>T2</sub>, with T<sub>1</sub> &gt; T<sub>2</sub>) than that with a larger In composition (x = 0.3) because the radiative recombination is dominant and the IE of the former is much larger than that of the latter. The hot/cold factors for QW structures with x = 0.1 and 0.3 are 0.85 and 0.71 at J = 100 A/cm<sup>2</sup>, respectively. On the other hand, in the case of a relatively large Auger recombination (&gt; C<sub>A</sub> = 10<sup>-28</sup> cm<sup>6</sup>s), the hot/cold factor (0.69) of the QW structure with a larger In composition is found to be larger than that (0.62) with a smaller In composition. This is attributed to the fact that the Auger recombination is dominant even for the QW structure with a small In composition and that the difference of the IE between two different temperatures decreases with increasing x.]]></abstract>
##     <issn><![CDATA[1943-0655]]></issn>
##     <htmlFlag><![CDATA[1]]></htmlFlag>
##     <arnumber><![CDATA[6899759]]></arnumber>
##     <doi><![CDATA[10.1109/JPHOT.2014.2356504]]></doi>
##     <publicationId><![CDATA[6899759]]></publicationId>
##     <mdurl><![CDATA[http://ieeexplore.ieee.org/xpl/articleDetails.jsp?tp=&arnumber=6899759&contentType=Journals+%26+Magazines]]></mdurl>
##     <pdf><![CDATA[http://ieeexplore.ieee.org/stamp/stamp.jsp?arnumber=6899759]]></pdf>
##   </document>
##   <document>
##     <rank>675</rank>
##     <title><![CDATA[A Moir&#x00E9;-Based Four-Channel Focusing and Leveling Scheme for Projection Lithography]]></title>
##     <authors><![CDATA[Chengliang Di;  Wei Yan;  Song Hu;  Yanli Li;  Didi Yin;  Yan Tang;  Junmin Tong]]></authors>
##     <affiliations><![CDATA[State Key Lab. of Opt. Technol. for Microfabrication, Inst. of Opt. & Electron., Chengdu, China]]></affiliations>
##     <controlledterms>
##       <term><![CDATA[diffraction gratings]]></term>
##       <term><![CDATA[focal planes]]></term>
##       <term><![CDATA[lithography]]></term>
##       <term><![CDATA[moire fringes]]></term>
##       <term><![CDATA[wafer level packaging]]></term>
##     </controlledterms>
##     <thesaurusterms>
##       <term><![CDATA[Accuracy]]></term>
##       <term><![CDATA[Diffraction gratings]]></term>
##       <term><![CDATA[Focusing]]></term>
##       <term><![CDATA[Gratings]]></term>
##       <term><![CDATA[Lithography]]></term>
##       <term><![CDATA[Optical imaging]]></term>
##       <term><![CDATA[Transforms]]></term>
##     </thesaurusterms>
##     <pubtitle><![CDATA[Photonics Journal, IEEE]]></pubtitle>
##     <punumber><![CDATA[4563994]]></punumber>
##     <pubtype><![CDATA[Journals & Magazines]]></pubtype>
##     <publisher><![CDATA[IEEE]]></publisher>
##     <volume><![CDATA[6]]></volume>
##     <issue><![CDATA[4]]></issue>
##     <py><![CDATA[2014]]></py>
##     <spage><![CDATA[1]]></spage>
##     <epage><![CDATA[12]]></epage>
##     <abstract><![CDATA[Focusing and leveling are two imperative processes to adjust the wafer onto the ideal focal plane of projection lithography tools. Based on moire&#x0301; fringes formed by particularly designed dual-grating marks, the four-channel focusing and leveling scheme is proposed and demonstrated. These relationships between the tilted amount of wafer, the vertical defocusing amount, and the phase distributions of moire&#x0301; fringes are deduced. A single-channel experimental setup is constructed to verify the performances of proposed method. Results indicate that the tilted amount and the vertical defocusing amount can be precisely detected with accuracy at 10<sup>-4</sup> rad and several nanometers level, respectively, and therefore meet the demand of the high-demanding focusing and leveling processes.]]></abstract>
##     <issn><![CDATA[1943-0655]]></issn>
##     <htmlFlag><![CDATA[1]]></htmlFlag>
##     <arnumber><![CDATA[6842663]]></arnumber>
##     <doi><![CDATA[10.1109/JPHOT.2014.2332559]]></doi>
##     <publicationId><![CDATA[6842663]]></publicationId>
##     <mdurl><![CDATA[http://ieeexplore.ieee.org/xpl/articleDetails.jsp?tp=&arnumber=6842663&contentType=Journals+%26+Magazines]]></mdurl>
##     <pdf><![CDATA[http://ieeexplore.ieee.org/stamp/stamp.jsp?arnumber=6842663]]></pdf>
##   </document>
##   <document>
##     <rank>676</rank>
##     <title><![CDATA[Multiple Fano Resonances Based on Different Waveguide Modes in a Symmetry Breaking Plasmonic System]]></title>
##     <authors><![CDATA[Zhao Chen;  Li Yu]]></authors>
##     <affiliations><![CDATA[State Key Lab. of Inf. Photonics & Opt. Commun., Beijing Univ. of Posts & Telecommun., Beijing, China]]></affiliations>
##     <controlledterms>
##       <term><![CDATA[MIM devices]]></term>
##       <term><![CDATA[nanophotonics]]></term>
##       <term><![CDATA[nanosensors]]></term>
##       <term><![CDATA[optical design techniques]]></term>
##       <term><![CDATA[optical resonators]]></term>
##       <term><![CDATA[optical waveguides]]></term>
##       <term><![CDATA[plasmonics]]></term>
##     </controlledterms>
##     <thesaurusterms>
##       <term><![CDATA[Educational institutions]]></term>
##       <term><![CDATA[Interference]]></term>
##       <term><![CDATA[Load flow]]></term>
##       <term><![CDATA[Optical resonators]]></term>
##       <term><![CDATA[Optical waveguides]]></term>
##       <term><![CDATA[Plasmons]]></term>
##     </thesaurusterms>
##     <pubtitle><![CDATA[Photonics Journal, IEEE]]></pubtitle>
##     <punumber><![CDATA[4563994]]></punumber>
##     <pubtype><![CDATA[Journals & Magazines]]></pubtype>
##     <publisher><![CDATA[IEEE]]></publisher>
##     <volume><![CDATA[6]]></volume>
##     <issue><![CDATA[6]]></issue>
##     <py><![CDATA[2014]]></py>
##     <spage><![CDATA[1]]></spage>
##     <epage><![CDATA[8]]></epage>
##     <abstract><![CDATA[Multiple Fano resonances are numerically investigated based on different waveguide modes in a nanoscale plasmonic waveguide resonator system, which consists of two grooves coupled with a metal-insulator-metal (MIM) waveguide. Simulation results show that by introducing a small structural breaking in the plasmonic resonator, both symmetric and antisymmetric waveguide modes can be excited. Due to the interaction of the symmetric and antisymmetric waveguide modes, the transmission spectra possess a sharp asymmetrical profile. Because of different origins, these Fano resonances exhibit different dependence on the parameters of the structure and can be easily tuned. These characteristics offer flexibility to design the device. This nanosensor yields a sensitivity of ~820 nm/RIU and a figure-of-merit of ~ 3.2&#x00D7; 10<sup>5</sup>. The utilization of the antisymmetric mode in the MIM waveguide provides a new possibility for designing high-performance plasmonic devices.]]></abstract>
##     <issn><![CDATA[1943-0655]]></issn>
##     <htmlFlag><![CDATA[1]]></htmlFlag>
##     <arnumber><![CDATA[6951358]]></arnumber>
##     <doi><![CDATA[10.1109/JPHOT.2014.2368779]]></doi>
##     <publicationId><![CDATA[6951358]]></publicationId>
##     <mdurl><![CDATA[http://ieeexplore.ieee.org/xpl/articleDetails.jsp?tp=&arnumber=6951358&contentType=Journals+%26+Magazines]]></mdurl>
##     <pdf><![CDATA[http://ieeexplore.ieee.org/stamp/stamp.jsp?arnumber=6951358]]></pdf>
##   </document>
##   <document>
##     <rank>677</rank>
##     <title><![CDATA[Improved Use of Foot Force Sensors and Mobile Phone GPS for Mobility Activity Recognition]]></title>
##     <authors><![CDATA[Zelun Zhang;  Poslad, S.]]></authors>
##     <affiliations><![CDATA[Sch. of Electron. Eng. & Comput. Sci., Queen Mary, Univ. of London, London, UK]]></affiliations>
##     <controlledterms>
##       <term><![CDATA[Global Positioning System]]></term>
##       <term><![CDATA[force measurement]]></term>
##       <term><![CDATA[force sensors]]></term>
##       <term><![CDATA[mobile handsets]]></term>
##     </controlledterms>
##     <thesaurusterms>
##       <term><![CDATA[Accuracy]]></term>
##       <term><![CDATA[Foot]]></term>
##       <term><![CDATA[Force]]></term>
##       <term><![CDATA[Global Positioning System]]></term>
##       <term><![CDATA[Legged locomotion]]></term>
##       <term><![CDATA[Monitoring]]></term>
##       <term><![CDATA[Sensors]]></term>
##     </thesaurusterms>
##     <pubtitle><![CDATA[Sensors Journal, IEEE]]></pubtitle>
##     <punumber><![CDATA[7361]]></punumber>
##     <pubtype><![CDATA[Journals & Magazines]]></pubtype>
##     <publisher><![CDATA[IEEE]]></publisher>
##     <volume><![CDATA[14]]></volume>
##     <issue><![CDATA[12]]></issue>
##     <py><![CDATA[2014]]></py>
##     <spage><![CDATA[4340]]></spage>
##     <epage><![CDATA[4347]]></epage>
##     <abstract><![CDATA[Recent advances in the development of multimodal wearable sensors enable us to gather richer contexts of mobile user activities. The combination of foot force sensor (FF) and GPS is able to afford fine-grained mobility activity recognition. We derive and identify 12 (out of 31) maximally informative FF features, and the minimal most effective insole positions (two per foot) for sensing, to improve the use of FF + GPS methods for mobility activity recognition. We tested the improved FF + GPS method using over 7000 samples collected from ten volunteers in a natural, unconstrained, environment. The results show that the improved FF + GPS can achieve an average accuracy of over 90% when detecting five different mobility activities, including walking, cycling, bus-passenger, car-passenger, and car-driver.]]></abstract>
##     <issn><![CDATA[1530-437X]]></issn>
##     <htmlFlag><![CDATA[1]]></htmlFlag>
##     <arnumber><![CDATA[6838960]]></arnumber>
##     <doi><![CDATA[10.1109/JSEN.2014.2331463]]></doi>
##     <publicationId><![CDATA[6838960]]></publicationId>
##     <mdurl><![CDATA[http://ieeexplore.ieee.org/xpl/articleDetails.jsp?tp=&arnumber=6838960&contentType=Journals+%26+Magazines]]></mdurl>
##     <pdf><![CDATA[http://ieeexplore.ieee.org/stamp/stamp.jsp?arnumber=6838960]]></pdf>
##   </document>
##   <document>
##     <rank>678</rank>
##     <title><![CDATA[Physical Layer Encryption in OFDM-PON Employing Time-Variable Keys From ONUs]]></title>
##     <authors><![CDATA[Pan Cao;  Xiaofeng Hu;  Jiayang Wu;  Liang Zhang;  Xinhong Jiang;  Yikai Su]]></authors>
##     <affiliations><![CDATA[Dept. of Electron. Eng., Shanghai Jiao Tong Univ., Shanghai, China]]></affiliations>
##     <controlledterms>
##       <term><![CDATA[OFDM modulation]]></term>
##       <term><![CDATA[cryptography]]></term>
##       <term><![CDATA[passive optical networks]]></term>
##       <term><![CDATA[quadrature amplitude modulation]]></term>
##       <term><![CDATA[telecommunication security]]></term>
##     </controlledterms>
##     <thesaurusterms>
##       <term><![CDATA[Encryption]]></term>
##       <term><![CDATA[OFDM]]></term>
##       <term><![CDATA[Optical network units]]></term>
##       <term><![CDATA[Passive optical networks]]></term>
##       <term><![CDATA[Physical layer]]></term>
##     </thesaurusterms>
##     <pubtitle><![CDATA[Photonics Journal, IEEE]]></pubtitle>
##     <punumber><![CDATA[4563994]]></punumber>
##     <pubtype><![CDATA[Journals & Magazines]]></pubtype>
##     <publisher><![CDATA[IEEE]]></publisher>
##     <volume><![CDATA[6]]></volume>
##     <issue><![CDATA[2]]></issue>
##     <py><![CDATA[2014]]></py>
##     <spage><![CDATA[1]]></spage>
##     <epage><![CDATA[6]]></epage>
##     <abstract><![CDATA[We propose and experimentally demonstrate a dynamic encryption method to realize physical layer security for orthogonal frequency division multiplexing passive optical network (OFDM-PON). In our scheme, encryption of the downstream signal is obtained by applying exclusive or (xor) operation between optical network units' (ONUs') downstream signals and received upstream signals at the optical line terminal side. The upstream signals are used as secure keys for corresponding ONUs. Then the encrypted downstream signals are sent to the ONU sides, where the downstream signal can be retrieved by applying xor operation again between the encrypted downstream signal and the stored upstream signal. Since each ONU cannot obtain the upstream signals of other ONUs, only the ONU itself can recover its downstream signal from the encrypted downstream signal. Moreover, the secure key is dynamically changing along with the upstream signal, significantly improving the security of the downstream signal for the OFDM-PON system. A 5-Gb/s 16-quadrature amplitude modulation OFDM signal with xor-based encryption has been successfully implemented over a 25-km standard single-mode fiber. Experimental results verify that the encryption scheme can effectively prevent eavesdropping by malicious users.]]></abstract>
##     <issn><![CDATA[1943-0655]]></issn>
##     <htmlFlag><![CDATA[1]]></htmlFlag>
##     <arnumber><![CDATA[6775279]]></arnumber>
##     <doi><![CDATA[10.1109/JPHOT.2014.2311451]]></doi>
##     <publicationId><![CDATA[6775279]]></publicationId>
##     <mdurl><![CDATA[http://ieeexplore.ieee.org/xpl/articleDetails.jsp?tp=&arnumber=6775279&contentType=Journals+%26+Magazines]]></mdurl>
##     <pdf><![CDATA[http://ieeexplore.ieee.org/stamp/stamp.jsp?arnumber=6775279]]></pdf>
##   </document>
##   <document>
##     <rank>679</rank>
##     <title><![CDATA[Source Code Revision History Visualization Tools: Do They Work and What Would it Take to Put Them to Work?]]></title>
##     <authors><![CDATA[Chang Liu;  Xin Ye;  En Ye]]></authors>
##     <affiliations><![CDATA[Sch. of Electr. & Eng. Comput. Sci., Ohio Univ., Athens, OH, USA]]></affiliations>
##     <controlledterms>
##       <term><![CDATA[configuration management]]></term>
##       <term><![CDATA[software engineering]]></term>
##       <term><![CDATA[software tools]]></term>
##       <term><![CDATA[source code (software)]]></term>
##     </controlledterms>
##     <thesaurusterms>
##       <term><![CDATA[Computational complexity]]></term>
##       <term><![CDATA[Data visualization]]></term>
##       <term><![CDATA[Source codes]]></term>
##       <term><![CDATA[VIsualization]]></term>
##     </thesaurusterms>
##     <pubtitle><![CDATA[Access, IEEE]]></pubtitle>
##     <punumber><![CDATA[6287639]]></punumber>
##     <pubtype><![CDATA[Journals & Magazines]]></pubtype>
##     <publisher><![CDATA[IEEE]]></publisher>
##     <volume><![CDATA[2]]></volume>
##     <py><![CDATA[2014]]></py>
##     <spage><![CDATA[404]]></spage>
##     <epage><![CDATA[426]]></epage>
##     <abstract><![CDATA[Source code revision history visualization tools have been around for over two decades. Yet, they have not become a mainstream tool in a typical programmer's toolbox and are not typically available in intergraded development environments. So, do they really work? And if they do, what would it take to put them to work? This paper seeks to answer these two questions through experiments, surveys, and interviews. A source code history visualization tool named TeamWATCH was implemented to visualize subversion code repositories. Two comparative controlled experiments were conducted to evaluate the effectiveness of TeamWATCH. The experimental results showed that the subjects using TeamWATCH spent less time than subjects using the command-line subversion client and TortoiseSVN in answering the same set of questions regarding source code revision history. In addition, surveys and interviews were conducted to identify obstacles in adopting source code history visualization tools. Collectively, the results show that source code history visualization tools do bring value to programmers. Key obstacles to wider adoption in practice include nontrivial overhead in using the tools and perceived complexity in visualization.]]></abstract>
##     <issn><![CDATA[2169-3536]]></issn>
##     <htmlFlag><![CDATA[1]]></htmlFlag>
##     <arnumber><![CDATA[6810769]]></arnumber>
##     <doi><![CDATA[10.1109/ACCESS.2014.2322102]]></doi>
##     <publicationId><![CDATA[6810769]]></publicationId>
##     <mdurl><![CDATA[http://ieeexplore.ieee.org/xpl/articleDetails.jsp?tp=&arnumber=6810769&contentType=Journals+%26+Magazines]]></mdurl>
##     <pdf><![CDATA[http://ieeexplore.ieee.org/stamp/stamp.jsp?arnumber=6810769]]></pdf>
##   </document>
##   <document>
##     <rank>680</rank>
##     <title><![CDATA[RFID Tag Helix Antenna Sensors for Wireless Drug Dosage Monitoring]]></title>
##     <authors><![CDATA[Haiyu Huang;  Peisen Zhao;  Pai-Yen Chen;  Yong Ren;  Xuewu Liu;  Ferrari, M.;  Ye Hu;  Akinwande, D.]]></authors>
##     <affiliations><![CDATA[Methodist Hosp. Res. Inst., Houston, TX, USA]]></affiliations>
##     <controlledterms>
##       <term><![CDATA[antennas]]></term>
##       <term><![CDATA[biomedical electrodes]]></term>
##       <term><![CDATA[drug delivery systems]]></term>
##       <term><![CDATA[drugs]]></term>
##       <term><![CDATA[finite element analysis]]></term>
##       <term><![CDATA[radiofrequency identification]]></term>
##     </controlledterms>
##     <thesaurusterms>
##       <term><![CDATA[Drug delivery]]></term>
##       <term><![CDATA[Helical antennas]]></term>
##       <term><![CDATA[RFID tags]]></term>
##       <term><![CDATA[Radiofrequency identification]]></term>
##       <term><![CDATA[Resonant frequency]]></term>
##       <term><![CDATA[Sensors]]></term>
##       <term><![CDATA[Tagging]]></term>
##       <term><![CDATA[Wireless communication]]></term>
##     </thesaurusterms>
##     <pubtitle><![CDATA[Translational Engineering in Health and Medicine, IEEE Journal of]]></pubtitle>
##     <punumber><![CDATA[6221039]]></punumber>
##     <pubtype><![CDATA[Journals & Magazines]]></pubtype>
##     <publisher><![CDATA[IEEE]]></publisher>
##     <volume><![CDATA[2]]></volume>
##     <py><![CDATA[2014]]></py>
##     <spage><![CDATA[1]]></spage>
##     <epage><![CDATA[8]]></epage>
##     <abstract><![CDATA[Miniaturized helix antennas are integrated with drug reservoirs to function as RFID wireless tag sensors for real-time drug dosage monitoring. The general design procedure of this type of biomedical antenna sensors is proposed based on electromagnetic theory and finite element simulation. A cost effective fabrication process is utilized to encapsulate the antenna sensor within a biocompatible package layer using PDMS material, and at the same time form a drug storage or drug delivery unit inside the sensor. The in vitro experiment on two prototypes of antenna sensor-drug reservoir assembly have shown the ability to monitor the drug dosage by tracking antenna resonant frequency shift from 2.4-2.5-GHz ISM band with realized sensitivity of 1.27 &#x03BC; l/MHz for transdermal drug delivery monitoring and 2.76- &#x03BC; l/MHz sensitivity for implanted drug delivery monitoring.]]></abstract>
##     <issn><![CDATA[2168-2372]]></issn>
##     <htmlFlag><![CDATA[1]]></htmlFlag>
##     <arnumber><![CDATA[6754124]]></arnumber>
##     <doi><![CDATA[10.1109/JTEHM.2014.2309335]]></doi>
##     <publicationId><![CDATA[6754124]]></publicationId>
##     <mdurl><![CDATA[http://ieeexplore.ieee.org/xpl/articleDetails.jsp?tp=&arnumber=6754124&contentType=Journals+%26+Magazines]]></mdurl>
##     <pdf><![CDATA[http://ieeexplore.ieee.org/stamp/stamp.jsp?arnumber=6754124]]></pdf>
##   </document>
##   <document>
##     <rank>681</rank>
##     <title><![CDATA[An Analysis of Whole Body Tracer Kinetics in Dynamic PET Studies With Application to Image-Based Blood Input Function Extraction]]></title>
##     <authors><![CDATA[Jian Huang;  O'Sullivan, F.]]></authors>
##     <affiliations><![CDATA[Dept. of Stat., Univ. Coll. Cork, Cork, Ireland]]></affiliations>
##     <controlledterms>
##       <term><![CDATA[Markov processes]]></term>
##       <term><![CDATA[blood]]></term>
##       <term><![CDATA[blood vessels]]></term>
##       <term><![CDATA[haemodynamics]]></term>
##       <term><![CDATA[least squares approximations]]></term>
##       <term><![CDATA[medical image processing]]></term>
##       <term><![CDATA[positron emission tomography]]></term>
##       <term><![CDATA[statistical analysis]]></term>
##     </controlledterms>
##     <thesaurusterms>
##       <term><![CDATA[Adaptation models]]></term>
##       <term><![CDATA[Atomic measurements]]></term>
##       <term><![CDATA[Blood circulation]]></term>
##       <term><![CDATA[Blood pressure]]></term>
##       <term><![CDATA[Markov processes]]></term>
##       <term><![CDATA[Numerical models]]></term>
##       <term><![CDATA[Positron emission tomography]]></term>
##     </thesaurusterms>
##     <pubtitle><![CDATA[Medical Imaging, IEEE Transactions on]]></pubtitle>
##     <punumber><![CDATA[42]]></punumber>
##     <pubtype><![CDATA[Journals & Magazines]]></pubtype>
##     <publisher><![CDATA[IEEE]]></publisher>
##     <volume><![CDATA[33]]></volume>
##     <issue><![CDATA[5]]></issue>
##     <py><![CDATA[2014]]></py>
##     <spage><![CDATA[1093]]></spage>
##     <epage><![CDATA[1108]]></epage>
##     <abstract><![CDATA[In a positron emission tomography (PET) study, the local uptake of the tracer is dependent on vascular delivery and retention. For dynamic studies the measured uptake time-course information can be best interpreted when knowledge of the time-course of tracer in the blood is available. This is certainly true for the most established tracers such as <sup>18</sup>F-Fluorodeoxyglucose (FDG) and <sup>15</sup>O-Water (H 2O). Since direct sampling of blood as part of PET studies is increasingly impractical, there is ongoing interest in image-extraction of blood time-course information. But analysis of PET-measured blood pool signals is complicated because they will typically involve a combination of arterial, venous and tissue information. Thus, a careful appreciation of these components is needed to interpret the available data. To facilitate this process, we propose a novel Markov chain model for representation of the circulation of a tracer atom in the body. The model represents both arterial and venous time-course patterns. Under reasonable conditions equilibration of tracer activity in arterial and venous blood is achieved by the end of the PET study-consistent with empirical measurement. Statistical inference for Markov model parameters is a challenge. A penalized nonlinear least squares process, incorporating a generalized cross-validation score, is proposed. Random effects analysis is used to adaptively specify the structure of the penalty function based on historical samples of directly measured blood data. A collection of arterially sampled data from PET studies with FDG and H 2O is used to illustrate the methodology. These data analyses are highly supportive of the overall modeling approach. An adaptation of the model to the problem of extraction of arterial blood signals from imaging data is also developed and promising preliminary results for cerebral and thoracic imaging studies with FDG and H<sub>2</sub>O are obtained.]]></abstract>
##     <issn><![CDATA[0278-0062]]></issn>
##     <htmlFlag><![CDATA[1]]></htmlFlag>
##     <arnumber><![CDATA[6734713]]></arnumber>
##     <doi><![CDATA[10.1109/TMI.2014.2305113]]></doi>
##     <publicationId><![CDATA[6734713]]></publicationId>
##     <mdurl><![CDATA[http://ieeexplore.ieee.org/xpl/articleDetails.jsp?tp=&arnumber=6734713&contentType=Journals+%26+Magazines]]></mdurl>
##     <pdf><![CDATA[http://ieeexplore.ieee.org/stamp/stamp.jsp?arnumber=6734713]]></pdf>
##   </document>
##   <document>
##     <rank>682</rank>
##     <title><![CDATA[Breakthroughs in Photonics 2013: THz Communications Based on Photonics]]></title>
##     <authors><![CDATA[Nagatsuma, T.]]></authors>
##     <affiliations><![CDATA[Grad. Sch. of Eng. Sci., Osaka Univ., Toyonaka, Japan]]></affiliations>
##     <controlledterms>
##       <term><![CDATA[optical fibre communication]]></term>
##       <term><![CDATA[signal detection]]></term>
##       <term><![CDATA[terahertz wave detectors]]></term>
##     </controlledterms>
##     <thesaurusterms>
##       <term><![CDATA[Bandwidth]]></term>
##       <term><![CDATA[Detectors]]></term>
##       <term><![CDATA[Modulation]]></term>
##       <term><![CDATA[Optical transmitters]]></term>
##       <term><![CDATA[Photodiodes]]></term>
##       <term><![CDATA[Photonics]]></term>
##       <term><![CDATA[Wireless communication]]></term>
##     </thesaurusterms>
##     <pubtitle><![CDATA[Photonics Journal, IEEE]]></pubtitle>
##     <punumber><![CDATA[4563994]]></punumber>
##     <pubtype><![CDATA[Journals & Magazines]]></pubtype>
##     <publisher><![CDATA[IEEE]]></publisher>
##     <volume><![CDATA[6]]></volume>
##     <issue><![CDATA[2]]></issue>
##     <py><![CDATA[2014]]></py>
##     <spage><![CDATA[1]]></spage>
##     <epage><![CDATA[5]]></epage>
##     <abstract><![CDATA[There has been an increasing interest in the application of terahertz (THz) waves to broadband wireless communications. In particular, use of frequencies above 275 GHz is one of the big concerns among radio scientists and engineers, because these frequency bands have not yet been allocated at specific active services, and there is a possibility to employ extremely large bandwidths for ultrabroadband wireless communications. Introduction of photonics technologies for signal generation, modulation, and detection is effective not only to enhance the bandwidth and/or the data rate but also to combine fiber-optic and wireless networks. This paper reviews recent progress in THz wireless communications using telecom-based photonics technologies towards 100 Gb/s.]]></abstract>
##     <issn><![CDATA[1943-0655]]></issn>
##     <htmlFlag><![CDATA[1]]></htmlFlag>
##     <arnumber><![CDATA[6758380]]></arnumber>
##     <doi><![CDATA[10.1109/JPHOT.2014.2309643]]></doi>
##     <publicationId><![CDATA[6758380]]></publicationId>
##     <mdurl><![CDATA[http://ieeexplore.ieee.org/xpl/articleDetails.jsp?tp=&arnumber=6758380&contentType=Journals+%26+Magazines]]></mdurl>
##     <pdf><![CDATA[http://ieeexplore.ieee.org/stamp/stamp.jsp?arnumber=6758380]]></pdf>
##   </document>
##   <document>
##     <rank>683</rank>
##     <title><![CDATA[Modernization of National Oil Industry in Mexico: Upgrading With IEC61850]]></title>
##     <authors><![CDATA[Ruiz Flores, L.I.;  Garcia, J.A.E.]]></authors>
##     <affiliations><![CDATA[Dept. of Electr. Syst., Inst. de Investig. Electr., Cuernavaca, Mexico]]></affiliations>
##     <controlledterms>
##       <term><![CDATA[cost accounting]]></term>
##       <term><![CDATA[factory automation]]></term>
##       <term><![CDATA[maintenance engineering]]></term>
##       <term><![CDATA[petroleum industry]]></term>
##       <term><![CDATA[quality control]]></term>
##       <term><![CDATA[synchronisation]]></term>
##     </controlledterms>
##     <thesaurusterms>
##       <term><![CDATA[Automation]]></term>
##       <term><![CDATA[IEC standards]]></term>
##       <term><![CDATA[Maintenance engineering]]></term>
##       <term><![CDATA[Mexico]]></term>
##       <term><![CDATA[Oil refineries]]></term>
##       <term><![CDATA[Petroleum industry]]></term>
##       <term><![CDATA[Power system planning]]></term>
##       <term><![CDATA[Synchronization]]></term>
##     </thesaurusterms>
##     <pubtitle><![CDATA[Access, IEEE]]></pubtitle>
##     <punumber><![CDATA[6287639]]></punumber>
##     <pubtype><![CDATA[Journals & Magazines]]></pubtype>
##     <publisher><![CDATA[IEEE]]></publisher>
##     <volume><![CDATA[2]]></volume>
##     <py><![CDATA[2014]]></py>
##     <spage><![CDATA[571]]></spage>
##     <epage><![CDATA[576]]></epage>
##     <abstract><![CDATA[This paper describes the most important topics on automation modernization of electrical repowering system of oil refinery in Mexico with IEC-61850. A strategic automation approach for protection, metering, and control is presented, as well as a description on how to automate the electrical equipment in order to reach a thorough communication level according to the standard IEC-61850, including station, bay, and, process buses and their interoperability concepts. A proposal for process bus, considering several scenarios with merging units interface, is presented. On the other hand, the repowering electrical system have several energy sources, with different conditions and challenges to solve, one of them is the integration of equipment with IEC 61850 protocols, other scenario is the system synchronization. Furthermore, the benefits on improvement in power quality and a scheme to avoid the high costs of maintenance are discussed. With the upgrading of the electrical system, the electrical demand is considered cover for all refining processes, including the new production plant of ultralow sulfur clean gasoline.]]></abstract>
##     <issn><![CDATA[2169-3536]]></issn>
##     <htmlFlag><![CDATA[1]]></htmlFlag>
##     <arnumber><![CDATA[6807631]]></arnumber>
##     <doi><![CDATA[10.1109/ACCESS.2014.2320507]]></doi>
##     <publicationId><![CDATA[6807631]]></publicationId>
##     <mdurl><![CDATA[http://ieeexplore.ieee.org/xpl/articleDetails.jsp?tp=&arnumber=6807631&contentType=Journals+%26+Magazines]]></mdurl>
##     <pdf><![CDATA[http://ieeexplore.ieee.org/stamp/stamp.jsp?arnumber=6807631]]></pdf>
##   </document>
##   <document>
##     <rank>684</rank>
##     <title><![CDATA[Robust Path Diversity for Network Quality of Service in Cyber-Physical Systems]]></title>
##     <authors><![CDATA[Kyung-Joon Park;  Jaemin Kim;  Hyuk Lim;  Yongsoon Eun]]></authors>
##     <affiliations><![CDATA[Dept. of Inf. & Commun. Eng., Daegu Gyeongbuk Inst. of Sci. & Technol. (DGIST), Daegu, South Korea]]></affiliations>
##     <controlledterms>
##       <term><![CDATA[control engineering computing]]></term>
##       <term><![CDATA[delays]]></term>
##       <term><![CDATA[networked control systems]]></term>
##       <term><![CDATA[probability]]></term>
##       <term><![CDATA[quality of service]]></term>
##       <term><![CDATA[reliability]]></term>
##     </controlledterms>
##     <thesaurusterms>
##       <term><![CDATA[Algorithm design and analysis]]></term>
##       <term><![CDATA[Delays]]></term>
##       <term><![CDATA[Optimization]]></term>
##       <term><![CDATA[Quality of service]]></term>
##       <term><![CDATA[Robustness]]></term>
##       <term><![CDATA[Uncertainty]]></term>
##     </thesaurusterms>
##     <pubtitle><![CDATA[Industrial Informatics, IEEE Transactions on]]></pubtitle>
##     <punumber><![CDATA[9424]]></punumber>
##     <pubtype><![CDATA[Journals & Magazines]]></pubtype>
##     <publisher><![CDATA[IEEE]]></publisher>
##     <volume><![CDATA[10]]></volume>
##     <issue><![CDATA[4]]></issue>
##     <py><![CDATA[2014]]></py>
##     <spage><![CDATA[2204]]></spage>
##     <epage><![CDATA[2215]]></epage>
##     <abstract><![CDATA[The reliability of control in cyber-physical systems (CPSs) heavily depends on the network-induced delay. The problem of obtaining a maximum allowable delay bound has been widely studied in the networked control systems (NCS) area. Once the delay bound is derived, the remaining question is how to make a network satisfy the bound. In this paper, we propose a robust path selection algorithm, which exploits multipath diversity for providing robust network performance against intrinsic randomness in delay. Our path selection algorithm gives the required paths for any given robustness level parameterized by the reliability violation probability. Based on extensive experimental results with our testbed, we empirically show that the proposed scheme can provide the required network quality of service (QoS) for system robustness.]]></abstract>
##     <issn><![CDATA[1551-3203]]></issn>
##     <htmlFlag><![CDATA[1]]></htmlFlag>
##     <arnumber><![CDATA[6882779]]></arnumber>
##     <doi><![CDATA[10.1109/TII.2014.2351753]]></doi>
##     <publicationId><![CDATA[6882779]]></publicationId>
##     <mdurl><![CDATA[http://ieeexplore.ieee.org/xpl/articleDetails.jsp?tp=&arnumber=6882779&contentType=Journals+%26+Magazines]]></mdurl>
##     <pdf><![CDATA[http://ieeexplore.ieee.org/stamp/stamp.jsp?arnumber=6882779]]></pdf>
##   </document>
##   <document>
##     <rank>685</rank>
##     <title><![CDATA[Joint Optimization of Transmission-Order Selection and Channel Allocation for Bidirectional Wireless Links&#x2014;Part I: Game Theoretic Analysis]]></title>
##     <authors><![CDATA[Uykan, Z.;  Jantti, R.]]></authors>
##     <affiliations><![CDATA[Dept. of Commun. & Networking (COMNET), Aalto Univ., Espoo, Finland]]></affiliations>
##     <controlledterms>
##       <term><![CDATA[cellular radio]]></term>
##       <term><![CDATA[channel allocation]]></term>
##       <term><![CDATA[game theory]]></term>
##       <term><![CDATA[graph theory]]></term>
##       <term><![CDATA[minimisation]]></term>
##       <term><![CDATA[radio transmitters]]></term>
##       <term><![CDATA[radiofrequency interference]]></term>
##     </controlledterms>
##     <thesaurusterms>
##       <term><![CDATA[Games]]></term>
##       <term><![CDATA[Indexes]]></term>
##       <term><![CDATA[Interference]]></term>
##       <term><![CDATA[Joints]]></term>
##       <term><![CDATA[Optimization]]></term>
##       <term><![CDATA[Signal to noise ratio]]></term>
##       <term><![CDATA[Wireless communication]]></term>
##     </thesaurusterms>
##     <pubtitle><![CDATA[Wireless Communications, IEEE Transactions on]]></pubtitle>
##     <punumber><![CDATA[7693]]></punumber>
##     <pubtype><![CDATA[Journals & Magazines]]></pubtype>
##     <publisher><![CDATA[IEEE]]></publisher>
##     <volume><![CDATA[13]]></volume>
##     <issue><![CDATA[7]]></issue>
##     <py><![CDATA[2014]]></py>
##     <spage><![CDATA[4003]]></spage>
##     <epage><![CDATA[4013]]></epage>
##     <abstract><![CDATA[In this two-part paper, we consider a system consisting of bidirectional wireless links that interfere with each other, which has been the of focus of intensive research recently in emerging wireless systems like Device-to-Device (D2D) communications underlaying cellular networks, heterogeneous networks, and small-cell networks. The problem is allocating the time resources (transmission orders) and frequency resources (channels) among the transmitters such that the overall network interference is minimized. Here, transmission order (TO) is a novel dimension for optimization. In Part I, we analyze the TO optimization problem in the presence of channel allocation (CA), i.e., joint CA and the TO optimization problem from a game theoretic perspective, and prove that the joint optimization problem can be formulated as an exact potential game, which has at least one &#x201C;pure strategy Nash Equilibrium&#x201D; for any subset of the complete CA-TO action set and for any initial CA and TO conditions. We also show that the proposed joint CA and TO game is equal to the max-cut of a novel TO-dependent holistic system interference graph. In Part II, we present novel joint CA-TO algorithms and their performance analysis in D2D communications underlays.]]></abstract>
##     <issn><![CDATA[1536-1276]]></issn>
##     <htmlFlag><![CDATA[1]]></htmlFlag>
##     <arnumber><![CDATA[6783803]]></arnumber>
##     <doi><![CDATA[10.1109/TWC.2014.2314633]]></doi>
##     <publicationId><![CDATA[6783803]]></publicationId>
##     <mdurl><![CDATA[http://ieeexplore.ieee.org/xpl/articleDetails.jsp?tp=&arnumber=6783803&contentType=Journals+%26+Magazines]]></mdurl>
##     <pdf><![CDATA[http://ieeexplore.ieee.org/stamp/stamp.jsp?arnumber=6783803]]></pdf>
##   </document>
##   <document>
##     <rank>686</rank>
##     <title><![CDATA[All-Optical OFDM With Cyclic Prefix Insertion Using Flexible Wavelength Selective Switch Optical Processing]]></title>
##     <authors><![CDATA[Schroder, J.;  Liang Bangyuan Du;  Carpenter, J.;  Eggleton, B.J.;  Lowery, A.J.]]></authors>
##     <affiliations><![CDATA[Centre for Ultrahigh bandwidth Devices for Opt. Syst. (CUDOS), Univ. of Sydney, Sydney, NSW, Australia]]></affiliations>
##     <controlledterms>
##       <term><![CDATA[Fourier transforms]]></term>
##       <term><![CDATA[OFDM modulation]]></term>
##       <term><![CDATA[digital subscriber lines]]></term>
##       <term><![CDATA[modulators]]></term>
##       <term><![CDATA[optical switches]]></term>
##       <term><![CDATA[optical transmitters]]></term>
##       <term><![CDATA[wavelength assignment]]></term>
##     </controlledterms>
##     <thesaurusterms>
##       <term><![CDATA[OFDM]]></term>
##       <term><![CDATA[Optical modulation]]></term>
##       <term><![CDATA[Optical pulses]]></term>
##       <term><![CDATA[Optical receivers]]></term>
##       <term><![CDATA[Optical signal processing]]></term>
##       <term><![CDATA[Optical transmitters]]></term>
##     </thesaurusterms>
##     <pubtitle><![CDATA[Lightwave Technology, Journal of]]></pubtitle>
##     <punumber><![CDATA[50]]></punumber>
##     <pubtype><![CDATA[Journals & Magazines]]></pubtype>
##     <publisher><![CDATA[IEEE]]></publisher>
##     <volume><![CDATA[32]]></volume>
##     <issue><![CDATA[4]]></issue>
##     <py><![CDATA[2014]]></py>
##     <spage><![CDATA[752]]></spage>
##     <epage><![CDATA[759]]></epage>
##     <abstract><![CDATA[We demonstrate that the optical Fourier transform and cyclic prefix in an all-optical OFDM transmitter can be simultaneously implemented using a liquid crystal on silicon wavelength selective switch (WSS). The design uses phase-modulated optical pulses at the inputs of the WSS; this has the advantage that the optical modulators are only sampled by the optical pulses once per data-symbol, so that the transition times between the data symbols are irrelevant to the performance of the system, allowing slow optical modulators to be used. Furthermore, each input of the WSS can be assigned to any combination of output subcarrier frequencies, including frequencies unrelated to the modal frequencies of the comb source. This is especially useful for testing in-service ultra-high bandwidth systems by applying additional wavelengths. As an example, we generate a 10.08 Tb/s signal and transmit along 857.4 km of fiber using 252 10-Gbaud subcarriers with a 10% cyclic prefix. We use an optically-banded digital subcarrier demultiplexer to simultaneously detect three subcarriers using a single coherent receiver.]]></abstract>
##     <issn><![CDATA[0733-8724]]></issn>
##     <htmlFlag><![CDATA[1]]></htmlFlag>
##     <arnumber><![CDATA[6655942]]></arnumber>
##     <doi><![CDATA[10.1109/JLT.2013.2288638]]></doi>
##     <publicationId><![CDATA[6655942]]></publicationId>
##     <mdurl><![CDATA[http://ieeexplore.ieee.org/xpl/articleDetails.jsp?tp=&arnumber=6655942&contentType=Journals+%26+Magazines]]></mdurl>
##     <pdf><![CDATA[http://ieeexplore.ieee.org/stamp/stamp.jsp?arnumber=6655942]]></pdf>
##   </document>
##   <document>
##     <rank>687</rank>
##     <title><![CDATA[Perfect Anomalous Absorption of TM Polarized Light in Metallic Grating Situated in Asymmetric Surroundings]]></title>
##     <authors><![CDATA[Jing Nie;  Hu-Quan Li;  Wen Liu]]></authors>
##     <affiliations><![CDATA[Wuhan Nat. Lab. for Optoelectron., Huazhong Univ. of Sci. & Technol., Wuhan, China]]></affiliations>
##     <controlledterms>
##       <term><![CDATA[diffraction gratings]]></term>
##       <term><![CDATA[light absorption]]></term>
##     </controlledterms>
##     <thesaurusterms>
##       <term><![CDATA[Absorption]]></term>
##       <term><![CDATA[Gratings]]></term>
##       <term><![CDATA[Optical films]]></term>
##       <term><![CDATA[Optical polarization]]></term>
##       <term><![CDATA[Optical reflection]]></term>
##       <term><![CDATA[Plasmons]]></term>
##     </thesaurusterms>
##     <pubtitle><![CDATA[Photonics Journal, IEEE]]></pubtitle>
##     <punumber><![CDATA[4563994]]></punumber>
##     <pubtype><![CDATA[Journals & Magazines]]></pubtype>
##     <publisher><![CDATA[IEEE]]></publisher>
##     <volume><![CDATA[6]]></volume>
##     <issue><![CDATA[6]]></issue>
##     <py><![CDATA[2014]]></py>
##     <spage><![CDATA[1]]></spage>
##     <epage><![CDATA[8]]></epage>
##     <abstract><![CDATA[In this paper, we predict an unexpected perfect optical absorption phenomenon of oblique-incident transverse magnetic polarized light in a metallic grating situated in asymmetric surroundings. Two physical processes result in the anomalous absorption: One is the total internal reflection at the grating's interface; another is the Fabry-Pe&#x0301;rot-like resonance supported in the grating's slits. For small filling factor grating (f &lt;; 0.9), the enhanced anomalous absorption only appears at grazing incidence with incident angle larger than 85&#x00B0;; however, if the filling factor is sufficiently large (f &gt; 0.9), anomalous absorption is realizable only if the total internal reflection occurs. The influences of structural parameters, such as grating thickness, slits material on the anomalous absorption are investigated. Based on these properties, a plasmonic absorption device with a designed absorption line shape is proposed.]]></abstract>
##     <issn><![CDATA[1943-0655]]></issn>
##     <htmlFlag><![CDATA[1]]></htmlFlag>
##     <arnumber><![CDATA[6926753]]></arnumber>
##     <doi><![CDATA[10.1109/JPHOT.2014.2363439]]></doi>
##     <publicationId><![CDATA[6926753]]></publicationId>
##     <mdurl><![CDATA[http://ieeexplore.ieee.org/xpl/articleDetails.jsp?tp=&arnumber=6926753&contentType=Journals+%26+Magazines]]></mdurl>
##     <pdf><![CDATA[http://ieeexplore.ieee.org/stamp/stamp.jsp?arnumber=6926753]]></pdf>
##   </document>
##   <document>
##     <rank>688</rank>
##     <title><![CDATA[Multitemporal Wetland Monitoring in Sub-Saharan West-Africa Using Medium Resolution Optical Satellite Data]]></title>
##     <authors><![CDATA[Moser, L.;  Voigt, S.;  Schoepfer, E.;  Palmer, S.]]></authors>
##     <affiliations><![CDATA[German Aerosp. Center (DLR), German Remote Sensing Data Center (DFD), Wessling, Germany]]></affiliations>
##     <controlledterms>
##       <term><![CDATA[lakes]]></term>
##       <term><![CDATA[land use]]></term>
##       <term><![CDATA[rain]]></term>
##       <term><![CDATA[remote sensing]]></term>
##     </controlledterms>
##     <thesaurusterms>
##       <term><![CDATA[Earth]]></term>
##       <term><![CDATA[Indexes]]></term>
##       <term><![CDATA[MODIS]]></term>
##       <term><![CDATA[Monitoring]]></term>
##       <term><![CDATA[Remote sensing]]></term>
##       <term><![CDATA[Time series analysis]]></term>
##       <term><![CDATA[Water resources]]></term>
##     </thesaurusterms>
##     <pubtitle><![CDATA[Selected Topics in Applied Earth Observations and Remote Sensing, IEEE Journal of]]></pubtitle>
##     <punumber><![CDATA[4609443]]></punumber>
##     <pubtype><![CDATA[Journals & Magazines]]></pubtype>
##     <publisher><![CDATA[IEEE]]></publisher>
##     <volume><![CDATA[7]]></volume>
##     <issue><![CDATA[8]]></issue>
##     <py><![CDATA[2014]]></py>
##     <spage><![CDATA[3402]]></spage>
##     <epage><![CDATA[3415]]></epage>
##     <abstract><![CDATA[Surface water is a critical resource in semiarid West-African regions that are frequently exposed to droughts. Natural and artificial wetlands are of high importance for different livelihoods, particularly during the dry season, from October/November until May. However, wetlands largely go unmonitored. In this work, remote sensing is used to monitor wetlands in semiarid Burkina Faso over large areal extents along a gradient of different rainfall and land use characteristics. Time series of data from the Moderate Resolution Imaging Spectrometer (MODIS) from 2000 to 2012 is used for near-infrared (NIR)-based water monitoring using a latitudinal threshold gradient approach. The occurrence of 21 new water bodies with a size larger than 0.5 km<sup>2</sup> over the 13-year analysis period results from a postclassification change detection. Yearly cumulative spatiotemporal analysis shows lower water extents in the drought seasons of 2000-2001, 2004-2005, and 2011-2012. Multiple wetlands indicate a positive trend toward a larger yearly maximum area, but a negative trend toward shorter flooding duration. Such a negative trend is observed particularly for natural wetlands. The temporal behavior of five selected case studies demonstrates that monthly negative anomalies of watercovered areas coincide with the occurrence of drought seasons. The successful application of remote sensing time series as a tool to monitor wetlands in semiarid regions is presented, and the potential of novel early warning indicators of drought from remote sensing is demonstrated.]]></abstract>
##     <issn><![CDATA[1939-1404]]></issn>
##     <htmlFlag><![CDATA[1]]></htmlFlag>
##     <arnumber><![CDATA[6875901]]></arnumber>
##     <doi><![CDATA[10.1109/JSTARS.2014.2336875]]></doi>
##     <publicationId><![CDATA[6875901]]></publicationId>
##     <mdurl><![CDATA[http://ieeexplore.ieee.org/xpl/articleDetails.jsp?tp=&arnumber=6875901&contentType=Journals+%26+Magazines]]></mdurl>
##     <pdf><![CDATA[http://ieeexplore.ieee.org/stamp/stamp.jsp?arnumber=6875901]]></pdf>
##   </document>
##   <document>
##     <rank>689</rank>
##     <title><![CDATA[Design Challenges in Silicon Photonics]]></title>
##     <authors><![CDATA[Bogaerts, W.;  Fiers, M.;  Dumon, P.]]></authors>
##     <affiliations><![CDATA[IMEC, Dept. of Inf. Technol., Ghent Univ., Ghent, Belgium]]></affiliations>
##     <controlledterms>
##       <term><![CDATA[elemental semiconductors]]></term>
##       <term><![CDATA[integrated optoelectronics]]></term>
##       <term><![CDATA[optical design techniques]]></term>
##       <term><![CDATA[silicon]]></term>
##     </controlledterms>
##     <thesaurusterms>
##       <term><![CDATA[Integrated circuit modeling]]></term>
##       <term><![CDATA[Nonlinear optics]]></term>
##       <term><![CDATA[Optical resonators]]></term>
##       <term><![CDATA[Optical sensors]]></term>
##       <term><![CDATA[Optical waveguides]]></term>
##       <term><![CDATA[Photonics]]></term>
##       <term><![CDATA[Silicon]]></term>
##     </thesaurusterms>
##     <pubtitle><![CDATA[Selected Topics in Quantum Electronics, IEEE Journal of]]></pubtitle>
##     <punumber><![CDATA[2944]]></punumber>
##     <pubtype><![CDATA[Journals & Magazines]]></pubtype>
##     <publisher><![CDATA[IEEE]]></publisher>
##     <volume><![CDATA[20]]></volume>
##     <issue><![CDATA[4]]></issue>
##     <py><![CDATA[2014]]></py>
##     <spage><![CDATA[1]]></spage>
##     <epage><![CDATA[8]]></epage>
##     <abstract><![CDATA[Silicon photonics is rapidly gaining maturity in high-bandwidth optical communication, with applications in datacom, access networks, and I/O for bandwidth-intensive electronics. Also, applications are emerging in spectroscopy and sensing. To get the best performance out of the photonics, co-integration with electronics is needed: side-by-side, stacked, or on the same chip. However, the combination of photonics and electronics introduces a range of new problems on the design side: Codesign and cosimulation of complex photonic and electronic circuits, tolerance to variability, and verification algorithms that can handle photonic circuits. We will discuss these challenges and give an outlook on how tools need to evolve to address the needs of photonic-electronic IC designers.]]></abstract>
##     <issn><![CDATA[1077-260X]]></issn>
##     <htmlFlag><![CDATA[1]]></htmlFlag>
##     <arnumber><![CDATA[6691908]]></arnumber>
##     <doi><![CDATA[10.1109/JSTQE.2013.2295882]]></doi>
##     <publicationId><![CDATA[6691908]]></publicationId>
##     <mdurl><![CDATA[http://ieeexplore.ieee.org/xpl/articleDetails.jsp?tp=&arnumber=6691908&contentType=Journals+%26+Magazines]]></mdurl>
##     <pdf><![CDATA[http://ieeexplore.ieee.org/stamp/stamp.jsp?arnumber=6691908]]></pdf>
##   </document>
##   <document>
##     <rank>690</rank>
##     <title><![CDATA[Optical Sideband Processing Approach for Highly Linear Phase-Modulation/Direct-Detection Microwave Photonics Link]]></title>
##     <authors><![CDATA[Jian Li;  Yi-Chen Zhang;  Song Yu;  Wanyi Gu]]></authors>
##     <affiliations><![CDATA[State Key Lab. of Inf. Photonics & Opt. Commun., Beijing Univ. of Posts & Telecommun., Beijing, China]]></affiliations>
##     <controlledterms>
##       <term><![CDATA[intermodulation distortion]]></term>
##       <term><![CDATA[microwave photonics]]></term>
##       <term><![CDATA[optical links]]></term>
##       <term><![CDATA[optical modulation]]></term>
##       <term><![CDATA[phase modulation]]></term>
##     </controlledterms>
##     <thesaurusterms>
##       <term><![CDATA[Amplitude modulation]]></term>
##       <term><![CDATA[Microwave filters]]></term>
##       <term><![CDATA[Optical attenuators]]></term>
##       <term><![CDATA[Optical fiber communication]]></term>
##       <term><![CDATA[Optical filters]]></term>
##       <term><![CDATA[Optical modulation]]></term>
##       <term><![CDATA[Phase modulation]]></term>
##     </thesaurusterms>
##     <pubtitle><![CDATA[Photonics Journal, IEEE]]></pubtitle>
##     <punumber><![CDATA[4563994]]></punumber>
##     <pubtype><![CDATA[Journals & Magazines]]></pubtype>
##     <publisher><![CDATA[IEEE]]></publisher>
##     <volume><![CDATA[6]]></volume>
##     <issue><![CDATA[5]]></issue>
##     <py><![CDATA[2014]]></py>
##     <spage><![CDATA[1]]></spage>
##     <epage><![CDATA[10]]></epage>
##     <abstract><![CDATA[We propose an optical sideband processing approach for phase-modulation direct-detection microwave photonics link in this paper. With optical sideband processing, phase modulation signal is converted to intensity modulation signal for direct detection and nonlinearity compensation is realized to suppress third-order intermodulation distortion simultaneously. Theoretical analysis shows that, by imposing proper phase shifts on zero and second-order optical sidebands, corresponding third-order intermodulation distortion components can cancel each other. A proof-of-concept experiment is carried out for verification. Experimental results demonstrate a 30.06 dB suppression of third-order intermodulation distortion and the linearized link's dynamic range can also be improved by 17.42 dB compared with the nonlinearized link, which verify well a highly linearized phase-modulation direct-detection microwave photonics link using this proposal.]]></abstract>
##     <issn><![CDATA[1943-0655]]></issn>
##     <htmlFlag><![CDATA[1]]></htmlFlag>
##     <arnumber><![CDATA[6910245]]></arnumber>
##     <doi><![CDATA[10.1109/JPHOT.2014.2360293]]></doi>
##     <publicationId><![CDATA[6910245]]></publicationId>
##     <mdurl><![CDATA[http://ieeexplore.ieee.org/xpl/articleDetails.jsp?tp=&arnumber=6910245&contentType=Journals+%26+Magazines]]></mdurl>
##     <pdf><![CDATA[http://ieeexplore.ieee.org/stamp/stamp.jsp?arnumber=6910245]]></pdf>
##   </document>
##   <document>
##     <rank>691</rank>
##     <title><![CDATA[Combined EEG-fNIRS Decoding of Motor Attempt and Imagery for Brain Switch Control: An Offline Study in Patients With Tetraplegia]]></title>
##     <authors><![CDATA[Blokland, Y.;  Spyrou, L.;  Thijssen, D.;  Eijsvogels, T.;  Colier, W.;  Floor-Westerdijk, M.;  Vlek, R.;  Bruhn, J.;  Farquhar, J.]]></authors>
##     <affiliations><![CDATA[Donders Inst. for Brain, Cognition & Behaviour, Radboud Univ. Nijmegen, Nijmegen, Netherlands]]></affiliations>
##     <controlledterms>
##       <term><![CDATA[biomedical optical imaging]]></term>
##       <term><![CDATA[electroencephalography]]></term>
##       <term><![CDATA[feature extraction]]></term>
##       <term><![CDATA[haemodynamics]]></term>
##       <term><![CDATA[infrared imaging]]></term>
##       <term><![CDATA[medical disorders]]></term>
##       <term><![CDATA[medical signal processing]]></term>
##       <term><![CDATA[signal classification]]></term>
##     </controlledterms>
##     <thesaurusterms>
##       <term><![CDATA[Biomedical imaging]]></term>
##       <term><![CDATA[Educational institutions]]></term>
##       <term><![CDATA[Electroencephalography]]></term>
##       <term><![CDATA[Hemodynamics]]></term>
##       <term><![CDATA[Optical switches]]></term>
##       <term><![CDATA[Optical transmitters]]></term>
##     </thesaurusterms>
##     <pubtitle><![CDATA[Neural Systems and Rehabilitation Engineering, IEEE Transactions on]]></pubtitle>
##     <punumber><![CDATA[7333]]></punumber>
##     <pubtype><![CDATA[Journals & Magazines]]></pubtype>
##     <publisher><![CDATA[IEEE]]></publisher>
##     <volume><![CDATA[22]]></volume>
##     <issue><![CDATA[2]]></issue>
##     <py><![CDATA[2014]]></py>
##     <spage><![CDATA[222]]></spage>
##     <epage><![CDATA[229]]></epage>
##     <abstract><![CDATA[Combining electrophysiological and hemodynamic features is a novel approach for improving current performance of brain switches based on sensorimotor rhythms (SMR). This study was conducted with a dual purpose: to test the feasibility of using a combined electroencephalogram/functional near-infrared spectroscopy (EEG-fNIRS) SMR-based brain switch in patients with tetraplegia, and to examine the performance difference between motor imagery and motor attempt for this user group. A general improvement was found when using both EEG and fNIRS features for classification as compared to using the single-modality EEG classifier, with average classification rates of 79% for attempted movement and 70% for imagined movement. For the control group, rates of 87% and 79% were obtained, respectively, where the &#x201C;attempted movement&#x201D; condition was replaced with &#x201C;actual movement.&#x201D; A combined EEG-fNIRS system might be especially beneficial for users who lack sufficient control of current EEG-based brain switches. The average classification performance in the patient group for attempted movement was significantly higher than for imagined movement using the EEG-only as well as the combined classifier, arguing for the case of a paradigm shift in current brain switch research.]]></abstract>
##     <issn><![CDATA[1534-4320]]></issn>
##     <htmlFlag><![CDATA[1]]></htmlFlag>
##     <arnumber><![CDATA[6678785]]></arnumber>
##     <doi><![CDATA[10.1109/TNSRE.2013.2292995]]></doi>
##     <publicationId><![CDATA[6678785]]></publicationId>
##     <mdurl><![CDATA[http://ieeexplore.ieee.org/xpl/articleDetails.jsp?tp=&arnumber=6678785&contentType=Journals+%26+Magazines]]></mdurl>
##     <pdf><![CDATA[http://ieeexplore.ieee.org/stamp/stamp.jsp?arnumber=6678785]]></pdf>
##   </document>
##   <document>
##     <rank>692</rank>
##     <title><![CDATA[Transferable Multiparty Computation With Applications to the Smart Grid]]></title>
##     <authors><![CDATA[Clark, M.R.;  Hopkinson, K.M.]]></authors>
##     <affiliations><![CDATA[Dept. of Electr. & Comput. Eng., Air Force Inst. of Technol., Wright-Patterson AFB, OH, USA]]></affiliations>
##     <controlledterms>
##       <term><![CDATA[cryptographic protocols]]></term>
##       <term><![CDATA[data privacy]]></term>
##       <term><![CDATA[optimisation]]></term>
##       <term><![CDATA[power engineering computing]]></term>
##       <term><![CDATA[smart meters]]></term>
##       <term><![CDATA[smart power grids]]></term>
##     </controlledterms>
##     <thesaurusterms>
##       <term><![CDATA[Computational modeling]]></term>
##       <term><![CDATA[Cryptography]]></term>
##       <term><![CDATA[Privacy]]></term>
##       <term><![CDATA[Protocols]]></term>
##       <term><![CDATA[Smart grids]]></term>
##       <term><![CDATA[Standards]]></term>
##     </thesaurusterms>
##     <pubtitle><![CDATA[Information Forensics and Security, IEEE Transactions on]]></pubtitle>
##     <punumber><![CDATA[10206]]></punumber>
##     <pubtype><![CDATA[Journals & Magazines]]></pubtype>
##     <publisher><![CDATA[IEEE]]></publisher>
##     <volume><![CDATA[9]]></volume>
##     <issue><![CDATA[9]]></issue>
##     <py><![CDATA[2014]]></py>
##     <spage><![CDATA[1356]]></spage>
##     <epage><![CDATA[1366]]></epage>
##     <abstract><![CDATA[The smart grid offers an exciting way to better manage various aspects of a given utility. This comes with an increased threat of privacy due to fine-grained information reporting from each smart meter. Privacy preserving protocols have been proposed in the literature to deal with this problem. In this paper, we look at optimizing one technique, secure multiparty computation (MPC), for application to in-network, privacy preserving computation for smart meter networks. We propose a new construction for secure MPC, which we call transferable MPC. We present protocols for this construction, and show how this leads to more efficient and scalable multiparty computations in a smart metering application.]]></abstract>
##     <issn><![CDATA[1556-6013]]></issn>
##     <htmlFlag><![CDATA[1]]></htmlFlag>
##     <arnumber><![CDATA[6839004]]></arnumber>
##     <doi><![CDATA[10.1109/TIFS.2014.2331753]]></doi>
##     <publicationId><![CDATA[6839004]]></publicationId>
##     <mdurl><![CDATA[http://ieeexplore.ieee.org/xpl/articleDetails.jsp?tp=&arnumber=6839004&contentType=Journals+%26+Magazines]]></mdurl>
##     <pdf><![CDATA[http://ieeexplore.ieee.org/stamp/stamp.jsp?arnumber=6839004]]></pdf>
##   </document>
##   <document>
##     <rank>693</rank>
##     <title><![CDATA[Flexible Nyquist Pulse Sequence Generation With Variable Bandwidth and Repetition Rate]]></title>
##     <authors><![CDATA[Preussler, S.;  Wenzel, N.;  Schneider, T.]]></authors>
##     <affiliations><![CDATA[Inst. fur Hochfrequenztech., Hochschule fur Telekommunikation Leipzig, Leipzig, Germany]]></affiliations>
##     <controlledterms>
##       <term><![CDATA[microwave photonics]]></term>
##       <term><![CDATA[optical filters]]></term>
##       <term><![CDATA[optical information processing]]></term>
##       <term><![CDATA[optical modulation]]></term>
##       <term><![CDATA[optical pulse generation]]></term>
##     </controlledterms>
##     <thesaurusterms>
##       <term><![CDATA[Bandwidth]]></term>
##       <term><![CDATA[Frequency modulation]]></term>
##       <term><![CDATA[Generators]]></term>
##       <term><![CDATA[Optical modulation]]></term>
##       <term><![CDATA[Optical pulse generation]]></term>
##       <term><![CDATA[Optical signal processing]]></term>
##     </thesaurusterms>
##     <pubtitle><![CDATA[Photonics Journal, IEEE]]></pubtitle>
##     <punumber><![CDATA[4563994]]></punumber>
##     <pubtype><![CDATA[Journals & Magazines]]></pubtype>
##     <publisher><![CDATA[IEEE]]></publisher>
##     <volume><![CDATA[6]]></volume>
##     <issue><![CDATA[4]]></issue>
##     <py><![CDATA[2014]]></py>
##     <spage><![CDATA[1]]></spage>
##     <epage><![CDATA[8]]></epage>
##     <abstract><![CDATA[The rectangular spectrum of sinc-shaped Nyquist pulses enables the encoding of data in a minimum spectral width. Sinc pulses can improve optical sampling devices, could enable the implementation of ideal rectangular microwave photonics filters, and can be used for all-optical signal processing, spectroscopy, and light storage. Recently, the generation of sinc-pulse sequences with extraordinary quality was shown by the utilization of cascaded modulators. However, the line width and repetition rate of the pulses is limited by the modulator bandwidth. Here, we present the nonrestricted generation of flexible Nyquist pulse sequences. Therefore, multiple single lines of a comb generator are extracted with optical filters and subsequently processed by cascaded modulators. In a first proof-of-concept experiment, we achieved almost ideally sinc-shaped Nyquist pulses with a bandwidth of 286 GHz, a pulse width of 3.5 ps, and a duty cycle of 2.2%. However, sinc-shaped Nyquist pulse sequences in the femtosecond range with terahertz bandwidths would be possible with the method.]]></abstract>
##     <issn><![CDATA[1943-0655]]></issn>
##     <htmlFlag><![CDATA[1]]></htmlFlag>
##     <arnumber><![CDATA[6838957]]></arnumber>
##     <doi><![CDATA[10.1109/JPHOT.2014.2331240]]></doi>
##     <publicationId><![CDATA[6838957]]></publicationId>
##     <mdurl><![CDATA[http://ieeexplore.ieee.org/xpl/articleDetails.jsp?tp=&arnumber=6838957&contentType=Journals+%26+Magazines]]></mdurl>
##     <pdf><![CDATA[http://ieeexplore.ieee.org/stamp/stamp.jsp?arnumber=6838957]]></pdf>
##   </document>
##   <document>
##     <rank>694</rank>
##     <title><![CDATA[Magnetic-Resonance-Based Electrical Properties Tomography: A Review]]></title>
##     <authors><![CDATA[Xiaotong Zhang;  Jiaen Liu;  Bin He]]></authors>
##     <affiliations><![CDATA[Dept. of Biomed. Eng., Univ. of Minnesota, Minneapolis, MN, USA]]></affiliations>
##     <controlledterms>
##       <term><![CDATA[bioelectric phenomena]]></term>
##       <term><![CDATA[biomedical MRI]]></term>
##       <term><![CDATA[electric impedance imaging]]></term>
##       <term><![CDATA[electrical conductivity]]></term>
##       <term><![CDATA[image reconstruction]]></term>
##       <term><![CDATA[medical image processing]]></term>
##       <term><![CDATA[permittivity]]></term>
##       <term><![CDATA[phantoms]]></term>
##       <term><![CDATA[reviews]]></term>
##       <term><![CDATA[tumours]]></term>
##     </controlledterms>
##     <thesaurusterms>
##       <term><![CDATA[Bioimpedance]]></term>
##       <term><![CDATA[Biomedical measurement]]></term>
##       <term><![CDATA[Image reconstruction]]></term>
##       <term><![CDATA[Magnetic field measurement]]></term>
##       <term><![CDATA[Magnetic fields]]></term>
##       <term><![CDATA[Magnetic resonance imaging]]></term>
##       <term><![CDATA[Radio frequency]]></term>
##       <term><![CDATA[Specific absorption rate]]></term>
##       <term><![CDATA[Tomography]]></term>
##     </thesaurusterms>
##     <pubtitle><![CDATA[Biomedical Engineering, IEEE Reviews in]]></pubtitle>
##     <punumber><![CDATA[4664312]]></punumber>
##     <pubtype><![CDATA[Journals & Magazines]]></pubtype>
##     <publisher><![CDATA[IEEE]]></publisher>
##     <volume><![CDATA[7]]></volume>
##     <py><![CDATA[2014]]></py>
##     <spage><![CDATA[87]]></spage>
##     <epage><![CDATA[96]]></epage>
##     <abstract><![CDATA[Frequency-dependent electrical properties (EPs; conductivity and permittivity) of biological tissues provide important diagnostic information (e.g., tumor characterization), and also play an important role in quantifying radiofrequency (RF) coil induced specific absorption rate (SAR), which is a major safety concern in high- and ultrahigh-field magnetic resonance imaging (MRI) applications. Cross-sectional imaging of EPs has been pursued for decades. Recently introduced electrical properties tomography (EPT) approaches utilize the measurable RF magnetic field induced by the RF coil in an MRI system to quantitatively reconstruct the EP distribution in vivo and noninvasively with a spatial resolution of a few millimeters or less. This paper reviews the EPT approach from its basic theory in electromagnetism to the state-of-the-art research outcomes. Emphasizing on the imaging reconstruction methods rather than experimentation techniques, we review the developed imaging algorithms, validation results in physical phantoms and biological tissues, as well as their applications in in vivo tumor detection and subject-specific SAR prediction. Challenges for future research are also discussed.]]></abstract>
##     <issn><![CDATA[1937-3333]]></issn>
##     <htmlFlag><![CDATA[1]]></htmlFlag>
##     <arnumber><![CDATA[6701123]]></arnumber>
##     <doi><![CDATA[10.1109/RBME.2013.2297206]]></doi>
##     <publicationId><![CDATA[6701123]]></publicationId>
##     <mdurl><![CDATA[http://ieeexplore.ieee.org/xpl/articleDetails.jsp?tp=&arnumber=6701123&contentType=Journals+%26+Magazines]]></mdurl>
##     <pdf><![CDATA[http://ieeexplore.ieee.org/stamp/stamp.jsp?arnumber=6701123]]></pdf>
##   </document>
##   <document>
##     <rank>695</rank>
##     <title><![CDATA[Strip Detectors Processed on High-Resistivity 6-inch Diameter Magnetic Czochralski Silicon (MCz-Si) Substrates]]></title>
##     <authors><![CDATA[Wu, X.;  Harkonen, J.;  Kalliopuska, J.;  Tuominen, E.;  Maenpaa, T.;  Luukka, P.;  Tuovinen, E.;  Karadzhinova, A.;  Spiegel, L.;  Eranen, S.;  Oja, A.;  Haapalinna, A.]]></authors>
##     <affiliations><![CDATA[VTT Tech. Res. Center of Finland, Espoo, Finland]]></affiliations>
##     <controlledterms>
##       <term><![CDATA[current density]]></term>
##       <term><![CDATA[data acquisition]]></term>
##       <term><![CDATA[leakage currents]]></term>
##       <term><![CDATA[nuclear electronics]]></term>
##       <term><![CDATA[readout electronics]]></term>
##       <term><![CDATA[resistors]]></term>
##       <term><![CDATA[silicon radiation detectors]]></term>
##     </controlledterms>
##     <thesaurusterms>
##       <term><![CDATA[Current measurement]]></term>
##       <term><![CDATA[Detectors]]></term>
##       <term><![CDATA[Semiconductor device measurement]]></term>
##       <term><![CDATA[Silicon]]></term>
##       <term><![CDATA[Strips]]></term>
##       <term><![CDATA[Voltage measurement]]></term>
##     </thesaurusterms>
##     <pubtitle><![CDATA[Nuclear Science, IEEE Transactions on]]></pubtitle>
##     <punumber><![CDATA[23]]></punumber>
##     <pubtype><![CDATA[Journals & Magazines]]></pubtype>
##     <publisher><![CDATA[IEEE]]></publisher>
##     <volume><![CDATA[61]]></volume>
##     <issue><![CDATA[1]]></issue>
##     <part><![CDATA[3]]></part>
##     <py><![CDATA[2014]]></py>
##     <spage><![CDATA[611]]></spage>
##     <epage><![CDATA[618]]></epage>
##     <abstract><![CDATA[Tracking detectors for future high-luminosity particle physics experiments have to be simultaneously radiation hard and cost efficient. This paper describes processing and characterization of p<sup>+</sup> /n<sup>-</sup>/n<sup>+</sup> (n-type silicon bulk) detectors made of high-resistivity Magnetic Czochralski silicon (MCz-Si) substrates with 6-inch wafer diameter. The processing was carried out on a line used for large-scale production of sensors using standard fabrication methods, such as implanting polysilicon resistors to bias individual sensor strips. Special care was taken to avoid the creation of Thermal Donors (TD) during processing. The sensors have a full depletion voltage of 120-150 V which are uniform over the investigated sensors. All of the leakage current densities were below 55 nA/cm<sup>2</sup> at 200 V bias voltage. A strip sensor with 768 channels was attached to readout electronics and tested in particle beam with a data acquisition (DAQ) similar to the system used by the CMS experiment at the CERN LHC. The test beam results show a signal-to-noise ratio greater than 40 for the test beam sensor. The results demonstrate that MCz-Si detectors can reliably be manufactured in the industrial scale semiconductor process.]]></abstract>
##     <issn><![CDATA[0018-9499]]></issn>
##     <htmlFlag><![CDATA[1]]></htmlFlag>
##     <arnumber><![CDATA[6733408]]></arnumber>
##     <doi><![CDATA[10.1109/TNS.2013.2295430]]></doi>
##     <publicationId><![CDATA[6733408]]></publicationId>
##     <mdurl><![CDATA[http://ieeexplore.ieee.org/xpl/articleDetails.jsp?tp=&arnumber=6733408&contentType=Journals+%26+Magazines]]></mdurl>
##     <pdf><![CDATA[http://ieeexplore.ieee.org/stamp/stamp.jsp?arnumber=6733408]]></pdf>
##   </document>
##   <document>
##     <rank>696</rank>
##     <title><![CDATA[Design and Optimization of an Optofluidic Ring Resonator Based on Liquid-Core Hybrid ARROWs]]></title>
##     <authors><![CDATA[Testa, G.;  Persichetti, G.;  Bernini, R.]]></authors>
##     <affiliations><![CDATA[Ist. per il Rilevamento Elettromagnetico dell'Ambiente, Consiglio Naz. delle Ric., Naples, Italy]]></affiliations>
##     <controlledterms>
##       <term><![CDATA[elemental semiconductors]]></term>
##       <term><![CDATA[light interference]]></term>
##       <term><![CDATA[optical losses]]></term>
##       <term><![CDATA[optical resonators]]></term>
##       <term><![CDATA[optical waveguides]]></term>
##       <term><![CDATA[polymers]]></term>
##       <term><![CDATA[silicon]]></term>
##     </controlledterms>
##     <thesaurusterms>
##       <term><![CDATA[Attenuation]]></term>
##       <term><![CDATA[Holographic optical components]]></term>
##       <term><![CDATA[Holography]]></term>
##       <term><![CDATA[Optical imaging]]></term>
##       <term><![CDATA[Optical losses]]></term>
##       <term><![CDATA[Optical ring resonators]]></term>
##       <term><![CDATA[Optical waveguides]]></term>
##     </thesaurusterms>
##     <pubtitle><![CDATA[Photonics Journal, IEEE]]></pubtitle>
##     <punumber><![CDATA[4563994]]></punumber>
##     <pubtype><![CDATA[Journals & Magazines]]></pubtype>
##     <publisher><![CDATA[IEEE]]></publisher>
##     <volume><![CDATA[6]]></volume>
##     <issue><![CDATA[5]]></issue>
##     <py><![CDATA[2014]]></py>
##     <spage><![CDATA[1]]></spage>
##     <epage><![CDATA[14]]></epage>
##     <abstract><![CDATA[In this paper, we present the design and analysis of an integrated optofluidic ring resonator based on liquid-core hybrid polymer-silicon antiresonant reflecting optical waveguide (h-ARROW). We perform a modal analysis of h-ARROW using the finite-difference method, in order to find the optimized optical configuration, which accomplishes single-mode operation and reduced attenuation losses. An accurate investigation of the bend sections is performed to preserve the single-mode behavior with reduced propagation losses. A hybrid liquid-core multimode interference (MMI) device is used as a coupling element in the ring layout, and three possible MMI configurations are simulated and compared. By properly designing and optimizing each optical element, we demonstrate, by simulations, the possibility to achieve a quality factor up to 4 &#x00D7; 10<sup>4</sup> with the extinction ratio of about 31 dB. Bulk and surface sensing performances of the device are also simulated and discussed.]]></abstract>
##     <issn><![CDATA[1943-0655]]></issn>
##     <htmlFlag><![CDATA[1]]></htmlFlag>
##     <arnumber><![CDATA[6892927]]></arnumber>
##     <doi><![CDATA[10.1109/JPHOT.2014.2352615]]></doi>
##     <publicationId><![CDATA[6892927]]></publicationId>
##     <mdurl><![CDATA[http://ieeexplore.ieee.org/xpl/articleDetails.jsp?tp=&arnumber=6892927&contentType=Journals+%26+Magazines]]></mdurl>
##     <pdf><![CDATA[http://ieeexplore.ieee.org/stamp/stamp.jsp?arnumber=6892927]]></pdf>
##   </document>
##   <document>
##     <rank>697</rank>
##     <title><![CDATA[ACO-OFDM-Specified Recoverable Upper Clipping With Efficient Detection for Optical Wireless Communications]]></title>
##     <authors><![CDATA[Wei Xu;  Man Wu;  Hua Zhang;  Xiaohu You;  Chunming Zhao]]></authors>
##     <affiliations><![CDATA[Nat. Mobile Commun. Res. Lab., Southeast Univ., Nanjing, China]]></affiliations>
##     <controlledterms>
##       <term><![CDATA[OFDM modulation]]></term>
##       <term><![CDATA[optical communication]]></term>
##     </controlledterms>
##     <thesaurusterms>
##       <term><![CDATA[Adaptive optics]]></term>
##       <term><![CDATA[Optical modulation]]></term>
##       <term><![CDATA[Optical receivers]]></term>
##       <term><![CDATA[Peak to average power ratio]]></term>
##       <term><![CDATA[Radio frequency]]></term>
##       <term><![CDATA[Time-domain analysis]]></term>
##     </thesaurusterms>
##     <pubtitle><![CDATA[Photonics Journal, IEEE]]></pubtitle>
##     <punumber><![CDATA[4563994]]></punumber>
##     <pubtype><![CDATA[Journals & Magazines]]></pubtype>
##     <publisher><![CDATA[IEEE]]></publisher>
##     <volume><![CDATA[6]]></volume>
##     <issue><![CDATA[5]]></issue>
##     <py><![CDATA[2014]]></py>
##     <spage><![CDATA[1]]></spage>
##     <epage><![CDATA[17]]></epage>
##     <abstract><![CDATA[The high peak-to-average-power ratio (PAPR) of orthogonal frequency-division multiplexing (OFDM) degrades the performance in optical wireless communication systems. This paper proposes a modified asymmetrically clipped optical OFDM (ACO-OFDM) with low PAPR via introducing a recoverable upper-clipping (RoC) procedure. Although some information is clipped by a predetermined peak threshold, the clipped error information is kept and repositioned in our proposed scheme, which is named RoC-ACO-OFDM, instead of simply being dropped in conventional schemes. The proposed method makes full use of the specific structure of ACO-OFDM signals in the time domain, where half of the positions are forced to zeros within an OFDM symbol. The zero-valued positions are utilized to carry the clipped error information. Moreover, we accordingly present an optimal maximum a posteriori (MAP) detection for the RoC-ACO-OFDM system. To facilitate the usage of RoC-ACO-OFDM in practical applications, an efficient detection method is further developed with near-optimal performance. Simulation results show that the proposed RoC-ACO-OFDM achieves a significant PAPR reduction, while maintaining a competitive bit-error rate performance compared with the conventional schemes.]]></abstract>
##     <issn><![CDATA[1943-0655]]></issn>
##     <htmlFlag><![CDATA[1]]></htmlFlag>
##     <arnumber><![CDATA[6887289]]></arnumber>
##     <doi><![CDATA[10.1109/JPHOT.2014.2352643]]></doi>
##     <publicationId><![CDATA[6887289]]></publicationId>
##     <mdurl><![CDATA[http://ieeexplore.ieee.org/xpl/articleDetails.jsp?tp=&arnumber=6887289&contentType=Journals+%26+Magazines]]></mdurl>
##     <pdf><![CDATA[http://ieeexplore.ieee.org/stamp/stamp.jsp?arnumber=6887289]]></pdf>
##   </document>
##   <document>
##     <rank>698</rank>
##     <title><![CDATA[Efficiently Maintaining the Fast Updated Sequential Pattern Trees With Sequence Deletion]]></title>
##     <authors><![CDATA[Lin, J.C.-W.;  Wensheng Gan;  Tzung-Pei Hong]]></authors>
##     <affiliations><![CDATA[Innovative Inf. Ind. Res. Center, Harbin Inst. of Technol., Shenzhen, China]]></affiliations>
##     <controlledterms>
##       <term><![CDATA[data mining]]></term>
##       <term><![CDATA[tree data structures]]></term>
##     </controlledterms>
##     <thesaurusterms>
##       <term><![CDATA[Algorithm design and analysis]]></term>
##       <term><![CDATA[Data mining]]></term>
##       <term><![CDATA[Database systems]]></term>
##       <term><![CDATA[Pattern recognition]]></term>
##       <term><![CDATA[Sequential analysis]]></term>
##       <term><![CDATA[Tree data structures]]></term>
##     </thesaurusterms>
##     <pubtitle><![CDATA[Access, IEEE]]></pubtitle>
##     <punumber><![CDATA[6287639]]></punumber>
##     <pubtype><![CDATA[Journals & Magazines]]></pubtype>
##     <publisher><![CDATA[IEEE]]></publisher>
##     <volume><![CDATA[2]]></volume>
##     <py><![CDATA[2014]]></py>
##     <spage><![CDATA[1374]]></spage>
##     <epage><![CDATA[1383]]></epage>
##     <abstract><![CDATA[Among the discovered knowledge, sequential-pattern mining is used to discover the frequent subsequences from a sequence database. Most research handles the static database in batch mode to discover the desired sequential patterns. In the past, the fast updated (FUP) and Fast UPdated 2 (FUP2) concepts were adopted to, respectively, maintain and update the discovered sequential patterns with sequence insertion and sequence deletion based on the designed FUP sequential pattern (FUSP)-tree structure. Based on the FUP or FUP2 concepts, original customer sequences are required to be rescanned if it is necessary to maintain and update the unpromising (small) sequences from the original database. In the past, pre-large concept was designed to keep the prelarge itemsets as the buffer to avoid the database rescan each time whether transaction insertion or deletion in the dynamic databases. In this paper, the prelarge concept is adopted to handle the discovered sequential patterns with sequence deletion. An FUSP tree is first built to keep only the frequent 1-sequences from the original database. The prelarge 1-sequences are also kept in a set for later maintenance approach. When some sequences are deleted from the original database, the proposed algorithm is then performed to divide the kept frequent 1-sequences and prelarge 1-sequences from the original database and the mined 1-sequences from the deleted customer sequences into three parts with nine cases. Each case is then processed by the designed algorithm to maintain and update the built FUSP tree. When the number of deleted customer sequences is smaller than the safety bound of the prelarge concept, the original customer sequences are unnecessary to be rescanned, but the sequential patterns can still be actually maintained and updated. Experiments are conducted to show the performance of the proposed algorithm in terms of execution time and the number of tree nodes.]]></abstract>
##     <issn><![CDATA[2169-3536]]></issn>
##     <htmlFlag><![CDATA[1]]></htmlFlag>
##     <arnumber><![CDATA[6965610]]></arnumber>
##     <doi><![CDATA[10.1109/ACCESS.2014.2373433]]></doi>
##     <publicationId><![CDATA[6965610]]></publicationId>
##     <mdurl><![CDATA[http://ieeexplore.ieee.org/xpl/articleDetails.jsp?tp=&arnumber=6965610&contentType=Journals+%26+Magazines]]></mdurl>
##     <pdf><![CDATA[http://ieeexplore.ieee.org/stamp/stamp.jsp?arnumber=6965610]]></pdf>
##   </document>
##   <document>
##     <rank>699</rank>
##     <title><![CDATA[Dynamic Image-to-Class Warping for Occluded Face Recognition]]></title>
##     <authors><![CDATA[Xingjie Wei;  Chang-Tsun Li;  Zhen Lei;  Dong Yi;  Li, S.Z.]]></authors>
##     <affiliations><![CDATA[Dept. of Comput. Sci., Univ. of Warwick, Coventry, UK]]></affiliations>
##     <controlledterms>
##       <term><![CDATA[face recognition]]></term>
##       <term><![CDATA[image sequences]]></term>
##       <term><![CDATA[interference]]></term>
##       <term><![CDATA[visual databases]]></term>
##     </controlledterms>
##     <thesaurusterms>
##       <term><![CDATA[Biometrics (access control)]]></term>
##       <term><![CDATA[Face recognition]]></term>
##       <term><![CDATA[Facial features]]></term>
##       <term><![CDATA[Image recognition]]></term>
##       <term><![CDATA[Time warp simulation]]></term>
##     </thesaurusterms>
##     <pubtitle><![CDATA[Information Forensics and Security, IEEE Transactions on]]></pubtitle>
##     <punumber><![CDATA[10206]]></punumber>
##     <pubtype><![CDATA[Journals & Magazines]]></pubtype>
##     <publisher><![CDATA[IEEE]]></publisher>
##     <volume><![CDATA[9]]></volume>
##     <issue><![CDATA[12]]></issue>
##     <py><![CDATA[2014]]></py>
##     <spage><![CDATA[2035]]></spage>
##     <epage><![CDATA[2050]]></epage>
##     <abstract><![CDATA[Face recognition (FR) systems in real-world applications need to deal with a wide range of interferences, such as occlusions and disguises in face images. Compared with other forms of interferences such as nonuniform illumination and pose changes, face with occlusions has not attracted enough attention yet. A novel approach, coined dynamic image-to-class warping (DICW), is proposed in this work to deal with this challenge in FR. The face consists of the forehead, eyes, nose, mouth, and chin in a natural order and this order does not change despite occlusions. Thus, a face image is partitioned into patches, which are then concatenated in the raster scan order to form an ordered sequence. Considering this order information, DICW computes the image-to-class distance between a query face and those of an enrolled subject by finding the optimal alignment between the query sequence and all sequences of that subject along both the time dimension and within-class dimension. Unlike most existing methods, our method is able to deal with occlusions which exist in both gallery and probe images. Extensive experiments on public face databases with various types of occlusions have confirmed the effectiveness of the proposed method.]]></abstract>
##     <issn><![CDATA[1556-6013]]></issn>
##     <htmlFlag><![CDATA[1]]></htmlFlag>
##     <arnumber><![CDATA[6906281]]></arnumber>
##     <doi><![CDATA[10.1109/TIFS.2014.2359632]]></doi>
##     <publicationId><![CDATA[6906281]]></publicationId>
##     <mdurl><![CDATA[http://ieeexplore.ieee.org/xpl/articleDetails.jsp?tp=&arnumber=6906281&contentType=Journals+%26+Magazines]]></mdurl>
##     <pdf><![CDATA[http://ieeexplore.ieee.org/stamp/stamp.jsp?arnumber=6906281]]></pdf>
##   </document>
##   <document>
##     <rank>700</rank>
##     <title><![CDATA[An Interphase Distance Relaying Algorithm for Series-Compensated Transmission Lines]]></title>
##     <authors><![CDATA[Xu, Z.Y.;  Su, Z.P.;  Zhang, J.H.;  Wen, A.;  Yang, Q.X.]]></authors>
##     <affiliations><![CDATA[Sch. of Electr. Electron. Eng., North China Electr. Power Univ., Beijing, China]]></affiliations>
##     <controlledterms>
##       <term><![CDATA[power capacitors]]></term>
##       <term><![CDATA[power transmission faults]]></term>
##       <term><![CDATA[power transmission lines]]></term>
##       <term><![CDATA[power transmission protection]]></term>
##       <term><![CDATA[relay protection]]></term>
##     </controlledterms>
##     <thesaurusterms>
##       <term><![CDATA[Algorithm design and analysis]]></term>
##       <term><![CDATA[Capacitors]]></term>
##       <term><![CDATA[Circuit faults]]></term>
##       <term><![CDATA[Impedance]]></term>
##       <term><![CDATA[Power transmission lines]]></term>
##       <term><![CDATA[Relays]]></term>
##       <term><![CDATA[Transmission line measurements]]></term>
##     </thesaurusterms>
##     <pubtitle><![CDATA[Power Delivery, IEEE Transactions on]]></pubtitle>
##     <punumber><![CDATA[61]]></punumber>
##     <pubtype><![CDATA[Journals & Magazines]]></pubtype>
##     <publisher><![CDATA[IEEE]]></publisher>
##     <volume><![CDATA[29]]></volume>
##     <issue><![CDATA[2]]></issue>
##     <py><![CDATA[2014]]></py>
##     <spage><![CDATA[834]]></spage>
##     <epage><![CDATA[841]]></epage>
##     <abstract><![CDATA[A new interphase distance relaying algorithm for series-compensated transmission lines is proposed in this paper. The fault impedance can be correctively calculated using the proposed algorithm without requirement of the parameters of series capacitor device. The series capacitor can be simplified as a linearized impedance during the fault period, but its parameters can be used as variables in fault loop equations. The real-time digital simulator tests show that the proposed algorithm can obtain more accurate results than other traditional distance relaying algorithms for series-compensated transmission lines.]]></abstract>
##     <issn><![CDATA[0885-8977]]></issn>
##     <htmlFlag><![CDATA[1]]></htmlFlag>
##     <arnumber><![CDATA[6612758]]></arnumber>
##     <doi><![CDATA[10.1109/TPWRD.2013.2281067]]></doi>
##     <publicationId><![CDATA[6612758]]></publicationId>
##     <mdurl><![CDATA[http://ieeexplore.ieee.org/xpl/articleDetails.jsp?tp=&arnumber=6612758&contentType=Journals+%26+Magazines]]></mdurl>
##     <pdf><![CDATA[http://ieeexplore.ieee.org/stamp/stamp.jsp?arnumber=6612758]]></pdf>
##   </document>
##   <document>
##     <rank>701</rank>
##     <title><![CDATA[MIMO DWDM System Using Uncooled DFB Lasers With Adaptive Laser Bias Control and Postphotodetection Crosstalk Cancellation]]></title>
##     <authors><![CDATA[Jiannan Zhu;  Ingham, J.D.;  von Lindeiner, J.B.;  Wonfor, A.;  Penty, R.V.;  White, I.H.]]></authors>
##     <affiliations><![CDATA[Dept. of Eng., Univ. of Cambridge, Cambridge, UK]]></affiliations>
##     <controlledterms>
##       <term><![CDATA[MIMO communication]]></term>
##       <term><![CDATA[adaptive control]]></term>
##       <term><![CDATA[adjacent channel interference]]></term>
##       <term><![CDATA[crosstalk]]></term>
##       <term><![CDATA[decoding]]></term>
##       <term><![CDATA[distributed feedback lasers]]></term>
##       <term><![CDATA[intensity modulation]]></term>
##       <term><![CDATA[interference suppression]]></term>
##       <term><![CDATA[mean square error methods]]></term>
##       <term><![CDATA[photodetectors]]></term>
##       <term><![CDATA[radio receivers]]></term>
##       <term><![CDATA[wavelength division multiplexing]]></term>
##     </controlledterms>
##     <thesaurusterms>
##       <term><![CDATA[Crosstalk]]></term>
##       <term><![CDATA[Decoding]]></term>
##       <term><![CDATA[Laser tuning]]></term>
##       <term><![CDATA[MIMO]]></term>
##       <term><![CDATA[Receivers]]></term>
##       <term><![CDATA[Wavelength division multiplexing]]></term>
##     </thesaurusterms>
##     <pubtitle><![CDATA[Lightwave Technology, Journal of]]></pubtitle>
##     <punumber><![CDATA[50]]></punumber>
##     <pubtype><![CDATA[Journals & Magazines]]></pubtype>
##     <publisher><![CDATA[IEEE]]></publisher>
##     <volume><![CDATA[32]]></volume>
##     <issue><![CDATA[21]]></issue>
##     <py><![CDATA[2014]]></py>
##     <spage><![CDATA[3974]]></spage>
##     <epage><![CDATA[3981]]></epage>
##     <abstract><![CDATA[A proof-of-principle demonstration of a multiple input-multiple output (MIMO) dense wavelength division multiplexing (DWDM) system is reported. It uses standard uncooled distributed feedback lasers with intensity modulation-direction detection (IM-DD), in which the temperature of each laser is allowed to drift independently within a 50 &#x00B0;C temperature range. A feedback-based laser bias control algorithm is introduced to guarantee acceptable wavelength spacing and a postphotodetection minimum mean square error decoder is applied to cancel the interchannel crosstalk. The relative sensitivity of the MIMO receiver in both a random laser temperature drift scenario and a worst-case scenario are investigated by simulations in MATLAB. Experimental results for a 40-channel &#x00D7; 12.5 Gb/s DWDM system transmitting over 28 km of single-mode fiber with worst possible wavelength distribution prove the feasibility of the technique.]]></abstract>
##     <issn><![CDATA[0733-8724]]></issn>
##     <htmlFlag><![CDATA[1]]></htmlFlag>
##     <arnumber><![CDATA[6847154]]></arnumber>
##     <doi><![CDATA[10.1109/JLT.2014.2334474]]></doi>
##     <publicationId><![CDATA[6847154]]></publicationId>
##     <mdurl><![CDATA[http://ieeexplore.ieee.org/xpl/articleDetails.jsp?tp=&arnumber=6847154&contentType=Journals+%26+Magazines]]></mdurl>
##     <pdf><![CDATA[http://ieeexplore.ieee.org/stamp/stamp.jsp?arnumber=6847154]]></pdf>
##   </document>
##   <document>
##     <rank>702</rank>
##     <title><![CDATA[Breakthroughs in Photonics 2013: Photoacoustic Tomography in Biomedicine]]></title>
##     <authors><![CDATA[Junjie Yao;  Wang, L.V.]]></authors>
##     <affiliations><![CDATA[Dept. of Biomed. Eng., Washington Univ. in St. Louis, St. Louis, MO, USA]]></affiliations>
##     <controlledterms>
##       <term><![CDATA[acoustic tomography]]></term>
##       <term><![CDATA[biomedical optical imaging]]></term>
##       <term><![CDATA[biomedical ultrasonics]]></term>
##       <term><![CDATA[blood flow measurement]]></term>
##       <term><![CDATA[brain]]></term>
##       <term><![CDATA[image resolution]]></term>
##       <term><![CDATA[medical image processing]]></term>
##       <term><![CDATA[neurophysiology]]></term>
##       <term><![CDATA[reviews]]></term>
##       <term><![CDATA[ultrasonic imaging]]></term>
##     </controlledterms>
##     <thesaurusterms>
##       <term><![CDATA[Biomedical optical imaging]]></term>
##       <term><![CDATA[Image resolution]]></term>
##       <term><![CDATA[Nonlinear optics]]></term>
##       <term><![CDATA[Optical diffraction]]></term>
##       <term><![CDATA[Optical imaging]]></term>
##       <term><![CDATA[Optical sensors]]></term>
##     </thesaurusterms>
##     <pubtitle><![CDATA[Photonics Journal, IEEE]]></pubtitle>
##     <punumber><![CDATA[4563994]]></punumber>
##     <pubtype><![CDATA[Journals & Magazines]]></pubtype>
##     <publisher><![CDATA[IEEE]]></publisher>
##     <volume><![CDATA[6]]></volume>
##     <issue><![CDATA[2]]></issue>
##     <py><![CDATA[2014]]></py>
##     <spage><![CDATA[1]]></spage>
##     <epage><![CDATA[6]]></epage>
##     <abstract><![CDATA[Photoacoustic tomography (PAT) is one of the fastest growing biomedical imaging modalities in the last decade. Building on its high scalability and complementary imaging contrast to other mainstream modalities, PAT has gained substantial momentum in both preclinical and clinical studies. In 2013, PAT has grown markedly in both its technological capabilities and biomedical applications. In particular, breakthroughs have been made in super-resolution imaging, deep blood flow measurement, small animal resting state brain mapping, video rate functional human imaging, and human breast imaging. These breakthroughs have either successfully solved long-standing technical issues in PAT or significantly enhanced its imaging capability. This review will summarize state-of-the-art developments in PAT and highlight a few representative achievements of the year 2013.]]></abstract>
##     <issn><![CDATA[1943-0655]]></issn>
##     <htmlFlag><![CDATA[1]]></htmlFlag>
##     <arnumber><![CDATA[6762866]]></arnumber>
##     <doi><![CDATA[10.1109/JPHOT.2014.2310197]]></doi>
##     <publicationId><![CDATA[6762866]]></publicationId>
##     <mdurl><![CDATA[http://ieeexplore.ieee.org/xpl/articleDetails.jsp?tp=&arnumber=6762866&contentType=Journals+%26+Magazines]]></mdurl>
##     <pdf><![CDATA[http://ieeexplore.ieee.org/stamp/stamp.jsp?arnumber=6762866]]></pdf>
##   </document>
##   <document>
##     <rank>703</rank>
##     <title><![CDATA[40G-OCDMA-PON System With an Asymmetric Structure Using a Single Multi-Port and Sampled SSFBG Encoder/Decoders]]></title>
##     <authors><![CDATA[Matsumoto, R.;  Kodama, T.;  Shimizu, S.;  Nomura, R.;  Omichi, K.;  Wada, N.;  Kitayama, K.-I.]]></authors>
##     <affiliations><![CDATA[Dept. of Electr., Electron., & Inf. Eng., Osaka Univ., Suita, Japan]]></affiliations>
##     <controlledterms>
##       <term><![CDATA[Bragg gratings]]></term>
##       <term><![CDATA[code division multiplexing]]></term>
##       <term><![CDATA[passive optical networks]]></term>
##     </controlledterms>
##     <thesaurusterms>
##       <term><![CDATA[Decoding]]></term>
##       <term><![CDATA[Fiber gratings]]></term>
##       <term><![CDATA[Optical network units]]></term>
##       <term><![CDATA[Optical refraction]]></term>
##       <term><![CDATA[Optical variables control]]></term>
##       <term><![CDATA[Passive optical networks]]></term>
##     </thesaurusterms>
##     <pubtitle><![CDATA[Lightwave Technology, Journal of]]></pubtitle>
##     <punumber><![CDATA[50]]></punumber>
##     <pubtype><![CDATA[Journals & Magazines]]></pubtype>
##     <publisher><![CDATA[IEEE]]></publisher>
##     <volume><![CDATA[32]]></volume>
##     <issue><![CDATA[6]]></issue>
##     <py><![CDATA[2014]]></py>
##     <spage><![CDATA[1132]]></spage>
##     <epage><![CDATA[1143]]></epage>
##     <abstract><![CDATA[In the hybrid optical code division multiple access (OCDMA) architectures using different types of optical encoder/decoder (E/Ds) between an optical line terminal and optical network units, a hybrid 40G-OCDMA-PON with a single multi-port and super-structured fiber Bragg grating (SSFBG) E/Ds is studied as a candidate system beyond next-generation PON stage 2 (NG-PON2). In this paper, we have developed uniformed and sampled SSFBG E/Ds with different refractive index profiles and have compared their code performances in a hybrid 40G-OCDMA-PON system. The experimental results show that the sampled profile improves the code dependent performances presented in many OCDMA systems. We have also demonstrated a full-duplex 4-user &#x00D7; 40 Gb/s hybrid OCDMA-PON system by applying the sampled SSFBG E/Ds. An asynchronous full-duplex 50 km transmission with a total capacity of 160 Gb/s (4-user &#x00D7; 40 Gb/s) has been achieved for the first time. Furthermore, we focus on major requirements in NG-PON2 and discuss several issues to introduce the hybrid 40G-OCDMA-PON for optical access networks.]]></abstract>
##     <issn><![CDATA[0733-8724]]></issn>
##     <htmlFlag><![CDATA[1]]></htmlFlag>
##     <arnumber><![CDATA[6705616]]></arnumber>
##     <doi><![CDATA[10.1109/JLT.2014.2299070]]></doi>
##     <publicationId><![CDATA[6705616]]></publicationId>
##     <mdurl><![CDATA[http://ieeexplore.ieee.org/xpl/articleDetails.jsp?tp=&arnumber=6705616&contentType=Journals+%26+Magazines]]></mdurl>
##     <pdf><![CDATA[http://ieeexplore.ieee.org/stamp/stamp.jsp?arnumber=6705616]]></pdf>
##   </document>
##   <document>
##     <rank>704</rank>
##     <title><![CDATA[A Nonparametric Temperature Controller With Nonlinear Negative Reaction for Multi-Point Rapid MR-Guided HIFU Ablation]]></title>
##     <authors><![CDATA[Petrusca, L.;  Auboiroux, V.;  Goget, T.;  Viallon, M.;  Muller, A.;  Gross, P.;  Becker, C.D.;  Salomir, R.]]></authors>
##     <affiliations><![CDATA[Fac. of Med., Univ. of Geneva, Geneva, Switzerland]]></affiliations>
##     <controlledterms>
##       <term><![CDATA[biomedical MRI]]></term>
##       <term><![CDATA[dosimetry]]></term>
##       <term><![CDATA[temperature control]]></term>
##       <term><![CDATA[temperature measurement]]></term>
##       <term><![CDATA[ultrasonic therapy]]></term>
##     </controlledterms>
##     <thesaurusterms>
##       <term><![CDATA[Convergence]]></term>
##       <term><![CDATA[Equations]]></term>
##       <term><![CDATA[Heating]]></term>
##       <term><![CDATA[Steady-state]]></term>
##       <term><![CDATA[Temperature measurement]]></term>
##       <term><![CDATA[Ultrasonic imaging]]></term>
##     </thesaurusterms>
##     <pubtitle><![CDATA[Medical Imaging, IEEE Transactions on]]></pubtitle>
##     <punumber><![CDATA[42]]></punumber>
##     <pubtype><![CDATA[Journals & Magazines]]></pubtype>
##     <publisher><![CDATA[IEEE]]></publisher>
##     <volume><![CDATA[33]]></volume>
##     <issue><![CDATA[6]]></issue>
##     <py><![CDATA[2014]]></py>
##     <spage><![CDATA[1324]]></spage>
##     <epage><![CDATA[1337]]></epage>
##     <abstract><![CDATA[Magnetic resonance-guided high intensity focused ultrasound (MRgHIFU) is a noninvasive method for thermal ablation, which exploits the capabilities of magnetic resonance imaging (MRI) for excellent visualization of the target and for near real-time thermometry. Oncological quality of ablation may be obtained by volumetric sonication under automatic feedback control of the temperature. For this purpose, a new nonparametric (i.e., model independent) temperature controller, using nonlinear negative reaction, was designed and evaluated for the iterated sonication of a prescribed pattern of foci. The main objective was to achieve the same thermal history at each sonication point during volumetric MRgHIFU. Differently sized linear and circular trajectories were investigated ex vivo and in vivo using a phased-array HIFU transducer. A clinical 3T MRI scanner was used and the temperature elevation was measured in five slices simultaneously with a voxel size of 1 &#x00D7; 1 &#x00D7; 5 mm<sup>3</sup> and temporal resolution of 4 s. In vivo results indicated a similar thermal history of each sonicated focus along the prescribed pattern, that was 17.3 &#x00B1; 0.5&#x00B0;C as compared to 16&#x00B0;C prescribed temperature elevation. The spatio-temporal control of the temperature also enabled meaningful comparison of various sonication patterns in terms of dosimetry and near-field safety. The thermal build-up tended to drift downwards in the HIFU transducer with a circular scan.]]></abstract>
##     <issn><![CDATA[0278-0062]]></issn>
##     <arnumber><![CDATA[6763076]]></arnumber>
##     <doi><![CDATA[10.1109/TMI.2014.2310704]]></doi>
##     <publicationId><![CDATA[6763076]]></publicationId>
##     <mdurl><![CDATA[http://ieeexplore.ieee.org/xpl/articleDetails.jsp?tp=&arnumber=6763076&contentType=Journals+%26+Magazines]]></mdurl>
##     <pdf><![CDATA[http://ieeexplore.ieee.org/stamp/stamp.jsp?arnumber=6763076]]></pdf>
##   </document>
##   <document>
##     <rank>705</rank>
##     <title><![CDATA[When Does Relay Transmission Give a More Secure Connection in Wireless Ad Hoc Networks?]]></title>
##     <authors><![CDATA[Chunxiao Cai;  Yueming Cai;  Xiangyun Zhou;  Weiwei Yang;  Wendong Yang]]></authors>
##     <affiliations><![CDATA[Coll. of Commun. Eng., People's Liberation Army Univ. of Sci. & Technol., Nanjing, China]]></affiliations>
##     <controlledterms>
##       <term><![CDATA[ad hoc networks]]></term>
##       <term><![CDATA[relay networks (telecommunication)]]></term>
##       <term><![CDATA[telecommunication security]]></term>
##     </controlledterms>
##     <thesaurusterms>
##       <term><![CDATA[Educational institutions]]></term>
##       <term><![CDATA[Licenses]]></term>
##       <term><![CDATA[Relays]]></term>
##       <term><![CDATA[Security]]></term>
##       <term><![CDATA[Signal to noise ratio]]></term>
##       <term><![CDATA[Wireless networks]]></term>
##     </thesaurusterms>
##     <pubtitle><![CDATA[Information Forensics and Security, IEEE Transactions on]]></pubtitle>
##     <punumber><![CDATA[10206]]></punumber>
##     <pubtype><![CDATA[Journals & Magazines]]></pubtype>
##     <publisher><![CDATA[IEEE]]></publisher>
##     <volume><![CDATA[9]]></volume>
##     <issue><![CDATA[4]]></issue>
##     <py><![CDATA[2014]]></py>
##     <spage><![CDATA[624]]></spage>
##     <epage><![CDATA[632]]></epage>
##     <abstract><![CDATA[Relay transmission can enhance coverage and throughput, whereas it can be vulnerable to eavesdropping attacks due to the additional transmission of the source message at the relay. Thus, whether or not one should use relay transmission for secure communication is an interesting and important problem. In this paper, we consider the transmission of a confidential message from a source to a destination in a decentralized wireless network in the presence of randomly distributed eavesdroppers. The source-destination pair can be potentially assisted by randomly distributed relays. For an arbitrary relay, we derive exact expressions of secure connection probability for both colluding and noncolluding eavesdroppers. We further obtain lower bound expressions on the secure connection probability, which are accurate when the eavesdropper density is small. Using these lower bound expressions, we propose a relay selection strategy to improve the secure connection probability. By analytically comparing the secure connection probability for direct transmission and relay transmission, we address the important problem of whether or not to relay and discuss the condition for relay transmission in terms of the relay density and source-destination distance. These analytical results are accurate in the small eavesdropper density regime.]]></abstract>
##     <issn><![CDATA[1556-6013]]></issn>
##     <htmlFlag><![CDATA[1]]></htmlFlag>
##     <arnumber><![CDATA[6702428]]></arnumber>
##     <doi><![CDATA[10.1109/TIFS.2013.2297835]]></doi>
##     <publicationId><![CDATA[6702428]]></publicationId>
##     <mdurl><![CDATA[http://ieeexplore.ieee.org/xpl/articleDetails.jsp?tp=&arnumber=6702428&contentType=Journals+%26+Magazines]]></mdurl>
##     <pdf><![CDATA[http://ieeexplore.ieee.org/stamp/stamp.jsp?arnumber=6702428]]></pdf>
##   </document>
##   <document>
##     <rank>706</rank>
##     <title><![CDATA[Quantum Information Transmission Over a Partially Degradable Channel]]></title>
##     <authors><![CDATA[Gyongyosi, L.]]></authors>
##     <affiliations><![CDATA[Dept. of Telecommun., Budapest Univ. of Technol. & Econ., Budapest, Hungary]]></affiliations>
##     <controlledterms>
##       <term><![CDATA[data communication]]></term>
##       <term><![CDATA[encoding]]></term>
##       <term><![CDATA[protocols]]></term>
##       <term><![CDATA[quantum communication]]></term>
##       <term><![CDATA[telecommunication channels]]></term>
##     </controlledterms>
##     <thesaurusterms>
##       <term><![CDATA[Channel capacity]]></term>
##       <term><![CDATA[Encoding]]></term>
##       <term><![CDATA[Noise measurement]]></term>
##       <term><![CDATA[Protocols]]></term>
##       <term><![CDATA[Quantum computing]]></term>
##       <term><![CDATA[Quantum mechanics]]></term>
##       <term><![CDATA[Standards]]></term>
##     </thesaurusterms>
##     <pubtitle><![CDATA[Access, IEEE]]></pubtitle>
##     <punumber><![CDATA[6287639]]></punumber>
##     <pubtype><![CDATA[Journals & Magazines]]></pubtype>
##     <publisher><![CDATA[IEEE]]></publisher>
##     <volume><![CDATA[2]]></volume>
##     <py><![CDATA[2014]]></py>
##     <spage><![CDATA[195]]></spage>
##     <epage><![CDATA[198]]></epage>
##     <abstract><![CDATA[We investigate a quantum coding for quantum communication over a partially degradable (PD) quantum channel. For a PD channel, the degraded environment state can be expressed from the channel output state up to a degrading map. PD channels can be restricted to the set of optical channels, which allows for the parties to exploit the benefits in experimental quantum communications. We show that for a PD channel, the partial degradability property leads to higher quantum data rates in comparison with those of a degradable channel. The PD property is particularly convenient for quantum communications and allows one to implement the experimental quantum protocols with higher performance. We define a coding scheme for PD channels and give the achievable rates of quantum communication.]]></abstract>
##     <issn><![CDATA[2169-3536]]></issn>
##     <htmlFlag><![CDATA[1]]></htmlFlag>
##     <arnumber><![CDATA[6748877]]></arnumber>
##     <doi><![CDATA[10.1109/ACCESS.2014.2308574]]></doi>
##     <publicationId><![CDATA[6748877]]></publicationId>
##     <mdurl><![CDATA[http://ieeexplore.ieee.org/xpl/articleDetails.jsp?tp=&arnumber=6748877&contentType=Journals+%26+Magazines]]></mdurl>
##     <pdf><![CDATA[http://ieeexplore.ieee.org/stamp/stamp.jsp?arnumber=6748877]]></pdf>
##   </document>
##   <document>
##     <rank>707</rank>
##     <title><![CDATA[A Hybrid Power Control Concept for PV Inverters With Reduced Thermal Loading]]></title>
##     <authors><![CDATA[Yongheng Yang;  Huai Wang;  Blaabjerg, F.;  Kerekes, T.]]></authors>
##     <affiliations><![CDATA[Dept. of Energy Technol., Aalborg Univ., Aalborg, Denmark]]></affiliations>
##     <controlledterms>
##       <term><![CDATA[invertors]]></term>
##       <term><![CDATA[maximum power point trackers]]></term>
##       <term><![CDATA[power control]]></term>
##       <term><![CDATA[power generation control]]></term>
##       <term><![CDATA[power grids]]></term>
##       <term><![CDATA[solar cells]]></term>
##     </controlledterms>
##     <thesaurusterms>
##       <term><![CDATA[Hybrid power systems]]></term>
##       <term><![CDATA[Inverters]]></term>
##       <term><![CDATA[Loading]]></term>
##       <term><![CDATA[Power control]]></term>
##       <term><![CDATA[Power generation]]></term>
##       <term><![CDATA[Temperature control]]></term>
##       <term><![CDATA[Thermal loading]]></term>
##     </thesaurusterms>
##     <pubtitle><![CDATA[Power Electronics, IEEE Transactions on]]></pubtitle>
##     <punumber><![CDATA[63]]></punumber>
##     <pubtype><![CDATA[Journals & Magazines]]></pubtype>
##     <publisher><![CDATA[IEEE]]></publisher>
##     <volume><![CDATA[29]]></volume>
##     <issue><![CDATA[12]]></issue>
##     <py><![CDATA[2014]]></py>
##     <spage><![CDATA[6271]]></spage>
##     <epage><![CDATA[6275]]></epage>
##     <abstract><![CDATA[This letter proposes a hybrid power control concept for grid-connected photovoltaic (PV) inverters. The control strategy is based on either a maximum power point tracking control or a constant power generation (CPG) control depending on the instantaneous available power from the PV panels. The essence of the proposed concept lies in the selection of an appropriate power limit for the CPG control to achieve an improved thermal performance and an increased utilization factor of PV inverters, and thus, to cater for a higher penetration level of PV systems with intermittent nature. A case study on a single-phase PV inverter under yearly operation is presented with analyses of the thermal loading, lifetime, and annual energy yield. It has revealed the trade-off factors to select the power limit and also verified the feasibility and the effectiveness of the proposed control concept.]]></abstract>
##     <issn><![CDATA[0885-8993]]></issn>
##     <htmlFlag><![CDATA[1]]></htmlFlag>
##     <arnumber><![CDATA[6844045]]></arnumber>
##     <doi><![CDATA[10.1109/TPEL.2014.2332754]]></doi>
##     <publicationId><![CDATA[6844045]]></publicationId>
##     <mdurl><![CDATA[http://ieeexplore.ieee.org/xpl/articleDetails.jsp?tp=&arnumber=6844045&contentType=Journals+%26+Magazines]]></mdurl>
##     <pdf><![CDATA[http://ieeexplore.ieee.org/stamp/stamp.jsp?arnumber=6844045]]></pdf>
##   </document>
##   <document>
##     <rank>708</rank>
##     <title><![CDATA[Automatic Generation of Statistical Pose and Shape Models for Articulated Joints]]></title>
##     <authors><![CDATA[Xin Chen;  Graham, J.;  Hutchinson, C.;  Muir, L.]]></authors>
##     <affiliations><![CDATA[Centre for Imaging Sci., Univ. of Manchester, Manchester, UK]]></affiliations>
##     <controlledterms>
##       <term><![CDATA[biomedical MRI]]></term>
##       <term><![CDATA[bone]]></term>
##       <term><![CDATA[computerised tomography]]></term>
##       <term><![CDATA[diseases]]></term>
##       <term><![CDATA[image registration]]></term>
##       <term><![CDATA[image segmentation]]></term>
##       <term><![CDATA[iterative methods]]></term>
##       <term><![CDATA[medical image processing]]></term>
##       <term><![CDATA[statistical analysis]]></term>
##     </controlledterms>
##     <thesaurusterms>
##       <term><![CDATA[Bones]]></term>
##       <term><![CDATA[Image segmentation]]></term>
##       <term><![CDATA[Joints]]></term>
##       <term><![CDATA[Mathematical model]]></term>
##       <term><![CDATA[Radio access networks]]></term>
##       <term><![CDATA[Three-dimensional displays]]></term>
##       <term><![CDATA[Wrist]]></term>
##     </thesaurusterms>
##     <pubtitle><![CDATA[Medical Imaging, IEEE Transactions on]]></pubtitle>
##     <punumber><![CDATA[42]]></punumber>
##     <pubtype><![CDATA[Journals & Magazines]]></pubtype>
##     <publisher><![CDATA[IEEE]]></publisher>
##     <volume><![CDATA[33]]></volume>
##     <issue><![CDATA[2]]></issue>
##     <py><![CDATA[2014]]></py>
##     <spage><![CDATA[372]]></spage>
##     <epage><![CDATA[383]]></epage>
##     <abstract><![CDATA[Statistical analysis of motion patterns of body joints is potentially useful for detecting and quantifying pathologies. However, building a statistical motion model across different subjects remains a challenging task, especially for a complex joint like the wrist. We present a novel framework for simultaneous registration and segmentation of multiple 3-D (CT or MR) volumes of different subjects at various articulated positions. The framework starts with a pose model generated from 3-D volumes captured at different articulated positions of a single subject (template). This initial pose model is used to register the template volume to image volumes from new subjects. During this process, the Grow-Cut algorithm is used in an iterative refinement of the segmentation of the bone along with the pose parameters. As each new subject is registered and segmented, the pose model is updated, improving the accuracy of successive registrations. We applied the algorithm to CT images of the wrist from 25 subjects, each at five different wrist positions and demonstrated that it performed robustly and accurately. More importantly, the resulting segmentations allowed a statistical pose model of the carpal bones to be generated automatically without interaction. The evaluation results show that our proposed framework achieved accurate registration with an average mean target registration error of 0.34 &#x00B1;0.27 mm. The automatic segmentation results also show high consistency with the ground truth obtained semi-automatically. Furthermore, we demonstrated the capability of the resulting statistical pose and shape models by using them to generate a measurement tool for scaphoid-lunate dissociation diagnosis, which achieved 90% sensitivity and specificity.]]></abstract>
##     <issn><![CDATA[0278-0062]]></issn>
##     <htmlFlag><![CDATA[1]]></htmlFlag>
##     <arnumber><![CDATA[6630071]]></arnumber>
##     <doi><![CDATA[10.1109/TMI.2013.2285503]]></doi>
##     <publicationId><![CDATA[6630071]]></publicationId>
##     <mdurl><![CDATA[http://ieeexplore.ieee.org/xpl/articleDetails.jsp?tp=&arnumber=6630071&contentType=Journals+%26+Magazines]]></mdurl>
##     <pdf><![CDATA[http://ieeexplore.ieee.org/stamp/stamp.jsp?arnumber=6630071]]></pdf>
##   </document>
##   <document>
##     <rank>709</rank>
##     <title><![CDATA[The Sensitive and Efficient Detection of Quadriceps Muscle Thickness Changes in Cross-Sectional Plane Using Ultrasonography: A Feasibility Investigation]]></title>
##     <authors><![CDATA[Jizhou Li;  Yongjin Zhou;  Yi Lu;  Guangquan Zhou;  Lei Wang;  Yong-Ping Zheng]]></authors>
##     <affiliations><![CDATA[Shenzhen Inst. of Adv. Technol., Shenzhen, China]]></affiliations>
##     <controlledterms>
##       <term><![CDATA[biomechanics]]></term>
##       <term><![CDATA[biomedical ultrasonics]]></term>
##       <term><![CDATA[medical image processing]]></term>
##       <term><![CDATA[muscle]]></term>
##       <term><![CDATA[thickness measurement]]></term>
##     </controlledterms>
##     <thesaurusterms>
##       <term><![CDATA[Biomedical measurement]]></term>
##       <term><![CDATA[Manuals]]></term>
##       <term><![CDATA[Muscles]]></term>
##       <term><![CDATA[Probes]]></term>
##       <term><![CDATA[Torque]]></term>
##       <term><![CDATA[Ultrasonic imaging]]></term>
##       <term><![CDATA[Ultrasonic variables measurement]]></term>
##     </thesaurusterms>
##     <pubtitle><![CDATA[Biomedical and Health Informatics, IEEE Journal of]]></pubtitle>
##     <punumber><![CDATA[6221020]]></punumber>
##     <pubtype><![CDATA[Journals & Magazines]]></pubtype>
##     <publisher><![CDATA[IEEE]]></publisher>
##     <volume><![CDATA[18]]></volume>
##     <issue><![CDATA[2]]></issue>
##     <py><![CDATA[2014]]></py>
##     <spage><![CDATA[628]]></spage>
##     <epage><![CDATA[635]]></epage>
##     <abstract><![CDATA[As a direct determinant parameter to quantify muscle activity, the muscle thickness (MT) has been investigated in many aspects and for various purposes. Ultrasonography (US) is a promising modality to detect muscle morphological changes during contractions since it is portable, noninvasive, and real time. However, there are few reports on sensitive and efficient estimation of changes of MT in a cross-sectional plane. In this feasibility investigation, we proposed a coarse-to-fine method based on a compressive-tracking algorithm for estimation of MT changes during an example task of isometric knee extension using ultrasound images. The sensitivity and efficiency are evaluated with 1920 US images from quadriceps muscle (QM) in eight subjects. The detection results were compared with those obtained from both traditional manual measurement and the well known normalized cross-correlation method, and the effect of the size of tracking window on detection performance was evaluated as well. It is demonstrated that the proposed method agrees well with the manual measurement. Meanwhile, it is not only sensitive to relatively small changes of MT but also computationally efficient.]]></abstract>
##     <issn><![CDATA[2168-2194]]></issn>
##     <htmlFlag><![CDATA[1]]></htmlFlag>
##     <arnumber><![CDATA[6570488]]></arnumber>
##     <doi><![CDATA[10.1109/JBHI.2013.2275002]]></doi>
##     <publicationId><![CDATA[6570488]]></publicationId>
##     <mdurl><![CDATA[http://ieeexplore.ieee.org/xpl/articleDetails.jsp?tp=&arnumber=6570488&contentType=Journals+%26+Magazines]]></mdurl>
##     <pdf><![CDATA[http://ieeexplore.ieee.org/stamp/stamp.jsp?arnumber=6570488]]></pdf>
##   </document>
##   <document>
##     <rank>710</rank>
##     <title><![CDATA[A Transmission Model of Analog Signals in Photonic Links]]></title>
##     <authors><![CDATA[Zhiyu Chen;  Lianshan Yan;  Wei Pan;  Bin Luo;  Xihua Zou;  Hengyun Jiang]]></authors>
##     <affiliations><![CDATA[Center for Inf. Photonics & Commun., Southwest Jiaotong Univ., Chengdu, China]]></affiliations>
##     <controlledterms>
##       <term><![CDATA[analogue circuits]]></term>
##       <term><![CDATA[optical Kerr effect]]></term>
##       <term><![CDATA[optical fibre communication]]></term>
##       <term><![CDATA[optical fibre dispersion]]></term>
##       <term><![CDATA[optical fibre polarisation]]></term>
##     </controlledterms>
##     <thesaurusterms>
##       <term><![CDATA[Equations]]></term>
##       <term><![CDATA[Optical fiber communication]]></term>
##       <term><![CDATA[Optical fiber dispersion]]></term>
##       <term><![CDATA[Optical fiber polarization]]></term>
##       <term><![CDATA[Optical fiber theory]]></term>
##       <term><![CDATA[Photonics]]></term>
##     </thesaurusterms>
##     <pubtitle><![CDATA[Photonics Journal, IEEE]]></pubtitle>
##     <punumber><![CDATA[4563994]]></punumber>
##     <pubtype><![CDATA[Journals & Magazines]]></pubtype>
##     <publisher><![CDATA[IEEE]]></publisher>
##     <volume><![CDATA[6]]></volume>
##     <issue><![CDATA[6]]></issue>
##     <py><![CDATA[2014]]></py>
##     <spage><![CDATA[1]]></spage>
##     <epage><![CDATA[13]]></epage>
##     <abstract><![CDATA[We investigate the transmission of analog photonic signals based on coupled-mode theory and small-signal analysis and provide general expressions for signal evolution along standard single-mode-fiber links. Such model consists of a set of terms that correspond to polarization effects, chromatic dispersion (CD), and fiber nonlinearity. Based on the proposed complete model, we further investigate two typical cases, including the interaction between polarization effects and CD, as well as CD and Kerr effects. In addition, the limitations of the noise figure and spur-free dynamic range are investigated in dispersive nonlinear links.]]></abstract>
##     <issn><![CDATA[1943-0655]]></issn>
##     <htmlFlag><![CDATA[1]]></htmlFlag>
##     <arnumber><![CDATA[6945242]]></arnumber>
##     <doi><![CDATA[10.1109/JPHOT.2014.2366162]]></doi>
##     <publicationId><![CDATA[6945242]]></publicationId>
##     <mdurl><![CDATA[http://ieeexplore.ieee.org/xpl/articleDetails.jsp?tp=&arnumber=6945242&contentType=Journals+%26+Magazines]]></mdurl>
##     <pdf><![CDATA[http://ieeexplore.ieee.org/stamp/stamp.jsp?arnumber=6945242]]></pdf>
##   </document>
##   <document>
##     <rank>711</rank>
##     <title><![CDATA[Nonlinearity Cancellation in Fiber Optic Links Based on Frequency Referenced Carriers]]></title>
##     <authors><![CDATA[Alic, N.;  Myslivets, E.;  Temprana, E.;  Kuo, B.P.-P.;  Radic, S.]]></authors>
##     <affiliations><![CDATA[Qualcomm Inst., Univ. of California San Diego, La Jolla, CA, USA]]></affiliations>
##     <controlledterms>
##       <term><![CDATA[nonlinear optics]]></term>
##       <term><![CDATA[optical fibre communication]]></term>
##       <term><![CDATA[optical links]]></term>
##       <term><![CDATA[optical receivers]]></term>
##       <term><![CDATA[optical transmitters]]></term>
##       <term><![CDATA[stochastic processes]]></term>
##     </controlledterms>
##     <thesaurusterms>
##       <term><![CDATA[Nonlinear optics]]></term>
##       <term><![CDATA[Optical fiber communication]]></term>
##       <term><![CDATA[Optical fiber dispersion]]></term>
##       <term><![CDATA[Optical fibers]]></term>
##       <term><![CDATA[Standards]]></term>
##       <term><![CDATA[Wavelength division multiplexing]]></term>
##     </thesaurusterms>
##     <pubtitle><![CDATA[Lightwave Technology, Journal of]]></pubtitle>
##     <punumber><![CDATA[50]]></punumber>
##     <pubtype><![CDATA[Journals & Magazines]]></pubtype>
##     <publisher><![CDATA[IEEE]]></publisher>
##     <volume><![CDATA[32]]></volume>
##     <issue><![CDATA[15]]></issue>
##     <py><![CDATA[2014]]></py>
##     <spage><![CDATA[2690]]></spage>
##     <epage><![CDATA[2698]]></epage>
##     <abstract><![CDATA[We study the limitations and their origins in the nonlinear effects mitigation in fiber-optic communication systems. The carrier frequencies uncertainty and their stochastic variations are identified as the major impeding factor for successful inter-channel nonlinear impairments management. Furthermore, the results clearly point out to the significant benefits of employing fully frequency referenced carriers in transmission, with frequency combs representing an immediately available solution. Finally, frequency referenced transmitters and/or receivers are shown as critical for availing longer reach at high spectral efficiencies in transmission.]]></abstract>
##     <issn><![CDATA[0733-8724]]></issn>
##     <htmlFlag><![CDATA[1]]></htmlFlag>
##     <arnumber><![CDATA[6840964]]></arnumber>
##     <doi><![CDATA[10.1109/JLT.2014.2332234]]></doi>
##     <publicationId><![CDATA[6840964]]></publicationId>
##     <mdurl><![CDATA[http://ieeexplore.ieee.org/xpl/articleDetails.jsp?tp=&arnumber=6840964&contentType=Journals+%26+Magazines]]></mdurl>
##     <pdf><![CDATA[http://ieeexplore.ieee.org/stamp/stamp.jsp?arnumber=6840964]]></pdf>
##   </document>
##   <document>
##     <rank>712</rank>
##     <title><![CDATA[Microwave Photonics Filtering Technique for Interrogating a Very-Weak Fiber Bragg Grating Cascade Sensor]]></title>
##     <authors><![CDATA[Ricchiuti, A.L.;  Hervas, J.;  Barrera, D.;  Sales, S.;  Capmany, J.]]></authors>
##     <affiliations><![CDATA[ITEAM Res. Inst., Univ. Politec. de Valencia, Valencia, Spain]]></affiliations>
##     <controlledterms>
##       <term><![CDATA[Bragg gratings]]></term>
##       <term><![CDATA[distributed sensors]]></term>
##       <term><![CDATA[fibre optic sensors]]></term>
##       <term><![CDATA[microwave photonics]]></term>
##       <term><![CDATA[optical fibre filters]]></term>
##       <term><![CDATA[photodetectors]]></term>
##     </controlledterms>
##     <thesaurusterms>
##       <term><![CDATA[Bragg gratings]]></term>
##       <term><![CDATA[Fiber gratings]]></term>
##       <term><![CDATA[Microwave filters]]></term>
##       <term><![CDATA[Microwave photonics]]></term>
##       <term><![CDATA[Optical fiber sensors]]></term>
##       <term><![CDATA[Optical filters]]></term>
##     </thesaurusterms>
##     <pubtitle><![CDATA[Photonics Journal, IEEE]]></pubtitle>
##     <punumber><![CDATA[4563994]]></punumber>
##     <pubtype><![CDATA[Journals & Magazines]]></pubtype>
##     <publisher><![CDATA[IEEE]]></publisher>
##     <volume><![CDATA[6]]></volume>
##     <issue><![CDATA[6]]></issue>
##     <py><![CDATA[2014]]></py>
##     <spage><![CDATA[1]]></spage>
##     <epage><![CDATA[10]]></epage>
##     <abstract><![CDATA[A system to interrogate photonic sensors based on a very weak fiber Bragg grating cascade fiber is presented and experimentally validated and dedicated to detecting the presence and location of a spot event. The distributed sensor proposed consists of a 5-m-long fiber, containing 500 9-mm-long Bragg gratings with a grating separation of 10.21 mm. The principle of operation is based on a technique used to analyze microwave photonics filters. The detection of spot events along the sensor is demonstrated with remarkable accuracy under 1 mm, using a photodetector and a modulator with a bandwidth of only 500 MHz. The simple proposed scheme is intrinsically robust against environmental changes and is easy to reconfigure.]]></abstract>
##     <issn><![CDATA[1943-0655]]></issn>
##     <htmlFlag><![CDATA[1]]></htmlFlag>
##     <arnumber><![CDATA[6928450]]></arnumber>
##     <doi><![CDATA[10.1109/JPHOT.2014.2363443]]></doi>
##     <publicationId><![CDATA[6928450]]></publicationId>
##     <mdurl><![CDATA[http://ieeexplore.ieee.org/xpl/articleDetails.jsp?tp=&arnumber=6928450&contentType=Journals+%26+Magazines]]></mdurl>
##     <pdf><![CDATA[http://ieeexplore.ieee.org/stamp/stamp.jsp?arnumber=6928450]]></pdf>
##   </document>
##   <document>
##     <rank>713</rank>
##     <title><![CDATA[Position Control of Electric Clutch Actuator Using a Triple-Step Nonlinear Method]]></title>
##     <authors><![CDATA[Bingzhao Gao;  Hong Chen;  Qifang Liu;  Hongqing Chu]]></authors>
##     <affiliations><![CDATA[State Key Lab. of Automotive Simulation & Control, Jilin Univ., Changchun, China]]></affiliations>
##     <controlledterms>
##       <term><![CDATA[automobiles]]></term>
##       <term><![CDATA[clutches]]></term>
##       <term><![CDATA[control system synthesis]]></term>
##       <term><![CDATA[electric actuators]]></term>
##       <term><![CDATA[feedback]]></term>
##       <term><![CDATA[feedforward]]></term>
##       <term><![CDATA[nonlinear control systems]]></term>
##       <term><![CDATA[position control]]></term>
##       <term><![CDATA[three-term control]]></term>
##     </controlledterms>
##     <thesaurusterms>
##       <term><![CDATA[Actuators]]></term>
##       <term><![CDATA[Automotive engineering]]></term>
##       <term><![CDATA[Nonlinear control systems]]></term>
##       <term><![CDATA[Position control]]></term>
##       <term><![CDATA[Steady-state]]></term>
##       <term><![CDATA[Vehicle dynamics]]></term>
##     </thesaurusterms>
##     <pubtitle><![CDATA[Industrial Electronics, IEEE Transactions on]]></pubtitle>
##     <punumber><![CDATA[41]]></punumber>
##     <pubtype><![CDATA[Journals & Magazines]]></pubtype>
##     <publisher><![CDATA[IEEE]]></publisher>
##     <volume><![CDATA[61]]></volume>
##     <issue><![CDATA[12]]></issue>
##     <py><![CDATA[2014]]></py>
##     <spage><![CDATA[6995]]></spage>
##     <epage><![CDATA[7003]]></epage>
##     <abstract><![CDATA[For a novel electric clutch actuator, a nonlinear feedforward-feedback control scheme is proposed to improve the performance of the position tracking control. The design procedure is formalized as a triple-step deduction, and the derived controller consists of three parts: steady-state-like control; feedforward control based on reference dynamics; and state-dependent feedback control. The structure of the proposed nonlinear controller is concise and is also comparable to those widely used in modern automotive control. Finally, the designed controller is evaluated through simulations and experimental tests, which show that the proposed controller satisfied the control requirement. Comparison with proportional-integral-derivative control is given as well.]]></abstract>
##     <issn><![CDATA[0278-0046]]></issn>
##     <htmlFlag><![CDATA[1]]></htmlFlag>
##     <arnumber><![CDATA[6797898]]></arnumber>
##     <doi><![CDATA[10.1109/TIE.2014.2317131]]></doi>
##     <publicationId><![CDATA[6797898]]></publicationId>
##     <mdurl><![CDATA[http://ieeexplore.ieee.org/xpl/articleDetails.jsp?tp=&arnumber=6797898&contentType=Journals+%26+Magazines]]></mdurl>
##     <pdf><![CDATA[http://ieeexplore.ieee.org/stamp/stamp.jsp?arnumber=6797898]]></pdf>
##   </document>
##   <document>
##     <rank>714</rank>
##     <title><![CDATA[Efficiently Representing Membershipfor Variable Large Data Sets]]></title>
##     <authors><![CDATA[Jiansheng Wei;  Hong Jiang;  Ke Zhou;  Dan Feng]]></authors>
##     <affiliations><![CDATA[Sch. of Comput. Sci. & Technol., Huazhong Univ. of Sci. & Technol., Wuhan, China]]></affiliations>
##     <controlledterms>
##       <term><![CDATA[cloud computing]]></term>
##       <term><![CDATA[data handling]]></term>
##       <term><![CDATA[data structures]]></term>
##       <term><![CDATA[query processing]]></term>
##     </controlledterms>
##     <thesaurusterms>
##       <term><![CDATA[Arrays]]></term>
##       <term><![CDATA[Distributed databases]]></term>
##       <term><![CDATA[Error analysis]]></term>
##       <term><![CDATA[Indexes]]></term>
##       <term><![CDATA[Peer-to-peer computing]]></term>
##       <term><![CDATA[Random access memory]]></term>
##       <term><![CDATA[Servers]]></term>
##     </thesaurusterms>
##     <pubtitle><![CDATA[Parallel and Distributed Systems, IEEE Transactions on]]></pubtitle>
##     <punumber><![CDATA[71]]></punumber>
##     <pubtype><![CDATA[Journals & Magazines]]></pubtype>
##     <publisher><![CDATA[IEEE]]></publisher>
##     <volume><![CDATA[25]]></volume>
##     <issue><![CDATA[4]]></issue>
##     <py><![CDATA[2014]]></py>
##     <spage><![CDATA[960]]></spage>
##     <epage><![CDATA[970]]></epage>
##     <abstract><![CDATA[Cloud computing has raised new challenges for the membership representation scheme of storage systems that manage very large data sets. This paper proposes DBA, a dynamic Bloom filter array aimed at representing membership for variable large data sets in storage systems in a scalable way. DBA consists of dynamically created groups of space-efficient Bloom filters (BFs) to accommodate changes in set sizes. Within a group, BFs are homogeneous and the data layout is optimized at the bit level to enable parallel access and thus achieve high query performance. DBA can effectively control its query accuracy by partially adjusting the error rate of the constructing BFs, where each BF only represents an independent subset to help locate elements and confirm membership. Further, DBA supports element deletion by introducing a lazy update policy. We prototype and evaluate our DBA scheme as a scalable fast index in the MAD2 deduplication storage system. Experimental results reveal that DBA (with 64 BFs per group) shows significantly higher query performance than the state-of-the-art approach while scaling up to 160 BFs. DBA is also shown to excel in scalability, query accuracy, and space efficiency by theoretical analysis and experimental evaluation.]]></abstract>
##     <issn><![CDATA[1045-9219]]></issn>
##     <htmlFlag><![CDATA[1]]></htmlFlag>
##     <arnumber><![CDATA[6471979]]></arnumber>
##     <doi><![CDATA[10.1109/TPDS.2013.66]]></doi>
##     <publicationId><![CDATA[6471979]]></publicationId>
##     <mdurl><![CDATA[http://ieeexplore.ieee.org/xpl/articleDetails.jsp?tp=&arnumber=6471979&contentType=Journals+%26+Magazines]]></mdurl>
##     <pdf><![CDATA[http://ieeexplore.ieee.org/stamp/stamp.jsp?arnumber=6471979]]></pdf>
##   </document>
##   <document>
##     <rank>715</rank>
##     <title><![CDATA[Tunable CMOS Delay Gate With Improved Matching Properties]]></title>
##     <authors><![CDATA[Mroszczyk, P.;  Dudek, P.]]></authors>
##     <affiliations><![CDATA[Sch. of Electr. & Electron. Eng., Univ. of Manchester, Manchester, UK]]></affiliations>
##     <controlledterms>
##       <term><![CDATA[CMOS logic circuits]]></term>
##       <term><![CDATA[MOSFET circuits]]></term>
##       <term><![CDATA[current limiters]]></term>
##       <term><![CDATA[delay lines]]></term>
##       <term><![CDATA[logic gates]]></term>
##     </controlledterms>
##     <thesaurusterms>
##       <term><![CDATA[Capacitance]]></term>
##       <term><![CDATA[Delays]]></term>
##       <term><![CDATA[Discharges (electric)]]></term>
##       <term><![CDATA[Limiting]]></term>
##       <term><![CDATA[Logic gates]]></term>
##       <term><![CDATA[Open systems]]></term>
##       <term><![CDATA[Transistors]]></term>
##     </thesaurusterms>
##     <pubtitle><![CDATA[Circuits and Systems I: Regular Papers, IEEE Transactions on]]></pubtitle>
##     <punumber><![CDATA[8919]]></punumber>
##     <pubtype><![CDATA[Journals & Magazines]]></pubtype>
##     <publisher><![CDATA[IEEE]]></publisher>
##     <volume><![CDATA[61]]></volume>
##     <issue><![CDATA[9]]></issue>
##     <py><![CDATA[2014]]></py>
##     <spage><![CDATA[2586]]></spage>
##     <epage><![CDATA[2595]]></epage>
##     <abstract><![CDATA[This paper presents the analysis and design of a tunable CMOS delay gate with improved matching properties as compared with the commonly used &#x201C;current starved inverter&#x201D; (CSI). The main difference between these structures lies in the location of the current limiting transistor on the inverter's output rather than on the side of the power rail. This improves the dynamic performance of the proposed &#x201C;output split inverter&#x201D; (OSI) circuit reducing the influence of the MOS transistor mismatch on the generated delay time variability. A test chip including two arrays consisting of 512 16-stage delay lines employing the CSI and OSI gates has been designed and fabricated in a standard 90 nm CMOS technology. The experimental results show that the proposed OSI circuit is about 10-50% more accurate than the conventional current starved inverter with no penalty in terms of the increased area, power consumption or complexity. Applications of the proposed circuit are in the design of time-to-digital converters (TDCs), delay locked loops, readout circuits for particle detection and time-based asynchronous computation systems.]]></abstract>
##     <issn><![CDATA[1549-8328]]></issn>
##     <htmlFlag><![CDATA[1]]></htmlFlag>
##     <arnumber><![CDATA[6807526]]></arnumber>
##     <doi><![CDATA[10.1109/TCSI.2014.2312491]]></doi>
##     <publicationId><![CDATA[6807526]]></publicationId>
##     <mdurl><![CDATA[http://ieeexplore.ieee.org/xpl/articleDetails.jsp?tp=&arnumber=6807526&contentType=Journals+%26+Magazines]]></mdurl>
##     <pdf><![CDATA[http://ieeexplore.ieee.org/stamp/stamp.jsp?arnumber=6807526]]></pdf>
##   </document>
##   <document>
##     <rank>716</rank>
##     <title><![CDATA[Processing Time Analysis and Estimation for 3-D Applications]]></title>
##     <authors><![CDATA[Issa, J.A.]]></authors>
##     <affiliations><![CDATA[Dept. of Electr. & Comput. Eng., Notre Dame Univ., Zouk Mosbeh, Lebanon]]></affiliations>
##     <controlledterms>
##       <term><![CDATA[graphics processing units]]></term>
##       <term><![CDATA[performance evaluation]]></term>
##       <term><![CDATA[regression analysis]]></term>
##     </controlledterms>
##     <thesaurusterms>
##       <term><![CDATA[Benchmark testing]]></term>
##       <term><![CDATA[Frequency estimation]]></term>
##       <term><![CDATA[Graphics processing units]]></term>
##       <term><![CDATA[Three dimensional displays]]></term>
##       <term><![CDATA[Time measurement]]></term>
##       <term><![CDATA[Time-frequency analysis]]></term>
##     </thesaurusterms>
##     <pubtitle><![CDATA[Access, IEEE]]></pubtitle>
##     <punumber><![CDATA[6287639]]></punumber>
##     <pubtype><![CDATA[Journals & Magazines]]></pubtype>
##     <publisher><![CDATA[IEEE]]></publisher>
##     <volume><![CDATA[2]]></volume>
##     <py><![CDATA[2014]]></py>
##     <spage><![CDATA[1177]]></spage>
##     <epage><![CDATA[1186]]></epage>
##     <abstract><![CDATA[The performance analysis of a graphics processing unit (GPU) is important for analyzing and fine tuning current and future graphics processors as well as for comparing the performance of different architectures. In this paper, we present an analytical model to calculate the total time it takes for a GPU to retire one frame on a given benchmark. The model also estimates the total retirement time for the same frame on a different GPU using regression estimation model. The model consists of two stages. The first stage entails establishing the measured baseline for a specific frame on a given graphics card, and the second stage entails adjusting the measured baseline and estimating the time it takes to process all draw calls for the same frame on a different graphics card. The model considers the impact of pipeline bottlenecks to process a specific frame, estimates the minimum time it takes to process that frame, and reparameterize the baseline for a different graphics card to calculate new frame retirement times at two different memory frequencies. We used Amdahl's law model to estimate frame retirement time for a different graphics card at higher memory frequencies based on the new adjusted measured baseline with error margin is &lt;;5%.]]></abstract>
##     <issn><![CDATA[2169-3536]]></issn>
##     <htmlFlag><![CDATA[1]]></htmlFlag>
##     <arnumber><![CDATA[6924756]]></arnumber>
##     <doi><![CDATA[10.1109/ACCESS.2014.2363113]]></doi>
##     <publicationId><![CDATA[6924756]]></publicationId>
##     <mdurl><![CDATA[http://ieeexplore.ieee.org/xpl/articleDetails.jsp?tp=&arnumber=6924756&contentType=Journals+%26+Magazines]]></mdurl>
##     <pdf><![CDATA[http://ieeexplore.ieee.org/stamp/stamp.jsp?arnumber=6924756]]></pdf>
##   </document>
##   <document>
##     <rank>717</rank>
##     <title><![CDATA[SMoW: An Energy-Bandwidth Aware Web Browsing Technique for Smartphones]]></title>
##     <authors><![CDATA[Albasir, A.A.;  Naik, K.]]></authors>
##     <affiliations><![CDATA[Dept. of Electr. & Comput. Eng., Univ. of Waterloo, Waterloo, ON, Canada]]></affiliations>
##     <controlledterms>
##       <term><![CDATA[Web sites]]></term>
##       <term><![CDATA[advertising]]></term>
##       <term><![CDATA[online front-ends]]></term>
##       <term><![CDATA[power aware computing]]></term>
##       <term><![CDATA[smart phones]]></term>
##     </controlledterms>
##     <thesaurusterms>
##       <term><![CDATA[Browsers]]></term>
##       <term><![CDATA[Computer applications]]></term>
##       <term><![CDATA[Energy efficiency]]></term>
##       <term><![CDATA[Mobile communication]]></term>
##       <term><![CDATA[Smart phones]]></term>
##       <term><![CDATA[Web and internet services]]></term>
##     </thesaurusterms>
##     <pubtitle><![CDATA[Access, IEEE]]></pubtitle>
##     <punumber><![CDATA[6287639]]></punumber>
##     <pubtype><![CDATA[Journals & Magazines]]></pubtype>
##     <publisher><![CDATA[IEEE]]></publisher>
##     <volume><![CDATA[2]]></volume>
##     <py><![CDATA[2014]]></py>
##     <spage><![CDATA[1427]]></spage>
##     <epage><![CDATA[1441]]></epage>
##     <abstract><![CDATA[With the rapid advancement of mobile devices, people have become more attached to them than ever. This rapid growth combined with millions of applications (apps) make smart phones a favorite means of communication among users. In general, the available contents on smart phones, apps, and Web, come in two versions: (1) free content that is monetized via advertisements (ads) and (2) paid content that is monetized by user subscription fees. However, the resources, namely, energy, bandwidth, and processing power, on-board are limited, and the existence of ads in Web sites and free apps can significantly increase the usage of these resources. These issues necessitate a good understanding of the mobile advertising eco-system and how such limited resources can be efficiently used. In this paper, we present the results of a novel Web browsing technique that adapts the Web pages delivered to smart phone, based on the smart phone's current battery level and the network type. Web pages are adapted by controlling the amount of ads to be displayed. Validation tests confirm that the system can extend smart phone battery life by up to 30% and save wireless bandwidth up to ~44%.]]></abstract>
##     <issn><![CDATA[2169-3536]]></issn>
##     <htmlFlag><![CDATA[1]]></htmlFlag>
##     <arnumber><![CDATA[6936284]]></arnumber>
##     <doi><![CDATA[10.1109/ACCESS.2014.2365091]]></doi>
##     <publicationId><![CDATA[6936284]]></publicationId>
##     <mdurl><![CDATA[http://ieeexplore.ieee.org/xpl/articleDetails.jsp?tp=&arnumber=6936284&contentType=Journals+%26+Magazines]]></mdurl>
##     <pdf><![CDATA[http://ieeexplore.ieee.org/stamp/stamp.jsp?arnumber=6936284]]></pdf>
##   </document>
##   <document>
##     <rank>718</rank>
##     <title><![CDATA[FILOSE for Svenning: A Flow Sensing Bioinspired Robot]]></title>
##     <authors><![CDATA[Kruusmaa, M.;  Fiorini, P.;  Megill, W.;  De Vittorio, M.;  Akanyeti, O.;  Visentin, F.;  Chambers, L.;  El Daou, H.;  Fiazza, M.-C.;  Jezov, J.;  Listak, M.;  Rossi, L.;  Salumae, T.;  Toming, G.;  Venturelli, R.;  Jung, D.S.;  Brown, J.;  Rizzi, F.;  Qualtieri, A.;  Maud, J.L.;  Liszewski, A.]]></authors>
##     <affiliations><![CDATA[Tallinn Univ. of Technol., Tallinn, Estonia]]></affiliations>
##     <controlledterms>
##       <term><![CDATA[biomimetics]]></term>
##       <term><![CDATA[flow control]]></term>
##       <term><![CDATA[marine engineering]]></term>
##       <term><![CDATA[mobile robots]]></term>
##     </controlledterms>
##     <thesaurusterms>
##       <term><![CDATA[Biomimetics]]></term>
##       <term><![CDATA[Hydrodynamics]]></term>
##       <term><![CDATA[Marine animals]]></term>
##       <term><![CDATA[Mathematical model]]></term>
##       <term><![CDATA[Robot sensing systems]]></term>
##       <term><![CDATA[Underwater equipment]]></term>
##     </thesaurusterms>
##     <pubtitle><![CDATA[Robotics & Automation Magazine, IEEE]]></pubtitle>
##     <punumber><![CDATA[100]]></punumber>
##     <pubtype><![CDATA[Journals & Magazines]]></pubtype>
##     <publisher><![CDATA[IEEE]]></publisher>
##     <volume><![CDATA[21]]></volume>
##     <issue><![CDATA[3]]></issue>
##     <py><![CDATA[2014]]></py>
##     <spage><![CDATA[51]]></spage>
##     <epage><![CDATA[62]]></epage>
##     <abstract><![CDATA[The trend of biomimetic underwater robots has emerged as a search for an alternative to traditional propeller-driven underwater vehicles. The drive of this trend, as in any other areas of bioinspired and biomimetic robotics, is the belief that exploiting solutions that evolution has already optimized leads to more advanced technologies and devices. In underwater robotics, bioinspired design is expected to offer more energy-efficient, highly maneuverable, agile, robust, and stable underwater robots. The 30,000 fish species have inspired roboticists to mimic tuna [1], rays [2], boxfish [3], eels [4], and others. The development of the first commercialized fish robot Ghostswimmer by Boston Engineering and the development of fish robots for field trials with specific applications in mind (http://www.roboshoal. com) mark a new degree of maturity of this engineering discipline after decades of laboratory trials.]]></abstract>
##     <issn><![CDATA[1070-9932]]></issn>
##     <arnumber><![CDATA[6894715]]></arnumber>
##     <doi><![CDATA[10.1109/MRA.2014.2322287]]></doi>
##     <publicationId><![CDATA[6894715]]></publicationId>
##     <mdurl><![CDATA[http://ieeexplore.ieee.org/xpl/articleDetails.jsp?tp=&arnumber=6894715&contentType=Journals+%26+Magazines]]></mdurl>
##     <pdf><![CDATA[http://ieeexplore.ieee.org/stamp/stamp.jsp?arnumber=6894715]]></pdf>
##   </document>
##   <document>
##     <rank>719</rank>
##     <title><![CDATA[Analysis of GNSS Performance Index Using Feature Points of Sky-View Image]]></title>
##     <authors><![CDATA[Woonki Hong;  Kwangsik Choi;  Eunsung Lee;  Sunghyuck Im;  Moonbeom Heo]]></authors>
##     <affiliations><![CDATA[Korea Aerosp. Res. Inst., Daejeon, South Korea]]></affiliations>
##     <controlledterms>
##       <term><![CDATA[feature extraction]]></term>
##       <term><![CDATA[satellite navigation]]></term>
##     </controlledterms>
##     <thesaurusterms>
##       <term><![CDATA[Global Navigation Satellite Systems]]></term>
##       <term><![CDATA[Image edge detection]]></term>
##       <term><![CDATA[Land transportation]]></term>
##       <term><![CDATA[Lenses]]></term>
##       <term><![CDATA[Performance analysis]]></term>
##       <term><![CDATA[Satellites]]></term>
##     </thesaurusterms>
##     <pubtitle><![CDATA[Intelligent Transportation Systems, IEEE Transactions on]]></pubtitle>
##     <punumber><![CDATA[6979]]></punumber>
##     <pubtype><![CDATA[Journals & Magazines]]></pubtype>
##     <publisher><![CDATA[IEEE]]></publisher>
##     <volume><![CDATA[15]]></volume>
##     <issue><![CDATA[2]]></issue>
##     <py><![CDATA[2014]]></py>
##     <spage><![CDATA[889]]></spage>
##     <epage><![CDATA[895]]></epage>
##     <abstract><![CDATA[When sky-view factor (SVF) is used to predict the positioning performance of the global navigation satellite system (GNSS), it is easy to use the SVF as a performance index without a specific database, as it is used for topographic maps, not only in an open-sky land but also in regions where there are many tall buildings. However, conventional SVF is only able to express the degree of openness of the sky as a ratio, and it is limited to being used as a performance index for the positioning that uses the GNSS. When the conventional SVF is used in a land transportation environment, the predicted value for the positioning performance of the GNSS is often not consistent with the actual positioning error, but when sky-view-based dilution of precision (SVDOP) is applied, we confirmed that it was substantially close to the actual positioning error. This confirms our expectation that the utilization of the proposed method rather than the utilization of SVF alone in land transportation environments will make the analysis easier. In this paper, the SVDOP is calculated with real Global Positioning System data, and its usefulness is validated by comparing it with the conventional SVF and the DOP.]]></abstract>
##     <issn><![CDATA[1524-9050]]></issn>
##     <arnumber><![CDATA[6648430]]></arnumber>
##     <doi><![CDATA[10.1109/TITS.2013.2282631]]></doi>
##     <publicationId><![CDATA[6648430]]></publicationId>
##     <mdurl><![CDATA[http://ieeexplore.ieee.org/xpl/articleDetails.jsp?tp=&arnumber=6648430&contentType=Journals+%26+Magazines]]></mdurl>
##     <pdf><![CDATA[http://ieeexplore.ieee.org/stamp/stamp.jsp?arnumber=6648430]]></pdf>
##   </document>
##   <document>
##     <rank>720</rank>
##     <title><![CDATA[A Hybrid Integrated Light Source on a Silicon Platform Using a Trident Spot-Size Converter]]></title>
##     <authors><![CDATA[Hatori, N.;  Shimizu, T.;  Okano, M.;  Ishizaka, M.;  Yamamoto, T.;  Urino, Y.;  Mori, M.;  Nakamura, T.;  Arakawa, Y.]]></authors>
##     <affiliations><![CDATA[Photonics Electron. Technol. Res. Assoc. (PETRA), Tsukuba, Japan]]></affiliations>
##     <controlledterms>
##       <term><![CDATA[elemental semiconductors]]></term>
##       <term><![CDATA[integrated optoelectronics]]></term>
##       <term><![CDATA[optical fabrication]]></term>
##       <term><![CDATA[optical losses]]></term>
##       <term><![CDATA[optical planar waveguides]]></term>
##       <term><![CDATA[quantum dot lasers]]></term>
##       <term><![CDATA[quantum well lasers]]></term>
##       <term><![CDATA[silicon]]></term>
##     </controlledterms>
##     <thesaurusterms>
##       <term><![CDATA[Couplings]]></term>
##       <term><![CDATA[Light sources]]></term>
##       <term><![CDATA[Loss measurement]]></term>
##       <term><![CDATA[Optical device fabrication]]></term>
##       <term><![CDATA[Optical losses]]></term>
##       <term><![CDATA[Optical waveguides]]></term>
##       <term><![CDATA[Silicon]]></term>
##     </thesaurusterms>
##     <pubtitle><![CDATA[Lightwave Technology, Journal of]]></pubtitle>
##     <punumber><![CDATA[50]]></punumber>
##     <pubtype><![CDATA[Journals & Magazines]]></pubtype>
##     <publisher><![CDATA[IEEE]]></publisher>
##     <volume><![CDATA[32]]></volume>
##     <issue><![CDATA[7]]></issue>
##     <py><![CDATA[2014]]></py>
##     <spage><![CDATA[1329]]></spage>
##     <epage><![CDATA[1336]]></epage>
##     <abstract><![CDATA[This paper reports a hybrid integrated light source fabricated on a Si platform using a spot-size converter (SSC) with a trident Si waveguide. Low-loss coupling for 1.55 &#x03BC;m and 1.3 &#x03BC;m wavelengths was achieved with merely the simple planar form of a Si waveguide with no use of complicated structures such as vertical tapers or an extra dielectric core overlaid on the waveguide. The coupling loss tolerance up to a 1 dB loss increase was larger than the accuracy of our passive alignment technology. The coupling efficiency was quite robust against manufacturing variations in the waveguide width compared with that of a conventional SSC with an inverse taper waveguide. A multi-channel light source with highly uniform output power and a high-temperature light source were fabricated with a 1.55 &#x03BC;m quantum well laser and a 1.3 &#x03BC;m quantum dot laser, respectively. The integration scheme we report can be used to fabricate light sources for high-density, multi-channel Si optical interposers.]]></abstract>
##     <issn><![CDATA[0733-8724]]></issn>
##     <htmlFlag><![CDATA[1]]></htmlFlag>
##     <arnumber><![CDATA[6733299]]></arnumber>
##     <doi><![CDATA[10.1109/JLT.2014.2304305]]></doi>
##     <publicationId><![CDATA[6733299]]></publicationId>
##     <mdurl><![CDATA[http://ieeexplore.ieee.org/xpl/articleDetails.jsp?tp=&arnumber=6733299&contentType=Journals+%26+Magazines]]></mdurl>
##     <pdf><![CDATA[http://ieeexplore.ieee.org/stamp/stamp.jsp?arnumber=6733299]]></pdf>
##   </document>
##   <document>
##     <rank>721</rank>
##     <title><![CDATA[Intermittent Electrical Contact Resistance as a Contributory Factor in the Loss of Automobile Speed Control Functional Integrity]]></title>
##     <authors><![CDATA[Anderson, A.F.]]></authors>
##     <controlledterms>
##       <term><![CDATA[automobiles]]></term>
##       <term><![CDATA[automotive electronics]]></term>
##       <term><![CDATA[electrical contacts]]></term>
##       <term><![CDATA[road safety]]></term>
##       <term><![CDATA[velocimeters]]></term>
##       <term><![CDATA[velocity control]]></term>
##     </controlledterms>
##     <thesaurusterms>
##       <term><![CDATA[Acceleration]]></term>
##       <term><![CDATA[Automotive electronics]]></term>
##       <term><![CDATA[Electric contact resistance]]></term>
##       <term><![CDATA[Engines]]></term>
##       <term><![CDATA[Fault diagnosis]]></term>
##       <term><![CDATA[Velocity control]]></term>
##     </thesaurusterms>
##     <pubtitle><![CDATA[Access, IEEE]]></pubtitle>
##     <punumber><![CDATA[6287639]]></punumber>
##     <pubtype><![CDATA[Journals & Magazines]]></pubtype>
##     <publisher><![CDATA[IEEE]]></publisher>
##     <volume><![CDATA[2]]></volume>
##     <py><![CDATA[2014]]></py>
##     <spage><![CDATA[258]]></spage>
##     <epage><![CDATA[289]]></epage>
##     <abstract><![CDATA[For three decades, sudden acceleration (SA) incidents have been reported, where automobiles accelerate without warning. These incidents are often diagnosed as no fault found. Investigators, who follow the line of diagnostic reasoning from the 1989 National Highway Traffic Safety Administration (NHTSA) SA report, tend to conclude that SAs are caused by driver pedal error. This paper reviews the diagnostic process in the NHTSA report and finds that: 1) it assumes that an intermittent electronic malfunction should be reproducible either through in-vehicle or laboratory bench tests without saying why and 2) the consequence of this assumption, for which there appears to be no forensic precedent, is to recategorize possible intermittent electronic failures as proven to be nonelectronic. Showing that the supposedly inescapable conclusions of the NHTSA report concerning electronic malfunctions are without foundation opens the way for this paper to discuss electronic intermittency as a potential factor in SA incidents. It then reports a simple practical experiment that shows how mechanically induced electrical contact intermittencies can generate false speed signals that an automobile speed control system may accept as true and that do not trigger any diagnostic fault codes. Since the generation of accurate speed signals is essential for the proper functioning of a number of other automobile safety-critical control systems, the apparent ease with which false speed signals can be generated by vibration of a poor electrical contact is obviously a matter of general concern. Various ways of reducing the likelihood of SAs are discussed, including electrical contact improvements to reduce the likelihood of generating false speed signals, improved battery maintenance, and the incorporation of an independent fail-safe that reduces engine power in an emergency, such as a kill switch.]]></abstract>
##     <issn><![CDATA[2169-3536]]></issn>
##     <htmlFlag><![CDATA[1]]></htmlFlag>
##     <arnumber><![CDATA[6777269]]></arnumber>
##     <doi><![CDATA[10.1109/ACCESS.2014.2313296]]></doi>
##     <publicationId><![CDATA[6777269]]></publicationId>
##     <mdurl><![CDATA[http://ieeexplore.ieee.org/xpl/articleDetails.jsp?tp=&arnumber=6777269&contentType=Journals+%26+Magazines]]></mdurl>
##     <pdf><![CDATA[http://ieeexplore.ieee.org/stamp/stamp.jsp?arnumber=6777269]]></pdf>
##   </document>
##   <document>
##     <rank>722</rank>
##     <title><![CDATA[Adaptable Robot Formation Control: Adaptive and Predictive Formation Control of Autonomous Vehicles]]></title>
##     <authors><![CDATA[Guillet, A.;  Lenain, R.;  Thuilot, B.;  Martinet, P.]]></authors>
##     <affiliations><![CDATA[Irstea, Clermont-Ferrand, France]]></affiliations>
##     <controlledterms>
##       <term><![CDATA[adaptive control]]></term>
##       <term><![CDATA[agricultural machinery]]></term>
##       <term><![CDATA[decentralised control]]></term>
##       <term><![CDATA[mobile robots]]></term>
##       <term><![CDATA[position control]]></term>
##       <term><![CDATA[predictive control]]></term>
##       <term><![CDATA[vehicles]]></term>
##     </controlledterms>
##     <thesaurusterms>
##       <term><![CDATA[Autonomous vehicles]]></term>
##       <term><![CDATA[Mathematical model]]></term>
##       <term><![CDATA[Mobile robots]]></term>
##       <term><![CDATA[Navigation]]></term>
##       <term><![CDATA[Predictive methods]]></term>
##       <term><![CDATA[Robot kinematics]]></term>
##     </thesaurusterms>
##     <pubtitle><![CDATA[Robotics & Automation Magazine, IEEE]]></pubtitle>
##     <punumber><![CDATA[100]]></punumber>
##     <pubtype><![CDATA[Journals & Magazines]]></pubtype>
##     <publisher><![CDATA[IEEE]]></publisher>
##     <volume><![CDATA[21]]></volume>
##     <issue><![CDATA[1]]></issue>
##     <py><![CDATA[2014]]></py>
##     <spage><![CDATA[28]]></spage>
##     <epage><![CDATA[39]]></epage>
##     <abstract><![CDATA[The ability to use cooperative small vehicles is of interest in many applications. From material transportation to farming operations, the use of small machines achieving small tasks, but able to work together to complete larger tasks, permits us to rely on a unique kind of vehicle. To be efficient, such a point of view requires the vehicles to be, at least partially, autonomous and their motion must be accurately coordinated for the tasks to be properly achieved. This article proposes a control framework dedicated to the accurate control of a fleet of mobile robots operating in formation. Decentralized control relying on interrobot communication has been favored. To ensure a high relative positioning, adaptive and predictive control techniques are considered, allowing us to account for the influence of several phenomena (such as dynamic perturbations or bad grip conditions) depreciating the relevance of classical approaches based on ideal robots and ideal contact conditions assumptions.]]></abstract>
##     <issn><![CDATA[1070-9932]]></issn>
##     <htmlFlag><![CDATA[1]]></htmlFlag>
##     <arnumber><![CDATA[6740018]]></arnumber>
##     <doi><![CDATA[10.1109/MRA.2013.2295946]]></doi>
##     <publicationId><![CDATA[6740018]]></publicationId>
##     <mdurl><![CDATA[http://ieeexplore.ieee.org/xpl/articleDetails.jsp?tp=&arnumber=6740018&contentType=Journals+%26+Magazines]]></mdurl>
##     <pdf><![CDATA[http://ieeexplore.ieee.org/stamp/stamp.jsp?arnumber=6740018]]></pdf>
##   </document>
##   <document>
##     <rank>723</rank>
##     <title><![CDATA[Calibration-Free Electrical Spectrum Analysis for Microwave Characterization of Optical Phase Modulators Using Frequency-Shifted Heterodyning]]></title>
##     <authors><![CDATA[Shangjian Zhang;  Heng Wang;  Xinhai Zou;  Yali Zhang;  Rongguo Lu;  Yong Liu]]></authors>
##     <affiliations><![CDATA[State Key Lab. of Electron. Thin Films & Integrated Devices, Univ. of Electron. Sci. & Technol. of China, Chengdu, China]]></affiliations>
##     <controlledterms>
##       <term><![CDATA[electro-optical modulation]]></term>
##       <term><![CDATA[heterodyne detection]]></term>
##       <term><![CDATA[microwave photonics]]></term>
##       <term><![CDATA[optical frequency conversion]]></term>
##       <term><![CDATA[phase modulation]]></term>
##     </controlledterms>
##     <thesaurusterms>
##       <term><![CDATA[Frequency measurement]]></term>
##       <term><![CDATA[Frequency modulation]]></term>
##       <term><![CDATA[Optical interferometry]]></term>
##       <term><![CDATA[Optical mixing]]></term>
##       <term><![CDATA[Optical modulation]]></term>
##       <term><![CDATA[Optical variables measurement]]></term>
##       <term><![CDATA[Phase modulation]]></term>
##     </thesaurusterms>
##     <pubtitle><![CDATA[Photonics Journal, IEEE]]></pubtitle>
##     <punumber><![CDATA[4563994]]></punumber>
##     <pubtype><![CDATA[Journals & Magazines]]></pubtype>
##     <publisher><![CDATA[IEEE]]></publisher>
##     <volume><![CDATA[6]]></volume>
##     <issue><![CDATA[4]]></issue>
##     <py><![CDATA[2014]]></py>
##     <spage><![CDATA[1]]></spage>
##     <epage><![CDATA[8]]></epage>
##     <abstract><![CDATA[A novel calibration-free electrical spectrum analysis method for microwave characterization of electrooptic phase modulators is proposed and experimentally demonstrated based on frequency-shifted optical heterodyning. The method achieves the electrical domain measurement of the modulation efficiency of phase modulators without the need for correcting the responsivity fluctuation in the photodetection. Moreover, it extends double the measuring frequency range through setting a specific frequency relationship between the driving microwave signals. Modulation depth and half-wave voltage of phase modulators are experimentally extracted from the heterodyning spectrum of two phase-modulated signals with and without frequency shifting, and the measured results are compared to those obtained with the traditional optical spectrum analysis method to check the consistency. The proposed method provides calibration-free and accurate measurement for high-speed optical phase modulators with the high-resolution electrical spectrum analysis.]]></abstract>
##     <issn><![CDATA[1943-0655]]></issn>
##     <htmlFlag><![CDATA[1]]></htmlFlag>
##     <arnumber><![CDATA[6872812]]></arnumber>
##     <doi><![CDATA[10.1109/JPHOT.2014.2343991]]></doi>
##     <publicationId><![CDATA[6872812]]></publicationId>
##     <mdurl><![CDATA[http://ieeexplore.ieee.org/xpl/articleDetails.jsp?tp=&arnumber=6872812&contentType=Journals+%26+Magazines]]></mdurl>
##     <pdf><![CDATA[http://ieeexplore.ieee.org/stamp/stamp.jsp?arnumber=6872812]]></pdf>
##   </document>
##   <document>
##     <rank>724</rank>
##     <title><![CDATA[Enhanced Spurious-Free Dynamic Range in Intensity-Modulated Analog Photonic Link Using Digital Postprocessing]]></title>
##     <authors><![CDATA[Yan Cui;  Yitang Dai;  Feifei Yin;  Qiang Lv;  Jianqiang Li;  Kun Xu;  Jintong Lin]]></authors>
##     <affiliations><![CDATA[State Key Lab. of Inf. Photonics & Opt. Commun., Beijing Univ. of Posts & Telecommun., Beijing, China]]></affiliations>
##     <controlledterms>
##       <term><![CDATA[analogue circuits]]></term>
##       <term><![CDATA[intensity modulation]]></term>
##       <term><![CDATA[linearisation techniques]]></term>
##       <term><![CDATA[microwave photonics]]></term>
##       <term><![CDATA[optical fibre communication]]></term>
##       <term><![CDATA[optical links]]></term>
##       <term><![CDATA[optical modulation]]></term>
##       <term><![CDATA[signal processing]]></term>
##     </controlledterms>
##     <thesaurusterms>
##       <term><![CDATA[Nonlinear optics]]></term>
##       <term><![CDATA[Optical distortion]]></term>
##       <term><![CDATA[Optical fiber communication]]></term>
##       <term><![CDATA[Optical fibers]]></term>
##       <term><![CDATA[Optical filters]]></term>
##       <term><![CDATA[Photonics]]></term>
##       <term><![CDATA[Radio frequency]]></term>
##     </thesaurusterms>
##     <pubtitle><![CDATA[Photonics Journal, IEEE]]></pubtitle>
##     <punumber><![CDATA[4563994]]></punumber>
##     <pubtype><![CDATA[Journals & Magazines]]></pubtype>
##     <publisher><![CDATA[IEEE]]></publisher>
##     <volume><![CDATA[6]]></volume>
##     <issue><![CDATA[2]]></issue>
##     <py><![CDATA[2014]]></py>
##     <spage><![CDATA[1]]></spage>
##     <epage><![CDATA[8]]></epage>
##     <abstract><![CDATA[In this paper, a post digital linearization technique is proposed and demonstrated for a conventional intensity-modulation direct-detection analog photonic link. Instead of precisely emulating the transfer function of the whole nonlinear system in the digital domain, the key distortion information is directly acquired from hardware. By low biasing the Mach-Zehnder modulator, linearization can be achieved through simply multiplying the distorted signal by the coreceived low-pass band. Theory analysis shows that only the bias angle should be known by the algorithm, whereas the simulation shows adequate tolerance. The low bias angle also results in an improved link gain. The experiment shows a well-suppressed third-order intermodulation distortion (IMD3). With the optical down conversion, we demonstrate a spurious-free dynamic range increasing from 103.2 to 123.3 dB in a 1-Hz bandwidth.]]></abstract>
##     <issn><![CDATA[1943-0655]]></issn>
##     <htmlFlag><![CDATA[1]]></htmlFlag>
##     <arnumber><![CDATA[6748005]]></arnumber>
##     <doi><![CDATA[10.1109/JPHOT.2014.2308196]]></doi>
##     <publicationId><![CDATA[6748005]]></publicationId>
##     <mdurl><![CDATA[http://ieeexplore.ieee.org/xpl/articleDetails.jsp?tp=&arnumber=6748005&contentType=Journals+%26+Magazines]]></mdurl>
##     <pdf><![CDATA[http://ieeexplore.ieee.org/stamp/stamp.jsp?arnumber=6748005]]></pdf>
##   </document>
##   <document>
##     <rank>725</rank>
##     <title><![CDATA[Reflective Optical Encoder Capitalizing on an Index Grating Imbedded in a Compact Smart Frame]]></title>
##     <authors><![CDATA[Hak-Soon Lee;  Sang-Shin Lee]]></authors>
##     <affiliations><![CDATA[Dept. of Electron. Eng., Kwangwoon Univ., Seoul, South Korea]]></affiliations>
##     <controlledterms>
##       <term><![CDATA[aluminium]]></term>
##       <term><![CDATA[diffraction gratings]]></term>
##       <term><![CDATA[encoding]]></term>
##       <term><![CDATA[integrated optics]]></term>
##       <term><![CDATA[laser cavity resonators]]></term>
##       <term><![CDATA[lenses]]></term>
##       <term><![CDATA[photodetectors]]></term>
##       <term><![CDATA[reflectivity]]></term>
##       <term><![CDATA[surface emitting lasers]]></term>
##     </controlledterms>
##     <thesaurusterms>
##       <term><![CDATA[Gratings]]></term>
##       <term><![CDATA[Indexes]]></term>
##       <term><![CDATA[Laser beams]]></term>
##       <term><![CDATA[Lenses]]></term>
##       <term><![CDATA[Optical reflection]]></term>
##       <term><![CDATA[Optical sensors]]></term>
##       <term><![CDATA[Vertical cavity surface emitting lasers]]></term>
##     </thesaurusterms>
##     <pubtitle><![CDATA[Photonics Journal, IEEE]]></pubtitle>
##     <punumber><![CDATA[4563994]]></punumber>
##     <pubtype><![CDATA[Journals & Magazines]]></pubtype>
##     <publisher><![CDATA[IEEE]]></publisher>
##     <volume><![CDATA[6]]></volume>
##     <issue><![CDATA[2]]></issue>
##     <py><![CDATA[2014]]></py>
##     <spage><![CDATA[1]]></spage>
##     <epage><![CDATA[8]]></epage>
##     <abstract><![CDATA[A reflective optical encoder has been proposed and realized capitalizing on a miniaturized sensing head, which integrates both a pair of index gratings and a beam router in a plastic smart frame. The index gratings play a role in selectively transmitting the incident beam carrying the periodic grating pattern of a code scale. The beam router, consisting of a collimating lens in conjunction with a prism, is used to collimate and steer the beam originating from a vertical-cavity surface-emitting laser (VCSEL) toward the code scale. The proposed encoder is rigorously designed by ray optic simulations and manufactured by passively aligning a smart frame, which is produced by plastic injection molding, with a VCSEL at &#x03BB; = 850 nm and two semicircular photodetectors. The index grating is created by forming an Al grating pattern of a 10- &#x03BC;m pitch on a silica substrate. Two output signals provide sinusoidal signals, which are 90 &#x00B0; out of phase from each other. The period of the signals is 4 &#x03BC;s at a frequency of 250 kHz, which is equivalent to a period of 10 &#x03BC;m. As expected, the relative phase relationship is reversed by altering the direction of rotation of the code scale. By examining either of the two output signals, we can simply demonstrate positional and angular resolutions of about 10 &#x03BC;m and 0.04 &#x00B0;, respectively. The attained resolutions can be readily enhanced to 2.5 &#x03BC;m and 0.01 &#x00B0;, by simultaneously taking into account the two signals with a distinct 90 &#x00B0; phase difference.]]></abstract>
##     <issn><![CDATA[1943-0655]]></issn>
##     <htmlFlag><![CDATA[1]]></htmlFlag>
##     <arnumber><![CDATA[6746062]]></arnumber>
##     <doi><![CDATA[10.1109/JPHOT.2014.2307554]]></doi>
##     <publicationId><![CDATA[6746062]]></publicationId>
##     <mdurl><![CDATA[http://ieeexplore.ieee.org/xpl/articleDetails.jsp?tp=&arnumber=6746062&contentType=Journals+%26+Magazines]]></mdurl>
##     <pdf><![CDATA[http://ieeexplore.ieee.org/stamp/stamp.jsp?arnumber=6746062]]></pdf>
##   </document>
##   <document>
##     <rank>726</rank>
##     <title><![CDATA[Some Consequences of Refusing to Participate in Peer Review]]></title>
##     <authors><![CDATA[Dacso, C.C.]]></authors>
##     <affiliations><![CDATA[, Baylor College of Medicine, Houston, TX, USA]]></affiliations>
##     <thesaurusterms>
##       <term><![CDATA[Bibliometrics]]></term>
##       <term><![CDATA[Engineering profession]]></term>
##       <term><![CDATA[Publishing]]></term>
##       <term><![CDATA[Scientific publishing]]></term>
##       <term><![CDATA[Writing]]></term>
##     </thesaurusterms>
##     <pubtitle><![CDATA[Translational Engineering in Health and Medicine, IEEE Journal of]]></pubtitle>
##     <punumber><![CDATA[6221039]]></punumber>
##     <pubtype><![CDATA[Journals & Magazines]]></pubtype>
##     <publisher><![CDATA[IEEE]]></publisher>
##     <volume><![CDATA[2]]></volume>
##     <py><![CDATA[2014]]></py>
##     <spage><![CDATA[1]]></spage>
##     <epage><![CDATA[3]]></epage>
##     <abstract><![CDATA[The proliferation of journals has had an unexpected side effect: it is now difficult to find qualified reviewers willing to devote the time necessary for assessing journal contributions. Although it is difficult to find data, most scientists involved in the academic world have their inboxes deluged with a cornucopia of invitations to submit to new journals, speak at conferences (as a keynote, for sure!), and review articles. New online journals (such as ours) strive to publish the finest and most relevant work for our readers while at the same time maintaining rapid &#x201C;turn around times&#x201D; for authors. This requires that a complex system function flawlessly. But, there are some aspects of the system that are creaky; some are broken entirely.]]></abstract>
##     <issn><![CDATA[2168-2372]]></issn>
##     <htmlFlag><![CDATA[1]]></htmlFlag>
##     <arnumber><![CDATA[7024956]]></arnumber>
##     <doi><![CDATA[10.1109/JTEHM.2015.2392271]]></doi>
##     <publicationId><![CDATA[7024956]]></publicationId>
##     <mdurl><![CDATA[http://ieeexplore.ieee.org/xpl/articleDetails.jsp?tp=&arnumber=7024956&contentType=Journals+%26+Magazines]]></mdurl>
##     <pdf><![CDATA[http://ieeexplore.ieee.org/stamp/stamp.jsp?arnumber=7024956]]></pdf>
##   </document>
##   <document>
##     <rank>727</rank>
##     <title><![CDATA[Polarization-Independent Receivers for Low-Cost Coherent OOK Systems]]></title>
##     <authors><![CDATA[Ciaramella, E.]]></authors>
##     <affiliations><![CDATA[Scuola Superiore Sant'Anna, Pisa, Italy]]></affiliations>
##     <controlledterms>
##       <term><![CDATA[amplitude shift keying]]></term>
##       <term><![CDATA[low-pass filters]]></term>
##       <term><![CDATA[numerical analysis]]></term>
##       <term><![CDATA[optical distortion]]></term>
##       <term><![CDATA[optical fibre couplers]]></term>
##       <term><![CDATA[optical fibre filters]]></term>
##       <term><![CDATA[optical fibre polarisation]]></term>
##       <term><![CDATA[optical receivers]]></term>
##       <term><![CDATA[optical tuning]]></term>
##       <term><![CDATA[passive optical networks]]></term>
##       <term><![CDATA[photodiodes]]></term>
##     </controlledterms>
##     <thesaurusterms>
##       <term><![CDATA[Bandwidth]]></term>
##       <term><![CDATA[Couplers]]></term>
##       <term><![CDATA[Optical fibers]]></term>
##       <term><![CDATA[Optical polarization]]></term>
##       <term><![CDATA[Optical receivers]]></term>
##       <term><![CDATA[Passive optical networks]]></term>
##     </thesaurusterms>
##     <pubtitle><![CDATA[Photonics Technology Letters, IEEE]]></pubtitle>
##     <punumber><![CDATA[68]]></punumber>
##     <pubtype><![CDATA[Journals & Magazines]]></pubtype>
##     <publisher><![CDATA[IEEE]]></publisher>
##     <volume><![CDATA[26]]></volume>
##     <issue><![CDATA[6]]></issue>
##     <py><![CDATA[2014]]></py>
##     <spage><![CDATA[548]]></spage>
##     <epage><![CDATA[551]]></epage>
##     <abstract><![CDATA[We demonstrate analytically a novel scheme to obtain polarization-independent operation in low-cost coherent OOK receivers, suitable for access networks. We move from a well-known phase-diversity receiver, exploiting a 3 &#x00D7; 3 coupler, three photodiodes, and basic analogue processing. If that receiver is modified to inject the local oscillator (or the signal) into two inputs with proper polarization states, we show, by both mathematical derivation and numerical simulations, that the resulting electrical signal can be polarization-independent; this is attained only if the frequency-detuning between the signal carrier and the local oscillator is high enough, so that intrinsic second order distortions can be spectrally isolated and suppressed by the low-pass filtering of the receiver.]]></abstract>
##     <issn><![CDATA[1041-1135]]></issn>
##     <htmlFlag><![CDATA[1]]></htmlFlag>
##     <arnumber><![CDATA[6701316]]></arnumber>
##     <doi><![CDATA[10.1109/LPT.2013.2296943]]></doi>
##     <publicationId><![CDATA[6701316]]></publicationId>
##     <mdurl><![CDATA[http://ieeexplore.ieee.org/xpl/articleDetails.jsp?tp=&arnumber=6701316&contentType=Journals+%26+Magazines]]></mdurl>
##     <pdf><![CDATA[http://ieeexplore.ieee.org/stamp/stamp.jsp?arnumber=6701316]]></pdf>
##   </document>
##   <document>
##     <rank>728</rank>
##     <title><![CDATA[Study of the Magnetization Method Suitable for Fractional-Slot Concentrated-Winding Variable Magnetomotive-Force Memory Motor]]></title>
##     <authors><![CDATA[Maekawa, S.;  Yuki, K.;  Matsushita, M.;  Nitta, I.;  Hasegawa, Y.;  Shiga, T.;  Hosoito, T.;  Nagai, K.;  Kubota, H.]]></authors>
##     <affiliations><![CDATA[Toshiba Corp. FUCHU Oper. Power & Ind. Syst. R&D Center, Fuchu, Japan]]></affiliations>
##     <controlledterms>
##       <term><![CDATA[machine windings]]></term>
##       <term><![CDATA[magnetisation]]></term>
##       <term><![CDATA[optimal control]]></term>
##       <term><![CDATA[permanent magnet motors]]></term>
##       <term><![CDATA[synchronous motors]]></term>
##     </controlledterms>
##     <thesaurusterms>
##       <term><![CDATA[Magnetic circuits]]></term>
##       <term><![CDATA[Magnetic fields]]></term>
##       <term><![CDATA[Magnetic flux]]></term>
##       <term><![CDATA[Magnetization]]></term>
##       <term><![CDATA[Permanent magnet motors]]></term>
##       <term><![CDATA[Saturation magnetization]]></term>
##       <term><![CDATA[Windings]]></term>
##     </thesaurusterms>
##     <pubtitle><![CDATA[Power Electronics, IEEE Transactions on]]></pubtitle>
##     <punumber><![CDATA[63]]></punumber>
##     <pubtype><![CDATA[Journals & Magazines]]></pubtype>
##     <publisher><![CDATA[IEEE]]></publisher>
##     <volume><![CDATA[29]]></volume>
##     <issue><![CDATA[9]]></issue>
##     <py><![CDATA[2014]]></py>
##     <spage><![CDATA[4877]]></spage>
##     <epage><![CDATA[4887]]></epage>
##     <abstract><![CDATA[This paper describes a variable magnetomotive-force memory motor with a fractional-slot concentrated winding. We propose a method for optimal control of magnetization suitable for a fractional-slot concentrated winding. Our result shows that variable magnetized magnets are more highly magnetized when a fractional-slot concentrated winding is used than those using the conventional method.]]></abstract>
##     <issn><![CDATA[0885-8993]]></issn>
##     <htmlFlag><![CDATA[1]]></htmlFlag>
##     <arnumber><![CDATA[6656020]]></arnumber>
##     <doi><![CDATA[10.1109/TPEL.2013.2288635]]></doi>
##     <publicationId><![CDATA[6656020]]></publicationId>
##     <mdurl><![CDATA[http://ieeexplore.ieee.org/xpl/articleDetails.jsp?tp=&arnumber=6656020&contentType=Journals+%26+Magazines]]></mdurl>
##     <pdf><![CDATA[http://ieeexplore.ieee.org/stamp/stamp.jsp?arnumber=6656020]]></pdf>
##   </document>
##   <document>
##     <rank>729</rank>
##     <title><![CDATA[Material Gain in Ga<sub>0.66</sub>In<sub>0.34</sub>NyAs<sub>1&#x2013;y</sub>, GaN<sub>y</sub>As<sub>0.69&#x2013;y</sub>Sb<sub>0.31</sub>, and GaN<sub>y</sub>P<sub>0.46</sub>Sb<sub>0.54&#x2013;y</sub> Quantum Wells Grown on GaAs Substrates: Comparative Theoretical Studies]]></title>
##     <authors><![CDATA[Gladysiewicz, M.;  Kudrawiec, R.;  Wartak, M.S.]]></authors>
##     <affiliations><![CDATA[Inst. of Phys., Wroclaw Univ. of Technol., Wroc&#x0142;aw, Poland]]></affiliations>
##     <controlledterms>
##       <term><![CDATA[III-V semiconductors]]></term>
##       <term><![CDATA[band structure]]></term>
##       <term><![CDATA[effective mass]]></term>
##       <term><![CDATA[energy gap]]></term>
##       <term><![CDATA[gallium arsenide]]></term>
##       <term><![CDATA[gallium compounds]]></term>
##       <term><![CDATA[indium compounds]]></term>
##       <term><![CDATA[nitrogen compounds]]></term>
##       <term><![CDATA[optical materials]]></term>
##       <term><![CDATA[phosphorus compounds]]></term>
##       <term><![CDATA[quantum well lasers]]></term>
##       <term><![CDATA[semiconductor growth]]></term>
##       <term><![CDATA[semiconductor quantum wells]]></term>
##       <term><![CDATA[spectral line shift]]></term>
##     </controlledterms>
##     <thesaurusterms>
##       <term><![CDATA[Effective mass]]></term>
##       <term><![CDATA[Gallium arsenide]]></term>
##       <term><![CDATA[Manganese]]></term>
##       <term><![CDATA[Nitrogen]]></term>
##       <term><![CDATA[Photonic band gap]]></term>
##     </thesaurusterms>
##     <pubtitle><![CDATA[Quantum Electronics, IEEE Journal of]]></pubtitle>
##     <punumber><![CDATA[3]]></punumber>
##     <pubtype><![CDATA[Journals & Magazines]]></pubtype>
##     <publisher><![CDATA[IEEE]]></publisher>
##     <volume><![CDATA[50]]></volume>
##     <issue><![CDATA[12]]></issue>
##     <py><![CDATA[2014]]></py>
##     <spage><![CDATA[996]]></spage>
##     <epage><![CDATA[1005]]></epage>
##     <abstract><![CDATA[Electronic band structure and material gain are calculated using the 8-band kp Hamiltonian for three quantum well (QW) systems grown on GaAs substrates: 1) Ga<sub>0.66</sub>In<sub>0.34</sub>N<sub>y</sub>As<sub>1-y</sub>/GaAs; 2) GaN<sub>y</sub>As<sub>0.69-y</sub>Sb<sub>0.31</sub>/GaAs; and 3) GaN<sub>y</sub>P<sub>0.46</sub>Sb<sub>0.54-y</sub>/GaAs QWs with various nitrogen concentrations. Bandgap and electron effective mass for Ga<sub>0.66</sub>In<sub>0.34</sub>N<sub>y</sub>As<sub>1-y</sub>, GaN<sub>y</sub>As<sub>0.69-y</sub>Sb<sub>0.31</sub>, and GaN<sub>y</sub>P<sub>0.46</sub> Sb<sub>0.54-y</sub> alloys are determined within band anticrossing model using the proper alloying approximations. The intensity and spectral position of gain peak are directly compared for QWs with 2% of nitrogen and 2% compressive strain in the QW layer. The largest shift of gain peak has been observed for GaN<sub>0.02</sub>P<sub>0.46</sub>Sb<sub>0.52</sub>/GaAs QW. In this case, the gain peak (transverse electric (TE) mode) is located at 1.77 &#x03BC;m that is longer by 390 nm than the gain peak (TE mode) for Ga<sub>0.66</sub>In<sub>0.34</sub>N<sub>0.02</sub>As<sub>0.98</sub>/GaAs QW. In addition, for GaN<sub>0.02</sub>As<sub>0.67</sub>Sb<sup>0.31</sup>/GaAs QW the largest shift of gain peak (TE mode) is observed for Ga<sub>0.66</sub>In<sub>0.34</sub>N<sub>0.02</sub>As<sub>0.98</sub>/GaAs QW (1.67 versus 1.38~&#x03BC;m. The intensities of TE modes of material gain are comparable for all three systems. The above indicates that GaN<sub>y</sub>As<sub>0.69-y</sub>Sb<sub>0.31</sub>/GaAs and GaN<sub>y</sub>P<sub>0.46</sub>Sb<sub>0.54-y</sub>/GaAs QW systems are promising candidates for GaAs-based lasers dedicated for emission in very long-wavelength telecommunication windows. The required nitrogen concentration for achieving gain peak position (TE mode) at 1.3 and 1.55~&#x03BC;m are 0.5% and 1.5% N for GaN<sub>y</sub>As<sub>0.69-y</sub>Sb<sub>0.31</sub>/GaAs QWs, respectively, while for GaN- sub>y</sub>P<sub>0.46</sub>Sb<sub>0.54-y</sub>/GaAs QWs are 0.3% and 1.2% N. Those values are significantly less than in Ga<sub>0.66</sub>In<sub>0.34</sub>N<sub>y</sub>As<sub>1-y</sub>/GaAs QWs. Considering a possible blue shift of QW emission upon annealing (i.e., the post grown procedure, which is usually used for improving optical quality of these QWs) the required nitrogen concentration can be larger by ~0.5% N than the one calculated for as-grown QWs.]]></abstract>
##     <issn><![CDATA[0018-9197]]></issn>
##     <htmlFlag><![CDATA[1]]></htmlFlag>
##     <arnumber><![CDATA[6940208]]></arnumber>
##     <doi><![CDATA[10.1109/JQE.2014.2363763]]></doi>
##     <publicationId><![CDATA[6940208]]></publicationId>
##     <mdurl><![CDATA[http://ieeexplore.ieee.org/xpl/articleDetails.jsp?tp=&arnumber=6940208&contentType=Journals+%26+Magazines]]></mdurl>
##     <pdf><![CDATA[http://ieeexplore.ieee.org/stamp/stamp.jsp?arnumber=6940208]]></pdf>
##   </document>
##   <document>
##     <rank>730</rank>
##     <title><![CDATA[Optical Pulse Generation Based on an Optoelectronic Oscillator With Cascaded Nonlinear Semiconductor Optical Amplifiers]]></title>
##     <authors><![CDATA[Ningbo Huang;  Ming Li;  Ye Deng;  Ning Hua Zhu]]></authors>
##     <affiliations><![CDATA[State Key Lab. on Integrated Optoelectron., Inst. of Semicond., Beijing, China]]></affiliations>
##     <controlledterms>
##       <term><![CDATA[microwave photonics]]></term>
##       <term><![CDATA[multiwave mixing]]></term>
##       <term><![CDATA[optical pulse generation]]></term>
##       <term><![CDATA[oscillators]]></term>
##       <term><![CDATA[semiconductor optical amplifiers]]></term>
##     </controlledterms>
##     <thesaurusterms>
##       <term><![CDATA[Nonlinear optics]]></term>
##       <term><![CDATA[Optical fibers]]></term>
##       <term><![CDATA[Optical filters]]></term>
##       <term><![CDATA[Optical polarization]]></term>
##       <term><![CDATA[Optical pulses]]></term>
##       <term><![CDATA[Semiconductor optical amplifiers]]></term>
##     </thesaurusterms>
##     <pubtitle><![CDATA[Photonics Journal, IEEE]]></pubtitle>
##     <punumber><![CDATA[4563994]]></punumber>
##     <pubtype><![CDATA[Journals & Magazines]]></pubtype>
##     <publisher><![CDATA[IEEE]]></publisher>
##     <volume><![CDATA[6]]></volume>
##     <issue><![CDATA[1]]></issue>
##     <py><![CDATA[2014]]></py>
##     <spage><![CDATA[1]]></spage>
##     <epage><![CDATA[8]]></epage>
##     <abstract><![CDATA[This paper presents a high-repetition-rate pulse train generator incorporating an optoelectronic oscillator based on four-wave mixing (FWM) in a semiconductor optical amplifier (SOA). The optoelectronic oscillator is used to generate a high-stability microwave signal, by which an optical pulse train is generated using FWM effect in cascaded SOAs. The key feature of this work is that no external microwave signal source is needed for generating an optical pulse train. An optical pulse train with a repetition rate of 10 GHz and a pulsewidth of 19 ps is experimentally generated. The pulsewidth and the repetition rate can be tunable and are experimentally demonstrated.]]></abstract>
##     <issn><![CDATA[1943-0655]]></issn>
##     <htmlFlag><![CDATA[1]]></htmlFlag>
##     <arnumber><![CDATA[6731528]]></arnumber>
##     <doi><![CDATA[10.1109/JPHOT.2014.2304552]]></doi>
##     <publicationId><![CDATA[6731528]]></publicationId>
##     <mdurl><![CDATA[http://ieeexplore.ieee.org/xpl/articleDetails.jsp?tp=&arnumber=6731528&contentType=Journals+%26+Magazines]]></mdurl>
##     <pdf><![CDATA[http://ieeexplore.ieee.org/stamp/stamp.jsp?arnumber=6731528]]></pdf>
##   </document>
##   <document>
##     <rank>731</rank>
##     <title><![CDATA[Theoretical Analysis of Plasmonic Modes in a Symmetric Conductor&#x2013;Gap&#x2013;Dielectric Structure for Nanoscale Confinement]]></title>
##     <authors><![CDATA[Li Wei;  Aldawsari, S.;  Wing-Ki Liu;  West, B.R.]]></authors>
##     <affiliations><![CDATA[Dept. of Phys. & Comput. Sci., Wilfrid Laurier Univ., Waterloo, ON, Canada]]></affiliations>
##     <controlledterms>
##       <term><![CDATA[metallic thin films]]></term>
##       <term><![CDATA[nanophotonics]]></term>
##       <term><![CDATA[optical waveguides]]></term>
##       <term><![CDATA[plasmonics]]></term>
##       <term><![CDATA[polaritons]]></term>
##       <term><![CDATA[surface plasmons]]></term>
##     </controlledterms>
##     <thesaurusterms>
##       <term><![CDATA[Equations]]></term>
##       <term><![CDATA[Films]]></term>
##       <term><![CDATA[Indexes]]></term>
##       <term><![CDATA[Metals]]></term>
##       <term><![CDATA[Optical waveguides]]></term>
##       <term><![CDATA[Plasmons]]></term>
##       <term><![CDATA[Propagation losses]]></term>
##     </thesaurusterms>
##     <pubtitle><![CDATA[Photonics Journal, IEEE]]></pubtitle>
##     <punumber><![CDATA[4563994]]></punumber>
##     <pubtype><![CDATA[Journals & Magazines]]></pubtype>
##     <publisher><![CDATA[IEEE]]></publisher>
##     <volume><![CDATA[6]]></volume>
##     <issue><![CDATA[3]]></issue>
##     <py><![CDATA[2014]]></py>
##     <spage><![CDATA[1]]></spage>
##     <epage><![CDATA[10]]></epage>
##     <abstract><![CDATA[A hybrid plasmonic waveguide is considered as one of the most promising architectures for long-range subwavelength guiding. The objective of this paper is to present a theoretical analysis of plasmonic guided modes in a symmetric conductor-gap-dielectric (SCGD) system. It consists of a thin metal conductor symmetrically sandwiched by two-layer dielectrics with low-index nanoscale gaps inside. The SCGD waveguide can support ultra-long range surface plasmon-polariton mode when the thickness of a low-index gap is smaller than a cutoff gap thickness. For relatively high index contrast ratios of the cladding to gap layers, the cutoff gap thickness is only a few nanometers, within which the electric field of the guided SCGD mode is tightly confined. The dispersion equations and approximate analytical expressions of the cutoff gap thickness are derived in order to characterize the properties of the guided mode. Our simulation results show that the cutoff gap thickness can be tailored by the metal film thickness and the indices of the cladding and gap materials. The geometrical scheme for lateral confinement is also presented. Such a structure with unique features of low-loss and strong confinement has applications in the fabrication of active and passive plasmonic devices.]]></abstract>
##     <issn><![CDATA[1943-0655]]></issn>
##     <htmlFlag><![CDATA[1]]></htmlFlag>
##     <arnumber><![CDATA[6823089]]></arnumber>
##     <doi><![CDATA[10.1109/JPHOT.2014.2326677]]></doi>
##     <publicationId><![CDATA[6823089]]></publicationId>
##     <mdurl><![CDATA[http://ieeexplore.ieee.org/xpl/articleDetails.jsp?tp=&arnumber=6823089&contentType=Journals+%26+Magazines]]></mdurl>
##     <pdf><![CDATA[http://ieeexplore.ieee.org/stamp/stamp.jsp?arnumber=6823089]]></pdf>
##   </document>
##   <document>
##     <rank>732</rank>
##     <title><![CDATA[Proposal and Fabrication of an Electrooptically Controlled Multimode Microresonator for Continuous Fast-to-Slow Light Tuning]]></title>
##     <authors><![CDATA[Qingzhong Huang;  Ge Song;  Juguang Chen;  Zhan Shu;  Jinzhong Yu]]></authors>
##     <affiliations><![CDATA[Wuhan Nat. Lab. for Optoelectron., Huazhong Univ. of Sci. & Technol., Wuhan, China]]></affiliations>
##     <controlledterms>
##       <term><![CDATA[delays]]></term>
##       <term><![CDATA[electro-optical devices]]></term>
##       <term><![CDATA[light propagation]]></term>
##       <term><![CDATA[micro-optomechanical devices]]></term>
##       <term><![CDATA[microcavities]]></term>
##       <term><![CDATA[micromechanical resonators]]></term>
##       <term><![CDATA[optical fabrication]]></term>
##       <term><![CDATA[optical tuning]]></term>
##       <term><![CDATA[slow light]]></term>
##     </controlledterms>
##     <thesaurusterms>
##       <term><![CDATA[Bandwidth]]></term>
##       <term><![CDATA[Delay effects]]></term>
##       <term><![CDATA[Delays]]></term>
##       <term><![CDATA[Electrooptical waveguides]]></term>
##       <term><![CDATA[Optical resonators]]></term>
##       <term><![CDATA[Tuning]]></term>
##     </thesaurusterms>
##     <pubtitle><![CDATA[Photonics Journal, IEEE]]></pubtitle>
##     <punumber><![CDATA[4563994]]></punumber>
##     <pubtype><![CDATA[Journals & Magazines]]></pubtype>
##     <publisher><![CDATA[IEEE]]></publisher>
##     <volume><![CDATA[6]]></volume>
##     <issue><![CDATA[4]]></issue>
##     <py><![CDATA[2014]]></py>
##     <spage><![CDATA[1]]></spage>
##     <epage><![CDATA[11]]></epage>
##     <abstract><![CDATA[We propose a scheme to utilize a multimode microdisk resonator in silicon controlled through free-carrier injection for continuous fast-to-slow light tuning. Two nearby resonant modes are employed, i.e., undercoupling and overcoupling. The optical properties of a device are analytically investigated considering the resonance spacing, bandwidth, and density change of free carriers. Pulse propagation simulations are implemented, and the pulse delay/advance is studied for wavelengths between the two resonances. As the density of injected free carriers increases, a continuous transition from fast to slow light occurs for a range of wavelengths. Consequently, a microdisk resonator in silicon is experimentally realized with the desired two resonances, exhibiting electrooptical tuning with a fast response. The experimental results agree well with the simulation in power transmission, and the tunable time delay and advance are also predicted.]]></abstract>
##     <issn><![CDATA[1943-0655]]></issn>
##     <htmlFlag><![CDATA[1]]></htmlFlag>
##     <arnumber><![CDATA[6842588]]></arnumber>
##     <doi><![CDATA[10.1109/JPHOT.2014.2332473]]></doi>
##     <publicationId><![CDATA[6842588]]></publicationId>
##     <mdurl><![CDATA[http://ieeexplore.ieee.org/xpl/articleDetails.jsp?tp=&arnumber=6842588&contentType=Journals+%26+Magazines]]></mdurl>
##     <pdf><![CDATA[http://ieeexplore.ieee.org/stamp/stamp.jsp?arnumber=6842588]]></pdf>
##   </document>
##   <document>
##     <rank>733</rank>
##     <title><![CDATA[Big Data Deep Learning: Challenges and Perspectives]]></title>
##     <authors><![CDATA[Xue-Wen Chen;  Xiaotong Lin]]></authors>
##     <affiliations><![CDATA[Dept. of Comput. Sci., Wayne State Univ., Detroit, MI, USA]]></affiliations>
##     <controlledterms>
##       <term><![CDATA[Big Data]]></term>
##       <term><![CDATA[data analysis]]></term>
##       <term><![CDATA[learning (artificial intelligence)]]></term>
##     </controlledterms>
##     <thesaurusterms>
##       <term><![CDATA[Big data]]></term>
##       <term><![CDATA[Data processing]]></term>
##       <term><![CDATA[Information analysis]]></term>
##       <term><![CDATA[Machine learning]]></term>
##       <term><![CDATA[Natural language processing]]></term>
##       <term><![CDATA[Pattern recognition]]></term>
##     </thesaurusterms>
##     <pubtitle><![CDATA[Access, IEEE]]></pubtitle>
##     <punumber><![CDATA[6287639]]></punumber>
##     <pubtype><![CDATA[Journals & Magazines]]></pubtype>
##     <publisher><![CDATA[IEEE]]></publisher>
##     <volume><![CDATA[2]]></volume>
##     <py><![CDATA[2014]]></py>
##     <spage><![CDATA[514]]></spage>
##     <epage><![CDATA[525]]></epage>
##     <abstract><![CDATA[Deep learning is currently an extremely active research area in machine learning and pattern recognition society. It has gained huge successes in a broad area of applications such as speech recognition, computer vision, and natural language processing. With the sheer size of data available today, big data brings big opportunities and transformative potential for various sectors; on the other hand, it also presents unprecedented challenges to harnessing data and information. As the data keeps getting bigger, deep learning is coming to play a key role in providing big data predictive analytics solutions. In this paper, we provide a brief overview of deep learning, and highlight current research efforts and the challenges to big data, as well as the future trends.]]></abstract>
##     <issn><![CDATA[2169-3536]]></issn>
##     <htmlFlag><![CDATA[1]]></htmlFlag>
##     <arnumber><![CDATA[6817512]]></arnumber>
##     <doi><![CDATA[10.1109/ACCESS.2014.2325029]]></doi>
##     <publicationId><![CDATA[6817512]]></publicationId>
##     <mdurl><![CDATA[http://ieeexplore.ieee.org/xpl/articleDetails.jsp?tp=&arnumber=6817512&contentType=Journals+%26+Magazines]]></mdurl>
##     <pdf><![CDATA[http://ieeexplore.ieee.org/stamp/stamp.jsp?arnumber=6817512]]></pdf>
##   </document>
##   <document>
##     <rank>734</rank>
##     <title><![CDATA[Delta Modulation Technique for Improving the Sensitivity of Monobit Subsamplers in Radar and Coherent Receiver Applications]]></title>
##     <authors><![CDATA[Rodenbeck, C.T.;  Tracey, K.J.;  Barkley, K.R.;  DuVerneay, B.B.]]></authors>
##     <affiliations><![CDATA[Sandia Nat. Labs., Albuquerque, NM, USA]]></affiliations>
##     <controlledterms>
##       <term><![CDATA[analogue-digital conversion]]></term>
##       <term><![CDATA[radar receivers]]></term>
##     </controlledterms>
##     <thesaurusterms>
##       <term><![CDATA[Capacitors]]></term>
##       <term><![CDATA[Charge pumps]]></term>
##       <term><![CDATA[Radar]]></term>
##       <term><![CDATA[Radiation detectors]]></term>
##       <term><![CDATA[Radio frequency]]></term>
##       <term><![CDATA[Receivers]]></term>
##       <term><![CDATA[Signal to noise ratio]]></term>
##     </thesaurusterms>
##     <pubtitle><![CDATA[Microwave Theory and Techniques, IEEE Transactions on]]></pubtitle>
##     <punumber><![CDATA[22]]></punumber>
##     <pubtype><![CDATA[Journals & Magazines]]></pubtype>
##     <publisher><![CDATA[IEEE]]></publisher>
##     <volume><![CDATA[62]]></volume>
##     <issue><![CDATA[8]]></issue>
##     <py><![CDATA[2014]]></py>
##     <spage><![CDATA[1811]]></spage>
##     <epage><![CDATA[1822]]></epage>
##     <abstract><![CDATA[This paper introduces a technique for improving the sensitivity of RF subsamplers in radar and coherent receiver applications. The technique, referred to herein as &#x201C;delta modulation&#x201D; (DM), feeds the time-average output of a monobit analog-to-digital converter (ADC) back to the ADC input, but with opposite polarity. Assuming pseudostationary modulation statistics on the sampled RF waveform, the feedback signal corrects for aggregate dc offsets present in the ADC that otherwise degrade ADC sensitivity. Two RF integrated circuits (RFICs) are designed to demonstrate the approach. One uses analog DM to create the feedback signal; the other uses digital DM to achieve the same result. A series of tests validates the designs. The dynamic time-domain response confirms the feedback loop's basic operation. Measured output quantization imbalance, under noise-only input drive, significantly improves with the use of the DM circuit, even for large, deliberately induced dc offsets and wide temperature variation from -55 &#x00B0;C to +85 &#x00B0;C. Examination of the corrected versus uncorrected baseband spectrum under swept input signal-to-noise ratio (SNR) conditions demonstrates the effectiveness of this approach for realistic radar and coherent receiver applications. Two-tone testing shows no impact of the DM technique on ADC linearity.]]></abstract>
##     <issn><![CDATA[0018-9480]]></issn>
##     <htmlFlag><![CDATA[1]]></htmlFlag>
##     <arnumber><![CDATA[6857428]]></arnumber>
##     <doi><![CDATA[10.1109/TMTT.2014.2332433]]></doi>
##     <publicationId><![CDATA[6857428]]></publicationId>
##     <mdurl><![CDATA[http://ieeexplore.ieee.org/xpl/articleDetails.jsp?tp=&arnumber=6857428&contentType=Journals+%26+Magazines]]></mdurl>
##     <pdf><![CDATA[http://ieeexplore.ieee.org/stamp/stamp.jsp?arnumber=6857428]]></pdf>
##   </document>
##   <document>
##     <rank>735</rank>
##     <title><![CDATA[The Prototype Inductive Adder With Droop Compensation for the CLIC Kicker Systems]]></title>
##     <authors><![CDATA[Holma, J.;  Barnes, M.J.]]></authors>
##     <affiliations><![CDATA[Eur. Lab. for Particle Phys., CERN, Geneva, Switzerland]]></affiliations>
##     <controlledterms>
##       <term><![CDATA[adders]]></term>
##       <term><![CDATA[electron accelerators]]></term>
##       <term><![CDATA[linear colliders]]></term>
##       <term><![CDATA[nuclear electronics]]></term>
##       <term><![CDATA[particle beam bunching]]></term>
##       <term><![CDATA[particle beam extraction]]></term>
##       <term><![CDATA[synchrotron radiation]]></term>
##     </controlledterms>
##     <thesaurusterms>
##       <term><![CDATA[Adders]]></term>
##       <term><![CDATA[Capacitors]]></term>
##       <term><![CDATA[Inductance]]></term>
##       <term><![CDATA[Modulation]]></term>
##       <term><![CDATA[Prototypes]]></term>
##       <term><![CDATA[Resistance]]></term>
##       <term><![CDATA[Resistors]]></term>
##     </thesaurusterms>
##     <pubtitle><![CDATA[Plasma Science, IEEE Transactions on]]></pubtitle>
##     <punumber><![CDATA[27]]></punumber>
##     <pubtype><![CDATA[Journals & Magazines]]></pubtype>
##     <publisher><![CDATA[IEEE]]></publisher>
##     <volume><![CDATA[42]]></volume>
##     <issue><![CDATA[10]]></issue>
##     <part><![CDATA[2]]></part>
##     <py><![CDATA[2014]]></py>
##     <spage><![CDATA[2899]]></spage>
##     <epage><![CDATA[2908]]></epage>
##     <abstract><![CDATA[The Compact Linear Collider (CLIC) study is exploring the scheme for an electron-positron collider with high luminosity and a nominal center-of-mass energy of 3 TeV. The CLIC predamping rings and damping rings (DRs) will produce, through synchrotron radiation, an ultralow emittance beam with high bunch charge. To avoid beam emittance increase, the DR kicker systems must provide extremely flat, high-voltage, pulses. The specifications for the extraction kickers of the DRs are particularly demanding: the flattops of the pulses must be &#x00B1;12.5 kV with a combined ripple and droop of not more than &#x00B1;0.02% (&#x00B1;2.5 V). An inductive adder is a very promising approach to meeting the specifications. Recently, a five-layer prototype has been built at CERN. Passive analog modulation has been applied to compensate the voltage droop, for example of the pulse capacitors. The output waveforms of the prototype inductive adder have been compared with predictions of the voltage droop and pulse shape. Conclusions are drawn concerning the design of the full-scale prototype inductive adder.]]></abstract>
##     <issn><![CDATA[0093-3813]]></issn>
##     <htmlFlag><![CDATA[1]]></htmlFlag>
##     <arnumber><![CDATA[6781555]]></arnumber>
##     <doi><![CDATA[10.1109/TPS.2014.2312033]]></doi>
##     <publicationId><![CDATA[6781555]]></publicationId>
##     <mdurl><![CDATA[http://ieeexplore.ieee.org/xpl/articleDetails.jsp?tp=&arnumber=6781555&contentType=Journals+%26+Magazines]]></mdurl>
##     <pdf><![CDATA[http://ieeexplore.ieee.org/stamp/stamp.jsp?arnumber=6781555]]></pdf>
##   </document>
##   <document>
##     <rank>736</rank>
##     <title><![CDATA[Practical Switching-Based Hybrid FSO/RF Transmission and Its Performance Analysis]]></title>
##     <authors><![CDATA[Usman, M.;  Hong-Chuan Yang;  Alouini, M.-S.]]></authors>
##     <affiliations><![CDATA[Dept. of Electr. & Comput. Eng., Univ. of Victoria, Victoria, BC, Canada]]></affiliations>
##     <controlledterms>
##       <term><![CDATA[microwave photonics]]></term>
##       <term><![CDATA[optical links]]></term>
##       <term><![CDATA[optical switches]]></term>
##     </controlledterms>
##     <thesaurusterms>
##       <term><![CDATA[Adaptive optics]]></term>
##       <term><![CDATA[Bit error rate]]></term>
##       <term><![CDATA[Fading]]></term>
##       <term><![CDATA[Modulation]]></term>
##       <term><![CDATA[Optical transmitters]]></term>
##       <term><![CDATA[Radio frequency]]></term>
##       <term><![CDATA[Signal to noise ratio]]></term>
##     </thesaurusterms>
##     <pubtitle><![CDATA[Photonics Journal, IEEE]]></pubtitle>
##     <punumber><![CDATA[4563994]]></punumber>
##     <pubtype><![CDATA[Journals & Magazines]]></pubtype>
##     <publisher><![CDATA[IEEE]]></publisher>
##     <volume><![CDATA[6]]></volume>
##     <issue><![CDATA[5]]></issue>
##     <py><![CDATA[2014]]></py>
##     <spage><![CDATA[1]]></spage>
##     <epage><![CDATA[13]]></epage>
##     <abstract><![CDATA[Hybrid free-space optical (FSO)/radio-frequency (RF) systems have emerged as a promising solution for high-data-rate wireless backhaul. We present and analyze a switching-based transmission scheme for the hybrid FSO/RF system. Specifically, either the FSO or RF link will be active at a certain time instance, with the FSO link enjoying a higher priority. We considered both a single-threshold case and a dual-threshold case for FSO link operation. Analytical expressions have been obtained for the outage probability, average bit error rate, and ergodic capacity for the resulting system. Numerical examples are presented to compare the performance of the hybrid scheme with the FSO-only scenario.]]></abstract>
##     <issn><![CDATA[1943-0655]]></issn>
##     <htmlFlag><![CDATA[1]]></htmlFlag>
##     <arnumber><![CDATA[6887284]]></arnumber>
##     <doi><![CDATA[10.1109/JPHOT.2014.2352629]]></doi>
##     <publicationId><![CDATA[6887284]]></publicationId>
##     <mdurl><![CDATA[http://ieeexplore.ieee.org/xpl/articleDetails.jsp?tp=&arnumber=6887284&contentType=Journals+%26+Magazines]]></mdurl>
##     <pdf><![CDATA[http://ieeexplore.ieee.org/stamp/stamp.jsp?arnumber=6887284]]></pdf>
##   </document>
##   <document>
##     <rank>737</rank>
##     <title><![CDATA[A Simple Closed-Form Approximation for the Packet Loss Rate of a TCP Connection Over Wireless Links]]></title>
##     <authors><![CDATA[Mata-Diaz, J.;  Alins, J.;  Munoz, J.L.;  Esparza, O.]]></authors>
##     <affiliations><![CDATA[Univ. Politec. de Catalunya, Barcelona, Spain]]></affiliations>
##     <controlledterms>
##       <term><![CDATA[approximation theory]]></term>
##       <term><![CDATA[packet radio networks]]></term>
##       <term><![CDATA[probability]]></term>
##       <term><![CDATA[telecommunication congestion control]]></term>
##       <term><![CDATA[transport protocols]]></term>
##     </controlledterms>
##     <thesaurusterms>
##       <term><![CDATA[Analytical models]]></term>
##       <term><![CDATA[Approximation methods]]></term>
##       <term><![CDATA[Packet loss]]></term>
##       <term><![CDATA[Propagation losses]]></term>
##       <term><![CDATA[Throughput]]></term>
##       <term><![CDATA[Wireless communication]]></term>
##     </thesaurusterms>
##     <pubtitle><![CDATA[Communications Letters, IEEE]]></pubtitle>
##     <punumber><![CDATA[4234]]></punumber>
##     <pubtype><![CDATA[Journals & Magazines]]></pubtype>
##     <publisher><![CDATA[IEEE]]></publisher>
##     <volume><![CDATA[18]]></volume>
##     <issue><![CDATA[9]]></issue>
##     <py><![CDATA[2014]]></py>
##     <spage><![CDATA[1595]]></spage>
##     <epage><![CDATA[1598]]></epage>
##     <abstract><![CDATA[This letter presents a new and simple model for a TCP flow experiencing random packet losses due to both transmission errors and congestion events. From this model, we will derive a straightforward expression of a unified loss probability (ULP). This ULP gives the opportunity to reuse classical analytic models to analyze the performance of TCP and to size the buffer to optimize the wireless link utilization. Extensive simulations using TCP Reno in ns-2 demonstrate that our model is valid not only for the extreme cases where either transmission errors or congestion losses dominate but also in the situations where both types of losses are significant.]]></abstract>
##     <issn><![CDATA[1089-7798]]></issn>
##     <htmlFlag><![CDATA[1]]></htmlFlag>
##     <arnumber><![CDATA[6855360]]></arnumber>
##     <doi><![CDATA[10.1109/LCOMM.2014.2336844]]></doi>
##     <publicationId><![CDATA[6855360]]></publicationId>
##     <mdurl><![CDATA[http://ieeexplore.ieee.org/xpl/articleDetails.jsp?tp=&arnumber=6855360&contentType=Journals+%26+Magazines]]></mdurl>
##     <pdf><![CDATA[http://ieeexplore.ieee.org/stamp/stamp.jsp?arnumber=6855360]]></pdf>
##   </document>
##   <document>
##     <rank>738</rank>
##     <title><![CDATA[All-Optical Frequency Shifter Based on Stimulated Brillouin Scattering in an Optical Fiber]]></title>
##     <authors><![CDATA[Chan, E.H.W.;  Minasian, R.A.]]></authors>
##     <affiliations><![CDATA[Sch. of Electr. & Inf. Eng., Univ. of Sydney, Sydney, NSW, Australia]]></affiliations>
##     <controlledterms>
##       <term><![CDATA[microwave photonics]]></term>
##       <term><![CDATA[optical communication equipment]]></term>
##       <term><![CDATA[optical fibre losses]]></term>
##       <term><![CDATA[optical fibres]]></term>
##       <term><![CDATA[optical frequency conversion]]></term>
##       <term><![CDATA[optical modulation]]></term>
##       <term><![CDATA[optical tuning]]></term>
##       <term><![CDATA[radio-over-fibre]]></term>
##       <term><![CDATA[stimulated Brillouin scattering]]></term>
##     </controlledterms>
##     <thesaurusterms>
##       <term><![CDATA[Frequency modulation]]></term>
##       <term><![CDATA[Optical fibers]]></term>
##       <term><![CDATA[Optical modulation]]></term>
##       <term><![CDATA[Optical pumping]]></term>
##       <term><![CDATA[Scattering]]></term>
##     </thesaurusterms>
##     <pubtitle><![CDATA[Photonics Journal, IEEE]]></pubtitle>
##     <punumber><![CDATA[4563994]]></punumber>
##     <pubtype><![CDATA[Journals & Magazines]]></pubtype>
##     <publisher><![CDATA[IEEE]]></publisher>
##     <volume><![CDATA[6]]></volume>
##     <issue><![CDATA[2]]></issue>
##     <py><![CDATA[2014]]></py>
##     <spage><![CDATA[1]]></spage>
##     <epage><![CDATA[10]]></epage>
##     <abstract><![CDATA[A new optical frequency shifter that can shift the light frequency in the microwave or millimeter-wave frequency range is presented. It is based on the use of stimulated Brillouin scattering gain and loss spectrums in an optical fiber to suppress one sideband of a double-sideband suppressed carrier modulation signal while amplifying the other sideband. The new optical frequency shifter only involves optical components and is capable of operating over a wide frequency range with low spurious generation and with tunable capabilities. It can be also used as a single-sideband suppressed carrier modulator. Experimental results demonstrate a 20 GHz frequency shift with a 25 dB signal-to-noise ratio and a widely tunable frequency shifting operation.]]></abstract>
##     <issn><![CDATA[1943-0655]]></issn>
##     <htmlFlag><![CDATA[1]]></htmlFlag>
##     <arnumber><![CDATA[6776398]]></arnumber>
##     <doi><![CDATA[10.1109/JPHOT.2014.2312921]]></doi>
##     <publicationId><![CDATA[6776398]]></publicationId>
##     <mdurl><![CDATA[http://ieeexplore.ieee.org/xpl/articleDetails.jsp?tp=&arnumber=6776398&contentType=Journals+%26+Magazines]]></mdurl>
##     <pdf><![CDATA[http://ieeexplore.ieee.org/stamp/stamp.jsp?arnumber=6776398]]></pdf>
##   </document>
##   <document>
##     <rank>739</rank>
##     <title><![CDATA[Performance Analysis of Free-Space Optical Communication Systems With Multiuser Diversity Over Atmospheric Turbulence Channels]]></title>
##     <authors><![CDATA[Liang Yang;  Xiqi Gao;  Alouini, M.-S.]]></authors>
##     <affiliations><![CDATA[Sch. of Inf. Eng., Guangdong Univ. of Technol., Guangzhou, China]]></affiliations>
##     <controlledterms>
##       <term><![CDATA[atmospheric turbulence]]></term>
##       <term><![CDATA[channel capacity]]></term>
##       <term><![CDATA[error statistics]]></term>
##       <term><![CDATA[fading channels]]></term>
##       <term><![CDATA[optical communication]]></term>
##       <term><![CDATA[statistical analysis]]></term>
##     </controlledterms>
##     <thesaurusterms>
##       <term><![CDATA[Adaptive optics]]></term>
##       <term><![CDATA[Apertures]]></term>
##       <term><![CDATA[Atmospheric modeling]]></term>
##       <term><![CDATA[Fading]]></term>
##       <term><![CDATA[Log-normal distribution]]></term>
##       <term><![CDATA[MIMO]]></term>
##       <term><![CDATA[Optical fiber communication]]></term>
##     </thesaurusterms>
##     <pubtitle><![CDATA[Photonics Journal, IEEE]]></pubtitle>
##     <punumber><![CDATA[4563994]]></punumber>
##     <pubtype><![CDATA[Journals & Magazines]]></pubtype>
##     <publisher><![CDATA[IEEE]]></publisher>
##     <volume><![CDATA[6]]></volume>
##     <issue><![CDATA[2]]></issue>
##     <py><![CDATA[2014]]></py>
##     <spage><![CDATA[1]]></spage>
##     <epage><![CDATA[17]]></epage>
##     <abstract><![CDATA[Free-space optical (FSO) communication has become a cost-effective method to provide high data rates. However, the turbulence-induced fading limits its application to short-range applications. To address this, we propose a multiuser diversity (MD) FSO scheme in which the Nth best user is selected and the channel fluctuations can be effectively exploited to produce a selection diversity gain. More specifically, we first present the statistics analysis for the considered system over both weak and strong atmospheric turbulence channels. Based on these statistics, the outage probability, bit-error rate performance, average capacity, diversity order, and coverage are analyzed. Results show that the diversity order for the gamma-gamma fading is N min{&#x03B1;, &#x03B2;}/2, where N is the number of users, and &#x03B1; and &#x03B2; are the channel fading parameters related to the effective atmospheric conditions of the link.]]></abstract>
##     <issn><![CDATA[1943-0655]]></issn>
##     <htmlFlag><![CDATA[1]]></htmlFlag>
##     <arnumber><![CDATA[6766188]]></arnumber>
##     <doi><![CDATA[10.1109/JPHOT.2014.2311446]]></doi>
##     <publicationId><![CDATA[6766188]]></publicationId>
##     <mdurl><![CDATA[http://ieeexplore.ieee.org/xpl/articleDetails.jsp?tp=&arnumber=6766188&contentType=Journals+%26+Magazines]]></mdurl>
##     <pdf><![CDATA[http://ieeexplore.ieee.org/stamp/stamp.jsp?arnumber=6766188]]></pdf>
##   </document>
##   <document>
##     <rank>740</rank>
##     <title><![CDATA[Self-Aligned Electrodes on SU-8 Negative Photoresist Pedestals]]></title>
##     <authors><![CDATA[Fok, H.R.;  Jackson, T.N.]]></authors>
##     <affiliations><![CDATA[Dept. of Electr. Eng., Pennsylvania State Univ., University Park, PA, USA]]></affiliations>
##     <controlledterms>
##       <term><![CDATA[aluminium]]></term>
##       <term><![CDATA[electrodes]]></term>
##       <term><![CDATA[masks]]></term>
##       <term><![CDATA[micromechanical devices]]></term>
##       <term><![CDATA[photoresists]]></term>
##       <term><![CDATA[self-assembly]]></term>
##       <term><![CDATA[thin films]]></term>
##       <term><![CDATA[titanium]]></term>
##     </controlledterms>
##     <thesaurusterms>
##       <term><![CDATA[Aluminum]]></term>
##       <term><![CDATA[Electrodes]]></term>
##       <term><![CDATA[Etching]]></term>
##       <term><![CDATA[Micromechanical devices]]></term>
##       <term><![CDATA[Resists]]></term>
##       <term><![CDATA[Titanium]]></term>
##     </thesaurusterms>
##     <pubtitle><![CDATA[Microelectromechanical Systems, Journal of]]></pubtitle>
##     <punumber><![CDATA[84]]></punumber>
##     <pubtype><![CDATA[Journals & Magazines]]></pubtype>
##     <publisher><![CDATA[IEEE]]></publisher>
##     <volume><![CDATA[23]]></volume>
##     <issue><![CDATA[3]]></issue>
##     <py><![CDATA[2014]]></py>
##     <spage><![CDATA[508]]></spage>
##     <epage><![CDATA[510]]></epage>
##     <abstract><![CDATA[We report a novel technique by which self-aligned thin-film electrodes are fabricated on SU-8 negative photoresist pedestals. A bilayer aluminum and titanium structure is used to align the thin-film electrode and to serve as the optical mask for the UV exposure of the SU-8. The SU-8 developer is used to remove both the unexposed SU-8 and the bilayer structure of aluminum and titanium. The result is a thin-film electrode aligned on an SU-8 pedestal with a minimal undercut beneath the thin-film electrode.]]></abstract>
##     <issn><![CDATA[1057-7157]]></issn>
##     <htmlFlag><![CDATA[1]]></htmlFlag>
##     <arnumber><![CDATA[6784311]]></arnumber>
##     <doi><![CDATA[10.1109/JMEMS.2014.2313723]]></doi>
##     <publicationId><![CDATA[6784311]]></publicationId>
##     <mdurl><![CDATA[http://ieeexplore.ieee.org/xpl/articleDetails.jsp?tp=&arnumber=6784311&contentType=Journals+%26+Magazines]]></mdurl>
##     <pdf><![CDATA[http://ieeexplore.ieee.org/stamp/stamp.jsp?arnumber=6784311]]></pdf>
##   </document>
##   <document>
##     <rank>741</rank>
##     <title><![CDATA[Optical Vector Network Analyzer With Improved Accuracy Based on Brillouin-Assisted Optical Carrier Processing]]></title>
##     <authors><![CDATA[Wen Ting Wang;  Wei Li;  Jian Guo Liu;  Wen Hui Sun;  Wei Yu Wang;  Ning Hua Zhu]]></authors>
##     <affiliations><![CDATA[State Key Lab. on Integrated Optoelectron., Inst. of Semicond., Beijing, China]]></affiliations>
##     <controlledterms>
##       <term><![CDATA[measurement errors]]></term>
##       <term><![CDATA[microwave photonics]]></term>
##       <term><![CDATA[optical fibre dispersion]]></term>
##       <term><![CDATA[stimulated Brillouin scattering]]></term>
##       <term><![CDATA[vectors]]></term>
##     </controlledterms>
##     <thesaurusterms>
##       <term><![CDATA[Adaptive optics]]></term>
##       <term><![CDATA[Amplitude modulation]]></term>
##       <term><![CDATA[Optical attenuators]]></term>
##       <term><![CDATA[Optical fibers]]></term>
##       <term><![CDATA[Optical filters]]></term>
##       <term><![CDATA[Optical modulation]]></term>
##     </thesaurusterms>
##     <pubtitle><![CDATA[Photonics Journal, IEEE]]></pubtitle>
##     <punumber><![CDATA[4563994]]></punumber>
##     <pubtype><![CDATA[Journals & Magazines]]></pubtype>
##     <publisher><![CDATA[IEEE]]></publisher>
##     <volume><![CDATA[6]]></volume>
##     <issue><![CDATA[6]]></issue>
##     <py><![CDATA[2014]]></py>
##     <spage><![CDATA[1]]></spage>
##     <epage><![CDATA[10]]></epage>
##     <abstract><![CDATA[We propose a new method to improve the accuracy of an optical vector network analyzer (OVNA) based on stimulated Brillouin scattering (SBS) in a dispersion-shifted fiber (DSF). Generally, the measurement error of the OVNA mainly derives from the beat signals between adjacent optical sidebands. In this paper, the measurement error is significantly suppressed using a two-step measurement process. For the first measurement, both the transmission response of an optical device-under-test (ODUT) and the measurement error are recorded. For the second measurement, only the measurement error is obtained by suppressing the optical carrier using SBS in the DSF. The accurate transmission response of the ODUT is subsequently extracted by subtracting the error from the first measurement. The proposed method is theoretically investigated and experimentally demonstrated. The experimental results show that the accuracy of the OVNA is significantly improved.]]></abstract>
##     <issn><![CDATA[1943-0655]]></issn>
##     <htmlFlag><![CDATA[1]]></htmlFlag>
##     <arnumber><![CDATA[6923459]]></arnumber>
##     <doi><![CDATA[10.1109/JPHOT.2014.2363430]]></doi>
##     <publicationId><![CDATA[6923459]]></publicationId>
##     <mdurl><![CDATA[http://ieeexplore.ieee.org/xpl/articleDetails.jsp?tp=&arnumber=6923459&contentType=Journals+%26+Magazines]]></mdurl>
##     <pdf><![CDATA[http://ieeexplore.ieee.org/stamp/stamp.jsp?arnumber=6923459]]></pdf>
##   </document>
##   <document>
##     <rank>742</rank>
##     <title><![CDATA[Reflectivity and Bandwidth Modulation of Fiber Bragg Gratings in a Suspended Core Fiber by Tunable Acoustic Waves]]></title>
##     <authors><![CDATA[Silva, R.E.;  Becker, M.;  Hartung, A.;  Rothhardt, M.;  Pohl, A.A.P.;  Bartelt, H.]]></authors>
##     <affiliations><![CDATA[Leibniz Inst. of Photonic Technol. (IPHT), Jena, Germany]]></affiliations>
##     <controlledterms>
##       <term><![CDATA[Bragg gratings]]></term>
##       <term><![CDATA[acousto-optical devices]]></term>
##       <term><![CDATA[acousto-optical effects]]></term>
##       <term><![CDATA[optical fibres]]></term>
##       <term><![CDATA[optical modulation]]></term>
##       <term><![CDATA[reflectivity]]></term>
##     </controlledterms>
##     <thesaurusterms>
##       <term><![CDATA[Bandwidth]]></term>
##       <term><![CDATA[Fiber gratings]]></term>
##       <term><![CDATA[Gratings]]></term>
##       <term><![CDATA[Modulation]]></term>
##       <term><![CDATA[Optical fiber devices]]></term>
##       <term><![CDATA[Reflectivity]]></term>
##     </thesaurusterms>
##     <pubtitle><![CDATA[Photonics Journal, IEEE]]></pubtitle>
##     <punumber><![CDATA[4563994]]></punumber>
##     <pubtype><![CDATA[Journals & Magazines]]></pubtype>
##     <publisher><![CDATA[IEEE]]></publisher>
##     <volume><![CDATA[6]]></volume>
##     <issue><![CDATA[6]]></issue>
##     <py><![CDATA[2014]]></py>
##     <spage><![CDATA[1]]></spage>
##     <epage><![CDATA[8]]></epage>
##     <abstract><![CDATA[The acousto-optic modulation of fiber Bragg gratings in a four-hole suspended core fiber is experimentally demonstrated. Strong modulations with a reflectivity amplitude decrease by up to 67% and a 57% bandwidth increase in the Bragg resonance are obtained for gratings of 0.26- and 1-nm 3-dB bandwidths, respectively. The reduction of the required acoustic power for achieving the acousto-optic modulation compared to conventional solid-core single-mode fibers points to more efficient modulator devices in suspended core fibers.]]></abstract>
##     <issn><![CDATA[1943-0655]]></issn>
##     <htmlFlag><![CDATA[1]]></htmlFlag>
##     <arnumber><![CDATA[6945243]]></arnumber>
##     <doi><![CDATA[10.1109/JPHOT.2014.2366161]]></doi>
##     <publicationId><![CDATA[6945243]]></publicationId>
##     <mdurl><![CDATA[http://ieeexplore.ieee.org/xpl/articleDetails.jsp?tp=&arnumber=6945243&contentType=Journals+%26+Magazines]]></mdurl>
##     <pdf><![CDATA[http://ieeexplore.ieee.org/stamp/stamp.jsp?arnumber=6945243]]></pdf>
##   </document>
##   <document>
##     <rank>743</rank>
##     <title><![CDATA[Comparison of Split-Step Fourier Schemes for Simulating Fiber Optic Communication Systems]]></title>
##     <authors><![CDATA[Jing Shao;  Xiaojun Liang;  Kumar, S.]]></authors>
##     <affiliations><![CDATA[Dept. of Electr. & Comput. Eng., McMaster Univ., Hamilton, ON, Canada]]></affiliations>
##     <controlledterms>
##       <term><![CDATA[Fourier transform optics]]></term>
##       <term><![CDATA[Schrodinger equation]]></term>
##       <term><![CDATA[fast Fourier transforms]]></term>
##       <term><![CDATA[gradient methods]]></term>
##       <term><![CDATA[nonlinear optics]]></term>
##       <term><![CDATA[optical fibre communication]]></term>
##       <term><![CDATA[optical fibre dispersion]]></term>
##       <term><![CDATA[optical fibre losses]]></term>
##       <term><![CDATA[quadrature phase shift keying]]></term>
##     </controlledterms>
##     <thesaurusterms>
##       <term><![CDATA[Accuracy]]></term>
##       <term><![CDATA[Adaptive optics]]></term>
##       <term><![CDATA[Computational efficiency]]></term>
##       <term><![CDATA[Optical fiber dispersion]]></term>
##       <term><![CDATA[Optical fibers]]></term>
##       <term><![CDATA[Optical losses]]></term>
##     </thesaurusterms>
##     <pubtitle><![CDATA[Photonics Journal, IEEE]]></pubtitle>
##     <punumber><![CDATA[4563994]]></punumber>
##     <pubtype><![CDATA[Journals & Magazines]]></pubtype>
##     <publisher><![CDATA[IEEE]]></publisher>
##     <volume><![CDATA[6]]></volume>
##     <issue><![CDATA[4]]></issue>
##     <py><![CDATA[2014]]></py>
##     <spage><![CDATA[1]]></spage>
##     <epage><![CDATA[15]]></epage>
##     <abstract><![CDATA[This paper mainly focuses on efficient schemes for simulating propagation in optical fibers. Various schemes based on split-step Fourier techniques to solve the nonlinear Schro&#x0308;dinger equation (NLSE), which describes the propagation in optical fibers, are compared. In general, the schemes in which the loss operator is combined with nonlinearity operator are found to be more computationally efficient than the schemes in which the loss is combined with dispersion. When the global error is large, the schemes with variable step size outperform the ones with uniform step size. The schemes based on local error and/or minimum area mismatch (MAM) further improve the computational efficiency. In this scheme, by minimizing the area mismatch between the exponential profile and its stepwise approximation, an optimal step size distribution is found. The optimization problem is solved by the steepest descent algorithm. The number of steps to get the desired accuracy is determined by the local error method. The proposed scheme is found to have higher computational efficiency than the other schemes studied in this paper. For QPSK systems, when the global error is 10<sup>-8</sup>, the number of fast Fourier transforms (FFTs) needed for the conventional scheme (loss combined with dispersion and uniform step size) is 5.8 times that of the proposed scheme. When the global error is 10<sup>-6</sup>, the number of FFTs needed for the conventional scheme is 3.7 times that of the proposed scheme.]]></abstract>
##     <issn><![CDATA[1943-0655]]></issn>
##     <htmlFlag><![CDATA[1]]></htmlFlag>
##     <arnumber><![CDATA[6860304]]></arnumber>
##     <doi><![CDATA[10.1109/JPHOT.2014.2340993]]></doi>
##     <publicationId><![CDATA[6860304]]></publicationId>
##     <mdurl><![CDATA[http://ieeexplore.ieee.org/xpl/articleDetails.jsp?tp=&arnumber=6860304&contentType=Journals+%26+Magazines]]></mdurl>
##     <pdf><![CDATA[http://ieeexplore.ieee.org/stamp/stamp.jsp?arnumber=6860304]]></pdf>
##   </document>
##   <document>
##     <rank>744</rank>
##     <title><![CDATA[Robust Silicon Waveguide Polarization Rotator With an Amorphous Silicon Overlayer]]></title>
##     <authors><![CDATA[Yule Xiong;  Dan-Xia Xu;  Schmid, J.H.;  Cheben, P.;  Janz, S.;  Ye, W.N.]]></authors>
##     <affiliations><![CDATA[Nat. Res. Council Canada, Inf. & Commun. Technol., Ottawa, ON, Canada]]></affiliations>
##     <controlledterms>
##       <term><![CDATA[amorphous semiconductors]]></term>
##       <term><![CDATA[elemental semiconductors]]></term>
##       <term><![CDATA[optical polarisers]]></term>
##       <term><![CDATA[optical rotation]]></term>
##       <term><![CDATA[optical waveguide components]]></term>
##       <term><![CDATA[silicon]]></term>
##       <term><![CDATA[silicon compounds]]></term>
##     </controlledterms>
##     <thesaurusterms>
##       <term><![CDATA[Amorphous silicon]]></term>
##       <term><![CDATA[Fabrication]]></term>
##       <term><![CDATA[Optical waveguides]]></term>
##       <term><![CDATA[Photonics]]></term>
##       <term><![CDATA[Strips]]></term>
##       <term><![CDATA[Wires]]></term>
##     </thesaurusterms>
##     <pubtitle><![CDATA[Photonics Journal, IEEE]]></pubtitle>
##     <punumber><![CDATA[4563994]]></punumber>
##     <pubtype><![CDATA[Journals & Magazines]]></pubtype>
##     <publisher><![CDATA[IEEE]]></publisher>
##     <volume><![CDATA[6]]></volume>
##     <issue><![CDATA[2]]></issue>
##     <py><![CDATA[2014]]></py>
##     <spage><![CDATA[1]]></spage>
##     <epage><![CDATA[8]]></epage>
##     <abstract><![CDATA[We propose a robust polarization rotator based on the mode-evolution mechanism. The polarization rotation in a silicon wire waveguide is achieved by forming an amorphous silicon (a-Si) overlayer and an SiO2 spacer on top of the waveguide. A strip pattern of a constant width is designed to be etched through the overlayer at a specific angle with respect to the Si waveguide. The asymmetry in the a-Si overlayer affects the waveguide mode by rotating the modal axis. This polarization rotator design is amenable to comparatively simple fabrication compatible with standard silicon photonic processing for integration. The length of the rotation section is 17 &#x03BC;m, and the broadband operation is achieved with a rotation efficiency higher than 90% for a wavelength range exceeding 135 nm. A maximum polarization rotation efficiency of 99.5% is predicted by calculation.]]></abstract>
##     <issn><![CDATA[1943-0655]]></issn>
##     <htmlFlag><![CDATA[1]]></htmlFlag>
##     <arnumber><![CDATA[6744600]]></arnumber>
##     <doi><![CDATA[10.1109/JPHOT.2014.2306827]]></doi>
##     <publicationId><![CDATA[6744600]]></publicationId>
##     <mdurl><![CDATA[http://ieeexplore.ieee.org/xpl/articleDetails.jsp?tp=&arnumber=6744600&contentType=Journals+%26+Magazines]]></mdurl>
##     <pdf><![CDATA[http://ieeexplore.ieee.org/stamp/stamp.jsp?arnumber=6744600]]></pdf>
##   </document>
##   <document>
##     <rank>745</rank>
##     <title><![CDATA[Accurate and Fully Automatic Hippocampus Segmentation Using Subject-Specific 3D Optimal Local Maps Into a Hybrid Active Contour Model]]></title>
##     <authors><![CDATA[Zarpalas, D.;  Gkontra, P.;  Daras, P.;  Maglaveras, N.]]></authors>
##     <affiliations><![CDATA[Centre for Res. & Technol. Hellas, Inf. Technol. Inst., Thessaloniki, Greece]]></affiliations>
##     <controlledterms>
##       <term><![CDATA[biomedical MRI]]></term>
##       <term><![CDATA[brain]]></term>
##       <term><![CDATA[image segmentation]]></term>
##       <term><![CDATA[medical disorders]]></term>
##       <term><![CDATA[medical image processing]]></term>
##       <term><![CDATA[optimisation]]></term>
##     </controlledterms>
##     <thesaurusterms>
##       <term><![CDATA[Active appearance model]]></term>
##       <term><![CDATA[Active contours]]></term>
##       <term><![CDATA[Algorithm design and analysis]]></term>
##       <term><![CDATA[Biomedical imaging]]></term>
##       <term><![CDATA[Hippocampus]]></term>
##       <term><![CDATA[Image edge detection]]></term>
##       <term><![CDATA[Image segmentation]]></term>
##       <term><![CDATA[Solid modeling]]></term>
##     </thesaurusterms>
##     <pubtitle><![CDATA[Translational Engineering in Health and Medicine, IEEE Journal of]]></pubtitle>
##     <punumber><![CDATA[6221039]]></punumber>
##     <pubtype><![CDATA[Journals & Magazines]]></pubtype>
##     <publisher><![CDATA[IEEE]]></publisher>
##     <volume><![CDATA[2]]></volume>
##     <py><![CDATA[2014]]></py>
##     <spage><![CDATA[1]]></spage>
##     <epage><![CDATA[16]]></epage>
##     <abstract><![CDATA[Assessing the structural integrity of the hippocampus (HC) is an essential step toward prevention, diagnosis, and follow-up of various brain disorders due to the implication of the structural changes of the HC in those disorders. In this respect, the development of automatic segmentation methods that can accurately, reliably, and reproducibly segment the HC has attracted considerable attention over the past decades. This paper presents an innovative 3-D fully automatic method to be used on top of the multiatlas concept for the HC segmentation. The method is based on a subject-specific set of 3-D optimal local maps (OLMs) that locally control the influence of each energy term of a hybrid active contour model (ACM). The complete set of the OLMs for a set of training images is defined simultaneously via an optimization scheme. At the same time, the optimal ACM parameters are also calculated. Therefore, heuristic parameter fine-tuning is not required. Training OLMs are subsequently combined, by applying an extended multiatlas concept, to produce the OLMs that are anatomically more suitable to the test image. The proposed algorithm was tested on three different and publicly available data sets. Its accuracy was compared with that of state-of-the-art methods demonstrating the efficacy and robustness of the proposed method.]]></abstract>
##     <issn><![CDATA[2168-2372]]></issn>
##     <htmlFlag><![CDATA[1]]></htmlFlag>
##     <arnumber><![CDATA[6702429]]></arnumber>
##     <doi><![CDATA[10.1109/JTEHM.2014.2297953]]></doi>
##     <publicationId><![CDATA[6702429]]></publicationId>
##     <mdurl><![CDATA[http://ieeexplore.ieee.org/xpl/articleDetails.jsp?tp=&arnumber=6702429&contentType=Journals+%26+Magazines]]></mdurl>
##     <pdf><![CDATA[http://ieeexplore.ieee.org/stamp/stamp.jsp?arnumber=6702429]]></pdf>
##   </document>
##   <document>
##     <rank>746</rank>
##     <title><![CDATA[Time-Frequency Analysis as Probabilistic Inference]]></title>
##     <authors><![CDATA[Turner, R.E.;  Sahani, M.]]></authors>
##     <affiliations><![CDATA[Dept. of Eng., Univ. of Cambridge, Cambridge, UK]]></affiliations>
##     <controlledterms>
##       <term><![CDATA[Bayes methods]]></term>
##       <term><![CDATA[Fourier transforms]]></term>
##       <term><![CDATA[Kalman filters]]></term>
##       <term><![CDATA[Wiener filters]]></term>
##       <term><![CDATA[channel bank filters]]></term>
##       <term><![CDATA[matrix decomposition]]></term>
##       <term><![CDATA[source separation]]></term>
##       <term><![CDATA[time-frequency analysis]]></term>
##     </controlledterms>
##     <thesaurusterms>
##       <term><![CDATA[Hidden Markov models]]></term>
##       <term><![CDATA[Noise]]></term>
##       <term><![CDATA[Noise reduction]]></term>
##       <term><![CDATA[Probabilistic logic]]></term>
##       <term><![CDATA[Spectrogram]]></term>
##       <term><![CDATA[Time-frequency analysis]]></term>
##       <term><![CDATA[Uncertainty]]></term>
##     </thesaurusterms>
##     <pubtitle><![CDATA[Signal Processing, IEEE Transactions on]]></pubtitle>
##     <punumber><![CDATA[78]]></punumber>
##     <pubtype><![CDATA[Journals & Magazines]]></pubtype>
##     <publisher><![CDATA[IEEE]]></publisher>
##     <volume><![CDATA[62]]></volume>
##     <issue><![CDATA[23]]></issue>
##     <py><![CDATA[2014]]></py>
##     <spage><![CDATA[6171]]></spage>
##     <epage><![CDATA[6183]]></epage>
##     <abstract><![CDATA[This paper proposes a new view of time-frequency analysis framed in terms of probabilistic inference. Natural signals are assumed to be formed by the superposition of distinct time-frequency components, with the analytic goal being to infer these components by application of Bayes' rule. The framework serves to unify various existing models for natural time-series; it relates to both the Wiener and Kalman filters, and with suitable assumptions yields inferential interpretations of the short-time Fourier transform, spectrogram, filter bank, and wavelet representations. Value is gained by placing time-frequency analysis on the same probabilistic basis as is often employed in applications such as denoising, source separation, or recognition. Uncertainty in the time-frequency representation can be propagated correctly to application-specific stages, improving the handing of noise and missing data. Probabilistic learning allows modules to be co-adapted; thus, the time-frequency representation can be adapted to both the demands of the application and the time-varying statistics of the signal at hand. Similarly, the application module can be adapted to fine properties of the signal propagated by the initial time-frequency processing. We demonstrate these benefits by combining probabilistic time-frequency representations with non-negative matrix factorization, finding benefits in audio denoising and inpainting tasks, albeit with higher computational cost than incurred by the standard approach.]]></abstract>
##     <issn><![CDATA[1053-587X]]></issn>
##     <arnumber><![CDATA[6918491]]></arnumber>
##     <doi><![CDATA[10.1109/TSP.2014.2362100]]></doi>
##     <publicationId><![CDATA[6918491]]></publicationId>
##     <mdurl><![CDATA[http://ieeexplore.ieee.org/xpl/articleDetails.jsp?tp=&arnumber=6918491&contentType=Journals+%26+Magazines]]></mdurl>
##     <pdf><![CDATA[http://ieeexplore.ieee.org/stamp/stamp.jsp?arnumber=6918491]]></pdf>
##   </document>
##   <document>
##     <rank>747</rank>
##     <title><![CDATA[Effective LAI and CHP of a Single Tree From Small-Footprint Full-Waveform LiDAR]]></title>
##     <authors><![CDATA[Fieber, K.D.;  Davenport, I.J.;  Tanase, M.A.;  Ferryman, J.M.;  Gurney, R.J.;  Walker, J.P.;  Hacker, J.M.]]></authors>
##     <affiliations><![CDATA[Sch. of Syst. Eng., Univ. of Reading, Reading, UK]]></affiliations>
##     <controlledterms>
##       <term><![CDATA[geophysical techniques]]></term>
##       <term><![CDATA[remote sensing by laser beam]]></term>
##       <term><![CDATA[vegetation]]></term>
##     </controlledterms>
##     <thesaurusterms>
##       <term><![CDATA[Cameras]]></term>
##       <term><![CDATA[Cogeneration]]></term>
##       <term><![CDATA[Correlation]]></term>
##       <term><![CDATA[Indexes]]></term>
##       <term><![CDATA[Laser radar]]></term>
##       <term><![CDATA[Photography]]></term>
##       <term><![CDATA[Vegetation]]></term>
##     </thesaurusterms>
##     <pubtitle><![CDATA[Geoscience and Remote Sensing Letters, IEEE]]></pubtitle>
##     <punumber><![CDATA[8859]]></punumber>
##     <pubtype><![CDATA[Journals & Magazines]]></pubtype>
##     <publisher><![CDATA[IEEE]]></publisher>
##     <volume><![CDATA[11]]></volume>
##     <issue><![CDATA[9]]></issue>
##     <py><![CDATA[2014]]></py>
##     <spage><![CDATA[1634]]></spage>
##     <epage><![CDATA[1638]]></epage>
##     <abstract><![CDATA[This letter has tested the canopy height profile (CHP) methodology as a way of effective leaf area index (LAIe) and vertical vegetation profile retrieval at a single-tree level. Waveform and discrete airborne LiDAR data from six swaths, as well as from the combined data of six swaths, were used to extract the LAIe of a single live Callitris glaucophylla tree. LAIe was extracted from raw waveform as an intermediate step in the CHP methodology, with two different vegetation-ground reflectance ratios. Discrete point LAIe estimates were derived from the gap probability using the following: 1) single ground returns and 2) all ground returns. LiDAR LAIe retrievals were subsequently compared to hemispherical photography estimates, yielding mean values within &#x00B1;7% of the latter, depending on the method used. The CHP of a single dead Callitris glaucophylla tree, representing the distribution of vegetation material, was verified with a field profile manually reconstructed from convergent photographs taken with a fixed-focal-length camera. A binwise comparison of the two profiles showed very high correlation between the data reaching R<sup>2</sup> of 0.86 for the CHP from combined swaths. Using a study-area-adjusted reflectance ratio improved the correlation between the profiles, but only marginally in comparison to using an arbitrary ratio of 0.5 for the laser wavelength of 1550 nm.]]></abstract>
##     <issn><![CDATA[1545-598X]]></issn>
##     <htmlFlag><![CDATA[1]]></htmlFlag>
##     <arnumber><![CDATA[6744570]]></arnumber>
##     <doi><![CDATA[10.1109/LGRS.2014.2303500]]></doi>
##     <publicationId><![CDATA[6744570]]></publicationId>
##     <mdurl><![CDATA[http://ieeexplore.ieee.org/xpl/articleDetails.jsp?tp=&arnumber=6744570&contentType=Journals+%26+Magazines]]></mdurl>
##     <pdf><![CDATA[http://ieeexplore.ieee.org/stamp/stamp.jsp?arnumber=6744570]]></pdf>
##   </document>
##   <document>
##     <rank>748</rank>
##     <title><![CDATA[Fabrication and Characteristics of Ce-Doped Fiber for High-Resolution OCT Source]]></title>
##     <authors><![CDATA[Chun-Nien Liu;  Yi-Chung Huang;  Yen-Sheng Lin;  Sheng-Yuan Wang;  Pi-Ling Huang;  Tien-Tsorng Shih;  Sheng-Lung Huang;  Wood-Hi Cheng]]></authors>
##     <affiliations><![CDATA[Dept. of Photonics, Nat. Sun Yat-sen Univ., Kaohsiung, Taiwan]]></affiliations>
##     <controlledterms>
##       <term><![CDATA[biomedical optical imaging]]></term>
##       <term><![CDATA[cerium]]></term>
##       <term><![CDATA[fluorescence]]></term>
##       <term><![CDATA[optical fibre fabrication]]></term>
##       <term><![CDATA[optical tomography]]></term>
##     </controlledterms>
##     <thesaurusterms>
##       <term><![CDATA[Broadband communication]]></term>
##       <term><![CDATA[Coherence]]></term>
##       <term><![CDATA[Fluorescence]]></term>
##       <term><![CDATA[Light sources]]></term>
##       <term><![CDATA[Optical fiber amplifiers]]></term>
##     </thesaurusterms>
##     <pubtitle><![CDATA[Photonics Technology Letters, IEEE]]></pubtitle>
##     <punumber><![CDATA[68]]></punumber>
##     <pubtype><![CDATA[Journals & Magazines]]></pubtype>
##     <publisher><![CDATA[IEEE]]></publisher>
##     <volume><![CDATA[26]]></volume>
##     <issue><![CDATA[15]]></issue>
##     <py><![CDATA[2014]]></py>
##     <spage><![CDATA[1499]]></spage>
##     <epage><![CDATA[1502]]></epage>
##     <abstract><![CDATA[The fabrication of Ce-doped fibers (CeDFs) is demonstrated by drawing-tower method employing rod-in-tube technique. The fluorescence spectrum of CeDFs with a 16-&#x03BC;m core exhibited a 160-nm broadband emission with 1.45-&#x03BC;m axial resolution. This CeDF may be functioned as a high-resolution light source for optical coherence tomography applications.]]></abstract>
##     <issn><![CDATA[1041-1135]]></issn>
##     <htmlFlag><![CDATA[1]]></htmlFlag>
##     <arnumber><![CDATA[6823173]]></arnumber>
##     <doi><![CDATA[10.1109/LPT.2014.2327127]]></doi>
##     <publicationId><![CDATA[6823173]]></publicationId>
##     <mdurl><![CDATA[http://ieeexplore.ieee.org/xpl/articleDetails.jsp?tp=&arnumber=6823173&contentType=Journals+%26+Magazines]]></mdurl>
##     <pdf><![CDATA[http://ieeexplore.ieee.org/stamp/stamp.jsp?arnumber=6823173]]></pdf>
##   </document>
##   <document>
##     <rank>749</rank>
##     <title><![CDATA[Multiple Optical Compensation in Interferometric Fiber-Optic Gyroscope for Polarization Nonreciprocal Error Suppression]]></title>
##     <authors><![CDATA[Ping Lu;  Zinan Wang;  Yi Yang;  Duyu Zhao;  Siyadong Xiong;  Yulin Li;  Chao Peng;  Zhengbin Li]]></authors>
##     <affiliations><![CDATA[Dept. of Electron., Peking Univ., Beijing, China]]></affiliations>
##     <controlledterms>
##       <term><![CDATA[Sagnac interferometers]]></term>
##       <term><![CDATA[compensation]]></term>
##       <term><![CDATA[fibre optic gyroscopes]]></term>
##       <term><![CDATA[optical fibre polarisation]]></term>
##       <term><![CDATA[stability]]></term>
##     </controlledterms>
##     <thesaurusterms>
##       <term><![CDATA[Adaptive optics]]></term>
##       <term><![CDATA[Angular velocity]]></term>
##       <term><![CDATA[Coils]]></term>
##       <term><![CDATA[Optical fibers]]></term>
##       <term><![CDATA[Optical interferometry]]></term>
##       <term><![CDATA[Optical polarization]]></term>
##     </thesaurusterms>
##     <pubtitle><![CDATA[Photonics Journal, IEEE]]></pubtitle>
##     <punumber><![CDATA[4563994]]></punumber>
##     <pubtype><![CDATA[Journals & Magazines]]></pubtype>
##     <publisher><![CDATA[IEEE]]></publisher>
##     <volume><![CDATA[6]]></volume>
##     <issue><![CDATA[5]]></issue>
##     <py><![CDATA[2014]]></py>
##     <spage><![CDATA[1]]></spage>
##     <epage><![CDATA[8]]></epage>
##     <abstract><![CDATA[Polarization nonreciprocal (PN) errors within a novel interferometric fiber-optic gyroscope (IFOG) configuration are investigated both theoretically and experimentally. Different from those conventional IFOGs, here, two orthogonal polarizations coexist, and light can travel through multiple paths. The PN errors of individual paths possess opposite signs and, thus, can be effectively canceled out through a multiple optical compensation process. As experimentally demonstrated, the long-term stability of IFOG has been remarkably improved. From the perspective of optical compensation, the concept of &#x201C;reciprocity&#x201D; can be understood in a more generalized way.]]></abstract>
##     <issn><![CDATA[1943-0655]]></issn>
##     <htmlFlag><![CDATA[1]]></htmlFlag>
##     <arnumber><![CDATA[6872816]]></arnumber>
##     <doi><![CDATA[10.1109/JPHOT.2014.2345887]]></doi>
##     <publicationId><![CDATA[6872816]]></publicationId>
##     <mdurl><![CDATA[http://ieeexplore.ieee.org/xpl/articleDetails.jsp?tp=&arnumber=6872816&contentType=Journals+%26+Magazines]]></mdurl>
##     <pdf><![CDATA[http://ieeexplore.ieee.org/stamp/stamp.jsp?arnumber=6872816]]></pdf>
##   </document>
##   <document>
##     <rank>750</rank>
##     <title><![CDATA[Efficient Three-Process Frequency Conversion Based on Straddling Stimulated Raman Adiabatic Passage]]></title>
##     <authors><![CDATA[Chao Sun;  Changshui Chen;  Junxiong Wei;  Pengpeng Li]]></authors>
##     <affiliations><![CDATA[Key Lab. of Laser Life Sci. & Inst. of Laser Life Sci., South China Normal Univ., Guangzhou, China]]></affiliations>
##     <controlledterms>
##       <term><![CDATA[optical frequency conversion]]></term>
##       <term><![CDATA[stimulated Raman scattering]]></term>
##     </controlledterms>
##     <thesaurusterms>
##       <term><![CDATA[Couplings]]></term>
##       <term><![CDATA[Nonlinear optics]]></term>
##       <term><![CDATA[Optical coupling]]></term>
##       <term><![CDATA[Optical frequency conversion]]></term>
##       <term><![CDATA[Optical mixing]]></term>
##       <term><![CDATA[Wavelength conversion]]></term>
##     </thesaurusterms>
##     <pubtitle><![CDATA[Photonics Journal, IEEE]]></pubtitle>
##     <punumber><![CDATA[4563994]]></punumber>
##     <pubtype><![CDATA[Journals & Magazines]]></pubtype>
##     <publisher><![CDATA[IEEE]]></publisher>
##     <volume><![CDATA[6]]></volume>
##     <issue><![CDATA[6]]></issue>
##     <py><![CDATA[2014]]></py>
##     <spage><![CDATA[1]]></spage>
##     <epage><![CDATA[10]]></epage>
##     <abstract><![CDATA[The high-efficiency three-process frequency conversion is analyzed. By analysis of coupling equations, the optical straddling stimulated Raman adiabatic passage (S-STIRAP) scheme is proposed. Numerical simulation shows that the input power of the first frequency is monotonically converted into the output power of the last frequency, with the intermediate frequencies kept extremely low along the interacting length. We expect that this frequency conversion scheme has application in the frequency conversion between highly disparate frequencies, e.g., mid-IR to visible.]]></abstract>
##     <issn><![CDATA[1943-0655]]></issn>
##     <htmlFlag><![CDATA[1]]></htmlFlag>
##     <arnumber><![CDATA[6966718]]></arnumber>
##     <doi><![CDATA[10.1109/JPHOT.2014.2374616]]></doi>
##     <publicationId><![CDATA[6966718]]></publicationId>
##     <mdurl><![CDATA[http://ieeexplore.ieee.org/xpl/articleDetails.jsp?tp=&arnumber=6966718&contentType=Journals+%26+Magazines]]></mdurl>
##     <pdf><![CDATA[http://ieeexplore.ieee.org/stamp/stamp.jsp?arnumber=6966718]]></pdf>
##   </document>
##   <document>
##     <rank>751</rank>
##     <title><![CDATA[An SVM-Based Detection for Coherent Optical APSK Systems With Nonlinear Phase Noise]]></title>
##     <authors><![CDATA[Yi Han;  Song Yu;  Minliang Li;  Jie Yang;  Wanyi Gu]]></authors>
##     <affiliations><![CDATA[State Key Lab. of Inf. Photonics & Opt. Commun., Beijing Univ. of Posts & Telecommun., Beijing, China]]></affiliations>
##     <controlledterms>
##       <term><![CDATA[optical fibre communication]]></term>
##       <term><![CDATA[phase noise]]></term>
##       <term><![CDATA[phase shift keying]]></term>
##       <term><![CDATA[support vector machines]]></term>
##     </controlledterms>
##     <thesaurusterms>
##       <term><![CDATA[Detectors]]></term>
##       <term><![CDATA[Dispersion]]></term>
##       <term><![CDATA[Nonlinear optics]]></term>
##       <term><![CDATA[Optical noise]]></term>
##       <term><![CDATA[Phase noise]]></term>
##       <term><![CDATA[Quadrature amplitude modulation]]></term>
##       <term><![CDATA[Support vector machines]]></term>
##     </thesaurusterms>
##     <pubtitle><![CDATA[Photonics Journal, IEEE]]></pubtitle>
##     <punumber><![CDATA[4563994]]></punumber>
##     <pubtype><![CDATA[Journals & Magazines]]></pubtype>
##     <publisher><![CDATA[IEEE]]></publisher>
##     <volume><![CDATA[6]]></volume>
##     <issue><![CDATA[5]]></issue>
##     <py><![CDATA[2014]]></py>
##     <spage><![CDATA[1]]></spage>
##     <epage><![CDATA[10]]></epage>
##     <abstract><![CDATA[A support vector machine (SVM)-based data detection is proposed for coherent optical fiber amplitude phase-shift keying (APSK) communication systems where the nonlinear phase noise is the main system impairment. The performances of the detection with SVMs are investigated for three different 16-APSK modulation formats. In addition, three transmission scenarios with dispersion being considered or not are adopted to simulate and analyze the performances. Compared with the traditional two-stage maximum-likelihood detection, the SVM conducts detection without the need to know the information of transmission link, and it gains a relatively large improvement in the nonlinear system tolerance, particularly in the high nonlinear regime. Compared with quadrature amplitude modulation (QAM), the 16-APSK system can increase the nonlinear system tolerance by 4.88 dB at BER 1/4 1E - 3.]]></abstract>
##     <issn><![CDATA[1943-0655]]></issn>
##     <htmlFlag><![CDATA[1]]></htmlFlag>
##     <arnumber><![CDATA[6897918]]></arnumber>
##     <doi><![CDATA[10.1109/JPHOT.2014.2357424]]></doi>
##     <publicationId><![CDATA[6897918]]></publicationId>
##     <mdurl><![CDATA[http://ieeexplore.ieee.org/xpl/articleDetails.jsp?tp=&arnumber=6897918&contentType=Journals+%26+Magazines]]></mdurl>
##     <pdf><![CDATA[http://ieeexplore.ieee.org/stamp/stamp.jsp?arnumber=6897918]]></pdf>
##   </document>
##   <document>
##     <rank>752</rank>
##     <title><![CDATA[High-Sensitivity Charge-Transfer-Type pH Sensor With Quasi-Signal Removal Structure]]></title>
##     <authors><![CDATA[Nakazawa, H.;  Otake, R.;  Futagawa, M.;  Dasai, F.;  Ishida, M.;  Sawada, K.]]></authors>
##     <affiliations><![CDATA[Integrated Circuit & Sensor Syst. Group, Toyohashi Univ. of Technol., Toyohashi, Japan]]></affiliations>
##     <controlledterms>
##       <term><![CDATA[chemical sensors]]></term>
##       <term><![CDATA[ion sensitive field effect transistors]]></term>
##       <term><![CDATA[pH measurement]]></term>
##     </controlledterms>
##     <thesaurusterms>
##       <term><![CDATA[Actuators]]></term>
##       <term><![CDATA[Educational institutions]]></term>
##       <term><![CDATA[Electric potential]]></term>
##       <term><![CDATA[Electrodes]]></term>
##       <term><![CDATA[Sensitivity]]></term>
##       <term><![CDATA[Sensors]]></term>
##       <term><![CDATA[pH measurement]]></term>
##     </thesaurusterms>
##     <pubtitle><![CDATA[Electron Devices, IEEE Transactions on]]></pubtitle>
##     <punumber><![CDATA[16]]></punumber>
##     <pubtype><![CDATA[Journals & Magazines]]></pubtype>
##     <publisher><![CDATA[IEEE]]></publisher>
##     <volume><![CDATA[61]]></volume>
##     <issue><![CDATA[1]]></issue>
##     <py><![CDATA[2014]]></py>
##     <spage><![CDATA[136]]></spage>
##     <epage><![CDATA[140]]></epage>
##     <abstract><![CDATA[Charge-transfer-type pH sensors can be used to improve pH sensitivity with enhanced signal-to-noise ratio by applying the charge-accumulation technique. Theoretically, the pH sensitivity improves directly with the accumulated count. However, in a conventional sensor structure, a quasi-signal resulting from low charge transfer efficiency limits the accumulated count. In this paper, an effective solution to the problem of the quasi-signal and the novel sensor structure are investigated.]]></abstract>
##     <issn><![CDATA[0018-9383]]></issn>
##     <htmlFlag><![CDATA[1]]></htmlFlag>
##     <arnumber><![CDATA[6680697]]></arnumber>
##     <doi><![CDATA[10.1109/TED.2013.2292563]]></doi>
##     <publicationId><![CDATA[6680697]]></publicationId>
##     <mdurl><![CDATA[http://ieeexplore.ieee.org/xpl/articleDetails.jsp?tp=&arnumber=6680697&contentType=Journals+%26+Magazines]]></mdurl>
##     <pdf><![CDATA[http://ieeexplore.ieee.org/stamp/stamp.jsp?arnumber=6680697]]></pdf>
##   </document>
##   <document>
##     <rank>753</rank>
##     <title><![CDATA[Spatial Quantification of Cytosolic Ca<inline-formula> <img src="/images/tex/848.gif" alt="^{2+}"> </inline-formula> Accumulation in Nonexcitable Cells:An Analytical Study]]></title>
##     <authors><![CDATA[Lopez-Caamal, F.;  Oyarzun, D.A.;  Middleton, R.H.;  Garcia, M.R.]]></authors>
##     <affiliations><![CDATA[Hamilton Inst., Nat. Univ. of Ireland, Maynooth, Ireland]]></affiliations>
##     <controlledterms>
##       <term><![CDATA[biodiffusion]]></term>
##       <term><![CDATA[bioelectric potentials]]></term>
##       <term><![CDATA[biomembrane transport]]></term>
##       <term><![CDATA[boundary-value problems]]></term>
##       <term><![CDATA[calcium]]></term>
##       <term><![CDATA[filtering theory]]></term>
##       <term><![CDATA[low-pass filters]]></term>
##       <term><![CDATA[muscle]]></term>
##       <term><![CDATA[reaction-diffusion systems]]></term>
##     </controlledterms>
##     <thesaurusterms>
##       <term><![CDATA[Bioinformatics]]></term>
##       <term><![CDATA[Biological system modeling]]></term>
##       <term><![CDATA[Biomembranes]]></term>
##       <term><![CDATA[Computational biology]]></term>
##     </thesaurusterms>
##     <pubtitle><![CDATA[Computational Biology and Bioinformatics, IEEE/ACM Transactions on]]></pubtitle>
##     <punumber><![CDATA[8857]]></punumber>
##     <pubtype><![CDATA[Journals & Magazines]]></pubtype>
##     <publisher><![CDATA[IEEE]]></publisher>
##     <volume><![CDATA[11]]></volume>
##     <issue><![CDATA[3]]></issue>
##     <py><![CDATA[2014]]></py>
##     <spage><![CDATA[592]]></spage>
##     <epage><![CDATA[603]]></epage>
##     <abstract><![CDATA[Calcium ions act as messengers in a broad range of processes such as learning, apoptosis, and muscular movement. The transient profile and the temporal accumulation of calcium signals have been suggested as the two main characteristics in which calcium cues encode messages to be forwarded to downstream pathways. We address the analytical quantification of calcium temporal-accumulation in a long, thin section of a nonexcitable cell by solving a boundary value problem. In these expressions we note that the cytosolic Ca<sup>2+</sup> accumulation is independent of every intracellular calcium flux and depends on the Ca<sup>2+</sup> exchange across the membrane, cytosolic calcium diffusion, geometry of the cell, extracellular calcium perturbation, and initial concentrations. In particular, we analyse the time-integrated response of cytosolic calcium due to i) a localised initial concentration of cytosolic calcium and ii) transient extracellular perturbation of calcium. In these scenarios, we conclude that i) the range of calcium progression is confined to the vicinity of the initial concentration, thereby creating calcium microdomains; and ii) we observe a low-pass filtering effect in the response driven by extracellular Ca<sup>2+</sup> perturbations. Additionally, we note that our methodology can be used to analyse a broader range of stimuli and scenarios.]]></abstract>
##     <issn><![CDATA[1545-5963]]></issn>
##     <htmlFlag><![CDATA[1]]></htmlFlag>
##     <arnumber><![CDATA[6784136]]></arnumber>
##     <doi><![CDATA[10.1109/TCBB.2014.2316010]]></doi>
##     <publicationId><![CDATA[6784136]]></publicationId>
##     <mdurl><![CDATA[http://ieeexplore.ieee.org/xpl/articleDetails.jsp?tp=&arnumber=6784136&contentType=Journals+%26+Magazines]]></mdurl>
##     <pdf><![CDATA[http://ieeexplore.ieee.org/stamp/stamp.jsp?arnumber=6784136]]></pdf>
##   </document>
##   <document>
##     <rank>754</rank>
##     <title><![CDATA[A 1064- and 1074-nm Dual-Wavelength Nd:YAG Laser Using a Fabry&#x2013;Perot Band-pass Filter as Output Mirror]]></title>
##     <authors><![CDATA[Wang, X.Z.;  Wang, Z.F.;  Bu, Y.K.;  Chen, L.J.;  Cai, G.X.;  Cai, Z.P.]]></authors>
##     <affiliations><![CDATA[Dept. of Electron. Eng., Xiamen Univ., Xiamen, China]]></affiliations>
##     <controlledterms>
##       <term><![CDATA[band-pass filters]]></term>
##       <term><![CDATA[laser cavity resonators]]></term>
##       <term><![CDATA[laser mirrors]]></term>
##       <term><![CDATA[neodymium]]></term>
##       <term><![CDATA[optical design techniques]]></term>
##       <term><![CDATA[optical filters]]></term>
##       <term><![CDATA[optical pumping]]></term>
##       <term><![CDATA[semiconductor lasers]]></term>
##       <term><![CDATA[solid lasers]]></term>
##     </controlledterms>
##     <thesaurusterms>
##       <term><![CDATA[Band-pass filters]]></term>
##       <term><![CDATA[Cavity resonators]]></term>
##       <term><![CDATA[Fabry-Perot]]></term>
##       <term><![CDATA[Mirrors]]></term>
##       <term><![CDATA[Optical filters]]></term>
##       <term><![CDATA[Pump lasers]]></term>
##     </thesaurusterms>
##     <pubtitle><![CDATA[Photonics Journal, IEEE]]></pubtitle>
##     <punumber><![CDATA[4563994]]></punumber>
##     <pubtype><![CDATA[Journals & Magazines]]></pubtype>
##     <publisher><![CDATA[IEEE]]></publisher>
##     <volume><![CDATA[6]]></volume>
##     <issue><![CDATA[4]]></issue>
##     <py><![CDATA[2014]]></py>
##     <spage><![CDATA[1]]></spage>
##     <epage><![CDATA[6]]></epage>
##     <abstract><![CDATA[We propose and demonstrate a 1064- and 1074-nm dual-wavelength Nd:YAG laser by exploiting a dielectric Fabry-Perot bandpass filter (FPF) as laser output mirror. A fiber-pigtailed 808-nm laser diode array is used to pump an &lt;;111&gt;-cut Nd:YAG crystal with a plano-plano resonator cavity. The dielectric FPF as output mirror is specially designed to balance the net gain of 1064 and 1074 nm to obtain a dual-wavelength laser. Simultaneous dual-wavelength lasing at 1064 and 1074 nm is successfully achieved. The maximum output power of the laser is 581 mW, and the slope conversion efficiency is 18.8% with the threshold pump power of 2.1 W. The design of the FPF used as output mirror, including the relationship between FWHM and spectral separation, peak wavelength location, and peak transmission, are discussed. Compared with the coupled-cavity, etalon, or specially coated mirror methods, the FPF method presented in this paper is both easy in the selection of oscillating wavelength and simple in design and fabrication.]]></abstract>
##     <issn><![CDATA[1943-0655]]></issn>
##     <htmlFlag><![CDATA[1]]></htmlFlag>
##     <arnumber><![CDATA[6872535]]></arnumber>
##     <doi><![CDATA[10.1109/JPHOT.2014.2345884]]></doi>
##     <publicationId><![CDATA[6872535]]></publicationId>
##     <mdurl><![CDATA[http://ieeexplore.ieee.org/xpl/articleDetails.jsp?tp=&arnumber=6872535&contentType=Journals+%26+Magazines]]></mdurl>
##     <pdf><![CDATA[http://ieeexplore.ieee.org/stamp/stamp.jsp?arnumber=6872535]]></pdf>
##   </document>
##   <document>
##     <rank>755</rank>
##     <title><![CDATA[Partitioning Kinetic Energy During Freewheeling Wheelchair Maneuvers]]></title>
##     <authors><![CDATA[Medola, F.O.;  Dao, P.V.;  Caspall, J.J.;  Sprigle, S.]]></authors>
##     <affiliations><![CDATA[Programme of Post-graduation Interunits in Bioeng., Univ. of Sao Paulo, Sao Carlos, Brazil]]></affiliations>
##     <controlledterms>
##       <term><![CDATA[biomedical equipment]]></term>
##       <term><![CDATA[biomedical measurement]]></term>
##       <term><![CDATA[data acquisition]]></term>
##       <term><![CDATA[handicapped aids]]></term>
##       <term><![CDATA[patient rehabilitation]]></term>
##       <term><![CDATA[velocity measurement]]></term>
##       <term><![CDATA[wheelchairs]]></term>
##     </controlledterms>
##     <thesaurusterms>
##       <term><![CDATA[Acceleration]]></term>
##       <term><![CDATA[Kinematics]]></term>
##       <term><![CDATA[Manuals]]></term>
##       <term><![CDATA[Propulsion]]></term>
##       <term><![CDATA[Turning]]></term>
##       <term><![CDATA[Wheelchairs]]></term>
##       <term><![CDATA[Wheels]]></term>
##     </thesaurusterms>
##     <pubtitle><![CDATA[Neural Systems and Rehabilitation Engineering, IEEE Transactions on]]></pubtitle>
##     <punumber><![CDATA[7333]]></punumber>
##     <pubtype><![CDATA[Journals & Magazines]]></pubtype>
##     <publisher><![CDATA[IEEE]]></publisher>
##     <volume><![CDATA[22]]></volume>
##     <issue><![CDATA[2]]></issue>
##     <py><![CDATA[2014]]></py>
##     <spage><![CDATA[326]]></spage>
##     <epage><![CDATA[333]]></epage>
##     <abstract><![CDATA[This paper describes a systematic method to partition the kinetic energy (KE) of a free-wheeling wheelchair. An ultralightweight rigid frame wheelchair was instrumented with two axle-mounted encoders and data acquisition equipment to accurately measure the velocity of the drive wheels. A mathematical model was created combining physical specifications and geometry of the wheelchair and its components. Two able-bodied subjects propelled the wheelchair over four courses that involved straight and turning maneuvers at differing speeds. The KE of the wheelchair was divided into three components: translational, rotational, and turning energy. This technique was sensitive to the changing contributions of the three energy components across maneuvers. Translational energy represented the major component of total KE in all maneuvers except a zero radius turn in which turning energy was dominant. Both translational and rotational energies are directly related to wheelchair speed. Partitioning KE offers a useful means of investigating the dynamics of a moving wheelchair. The described technique permits analysis of KE imparted to the wheelchair during maneuvers involving changes in speed and direction, which are most representative of mobility in everyday life. This technique can be used to study the effort required to maneuver different types and configurations of wheelchairs.]]></abstract>
##     <issn><![CDATA[1534-4320]]></issn>
##     <htmlFlag><![CDATA[1]]></htmlFlag>
##     <arnumber><![CDATA[6656838]]></arnumber>
##     <doi><![CDATA[10.1109/TNSRE.2013.2289378]]></doi>
##     <publicationId><![CDATA[6656838]]></publicationId>
##     <mdurl><![CDATA[http://ieeexplore.ieee.org/xpl/articleDetails.jsp?tp=&arnumber=6656838&contentType=Journals+%26+Magazines]]></mdurl>
##     <pdf><![CDATA[http://ieeexplore.ieee.org/stamp/stamp.jsp?arnumber=6656838]]></pdf>
##   </document>
##   <document>
##     <rank>756</rank>
##     <title><![CDATA[AutoMashUpper: Automatic Creation of Multi-Song Music Mashups]]></title>
##     <authors><![CDATA[Davies, M.E.P.;  Hamel, P.;  Yoshii, K.;  Goto, M.]]></authors>
##     <affiliations><![CDATA[Nat. Inst. of Adv. Ind. Sci. & Technol. (AIST), Tsukuba, Japan]]></affiliations>
##     <controlledterms>
##       <term><![CDATA[information retrieval]]></term>
##       <term><![CDATA[music]]></term>
##     </controlledterms>
##     <thesaurusterms>
##       <term><![CDATA[Estimation]]></term>
##       <term><![CDATA[Feature extraction]]></term>
##       <term><![CDATA[Harmonic analysis]]></term>
##       <term><![CDATA[IEEE transactions]]></term>
##       <term><![CDATA[Mashups]]></term>
##       <term><![CDATA[Multiple signal classification]]></term>
##       <term><![CDATA[Spectrogram]]></term>
##     </thesaurusterms>
##     <pubtitle><![CDATA[Audio, Speech, and Language Processing, IEEE/ACM Transactions on]]></pubtitle>
##     <punumber><![CDATA[6570655]]></punumber>
##     <pubtype><![CDATA[Journals & Magazines]]></pubtype>
##     <publisher><![CDATA[IEEE]]></publisher>
##     <volume><![CDATA[22]]></volume>
##     <issue><![CDATA[12]]></issue>
##     <py><![CDATA[2014]]></py>
##     <spage><![CDATA[1726]]></spage>
##     <epage><![CDATA[1737]]></epage>
##     <abstract><![CDATA[In this paper we present a system, AutoMashUpper, for making multi-song music mashups. Central to our system is a measure of &#x201C;mashability&#x201D; calculated between phrase sections of an input song and songs in a music collection. We define mashability in terms of harmonic and rhythmic similarity and a measure of spectral balance. The principal novelty in our approach centres on the determination of how elements of songs can be made fit together using key transposition and tempo modification, rather than based on their unaltered properties. In this way, the properties of two songs used to model their mashability can be altered with respect to transformations performed to maximize their perceptual compatibility. AutoMashUpper has a user interface to allow users to control the parameterization of the mashability estimation. It allows users to define ranges for key shifts and tempo as well as adding, changing or removing elements from the created mashups. We evaluate AutoMashUpper by its ability to reliably segment music signals into phrase sections, and also via a listening test to examine the relationship between estimated mashability and user enjoyment.]]></abstract>
##     <issn><![CDATA[2329-9290]]></issn>
##     <htmlFlag><![CDATA[1]]></htmlFlag>
##     <arnumber><![CDATA[6876193]]></arnumber>
##     <doi><![CDATA[10.1109/TASLP.2014.2347135]]></doi>
##     <publicationId><![CDATA[6876193]]></publicationId>
##     <mdurl><![CDATA[http://ieeexplore.ieee.org/xpl/articleDetails.jsp?tp=&arnumber=6876193&contentType=Journals+%26+Magazines]]></mdurl>
##     <pdf><![CDATA[http://ieeexplore.ieee.org/stamp/stamp.jsp?arnumber=6876193]]></pdf>
##   </document>
##   <document>
##     <rank>757</rank>
##     <title><![CDATA[Reaching Law Approach to the Sliding Mode Control of Periodic Review Inventory Systems]]></title>
##     <authors><![CDATA[Bartoszewicz, A.;  Lesniewski, P.]]></authors>
##     <affiliations><![CDATA[Inst. of Autom. Control, Tech. Univ. of Lodz, &#x0141;o&#x0301;dz&#x0301;, Poland]]></affiliations>
##     <controlledterms>
##       <term><![CDATA[control system synthesis]]></term>
##       <term><![CDATA[discrete time systems]]></term>
##       <term><![CDATA[stock control]]></term>
##       <term><![CDATA[variable structure systems]]></term>
##     </controlledterms>
##     <thesaurusterms>
##       <term><![CDATA[Convergence]]></term>
##       <term><![CDATA[Discrete-time systems]]></term>
##       <term><![CDATA[Sliding mode control]]></term>
##       <term><![CDATA[Transportation]]></term>
##       <term><![CDATA[Uncertainty]]></term>
##       <term><![CDATA[Vectors]]></term>
##     </thesaurusterms>
##     <pubtitle><![CDATA[Automation Science and Engineering, IEEE Transactions on]]></pubtitle>
##     <punumber><![CDATA[8856]]></punumber>
##     <pubtype><![CDATA[Journals & Magazines]]></pubtype>
##     <publisher><![CDATA[IEEE]]></publisher>
##     <volume><![CDATA[11]]></volume>
##     <issue><![CDATA[3]]></issue>
##     <py><![CDATA[2014]]></py>
##     <spage><![CDATA[810]]></spage>
##     <epage><![CDATA[817]]></epage>
##     <abstract><![CDATA[In this paper, a discrete-time sliding mode inventory management strategy based on a novel non-switching type reaching law is introduced. The proposed reaching law eliminates undesirable chattering, and ensures that the sliding variable rate of change is upper bounded by a design parameter which does not depend on the system initial conditions. This approach guarantees fast convergence with non-negative, upper limited supply orders, and ensures that the maximum stock level may be specified a priori by the system designer. Furthermore, a sufficient condition for 100% customers' demand satisfaction is derived. The inventory replenishment system considered in this paper involves multiple suppliers with different lead times and different transportation losses in the delivery channels.]]></abstract>
##     <issn><![CDATA[1545-5955]]></issn>
##     <htmlFlag><![CDATA[1]]></htmlFlag>
##     <arnumber><![CDATA[6800105]]></arnumber>
##     <doi><![CDATA[10.1109/TASE.2014.2314690]]></doi>
##     <publicationId><![CDATA[6800105]]></publicationId>
##     <mdurl><![CDATA[http://ieeexplore.ieee.org/xpl/articleDetails.jsp?tp=&arnumber=6800105&contentType=Journals+%26+Magazines]]></mdurl>
##     <pdf><![CDATA[http://ieeexplore.ieee.org/stamp/stamp.jsp?arnumber=6800105]]></pdf>
##   </document>
##   <document>
##     <rank>758</rank>
##     <title><![CDATA[NSTX-U Digital Coil Protection System Software Detailed Design]]></title>
##     <authors><![CDATA[Erickson, K.G.;  Tchilinguirian, G.J.;  Hatcher, R.E.;  Davis, W.M.]]></authors>
##     <affiliations><![CDATA[Plasma Phys. Lab., Princeton Univ., Princeton, NJ, USA]]></affiliations>
##     <controlledterms>
##       <term><![CDATA[Linux]]></term>
##       <term><![CDATA[Tokamak devices]]></term>
##       <term><![CDATA[fusion reactor design]]></term>
##       <term><![CDATA[fusion reactor safety]]></term>
##       <term><![CDATA[nuclear engineering computing]]></term>
##       <term><![CDATA[plasma toroidal confinement]]></term>
##       <term><![CDATA[software architecture]]></term>
##     </controlledterms>
##     <thesaurusterms>
##       <term><![CDATA[Algorithm design and analysis]]></term>
##       <term><![CDATA[Coils]]></term>
##       <term><![CDATA[Kernel]]></term>
##       <term><![CDATA[Linux]]></term>
##       <term><![CDATA[Real-time systems]]></term>
##       <term><![CDATA[Unified modeling language]]></term>
##     </thesaurusterms>
##     <pubtitle><![CDATA[Plasma Science, IEEE Transactions on]]></pubtitle>
##     <punumber><![CDATA[27]]></punumber>
##     <pubtype><![CDATA[Journals & Magazines]]></pubtype>
##     <publisher><![CDATA[IEEE]]></publisher>
##     <volume><![CDATA[42]]></volume>
##     <issue><![CDATA[6]]></issue>
##     <part><![CDATA[2]]></part>
##     <py><![CDATA[2014]]></py>
##     <spage><![CDATA[1811]]></spage>
##     <epage><![CDATA[1818]]></epage>
##     <abstract><![CDATA[The national spherical torus experiment (NSTX) currently uses a collection of analog signal processing solutions for coil protection. Part of the NSTX upgrade (NSTX-U) entails replacing these analog systems with a software solution running on a conventional computing platform. The new digital coil protection system (DCPS) will replace the old systems entirely, while also providing an extensible framework that allows adding new functionality as desired. The development of the DCPS was a multidiscipline engineering effort. The fact that long-trusted yet presently inadequate protection mechanisms were being replaced with a first-of-a-kind system at NSTX-U has led to a carefully crafted, full-featured software design. Real-time concurrent RedHawk Linux provides the deterministic environment in which the software runs, and the software architecture follows a unified modeling language design with industry standard patterns.]]></abstract>
##     <issn><![CDATA[0093-3813]]></issn>
##     <htmlFlag><![CDATA[1]]></htmlFlag>
##     <arnumber><![CDATA[6822612]]></arnumber>
##     <doi><![CDATA[10.1109/TPS.2014.2321106]]></doi>
##     <publicationId><![CDATA[6822612]]></publicationId>
##     <mdurl><![CDATA[http://ieeexplore.ieee.org/xpl/articleDetails.jsp?tp=&arnumber=6822612&contentType=Journals+%26+Magazines]]></mdurl>
##     <pdf><![CDATA[http://ieeexplore.ieee.org/stamp/stamp.jsp?arnumber=6822612]]></pdf>
##   </document>
##   <document>
##     <rank>759</rank>
##     <title><![CDATA[Neuromorphic Character Recognition System With Two PCMO Memristors as a Synapse]]></title>
##     <authors><![CDATA[Sheri, A.M.;  Hyunsang Hwang;  Moongu Jeon;  Byung-geun Lee]]></authors>
##     <affiliations><![CDATA[Sch. of Inf. & Commun., Gwangju Inst. of Sci. & Technol., Gwangju, South Korea]]></affiliations>
##     <controlledterms>
##       <term><![CDATA[character recognition]]></term>
##       <term><![CDATA[memristors]]></term>
##       <term><![CDATA[neural net architecture]]></term>
##       <term><![CDATA[power engineering computing]]></term>
##     </controlledterms>
##     <thesaurusterms>
##       <term><![CDATA[Biological neural networks]]></term>
##       <term><![CDATA[Clocks]]></term>
##       <term><![CDATA[Memristors]]></term>
##       <term><![CDATA[Neuromorphics]]></term>
##       <term><![CDATA[Neurons]]></term>
##       <term><![CDATA[Threshold voltage]]></term>
##     </thesaurusterms>
##     <pubtitle><![CDATA[Industrial Electronics, IEEE Transactions on]]></pubtitle>
##     <punumber><![CDATA[41]]></punumber>
##     <pubtype><![CDATA[Journals & Magazines]]></pubtype>
##     <publisher><![CDATA[IEEE]]></publisher>
##     <volume><![CDATA[61]]></volume>
##     <issue><![CDATA[6]]></issue>
##     <py><![CDATA[2014]]></py>
##     <spage><![CDATA[2933]]></spage>
##     <epage><![CDATA[2941]]></epage>
##     <abstract><![CDATA[Using memristor devices as synaptic connections has been suggested with different neural architectures in the literature. Most of the published works focus on simulating some plasticity mechanism for changing memristor conductance. This paper presents a neural architecture of a character recognition neural system using Al/Pr<sub>0.7</sub>Ca<sub>0.3</sub>MnO<sub>3</sub> (PCMO) memristors. The PCMO memristor has an inhomogeneous barrier at the aluminum and PCMO interface which gives rise to an asymmetrical behavior when moving from high resistance to low resistance and vice versa. This paper details the design and simulations for solving this asymmetrical memristor behavior. Also, a general memory read/write framework is used to describe the running and plasticity of neural systems. The proposed neural system can be produced in hardware using a small 1 K crossbar memristor grid and CMOS neural nodes as presented in the simulation results.]]></abstract>
##     <issn><![CDATA[0278-0046]]></issn>
##     <htmlFlag><![CDATA[1]]></htmlFlag>
##     <arnumber><![CDATA[6573409]]></arnumber>
##     <doi><![CDATA[10.1109/TIE.2013.2275966]]></doi>
##     <publicationId><![CDATA[6573409]]></publicationId>
##     <mdurl><![CDATA[http://ieeexplore.ieee.org/xpl/articleDetails.jsp?tp=&arnumber=6573409&contentType=Journals+%26+Magazines]]></mdurl>
##     <pdf><![CDATA[http://ieeexplore.ieee.org/stamp/stamp.jsp?arnumber=6573409]]></pdf>
##   </document>
##   <document>
##     <rank>760</rank>
##     <title><![CDATA[Structural Optimization of Photonic Crystals for Enhancing Optical Absorption of Thin Film Silicon Solar Cell Structures]]></title>
##     <authors><![CDATA[Kawamoto, Y.;  Tanaka, Y.;  Ishizaki, K.;  De Zoysa, M.;  Asano, T.;  Noda, S.]]></authors>
##     <affiliations><![CDATA[Dept. of Electron. Sci. & Eng., Kyoto Univ., Kyoto, Japan]]></affiliations>
##     <controlledterms>
##       <term><![CDATA[light absorption]]></term>
##       <term><![CDATA[micro-optics]]></term>
##       <term><![CDATA[photonic crystals]]></term>
##       <term><![CDATA[solar cells]]></term>
##     </controlledterms>
##     <thesaurusterms>
##       <term><![CDATA[Absorption]]></term>
##       <term><![CDATA[Optical films]]></term>
##       <term><![CDATA[Optical reflection]]></term>
##       <term><![CDATA[Optical sensors]]></term>
##       <term><![CDATA[Optical superlattices]]></term>
##       <term><![CDATA[Photonic crystals]]></term>
##       <term><![CDATA[Silicon]]></term>
##     </thesaurusterms>
##     <pubtitle><![CDATA[Photonics Journal, IEEE]]></pubtitle>
##     <punumber><![CDATA[4563994]]></punumber>
##     <pubtype><![CDATA[Journals & Magazines]]></pubtype>
##     <publisher><![CDATA[IEEE]]></publisher>
##     <volume><![CDATA[6]]></volume>
##     <issue><![CDATA[1]]></issue>
##     <py><![CDATA[2014]]></py>
##     <spage><![CDATA[1]]></spage>
##     <epage><![CDATA[10]]></epage>
##     <abstract><![CDATA[We carry out the structural design of photonic crystals to enhance the optical absorption of thin-film microcrystalline silicon (&#x03BC;c-Si) solar cells using two methods. First, by exhaustive search, we choose a structure with the largest absorption within the investigated patterns. Then we employ a sensitivity analysis to finely modulate the structure for further increase of the optical absorption. The obtained &#x03BC;c-Si solar cell structure with a photonic crystal in this work has more than twice as much optical absorption as that without a photonic crystal.]]></abstract>
##     <issn><![CDATA[1943-0655]]></issn>
##     <htmlFlag><![CDATA[1]]></htmlFlag>
##     <arnumber><![CDATA[6727424]]></arnumber>
##     <doi><![CDATA[10.1109/JPHOT.2014.2302800]]></doi>
##     <publicationId><![CDATA[6727424]]></publicationId>
##     <mdurl><![CDATA[http://ieeexplore.ieee.org/xpl/articleDetails.jsp?tp=&arnumber=6727424&contentType=Journals+%26+Magazines]]></mdurl>
##     <pdf><![CDATA[http://ieeexplore.ieee.org/stamp/stamp.jsp?arnumber=6727424]]></pdf>
##   </document>
##   <document>
##     <rank>761</rank>
##     <title><![CDATA[A <inline-formula> <img src="/images/tex/21794.gif" alt="(10 \times 10)"> </inline-formula>-Gb/s DFB-LD Array Integrated With PLC-Based AWG for 100-Gb/s Transmission]]></title>
##     <authors><![CDATA[Oh Kee Kwon;  Young-Tak Han;  Young Ahn Leem;  Jang-Uk Shin;  Chul Wook Lee;  Ki Soo Kim]]></authors>
##     <affiliations><![CDATA[Opt./Wireless Convergence Component Res. Dept., Electron. & Telecommun. Res. Inst., Daejeon, South Korea]]></affiliations>
##     <controlledterms>
##       <term><![CDATA[arrayed waveguide gratings]]></term>
##       <term><![CDATA[distributed feedback lasers]]></term>
##       <term><![CDATA[electron beam lithography]]></term>
##       <term><![CDATA[integrated optoelectronics]]></term>
##       <term><![CDATA[light propagation]]></term>
##       <term><![CDATA[optical communication equipment]]></term>
##       <term><![CDATA[optical fabrication]]></term>
##       <term><![CDATA[optical losses]]></term>
##       <term><![CDATA[optical planar waveguides]]></term>
##       <term><![CDATA[semiconductor laser arrays]]></term>
##     </controlledterms>
##     <thesaurusterms>
##       <term><![CDATA[Arrayed waveguide gratings]]></term>
##       <term><![CDATA[Arrays]]></term>
##       <term><![CDATA[Couplings]]></term>
##       <term><![CDATA[Current measurement]]></term>
##       <term><![CDATA[Loss measurement]]></term>
##       <term><![CDATA[Wavelength measurement]]></term>
##     </thesaurusterms>
##     <pubtitle><![CDATA[Photonics Technology Letters, IEEE]]></pubtitle>
##     <punumber><![CDATA[68]]></punumber>
##     <pubtype><![CDATA[Journals & Magazines]]></pubtype>
##     <publisher><![CDATA[IEEE]]></publisher>
##     <volume><![CDATA[26]]></volume>
##     <issue><![CDATA[21]]></issue>
##     <py><![CDATA[2014]]></py>
##     <spage><![CDATA[2177]]></spage>
##     <epage><![CDATA[2180]]></epage>
##     <abstract><![CDATA[We describe a hybrid integration module of a 10 &#x00D7; 10-Gb/s distributed feedback laser diode array (DFB-LDA) and a planar light-wave circuit-based arrayed waveguide grating (AWG). For fabrication of the DFB-LDA, we adopted a selective area growth technique to tailor the channel gain properly, and E-beam lithography to accurately control the channel lasing wavelength and grating phase. For implementation of the AWG, we introduced a 2%-&#x0394; structure to reduce the coupling loss between the AWG and DFB-LDA, and designed the tapered and parabolic waveguides at the junctions of the free propagation regions to widen the spectral passband width (i.e., a -1-dB spectral width of 3 nm). The developed module shows a side-mode suppression ratio of &gt;45 dB, a clear eye opening with a dynamic extinction ratio of &gt;4.4 dB at 10 Gb/s, and a power penalty of &lt;;1.5 dB after a 2-km transmission for all channels.]]></abstract>
##     <issn><![CDATA[1041-1135]]></issn>
##     <htmlFlag><![CDATA[1]]></htmlFlag>
##     <arnumber><![CDATA[6881619]]></arnumber>
##     <doi><![CDATA[10.1109/LPT.2014.2349072]]></doi>
##     <publicationId><![CDATA[6881619]]></publicationId>
##     <mdurl><![CDATA[http://ieeexplore.ieee.org/xpl/articleDetails.jsp?tp=&arnumber=6881619&contentType=Journals+%26+Magazines]]></mdurl>
##     <pdf><![CDATA[http://ieeexplore.ieee.org/stamp/stamp.jsp?arnumber=6881619]]></pdf>
##   </document>
##   <document>
##     <rank>762</rank>
##     <title><![CDATA[Identifying Cancer Biomarkers From Microarray Data Using Feature Selection and Semisupervised Learning]]></title>
##     <authors><![CDATA[Chakraborty, D.;  Maulik, U.]]></authors>
##     <affiliations><![CDATA[Murshidabad Coll. of Eng. & Technol., Berhampore, India]]></affiliations>
##     <controlledterms>
##       <term><![CDATA[RNA]]></term>
##       <term><![CDATA[biomedical equipment]]></term>
##       <term><![CDATA[cancer]]></term>
##       <term><![CDATA[diseases]]></term>
##       <term><![CDATA[feature selection]]></term>
##       <term><![CDATA[fuzzy set theory]]></term>
##       <term><![CDATA[genetics]]></term>
##       <term><![CDATA[lab-on-a-chip]]></term>
##       <term><![CDATA[learning (artificial intelligence)]]></term>
##       <term><![CDATA[medical diagnostic computing]]></term>
##       <term><![CDATA[molecular biophysics]]></term>
##       <term><![CDATA[patient diagnosis]]></term>
##       <term><![CDATA[patient monitoring]]></term>
##       <term><![CDATA[patient treatment]]></term>
##       <term><![CDATA[pattern classification]]></term>
##       <term><![CDATA[statistical analysis]]></term>
##       <term><![CDATA[support vector machines]]></term>
##       <term><![CDATA[tumours]]></term>
##     </controlledterms>
##     <thesaurusterms>
##       <term><![CDATA[Biomarkers]]></term>
##       <term><![CDATA[Cancer]]></term>
##       <term><![CDATA[Gene expression]]></term>
##       <term><![CDATA[Kernel]]></term>
##       <term><![CDATA[Support vector machines]]></term>
##       <term><![CDATA[Training]]></term>
##       <term><![CDATA[Tumors]]></term>
##     </thesaurusterms>
##     <pubtitle><![CDATA[Translational Engineering in Health and Medicine, IEEE Journal of]]></pubtitle>
##     <punumber><![CDATA[6221039]]></punumber>
##     <pubtype><![CDATA[Journals & Magazines]]></pubtype>
##     <publisher><![CDATA[IEEE]]></publisher>
##     <volume><![CDATA[2]]></volume>
##     <py><![CDATA[2014]]></py>
##     <spage><![CDATA[1]]></spage>
##     <epage><![CDATA[11]]></epage>
##     <abstract><![CDATA[Microarrays have now gone from obscurity to being almost ubiquitous in biological research. At the same time, the statistical methodology for microarray analysis has progressed from simple visual assessments of results to novel algorithms for analyzing changes in expression profiles. In a micro-RNA (miRNA) or gene-expression profiling experiment, the expression levels of thousands of genes/miRNAs are simultaneously monitored to study the effects of certain treatments, diseases, and developmental stages on their expressions. Microarray-based gene expression profiling can be used to identify genes, whose expressions are changed in response to pathogens or other organisms by comparing gene expression in infected to that in uninfected cells or tissues. Recent studies have revealed that patterns of altered microarray expression profiles in cancer can serve as molecular biomarkers for tumor diagnosis, prognosis of disease-specific outcomes, and prediction of therapeutic responses. Microarray data sets containing expression profiles of a number of miRNAs or genes are used to identify biomarkers, which have dysregulation in normal and malignant tissues. However, small sample size remains a bottleneck to design successful classification methods. On the other hand, adequate number of microarray data that do not have clinical knowledge can be employed as additional source of information. In this paper, a combination of kernelized fuzzy rough set (KFRS) and semisupervised support vector machine (S<sup>3</sup>VM) is proposed for predicting cancer biomarkers from one miRNA and three gene expression data sets. Biomarkers are discovered employing three feature selection methods, including KFRS. The effectiveness of the proposed KFRS and S<sup>3</sup>VM combination on the microarray data sets is demonstrated, and the cancer biomarkers identified from miRNA data are reported. Furthermore, biological significance tests are conducted for miRNA cancer biomarkers.]]></abstract>
##     <issn><![CDATA[2168-2372]]></issn>
##     <htmlFlag><![CDATA[1]]></htmlFlag>
##     <arnumber><![CDATA[6971078]]></arnumber>
##     <doi><![CDATA[10.1109/JTEHM.2014.2375820]]></doi>
##     <publicationId><![CDATA[6971078]]></publicationId>
##     <mdurl><![CDATA[http://ieeexplore.ieee.org/xpl/articleDetails.jsp?tp=&arnumber=6971078&contentType=Journals+%26+Magazines]]></mdurl>
##     <pdf><![CDATA[http://ieeexplore.ieee.org/stamp/stamp.jsp?arnumber=6971078]]></pdf>
##   </document>
##   <document>
##     <rank>763</rank>
##     <title><![CDATA[Resilient Batch-Fabricated Planar Arrays of Miniaturized Langmuir Probes for Real-Time Measurement of Plasma Potential Fluctuations in the HF to Microwave Frequency Range]]></title>
##     <authors><![CDATA[Chimamkpam, E.F.C.;  Field, E.S.;  Akinwande, A.I.;  Velasquez-Garcia, L.F.]]></authors>
##     <affiliations><![CDATA[Massachusetts Inst. of Technol., Cambridge, MA, USA]]></affiliations>
##     <controlledterms>
##       <term><![CDATA[Langmuir probes]]></term>
##       <term><![CDATA[helicons]]></term>
##       <term><![CDATA[metallisation]]></term>
##       <term><![CDATA[microfabrication]]></term>
##       <term><![CDATA[microsensors]]></term>
##       <term><![CDATA[nickel]]></term>
##       <term><![CDATA[plasma density]]></term>
##       <term><![CDATA[plasma fluctuations]]></term>
##       <term><![CDATA[plasma sheaths]]></term>
##       <term><![CDATA[plasma sources]]></term>
##       <term><![CDATA[plasma transport processes]]></term>
##       <term><![CDATA[vias]]></term>
##     </controlledterms>
##     <thesaurusterms>
##       <term><![CDATA[Impedance]]></term>
##       <term><![CDATA[Micromechanical devices]]></term>
##       <term><![CDATA[Nickel]]></term>
##       <term><![CDATA[Plasma measurements]]></term>
##       <term><![CDATA[Plasmas]]></term>
##       <term><![CDATA[Probes]]></term>
##       <term><![CDATA[Substrates]]></term>
##     </thesaurusterms>
##     <pubtitle><![CDATA[Microelectromechanical Systems, Journal of]]></pubtitle>
##     <punumber><![CDATA[84]]></punumber>
##     <pubtype><![CDATA[Journals & Magazines]]></pubtype>
##     <publisher><![CDATA[IEEE]]></publisher>
##     <volume><![CDATA[23]]></volume>
##     <issue><![CDATA[5]]></issue>
##     <py><![CDATA[2014]]></py>
##     <spage><![CDATA[1131]]></spage>
##     <epage><![CDATA[1140]]></epage>
##     <abstract><![CDATA[We report the design, fabrication, and characterization of miniaturized, flush-mounted Langmuir probe arrays for RF diagnosis of plasmas in the HF to microwave range of frequencies. We developed probes of radii &#x2265;125 &#x03BC;m by electroless nickel metallization of ultrasonically drilled through-substrate vias. Planar arrays with as many as 25 probes spaced 1.6 mm apart (39 probes/cm2) in Pyrex, silicon carbide, and alumina substrates were produced. The sensor system was built to have a frequency response between 2 MHz and 3 GHz, and a probe impedance greater than or within close range of the plasma sheath impedance for plasma densities &#x2265;1016 m<sup>-3</sup>. We characterized a self-biasing nickel probe part of a 2&#x00D7;2 array with alumina substrate using a high-density magnetized helicon plasma source; we found that the measurement of the plasma potential from the MEMS probe compares well with independent measurements using a hot emissive probe and an ion sensitive probe. The sensor technology can be used to monitor plasma-based manufacturing systems, plasma-based energy generation systems, and as on-board plasma diagnostics in spacecraft including nanosatellites.]]></abstract>
##     <issn><![CDATA[1057-7157]]></issn>
##     <htmlFlag><![CDATA[1]]></htmlFlag>
##     <arnumber><![CDATA[6758352]]></arnumber>
##     <doi><![CDATA[10.1109/JMEMS.2014.2306631]]></doi>
##     <publicationId><![CDATA[6758352]]></publicationId>
##     <mdurl><![CDATA[http://ieeexplore.ieee.org/xpl/articleDetails.jsp?tp=&arnumber=6758352&contentType=Journals+%26+Magazines]]></mdurl>
##     <pdf><![CDATA[http://ieeexplore.ieee.org/stamp/stamp.jsp?arnumber=6758352]]></pdf>
##   </document>
##   <document>
##     <rank>764</rank>
##     <title><![CDATA[Regaining Trust in VLSI Design: Design-for-Trust Techniques]]></title>
##     <authors><![CDATA[Rajendran, J.;  Sinanoglu, O.;  Karri, R.]]></authors>
##     <affiliations><![CDATA[Electr. & Comput. Eng. Dept., New York Univ., New York, NY, USA]]></affiliations>
##     <controlledterms>
##       <term><![CDATA[VLSI]]></term>
##       <term><![CDATA[cryptography]]></term>
##       <term><![CDATA[integrated circuit design]]></term>
##       <term><![CDATA[logic circuits]]></term>
##       <term><![CDATA[microprocessor chips]]></term>
##       <term><![CDATA[reverse engineering]]></term>
##     </controlledterms>
##     <thesaurusterms>
##       <term><![CDATA[Design methodology]]></term>
##       <term><![CDATA[Encryption]]></term>
##       <term><![CDATA[Hardware]]></term>
##       <term><![CDATA[Integrated circuit modeling]]></term>
##       <term><![CDATA[Logic gates]]></term>
##       <term><![CDATA[Very large scale integration]]></term>
##     </thesaurusterms>
##     <pubtitle><![CDATA[Proceedings of the IEEE]]></pubtitle>
##     <punumber><![CDATA[5]]></punumber>
##     <pubtype><![CDATA[Journals & Magazines]]></pubtype>
##     <publisher><![CDATA[IEEE]]></publisher>
##     <volume><![CDATA[102]]></volume>
##     <issue><![CDATA[8]]></issue>
##     <py><![CDATA[2014]]></py>
##     <spage><![CDATA[1266]]></spage>
##     <epage><![CDATA[1282]]></epage>
##     <abstract><![CDATA[Designers use third-party intellectual property (IP) cores and outsource various steps in their integrated circuit (IC) design flow, including fabrication. As a result, security vulnerabilities have been emerging, forcing IC designers and end-users to reevaluate their trust in hardware. If an attacker gets hold of an unprotected design, attacks such as reverse engineering, insertion of malicious circuits, and IP piracy are possible. In this paper, we shed light on the vulnerabilities in very large scale integration (VLSI) design and fabrication flow, and survey design-for-trust (DfTr) techniques that aim at regaining trust in IC design. We elaborate on four DfTr techniques: logic encryption, split manufacturing, IC camouflaging, and Trojan activation. These techniques have been developed by reusing VLSI test principles.]]></abstract>
##     <issn><![CDATA[0018-9219]]></issn>
##     <htmlFlag><![CDATA[1]]></htmlFlag>
##     <arnumber><![CDATA[6856167]]></arnumber>
##     <doi><![CDATA[10.1109/JPROC.2014.2332154]]></doi>
##     <publicationId><![CDATA[6856167]]></publicationId>
##     <mdurl><![CDATA[http://ieeexplore.ieee.org/xpl/articleDetails.jsp?tp=&arnumber=6856167&contentType=Journals+%26+Magazines]]></mdurl>
##     <pdf><![CDATA[http://ieeexplore.ieee.org/stamp/stamp.jsp?arnumber=6856167]]></pdf>
##   </document>
##   <document>
##     <rank>765</rank>
##     <title><![CDATA[Variation-Aware Layer Assignment With Hierarchical Stochastic Optimization on a Multicore Platform]]></title>
##     <authors><![CDATA[Xiaodao Chen;  Dan Chen;  Lizhe Wang;  Ze Deng;  Ranjan, R.;  Zomaya, A.Y.;  Shiyan Hu]]></authors>
##     <affiliations><![CDATA[Dept. of Electr. & Comput. Eng., Michigan Technol. Univ., Houghton, MI, USA]]></affiliations>
##     <controlledterms>
##       <term><![CDATA[Monte Carlo methods]]></term>
##       <term><![CDATA[VLSI]]></term>
##       <term><![CDATA[logic design]]></term>
##       <term><![CDATA[multiprocessing systems]]></term>
##       <term><![CDATA[sampling methods]]></term>
##       <term><![CDATA[stochastic programming]]></term>
##     </controlledterms>
##     <thesaurusterms>
##       <term><![CDATA[Capacitance]]></term>
##       <term><![CDATA[Large-scale systems]]></term>
##       <term><![CDATA[Nanoscale devices]]></term>
##       <term><![CDATA[Programming]]></term>
##       <term><![CDATA[Stochastic processes]]></term>
##       <term><![CDATA[Very large scale integration]]></term>
##     </thesaurusterms>
##     <pubtitle><![CDATA[Emerging Topics in Computing, IEEE Transactions on]]></pubtitle>
##     <punumber><![CDATA[6245516]]></punumber>
##     <pubtype><![CDATA[Journals & Magazines]]></pubtype>
##     <publisher><![CDATA[IEEE]]></publisher>
##     <volume><![CDATA[2]]></volume>
##     <issue><![CDATA[4]]></issue>
##     <py><![CDATA[2014]]></py>
##     <spage><![CDATA[488]]></spage>
##     <epage><![CDATA[500]]></epage>
##     <abstract><![CDATA[As the very large scale integration (VLSI) technology enters the nanoscale regime, VLSI design is increasingly sensitive to variations on process, voltage, and temperature. Layer assignment technology plays a crucial role in industrial VLSI design flow. However, existing layer assignment approaches have largely ignored these variations, which can lead to significant timing violations. To address this issue, a variation-aware layer assignment approach for cost minimization is proposed in this paper. The proposed layer assignment approach is a single-stage stochastic program that directly controls the timing yield via a single parameter, and it is solved using Monte Carlo simulations and the Latin hypercube sampling technique. A hierarchical design is also adopted to enable the optimization process on a multicore platform. Experiments have been performed on 5000 industrial nets, and the results demonstrate that the proposed approach: 1) can significantly improve the timing yield by 64% in comparison with the nominal design and 2) can reduce the wire cost by 15.7% in comparison with the worst case design.]]></abstract>
##     <issn><![CDATA[2168-6750]]></issn>
##     <htmlFlag><![CDATA[1]]></htmlFlag>
##     <arnumber><![CDATA[6786337]]></arnumber>
##     <doi><![CDATA[10.1109/TETC.2014.2316503]]></doi>
##     <publicationId><![CDATA[6786337]]></publicationId>
##     <mdurl><![CDATA[http://ieeexplore.ieee.org/xpl/articleDetails.jsp?tp=&arnumber=6786337&contentType=Journals+%26+Magazines]]></mdurl>
##     <pdf><![CDATA[http://ieeexplore.ieee.org/stamp/stamp.jsp?arnumber=6786337]]></pdf>
##   </document>
##   <document>
##     <rank>766</rank>
##     <title><![CDATA[A Simple Transfer-Function-Based Approach for Estimating Material Parameters From Terahertz Time-Domain Data]]></title>
##     <authors><![CDATA[Tych, K.M.;  Wood, C.D.;  Tych, W.]]></authors>
##     <affiliations><![CDATA[Sch. of Phys. & Astron., Univ. of Leeds, Leeds, UK]]></affiliations>
##     <controlledterms>
##       <term><![CDATA[Fourier transform optics]]></term>
##       <term><![CDATA[measurement uncertainty]]></term>
##       <term><![CDATA[optical transfer function]]></term>
##       <term><![CDATA[refractive index]]></term>
##       <term><![CDATA[terahertz wave detectors]]></term>
##     </controlledterms>
##     <thesaurusterms>
##       <term><![CDATA[Estimation]]></term>
##       <term><![CDATA[Frequency-domain analysis]]></term>
##       <term><![CDATA[Materials]]></term>
##       <term><![CDATA[Mathematical model]]></term>
##       <term><![CDATA[Time-domain analysis]]></term>
##       <term><![CDATA[Transfer functions]]></term>
##       <term><![CDATA[Uncertainty]]></term>
##     </thesaurusterms>
##     <pubtitle><![CDATA[Photonics Journal, IEEE]]></pubtitle>
##     <punumber><![CDATA[4563994]]></punumber>
##     <pubtype><![CDATA[Journals & Magazines]]></pubtype>
##     <publisher><![CDATA[IEEE]]></publisher>
##     <volume><![CDATA[6]]></volume>
##     <issue><![CDATA[1]]></issue>
##     <py><![CDATA[2014]]></py>
##     <spage><![CDATA[1]]></spage>
##     <epage><![CDATA[11]]></epage>
##     <abstract><![CDATA[A novel parametrically efficient approach to estimating the spectra of short transient signals is proposed and evaluated, with an application to estimating material properties, including complex refractive index and absorption coefficient, in the terahertz frequency range. This technique includes uncertainty analysis of the obtained spectral estimates, allowing rigorous statistical comparison between samples. In the proposed approach, a simple few-parameter continuous-time transfer function model explains over 99.9% of the measured signal. The problem, normally solved using poorly numerically defined Fourier transform deconvolution methods, is reformulated and cast as a time-domain dynamic-system estimation problem, thus providing a true time-domain spectroscopy tool.]]></abstract>
##     <issn><![CDATA[1943-0655]]></issn>
##     <htmlFlag><![CDATA[1]]></htmlFlag>
##     <arnumber><![CDATA[6671984]]></arnumber>
##     <doi><![CDATA[10.1109/JPHOT.2013.2292337]]></doi>
##     <publicationId><![CDATA[6671984]]></publicationId>
##     <mdurl><![CDATA[http://ieeexplore.ieee.org/xpl/articleDetails.jsp?tp=&arnumber=6671984&contentType=Journals+%26+Magazines]]></mdurl>
##     <pdf><![CDATA[http://ieeexplore.ieee.org/stamp/stamp.jsp?arnumber=6671984]]></pdf>
##   </document>
##   <document>
##     <rank>767</rank>
##     <title><![CDATA[A Sufficient Condition Producing 16-QAM Golay Complementary Sequences]]></title>
##     <authors><![CDATA[Fanxin Zeng]]></authors>
##     <affiliations><![CDATA[Chongqing Key Lab. of Emergency Commun., Chongqing Commun. Inst., Chongqing, China]]></affiliations>
##     <controlledterms>
##       <term><![CDATA[Golay codes]]></term>
##       <term><![CDATA[quadrature amplitude modulation]]></term>
##     </controlledterms>
##     <thesaurusterms>
##       <term><![CDATA[Boolean functions]]></term>
##       <term><![CDATA[Cascading style sheets]]></term>
##       <term><![CDATA[Computers]]></term>
##       <term><![CDATA[OFDM]]></term>
##       <term><![CDATA[Quadrature amplitude modulation]]></term>
##       <term><![CDATA[Standards]]></term>
##       <term><![CDATA[Upper bound]]></term>
##     </thesaurusterms>
##     <pubtitle><![CDATA[Communications Letters, IEEE]]></pubtitle>
##     <punumber><![CDATA[4234]]></punumber>
##     <pubtype><![CDATA[Journals & Magazines]]></pubtype>
##     <publisher><![CDATA[IEEE]]></publisher>
##     <volume><![CDATA[18]]></volume>
##     <issue><![CDATA[11]]></issue>
##     <py><![CDATA[2014]]></py>
##     <spage><![CDATA[1875]]></spage>
##     <epage><![CDATA[1878]]></epage>
##     <abstract><![CDATA[This letter discusses the construction of 16-quadratic amplitude modulation (QAM) Golay complementary sequences of length N = 2<sup>m</sup>. Based on the standard binary Golay-Davis-Jedwab (GDJ) complementary sequences (CSs), we present a method to convert the aforementioned GDJ CSs into the required sequences. The resultant sequences have the upper bounds 3.6N, 2.8N, 2N, 1.2N, and 0.4N of peak envelope powers, respectively, depending on the choices of their offsets. The numbers of the proposed sequences, corresponding to five upper bounds referred to above, are (24m - 16)(m!/2)2<sup>m+1</sup>, 128(m - 1)(m!/2)2<sup>m+1</sup>, (176m - 160)(m!/2)2<sup>m+1</sup>, 128(m - 1)(m!/2)2<sup>m+1</sup>, and (24m - 16)(m!/2)2<sup>m+1</sup>. Our sequences can be potentially applied to the QAM systems whose input signals are binary signals.]]></abstract>
##     <issn><![CDATA[1089-7798]]></issn>
##     <htmlFlag><![CDATA[1]]></htmlFlag>
##     <arnumber><![CDATA[6912940]]></arnumber>
##     <doi><![CDATA[10.1109/LCOMM.2014.2360695]]></doi>
##     <publicationId><![CDATA[6912940]]></publicationId>
##     <mdurl><![CDATA[http://ieeexplore.ieee.org/xpl/articleDetails.jsp?tp=&arnumber=6912940&contentType=Journals+%26+Magazines]]></mdurl>
##     <pdf><![CDATA[http://ieeexplore.ieee.org/stamp/stamp.jsp?arnumber=6912940]]></pdf>
##   </document>
##   <document>
##     <rank>768</rank>
##     <title><![CDATA[Designs of Helix Metamaterials for Broadband and High-Transmission Polarization Rotation]]></title>
##     <authors><![CDATA[Hsiang-Hung Huang;  Yu-Chueh Hung]]></authors>
##     <affiliations><![CDATA[Inst. of Photonics Technol., Nat. Tsing Hua Univ., Hsinchu, Taiwan]]></affiliations>
##     <controlledterms>
##       <term><![CDATA[genetic algorithms]]></term>
##       <term><![CDATA[light polarisation]]></term>
##       <term><![CDATA[optical metamaterials]]></term>
##       <term><![CDATA[optical rotation]]></term>
##     </controlledterms>
##     <thesaurusterms>
##       <term><![CDATA[Biomedical optical imaging]]></term>
##       <term><![CDATA[Magnetic materials]]></term>
##       <term><![CDATA[Metamaterials]]></term>
##       <term><![CDATA[Optical device fabrication]]></term>
##       <term><![CDATA[Optical polarization]]></term>
##       <term><![CDATA[Optical ring resonators]]></term>
##       <term><![CDATA[Optical sensors]]></term>
##     </thesaurusterms>
##     <pubtitle><![CDATA[Photonics Journal, IEEE]]></pubtitle>
##     <punumber><![CDATA[4563994]]></punumber>
##     <pubtype><![CDATA[Journals & Magazines]]></pubtype>
##     <publisher><![CDATA[IEEE]]></publisher>
##     <volume><![CDATA[6]]></volume>
##     <issue><![CDATA[5]]></issue>
##     <py><![CDATA[2014]]></py>
##     <spage><![CDATA[1]]></spage>
##     <epage><![CDATA[7]]></epage>
##     <abstract><![CDATA[Artificial optical activity can be realized using chiral metamaterials, in which chirality is normally implemented with rotationally stacked planar structures. However, such geometries usually suffer from great loss and limited operation bandwidth, which may hinder their usage in many applications. In this study, by incorporating genetic algorithm for geometry optimization, we present designs of helix metamaterials that exhibit high transmission above 95% and a broad bandwidth with low ellipticity. The optical characteristics are described, and the nature behind the high transmission and broad bandwidth is discussed.]]></abstract>
##     <issn><![CDATA[1943-0655]]></issn>
##     <htmlFlag><![CDATA[1]]></htmlFlag>
##     <arnumber><![CDATA[6894270]]></arnumber>
##     <doi><![CDATA[10.1109/JPHOT.2014.2352636]]></doi>
##     <publicationId><![CDATA[6894270]]></publicationId>
##     <mdurl><![CDATA[http://ieeexplore.ieee.org/xpl/articleDetails.jsp?tp=&arnumber=6894270&contentType=Journals+%26+Magazines]]></mdurl>
##     <pdf><![CDATA[http://ieeexplore.ieee.org/stamp/stamp.jsp?arnumber=6894270]]></pdf>
##   </document>
##   <document>
##     <rank>769</rank>
##     <title><![CDATA[Interferometric Characterization of Laboratory Plasma Astrophysical Jets Produced by a 1-<inline-formula> <img src="/images/tex/21769.gif" alt="\mu "> </inline-formula>s Pulsed Power Driver]]></title>
##     <authors><![CDATA[Plouhinec, D.;  Zucchini, F.;  Loyen, A.;  Sol, D.;  Combes, P.;  Grunenwald, J.;  Hammer, D.A.]]></authors>
##     <affiliations><![CDATA[DAM, CEA, Gramat, France]]></affiliations>
##     <controlledterms>
##       <term><![CDATA[astrophysical plasma]]></term>
##       <term><![CDATA[plasma density]]></term>
##     </controlledterms>
##     <thesaurusterms>
##       <term><![CDATA[Arrays]]></term>
##       <term><![CDATA[Electric shock]]></term>
##       <term><![CDATA[Extraterrestrial measurements]]></term>
##       <term><![CDATA[Laboratories]]></term>
##       <term><![CDATA[Measurement by laser beam]]></term>
##       <term><![CDATA[Plasmas]]></term>
##       <term><![CDATA[Wires]]></term>
##     </thesaurusterms>
##     <pubtitle><![CDATA[Plasma Science, IEEE Transactions on]]></pubtitle>
##     <punumber><![CDATA[27]]></punumber>
##     <pubtype><![CDATA[Journals & Magazines]]></pubtype>
##     <publisher><![CDATA[IEEE]]></publisher>
##     <volume><![CDATA[42]]></volume>
##     <issue><![CDATA[10]]></issue>
##     <part><![CDATA[1]]></part>
##     <py><![CDATA[2014]]></py>
##     <spage><![CDATA[2666]]></spage>
##     <epage><![CDATA[2667]]></epage>
##     <abstract><![CDATA[A high current driver based on microsecond LTD technology has been used to perform laboratory plasma astrophysics studies using a conical wire array load coupled a 950 kA, 1.2-&#x03BC;s pulsed power generator. A plasma jet is generated as a result of the on-axis shock formed by the ablation streams from the wires of a conical tungsten wire-array load together with conservation of the axial momentum. The aim of this paper is to produce a scaled-down laboratory simulation of astrophysical Herbig-Haro plasma jets occurring during star formation along with some of their interactions with the interstellar medium, such as a crosswind. Due to the relatively long duration of the current pulse delivered by the driver, the jet develops on a 2-&#x03BC;s timescale and grows up to 100 mm. A time-resolved laser interferometer has been fielded to measure the plasma areal electron density as a function of time in and around the plasma jets. The setup consists of a continuous diode-pumped solid state laser (5 W-532 nm), a Mach-Zehnder interferometer and fast gated visible multiframe camera.]]></abstract>
##     <issn><![CDATA[0093-3813]]></issn>
##     <htmlFlag><![CDATA[1]]></htmlFlag>
##     <arnumber><![CDATA[6862900]]></arnumber>
##     <doi><![CDATA[10.1109/TPS.2014.2323575]]></doi>
##     <publicationId><![CDATA[6862900]]></publicationId>
##     <mdurl><![CDATA[http://ieeexplore.ieee.org/xpl/articleDetails.jsp?tp=&arnumber=6862900&contentType=Journals+%26+Magazines]]></mdurl>
##     <pdf><![CDATA[http://ieeexplore.ieee.org/stamp/stamp.jsp?arnumber=6862900]]></pdf>
##   </document>
##   <document>
##     <rank>770</rank>
##     <title><![CDATA[Novel System for Real-Time Integration of 3-D Echocardiography and Fluoroscopy for Image-Guided Cardiac Interventions: Preclinical Validation and Clinical Feasibility Evaluation]]></title>
##     <authors><![CDATA[Arujuna, A.V.;  Housden, R.J.;  Ma, Y.;  Rajani, R.;  Gao, G.;  Nijhof, N.;  Cathier, P.;  Bullens, R.;  Gijsbers, G.;  Parish, V.;  Kapetanakis, S.;  Hancock, J.;  Rinaldi, C.A.;  Cooklin, M.;  Gill, J.;  Thomas, M.;  O'neill, M.D.;  Razavi, R.;  Rhode, K.S.]]></authors>
##     <affiliations><![CDATA[Div. of Imaging Sci. & Biomed. Eng., King's Coll. London, London, UK]]></affiliations>
##     <controlledterms>
##       <term><![CDATA[catheters]]></term>
##       <term><![CDATA[diagnostic radiography]]></term>
##       <term><![CDATA[echocardiography]]></term>
##       <term><![CDATA[image fusion]]></term>
##       <term><![CDATA[medical disorders]]></term>
##       <term><![CDATA[medical image processing]]></term>
##       <term><![CDATA[phantoms]]></term>
##       <term><![CDATA[prosthetics]]></term>
##       <term><![CDATA[surgery]]></term>
##     </controlledterms>
##     <thesaurusterms>
##       <term><![CDATA[Biomedical imaging]]></term>
##       <term><![CDATA[Cardiology]]></term>
##       <term><![CDATA[Catheters]]></term>
##       <term><![CDATA[Real-time systems]]></term>
##       <term><![CDATA[Three-dimensional displays]]></term>
##       <term><![CDATA[X-ray imaging]]></term>
##     </thesaurusterms>
##     <pubtitle><![CDATA[Translational Engineering in Health and Medicine, IEEE Journal of]]></pubtitle>
##     <punumber><![CDATA[6221039]]></punumber>
##     <pubtype><![CDATA[Journals & Magazines]]></pubtype>
##     <publisher><![CDATA[IEEE]]></publisher>
##     <volume><![CDATA[2]]></volume>
##     <py><![CDATA[2014]]></py>
##     <spage><![CDATA[1]]></spage>
##     <epage><![CDATA[10]]></epage>
##     <abstract><![CDATA[Real-time imaging is required to guide minimally invasive catheter-based cardiac interventions. While transesophageal echocardiography allows for high-quality visualization of cardiac anatomy, X-ray fluoroscopy provides excellent visualization of devices. We have developed a novel image fusion system that allows real-time integration of 3-D echocardiography and the X-ray fluoroscopy. The system was validated in the following two stages: 1) preclinical to determine function and validate accuracy; and 2) in the clinical setting to assess clinical workflow feasibility and determine overall system accuracy. In the preclinical phase, the system was assessed using both phantom and porcine experimental studies. Median 2-D projection errors of 4.5 and 3.3 mm were found for the phantom and porcine studies, respectively. The clinical phase focused on extending the use of the system to interventions in patients undergoing either atrial fibrillation catheter ablation (CA) or transcatheter aortic valve implantation (TAVI). Eleven patients were studied with nine in the CA group and two in the TAVI group. Successful real-time view synchronization was achieved in all cases with a calculated median distance error of 2.2 mm in the CA group and 3.4 mm in the TAVI group. A standard clinical workflow was established using the image fusion system. These pilot data confirm the technical feasibility of accurate real-time echo-fluoroscopic image overlay in clinical practice, which may be a useful adjunct for real-time guidance during interventional cardiac procedures.]]></abstract>
##     <issn><![CDATA[2168-2372]]></issn>
##     <htmlFlag><![CDATA[1]]></htmlFlag>
##     <arnumber><![CDATA[6729029]]></arnumber>
##     <doi><![CDATA[10.1109/JTEHM.2014.2303799]]></doi>
##     <publicationId><![CDATA[6729029]]></publicationId>
##     <mdurl><![CDATA[http://ieeexplore.ieee.org/xpl/articleDetails.jsp?tp=&arnumber=6729029&contentType=Journals+%26+Magazines]]></mdurl>
##     <pdf><![CDATA[http://ieeexplore.ieee.org/stamp/stamp.jsp?arnumber=6729029]]></pdf>
##   </document>
##   <document>
##     <rank>771</rank>
##     <title><![CDATA[Plasmons-Enhanced Minority-Carrier Injection as a Measure of Potential Fluctuations in Heavily Doped Silicon]]></title>
##     <authors><![CDATA[Ming-Jer Chen;  Chuan-Li Chen;  Shang-Hsun Hsieh;  Li-Ming Chang]]></authors>
##     <affiliations><![CDATA[Dept. of Electron. Eng., Nat. Chiao Tung Univ., Hsinchu, Taiwan]]></affiliations>
##     <controlledterms>
##       <term><![CDATA[Gaussian distribution]]></term>
##       <term><![CDATA[bipolar transistors]]></term>
##       <term><![CDATA[carrier density]]></term>
##       <term><![CDATA[conduction bands]]></term>
##       <term><![CDATA[energy gap]]></term>
##       <term><![CDATA[field effect transistors]]></term>
##       <term><![CDATA[minority carriers]]></term>
##       <term><![CDATA[plasmons]]></term>
##       <term><![CDATA[valence bands]]></term>
##     </controlledterms>
##     <thesaurusterms>
##       <term><![CDATA[Doping]]></term>
##       <term><![CDATA[Electric potential]]></term>
##       <term><![CDATA[IEEE Potentials]]></term>
##       <term><![CDATA[MOSFET]]></term>
##       <term><![CDATA[Photonic band gap]]></term>
##       <term><![CDATA[Plasmons]]></term>
##       <term><![CDATA[Silicon]]></term>
##     </thesaurusterms>
##     <pubtitle><![CDATA[Electron Device Letters, IEEE]]></pubtitle>
##     <punumber><![CDATA[55]]></punumber>
##     <pubtype><![CDATA[Journals & Magazines]]></pubtype>
##     <publisher><![CDATA[IEEE]]></publisher>
##     <volume><![CDATA[35]]></volume>
##     <issue><![CDATA[7]]></issue>
##     <py><![CDATA[2014]]></py>
##     <spage><![CDATA[708]]></spage>
##     <epage><![CDATA[710]]></epage>
##     <abstract><![CDATA[Well-known apparent electrical silicon bandgap narrowing in a heavily doped region of a bipolar transistor is observed by means of an enhanced minority-carrier injection experiment. In the region of interest, plasmons-induced potential fluctuations are existent in nature and hence constitute the origin of apparent bandgap narrowing. In this letter, we extract the underlying potential fluctuations directly from the enhanced minority-carrier injection experiment published in the literature. The core of the extraction lies in a combination of the two existing theoretical frameworks. First, the Gaussian distribution can serve as a good approximation of potential fluctuations. Second, no change can be made in the real bandgap between fluctuating conduction- and valence-band edges. Extracted potential fluctuations come from plasmons, as verified by our published temperature dependences of plasmons limited mobility in the inversion layer of MOSFETs as well as theoretical calculation results on bulk silicon. More importantly, this letter can deliver potential applications in the modeling and simulation area of nanoscale FETs (MOSFETs, FinFETs, and so forth) and bulk semiconductors.]]></abstract>
##     <issn><![CDATA[0741-3106]]></issn>
##     <htmlFlag><![CDATA[1]]></htmlFlag>
##     <arnumber><![CDATA[6827917]]></arnumber>
##     <doi><![CDATA[10.1109/LED.2014.2325030]]></doi>
##     <publicationId><![CDATA[6827917]]></publicationId>
##     <mdurl><![CDATA[http://ieeexplore.ieee.org/xpl/articleDetails.jsp?tp=&arnumber=6827917&contentType=Journals+%26+Magazines]]></mdurl>
##     <pdf><![CDATA[http://ieeexplore.ieee.org/stamp/stamp.jsp?arnumber=6827917]]></pdf>
##   </document>
##   <document>
##     <rank>772</rank>
##     <title><![CDATA[A Layer Stripping Approach for EM Reconstruction of Stratified Media]]></title>
##     <authors><![CDATA[Caorsi, S.;  Stasolla, M.]]></authors>
##     <affiliations><![CDATA[Dipt. di Ing. Ind. e dell' Inf., Univ. of Pavia, Pavia, Italy]]></affiliations>
##     <controlledterms>
##       <term><![CDATA[geophysical signal processing]]></term>
##       <term><![CDATA[inhomogeneous media]]></term>
##       <term><![CDATA[permittivity]]></term>
##       <term><![CDATA[signal reconstruction]]></term>
##     </controlledterms>
##     <thesaurusterms>
##       <term><![CDATA[Algorithm design and analysis]]></term>
##       <term><![CDATA[Dielectrics]]></term>
##       <term><![CDATA[Image reconstruction]]></term>
##       <term><![CDATA[Media]]></term>
##       <term><![CDATA[Permittivity]]></term>
##       <term><![CDATA[Receivers]]></term>
##       <term><![CDATA[Time-domain analysis]]></term>
##     </thesaurusterms>
##     <pubtitle><![CDATA[Geoscience and Remote Sensing, IEEE Transactions on]]></pubtitle>
##     <punumber><![CDATA[36]]></punumber>
##     <pubtype><![CDATA[Journals & Magazines]]></pubtype>
##     <publisher><![CDATA[IEEE]]></publisher>
##     <volume><![CDATA[52]]></volume>
##     <issue><![CDATA[9]]></issue>
##     <py><![CDATA[2014]]></py>
##     <spage><![CDATA[5855]]></spage>
##     <epage><![CDATA[5869]]></epage>
##     <abstract><![CDATA[This paper presents an electromagnetic (EM) technique for the reconstruction of the physical and geometrical properties (permittivity and thickness) of stratified media. The key points of the approach, belonging to the so-called layer stripping algorithms, are the introduction of an equalization step that takes into account propagation effects, and the design of a procedure devoted to multiple reflections' removal. Furthermore, the proposed main processing block is an energy-based method able to accurately estimate amplitudes and time of delays of backscattered echoes in the time domain. A numerical analysis of the algorithm's potentialities will show that it can be successfully employed under different working conditions and in the presence of noisy data.]]></abstract>
##     <issn><![CDATA[0196-2892]]></issn>
##     <htmlFlag><![CDATA[1]]></htmlFlag>
##     <arnumber><![CDATA[6689325]]></arnumber>
##     <doi><![CDATA[10.1109/TGRS.2013.2293533]]></doi>
##     <publicationId><![CDATA[6689325]]></publicationId>
##     <mdurl><![CDATA[http://ieeexplore.ieee.org/xpl/articleDetails.jsp?tp=&arnumber=6689325&contentType=Journals+%26+Magazines]]></mdurl>
##     <pdf><![CDATA[http://ieeexplore.ieee.org/stamp/stamp.jsp?arnumber=6689325]]></pdf>
##   </document>
##   <document>
##     <rank>773</rank>
##     <title><![CDATA[<formula formulatype="inline"> <img src="/images/tex/233.gif" alt="M"> </formula>-Channel Oversampled Graph Filter Banks]]></title>
##     <authors><![CDATA[Tanaka, Y.;  Sakiyama, A.]]></authors>
##     <affiliations><![CDATA[Grad. Sch. of BASE, Tokyo Univ. of Agric. & Technol., Tokyo, Japan]]></affiliations>
##     <controlledterms>
##       <term><![CDATA[Laplace equations]]></term>
##       <term><![CDATA[channel bank filters]]></term>
##       <term><![CDATA[graph theory]]></term>
##       <term><![CDATA[matrix algebra]]></term>
##       <term><![CDATA[signal denoising]]></term>
##       <term><![CDATA[signal reconstruction]]></term>
##       <term><![CDATA[signal sampling]]></term>
##     </controlledterms>
##     <thesaurusterms>
##       <term><![CDATA[Bipartite graph]]></term>
##       <term><![CDATA[Image reconstruction]]></term>
##       <term><![CDATA[Laplace equations]]></term>
##       <term><![CDATA[Spectral analysis]]></term>
##       <term><![CDATA[Wavelet transforms]]></term>
##     </thesaurusterms>
##     <pubtitle><![CDATA[Signal Processing, IEEE Transactions on]]></pubtitle>
##     <punumber><![CDATA[78]]></punumber>
##     <pubtype><![CDATA[Journals & Magazines]]></pubtype>
##     <publisher><![CDATA[IEEE]]></publisher>
##     <volume><![CDATA[62]]></volume>
##     <issue><![CDATA[14]]></issue>
##     <py><![CDATA[2014]]></py>
##     <spage><![CDATA[3578]]></spage>
##     <epage><![CDATA[3590]]></epage>
##     <abstract><![CDATA[This paper proposes M-channel oversampled filter banks for graph signals. The filter set satisfies the perfect reconstruction condition. A method of designing oversampled graph filter banks is presented that allows us to design filters with arbitrary parameters, unlike the conventional critically sampled graph filter banks. The oversampled graph Laplacian matrix is also introduced with a discussion of the entire redundancy of the oversampled graph signal processing system. The practical performance of the proposed filter banks is validated through graph signal denoising experiments.]]></abstract>
##     <issn><![CDATA[1053-587X]]></issn>
##     <htmlFlag><![CDATA[1]]></htmlFlag>
##     <arnumber><![CDATA[6825829]]></arnumber>
##     <doi><![CDATA[10.1109/TSP.2014.2328983]]></doi>
##     <publicationId><![CDATA[6825829]]></publicationId>
##     <mdurl><![CDATA[http://ieeexplore.ieee.org/xpl/articleDetails.jsp?tp=&arnumber=6825829&contentType=Journals+%26+Magazines]]></mdurl>
##     <pdf><![CDATA[http://ieeexplore.ieee.org/stamp/stamp.jsp?arnumber=6825829]]></pdf>
##   </document>
##   <document>
##     <rank>774</rank>
##     <title><![CDATA[Multidimensional Optimal Signal Constellation Sets and Symbol Mappings for Block-Interleaved Coded-Modulation Enabling Ultrahigh-Speed Optical Transport]]></title>
##     <authors><![CDATA[Tao Liu;  Djordjevic, I.B.]]></authors>
##     <affiliations><![CDATA[Dept. of Electr. & Comput. Eng., Univ. of Arizona, Tucson, AZ, USA]]></affiliations>
##     <controlledterms>
##       <term><![CDATA[error statistics]]></term>
##       <term><![CDATA[iterative decoding]]></term>
##       <term><![CDATA[mean square error methods]]></term>
##       <term><![CDATA[minimisation]]></term>
##       <term><![CDATA[modulation coding]]></term>
##       <term><![CDATA[optical design techniques]]></term>
##       <term><![CDATA[optical fibre communication]]></term>
##       <term><![CDATA[parity check codes]]></term>
##     </controlledterms>
##     <thesaurusterms>
##       <term><![CDATA[Adaptive optics]]></term>
##       <term><![CDATA[Algorithm design and analysis]]></term>
##       <term><![CDATA[Constellation diagram]]></term>
##       <term><![CDATA[Euclidean distance]]></term>
##       <term><![CDATA[Iterative decoding]]></term>
##       <term><![CDATA[Training]]></term>
##     </thesaurusterms>
##     <pubtitle><![CDATA[Photonics Journal, IEEE]]></pubtitle>
##     <punumber><![CDATA[4563994]]></punumber>
##     <pubtype><![CDATA[Journals & Magazines]]></pubtype>
##     <publisher><![CDATA[IEEE]]></publisher>
##     <volume><![CDATA[6]]></volume>
##     <issue><![CDATA[4]]></issue>
##     <py><![CDATA[2014]]></py>
##     <spage><![CDATA[1]]></spage>
##     <epage><![CDATA[14]]></epage>
##     <abstract><![CDATA[In this paper, we introduce the optimal signal constellation design (OSCD) in both 2-D and multidimensional cases, which is obtained by minimization of mean square error of optimal source distribution. In addition, by analyzing block-interleaved coded modulation with iterative decoding, in which the bit error rate is reduced by iteration of extrinsic information between a multilevel/multidimensional demapper and an LDPC decoder, with the help of EXIT chart, we propose new mapping methods for OSCDs outperforming previously known ones. The simulation results show that optimized mappings for LDPC-coded OSCDs outperform natural mappings by 0.5 dB in 8-ary 2-D case and 0.6 dB in 16-ary 2-D case. Meanwhile, our proposed optimized mappings for 3-D-8-ary OSCD outperform natural mapping by 0.2 dB and sphere packing constellation by 0.5 dB in coherent optical OFDM few-mode fiber system.]]></abstract>
##     <issn><![CDATA[1943-0655]]></issn>
##     <htmlFlag><![CDATA[1]]></htmlFlag>
##     <arnumber><![CDATA[6844916]]></arnumber>
##     <doi><![CDATA[10.1109/JPHOT.2014.2331233]]></doi>
##     <publicationId><![CDATA[6844916]]></publicationId>
##     <mdurl><![CDATA[http://ieeexplore.ieee.org/xpl/articleDetails.jsp?tp=&arnumber=6844916&contentType=Journals+%26+Magazines]]></mdurl>
##     <pdf><![CDATA[http://ieeexplore.ieee.org/stamp/stamp.jsp?arnumber=6844916]]></pdf>
##   </document>
##   <document>
##     <rank>775</rank>
##     <title><![CDATA[Via Diode in Cu Backend Process for 3D Cross-Point RRAM Arrays]]></title>
##     <authors><![CDATA[Yu-Cheng Liao;  Hsin-Wei Pan;  Min-Che Hsieh;  Tzong-Sheng Chang;  Yu-Der Chih;  Ming-Jinn Tsai;  Chrong Jung Lin;  Ya-Chin King]]></authors>
##     <affiliations><![CDATA[Inst. of Electron. Eng., Nat. Tsing Hua Univ., Hsinchu, Taiwan]]></affiliations>
##     <controlledterms>
##       <term><![CDATA[CMOS memory circuits]]></term>
##       <term><![CDATA[copper]]></term>
##       <term><![CDATA[dielectric materials]]></term>
##       <term><![CDATA[diodes]]></term>
##       <term><![CDATA[embedded systems]]></term>
##       <term><![CDATA[integrated circuit metallisation]]></term>
##       <term><![CDATA[nanoelectronics]]></term>
##       <term><![CDATA[random-access storage]]></term>
##       <term><![CDATA[tantalum compounds]]></term>
##       <term><![CDATA[three-dimensional integrated circuits]]></term>
##     </controlledterms>
##     <thesaurusterms>
##       <term><![CDATA[CMOS technology]]></term>
##       <term><![CDATA[Copper]]></term>
##       <term><![CDATA[Leakage currents]]></term>
##       <term><![CDATA[Nonvolatile memory]]></term>
##       <term><![CDATA[Random access memory]]></term>
##     </thesaurusterms>
##     <pubtitle><![CDATA[Electron Devices Society, IEEE Journal of the]]></pubtitle>
##     <punumber><![CDATA[6245494]]></punumber>
##     <pubtype><![CDATA[Journals & Magazines]]></pubtype>
##     <publisher><![CDATA[IEEE]]></publisher>
##     <volume><![CDATA[2]]></volume>
##     <issue><![CDATA[6]]></issue>
##     <py><![CDATA[2014]]></py>
##     <spage><![CDATA[149]]></spage>
##     <epage><![CDATA[153]]></epage>
##     <abstract><![CDATA[In this paper, a fully logic compatible via diode is developed for high-density resistive random access memory (RRAM) array applications. This novel via diode is realized by advanced 28nm CMOS technology with Cu damascene via. The device is stacked between a top Cu via and a bottom Cu metal with a composite layer of TaN/TaON based dielectric film. An asymmetric current-voltage characteristic in this MIM structure provides a forward/reverse current ratio up to 10<sup>6</sup>. In a cross-point RRAM array, the suppression of sneak current path by incorporating this via diode enables array size to be greatly expended. Via diode provides an excellent solution for high-density embedded nonvolatile memory applications in the nano-scale CMOS technology.]]></abstract>
##     <issn><![CDATA[2168-6734]]></issn>
##     <htmlFlag><![CDATA[1]]></htmlFlag>
##     <arnumber><![CDATA[6857314]]></arnumber>
##     <doi><![CDATA[10.1109/JEDS.2014.2339296]]></doi>
##     <publicationId><![CDATA[6857314]]></publicationId>
##     <mdurl><![CDATA[http://ieeexplore.ieee.org/xpl/articleDetails.jsp?tp=&arnumber=6857314&contentType=Journals+%26+Magazines]]></mdurl>
##     <pdf><![CDATA[http://ieeexplore.ieee.org/stamp/stamp.jsp?arnumber=6857314]]></pdf>
##   </document>
##   <document>
##     <rank>776</rank>
##     <title><![CDATA[Synchronized Zero-Crossing-Based Self-Tuning Capacitor Time-Constant Estimator for Low-Power Digitally Controlled DC&#x2013;DC Converters]]></title>
##     <authors><![CDATA[Radic, A.;  Straka, A.;  Prodic, A.]]></authors>
##     <affiliations><![CDATA[Dept. of Electr. & Comput. Eng., Univ. of Toronto, Toronto, ON, Canada]]></affiliations>
##     <controlledterms>
##       <term><![CDATA[DC-DC power convertors]]></term>
##       <term><![CDATA[PWM power convertors]]></term>
##       <term><![CDATA[RC circuits]]></term>
##       <term><![CDATA[capacitors]]></term>
##       <term><![CDATA[equivalent circuits]]></term>
##       <term><![CDATA[switching convertors]]></term>
##       <term><![CDATA[zero current switching]]></term>
##       <term><![CDATA[zero voltage switching]]></term>
##     </controlledterms>
##     <thesaurusterms>
##       <term><![CDATA[Capacitors]]></term>
##       <term><![CDATA[Current measurement]]></term>
##       <term><![CDATA[Estimation]]></term>
##       <term><![CDATA[Resistors]]></term>
##       <term><![CDATA[Switched-mode power supply]]></term>
##       <term><![CDATA[Switches]]></term>
##     </thesaurusterms>
##     <pubtitle><![CDATA[Power Electronics, IEEE Transactions on]]></pubtitle>
##     <punumber><![CDATA[63]]></punumber>
##     <pubtype><![CDATA[Journals & Magazines]]></pubtype>
##     <publisher><![CDATA[IEEE]]></publisher>
##     <volume><![CDATA[29]]></volume>
##     <issue><![CDATA[10]]></issue>
##     <py><![CDATA[2014]]></py>
##     <spage><![CDATA[5106]]></spage>
##     <epage><![CDATA[5110]]></epage>
##     <abstract><![CDATA[A hardware efficient method and system for estimating the time constant and measuring the current of the output capacitors in digitally controlled switch-mode power supplies is introduced. The estimator emulates the equivalent RC circuit of the output capacitor with a much smaller version, placed in parallel, and adjusts its own resistance until the two circuits have the same time constant. The adjustment is based on a novel zero voltage crossing detection method and on synchronization with the digital pulse-width modulator operation. The effectiveness of the new estimator is verified with a 5 V to 1 V/5 A, 500-kHz buck converter prototype utilizing an optimal response controller. The experimental results show accuracy within a few tens of nanoseconds in the detection of capacitor zero-current crossing points, corresponding to a smaller than a 1.5% error in the time constant estimation, and, compared to an imperfectly tuned system, about 40% smaller voltage deviation during transients.]]></abstract>
##     <issn><![CDATA[0885-8993]]></issn>
##     <htmlFlag><![CDATA[1]]></htmlFlag>
##     <arnumber><![CDATA[6778751]]></arnumber>
##     <doi><![CDATA[10.1109/TPEL.2014.2313727]]></doi>
##     <publicationId><![CDATA[6778751]]></publicationId>
##     <mdurl><![CDATA[http://ieeexplore.ieee.org/xpl/articleDetails.jsp?tp=&arnumber=6778751&contentType=Journals+%26+Magazines]]></mdurl>
##     <pdf><![CDATA[http://ieeexplore.ieee.org/stamp/stamp.jsp?arnumber=6778751]]></pdf>
##   </document>
##   <document>
##     <rank>777</rank>
##     <title><![CDATA[Reflectance Response of Optical Fiber Coated With Carbon Nanotubes for Aqueous Ethanol Sensing]]></title>
##     <authors><![CDATA[Shabaneh, A.A.;  Girei, S.H.;  Arasu, P.T.;  Rashid, S.A.;  Yunusa, Z.;  Mahdi, M.A.;  Paiman, S.;  Ahmad, M.Z.;  Yaacob, M.H.]]></authors>
##     <affiliations><![CDATA[Wireless & Photonics Network Res. Centre, Univ. Putra Malaysia, Serdang, Malaysia]]></affiliations>
##     <controlledterms>
##       <term><![CDATA[Raman spectra]]></term>
##       <term><![CDATA[annealing]]></term>
##       <term><![CDATA[carbon nanotubes]]></term>
##       <term><![CDATA[casting]]></term>
##       <term><![CDATA[chemical sensors]]></term>
##       <term><![CDATA[fibre optic sensors]]></term>
##       <term><![CDATA[field emission electron microscopy]]></term>
##       <term><![CDATA[nanophotonics]]></term>
##       <term><![CDATA[optical fibre fabrication]]></term>
##       <term><![CDATA[optical films]]></term>
##       <term><![CDATA[organic compounds]]></term>
##       <term><![CDATA[reflectivity]]></term>
##       <term><![CDATA[scanning electron microscopy]]></term>
##       <term><![CDATA[silicon compounds]]></term>
##       <term><![CDATA[spectrophotometers]]></term>
##     </controlledterms>
##     <thesaurusterms>
##       <term><![CDATA[Ethanol]]></term>
##       <term><![CDATA[Optical fiber dispersion]]></term>
##       <term><![CDATA[Optical fiber sensors]]></term>
##       <term><![CDATA[Optical fibers]]></term>
##       <term><![CDATA[Reflectivity]]></term>
##       <term><![CDATA[Substrates]]></term>
##     </thesaurusterms>
##     <pubtitle><![CDATA[Photonics Journal, IEEE]]></pubtitle>
##     <punumber><![CDATA[4563994]]></punumber>
##     <pubtype><![CDATA[Journals & Magazines]]></pubtype>
##     <publisher><![CDATA[IEEE]]></publisher>
##     <volume><![CDATA[6]]></volume>
##     <issue><![CDATA[6]]></issue>
##     <py><![CDATA[2014]]></py>
##     <spage><![CDATA[1]]></spage>
##     <epage><![CDATA[10]]></epage>
##     <abstract><![CDATA[Ethanol is a highly flammable chemical and is widely used for medical and industrial applications. In this paper, optical sensing performance of aqueous ethanol with different concentrations is investigated using multimode fiber coated with carbon nanotubes (CNT). The multimode optical fiber tip is coated with CNT via a drop-casting technique and is annealed at 70 &#x00B0;C to improve the binding of the nanomaterial to the silica fiber. The optical fiber tip and the CNT sensing layer are microcharacterized using field emission scanning electron microscopy, Raman spectroscopy, and X-ray diffraction techniques. The reflectance response of the developed fiber sensor is measured using a spectrophotometer in the optical wavelength range of 500-800 nm. Upon exposure to ethanol with concentration ranges of 5%-80%, the sensor reflectance reduced proportionally. The dynamic response decreased by 4% when the sensor is exposed to ethanol with concentration of 80% in distilled water. It is found that the sensor shows fast response and recovery as low as 38 and 49 s, respectively.]]></abstract>
##     <issn><![CDATA[1943-0655]]></issn>
##     <htmlFlag><![CDATA[1]]></htmlFlag>
##     <arnumber><![CDATA[6926776]]></arnumber>
##     <doi><![CDATA[10.1109/JPHOT.2014.2363429]]></doi>
##     <publicationId><![CDATA[6926776]]></publicationId>
##     <mdurl><![CDATA[http://ieeexplore.ieee.org/xpl/articleDetails.jsp?tp=&arnumber=6926776&contentType=Journals+%26+Magazines]]></mdurl>
##     <pdf><![CDATA[http://ieeexplore.ieee.org/stamp/stamp.jsp?arnumber=6926776]]></pdf>
##   </document>
##   <document>
##     <rank>778</rank>
##     <title><![CDATA[Ka-Band Tunable Flat-Top Microwave Photonic Filter Using a Multi-Phase-Shifted Fiber Bragg Grating]]></title>
##     <authors><![CDATA[Ye Deng;  Ming Li;  Ningbo Huang;  Ninghua Zhu]]></authors>
##     <affiliations><![CDATA[Inst. of Semicond., Beijing, China]]></affiliations>
##     <controlledterms>
##       <term><![CDATA[Bragg gratings]]></term>
##       <term><![CDATA[band-pass filters]]></term>
##       <term><![CDATA[microwave photonics]]></term>
##       <term><![CDATA[optical fibre filters]]></term>
##       <term><![CDATA[optical transfer function]]></term>
##       <term><![CDATA[optical tuning]]></term>
##     </controlledterms>
##     <thesaurusterms>
##       <term><![CDATA[Fiber gratings]]></term>
##       <term><![CDATA[Microwave filters]]></term>
##       <term><![CDATA[Microwave photonics]]></term>
##       <term><![CDATA[Optical fibers]]></term>
##       <term><![CDATA[Optical filters]]></term>
##       <term><![CDATA[Passband]]></term>
##     </thesaurusterms>
##     <pubtitle><![CDATA[Photonics Journal, IEEE]]></pubtitle>
##     <punumber><![CDATA[4563994]]></punumber>
##     <pubtype><![CDATA[Journals & Magazines]]></pubtype>
##     <publisher><![CDATA[IEEE]]></publisher>
##     <volume><![CDATA[6]]></volume>
##     <issue><![CDATA[4]]></issue>
##     <py><![CDATA[2014]]></py>
##     <spage><![CDATA[1]]></spage>
##     <epage><![CDATA[8]]></epage>
##     <abstract><![CDATA[A tunable single-bandpass microwave photonic filter (MPF) with a narrow and flat-top shape operating in high frequency range is proposed and experimentally demonstrated, using a specially designed multi-phase-shift fiber Bragg grating (FBG) cascaded with a programmable optical filter (POF). The key device in the proposed MPF is the multi-phase-shift FBG, providing a flat-top passband with a narrow 3-dB bandwidth of 813 MHz in transmission. Consequently, in the MPF, a single passband with a 3- and 20-dB bandwidths of 840 MHz and 2.43 GHz is achieved, respectively. Moreover, an operating frequency range from 27.1 to 38.1 GHz is realized by tuning the wavelength of the optical carrier and adjusting the transfer function of the POF. To the best of our knowledge, this proposed filter provides the highest tunable frequency coverage in Ka-band ever reported for MPFs with a single narrow and flat-top passband.]]></abstract>
##     <issn><![CDATA[1943-0655]]></issn>
##     <htmlFlag><![CDATA[1]]></htmlFlag>
##     <arnumber><![CDATA[6863635]]></arnumber>
##     <doi><![CDATA[10.1109/JPHOT.2014.2339327]]></doi>
##     <publicationId><![CDATA[6863635]]></publicationId>
##     <mdurl><![CDATA[http://ieeexplore.ieee.org/xpl/articleDetails.jsp?tp=&arnumber=6863635&contentType=Journals+%26+Magazines]]></mdurl>
##     <pdf><![CDATA[http://ieeexplore.ieee.org/stamp/stamp.jsp?arnumber=6863635]]></pdf>
##   </document>
##   <document>
##     <rank>779</rank>
##     <title><![CDATA[Optimizing Low-Frequency Mode Stirring Performance Using Principal Component Analysis]]></title>
##     <authors><![CDATA[Arnaut, L.R.]]></authors>
##     <affiliations><![CDATA[George Green Inst. of Electromagn. Res., Univ. of Nottingham, Nottingham, UK]]></affiliations>
##     <controlledterms>
##       <term><![CDATA[electromagnetic devices]]></term>
##       <term><![CDATA[principal component analysis]]></term>
##       <term><![CDATA[reverberation chambers]]></term>
##     </controlledterms>
##     <thesaurusterms>
##       <term><![CDATA[Correlation]]></term>
##       <term><![CDATA[Frequency measurement]]></term>
##       <term><![CDATA[Loading]]></term>
##       <term><![CDATA[Manganese]]></term>
##       <term><![CDATA[Principal component analysis]]></term>
##       <term><![CDATA[Vectors]]></term>
##       <term><![CDATA[Zinc]]></term>
##     </thesaurusterms>
##     <pubtitle><![CDATA[Electromagnetic Compatibility, IEEE Transactions on]]></pubtitle>
##     <punumber><![CDATA[15]]></punumber>
##     <pubtype><![CDATA[Journals & Magazines]]></pubtype>
##     <publisher><![CDATA[IEEE]]></publisher>
##     <volume><![CDATA[56]]></volume>
##     <issue><![CDATA[1]]></issue>
##     <py><![CDATA[2014]]></py>
##     <spage><![CDATA[3]]></spage>
##     <epage><![CDATA[14]]></epage>
##     <abstract><![CDATA[We formulate and perform principal component analysis (PCA) of mechanically stirred fields, based on tuner sweep data collected across a frequency band at a single location of a sensor inside a reverberation chamber. Both covariance- and correlation-based PCA in undermoded and overmoded regime are performed and intercompared. The nonstationarity of the stir performance as a function of the angular position of the stirrer is demonstrated. It is shown that this nonuniformity can be quantified and exploited to select a set of optimal stir angles. The rotated principal components are found to be interpretable as energy stirred by specific angular sectors of the stirrer and are related to the correlation structure of the data. The analysis leads to the concept of eigen-stirrings (stir modes), which form an orthonormal set of empirical basis functions for expanding stir data.]]></abstract>
##     <issn><![CDATA[0018-9375]]></issn>
##     <htmlFlag><![CDATA[1]]></htmlFlag>
##     <arnumber><![CDATA[6572839]]></arnumber>
##     <doi><![CDATA[10.1109/TEMC.2013.2271903]]></doi>
##     <publicationId><![CDATA[6572839]]></publicationId>
##     <mdurl><![CDATA[http://ieeexplore.ieee.org/xpl/articleDetails.jsp?tp=&arnumber=6572839&contentType=Journals+%26+Magazines]]></mdurl>
##     <pdf><![CDATA[http://ieeexplore.ieee.org/stamp/stamp.jsp?arnumber=6572839]]></pdf>
##   </document>
##   <document>
##     <rank>780</rank>
##     <title><![CDATA[20-W Wavelength-Tunable Picosecond Yb-Doped Fiber MOPA Source Seeded by an NPR Mode-Locked Fiber Laser]]></title>
##     <authors><![CDATA[Huaiqin Lin;  Chunyu Guo;  Shuangchen Ruan;  Ruhua Wen;  Jun Yu;  Weiqi Liu]]></authors>
##     <affiliations><![CDATA[Shenzhen Key Lab. of Laser Eng., Shenzhen Univ., Shenzhen, China]]></affiliations>
##     <controlledterms>
##       <term><![CDATA[birefringence]]></term>
##       <term><![CDATA[high-speed optical techniques]]></term>
##       <term><![CDATA[laser mode locking]]></term>
##       <term><![CDATA[laser tuning]]></term>
##       <term><![CDATA[optical fibre amplifiers]]></term>
##       <term><![CDATA[ytterbium]]></term>
##     </controlledterms>
##     <thesaurusterms>
##       <term><![CDATA[Erbium-doped fiber lasers]]></term>
##       <term><![CDATA[Laser mode locking]]></term>
##       <term><![CDATA[Optical fiber amplifiers]]></term>
##       <term><![CDATA[Optical fiber dispersion]]></term>
##       <term><![CDATA[Optical fiber polarization]]></term>
##       <term><![CDATA[Power amplifiers]]></term>
##     </thesaurusterms>
##     <pubtitle><![CDATA[Photonics Journal, IEEE]]></pubtitle>
##     <punumber><![CDATA[4563994]]></punumber>
##     <pubtype><![CDATA[Journals & Magazines]]></pubtype>
##     <publisher><![CDATA[IEEE]]></publisher>
##     <volume><![CDATA[6]]></volume>
##     <issue><![CDATA[5]]></issue>
##     <py><![CDATA[2014]]></py>
##     <spage><![CDATA[1]]></spage>
##     <epage><![CDATA[6]]></epage>
##     <abstract><![CDATA[We demonstrated a high-power wavelength-tunable picosecond Yb-doped fiber master oscillator power amplifier source without any tunable elements for the first time. This wavelength-tunable output was realized based on the birefringence-induced filter effect of the nonlinear polarization rotation mode-locked fiber laser, which worked as the seed source. Through cascaded single-mode Yb-doped fiber preamplifiers together with two double-cladding Yb-doped fiber amplifiers, the pulses were amplified up to 21.2-W output power in the wavelength range from 1038.4 to 1060 nm with a pulse duration of 9.5 ps, an optical signal-to-noise ratio of 25 dB, and a beam quality M<sup>2</sup> of 1.16 at the repetition rate of 178 MHz.]]></abstract>
##     <issn><![CDATA[1943-0655]]></issn>
##     <htmlFlag><![CDATA[1]]></htmlFlag>
##     <arnumber><![CDATA[6895107]]></arnumber>
##     <doi><![CDATA[10.1109/JPHOT.2014.2356511]]></doi>
##     <publicationId><![CDATA[6895107]]></publicationId>
##     <mdurl><![CDATA[http://ieeexplore.ieee.org/xpl/articleDetails.jsp?tp=&arnumber=6895107&contentType=Journals+%26+Magazines]]></mdurl>
##     <pdf><![CDATA[http://ieeexplore.ieee.org/stamp/stamp.jsp?arnumber=6895107]]></pdf>
##   </document>
##   <document>
##     <rank>781</rank>
##     <title><![CDATA[High-Performance 1.55- <inline-formula> <img src="/images/tex/527.gif" alt="\mu\hbox {m}"> </inline-formula> Superluminescent Diode Based on Broad Gain InAs/InGaAlAs/InP Quantum Dash Active Region]]></title>
##     <authors><![CDATA[Khan, M.Z.M.;  Ng, T.K.;  Ooi, B.S.]]></authors>
##     <affiliations><![CDATA[Math. Sci. & Eng. Div., King Abdullah Univ. of Sci. & Technol. (KAUST), Thuwal, Saudi Arabia]]></affiliations>
##     <controlledterms>
##       <term><![CDATA[III-V semiconductors]]></term>
##       <term><![CDATA[aluminium compounds]]></term>
##       <term><![CDATA[gallium arsenide]]></term>
##       <term><![CDATA[indium compounds]]></term>
##       <term><![CDATA[quantum dash lasers]]></term>
##       <term><![CDATA[superluminescent diodes]]></term>
##     </controlledterms>
##     <thesaurusterms>
##       <term><![CDATA[Absorption]]></term>
##       <term><![CDATA[Bandwidth]]></term>
##       <term><![CDATA[Chirp]]></term>
##       <term><![CDATA[Current density]]></term>
##       <term><![CDATA[Materials]]></term>
##       <term><![CDATA[Superluminescent diodes]]></term>
##       <term><![CDATA[Temperature measurement]]></term>
##     </thesaurusterms>
##     <pubtitle><![CDATA[Photonics Journal, IEEE]]></pubtitle>
##     <punumber><![CDATA[4563994]]></punumber>
##     <pubtype><![CDATA[Journals & Magazines]]></pubtype>
##     <publisher><![CDATA[IEEE]]></publisher>
##     <volume><![CDATA[6]]></volume>
##     <issue><![CDATA[4]]></issue>
##     <py><![CDATA[2014]]></py>
##     <spage><![CDATA[1]]></spage>
##     <epage><![CDATA[8]]></epage>
##     <abstract><![CDATA[We report on the high-performance characteristics from superluminescent diodes (SLDs) based on four-stack InAs/InGaAlAs chirped-barrier thickness quantum dash (Qdash) in a well structure. The active region exhibits a measured broad gain spectrum of 140 nm, with a peak modal gain of ~41 cm<sup>-1</sup>. The noncoated two-section gainabsorber broad-area and ridge-waveguide device configuration exhibits an output power of &gt; 20 mW and &gt; 12 mW, respectively. The corresponding -3-dB bandwidths span ~82 nm and ~72 nm, with a small spectral ripple of &lt;; 0.2 dB, related largely to the contribution from dispersive height dash ensembles of the highly inhomogeneous active region. These C-L communication band devices will find applications in various cross-disciplinary fields of optical metrology, optical coherent tomography, etc.]]></abstract>
##     <issn><![CDATA[1943-0655]]></issn>
##     <htmlFlag><![CDATA[1]]></htmlFlag>
##     <arnumber><![CDATA[6851845]]></arnumber>
##     <doi><![CDATA[10.1109/JPHOT.2014.2337892]]></doi>
##     <publicationId><![CDATA[6851845]]></publicationId>
##     <mdurl><![CDATA[http://ieeexplore.ieee.org/xpl/articleDetails.jsp?tp=&arnumber=6851845&contentType=Journals+%26+Magazines]]></mdurl>
##     <pdf><![CDATA[http://ieeexplore.ieee.org/stamp/stamp.jsp?arnumber=6851845]]></pdf>
##   </document>
##   <document>
##     <rank>782</rank>
##     <title><![CDATA[A Phase Distribution Design Method for Phased Arrays Multibeam Independently Generating and 3-D Scanning]]></title>
##     <authors><![CDATA[Dong Wang;  Yu-Li Wu;  Bao-Quan Jin;  Peng Jia;  Dong-Mei Cai]]></authors>
##     <affiliations><![CDATA[Key Lab. of Adv. Transducers & Intell. Control Syst., Minist. of Educ., Taiyuan Univ. of Technol., Taiyuan, China]]></affiliations>
##     <controlledterms>
##       <term><![CDATA[optical arrays]]></term>
##       <term><![CDATA[optical scanners]]></term>
##     </controlledterms>
##     <thesaurusterms>
##       <term><![CDATA[Design methodology]]></term>
##       <term><![CDATA[Diffraction]]></term>
##       <term><![CDATA[Iterative methods]]></term>
##       <term><![CDATA[Laser beams]]></term>
##       <term><![CDATA[Lasers]]></term>
##       <term><![CDATA[Lenses]]></term>
##       <term><![CDATA[Phased arrays]]></term>
##     </thesaurusterms>
##     <pubtitle><![CDATA[Photonics Journal, IEEE]]></pubtitle>
##     <punumber><![CDATA[4563994]]></punumber>
##     <pubtype><![CDATA[Journals & Magazines]]></pubtype>
##     <publisher><![CDATA[IEEE]]></publisher>
##     <volume><![CDATA[6]]></volume>
##     <issue><![CDATA[5]]></issue>
##     <py><![CDATA[2014]]></py>
##     <spage><![CDATA[1]]></spage>
##     <epage><![CDATA[11]]></epage>
##     <abstract><![CDATA[This paper reports the realization of a phase distribution design method for phased arrays multibeam independently generating and 3-D scanning. The phase distribution design method inherits the projection optimization idea from the generalized adaptive-additive algorithm; however, a striking difference here is that the iteration plane is not just one but multiple, which means the method promises to be not only suitable for creating phase distributions for multibeam independently lateral scanning but capable of creating phase distributions for independently axial scanning as well. Both simulations and experiments were conducted to investigate the performance of the method; simulation agrees with experimental results, which validates the effectiveness of the method proposed. The technique described in this paper could provide a promising convenient multibeam generating and 3-D scanning for ladar, laser weapons, laser micromachining, etc.]]></abstract>
##     <issn><![CDATA[1943-0655]]></issn>
##     <htmlFlag><![CDATA[1]]></htmlFlag>
##     <arnumber><![CDATA[6872523]]></arnumber>
##     <doi><![CDATA[10.1109/JPHOT.2014.2345882]]></doi>
##     <publicationId><![CDATA[6872523]]></publicationId>
##     <mdurl><![CDATA[http://ieeexplore.ieee.org/xpl/articleDetails.jsp?tp=&arnumber=6872523&contentType=Journals+%26+Magazines]]></mdurl>
##     <pdf><![CDATA[http://ieeexplore.ieee.org/stamp/stamp.jsp?arnumber=6872523]]></pdf>
##   </document>
##   <document>
##     <rank>783</rank>
##     <title><![CDATA[Physical Unclonable Functions and Applications: A Tutorial]]></title>
##     <authors><![CDATA[Herder, C.;  Meng-Day Yu;  Koushanfar, F.;  Devadas, S.]]></authors>
##     <affiliations><![CDATA[Dept. of Electr. Eng. & Comput. Sci. (EECS), Massachusetts Inst. of Technol., Cambridge, MA, USA]]></affiliations>
##     <controlledterms>
##       <term><![CDATA[Random access memory]]></term>
##       <term><![CDATA[Ring oscillators]]></term>
##       <term><![CDATA[Tutorials]]></term>
##       <term><![CDATA[authorisation]]></term>
##       <term><![CDATA[cryptographic protocols]]></term>
##     </controlledterms>
##     <thesaurusterms>
##       <term><![CDATA[Coding]]></term>
##       <term><![CDATA[Cryptography]]></term>
##       <term><![CDATA[Hardware]]></term>
##       <term><![CDATA[Indexes]]></term>
##       <term><![CDATA[Logic gates]]></term>
##       <term><![CDATA[Oscillators]]></term>
##       <term><![CDATA[Random access memory]]></term>
##       <term><![CDATA[Ring oscillators]]></term>
##       <term><![CDATA[SRAM chips]]></term>
##       <term><![CDATA[Tutorials]]></term>
##     </thesaurusterms>
##     <pubtitle><![CDATA[Proceedings of the IEEE]]></pubtitle>
##     <punumber><![CDATA[5]]></punumber>
##     <pubtype><![CDATA[Journals & Magazines]]></pubtype>
##     <publisher><![CDATA[IEEE]]></publisher>
##     <volume><![CDATA[102]]></volume>
##     <issue><![CDATA[8]]></issue>
##     <py><![CDATA[2014]]></py>
##     <spage><![CDATA[1126]]></spage>
##     <epage><![CDATA[1141]]></epage>
##     <abstract><![CDATA[This paper describes the use of physical unclonable functions (PUFs) in low-cost authentication and key generation applications. First, it motivates the use of PUFs versus conventional secure nonvolatile memories and defines the two primary PUF types: &#x201C;strong PUFs&#x201D; and &#x201C;weak PUFs.&#x201D; It describes strong PUF implementations and their use for low-cost authentication. After this description, the paper covers both attacks and protocols to address errors. Next, the paper covers weak PUF implementations and their use in key generation applications. It covers error-correction schemes such as pattern matching and index-based coding. Finally, this paper reviews several emerging concepts in PUF technologies such as public model PUFs and new PUF implementation technologies.]]></abstract>
##     <issn><![CDATA[0018-9219]]></issn>
##     <htmlFlag><![CDATA[1]]></htmlFlag>
##     <arnumber><![CDATA[6823677]]></arnumber>
##     <doi><![CDATA[10.1109/JPROC.2014.2320516]]></doi>
##     <publicationId><![CDATA[6823677]]></publicationId>
##     <mdurl><![CDATA[http://ieeexplore.ieee.org/xpl/articleDetails.jsp?tp=&arnumber=6823677&contentType=Journals+%26+Magazines]]></mdurl>
##     <pdf><![CDATA[http://ieeexplore.ieee.org/stamp/stamp.jsp?arnumber=6823677]]></pdf>
##   </document>
##   <document>
##     <rank>784</rank>
##     <title><![CDATA[Hierarchical Spectral Clustering of Power Grids]]></title>
##     <authors><![CDATA[Sanchez-Garcia, R.J.;  Fennelly, M.;  Norris, S.;  Wright, N.;  Niblo, G.;  Brodzki, J.;  Bialek, J.W.]]></authors>
##     <affiliations><![CDATA[Sch. of Math., Univ. of Southampton, Southampton, UK]]></affiliations>
##     <controlledterms>
##       <term><![CDATA[eigenvalues and eigenfunctions]]></term>
##       <term><![CDATA[load flow]]></term>
##       <term><![CDATA[pattern clustering]]></term>
##       <term><![CDATA[power grids]]></term>
##       <term><![CDATA[power system analysis computing]]></term>
##       <term><![CDATA[power transmission lines]]></term>
##       <term><![CDATA[transmission networks]]></term>
##     </controlledterms>
##     <thesaurusterms>
##       <term><![CDATA[Admittance]]></term>
##       <term><![CDATA[Clustering algorithms]]></term>
##       <term><![CDATA[Eigenvalues and eigenfunctions]]></term>
##       <term><![CDATA[Laplace equations]]></term>
##       <term><![CDATA[Power grids]]></term>
##       <term><![CDATA[Standards]]></term>
##       <term><![CDATA[Symmetric matrices]]></term>
##     </thesaurusterms>
##     <pubtitle><![CDATA[Power Systems, IEEE Transactions on]]></pubtitle>
##     <punumber><![CDATA[59]]></punumber>
##     <pubtype><![CDATA[Journals & Magazines]]></pubtype>
##     <publisher><![CDATA[IEEE]]></publisher>
##     <volume><![CDATA[29]]></volume>
##     <issue><![CDATA[5]]></issue>
##     <py><![CDATA[2014]]></py>
##     <spage><![CDATA[2229]]></spage>
##     <epage><![CDATA[2237]]></epage>
##     <abstract><![CDATA[A power transmission system can be represented by a network with nodes and links representing buses and electrical transmission lines, respectively. Each line can be given a weight, representing some electrical property of the line, such as line admittance or average power flow at a given time. We use a hierarchical spectral clustering methodology to reveal the internal connectivity structure of such a network. Spectral clustering uses the eigenvalues and eigenvectors of a matrix associated to the network, it is computationally very efficient, and it works for any choice of weights. When using line admittances, it reveals the static internal connectivity structure of the underlying network, while using power flows highlights islands with minimal power flow disruption, and thus it naturally relates to controlled islanding. Our methodology goes beyond the standard k-means algorithm by instead representing the complete network substructure as a dendrogram. We provide a thorough theoretical justification of the use of spectral clustering in power systems, and we include the results of our methodology for several test systems of small, medium and large size, including a model of the Great Britain transmission network.]]></abstract>
##     <issn><![CDATA[0885-8950]]></issn>
##     <htmlFlag><![CDATA[1]]></htmlFlag>
##     <arnumber><![CDATA[6774471]]></arnumber>
##     <doi><![CDATA[10.1109/TPWRS.2014.2306756]]></doi>
##     <publicationId><![CDATA[6774471]]></publicationId>
##     <mdurl><![CDATA[http://ieeexplore.ieee.org/xpl/articleDetails.jsp?tp=&arnumber=6774471&contentType=Journals+%26+Magazines]]></mdurl>
##     <pdf><![CDATA[http://ieeexplore.ieee.org/stamp/stamp.jsp?arnumber=6774471]]></pdf>
##   </document>
##   <document>
##     <rank>785</rank>
##     <title><![CDATA[Vector-Valued Image Processing by Parallel Level Sets]]></title>
##     <authors><![CDATA[Ehrhardt, M.J.;  Arridge, S.R.]]></authors>
##     <affiliations><![CDATA[Med. Phys. & Bioeng. Dept., Univ. Coll. London, London, UK]]></affiliations>
##     <controlledterms>
##       <term><![CDATA[gradient methods]]></term>
##       <term><![CDATA[image colour analysis]]></term>
##       <term><![CDATA[image denoising]]></term>
##       <term><![CDATA[image enhancement]]></term>
##       <term><![CDATA[image segmentation]]></term>
##     </controlledterms>
##     <thesaurusterms>
##       <term><![CDATA[Biomedical imaging]]></term>
##       <term><![CDATA[Equations]]></term>
##       <term><![CDATA[Image color analysis]]></term>
##       <term><![CDATA[Level set]]></term>
##       <term><![CDATA[Mathematical model]]></term>
##       <term><![CDATA[Noise reduction]]></term>
##     </thesaurusterms>
##     <pubtitle><![CDATA[Image Processing, IEEE Transactions on]]></pubtitle>
##     <punumber><![CDATA[83]]></punumber>
##     <pubtype><![CDATA[Journals & Magazines]]></pubtype>
##     <publisher><![CDATA[IEEE]]></publisher>
##     <volume><![CDATA[23]]></volume>
##     <issue><![CDATA[1]]></issue>
##     <py><![CDATA[2014]]></py>
##     <spage><![CDATA[9]]></spage>
##     <epage><![CDATA[18]]></epage>
##     <abstract><![CDATA[Vector-valued images such as RGB color images or multimodal medical images show a strong interchannel correlation, which is not exploited by most image processing tools. We propose a new notion of treating vector-valued images which is based on the angle between the spatial gradients of their channels. Through minimizing a cost functional that penalizes large angles, images with parallel level sets can be obtained. After formally introducing this idea and the corresponding cost functionals, we discuss their Ga&#x0302;teaux derivatives that lead to a diffusion-like gradient descent scheme. We illustrate the properties of this cost functional by several examples in denoising and demosaicking of RGB color images. They show that parallel level sets are a suitable concept for color image enhancement. Demosaicking with parallel level sets gives visually perfect results for low noise levels. Furthermore, the proposed functional yields sharper images than the other approaches in comparison.]]></abstract>
##     <issn><![CDATA[1057-7149]]></issn>
##     <htmlFlag><![CDATA[1]]></htmlFlag>
##     <arnumber><![CDATA[6576903]]></arnumber>
##     <doi><![CDATA[10.1109/TIP.2013.2277775]]></doi>
##     <publicationId><![CDATA[6576903]]></publicationId>
##     <mdurl><![CDATA[http://ieeexplore.ieee.org/xpl/articleDetails.jsp?tp=&arnumber=6576903&contentType=Journals+%26+Magazines]]></mdurl>
##     <pdf><![CDATA[http://ieeexplore.ieee.org/stamp/stamp.jsp?arnumber=6576903]]></pdf>
##   </document>
##   <document>
##     <rank>786</rank>
##     <title><![CDATA[Security and Reliability Performance Analysis for Cloud Radio Access Networks With Channel Estimation Errors]]></title>
##     <authors><![CDATA[Jia You;  Zhangdui Zhong;  Gongpu Wang;  Bo Ai]]></authors>
##     <affiliations><![CDATA[State Key Lab. of Rail Traffic Control & Safety, Beijing Jiaotong Univ., Beijing, China]]></affiliations>
##     <controlledterms>
##       <term><![CDATA[channel estimation]]></term>
##       <term><![CDATA[cloud computing]]></term>
##       <term><![CDATA[mean square error methods]]></term>
##       <term><![CDATA[radio access networks]]></term>
##       <term><![CDATA[telecommunication network reliability]]></term>
##       <term><![CDATA[telecommunication security]]></term>
##     </controlledterms>
##     <thesaurusterms>
##       <term><![CDATA[Channel capacity]]></term>
##       <term><![CDATA[Channel estimation]]></term>
##       <term><![CDATA[Decoding]]></term>
##       <term><![CDATA[Network security]]></term>
##       <term><![CDATA[Performance evaluation]]></term>
##       <term><![CDATA[Power outages]]></term>
##       <term><![CDATA[Signal to noise ratio]]></term>
##       <term><![CDATA[Wireless communication]]></term>
##     </thesaurusterms>
##     <pubtitle><![CDATA[Access, IEEE]]></pubtitle>
##     <punumber><![CDATA[6287639]]></punumber>
##     <pubtype><![CDATA[Journals & Magazines]]></pubtype>
##     <publisher><![CDATA[IEEE]]></publisher>
##     <volume><![CDATA[2]]></volume>
##     <py><![CDATA[2014]]></py>
##     <spage><![CDATA[1348]]></spage>
##     <epage><![CDATA[1358]]></epage>
##     <abstract><![CDATA[Physical layer (PHY) security is recently regarded as a promising technique to improve the security performance of wireless communication networks. Current developments in PHY security are often based on the assumption of perfect channel state information (CSI). In this paper, both security and reliability performance for the downlink cloud radio access network with optimal remote radio heads (RRHs) node selection are investigated in a practical scenario by considering channel estimation (CE) errors. In particular, a three-phase transmission scheme is proposed and the linear minimum mean-square error (MMSE) estimation method is utilized to obtain the CSI. Based on the CSI estimates and the statistics of CE errors, the outage probability and intercept probability are derived in closed-form expression to evaluate the security and reliability performance, respectively. In addition, two possible cases (with or without intercepting signals from baseband unit) are considered for the eavesdropper. It is found that the suggested optimal RRHs selection scheme outperforms the nonselection scheme, and that the increasing number of RRHs can lower the outage probability as well as the intercept probability. It is also shown that there exists an optimal training number to minimize the sum of the outage probability and intercept probability. Finally, simulation results are provided to corroborate our proposed studies.]]></abstract>
##     <issn><![CDATA[2169-3536]]></issn>
##     <htmlFlag><![CDATA[1]]></htmlFlag>
##     <arnumber><![CDATA[6955788]]></arnumber>
##     <doi><![CDATA[10.1109/ACCESS.2014.2370391]]></doi>
##     <publicationId><![CDATA[6955788]]></publicationId>
##     <mdurl><![CDATA[http://ieeexplore.ieee.org/xpl/articleDetails.jsp?tp=&arnumber=6955788&contentType=Journals+%26+Magazines]]></mdurl>
##     <pdf><![CDATA[http://ieeexplore.ieee.org/stamp/stamp.jsp?arnumber=6955788]]></pdf>
##   </document>
##   <document>
##     <rank>787</rank>
##     <title><![CDATA[Test and Analysis of the Optical Kerr-Effect in Resonant Micro-Optic Gyros]]></title>
##     <authors><![CDATA[Xuehui Li;  Jianjie Zhang;  Huilian Ma;  Zhonghe Jin]]></authors>
##     <affiliations><![CDATA[Micro-Satellite Res. Center, Zhejiang Univ., Hangzhou, China]]></affiliations>
##     <controlledterms>
##       <term><![CDATA[acousto-optical modulation]]></term>
##       <term><![CDATA[gyroscopes]]></term>
##       <term><![CDATA[micro-optics]]></term>
##       <term><![CDATA[optical Kerr effect]]></term>
##       <term><![CDATA[optical testing]]></term>
##     </controlledterms>
##     <thesaurusterms>
##       <term><![CDATA[Adaptive optics]]></term>
##       <term><![CDATA[Nonlinear optics]]></term>
##       <term><![CDATA[Optical fibers]]></term>
##       <term><![CDATA[Optical polarization]]></term>
##       <term><![CDATA[Optical ring resonators]]></term>
##     </thesaurusterms>
##     <pubtitle><![CDATA[Photonics Journal, IEEE]]></pubtitle>
##     <punumber><![CDATA[4563994]]></punumber>
##     <pubtype><![CDATA[Journals & Magazines]]></pubtype>
##     <publisher><![CDATA[IEEE]]></publisher>
##     <volume><![CDATA[6]]></volume>
##     <issue><![CDATA[5]]></issue>
##     <py><![CDATA[2014]]></py>
##     <spage><![CDATA[1]]></spage>
##     <epage><![CDATA[7]]></epage>
##     <abstract><![CDATA[As one kind of the main optical error sources in the resonant micro-optic gyro (RMOG), the optical Kerr-effect brings a nonzero bias to the output of the RMOG. The optical Kerr-effect induced bias error is proportional to the difference between the clockwise (CW) and counterclockwise (CCW) light-intensities input to the resonator. A new method for testing the optical Kerr-effect induced bias error in the RMOG is proposed and demonstrated. A square-wave intensity-modulated optical signal with a symmetrical duty ratio of 50% is achieved by an acousto-optic modulator (AOM), thus making the input-intensity mismatch between the CW and CCW lightwaves change periodically. In this way, we obtain the optical Kerr-effect induced bias errors that vary periodically in accordance with the variation of the square-wave signal. The experimental results show a good agreement with the theoretical value. Moreover, the testing result of the closed-loop RMOG is consistent with the open-loop one. The above method is insusceptible to other noise, such as the backscattering-induced noise and the polarization fluctuation.]]></abstract>
##     <issn><![CDATA[1943-0655]]></issn>
##     <htmlFlag><![CDATA[1]]></htmlFlag>
##     <arnumber><![CDATA[6887329]]></arnumber>
##     <doi><![CDATA[10.1109/JPHOT.2014.2352635]]></doi>
##     <publicationId><![CDATA[6887329]]></publicationId>
##     <mdurl><![CDATA[http://ieeexplore.ieee.org/xpl/articleDetails.jsp?tp=&arnumber=6887329&contentType=Journals+%26+Magazines]]></mdurl>
##     <pdf><![CDATA[http://ieeexplore.ieee.org/stamp/stamp.jsp?arnumber=6887329]]></pdf>
##   </document>
##   <document>
##     <rank>788</rank>
##     <title><![CDATA[Real-Time Monte Carlo Tree Search in Ms Pac-Man]]></title>
##     <authors><![CDATA[Pepels, T.;  Winands, M.H.M.;  Lanctot, M.]]></authors>
##     <affiliations><![CDATA[Dept. of Knowledge Eng., Maastricht Univ., Maastricht, Netherlands]]></affiliations>
##     <controlledterms>
##       <term><![CDATA[Monte Carlo methods]]></term>
##       <term><![CDATA[computer games]]></term>
##       <term><![CDATA[digital simulation]]></term>
##       <term><![CDATA[tree searching]]></term>
##     </controlledterms>
##     <thesaurusterms>
##       <term><![CDATA[Artificial intelligence]]></term>
##       <term><![CDATA[Computational intelligence]]></term>
##       <term><![CDATA[Games]]></term>
##       <term><![CDATA[Junctions]]></term>
##       <term><![CDATA[Monte Carlo methods]]></term>
##       <term><![CDATA[Planning]]></term>
##       <term><![CDATA[Real-time systems]]></term>
##     </thesaurusterms>
##     <pubtitle><![CDATA[Computational Intelligence and AI in Games, IEEE Transactions on]]></pubtitle>
##     <punumber><![CDATA[4804728]]></punumber>
##     <pubtype><![CDATA[Journals & Magazines]]></pubtype>
##     <publisher><![CDATA[IEEE]]></publisher>
##     <volume><![CDATA[6]]></volume>
##     <issue><![CDATA[3]]></issue>
##     <py><![CDATA[2014]]></py>
##     <spage><![CDATA[245]]></spage>
##     <epage><![CDATA[257]]></epage>
##     <abstract><![CDATA[In this paper, Monte Carlo tree search (MCTS) is introduced for controlling the Pac-Man character in the real-time game Ms Pac-Man. MCTS is used to find an optimal path for an agent at each turn, determining the move to make based on the results of numerous randomized simulations. Several enhancements are introduced in order to adapt MCTS to the real-time domain. Ms Pac-Man is an arcade game, in which the protagonist has several goals but no conclusive terminal state. Unlike games such as Chess or Go there is no state in which the player wins the game. Instead, the game has two subgoals, 1) surviving and 2) scoring as many points as possible. Decisions must be made in a strict time constraint of 40 ms. The Pac-Man agent has to compete with a range of different ghost teams, hence limited assumptions can be made about their behavior. In order to expand the capabilities of existing MCTS agents, four enhancements are discussed: 1) a variable-depth tree; 2) simulation strategies for the ghost team and Pac-Man; 3) including long-term goals in scoring; and 4) reusing the search tree for several moves with a decay factor &#x03B3;. The agent described in this paper was entered in both the 2012 World Congress on Computational Intelligence (WCCI'12, Brisbane, Qld., Australia) and the 2012 IEEE Conference on Computational Intelligence and Games (CIG'12, Granada, Spain) Pac-Man Versus Ghost Team competitions, where it achieved second and first places, respectively. In the experiments, we show that using MCTS is a viable technique for the Pac-Man agent. Moreover, the enhancements improve overall performance against four different ghost teams.]]></abstract>
##     <issn><![CDATA[1943-068X]]></issn>
##     <htmlFlag><![CDATA[1]]></htmlFlag>
##     <arnumber><![CDATA[6731713]]></arnumber>
##     <doi><![CDATA[10.1109/TCIAIG.2013.2291577]]></doi>
##     <publicationId><![CDATA[6731713]]></publicationId>
##     <mdurl><![CDATA[http://ieeexplore.ieee.org/xpl/articleDetails.jsp?tp=&arnumber=6731713&contentType=Journals+%26+Magazines]]></mdurl>
##     <pdf><![CDATA[http://ieeexplore.ieee.org/stamp/stamp.jsp?arnumber=6731713]]></pdf>
##   </document>
##   <document>
##     <rank>789</rank>
##     <title><![CDATA[Loss Estimation Method for Three-Phase AC Reactors of Two Types of Structures Using Amorphous Wound Cores in 400-kVA UPS]]></title>
##     <authors><![CDATA[Kurita, N.;  Onda, K.;  Nakanoue, K.;  Inagaki, K.]]></authors>
##     <affiliations><![CDATA[Hitachi Res. Lab., Hitachi Ltd., Hitachi, Japan]]></affiliations>
##     <controlledterms>
##       <term><![CDATA[cores]]></term>
##       <term><![CDATA[finite element analysis]]></term>
##       <term><![CDATA[reactors (electric)]]></term>
##       <term><![CDATA[uninterruptible power supplies]]></term>
##     </controlledterms>
##     <thesaurusterms>
##       <term><![CDATA[Amorphous magnetic materials]]></term>
##       <term><![CDATA[Inductors]]></term>
##       <term><![CDATA[Loss measurement]]></term>
##       <term><![CDATA[Magnetic cores]]></term>
##       <term><![CDATA[Magnetic flux]]></term>
##       <term><![CDATA[Saturation magnetization]]></term>
##       <term><![CDATA[Toroidal magnetic fields]]></term>
##     </thesaurusterms>
##     <pubtitle><![CDATA[Power Electronics, IEEE Transactions on]]></pubtitle>
##     <punumber><![CDATA[63]]></punumber>
##     <pubtype><![CDATA[Journals & Magazines]]></pubtype>
##     <publisher><![CDATA[IEEE]]></publisher>
##     <volume><![CDATA[29]]></volume>
##     <issue><![CDATA[7]]></issue>
##     <py><![CDATA[2014]]></py>
##     <spage><![CDATA[3657]]></spage>
##     <epage><![CDATA[3668]]></epage>
##     <abstract><![CDATA[We have developed a loss estimation method that is applicable to three-phase ac reactors using amorphous cores in a 400-kVA UPS. The method is based on measured the B-H curves and iron losses of the cores, modeled with magnetic simulations of a finite-element method considering the directions of the amorphous ribbon planes. The reactors are formed from wound cores. Two types of magnetic legs are proposed: structure A, toroidal-shaped with slits, and structure B, sector prism-shaped cut from a wound core. Their unit volumes are shrunk by 9% and 19%, respectively, compared with that of a conventional silicon-steel core reactor. Both reactors have about half the total losses of the conventional silicon-steel core reactor, resulting in increased UPS efficiencies of up to 0.55%. The calculated iron losses at the pulse width modulation (PWM) carrier frequencies agree with measured losses within 10%. The accuracy of the loss estimation method for PWM-induced reactors is discussed. The method is confirmed to enable accurate design of a reactor for filtering in a high-efficiency inverter system.]]></abstract>
##     <issn><![CDATA[0885-8993]]></issn>
##     <htmlFlag><![CDATA[1]]></htmlFlag>
##     <arnumber><![CDATA[6583993]]></arnumber>
##     <doi><![CDATA[10.1109/TPEL.2013.2279256]]></doi>
##     <publicationId><![CDATA[6583993]]></publicationId>
##     <mdurl><![CDATA[http://ieeexplore.ieee.org/xpl/articleDetails.jsp?tp=&arnumber=6583993&contentType=Journals+%26+Magazines]]></mdurl>
##     <pdf><![CDATA[http://ieeexplore.ieee.org/stamp/stamp.jsp?arnumber=6583993]]></pdf>
##   </document>
##   <document>
##     <rank>790</rank>
##     <title><![CDATA[Enhanced Secure Strategy for OFDM-PON System by Using Hyperchaotic System and Fractional Fourier Transformation]]></title>
##     <authors><![CDATA[Cheng, M.;  Deng, L.;  Wang, X.;  Li, H.;  Tang, M.;  Ke, C.;  Shum, P.;  Liu, D.]]></authors>
##     <affiliations><![CDATA[Next Generation Internet Access Nat. Eng. Lab. (NGIA), Huazhong Univ. of Sci. & Technol., Wuhan, China]]></affiliations>
##     <controlledterms>
##       <term><![CDATA[Fourier transform optics]]></term>
##       <term><![CDATA[OFDM modulation]]></term>
##       <term><![CDATA[chaos]]></term>
##       <term><![CDATA[passive optical networks]]></term>
##       <term><![CDATA[quadrature amplitude modulation]]></term>
##       <term><![CDATA[security]]></term>
##     </controlledterms>
##     <thesaurusterms>
##       <term><![CDATA[Chaotic communication]]></term>
##       <term><![CDATA[Encryption]]></term>
##       <term><![CDATA[OFDM]]></term>
##       <term><![CDATA[Passive optical networks]]></term>
##     </thesaurusterms>
##     <pubtitle><![CDATA[Photonics Journal, IEEE]]></pubtitle>
##     <punumber><![CDATA[4563994]]></punumber>
##     <pubtype><![CDATA[Journals & Magazines]]></pubtype>
##     <publisher><![CDATA[IEEE]]></publisher>
##     <volume><![CDATA[6]]></volume>
##     <issue><![CDATA[6]]></issue>
##     <py><![CDATA[2014]]></py>
##     <spage><![CDATA[1]]></spage>
##     <epage><![CDATA[9]]></epage>
##     <abstract><![CDATA[We propose and experimentally demonstrate a scheme whereby hyperchaos and fractional Fourier transform (FrFT) techniques are integrated in an orthogonal frequency-division multiplexing (OFDM) passive optical network system. In our experiment, both security issues and transmission performance are investigated under an overall frame, and 7.64-Gb/s 16-quadrature-amplitude-modulation OFDM data with a four-level encryption scheme are successfully transmitted over a 25-km standard single-mode fiber. The results show that the system security and the transmission performance can be improved simultaneously. Moreover, the proposed scheme allows a flexible adjustment between the safety and the transmission performance according to the actual requirements.]]></abstract>
##     <issn><![CDATA[1943-0655]]></issn>
##     <htmlFlag><![CDATA[1]]></htmlFlag>
##     <arnumber><![CDATA[6923426]]></arnumber>
##     <doi><![CDATA[10.1109/JPHOT.2014.2363427]]></doi>
##     <publicationId><![CDATA[6923426]]></publicationId>
##     <mdurl><![CDATA[http://ieeexplore.ieee.org/xpl/articleDetails.jsp?tp=&arnumber=6923426&contentType=Journals+%26+Magazines]]></mdurl>
##     <pdf><![CDATA[http://ieeexplore.ieee.org/stamp/stamp.jsp?arnumber=6923426]]></pdf>
##   </document>
##   <document>
##     <rank>791</rank>
##     <title><![CDATA[CHOKeR: A Novel AQM Algorithm With Proportional Bandwidth Allocation and TCP Protection]]></title>
##     <authors><![CDATA[Lingyun Lu;  Haifeng Du;  Ren Ping Liu]]></authors>
##     <affiliations><![CDATA[Sch. of Comput. & Inf. Technol., Beijing JiaoTong Univ., Beijing, China]]></affiliations>
##     <controlledterms>
##       <term><![CDATA[DiffServ networks]]></term>
##       <term><![CDATA[bandwidth allocation]]></term>
##       <term><![CDATA[quality of service]]></term>
##       <term><![CDATA[queueing theory]]></term>
##       <term><![CDATA[telecommunication congestion control]]></term>
##       <term><![CDATA[telecommunication network management]]></term>
##       <term><![CDATA[transport protocols]]></term>
##     </controlledterms>
##     <thesaurusterms>
##       <term><![CDATA[Aggregates]]></term>
##       <term><![CDATA[Analytical models]]></term>
##       <term><![CDATA[Bandwidth]]></term>
##       <term><![CDATA[Channel allocation]]></term>
##       <term><![CDATA[Complexity theory]]></term>
##       <term><![CDATA[Diffserv networks]]></term>
##       <term><![CDATA[Inductors]]></term>
##     </thesaurusterms>
##     <pubtitle><![CDATA[Industrial Informatics, IEEE Transactions on]]></pubtitle>
##     <punumber><![CDATA[9424]]></punumber>
##     <pubtype><![CDATA[Journals & Magazines]]></pubtype>
##     <publisher><![CDATA[IEEE]]></publisher>
##     <volume><![CDATA[10]]></volume>
##     <issue><![CDATA[1]]></issue>
##     <py><![CDATA[2014]]></py>
##     <spage><![CDATA[637]]></spage>
##     <epage><![CDATA[644]]></epage>
##     <abstract><![CDATA[Although differentiated services (DiffServ) networks have been well discussed in the past several years, a conventional Active Queue Management (AQM) algorithm still cannot provide low-complexity and cost-effective differentiated bandwidth allocation in DiffServ. In this paper, a novel AQM scheme called CHOKeR is designed to protect TCP flows effectively. We adopt a method from CHOKeW to draw multiple packets randomly from the output buffer. CHOKeR enhances the drawing factor by using a multistep increase and single-step decrease (MISD) mechanism. In order to explain the features of CHOKeR, an analytical model is used, followed by extensive simulations to evaluate the performance of CHOKeR. The analytical model and simulation results demonstrate that CHOKeR achieves proportional bandwidth allocation between different priority levels, fairness guarantee among equal priority flows, and protection of TCP against high-speed unresponsive flows when network congestion occurs.]]></abstract>
##     <issn><![CDATA[1551-3203]]></issn>
##     <htmlFlag><![CDATA[1]]></htmlFlag>
##     <arnumber><![CDATA[6579777]]></arnumber>
##     <doi><![CDATA[10.1109/TII.2013.2278618]]></doi>
##     <publicationId><![CDATA[6579777]]></publicationId>
##     <mdurl><![CDATA[http://ieeexplore.ieee.org/xpl/articleDetails.jsp?tp=&arnumber=6579777&contentType=Journals+%26+Magazines]]></mdurl>
##     <pdf><![CDATA[http://ieeexplore.ieee.org/stamp/stamp.jsp?arnumber=6579777]]></pdf>
##   </document>
##   <document>
##     <rank>792</rank>
##     <title><![CDATA[Multivariable Static Ankle Mechanical Impedance With Active Muscles]]></title>
##     <authors><![CDATA[Hyunglae Lee;  Ho, P.;  Rastgaar, M.;  Krebs, H.I.;  Hogan, N.]]></authors>
##     <affiliations><![CDATA[Mech. Eng. Dept., Massachusetts Inst. of Technol., Cambridge, MA, USA]]></affiliations>
##     <controlledterms>
##       <term><![CDATA[biomechanics]]></term>
##       <term><![CDATA[elastic constants]]></term>
##       <term><![CDATA[electric impedance]]></term>
##       <term><![CDATA[electromyography]]></term>
##       <term><![CDATA[medical disorders]]></term>
##       <term><![CDATA[medical robotics]]></term>
##       <term><![CDATA[medical signal processing]]></term>
##       <term><![CDATA[multivariable control systems]]></term>
##       <term><![CDATA[neurophysiology]]></term>
##       <term><![CDATA[torque]]></term>
##     </controlledterms>
##     <thesaurusterms>
##       <term><![CDATA[Electromyography]]></term>
##       <term><![CDATA[Impedance]]></term>
##       <term><![CDATA[Muscles]]></term>
##       <term><![CDATA[Robots]]></term>
##       <term><![CDATA[Torque]]></term>
##       <term><![CDATA[Torque measurement]]></term>
##       <term><![CDATA[Vectors]]></term>
##     </thesaurusterms>
##     <pubtitle><![CDATA[Neural Systems and Rehabilitation Engineering, IEEE Transactions on]]></pubtitle>
##     <punumber><![CDATA[7333]]></punumber>
##     <pubtype><![CDATA[Journals & Magazines]]></pubtype>
##     <publisher><![CDATA[IEEE]]></publisher>
##     <volume><![CDATA[22]]></volume>
##     <issue><![CDATA[1]]></issue>
##     <py><![CDATA[2014]]></py>
##     <spage><![CDATA[44]]></spage>
##     <epage><![CDATA[52]]></epage>
##     <abstract><![CDATA[This paper reports quantification of multivariable static ankle mechanical impedance when muscles were active. Repetitive measurements using a highly backdrivable therapeutic robot combined with robust function approximation methods enabled reliable characterization of the nonlinear torque-angle relation at the ankle in two coupled degrees of freedom simultaneously, a combination of dorsiflexion-plantarflexion and inversion-eversion, and how it varied with muscle activation. Measurements on 10 young healthy seated subjects quantified the behavior of the human ankle when muscles were active at 10% of maximum voluntary contraction. Stiffness, a linear approximation to static ankle mechanical impedance, was estimated from the continuous vector field. As with previous measurements when muscles were maximally relaxed, we found that ankle stiffness was highly direction-dependent, being weakest in inversion/eversion. Predominantly activating a single muscle or co-contracting antagonistic muscles significantly increased ankle stiffness in all directions but it increased more in the sagittal plane than in the frontal plane, accentuating the relative weakness of the ankle in the inversion-eversion direction. Remarkably, the observed increase was not consistent with simple superposition of muscle-generated stiffness, which may be due to the contribution of unmonitored deep ankle muscles. Implications for the assessment of neuro-mechanical disorders are discussed.]]></abstract>
##     <issn><![CDATA[1534-4320]]></issn>
##     <htmlFlag><![CDATA[1]]></htmlFlag>
##     <arnumber><![CDATA[6605642]]></arnumber>
##     <doi><![CDATA[10.1109/TNSRE.2013.2262689]]></doi>
##     <publicationId><![CDATA[6605642]]></publicationId>
##     <mdurl><![CDATA[http://ieeexplore.ieee.org/xpl/articleDetails.jsp?tp=&arnumber=6605642&contentType=Journals+%26+Magazines]]></mdurl>
##     <pdf><![CDATA[http://ieeexplore.ieee.org/stamp/stamp.jsp?arnumber=6605642]]></pdf>
##   </document>
##   <document>
##     <rank>793</rank>
##     <title><![CDATA[Transthoracic ultrafast Doppler imaging of human left ventricular hemodynamic function]]></title>
##     <authors><![CDATA[Osmanski, B.-F.;  Maresca, D.;  Messas, E.;  Tanter, M.;  Pernot, M.]]></authors>
##     <affiliations><![CDATA[Inst. Langevin, ESPCI ParisTech, Paris, France]]></affiliations>
##     <controlledterms>
##       <term><![CDATA[Doppler measurement]]></term>
##       <term><![CDATA[blood flow measurement]]></term>
##       <term><![CDATA[blood vessels]]></term>
##       <term><![CDATA[cardiovascular system]]></term>
##       <term><![CDATA[diseases]]></term>
##       <term><![CDATA[echocardiography]]></term>
##       <term><![CDATA[image resolution]]></term>
##       <term><![CDATA[image sampling]]></term>
##       <term><![CDATA[medical image processing]]></term>
##       <term><![CDATA[pattern formation]]></term>
##     </controlledterms>
##     <thesaurusterms>
##       <term><![CDATA[Blood]]></term>
##       <term><![CDATA[Heart]]></term>
##       <term><![CDATA[Image color analysis]]></term>
##       <term><![CDATA[Imaging]]></term>
##       <term><![CDATA[Probes]]></term>
##       <term><![CDATA[Ultrasonic imaging]]></term>
##       <term><![CDATA[Valves]]></term>
##     </thesaurusterms>
##     <pubtitle><![CDATA[Ultrasonics, Ferroelectrics, and Frequency Control, IEEE Transactions on]]></pubtitle>
##     <punumber><![CDATA[58]]></punumber>
##     <pubtype><![CDATA[Journals & Magazines]]></pubtype>
##     <publisher><![CDATA[IEEE]]></publisher>
##     <volume><![CDATA[61]]></volume>
##     <issue><![CDATA[8]]></issue>
##     <py><![CDATA[2014]]></py>
##     <spage><![CDATA[1268]]></spage>
##     <epage><![CDATA[1275]]></epage>
##     <abstract><![CDATA[Heart diseases can affect intraventricular blood flow patterns. Real-time imaging of blood flow patterns is challenging because it requires both a high frame rate and a large field of view. To date, standard Doppler techniques can only perform blood flow estimation with high temporal resolution within small regions of interest. In this work, we used ultrafast imaging to map in 2-D human left ventricular blood flow patterns during the whole cardiac cycle. Cylindrical waves were transmitted at 4800 Hz with a transthoracic phased-array probe to achieve ultrafast Doppler imaging of the left ventricle. The high spatio-temporal sampling of ultrafast imaging permits reliance on a much more effective wall filtering and increased sensitivity when mapping blood flow patterns during the pre-ejection, ejection, early diastole, diastasis, and late diastole phases of the heart cycle. The superior sensitivity and temporal resolution of ultrafast Doppler imaging makes it a promising tool for the noninvasive study of intraventricular hemodynamic function.]]></abstract>
##     <issn><![CDATA[0885-3010]]></issn>
##     <htmlFlag><![CDATA[1]]></htmlFlag>
##     <arnumber><![CDATA[6863847]]></arnumber>
##     <doi><![CDATA[10.1109/TUFFC.2014.3033]]></doi>
##     <publicationId><![CDATA[6863847]]></publicationId>
##     <mdurl><![CDATA[http://ieeexplore.ieee.org/xpl/articleDetails.jsp?tp=&arnumber=6863847&contentType=Journals+%26+Magazines]]></mdurl>
##     <pdf><![CDATA[http://ieeexplore.ieee.org/stamp/stamp.jsp?arnumber=6863847]]></pdf>
##   </document>
##   <document>
##     <rank>794</rank>
##     <title><![CDATA[Fast 2-D ultrasound strain imaging: the benefits of using a GPU]]></title>
##     <authors><![CDATA[Idzenga, T.;  Gaburov, E.;  Vermin, W.;  Menssen, J.;  De Korte, C.]]></authors>
##     <affiliations><![CDATA[Dept. of Radiol., Radboud Univ. Nijmegen Med. Centre, Nijmegen, Netherlands]]></affiliations>
##     <controlledterms>
##       <term><![CDATA[biological tissues]]></term>
##       <term><![CDATA[biomechanics]]></term>
##       <term><![CDATA[biomedical ultrasonics]]></term>
##       <term><![CDATA[deformation]]></term>
##       <term><![CDATA[graphics processing units]]></term>
##       <term><![CDATA[medical image processing]]></term>
##       <term><![CDATA[parallel architectures]]></term>
##       <term><![CDATA[strain measurement]]></term>
##       <term><![CDATA[ultrasonic imaging]]></term>
##     </controlledterms>
##     <thesaurusterms>
##       <term><![CDATA[Correlation]]></term>
##       <term><![CDATA[Graphics processing units]]></term>
##       <term><![CDATA[Imaging]]></term>
##       <term><![CDATA[Instruction sets]]></term>
##       <term><![CDATA[Kernel]]></term>
##       <term><![CDATA[Strain]]></term>
##       <term><![CDATA[Ultrasonic imaging]]></term>
##     </thesaurusterms>
##     <pubtitle><![CDATA[Ultrasonics, Ferroelectrics, and Frequency Control, IEEE Transactions on]]></pubtitle>
##     <punumber><![CDATA[58]]></punumber>
##     <pubtype><![CDATA[Journals & Magazines]]></pubtype>
##     <publisher><![CDATA[IEEE]]></publisher>
##     <volume><![CDATA[61]]></volume>
##     <issue><![CDATA[1]]></issue>
##     <py><![CDATA[2014]]></py>
##     <spage><![CDATA[207]]></spage>
##     <epage><![CDATA[213]]></epage>
##     <abstract><![CDATA[Deformation of tissue can be accurately estimated from radio-frequency ultrasound data using a 2-dimensional normalized cross correlation (NCC)-based algorithm. This procedure, however, is very computationally time-consuming. A major time reduction can be achieved by parallelizing the numerous computations of NCC. In this paper, two approaches for parallelization have been investigated: the OpenMP interface on a multi-CPU system and Compute Unified Device Architecture (CUDA) on a graphics processing unit (GPU). The performance of the OpenMP and GPU approaches were compared with a conventional Matlab implementation of NCC. The OpenMP approach with 8 threads achieved a maximum speed-up factor of 132 on the computing of NCC, whereas the GPU approach on an Nvidia Tesla K20 achieved a maximum speed-up factor of 376. Neither parallelization approach resulted in a significant loss in image quality of the elastograms. Parallelization of the NCC computations using the GPU, therefore, significantly reduces the computation time and increases the frame rate for motion estimation.]]></abstract>
##     <issn><![CDATA[0885-3010]]></issn>
##     <htmlFlag><![CDATA[1]]></htmlFlag>
##     <arnumber><![CDATA[6689790]]></arnumber>
##     <doi><![CDATA[10.1109/TUFFC.2014.2893]]></doi>
##     <publicationId><![CDATA[6689790]]></publicationId>
##     <mdurl><![CDATA[http://ieeexplore.ieee.org/xpl/articleDetails.jsp?tp=&arnumber=6689790&contentType=Journals+%26+Magazines]]></mdurl>
##     <pdf><![CDATA[http://ieeexplore.ieee.org/stamp/stamp.jsp?arnumber=6689790]]></pdf>
##   </document>
##   <document>
##     <rank>795</rank>
##     <title><![CDATA[A CMOS LNA Using a Harmonic Rejection Technique to Enhance Its Linearity]]></title>
##     <authors><![CDATA[Jaehyuk Yoon;  Changkun Park]]></authors>
##     <affiliations><![CDATA[Sch. of Electron. Eng., Soongsil Univ., Seoul, South Korea]]></affiliations>
##     <controlledterms>
##       <term><![CDATA[CMOS analogue integrated circuits]]></term>
##       <term><![CDATA[RC circuits]]></term>
##       <term><![CDATA[differential amplifiers]]></term>
##       <term><![CDATA[feedback amplifiers]]></term>
##       <term><![CDATA[linearisation techniques]]></term>
##       <term><![CDATA[low noise amplifiers]]></term>
##     </controlledterms>
##     <thesaurusterms>
##       <term><![CDATA[CMOS integrated circuits]]></term>
##       <term><![CDATA[CMOS technology]]></term>
##       <term><![CDATA[Feedback loop]]></term>
##       <term><![CDATA[Harmonic analysis]]></term>
##       <term><![CDATA[Linearity]]></term>
##       <term><![CDATA[Noise]]></term>
##       <term><![CDATA[Wireless communication]]></term>
##     </thesaurusterms>
##     <pubtitle><![CDATA[Microwave and Wireless Components Letters, IEEE]]></pubtitle>
##     <punumber><![CDATA[7260]]></punumber>
##     <pubtype><![CDATA[Journals & Magazines]]></pubtype>
##     <publisher><![CDATA[IEEE]]></publisher>
##     <volume><![CDATA[24]]></volume>
##     <issue><![CDATA[9]]></issue>
##     <py><![CDATA[2014]]></py>
##     <spage><![CDATA[605]]></spage>
##     <epage><![CDATA[607]]></epage>
##     <abstract><![CDATA[In this study, we design a differential low-noise amplifier (LNA) using a 0.18- &#x03BC;m RF CMOS process. To improve its linearity, we propose a harmonic rejection technique using RC feedback at the gain stage. The third harmonic component of the drain node of the common-gate transistor is fed back to the source node of the common-gate transistor to restrict the generation of the third harmonic component at the output of the LNA. To verify the feasibility of the proposed technique for a linear amplifier, we designed a typical LNA and the proposed LNA in an identical process and with the same design parameters apart from the feedback loop of the proposed LNA. The measured improvement of the input-referred P1 dB of the proposed LNA is approximately 3 dB compared to that of the typical LNA. From these measured results, we successfully prove the feasibility of the proposed linearization technique.]]></abstract>
##     <issn><![CDATA[1531-1309]]></issn>
##     <htmlFlag><![CDATA[1]]></htmlFlag>
##     <arnumber><![CDATA[6842675]]></arnumber>
##     <doi><![CDATA[10.1109/LMWC.2014.2326518]]></doi>
##     <publicationId><![CDATA[6842675]]></publicationId>
##     <mdurl><![CDATA[http://ieeexplore.ieee.org/xpl/articleDetails.jsp?tp=&arnumber=6842675&contentType=Journals+%26+Magazines]]></mdurl>
##     <pdf><![CDATA[http://ieeexplore.ieee.org/stamp/stamp.jsp?arnumber=6842675]]></pdf>
##   </document>
##   <document>
##     <rank>796</rank>
##     <title><![CDATA[Breakthroughs in Photonics 2013: Holographic Imaging]]></title>
##     <authors><![CDATA[Memmolo, P.;  Bianco, V.;  Merola, F.;  Miccio, L.;  Paturzo, M.;  Ferraro, P.]]></authors>
##     <affiliations><![CDATA[Ist. Naz. di Ottica, Pozzuoli, Italy]]></affiliations>
##     <controlledterms>
##       <term><![CDATA[holography]]></term>
##     </controlledterms>
##     <thesaurusterms>
##       <term><![CDATA[DH-HEMTs]]></term>
##       <term><![CDATA[Holography]]></term>
##       <term><![CDATA[Image reconstruction]]></term>
##       <term><![CDATA[Image resolution]]></term>
##       <term><![CDATA[Microscopy]]></term>
##       <term><![CDATA[Three-dimensional displays]]></term>
##     </thesaurusterms>
##     <pubtitle><![CDATA[Photonics Journal, IEEE]]></pubtitle>
##     <punumber><![CDATA[4563994]]></punumber>
##     <pubtype><![CDATA[Journals & Magazines]]></pubtype>
##     <publisher><![CDATA[IEEE]]></publisher>
##     <volume><![CDATA[6]]></volume>
##     <issue><![CDATA[2]]></issue>
##     <py><![CDATA[2014]]></py>
##     <spage><![CDATA[1]]></spage>
##     <epage><![CDATA[6]]></epage>
##     <abstract><![CDATA[Although holography is topic that goes back to the 1950s, the research in this field continues to be very active worldwide. A continuous growth is confirmed by the publication of more than 2000 papers each year in archival journal on different holographic issues. Here we describe shortly what appeared to us to be the most significant achievements reached in 2013 on holographic imaging.]]></abstract>
##     <issn><![CDATA[1943-0655]]></issn>
##     <htmlFlag><![CDATA[1]]></htmlFlag>
##     <arnumber><![CDATA[6766234]]></arnumber>
##     <doi><![CDATA[10.1109/JPHOT.2014.2311448]]></doi>
##     <publicationId><![CDATA[6766234]]></publicationId>
##     <mdurl><![CDATA[http://ieeexplore.ieee.org/xpl/articleDetails.jsp?tp=&arnumber=6766234&contentType=Journals+%26+Magazines]]></mdurl>
##     <pdf><![CDATA[http://ieeexplore.ieee.org/stamp/stamp.jsp?arnumber=6766234]]></pdf>
##   </document>
##   <document>
##     <rank>797</rank>
##     <title><![CDATA[Scattering of near normal incidence SH waves by sinusoidal and rough surfaces in 3-D: Comparison to the scalar wave approximation]]></title>
##     <authors><![CDATA[Jarvis, A.J.C.;  Cegla, F.B.]]></authors>
##     <affiliations><![CDATA[Dept. of Mech. Eng., Imperial Coll. London, London, UK]]></affiliations>
##     <controlledterms>
##       <term><![CDATA[Gaussian distribution]]></term>
##       <term><![CDATA[acoustic transducer arrays]]></term>
##       <term><![CDATA[acoustic wave scattering]]></term>
##       <term><![CDATA[finite element analysis]]></term>
##       <term><![CDATA[rough surfaces]]></term>
##       <term><![CDATA[surface scattering]]></term>
##     </controlledterms>
##     <thesaurusterms>
##       <term><![CDATA[Acoustics]]></term>
##       <term><![CDATA[Approximation methods]]></term>
##       <term><![CDATA[Numerical models]]></term>
##       <term><![CDATA[Rough surfaces]]></term>
##       <term><![CDATA[Scattering]]></term>
##       <term><![CDATA[Surface roughness]]></term>
##       <term><![CDATA[Surface waves]]></term>
##     </thesaurusterms>
##     <pubtitle><![CDATA[Ultrasonics, Ferroelectrics, and Frequency Control, IEEE Transactions on]]></pubtitle>
##     <punumber><![CDATA[58]]></punumber>
##     <pubtype><![CDATA[Journals & Magazines]]></pubtype>
##     <publisher><![CDATA[IEEE]]></publisher>
##     <volume><![CDATA[61]]></volume>
##     <issue><![CDATA[7]]></issue>
##     <py><![CDATA[2014]]></py>
##     <spage><![CDATA[1179]]></spage>
##     <epage><![CDATA[1190]]></epage>
##     <abstract><![CDATA[The challenge of accurately simulating how incident scalar waves interact with rough boundaries has made it an important area of research within many scientific disciplines. Conventional methods, which in the majority of cases focus only on scattering in two dimensions, often suffer from long simulation times or reduced accuracy, neglecting phenomena such as multiple scattering and surface self-shadowing. A simulation based on the scalar wave distributed point source method (DPSM) is presented as an alternative which is computationally more efficient than fully meshed numerical methods while obtaining greater accuracy than approximate analytical techniques. Comparison is made to simulated results obtained using the finite element method for a sinusoidally periodic surface where scattering only occurs in two dimensions, showing very good agreement (&lt;;0.2 dB). In addition to twodimensional scattering, comparison to experimental results is also carried out for scattering in three dimensions when the surface has a Gaussian roughness distribution. Results indicate that for two-dimensional scattering and for rough surfaces with a correlation length equal to the incident wavelength (&#x03BB;) and a root mean square height less than 0.2&#x03BB;, the scalar wave approximation predicts reflected pulse shape change and envelope amplitudes generally to within 1 dB. Comparison between transducers within a three-element array also illustrate the sensitivity pulse amplitude can have to sensor position above a rough surface, differing by as much as 17 dB with a positional change of just 1.25&#x03BB;.]]></abstract>
##     <issn><![CDATA[0885-3010]]></issn>
##     <htmlFlag><![CDATA[1]]></htmlFlag>
##     <arnumber><![CDATA[6838813]]></arnumber>
##     <doi><![CDATA[10.1109/TUFFC.2014.3017]]></doi>
##     <publicationId><![CDATA[6838813]]></publicationId>
##     <mdurl><![CDATA[http://ieeexplore.ieee.org/xpl/articleDetails.jsp?tp=&arnumber=6838813&contentType=Journals+%26+Magazines]]></mdurl>
##     <pdf><![CDATA[http://ieeexplore.ieee.org/stamp/stamp.jsp?arnumber=6838813]]></pdf>
##   </document>
##   <document>
##     <rank>798</rank>
##     <title><![CDATA[Sensitivity of Microwave Imaging Systems Employing Scattering-Parameter Measurements]]></title>
##     <authors><![CDATA[Moussakhani, K.;  McCombe, J.J.;  Nikolova, N.K.]]></authors>
##     <affiliations><![CDATA[Dept. Electr. Eng., McMaster Univ., Hamilton, ON, Canada]]></affiliations>
##     <controlledterms>
##       <term><![CDATA[S-parameters]]></term>
##       <term><![CDATA[biomedical imaging]]></term>
##       <term><![CDATA[electric variables measurement]]></term>
##       <term><![CDATA[microwave antennas]]></term>
##       <term><![CDATA[microwave detectors]]></term>
##       <term><![CDATA[microwave imaging]]></term>
##     </controlledterms>
##     <thesaurusterms>
##       <term><![CDATA[Antenna measurements]]></term>
##       <term><![CDATA[Antennas]]></term>
##       <term><![CDATA[Microwave imaging]]></term>
##       <term><![CDATA[Microwave measurement]]></term>
##       <term><![CDATA[Permittivity]]></term>
##       <term><![CDATA[Sensitivity]]></term>
##     </thesaurusterms>
##     <pubtitle><![CDATA[Microwave Theory and Techniques, IEEE Transactions on]]></pubtitle>
##     <punumber><![CDATA[22]]></punumber>
##     <pubtype><![CDATA[Journals & Magazines]]></pubtype>
##     <publisher><![CDATA[IEEE]]></publisher>
##     <volume><![CDATA[62]]></volume>
##     <issue><![CDATA[10]]></issue>
##     <py><![CDATA[2014]]></py>
##     <spage><![CDATA[2447]]></spage>
##     <epage><![CDATA[2455]]></epage>
##     <abstract><![CDATA[The physical contrast sensitivity of microwave imaging systems employing scattering-parameter measurements is defined. Methodologies are proposed for its evaluation through measurements and through simulations. This enables the estimation of the smallest detectable target permittivity contrast or size for the system under evaluation. The outcomes of the proposed simulation-based and measurement-based methods are compared for the case of a realistic tissue-imaging system. The agreement between the simulated and measured sensitivity estimation validates the proposed methods. The intention of the proposed methodology is to provide common means to quantify and compare the sensitivity performance of microwave systems used in tissue imaging as well as the antennas used as sensors. We emphasize that the proposed method targets the performance of the hardware and it is not concerned with the image-reconstruction algorithms.]]></abstract>
##     <issn><![CDATA[0018-9480]]></issn>
##     <htmlFlag><![CDATA[1]]></htmlFlag>
##     <arnumber><![CDATA[6889125]]></arnumber>
##     <doi><![CDATA[10.1109/TMTT.2014.2350961]]></doi>
##     <publicationId><![CDATA[6889125]]></publicationId>
##     <mdurl><![CDATA[http://ieeexplore.ieee.org/xpl/articleDetails.jsp?tp=&arnumber=6889125&contentType=Journals+%26+Magazines]]></mdurl>
##     <pdf><![CDATA[http://ieeexplore.ieee.org/stamp/stamp.jsp?arnumber=6889125]]></pdf>
##   </document>
##   <document>
##     <rank>799</rank>
##     <title><![CDATA[Exploring Coordinated Multipoint Beamforming Strategies for 5G Cellular]]></title>
##     <authors><![CDATA[Schwarz, S.;  Rupp, M.]]></authors>
##     <affiliations><![CDATA[Inst. of Telecommun., Vienna Univ. of Technol., Vienna, Austria]]></affiliations>
##     <controlledterms>
##       <term><![CDATA[Internet]]></term>
##       <term><![CDATA[cellular radio]]></term>
##       <term><![CDATA[mobile computing]]></term>
##       <term><![CDATA[radiofrequency interference]]></term>
##     </controlledterms>
##     <thesaurusterms>
##       <term><![CDATA[Broadband communication]]></term>
##       <term><![CDATA[Cellular networks]]></term>
##       <term><![CDATA[Interference]]></term>
##       <term><![CDATA[Internet]]></term>
##       <term><![CDATA[Long Term Evolution]]></term>
##       <term><![CDATA[Mobile communication]]></term>
##       <term><![CDATA[Mobile computing]]></term>
##       <term><![CDATA[Network architecture]]></term>
##       <term><![CDATA[Transmitters]]></term>
##     </thesaurusterms>
##     <pubtitle><![CDATA[Access, IEEE]]></pubtitle>
##     <punumber><![CDATA[6287639]]></punumber>
##     <pubtype><![CDATA[Journals & Magazines]]></pubtype>
##     <publisher><![CDATA[IEEE]]></publisher>
##     <volume><![CDATA[2]]></volume>
##     <py><![CDATA[2014]]></py>
##     <spage><![CDATA[930]]></spage>
##     <epage><![CDATA[946]]></epage>
##     <abstract><![CDATA[Cellular networks are a central part of today's communication infrastructure. The global roll-out of 4G long-term evolution is underway, ideally enabling ubiquitous broadband Internet access. Mobile network operators, however, are currently facing an exponentially increasing demand for network capacity, necessitating densification of cellular base stations (keywords: small cells and heterogeneous networks) and causing a strongly deteriorated interference environment. Coordination among transmitters and receivers to mitigate and/or exploit interference is hence seen as a main path toward 5G mobile networks. We provide an overview of existing coordinated beamforming strategies for interference mitigation in broadcast and interference channels. To gain insight into their ergodic behavior in terms of signal to interference and noise ratio as well as achievable transmission rate, we focus on a simplified but representative scenario with two transmitters that serve two users. This analysis provides guidelines for selecting the best performing method depending on the particular transmission situation.]]></abstract>
##     <issn><![CDATA[2169-3536]]></issn>
##     <htmlFlag><![CDATA[1]]></htmlFlag>
##     <arnumber><![CDATA[6888496]]></arnumber>
##     <doi><![CDATA[10.1109/ACCESS.2014.2353137]]></doi>
##     <publicationId><![CDATA[6888496]]></publicationId>
##     <mdurl><![CDATA[http://ieeexplore.ieee.org/xpl/articleDetails.jsp?tp=&arnumber=6888496&contentType=Journals+%26+Magazines]]></mdurl>
##     <pdf><![CDATA[http://ieeexplore.ieee.org/stamp/stamp.jsp?arnumber=6888496]]></pdf>
##   </document>
##   <document>
##     <rank>800</rank>
##     <title><![CDATA[Quantitative Assessment of Single-Image Super-Resolution in Myocardial Scar Imaging]]></title>
##     <authors><![CDATA[Ashikaga, H.;  Estner, H.L.;  Herzka, D.A.;  Mcveigh, E.R.;  Halperin, H.R.]]></authors>
##     <affiliations><![CDATA[Division of Cardiology, The Johns Hopkins University School of Medicine, Baltimore, MD, USA]]></affiliations>
##     <thesaurusterms>
##       <term><![CDATA[Dictionaries]]></term>
##       <term><![CDATA[Heart]]></term>
##       <term><![CDATA[Interpolation]]></term>
##       <term><![CDATA[Myocardium]]></term>
##       <term><![CDATA[Spatial resolution]]></term>
##       <term><![CDATA[Training]]></term>
##     </thesaurusterms>
##     <pubtitle><![CDATA[Translational Engineering in Health and Medicine, IEEE Journal of]]></pubtitle>
##     <punumber><![CDATA[6221039]]></punumber>
##     <pubtype><![CDATA[Journals & Magazines]]></pubtype>
##     <publisher><![CDATA[IEEE]]></publisher>
##     <volume><![CDATA[2]]></volume>
##     <py><![CDATA[2014]]></py>
##     <spage><![CDATA[1]]></spage>
##     <epage><![CDATA[12]]></epage>
##     <abstract><![CDATA[Single-image super resolution is a process of obtaining a high-resolution image from a set of low-resolution observations by signal processing. While super resolution has been demonstrated to improve image quality in scaled down images in the image domain, its effects on the Fourier-based image acquisition technique, such as MRI, remains unknown. We performed high-resolution ex vivo late gadolinium enhancement (LGE) magnetic resonance imaging (0.4<formula formulatype="inline"> <img src="/images/tex/353.gif" alt=",\times,"> </formula>0.4<formula formulatype="inline"> <img src="/images/tex/353.gif" alt=",\times,"> </formula>0.4 <formula formulatype="inline"> <img src="/images/tex/23330.gif" alt="{\rm mm}^{{3}}"> </formula>) in postinfarction swine hearts <formula formulatype="inline"> <img src="/images/tex/23331.gif" alt="(n=24)"> </formula>. The swine hearts were divided into the training set <formula formulatype="inline"> <img src="/images/tex/23332.gif" alt="(n=14)"> </formula> and the test set <formula formulatype="inline"> <img src="/images/tex/23333.gif" alt="(n=10)"> </formula>, and in all hearts, low-resolution images were simulated from the high-resolution images. In the training set, super-resolution dictionaries with pairs of small matching patches of the high- and low-resolution images were created. In the test set, super resolution recovered high-resolution images from low-resolution images using the dictionaries. The same algorithm was also applied to patient LGE <formula formulatype="inline"> <img src="/images/tex/23334.gif" alt="(n=4)"> </formula> to assess its effects. Compared with interpolated images, super resolution significantly improved basic image quality indices <formula formulatype="inline"> <img src="/images/tex/23335.gif" alt="(P< 0.001)"> </formula>. Super resolution using Fourier-based zero padding achieved the best image quality. However, the magnitude of improvement was small in images with zero padding. Super resolution substantially improved the spatial resolution of the patient LGE images by sharpening the edges of the hear- and the scar. In conclusion, single-image super resolution significantly improves image errors. However, the magnitude of improvement was relatively small in images with Fourier-based zero padding. These findings provide evidence to support its potential use in myocardial scar imaging.]]></abstract>
##     <issn><![CDATA[2168-2372]]></issn>
##     <arnumber><![CDATA[6832680]]></arnumber>
##     <doi><![CDATA[10.1109/JTEHM.2014.2303806]]></doi>
##     <publicationId><![CDATA[6832680]]></publicationId>
##     <mdurl><![CDATA[http://ieeexplore.ieee.org/xpl/articleDetails.jsp?tp=&arnumber=6832680&contentType=Journals+%26+Magazines]]></mdurl>
##     <pdf><![CDATA[http://ieeexplore.ieee.org/stamp/stamp.jsp?arnumber=6832680]]></pdf>
##   </document>
##   <document>
##     <rank>801</rank>
##     <title><![CDATA[FingerSight: Fingertip Haptic Sensing of the Visual Environment]]></title>
##     <authors><![CDATA[Horvath, S.;  Galeotti, J.;  Bing Wu;  Klatzky, R.;  Siegel, M.;  Stetten, G.]]></authors>
##     <affiliations><![CDATA[Robot. Inst., Carnegie Mellon Univ., Pittsburgh, PA, USA]]></affiliations>
##     <controlledterms>
##       <term><![CDATA[biomedical optical imaging]]></term>
##       <term><![CDATA[cameras]]></term>
##       <term><![CDATA[feature extraction]]></term>
##       <term><![CDATA[handicapped aids]]></term>
##       <term><![CDATA[haptic interfaces]]></term>
##       <term><![CDATA[image representation]]></term>
##       <term><![CDATA[mechanoception]]></term>
##       <term><![CDATA[medical image processing]]></term>
##       <term><![CDATA[prototypes]]></term>
##       <term><![CDATA[vision]]></term>
##     </controlledterms>
##     <thesaurusterms>
##       <term><![CDATA[Cameras]]></term>
##       <term><![CDATA[Fingers]]></term>
##       <term><![CDATA[Haptic interfaces]]></term>
##       <term><![CDATA[Image edge detection]]></term>
##       <term><![CDATA[Visualization]]></term>
##     </thesaurusterms>
##     <pubtitle><![CDATA[Translational Engineering in Health and Medicine, IEEE Journal of]]></pubtitle>
##     <punumber><![CDATA[6221039]]></punumber>
##     <pubtype><![CDATA[Journals & Magazines]]></pubtype>
##     <publisher><![CDATA[IEEE]]></publisher>
##     <volume><![CDATA[2]]></volume>
##     <py><![CDATA[2014]]></py>
##     <spage><![CDATA[1]]></spage>
##     <epage><![CDATA[9]]></epage>
##     <abstract><![CDATA[We present a novel device mounted on the fingertip for acquiring and transmitting visual information through haptic channels. In contrast to previous systems in which the user interrogates an intermediate representation of visual information, such as a tactile display representing a camera generated image, our device uses a fingertip-mounted camera and haptic stimulator to allow the user to feel visual features directly from the environment. Visual features ranging from simple intensity or oriented edges to more complex information identified automatically about objects in the environment may be translated in this manner into haptic stimulation of the finger. Experiments using an initial prototype to trace a continuous straight edge have quantified the user's ability to discriminate the angle of the edge, a potentially useful feature for higher levels analysis of the visual scene.]]></abstract>
##     <issn><![CDATA[2168-2372]]></issn>
##     <htmlFlag><![CDATA[1]]></htmlFlag>
##     <arnumber><![CDATA[6758358]]></arnumber>
##     <doi><![CDATA[10.1109/JTEHM.2014.2309343]]></doi>
##     <publicationId><![CDATA[6758358]]></publicationId>
##     <mdurl><![CDATA[http://ieeexplore.ieee.org/xpl/articleDetails.jsp?tp=&arnumber=6758358&contentType=Journals+%26+Magazines]]></mdurl>
##     <pdf><![CDATA[http://ieeexplore.ieee.org/stamp/stamp.jsp?arnumber=6758358]]></pdf>
##   </document>
##   <document>
##     <rank>802</rank>
##     <title><![CDATA[MVCWalker: Random Walk-Based Most Valuable Collaborators Recommendation Exploiting Academic Factors]]></title>
##     <authors><![CDATA[Feng Xia;  Zhen Chen;  Wei Wang;  Jing Li;  Yang, L.T.]]></authors>
##     <affiliations><![CDATA[Sch. of Software, Dalian Univ. of Technol., Dalian, China]]></affiliations>
##     <controlledterms>
##       <term><![CDATA[educational administrative data processing]]></term>
##       <term><![CDATA[recommender systems]]></term>
##       <term><![CDATA[social networking (online)]]></term>
##     </controlledterms>
##     <thesaurusterms>
##       <term><![CDATA[Collaboration]]></term>
##       <term><![CDATA[Context]]></term>
##       <term><![CDATA[Data models]]></term>
##       <term><![CDATA[Educational institutions]]></term>
##       <term><![CDATA[Recommender systems]]></term>
##       <term><![CDATA[Social network services]]></term>
##       <term><![CDATA[Vectors]]></term>
##     </thesaurusterms>
##     <pubtitle><![CDATA[Emerging Topics in Computing, IEEE Transactions on]]></pubtitle>
##     <punumber><![CDATA[6245516]]></punumber>
##     <pubtype><![CDATA[Journals & Magazines]]></pubtype>
##     <publisher><![CDATA[IEEE]]></publisher>
##     <volume><![CDATA[2]]></volume>
##     <issue><![CDATA[3]]></issue>
##     <py><![CDATA[2014]]></py>
##     <spage><![CDATA[364]]></spage>
##     <epage><![CDATA[375]]></epage>
##     <abstract><![CDATA[In academia, scientific research achievements would be inconceivable without academic collaboration and cooperation among researchers. Previous studies have discovered that productive scholars tend to be more collaborative. However, it is often difficult and time-consuming for researchers to find the most valuable collaborators (MVCs) from a large volume of big scholarly data. In this paper, we present MVCWalker, an innovative method that stands on the shoulders of random walk with restart (RWR) for recommending collaborators to scholars. Three academic factors, i.e., coauthor order, latest collaboration time, and times of collaboration, are exploited to define link importance in academic social networks for the sake of recommendation quality. We conducted extensive experiments on DBLP data set in order to compare MVCWalker to the basic model of RWR and the common neighbor-based model friend of friends in various aspects, including, e.g., the impact of critical parameters and academic factors. Our experimental results show that incorporating the above factors into random walk model can improve the precision, recall rate, and coverage rate of academic collaboration recommendations.]]></abstract>
##     <issn><![CDATA[2168-6750]]></issn>
##     <htmlFlag><![CDATA[1]]></htmlFlag>
##     <arnumber><![CDATA[6894550]]></arnumber>
##     <doi><![CDATA[10.1109/TETC.2014.2356505]]></doi>
##     <publicationId><![CDATA[6894550]]></publicationId>
##     <mdurl><![CDATA[http://ieeexplore.ieee.org/xpl/articleDetails.jsp?tp=&arnumber=6894550&contentType=Journals+%26+Magazines]]></mdurl>
##     <pdf><![CDATA[http://ieeexplore.ieee.org/stamp/stamp.jsp?arnumber=6894550]]></pdf>
##   </document>
##   <document>
##     <rank>803</rank>
##     <title><![CDATA[A VCSEL-Composed Tunable Optical Band-Pass Filter for Long-Reach Optical Single Sideband Transport Systems]]></title>
##     <authors><![CDATA[Ching-Hung Chang;  Meng-Chun Tseng;  Jui-Hsuan Chang]]></authors>
##     <affiliations><![CDATA[Dept. of Electr. Eng., Nat. Chiayi Univ., Chiayi, Taiwan]]></affiliations>
##     <controlledterms>
##       <term><![CDATA[band-pass filters]]></term>
##       <term><![CDATA[laser cavity resonators]]></term>
##       <term><![CDATA[optical communication equipment]]></term>
##       <term><![CDATA[optical filters]]></term>
##       <term><![CDATA[optical modulation]]></term>
##       <term><![CDATA[optical tuning]]></term>
##       <term><![CDATA[phase modulation]]></term>
##       <term><![CDATA[radio-over-fibre]]></term>
##       <term><![CDATA[surface emitting lasers]]></term>
##     </controlledterms>
##     <thesaurusterms>
##       <term><![CDATA[Amplitude modulation]]></term>
##       <term><![CDATA[Optical attenuators]]></term>
##       <term><![CDATA[Optical fibers]]></term>
##       <term><![CDATA[Optical filters]]></term>
##       <term><![CDATA[Optical transmitters]]></term>
##       <term><![CDATA[Vertical cavity surface emitting lasers]]></term>
##     </thesaurusterms>
##     <pubtitle><![CDATA[Photonics Journal, IEEE]]></pubtitle>
##     <punumber><![CDATA[4563994]]></punumber>
##     <pubtype><![CDATA[Journals & Magazines]]></pubtype>
##     <publisher><![CDATA[IEEE]]></publisher>
##     <volume><![CDATA[6]]></volume>
##     <issue><![CDATA[4]]></issue>
##     <py><![CDATA[2014]]></py>
##     <spage><![CDATA[1]]></spage>
##     <epage><![CDATA[7]]></epage>
##     <abstract><![CDATA[This paper proposes a novel long-reach optical single sideband (OSSB) transport system based on a phase modulator and a vertical-cavity surface-emitting laser (VCSEL)-composed tunable optical band-pass filter (TOBPF). Experimental results show that the VCSEL-composed TOBPF can provide an approximately 5 GHz pass-band window, 350 GHz detuning frequency, and more than 13 dB stop-band attenuation to convert an optical phase-modulated radio frequency signal from optical double sideband (ODSB) format to OSSB format. Error-free transmissions and clear eye diagrams are experimentally obtained when a 1.25 Gbps/10 GHz RF signal is transmitted over a 60 km single mode fiber. The proposed system is shown to be efficient in promoting radio-over-fiber (RoF) transmission in long-distance passive optical networks.]]></abstract>
##     <issn><![CDATA[1943-0655]]></issn>
##     <htmlFlag><![CDATA[1]]></htmlFlag>
##     <arnumber><![CDATA[6866138]]></arnumber>
##     <doi><![CDATA[10.1109/JPHOT.2014.2343151]]></doi>
##     <publicationId><![CDATA[6866138]]></publicationId>
##     <mdurl><![CDATA[http://ieeexplore.ieee.org/xpl/articleDetails.jsp?tp=&arnumber=6866138&contentType=Journals+%26+Magazines]]></mdurl>
##     <pdf><![CDATA[http://ieeexplore.ieee.org/stamp/stamp.jsp?arnumber=6866138]]></pdf>
##   </document>
##   <document>
##     <rank>804</rank>
##     <title><![CDATA[A Combined Modulation of Set Current With Reset Voltage to Achieve 2-bit/cell Performance for Filament-Based RRAM]]></title>
##     <authors><![CDATA[Fang Yuan;  Zhigang Zhang;  Liyang Pan;  Jun Xu]]></authors>
##     <affiliations><![CDATA[Inst. of Microelectron., Tsinghua Univ., Beijing, China]]></affiliations>
##     <controlledterms>
##       <term><![CDATA[bipolar memory circuits]]></term>
##       <term><![CDATA[hafnium compounds]]></term>
##       <term><![CDATA[platinum]]></term>
##       <term><![CDATA[random-access storage]]></term>
##       <term><![CDATA[silver]]></term>
##     </controlledterms>
##     <thesaurusterms>
##       <term><![CDATA[Films]]></term>
##       <term><![CDATA[Hafnium compounds]]></term>
##       <term><![CDATA[Nonvolatile memory]]></term>
##       <term><![CDATA[Performance evaluation]]></term>
##       <term><![CDATA[Random access memory]]></term>
##       <term><![CDATA[Reliability]]></term>
##       <term><![CDATA[Resistance]]></term>
##     </thesaurusterms>
##     <pubtitle><![CDATA[Electron Devices Society, IEEE Journal of the]]></pubtitle>
##     <punumber><![CDATA[6245494]]></punumber>
##     <pubtype><![CDATA[Journals & Magazines]]></pubtype>
##     <publisher><![CDATA[IEEE]]></publisher>
##     <volume><![CDATA[2]]></volume>
##     <issue><![CDATA[6]]></issue>
##     <py><![CDATA[2014]]></py>
##     <spage><![CDATA[154]]></spage>
##     <epage><![CDATA[157]]></epage>
##     <abstract><![CDATA[A combined operation scheme to realize multibit switching in filament-based bipolar RRAM device is proposed. By combining the modulations of the current compliance in set operations with the stop voltage in reset operations together, the size or the quantity of the filaments in the film bulk can be controlled. An RRAM device with the structure of Ag/HfOx/Pt is fabricated and the 2-bit/cell memory function is achieved by the proposed modulation. Furthermore, the 2-bit/cell data reliability is satisfactory including the high-temperature retention, cycling endurance.]]></abstract>
##     <issn><![CDATA[2168-6734]]></issn>
##     <htmlFlag><![CDATA[1]]></htmlFlag>
##     <arnumber><![CDATA[6863671]]></arnumber>
##     <doi><![CDATA[10.1109/JEDS.2014.2342738]]></doi>
##     <publicationId><![CDATA[6863671]]></publicationId>
##     <mdurl><![CDATA[http://ieeexplore.ieee.org/xpl/articleDetails.jsp?tp=&arnumber=6863671&contentType=Journals+%26+Magazines]]></mdurl>
##     <pdf><![CDATA[http://ieeexplore.ieee.org/stamp/stamp.jsp?arnumber=6863671]]></pdf>
##   </document>
##   <document>
##     <rank>805</rank>
##     <title><![CDATA[Top orthogonal to bottom electrode (TOBE) 2-D CMUT arrays for 3-D photoacoustic imaging]]></title>
##     <authors><![CDATA[Chee, R.;  Sampaleanu, A.;  Rishi, D.;  Zemp, R.]]></authors>
##     <affiliations><![CDATA[Dept. of Electr. & Comput. Eng., Univ. of Alberta, Edmonton, AB, Canada]]></affiliations>
##     <controlledterms>
##       <term><![CDATA[acoustic imaging]]></term>
##       <term><![CDATA[biomedical optical imaging]]></term>
##       <term><![CDATA[biomedical transducers]]></term>
##       <term><![CDATA[biomedical ultrasonics]]></term>
##       <term><![CDATA[capacitive sensors]]></term>
##       <term><![CDATA[laser applications in medicine]]></term>
##       <term><![CDATA[phantoms]]></term>
##       <term><![CDATA[photoacoustic effect]]></term>
##       <term><![CDATA[ultrasonic transducer arrays]]></term>
##     </controlledterms>
##     <thesaurusterms>
##       <term><![CDATA[Arrays]]></term>
##       <term><![CDATA[Electrodes]]></term>
##       <term><![CDATA[Hair]]></term>
##       <term><![CDATA[Image resolution]]></term>
##       <term><![CDATA[Imaging]]></term>
##       <term><![CDATA[Transducers]]></term>
##       <term><![CDATA[Ultrasonic imaging]]></term>
##     </thesaurusterms>
##     <pubtitle><![CDATA[Ultrasonics, Ferroelectrics, and Frequency Control, IEEE Transactions on]]></pubtitle>
##     <punumber><![CDATA[58]]></punumber>
##     <pubtype><![CDATA[Journals & Magazines]]></pubtype>
##     <publisher><![CDATA[IEEE]]></publisher>
##     <volume><![CDATA[61]]></volume>
##     <issue><![CDATA[8]]></issue>
##     <py><![CDATA[2014]]></py>
##     <spage><![CDATA[1393]]></spage>
##     <epage><![CDATA[1395]]></epage>
##     <abstract><![CDATA[Top orthogonal to bottom electrode (TOBE) capacitive micromachined ultrasound transducers (CMUTs) are a new transducer architecture that permits large 2-D arrays to be addressed using row-column addressing. Here, we demonstrate the feasibility of 3-D photoacoustic imaging using N laser pulses and N receive channels. We used a synthetic aperture approach to simulate a large 2 &#x00D7; 2 cm array using a smaller die. A hair phantom in an oil immersion medium was excited by a laser, and the received signal was dynamically focused to obtain high-resolution images. We found the TOBE CMUT to have a center frequency of 3.7 MHz with a bandwidth of 3.9 MHz. Lateral and axial resolutions were 866 &#x03BC;m and 296 &#x03BC;m, respectively.]]></abstract>
##     <issn><![CDATA[0885-3010]]></issn>
##     <htmlFlag><![CDATA[1]]></htmlFlag>
##     <arnumber><![CDATA[6863862]]></arnumber>
##     <doi><![CDATA[10.1109/TUFFC.2014.3048]]></doi>
##     <publicationId><![CDATA[6863862]]></publicationId>
##     <mdurl><![CDATA[http://ieeexplore.ieee.org/xpl/articleDetails.jsp?tp=&arnumber=6863862&contentType=Journals+%26+Magazines]]></mdurl>
##     <pdf><![CDATA[http://ieeexplore.ieee.org/stamp/stamp.jsp?arnumber=6863862]]></pdf>
##   </document>
##   <document>
##     <rank>806</rank>
##     <title><![CDATA[Molecular Communication Among Biological Nanomachines: A Layered Architecture and Research Issues]]></title>
##     <authors><![CDATA[Nakano, T.;  Suda, T.;  Okaie, Y.;  Moore, M.J.;  Vasilakos, A.V.]]></authors>
##     <affiliations><![CDATA[Grad. Sch. of Frontier Biosci., Osaka Univ., Suita, Japan]]></affiliations>
##     <controlledterms>
##       <term><![CDATA[biocomputing]]></term>
##       <term><![CDATA[drug delivery systems]]></term>
##       <term><![CDATA[nanobiotechnology]]></term>
##       <term><![CDATA[tissue engineering]]></term>
##     </controlledterms>
##     <thesaurusterms>
##       <term><![CDATA[Biological cells]]></term>
##       <term><![CDATA[Computer architecture]]></term>
##       <term><![CDATA[Drugs]]></term>
##       <term><![CDATA[Molecular communication]]></term>
##       <term><![CDATA[Nanobioscience]]></term>
##       <term><![CDATA[Proteins]]></term>
##     </thesaurusterms>
##     <pubtitle><![CDATA[NanoBioscience, IEEE Transactions on]]></pubtitle>
##     <punumber><![CDATA[7728]]></punumber>
##     <pubtype><![CDATA[Journals & Magazines]]></pubtype>
##     <publisher><![CDATA[IEEE]]></publisher>
##     <volume><![CDATA[13]]></volume>
##     <issue><![CDATA[3]]></issue>
##     <py><![CDATA[2014]]></py>
##     <spage><![CDATA[169]]></spage>
##     <epage><![CDATA[197]]></epage>
##     <abstract><![CDATA[Molecular communication is an emerging communication paradigm for biological nanomachines. It allows biological nanomachines to communicate through exchanging molecules in an aqueous environment and to perform collaborative tasks through integrating functionalities of individual biological nanomachines. This paper develops the layered architecture of molecular communication and describes research issues that molecular communication faces at each layer of the architecture. Specifically, this paper applies a layered architecture approach, traditionally used in communication networks, to molecular communication, decomposes complex molecular communication functionality into a set of manageable layers, identifies basic functionalities of each layer, and develops a descriptive model consisting of key components of the layer for each layer. This paper also discusses open research issues that need to be addressed at each layer. In addition, this paper provides an example design of targeted drug delivery, a nanomedical application, to illustrate how the layered architecture helps design an application of molecular communication. The primary contribution of this paper is to provide an in-depth architectural view of molecular communication. Establishing a layered architecture of molecular communication helps organize various research issues and design concerns into layers that are relatively independent of each other, and thus accelerates research in each layer and facilitates the design and development of applications of molecular communication.]]></abstract>
##     <issn><![CDATA[1536-1241]]></issn>
##     <htmlFlag><![CDATA[1]]></htmlFlag>
##     <arnumber><![CDATA[6803950]]></arnumber>
##     <doi><![CDATA[10.1109/TNB.2014.2316674]]></doi>
##     <publicationId><![CDATA[6803950]]></publicationId>
##     <mdurl><![CDATA[http://ieeexplore.ieee.org/xpl/articleDetails.jsp?tp=&arnumber=6803950&contentType=Journals+%26+Magazines]]></mdurl>
##     <pdf><![CDATA[http://ieeexplore.ieee.org/stamp/stamp.jsp?arnumber=6803950]]></pdf>
##   </document>
##   <document>
##     <rank>807</rank>
##     <title><![CDATA[<inline-formula> <img src="/images/tex/24556.gif" alt="{mbi{\mu }}"> </inline-formula>-Foil Polymer Electrode Array for Intracortical Neural Recordings]]></title>
##     <authors><![CDATA[Ejserholm, F.;  Kohler, P.;  Granmo, M.;  Schouenborg, J.;  Bengtsson, M.;  Wallman, L.]]></authors>
##     <affiliations><![CDATA[Dept. of Biomed. Eng., Lund Univ., Lund, Sweden]]></affiliations>
##     <controlledterms>
##       <term><![CDATA[bioMEMS]]></term>
##       <term><![CDATA[bioelectric potentials]]></term>
##       <term><![CDATA[biological tissues]]></term>
##       <term><![CDATA[biomedical electrodes]]></term>
##       <term><![CDATA[biomedical measurement]]></term>
##       <term><![CDATA[brain]]></term>
##       <term><![CDATA[brain-computer interfaces]]></term>
##       <term><![CDATA[electric impedance]]></term>
##       <term><![CDATA[electroplating]]></term>
##       <term><![CDATA[foils]]></term>
##       <term><![CDATA[medical signal processing]]></term>
##       <term><![CDATA[microelectrodes]]></term>
##       <term><![CDATA[neurophysiology]]></term>
##       <term><![CDATA[platinum]]></term>
##       <term><![CDATA[polymers]]></term>
##     </controlledterms>
##     <thesaurusterms>
##       <term><![CDATA[Biological tissues]]></term>
##       <term><![CDATA[Brain modeling]]></term>
##       <term><![CDATA[Crosstalk]]></term>
##       <term><![CDATA[Electrodes]]></term>
##       <term><![CDATA[Impedance]]></term>
##       <term><![CDATA[In vitro]]></term>
##       <term><![CDATA[Multiaccess communication]]></term>
##     </thesaurusterms>
##     <pubtitle><![CDATA[Translational Engineering in Health and Medicine, IEEE Journal of]]></pubtitle>
##     <punumber><![CDATA[6221039]]></punumber>
##     <pubtype><![CDATA[Journals & Magazines]]></pubtype>
##     <publisher><![CDATA[IEEE]]></publisher>
##     <volume><![CDATA[2]]></volume>
##     <py><![CDATA[2014]]></py>
##     <spage><![CDATA[1]]></spage>
##     <epage><![CDATA[7]]></epage>
##     <abstract><![CDATA[We have developed a multichannel electrode array-termed &#x03BC;-foil-that comprises ultrathin and flexible electrodes protruding from a thin foil at fixed distances. In addition to allowing some of the active sites to reach less compromised tissue, the barb-like protrusions that also serves the purpose of anchoring the electrode array into the tissue. This paper is an early evaluation of technical aspects and performance of this electrode array in acute in vitro/in vivo experiments. The interface impedance was reduced by up to two decades by electroplating the active sites with platinum black. The platinum black also allowed for a reduced phase lag for higher frequency components. The distance between the protrusions of the electrode array was tailored to match the architecture of the rat cerebral cortex. In vivo acute measurements confirmed a high signal-to-noise ratio for the neural recordings, and no significant crosstalk between recording channels.]]></abstract>
##     <issn><![CDATA[2168-2372]]></issn>
##     <htmlFlag><![CDATA[1]]></htmlFlag>
##     <arnumber><![CDATA[6823093]]></arnumber>
##     <doi><![CDATA[10.1109/JTEHM.2014.2326859]]></doi>
##     <publicationId><![CDATA[6823093]]></publicationId>
##     <mdurl><![CDATA[http://ieeexplore.ieee.org/xpl/articleDetails.jsp?tp=&arnumber=6823093&contentType=Journals+%26+Magazines]]></mdurl>
##     <pdf><![CDATA[http://ieeexplore.ieee.org/stamp/stamp.jsp?arnumber=6823093]]></pdf>
##   </document>
##   <document>
##     <rank>808</rank>
##     <title><![CDATA[Flexible InGaN LEDs on a Polyimide Substrate Fabricated Using a Simple Direct-Transfer Method]]></title>
##     <authors><![CDATA[Won-Sik Choi;  Hyung Jo Park;  Si-Hyun Park;  Tak Jeong]]></authors>
##     <affiliations><![CDATA[Dept. of Electron. Eng., Yeungnam Univ., Gwangju, South Korea]]></affiliations>
##     <controlledterms>
##       <term><![CDATA[III-V semiconductors]]></term>
##       <term><![CDATA[etching]]></term>
##       <term><![CDATA[gallium compounds]]></term>
##       <term><![CDATA[indium compounds]]></term>
##       <term><![CDATA[laser materials processing]]></term>
##       <term><![CDATA[optical fabrication]]></term>
##       <term><![CDATA[organic light emitting diodes]]></term>
##       <term><![CDATA[polymers]]></term>
##       <term><![CDATA[sapphire]]></term>
##       <term><![CDATA[semiconductor epitaxial layers]]></term>
##     </controlledterms>
##     <thesaurusterms>
##       <term><![CDATA[Bonding]]></term>
##       <term><![CDATA[Gallium nitride]]></term>
##       <term><![CDATA[Light emitting diodes]]></term>
##       <term><![CDATA[Metals]]></term>
##       <term><![CDATA[Polyimides]]></term>
##       <term><![CDATA[Substrates]]></term>
##     </thesaurusterms>
##     <pubtitle><![CDATA[Photonics Technology Letters, IEEE]]></pubtitle>
##     <punumber><![CDATA[68]]></punumber>
##     <pubtype><![CDATA[Journals & Magazines]]></pubtype>
##     <publisher><![CDATA[IEEE]]></publisher>
##     <volume><![CDATA[26]]></volume>
##     <issue><![CDATA[21]]></issue>
##     <py><![CDATA[2014]]></py>
##     <spage><![CDATA[2115]]></spage>
##     <epage><![CDATA[2117]]></epage>
##     <abstract><![CDATA[An array of InGaN-based flexible light-emitting diodes (FLEDs) was fabricated on a full-scale 2-in polyimide substrate. An InGaN epitaxial layer on a sapphire substrate was directly bonded with a polyimide substrate, and the sapphire substrate was then removed via a laser lift-off process. A subsequent n-GaN etching process for a chip isolation finished LED chips over the entire 2-in polyimide substrate. Using this simple direct-transfer process, we obtained a production yield of over 97%. The FLED device operated linearly up to an input current level of 500 mA. Output power, operating voltage, and wavelength shift of the FLED up to 400-mA driving current were nearly the same as for a vertical LED on metal substrate.]]></abstract>
##     <issn><![CDATA[1041-1135]]></issn>
##     <htmlFlag><![CDATA[1]]></htmlFlag>
##     <arnumber><![CDATA[6879312]]></arnumber>
##     <doi><![CDATA[10.1109/LPT.2014.2348591]]></doi>
##     <publicationId><![CDATA[6879312]]></publicationId>
##     <mdurl><![CDATA[http://ieeexplore.ieee.org/xpl/articleDetails.jsp?tp=&arnumber=6879312&contentType=Journals+%26+Magazines]]></mdurl>
##     <pdf><![CDATA[http://ieeexplore.ieee.org/stamp/stamp.jsp?arnumber=6879312]]></pdf>
##   </document>
##   <document>
##     <rank>809</rank>
##     <title><![CDATA[Unified Differential Spatial Modulation]]></title>
##     <authors><![CDATA[Ishikawa, N.;  Sugiura, S.]]></authors>
##     <affiliations><![CDATA[Dept. of Comput. & Inf. Sci., Tokyo Univ. of Agric. & Technol., Koganei, Japan]]></affiliations>
##     <controlledterms>
##       <term><![CDATA[channel estimation]]></term>
##       <term><![CDATA[diversity reception]]></term>
##       <term><![CDATA[phase shift keying]]></term>
##       <term><![CDATA[radio receivers]]></term>
##       <term><![CDATA[radio transmitters]]></term>
##     </controlledterms>
##     <thesaurusterms>
##       <term><![CDATA[Diversity methods]]></term>
##       <term><![CDATA[Phase shift keying]]></term>
##       <term><![CDATA[Sparse matrices]]></term>
##       <term><![CDATA[Transmitting antennas]]></term>
##     </thesaurusterms>
##     <pubtitle><![CDATA[Wireless Communications Letters, IEEE]]></pubtitle>
##     <punumber><![CDATA[5962382]]></punumber>
##     <pubtype><![CDATA[Journals & Magazines]]></pubtype>
##     <publisher><![CDATA[IEEE]]></publisher>
##     <volume><![CDATA[3]]></volume>
##     <issue><![CDATA[4]]></issue>
##     <py><![CDATA[2014]]></py>
##     <spage><![CDATA[337]]></spage>
##     <epage><![CDATA[340]]></epage>
##     <abstract><![CDATA[This letter proposes a unified differential spatial modulation (DSM) architecture, where a flexible rate-diversity tradeoff is achieved, while enabling a simple single-RF transmitter structure along with non-coherent detection that dispenses with channel estimation at the receiver. In our proposed scheme, by assigning a set of sparse complex-valued antenna-index matrices, only one transmit antenna element is activated during each symbol interval and then a phase-shift keying (PSK) symbol is transmitted from the activated antenna element. The explicit benefit of the proposed scheme's universal DSM framework is ability to strike a balance between previous DSM schemes, such as the symbol-based and the block-based DSM schemes. Moreover, to attain a useful attainable diversity gain, we further extend the proposed DSM scheme in a manner that permits flexible planning of the number of symbols employed per antenna-index block.]]></abstract>
##     <issn><![CDATA[2162-2337]]></issn>
##     <htmlFlag><![CDATA[1]]></htmlFlag>
##     <arnumber><![CDATA[6783809]]></arnumber>
##     <doi><![CDATA[10.1109/LWC.2014.2315635]]></doi>
##     <publicationId><![CDATA[6783809]]></publicationId>
##     <mdurl><![CDATA[http://ieeexplore.ieee.org/xpl/articleDetails.jsp?tp=&arnumber=6783809&contentType=Journals+%26+Magazines]]></mdurl>
##     <pdf><![CDATA[http://ieeexplore.ieee.org/stamp/stamp.jsp?arnumber=6783809]]></pdf>
##   </document>
##   <document>
##     <rank>810</rank>
##     <title><![CDATA[Method for Improving Spatial Resolution and Amplitude by Optimized Deskew Filter in Long-Range OFDR]]></title>
##     <authors><![CDATA[Yang Du;  Tiegen Liu;  Zhenyang Ding;  Bowen Feng;  Xiaobo Li;  Kun Liu;  Junfeng Jiang]]></authors>
##     <affiliations><![CDATA[Coll. of Precision Instrum. & Opto-Electron. Eng., Tianjin Univ., Tianjin, China]]></affiliations>
##     <controlledterms>
##       <term><![CDATA[light interferometers]]></term>
##       <term><![CDATA[optical filters]]></term>
##       <term><![CDATA[reflectometry]]></term>
##     </controlledterms>
##     <thesaurusterms>
##       <term><![CDATA[Estimation]]></term>
##       <term><![CDATA[Optical fiber couplers]]></term>
##       <term><![CDATA[Optical fiber filters]]></term>
##       <term><![CDATA[Optical fiber polarization]]></term>
##       <term><![CDATA[Optical interferometry]]></term>
##       <term><![CDATA[Spatial resolution]]></term>
##       <term><![CDATA[Taylor series]]></term>
##     </thesaurusterms>
##     <pubtitle><![CDATA[Photonics Journal, IEEE]]></pubtitle>
##     <punumber><![CDATA[4563994]]></punumber>
##     <pubtype><![CDATA[Journals & Magazines]]></pubtype>
##     <publisher><![CDATA[IEEE]]></publisher>
##     <volume><![CDATA[6]]></volume>
##     <issue><![CDATA[5]]></issue>
##     <py><![CDATA[2014]]></py>
##     <spage><![CDATA[1]]></spage>
##     <epage><![CDATA[11]]></epage>
##     <abstract><![CDATA[We present a method for improving the spatial resolution and the amplitude by an optimized deskew filter in long-range optical frequency-domain reflectometry (OFDR). In a previous deskew-filter method, as the nonlinear phase estimated from an auxiliary interferometer is used to compensate for the nonlinearity effect in the beating signals generated from a main OFDR interferometer, the spatial resolution and amplitude of the reflection peak in a long range (i.e, 80 km) are deteriorated by a residual nonlinearity effect due to the estimation inaccuracy of the nonlinear phase. In the proposed optimized deskew-filter method, the estimation accuracy of the nonlinear phase is improved by the higher orders of Taylor expansion and the high accuracy of the estimation of the time delay in the auxiliary interferometer using a cepstrum. We experimentally demonstrate that the amplitude of a reflection peak at 80 km increases by 20.5 dB and that the spatial resolution is up to 80 cm, which is about 187 times enhancement when compared with that of the same OFDR trace without nonlinearity compensation.]]></abstract>
##     <issn><![CDATA[1943-0655]]></issn>
##     <htmlFlag><![CDATA[1]]></htmlFlag>
##     <arnumber><![CDATA[6887311]]></arnumber>
##     <doi><![CDATA[10.1109/JPHOT.2014.2352622]]></doi>
##     <publicationId><![CDATA[6887311]]></publicationId>
##     <mdurl><![CDATA[http://ieeexplore.ieee.org/xpl/articleDetails.jsp?tp=&arnumber=6887311&contentType=Journals+%26+Magazines]]></mdurl>
##     <pdf><![CDATA[http://ieeexplore.ieee.org/stamp/stamp.jsp?arnumber=6887311]]></pdf>
##   </document>
##   <document>
##     <rank>811</rank>
##     <title><![CDATA[General Factorization of Conjugate-Symmetric Hadamard Transforms]]></title>
##     <authors><![CDATA[Kyochi, S.;  Tanaka, Y.]]></authors>
##     <affiliations><![CDATA[Dept. of Inf. & Media Eng., Univ. of Kitakyushu, Fukuoka, Japan]]></affiliations>
##     <controlledterms>
##       <term><![CDATA[Hadamard transforms]]></term>
##       <term><![CDATA[computational complexity]]></term>
##       <term><![CDATA[discrete Fourier transforms]]></term>
##       <term><![CDATA[image coding]]></term>
##       <term><![CDATA[matrix decomposition]]></term>
##     </controlledterms>
##     <thesaurusterms>
##       <term><![CDATA[Computational complexity]]></term>
##       <term><![CDATA[Discrete Fourier transforms]]></term>
##       <term><![CDATA[Estimation]]></term>
##       <term><![CDATA[Image coding]]></term>
##       <term><![CDATA[Licenses]]></term>
##       <term><![CDATA[Signal processing]]></term>
##     </thesaurusterms>
##     <pubtitle><![CDATA[Signal Processing, IEEE Transactions on]]></pubtitle>
##     <punumber><![CDATA[78]]></punumber>
##     <pubtype><![CDATA[Journals & Magazines]]></pubtype>
##     <publisher><![CDATA[IEEE]]></publisher>
##     <volume><![CDATA[62]]></volume>
##     <issue><![CDATA[13]]></issue>
##     <py><![CDATA[2014]]></py>
##     <spage><![CDATA[3379]]></spage>
##     <epage><![CDATA[3392]]></epage>
##     <abstract><![CDATA[Complex-valued conjugate-symmetric Hadamard transforms (C-CSHT) are variants of complex Hadamard transforms and found applications in signal processing. In addition, their real-valued transform counterparts (R-CSHTs) perform comparably with Hadamard transforms (HTs) despite their lower computational complexity. Closed-form factorizations of C-CSHTs and R-CSHTs have recently been proposed to make calculations more efficient. However, there is still room to find effective and general factorizations. This paper presents a simple closed-form complete factorization of C-CSHTs based on that of R-CSHTs. The proposed factorization can be applied to both C- and R-CSHTs with one factorization and it provides several benefits: 1) It can save total implementation costs for both C-CSHTs and R-CSHTs; 2) the generalized R-CSHT factorization significantly reduces its computational cost; 3) memory-saved local orientation detection of images can be achieved; 4) a fast direction-aware transform can be attained; 5) it clarifies that C- and R-CSHTs are closely related to common block transforms, such as the discrete Fourier transform (DFT), binDCT, and HT; and 6) it achieves a new integer complex-valued transform, which can approximate the DFT better than the original C-CSHT. The image orientation estimation and performance in image coding of our R-CSHTs were evaluated through examples of practical applications based on the proposed factorization.]]></abstract>
##     <issn><![CDATA[1053-587X]]></issn>
##     <htmlFlag><![CDATA[1]]></htmlFlag>
##     <arnumber><![CDATA[6820782]]></arnumber>
##     <doi><![CDATA[10.1109/TSP.2014.2326620]]></doi>
##     <publicationId><![CDATA[6820782]]></publicationId>
##     <mdurl><![CDATA[http://ieeexplore.ieee.org/xpl/articleDetails.jsp?tp=&arnumber=6820782&contentType=Journals+%26+Magazines]]></mdurl>
##     <pdf><![CDATA[http://ieeexplore.ieee.org/stamp/stamp.jsp?arnumber=6820782]]></pdf>
##   </document>
##   <document>
##     <rank>812</rank>
##     <title><![CDATA[High-Precision Control of Ball-Screw-Driven Stage Based on Repetitive Control Using <formula formulatype="inline"> <img src="/images/tex/388.gif" alt="n"> </formula>-Times Learning Filter]]></title>
##     <authors><![CDATA[Fujimoto, H.;  Takemura, T.]]></authors>
##     <affiliations><![CDATA[Dept. of Adv. Energy, Univ. of Tokyo, Kashiwa, Japan]]></affiliations>
##     <controlledterms>
##       <term><![CDATA[adaptive control]]></term>
##       <term><![CDATA[ball screws]]></term>
##       <term><![CDATA[control system synthesis]]></term>
##       <term><![CDATA[friction]]></term>
##       <term><![CDATA[learning systems]]></term>
##       <term><![CDATA[position control]]></term>
##     </controlledterms>
##     <pubtitle><![CDATA[Industrial Electronics, IEEE Transactions on]]></pubtitle>
##     <punumber><![CDATA[41]]></punumber>
##     <pubtype><![CDATA[Journals & Magazines]]></pubtype>
##     <publisher><![CDATA[IEEE]]></publisher>
##     <volume><![CDATA[61]]></volume>
##     <issue><![CDATA[7]]></issue>
##     <py><![CDATA[2014]]></py>
##     <spage><![CDATA[3694]]></spage>
##     <epage><![CDATA[3703]]></epage>
##     <abstract><![CDATA[This paper presents a novel learning control method for ball-screw-driven stages. In recent years, many types of friction models that are based on complicated equations have been studied. However, it is difficult to treat friction models with equations because the level of precision that is associated with real friction characteristics and parameter tuning are difficult to achieve. In contrast, repetitive perfect tracking control (RPTC) is a repetitive control technique that achieves high-precision positioning. In this paper, we propose the use of RPTC with n-times learning filter. The n-times learning filter has a sharper rolloff property than conventional learning filters. With the use of the n-times learning filter, the proposed RPTC can converge tracking errors n times faster than the RPTC with the conventional learning filter. Simulations and experiments with a ball-screw-driven stage show the fast convergence of the proposed RPTC. Finally, the proposed learning control scheme is combined with data-based friction compensation, and the effectiveness of this combination is verified for the x-y stage of a numerically controlled machine tool.]]></abstract>
##     <issn><![CDATA[0278-0046]]></issn>
##     <htmlFlag><![CDATA[1]]></htmlFlag>
##     <arnumber><![CDATA[6661341]]></arnumber>
##     <doi><![CDATA[10.1109/TIE.2013.2290286]]></doi>
##     <publicationId><![CDATA[6661341]]></publicationId>
##     <mdurl><![CDATA[http://ieeexplore.ieee.org/xpl/articleDetails.jsp?tp=&arnumber=6661341&contentType=Journals+%26+Magazines]]></mdurl>
##     <pdf><![CDATA[http://ieeexplore.ieee.org/stamp/stamp.jsp?arnumber=6661341]]></pdf>
##   </document>
##   <document>
##     <rank>813</rank>
##     <title><![CDATA[Coverage Optimization and Power Reduction in SFN Using Simulated Annealing]]></title>
##     <authors><![CDATA[Lanza, M.;  Gutierrez, A.L.;  Perez, J.R.;  Morgade, J.;  Domingo, M.;  Valle, L.;  Angueira, P.;  Basterrechea, J.]]></authors>
##     <affiliations><![CDATA[Dept. de Ing. de Comun., Univ. of Cantabria, Santander, Spain]]></affiliations>
##     <controlledterms>
##       <term><![CDATA[digital video broadcasting]]></term>
##       <term><![CDATA[optimisation]]></term>
##       <term><![CDATA[radio receivers]]></term>
##       <term><![CDATA[radio transmitters]]></term>
##       <term><![CDATA[radiofrequency interference]]></term>
##       <term><![CDATA[transmitting antennas]]></term>
##     </controlledterms>
##     <thesaurusterms>
##       <term><![CDATA[Azimuth]]></term>
##       <term><![CDATA[Digital video broadcasting]]></term>
##       <term><![CDATA[OFDM]]></term>
##       <term><![CDATA[Optimization]]></term>
##       <term><![CDATA[Receivers]]></term>
##       <term><![CDATA[Transmitting antennas]]></term>
##     </thesaurusterms>
##     <pubtitle><![CDATA[Broadcasting, IEEE Transactions on]]></pubtitle>
##     <punumber><![CDATA[11]]></punumber>
##     <pubtype><![CDATA[Journals & Magazines]]></pubtype>
##     <publisher><![CDATA[IEEE]]></publisher>
##     <volume><![CDATA[60]]></volume>
##     <issue><![CDATA[3]]></issue>
##     <py><![CDATA[2014]]></py>
##     <spage><![CDATA[474]]></spage>
##     <epage><![CDATA[485]]></epage>
##     <abstract><![CDATA[An approach that predicts the propagation, models the terrestrial receivers and optimizes the performance of single frequency networks (SFN) for digital video broadcasting in terms of the final coverage achieved over any geographical region, enhancing the most populated areas, is proposed in this paper. The effective coverage improvement and thus, the self-interference reduction in the SFN is accomplished by optimizing the internal static delays, sector antenna gain, and both azimuth and elevation orientation for every transmitter within the network using the heuristic simulated annealing (SA) algorithm. Decimation and elevation filtering techniques have been considered and applied to reduce the computational cost of the SA-based approach, including results that demonstrate the improvements achieved. Further representative results for two SFN in different scenarios considering the effect on the final coverage of optimizing any of the transmitter parameters previously outlined or a combination of some of them are reported and discussed in order to show both, the performance of the method and how increasing gradually the complexity of the model for the transmitters leads to more realistic and accurate results.]]></abstract>
##     <issn><![CDATA[0018-9316]]></issn>
##     <htmlFlag><![CDATA[1]]></htmlFlag>
##     <arnumber><![CDATA[6874529]]></arnumber>
##     <doi><![CDATA[10.1109/TBC.2014.2333131]]></doi>
##     <publicationId><![CDATA[6874529]]></publicationId>
##     <mdurl><![CDATA[http://ieeexplore.ieee.org/xpl/articleDetails.jsp?tp=&arnumber=6874529&contentType=Journals+%26+Magazines]]></mdurl>
##     <pdf><![CDATA[http://ieeexplore.ieee.org/stamp/stamp.jsp?arnumber=6874529]]></pdf>
##   </document>
##   <document>
##     <rank>814</rank>
##     <title><![CDATA[Multifunctional Polymer Mach&#x2013;Zehnder Optical Switch/Filter Using Side-Coupled <inline-formula> <img src="/images/tex/16367.gif" alt="M\times N"> </inline-formula> Electrooptic Microring Array]]></title>
##     <authors><![CDATA[Zheng, C.T.;  Luo, Q.Q.;  Huang, X.L.;  Zhang, D.M.]]></authors>
##     <affiliations><![CDATA[State Key Lab. on Integrated Optoelectron., Jilin Univ., Changchun, China]]></affiliations>
##     <controlledterms>
##       <term><![CDATA[Mach-Zehnder interferometers]]></term>
##       <term><![CDATA[band-pass filters]]></term>
##       <term><![CDATA[electro-optical devices]]></term>
##       <term><![CDATA[optical arrays]]></term>
##       <term><![CDATA[optical crosstalk]]></term>
##       <term><![CDATA[optical design techniques]]></term>
##       <term><![CDATA[optical filters]]></term>
##       <term><![CDATA[optical losses]]></term>
##       <term><![CDATA[optical polymers]]></term>
##       <term><![CDATA[optical switches]]></term>
##     </controlledterms>
##     <thesaurusterms>
##       <term><![CDATA[Couplings]]></term>
##       <term><![CDATA[Electrooptical waveguides]]></term>
##       <term><![CDATA[Indexes]]></term>
##       <term><![CDATA[Optical filters]]></term>
##       <term><![CDATA[Optical switches]]></term>
##       <term><![CDATA[Waveguide discontinuities]]></term>
##     </thesaurusterms>
##     <pubtitle><![CDATA[Photonics Journal, IEEE]]></pubtitle>
##     <punumber><![CDATA[4563994]]></punumber>
##     <pubtype><![CDATA[Journals & Magazines]]></pubtype>
##     <publisher><![CDATA[IEEE]]></publisher>
##     <volume><![CDATA[6]]></volume>
##     <issue><![CDATA[4]]></issue>
##     <py><![CDATA[2014]]></py>
##     <spage><![CDATA[1]]></spage>
##     <epage><![CDATA[16]]></epage>
##     <abstract><![CDATA[Structure and design were proposed for a kind of novel multifunctional M &#x00D7; N microring array side-coupled Mach-Zehnder interferometer (MZI) electrooptic switch/ filter. The device can be operated as a notch or bandpass filter by applying voltages on the under-reference MZI arm, and it can also serve as an optical switch by applying voltage on microrings. The dependence relations of the device's performances on RSA size (M and N) are thoroughly analyzed and concluded. As a switch, for making driving voltages under both operation states below 5 V, N and M should be within the range of 2-8. For the seven switches (M = 1, 8 &#x2265; N &#x2265; 2), their insertion losses are within 0.85-0.35 dB at bar state and within 0.14-1.70 dB at cross state; the crosstalk are within -20.79- -31.11 dB at bar state and within -13.86--20.36 dB at cross state; and their 3-dB electrical bandwidths are 18.5-30.5 GHz, whereas their 3-dB optical bandwidths for ports 1 and 2 are 18.1-40.0 GHz and 19.0-55.0 GHz, respectively. In addition, the needed energy per bit under 10-GHz switching operations is within 0.092-0.697 pJ. Therefore, the proposed multifunctional device shows potential applications, including filtering, routing, and switching, in optical networks-on-chip.]]></abstract>
##     <issn><![CDATA[1943-0655]]></issn>
##     <htmlFlag><![CDATA[1]]></htmlFlag>
##     <arnumber><![CDATA[6842596]]></arnumber>
##     <doi><![CDATA[10.1109/JPHOT.2014.2332459]]></doi>
##     <publicationId><![CDATA[6842596]]></publicationId>
##     <mdurl><![CDATA[http://ieeexplore.ieee.org/xpl/articleDetails.jsp?tp=&arnumber=6842596&contentType=Journals+%26+Magazines]]></mdurl>
##     <pdf><![CDATA[http://ieeexplore.ieee.org/stamp/stamp.jsp?arnumber=6842596]]></pdf>
##   </document>
##   <document>
##     <rank>815</rank>
##     <title><![CDATA[Bio-Inspired Glucose Control in Diabetes Based on an Analogue Implementation of a <formula formulatype="inline"> <img src="/images/tex/21444.gif" alt="\beta "> </formula>-Cell Model]]></title>
##     <authors><![CDATA[Pagkalos, I.;  Herrero, P.;  Toumazou, C.;  Georgiou, P.]]></authors>
##     <affiliations><![CDATA[Dept. of Bionengineering, Imperial Coll. London, London, UK]]></affiliations>
##     <controlledterms>
##       <term><![CDATA[CMOS integrated circuits]]></term>
##       <term><![CDATA[cellular transport]]></term>
##       <term><![CDATA[diseases]]></term>
##       <term><![CDATA[drug delivery systems]]></term>
##       <term><![CDATA[low-power electronics]]></term>
##       <term><![CDATA[medical control systems]]></term>
##       <term><![CDATA[physiological models]]></term>
##       <term><![CDATA[sugar]]></term>
##       <term><![CDATA[three-term control]]></term>
##     </controlledterms>
##     <thesaurusterms>
##       <term><![CDATA[Blood]]></term>
##       <term><![CDATA[Diabetes]]></term>
##       <term><![CDATA[Insulin]]></term>
##       <term><![CDATA[Mathematical model]]></term>
##       <term><![CDATA[Pancreas]]></term>
##       <term><![CDATA[Sugar]]></term>
##       <term><![CDATA[Threshold voltage]]></term>
##     </thesaurusterms>
##     <pubtitle><![CDATA[Biomedical Circuits and Systems, IEEE Transactions on]]></pubtitle>
##     <punumber><![CDATA[4156126]]></punumber>
##     <pubtype><![CDATA[Journals & Magazines]]></pubtype>
##     <publisher><![CDATA[IEEE]]></publisher>
##     <volume><![CDATA[8]]></volume>
##     <issue><![CDATA[2]]></issue>
##     <py><![CDATA[2014]]></py>
##     <spage><![CDATA[186]]></spage>
##     <epage><![CDATA[195]]></epage>
##     <abstract><![CDATA[This paper presents a bio-inspired method for in-vivo control of blood glucose based on a model of the pancreatic &#x03B2;-cell. The proposed model is shown to be implementable using low-power analogue integrated circuits in CMOS, realizing a biologically faithful implementation which captures all the behaviours seen in physiology. This is then shown to be capable of glucose control using an in silico population of diabetic subjects achieving 93% of the time in tight glycemic target (i.e., [70, 140] mg/dl) . The proposed controller is then compared with a commonly used external physiological insulin delivery (ePID) controller for glucose control. Results confirm equivalent, or superior, performance in comparison with ePID. The system has been designed in a commercially available 0.35 &#x03BC;m CMOS process and achieves an overall power consumption of 1.907 mW.]]></abstract>
##     <issn><![CDATA[1932-4545]]></issn>
##     <htmlFlag><![CDATA[1]]></htmlFlag>
##     <arnumber><![CDATA[6778812]]></arnumber>
##     <doi><![CDATA[10.1109/TBCAS.2014.2301377]]></doi>
##     <publicationId><![CDATA[6778812]]></publicationId>
##     <mdurl><![CDATA[http://ieeexplore.ieee.org/xpl/articleDetails.jsp?tp=&arnumber=6778812&contentType=Journals+%26+Magazines]]></mdurl>
##     <pdf><![CDATA[http://ieeexplore.ieee.org/stamp/stamp.jsp?arnumber=6778812]]></pdf>
##   </document>
##   <document>
##     <rank>816</rank>
##     <title><![CDATA[A Simple Ladder Realization of Maximally Flat Allpass Fractional Delay Filters]]></title>
##     <authors><![CDATA[Koshita, S.;  Abe, M.;  Kawamata, M.]]></authors>
##     <affiliations><![CDATA[Dept. of Electron. Eng., Tohoku Univ., Sendai, Japan]]></affiliations>
##     <controlledterms>
##       <term><![CDATA[all-pass filters]]></term>
##       <term><![CDATA[delay filters]]></term>
##       <term><![CDATA[digital filters]]></term>
##       <term><![CDATA[ladder filters]]></term>
##     </controlledterms>
##     <thesaurusterms>
##       <term><![CDATA[Approximation methods]]></term>
##       <term><![CDATA[Circuit stability]]></term>
##       <term><![CDATA[Delays]]></term>
##       <term><![CDATA[Periodic structures]]></term>
##       <term><![CDATA[Prototypes]]></term>
##       <term><![CDATA[Speech]]></term>
##       <term><![CDATA[Transfer functions]]></term>
##     </thesaurusterms>
##     <pubtitle><![CDATA[Circuits and Systems II: Express Briefs, IEEE Transactions on]]></pubtitle>
##     <punumber><![CDATA[8920]]></punumber>
##     <pubtype><![CDATA[Journals & Magazines]]></pubtype>
##     <publisher><![CDATA[IEEE]]></publisher>
##     <volume><![CDATA[61]]></volume>
##     <issue><![CDATA[3]]></issue>
##     <py><![CDATA[2014]]></py>
##     <spage><![CDATA[203]]></spage>
##     <epage><![CDATA[207]]></epage>
##     <abstract><![CDATA[This brief proposes a new ladder structure for the Thiran fractional delay filter (i.e., maximally flat allpass fractional delay filter given by the Thiran approximation). The proposed ladder structure is based on a continued fraction representation. Although there exists a similar approach that was proposed by Tassart and Depalle, their structure is not realizable because of delay-free loops. On the other hand, we show that the proposed method avoids generating delay-free loops and thus successfully yields a realizable ladder structure for the Thiran fractional delay filter in a very simple form.]]></abstract>
##     <issn><![CDATA[1549-7747]]></issn>
##     <htmlFlag><![CDATA[1]]></htmlFlag>
##     <arnumber><![CDATA[6718008]]></arnumber>
##     <doi><![CDATA[10.1109/TCSII.2013.2296131]]></doi>
##     <publicationId><![CDATA[6718008]]></publicationId>
##     <mdurl><![CDATA[http://ieeexplore.ieee.org/xpl/articleDetails.jsp?tp=&arnumber=6718008&contentType=Journals+%26+Magazines]]></mdurl>
##     <pdf><![CDATA[http://ieeexplore.ieee.org/stamp/stamp.jsp?arnumber=6718008]]></pdf>
##   </document>
##   <document>
##     <rank>817</rank>
##     <title><![CDATA[Constrained <formula formulatype="inline"> <img src="/images/tex/21381.gif" alt="{\rm T}p{\rm V}"> </formula> Minimization for Enhanced Exploitation of Gradient Sparsity: Application to CT Image Reconstruction]]></title>
##     <authors><![CDATA[Sidky, E.Y.;  Chartrand, R.;  Boone, J.M.;  Xiaochuan Pan]]></authors>
##     <affiliations><![CDATA[Dept. of Radiol., Univ. of Chicago, Chicago, IL, USA]]></affiliations>
##     <controlledterms>
##       <term><![CDATA[computerised tomography]]></term>
##       <term><![CDATA[diagnostic radiography]]></term>
##       <term><![CDATA[image reconstruction]]></term>
##       <term><![CDATA[image sampling]]></term>
##       <term><![CDATA[mammography]]></term>
##       <term><![CDATA[medical image processing]]></term>
##       <term><![CDATA[minimisation]]></term>
##     </controlledterms>
##     <thesaurusterms>
##       <term><![CDATA[Computed tomography]]></term>
##       <term><![CDATA[Image reconstruction]]></term>
##       <term><![CDATA[Minimization]]></term>
##       <term><![CDATA[Optimization]]></term>
##       <term><![CDATA[Sparsity]]></term>
##       <term><![CDATA[X-ray detection]]></term>
##     </thesaurusterms>
##     <pubtitle><![CDATA[Translational Engineering in Health and Medicine, IEEE Journal of]]></pubtitle>
##     <punumber><![CDATA[6221039]]></punumber>
##     <pubtype><![CDATA[Journals & Magazines]]></pubtype>
##     <publisher><![CDATA[IEEE]]></publisher>
##     <volume><![CDATA[2]]></volume>
##     <py><![CDATA[2014]]></py>
##     <spage><![CDATA[1]]></spage>
##     <epage><![CDATA[18]]></epage>
##     <abstract><![CDATA[Exploiting sparsity in the image gradient magnitude has proved to be an effective means for reducing the sampling rate in the projection view angle in computed tomography (CT). Most of the image reconstruction algorithms, developed for this purpose, solve a nonsmooth convex optimization problem involving the image total variation (TV). The TV seminorm is the &#x2113;<sub>1</sub> norm of the image gradient magnitude, and reducing the &#x2113;<sub>1</sub> norm is known to encourage sparsity in its argument. Recently, there has been interest in employing nonconvex lp quasinorms with for sparsity exploiting image reconstruction, which is potentially more effective than &#x2113;<sub>1</sub> because nonconvex lp is closer to &#x2113;<sub>0</sub>-a direct measure of sparsity. This paper develops algorithms for constrained minimization of the total p-variation (TpV), lp of the image gradient. Use of the algorithms is illustrated in the context of breast CT-an imaging modality that is still in the research phase and for which constraints on X-ray dose are extremely tight. The TpV-based image reconstruction algorithms are demonstrated on computer simulated data for exploiting gradient magnitude sparsity to reduce the projection view angle sampling. The proposed algorithms are applied to projection data from a realistic breast CT simulation, where the total X-ray dose is equivalent to two-view digital mammography. Following the simulation survey, the algorithms are then demonstrated on a clinical breast CT data set.]]></abstract>
##     <issn><![CDATA[2168-2372]]></issn>
##     <htmlFlag><![CDATA[1]]></htmlFlag>
##     <arnumber><![CDATA[6714374]]></arnumber>
##     <doi><![CDATA[10.1109/JTEHM.2014.2300862]]></doi>
##     <publicationId><![CDATA[6714374]]></publicationId>
##     <mdurl><![CDATA[http://ieeexplore.ieee.org/xpl/articleDetails.jsp?tp=&arnumber=6714374&contentType=Journals+%26+Magazines]]></mdurl>
##     <pdf><![CDATA[http://ieeexplore.ieee.org/stamp/stamp.jsp?arnumber=6714374]]></pdf>
##   </document>
##   <document>
##     <rank>818</rank>
##     <title><![CDATA[Textile Antennas as Hybrid Energy-Harvesting Platforms]]></title>
##     <authors><![CDATA[Lemey, S.;  Declercq, F.;  Rogier, H.]]></authors>
##     <affiliations><![CDATA[Dept. of Inf. Technol., Ghent Univ., Ghent, Belgium]]></affiliations>
##     <controlledterms>
##       <term><![CDATA[antenna arrays]]></term>
##       <term><![CDATA[biomedical communication]]></term>
##       <term><![CDATA[clothing]]></term>
##       <term><![CDATA[energy harvesting]]></term>
##       <term><![CDATA[wearable antennas]]></term>
##     </controlledterms>
##     <thesaurusterms>
##       <term><![CDATA[Antennas]]></term>
##       <term><![CDATA[Batteries]]></term>
##       <term><![CDATA[Energy harvesting]]></term>
##       <term><![CDATA[Energy management]]></term>
##       <term><![CDATA[Photovoltaic cells]]></term>
##       <term><![CDATA[Radio frequency]]></term>
##       <term><![CDATA[Renewable energy sources]]></term>
##       <term><![CDATA[Sensors]]></term>
##       <term><![CDATA[Textiles]]></term>
##       <term><![CDATA[Wearable computers]]></term>
##     </thesaurusterms>
##     <pubtitle><![CDATA[Proceedings of the IEEE]]></pubtitle>
##     <punumber><![CDATA[5]]></punumber>
##     <pubtype><![CDATA[Journals & Magazines]]></pubtype>
##     <publisher><![CDATA[IEEE]]></publisher>
##     <volume><![CDATA[102]]></volume>
##     <issue><![CDATA[11]]></issue>
##     <py><![CDATA[2014]]></py>
##     <spage><![CDATA[1833]]></spage>
##     <epage><![CDATA[1857]]></epage>
##     <abstract><![CDATA[Smart-fabric interactive-textile systems offer exciting new possibilities, provided that they exhibit sufficient robustness and autonomy to be reliably deployed in critical applications. Textile multiantenna systems, unobtrusively integrated in a professional garment, are key components of such systems, as they set up energy-efficient and stable wireless body-centric communication links. Yet, their functionality may be further extended by exploiting their surface as energy-harvesting platform. Different state-of-the-art energy harvesters are suitable for compact integration onto a textile antenna. We demonstrate this by integrating a power management system, together with multiple diverse scavenging transducers and a storage module, on a well-chosen textile antenna topology. We provide guidelines to ensure that the additional hardware does not affect the textile antenna's performance. Simultaneous scavenging from different energy sources significantly increases the autonomy of a wearable system, in the meanwhile reducing battery size.]]></abstract>
##     <issn><![CDATA[0018-9219]]></issn>
##     <htmlFlag><![CDATA[1]]></htmlFlag>
##     <arnumber><![CDATA[6918390]]></arnumber>
##     <doi><![CDATA[10.1109/JPROC.2014.2355872]]></doi>
##     <publicationId><![CDATA[6918390]]></publicationId>
##     <mdurl><![CDATA[http://ieeexplore.ieee.org/xpl/articleDetails.jsp?tp=&arnumber=6918390&contentType=Journals+%26+Magazines]]></mdurl>
##     <pdf><![CDATA[http://ieeexplore.ieee.org/stamp/stamp.jsp?arnumber=6918390]]></pdf>
##   </document>
##   <document>
##     <rank>819</rank>
##     <title><![CDATA[A 1024<formula formulatype="inline"> <img src="/images/tex/353.gif" alt=",\times,"> </formula>8, 700-ps Time-Gated SPAD Line Sensor for Planetary Surface Exploration With Laser Raman Spectroscopy and LIBS]]></title>
##     <authors><![CDATA[Maruyama, Y.;  Blacksberg, J.;  Charbon, E.]]></authors>
##     <affiliations><![CDATA[Delft Univ. of Technol., Delft, Netherlands]]></affiliations>
##     <controlledterms>
##       <term><![CDATA[Raman spectra]]></term>
##       <term><![CDATA[astronomical instruments]]></term>
##       <term><![CDATA[astronomical techniques]]></term>
##       <term><![CDATA[avalanche diodes]]></term>
##       <term><![CDATA[planetary surfaces]]></term>
##     </controlledterms>
##     <thesaurusterms>
##       <term><![CDATA[CMOS integrated circuits]]></term>
##       <term><![CDATA[Imaging]]></term>
##       <term><![CDATA[Lasers]]></term>
##       <term><![CDATA[Logic gates]]></term>
##       <term><![CDATA[Optical sensors]]></term>
##       <term><![CDATA[Photonics]]></term>
##       <term><![CDATA[Raman scattering]]></term>
##     </thesaurusterms>
##     <pubtitle><![CDATA[Solid-State Circuits, IEEE Journal of]]></pubtitle>
##     <punumber><![CDATA[4]]></punumber>
##     <pubtype><![CDATA[Journals & Magazines]]></pubtype>
##     <publisher><![CDATA[IEEE]]></publisher>
##     <volume><![CDATA[49]]></volume>
##     <issue><![CDATA[1]]></issue>
##     <py><![CDATA[2014]]></py>
##     <spage><![CDATA[179]]></spage>
##     <epage><![CDATA[189]]></epage>
##     <abstract><![CDATA[A 1024 &#x00D7; 8 time-gated, single-photon avalanche diode line sensor is presented for time-resolved laser Raman spectroscopy and laser-induced breakdown spectroscopy. Two different chip geometries were implemented and characterized. A type-I sensor has a maximum photon detection efficiency of 0.3% and median dark count rate of 80 Hz at 3 V of excess bias. A type-II sensor offers a maximum photon detection efficiency of 19.3% and a median dark count rate of 5.7 kHz at 3 V of excess bias. Both chips have 250-ps temporal resolution and fast gating capability, with a minimum gate width of 1.8 ns for type I and 0.7 ns for type II. Raman spectra were successfully observed from natural minerals, such as calcite and willemite. With the use of subnanosecond gating, background fluorescence was significantly reduced.]]></abstract>
##     <issn><![CDATA[0018-9200]]></issn>
##     <htmlFlag><![CDATA[1]]></htmlFlag>
##     <arnumber><![CDATA[6619445]]></arnumber>
##     <doi><![CDATA[10.1109/JSSC.2013.2282091]]></doi>
##     <publicationId><![CDATA[6619445]]></publicationId>
##     <mdurl><![CDATA[http://ieeexplore.ieee.org/xpl/articleDetails.jsp?tp=&arnumber=6619445&contentType=Journals+%26+Magazines]]></mdurl>
##     <pdf><![CDATA[http://ieeexplore.ieee.org/stamp/stamp.jsp?arnumber=6619445]]></pdf>
##   </document>
##   <document>
##     <rank>820</rank>
##     <title><![CDATA[Consonance of Vibrotactile Chords]]></title>
##     <authors><![CDATA[Yongjae Yoo;  Inwook Hwang;  Seungmoon Choi]]></authors>
##     <affiliations><![CDATA[Dept. of Comput. Sci. & Eng., Pohang Univ. of Sci. & Technol., Pohang, South Korea]]></affiliations>
##     <controlledterms>
##       <term><![CDATA[electroactive polymer actuators]]></term>
##       <term><![CDATA[haptic interfaces]]></term>
##       <term><![CDATA[piezoelectric actuators]]></term>
##     </controlledterms>
##     <thesaurusterms>
##       <term><![CDATA[Actuators]]></term>
##       <term><![CDATA[Correlation]]></term>
##       <term><![CDATA[Haptic interfaces]]></term>
##       <term><![CDATA[Histograms]]></term>
##       <term><![CDATA[IEEE transactions]]></term>
##       <term><![CDATA[Mobile handsets]]></term>
##       <term><![CDATA[Vibrations]]></term>
##     </thesaurusterms>
##     <pubtitle><![CDATA[Haptics, IEEE Transactions on]]></pubtitle>
##     <punumber><![CDATA[4543165]]></punumber>
##     <pubtype><![CDATA[Journals & Magazines]]></pubtype>
##     <publisher><![CDATA[IEEE]]></publisher>
##     <volume><![CDATA[7]]></volume>
##     <issue><![CDATA[1]]></issue>
##     <py><![CDATA[2014]]></py>
##     <spage><![CDATA[3]]></spage>
##     <epage><![CDATA[13]]></epage>
##     <abstract><![CDATA[This paper is concerned with the perception of complex vibrotactile stimuli in which a few sinusoidal vibrations with different frequencies are superimposed. We begin with an observation that such vibrotactile signals are analogous to musical chords in which multiple notes are played simultaneously. A set of so-called &#x201C;vibrotactile chords&#x201D; are designed on the basis of musical chords, and their degrees of consonance (harmony) that participants perceive are evaluated through a perceptual experiment. Experimental results indicate that participants can reliably rate the degrees of consonance of vibrotactile chords and establish a well-defined function that relates the degree of consonance to the base and chordal frequency of a vibrotactile chord. These findings have direct implications for the design of complex vibrotactile signals that can be produced by current wideband actuators such as voice-coil, piezoelectric, and electroactive polymer actuators.]]></abstract>
##     <issn><![CDATA[1939-1412]]></issn>
##     <htmlFlag><![CDATA[1]]></htmlFlag>
##     <arnumber><![CDATA[6671586]]></arnumber>
##     <doi><![CDATA[10.1109/TOH.2013.57]]></doi>
##     <publicationId><![CDATA[6671586]]></publicationId>
##     <mdurl><![CDATA[http://ieeexplore.ieee.org/xpl/articleDetails.jsp?tp=&arnumber=6671586&contentType=Journals+%26+Magazines]]></mdurl>
##     <pdf><![CDATA[http://ieeexplore.ieee.org/stamp/stamp.jsp?arnumber=6671586]]></pdf>
##   </document>
##   <document>
##     <rank>821</rank>
##     <title><![CDATA[New Analytical Model for Nanoscale Tri-Gate SOI MOSFETs Including Quantum Effects]]></title>
##     <authors><![CDATA[Vimala, P.;  Balamurugan, N.B.]]></authors>
##     <affiliations><![CDATA[Dept. of Electron. & Commun. Eng., Thiagarajar Coll. of Eng., Madurai, India]]></affiliations>
##     <controlledterms>
##       <term><![CDATA[MOSFET]]></term>
##       <term><![CDATA[Poisson equation]]></term>
##       <term><![CDATA[Schrodinger equation]]></term>
##       <term><![CDATA[capacitance]]></term>
##       <term><![CDATA[semiconductor device models]]></term>
##       <term><![CDATA[silicon-on-insulator]]></term>
##       <term><![CDATA[variational techniques]]></term>
##       <term><![CDATA[wave functions]]></term>
##     </controlledterms>
##     <thesaurusterms>
##       <term><![CDATA[Analytical models]]></term>
##       <term><![CDATA[Equations]]></term>
##       <term><![CDATA[Logic gates]]></term>
##       <term><![CDATA[MOSFET]]></term>
##       <term><![CDATA[Mathematical model]]></term>
##       <term><![CDATA[Semiconductor device modeling]]></term>
##       <term><![CDATA[Silicon]]></term>
##     </thesaurusterms>
##     <pubtitle><![CDATA[Electron Devices Society, IEEE Journal of the]]></pubtitle>
##     <punumber><![CDATA[6245494]]></punumber>
##     <pubtype><![CDATA[Journals & Magazines]]></pubtype>
##     <publisher><![CDATA[IEEE]]></publisher>
##     <volume><![CDATA[2]]></volume>
##     <issue><![CDATA[1]]></issue>
##     <py><![CDATA[2014]]></py>
##     <spage><![CDATA[1]]></spage>
##     <epage><![CDATA[7]]></epage>
##     <abstract><![CDATA[In this paper, an analytical model for tri-Gate (TG) MOSFETs considering quantum effects is presented. The proposed model is based on the analytical solution of Schrodinger-Poisson's equation using variational approach. An analytical expression of the inversion charge distribution function (ICDF) or wave function for the TG MOSFETs has been developed. This obtained ICDF is used to calculate the device parameters, such as the inversion charge centroid, threshold voltage, inversion charge, gate capacitance, and drain current. These parameters are modeled for various device dimensions and applied bias. The results are validated against the TCAD simulation results.]]></abstract>
##     <issn><![CDATA[2168-6734]]></issn>
##     <htmlFlag><![CDATA[1]]></htmlFlag>
##     <arnumber><![CDATA[6705594]]></arnumber>
##     <doi><![CDATA[10.1109/JEDS.2014.2298915]]></doi>
##     <publicationId><![CDATA[6705594]]></publicationId>
##     <mdurl><![CDATA[http://ieeexplore.ieee.org/xpl/articleDetails.jsp?tp=&arnumber=6705594&contentType=Journals+%26+Magazines]]></mdurl>
##     <pdf><![CDATA[http://ieeexplore.ieee.org/stamp/stamp.jsp?arnumber=6705594]]></pdf>
##   </document>
##   <document>
##     <rank>822</rank>
##     <title><![CDATA[Reconfigurable UWB Pulse Generation Based on a Dual-Drive Mach&#x2013;Zehnder Modulator]]></title>
##     <authors><![CDATA[Pan Cao;  Xiaofeng Hu;  Jiayang Wu;  Liang Zhang;  Xinhong Jiang;  Yikai Su]]></authors>
##     <affiliations><![CDATA[Dept. of Electron. Eng., Shanghai Jiao Tong Univ., Shanghai, China]]></affiliations>
##     <controlledterms>
##       <term><![CDATA[microwave photonics]]></term>
##       <term><![CDATA[optical modulation]]></term>
##       <term><![CDATA[optical pulse generation]]></term>
##       <term><![CDATA[phase modulation]]></term>
##       <term><![CDATA[radio-over-fibre]]></term>
##       <term><![CDATA[ultra wideband communication]]></term>
##     </controlledterms>
##     <thesaurusterms>
##       <term><![CDATA[Bandwidth]]></term>
##       <term><![CDATA[Optical filters]]></term>
##       <term><![CDATA[Optical pulse shaping]]></term>
##       <term><![CDATA[Photonics]]></term>
##       <term><![CDATA[Pulse generation]]></term>
##       <term><![CDATA[Shape]]></term>
##       <term><![CDATA[Ultra wideband technology]]></term>
##     </thesaurusterms>
##     <pubtitle><![CDATA[Photonics Journal, IEEE]]></pubtitle>
##     <punumber><![CDATA[4563994]]></punumber>
##     <pubtype><![CDATA[Journals & Magazines]]></pubtype>
##     <publisher><![CDATA[IEEE]]></publisher>
##     <volume><![CDATA[6]]></volume>
##     <issue><![CDATA[5]]></issue>
##     <py><![CDATA[2014]]></py>
##     <spage><![CDATA[1]]></spage>
##     <epage><![CDATA[6]]></epage>
##     <abstract><![CDATA[We propose and experimentally demonstrate a reconfigurable ultrawideband (UWB) pulse generation scheme by using a dual-drive Mach-Zehnder modulator (DDMZM). In the proposed method, two phase modulators of the DDMZM are respectively driven by properly designed electrical signals. By carefully setting the parameters of the two electrical signals and the bias voltage of the DDMZM, different UWB pulses can be achieved at the interference output of the DDMZM. A proof-of-concept experiment verifies the feasibility of the proposed method. Photonic generations of UWB monocycle, doublet, triplet, and quintuple pulses have been successfully demonstrated, with central frequencies of 4.69, 4.69, 5.47, and 6.25 GHz and fractional bandwidths of 150%, 133%, 186%, and 75%, respectively.]]></abstract>
##     <issn><![CDATA[1943-0655]]></issn>
##     <htmlFlag><![CDATA[1]]></htmlFlag>
##     <arnumber><![CDATA[6908997]]></arnumber>
##     <doi><![CDATA[10.1109/JPHOT.2014.2352632]]></doi>
##     <publicationId><![CDATA[6908997]]></publicationId>
##     <mdurl><![CDATA[http://ieeexplore.ieee.org/xpl/articleDetails.jsp?tp=&arnumber=6908997&contentType=Journals+%26+Magazines]]></mdurl>
##     <pdf><![CDATA[http://ieeexplore.ieee.org/stamp/stamp.jsp?arnumber=6908997]]></pdf>
##   </document>
##   <document>
##     <rank>823</rank>
##     <title><![CDATA[A Survey of Clustering Algorithms for Big Data: Taxonomy and Empirical Analysis]]></title>
##     <authors><![CDATA[Fahad, A.;  Alshatri, N.;  Tari, Z.;  Alamri, A.;  Khalil, I.;  Zomaya, A.Y.;  Foufou, S.;  Bouras, A.]]></authors>
##     <affiliations><![CDATA[Sch. of Comput. Sci. & Inf. Technol., R. Melbourne Inst. of Technol., Melbourne, VIC, Australia]]></affiliations>
##     <controlledterms>
##       <term><![CDATA[Big Data]]></term>
##       <term><![CDATA[learning (artificial intelligence)]]></term>
##       <term><![CDATA[pattern clustering]]></term>
##     </controlledterms>
##     <thesaurusterms>
##       <term><![CDATA[Algorithm design and analysis]]></term>
##       <term><![CDATA[Big data]]></term>
##       <term><![CDATA[Clustering algorithms]]></term>
##       <term><![CDATA[Clustering methods]]></term>
##       <term><![CDATA[Neural networks]]></term>
##       <term><![CDATA[Partitioning algorithms]]></term>
##       <term><![CDATA[Taxonomies]]></term>
##     </thesaurusterms>
##     <pubtitle><![CDATA[Emerging Topics in Computing, IEEE Transactions on]]></pubtitle>
##     <punumber><![CDATA[6245516]]></punumber>
##     <pubtype><![CDATA[Journals & Magazines]]></pubtype>
##     <publisher><![CDATA[IEEE]]></publisher>
##     <volume><![CDATA[2]]></volume>
##     <issue><![CDATA[3]]></issue>
##     <py><![CDATA[2014]]></py>
##     <spage><![CDATA[267]]></spage>
##     <epage><![CDATA[279]]></epage>
##     <abstract><![CDATA[Clustering algorithms have emerged as an alternative powerful meta-learning tool to accurately analyze the massive volume of data generated by modern applications. In particular, their main goal is to categorize data into clusters such that objects are grouped in the same cluster when they are similar according to specific metrics. There is a vast body of knowledge in the area of clustering and there has been attempts to analyze and categorize them for a larger number of applications. However, one of the major issues in using clustering algorithms for big data that causes confusion amongst practitioners is the lack of consensus in the definition of their properties as well as a lack of formal categorization. With the intention of alleviating these problems, this paper introduces concepts and algorithms related to clustering, a concise survey of existing (clustering) algorithms as well as providing a comparison, both from a theoretical and an empirical perspective. From a theoretical perspective, we developed a categorizing framework based on the main properties pointed out in previous studies. Empirically, we conducted extensive experiments where we compared the most representative algorithm from each of the categories using a large number of real (big) data sets. The effectiveness of the candidate clustering algorithms is measured through a number of internal and external validity metrics, stability, runtime, and scalability tests. In addition, we highlighted the set of clustering algorithms that are the best performing for big data.]]></abstract>
##     <issn><![CDATA[2168-6750]]></issn>
##     <htmlFlag><![CDATA[1]]></htmlFlag>
##     <arnumber><![CDATA[6832486]]></arnumber>
##     <doi><![CDATA[10.1109/TETC.2014.2330519]]></doi>
##     <publicationId><![CDATA[6832486]]></publicationId>
##     <mdurl><![CDATA[http://ieeexplore.ieee.org/xpl/articleDetails.jsp?tp=&arnumber=6832486&contentType=Journals+%26+Magazines]]></mdurl>
##     <pdf><![CDATA[http://ieeexplore.ieee.org/stamp/stamp.jsp?arnumber=6832486]]></pdf>
##   </document>
##   <document>
##     <rank>824</rank>
##     <title><![CDATA[Q-Switching and Mode-Locking in Highly Doped Zr<formula formulatype="inline"> <img src="/images/tex/517.gif" alt="_{2}"> </formula>O<formula formulatype="inline"> <img src="/images/tex/602.gif" alt="_{3}"> </formula>&#x2013;Al<formula formulatype="inline"> <img src="/images/tex/21142.gif" alt="_{2}"> </formula>O<formula formulatype="inline"> <img src="/images/tex/602.gif" alt="_{3}"> </formula>&#x2013;Er<formula formulatype="inline"> <img src="/images/tex/21117.gif" alt="_{2}"> </formula>O<formula formulatype="inline"> <img src="/images/tex/602.gif" alt="_{3}"> </formula>-Doped Fiber Lasers Using Graphene as a Saturable Absorber]]></title>
##     <authors><![CDATA[Ahmad, H.;  Thambiratnam, K.;  Muhammad, F.D.;  Zulkifli, M.Z.;  Zulkifli, A.Z.;  Paul, M.C.;  Harun, S.W.]]></authors>
##     <affiliations><![CDATA[Photonics Res. Centre, Univ. of Malaya, Kuala Lumpur, Malaysia]]></affiliations>
##     <controlledterms>
##       <term><![CDATA[Q-switching]]></term>
##       <term><![CDATA[aluminium compounds]]></term>
##       <term><![CDATA[erbium compounds]]></term>
##       <term><![CDATA[fibre lasers]]></term>
##       <term><![CDATA[graphene]]></term>
##       <term><![CDATA[laser mode locking]]></term>
##       <term><![CDATA[optical pulse generation]]></term>
##       <term><![CDATA[optical saturable absorption]]></term>
##       <term><![CDATA[zirconium compounds]]></term>
##     </controlledterms>
##     <thesaurusterms>
##       <term><![CDATA[Graphene]]></term>
##       <term><![CDATA[Laser excitation]]></term>
##       <term><![CDATA[Laser mode locking]]></term>
##       <term><![CDATA[Optical fiber amplifiers]]></term>
##       <term><![CDATA[Power lasers]]></term>
##       <term><![CDATA[Pump lasers]]></term>
##     </thesaurusterms>
##     <pubtitle><![CDATA[Selected Topics in Quantum Electronics, IEEE Journal of]]></pubtitle>
##     <punumber><![CDATA[2944]]></punumber>
##     <pubtype><![CDATA[Journals & Magazines]]></pubtype>
##     <publisher><![CDATA[IEEE]]></publisher>
##     <volume><![CDATA[20]]></volume>
##     <issue><![CDATA[1]]></issue>
##     <py><![CDATA[2014]]></py>
##     <spage><![CDATA[9]]></spage>
##     <epage><![CDATA[16]]></epage>
##     <abstract><![CDATA[The application of graphene as a saturable absorber (SA) for generating Q-switched and mode-locked pulses in a Zirconia-Erbium-doped fiber (Zr-EDF) laser is explored. Graphene-based SAs have a very wide operational range, which complements the extended operational bandwidth of the Zr-EDF. The Zr-EDF has an erbium concentration of about 4320 ppm, with absorption rates of 22.0 and 58.0 dB/m at 987 and 1550 nm. The system is capable of generating Q-switched pulses with pulsewidths and energies of 4.6 &#x03BC;s and 16.8 nJ, respectively, as well as peak powers of 3.6 mW at a repetition rate of 50.1 kHz. The Zr-EDF laser can also generate mode-locked pulses with pulsewidths, average output powers, pulse energies, and peak powers of 730 fs, 1.6 mW, 23.1 pJ, and 31.6 W, respectively, at a repetition rate of 69.3 MHz. Both the Q-switched and mode-locked output pulses are highly stable, allowing for their application in a multitude of real-world applications.]]></abstract>
##     <issn><![CDATA[1077-260X]]></issn>
##     <htmlFlag><![CDATA[1]]></htmlFlag>
##     <arnumber><![CDATA[6578105]]></arnumber>
##     <doi><![CDATA[10.1109/JSTQE.2013.2272459]]></doi>
##     <publicationId><![CDATA[6578105]]></publicationId>
##     <mdurl><![CDATA[http://ieeexplore.ieee.org/xpl/articleDetails.jsp?tp=&arnumber=6578105&contentType=Journals+%26+Magazines]]></mdurl>
##     <pdf><![CDATA[http://ieeexplore.ieee.org/stamp/stamp.jsp?arnumber=6578105]]></pdf>
##   </document>
##   <document>
##     <rank>825</rank>
##     <title><![CDATA[Essential Factor for Determining Optical Output of Phosphor-Converted LEDs]]></title>
##     <authors><![CDATA[Tsung-Hsun Yang;  Cheng-Chien Chen;  Ching-Yi Chen;  Yu-Yu Chang;  Ching-Cherng Sun]]></authors>
##     <affiliations><![CDATA[Dept. of Opt. & Photonics, Nat. Central Univ., Chungli, Taiwan]]></affiliations>
##     <controlledterms>
##       <term><![CDATA[colour]]></term>
##       <term><![CDATA[light emitting diodes]]></term>
##       <term><![CDATA[phosphors]]></term>
##     </controlledterms>
##     <thesaurusterms>
##       <term><![CDATA[Color]]></term>
##       <term><![CDATA[Licenses]]></term>
##       <term><![CDATA[Light emitting diodes]]></term>
##       <term><![CDATA[Optical mixing]]></term>
##       <term><![CDATA[Packaging]]></term>
##       <term><![CDATA[Phosphors]]></term>
##       <term><![CDATA[Resins]]></term>
##     </thesaurusterms>
##     <pubtitle><![CDATA[Photonics Journal, IEEE]]></pubtitle>
##     <punumber><![CDATA[4563994]]></punumber>
##     <pubtype><![CDATA[Journals & Magazines]]></pubtype>
##     <publisher><![CDATA[IEEE]]></publisher>
##     <volume><![CDATA[6]]></volume>
##     <issue><![CDATA[2]]></issue>
##     <py><![CDATA[2014]]></py>
##     <spage><![CDATA[1]]></spage>
##     <epage><![CDATA[9]]></epage>
##     <abstract><![CDATA[An essential factor of the particle number is exploited for the phosphor excitation in phosphor-converted white LEDs. The particle number can clearly reveal the dependence of the light output flux and the correlated color temperature upon the conventional parameters, thickness, and concentration of the phosphors in a simpler way. In addition, we also find that there might exist an optimal particle number for the maximal luminous light output. An empirical function is then proposed for successfully modeling the relation between the output light and the particle number.]]></abstract>
##     <issn><![CDATA[1943-0655]]></issn>
##     <htmlFlag><![CDATA[1]]></htmlFlag>
##     <arnumber><![CDATA[6748849]]></arnumber>
##     <doi><![CDATA[10.1109/JPHOT.2014.2308630]]></doi>
##     <publicationId><![CDATA[6748849]]></publicationId>
##     <mdurl><![CDATA[http://ieeexplore.ieee.org/xpl/articleDetails.jsp?tp=&arnumber=6748849&contentType=Journals+%26+Magazines]]></mdurl>
##     <pdf><![CDATA[http://ieeexplore.ieee.org/stamp/stamp.jsp?arnumber=6748849]]></pdf>
##   </document>
##   <document>
##     <rank>826</rank>
##     <title><![CDATA[Analysis of Hysteresis Width on Optical Bistability for the Realization of Optical SR Flip-Flop Using SMFP-LDs With Simultaneous Inverted and Non-Inverted Outputs]]></title>
##     <authors><![CDATA[Nakarmi, B.;  Hoai, T.Q.;  Won, Y.H.;  Xuping Zhang]]></authors>
##     <affiliations><![CDATA[Inst. of Opt. Commun. Eng., Nanjing Univ., Nanjing, China]]></affiliations>
##     <controlledterms>
##       <term><![CDATA[error statistics]]></term>
##       <term><![CDATA[extinction coefficients]]></term>
##       <term><![CDATA[flip-flops]]></term>
##       <term><![CDATA[hysteresis]]></term>
##       <term><![CDATA[laser modes]]></term>
##       <term><![CDATA[optical bistability]]></term>
##       <term><![CDATA[optical logic]]></term>
##       <term><![CDATA[optical tuning]]></term>
##       <term><![CDATA[semiconductor lasers]]></term>
##     </controlledterms>
##     <thesaurusterms>
##       <term><![CDATA[Hysteresis]]></term>
##       <term><![CDATA[Injection-locked oscillators]]></term>
##       <term><![CDATA[Laser beams]]></term>
##       <term><![CDATA[Optical bistability]]></term>
##       <term><![CDATA[Optical pulses]]></term>
##       <term><![CDATA[Optical signal processing]]></term>
##       <term><![CDATA[Particle beam injection]]></term>
##     </thesaurusterms>
##     <pubtitle><![CDATA[Photonics Journal, IEEE]]></pubtitle>
##     <punumber><![CDATA[4563994]]></punumber>
##     <pubtype><![CDATA[Journals & Magazines]]></pubtype>
##     <publisher><![CDATA[IEEE]]></publisher>
##     <volume><![CDATA[6]]></volume>
##     <issue><![CDATA[3]]></issue>
##     <py><![CDATA[2014]]></py>
##     <spage><![CDATA[1]]></spage>
##     <epage><![CDATA[12]]></epage>
##     <abstract><![CDATA[The width of hysteresis in optical bistability plays an important role on latching operation. The injected input optical power and wavelength detuning in single mode Fabry-Perot laser diode (SMFP-LD) are two parameters that can impact the width of hysteresis in bistability of dominant mode and injected mode. In this paper, we analyze the effect of wavelength detuning on width of the hysteresis loop, the dynamic power range for optical bistability, and the rising-falling time of output waveform. With this analysis, optical Set Reset flip-flop with simultaneous inverted and non-inverted output at the data rate of 8.5 Gb/s are obtained using SMFP-LDs. The principle is based on the optical bistability phenomena during injection-locking of the Fabry-Perot laser diode. The set and reset pulses are maintained in low power of about -14 and -6 dBm for the power effective configuration. The contrast ratios of more than 20 dB with simultaneous non-inverted and inverted output Q and Q are observed. Clear output spectrum domain results, waveforms, eye diagram with extinction ratio of more than 12 dB, a rising/falling time of about 40 ps, and bit error rate (BER) measurement without noise floor up to 10 <sup>-12</sup> are observed. The maximum power penalty is about 1.8 dB at BER of 10 <sup>-9</sup>.]]></abstract>
##     <issn><![CDATA[1943-0655]]></issn>
##     <htmlFlag><![CDATA[1]]></htmlFlag>
##     <arnumber><![CDATA[6814835]]></arnumber>
##     <doi><![CDATA[10.1109/JPHOT.2014.2323303]]></doi>
##     <publicationId><![CDATA[6814835]]></publicationId>
##     <mdurl><![CDATA[http://ieeexplore.ieee.org/xpl/articleDetails.jsp?tp=&arnumber=6814835&contentType=Journals+%26+Magazines]]></mdurl>
##     <pdf><![CDATA[http://ieeexplore.ieee.org/stamp/stamp.jsp?arnumber=6814835]]></pdf>
##   </document>
##   <document>
##     <rank>827</rank>
##     <title><![CDATA[Low Noise Nanometer Scale Room-Temperature <formula formulatype="inline"> <img src="/images/tex/21507.gif" alt="{\hbox {YBa}}_{2}{\hbox {Cu}}_{3}{\hbox {O}}_{7-x}"> </formula> Bolometers for THz Direct Detection]]></title>
##     <authors><![CDATA[Bevilacqua, S.;  Cherednichenko, S.]]></authors>
##     <affiliations><![CDATA[Dept. of Microtechnol. & Nanosci. (MC2), Chalmers Univ. of Technol., Goteborg, Sweden]]></affiliations>
##     <controlledterms>
##       <term><![CDATA[1/f noise]]></term>
##       <term><![CDATA[barium compounds]]></term>
##       <term><![CDATA[bolometers]]></term>
##       <term><![CDATA[high-temperature superconductors]]></term>
##       <term><![CDATA[planar antennas]]></term>
##       <term><![CDATA[spiral antennas]]></term>
##       <term><![CDATA[superconducting photodetectors]]></term>
##       <term><![CDATA[superconducting thin films]]></term>
##       <term><![CDATA[terahertz wave detectors]]></term>
##       <term><![CDATA[yttrium compounds]]></term>
##     </controlledterms>
##     <thesaurusterms>
##       <term><![CDATA[Bolometers]]></term>
##       <term><![CDATA[Detectors]]></term>
##       <term><![CDATA[Noise]]></term>
##       <term><![CDATA[Optical variables measurement]]></term>
##       <term><![CDATA[Resistance]]></term>
##       <term><![CDATA[Temperature measurement]]></term>
##       <term><![CDATA[Yttrium barium copper oxide]]></term>
##     </thesaurusterms>
##     <pubtitle><![CDATA[Terahertz Science and Technology, IEEE Transactions on]]></pubtitle>
##     <punumber><![CDATA[5503871]]></punumber>
##     <pubtype><![CDATA[Journals & Magazines]]></pubtype>
##     <publisher><![CDATA[IEEE]]></publisher>
##     <volume><![CDATA[4]]></volume>
##     <issue><![CDATA[6]]></issue>
##     <py><![CDATA[2014]]></py>
##     <spage><![CDATA[653]]></spage>
##     <epage><![CDATA[660]]></epage>
##     <abstract><![CDATA[We present a detailed investigation of the responsivity and the noise in room temperature THz direct detectors made of YBa<sub>2</sub>Cu<sub>3</sub>O<sub>7-x</sub> (YBCO) thin-film nano-bolometers. The YBCO nano-bolometers are integrated with planar spiral antennas covering a frequency range from 100 GHz to 2 THz. The detectors were characterized at 1.6 THz, 0.7 THz, 400 GHz and 100 GHz. The maximum electrical responsivity of 70 V/W and a minimum noise equivalent power (NEP) of 50 pW/Hz<sup>0.5</sup> were measured, whereas the highest optical responsivity was 45 V/W. The (1/f) noise in nano-bolometers is independent on the device volume and can be found as (V<sub>N</sub>/V)<sup>2</sup>=6&#x00D7;10<sup>-11</sup>&#x00D7;1/f Hz<sup>-1</sup> for a given modulation frequency f and a dc voltage V.]]></abstract>
##     <issn><![CDATA[2156-342X]]></issn>
##     <htmlFlag><![CDATA[1]]></htmlFlag>
##     <arnumber><![CDATA[6872609]]></arnumber>
##     <doi><![CDATA[10.1109/TTHZ.2014.2344435]]></doi>
##     <publicationId><![CDATA[6872609]]></publicationId>
##     <mdurl><![CDATA[http://ieeexplore.ieee.org/xpl/articleDetails.jsp?tp=&arnumber=6872609&contentType=Journals+%26+Magazines]]></mdurl>
##     <pdf><![CDATA[http://ieeexplore.ieee.org/stamp/stamp.jsp?arnumber=6872609]]></pdf>
##   </document>
##   <document>
##     <rank>828</rank>
##     <title><![CDATA[S-Band Gain Improvement Using a Thulium&#x2013;Aluminum Co-Doped Photonic Crystal Fiber Amplifier]]></title>
##     <authors><![CDATA[Emami, S.D.;  Muhammad, A.R.;  Muhamad-Yasin, S.Z.;  Mat-Sharif, K.A.;  Zulkifli, M.I.;  Adikan, F.R.M.;  Ahmad, H.;  Abdul-Rashid, H.A.]]></authors>
##     <affiliations><![CDATA[Fac. of Eng., Multimedia Univ., Cyberjaya, Malaysia]]></affiliations>
##     <controlledterms>
##       <term><![CDATA[aluminium]]></term>
##       <term><![CDATA[holey fibres]]></term>
##       <term><![CDATA[laser noise]]></term>
##       <term><![CDATA[optical fibre amplifiers]]></term>
##       <term><![CDATA[optical fibre losses]]></term>
##       <term><![CDATA[photonic crystals]]></term>
##       <term><![CDATA[population inversion]]></term>
##       <term><![CDATA[superradiance]]></term>
##       <term><![CDATA[thulium]]></term>
##     </controlledterms>
##     <thesaurusterms>
##       <term><![CDATA[Band-pass filters]]></term>
##       <term><![CDATA[Doping]]></term>
##       <term><![CDATA[Gain]]></term>
##       <term><![CDATA[Indexes]]></term>
##       <term><![CDATA[Optical fiber amplifiers]]></term>
##       <term><![CDATA[Photonic crystal fibers]]></term>
##       <term><![CDATA[Refractive index]]></term>
##     </thesaurusterms>
##     <pubtitle><![CDATA[Photonics Journal, IEEE]]></pubtitle>
##     <punumber><![CDATA[4563994]]></punumber>
##     <pubtype><![CDATA[Journals & Magazines]]></pubtype>
##     <publisher><![CDATA[IEEE]]></publisher>
##     <volume><![CDATA[6]]></volume>
##     <issue><![CDATA[6]]></issue>
##     <py><![CDATA[2014]]></py>
##     <spage><![CDATA[1]]></spage>
##     <epage><![CDATA[10]]></epage>
##     <abstract><![CDATA[An extended method for gain and noise figure enhancement in the S-band using a thulium-doped photonic crystal fiber amplifier (TD-PCFA) is proposed and shown by numerical simulation. The principle behind the enhancement is the suppression of unwanted amplified spontaneous emission (ASE) using the PCF structure. This proposed PCF achieves the intended band-pass by doping the cladding with high index material and realizes appropriate short and long cut-off wavelengths by enlarging the air-holes surrounding the doped core region. The PCF geometrical structure is optimized so that high losses occur below the short cut-off wavelength (800 nm) and beyond the long cut-off wavelength (1750 nm). Furthermore, the PCF geometrical structure design allows for high ASE suppression at 800- and 1800-nm band, thus increasing the population inversion needed for amplification in S-band region as the 1050-nm pump propagates light in the band-pass. The proposed TD-PCFA demonstrates gain enhancements of 3-6 dB between 1420 and 1470 nm.]]></abstract>
##     <issn><![CDATA[1943-0655]]></issn>
##     <htmlFlag><![CDATA[1]]></htmlFlag>
##     <arnumber><![CDATA[6963583]]></arnumber>
##     <doi><![CDATA[10.1109/JPHOT.2014.2366158]]></doi>
##     <publicationId><![CDATA[6963583]]></publicationId>
##     <mdurl><![CDATA[http://ieeexplore.ieee.org/xpl/articleDetails.jsp?tp=&arnumber=6963583&contentType=Journals+%26+Magazines]]></mdurl>
##     <pdf><![CDATA[http://ieeexplore.ieee.org/stamp/stamp.jsp?arnumber=6963583]]></pdf>
##   </document>
##   <document>
##     <rank>829</rank>
##     <title><![CDATA[Some Fundamental Cybersecurity Concepts]]></title>
##     <authors><![CDATA[Wilson, K.S.;  Kiy, M.A.]]></authors>
##     <affiliations><![CDATA[Patent Litigation, BlackBerry, TX, USA]]></affiliations>
##     <controlledterms>
##       <term><![CDATA[data protection]]></term>
##     </controlledterms>
##     <thesaurusterms>
##       <term><![CDATA[Computer hacking]]></term>
##       <term><![CDATA[Computer security]]></term>
##       <term><![CDATA[Security of data]]></term>
##       <term><![CDATA[Statistical analysis]]></term>
##     </thesaurusterms>
##     <pubtitle><![CDATA[Access, IEEE]]></pubtitle>
##     <punumber><![CDATA[6287639]]></punumber>
##     <pubtype><![CDATA[Journals & Magazines]]></pubtype>
##     <publisher><![CDATA[IEEE]]></publisher>
##     <volume><![CDATA[2]]></volume>
##     <py><![CDATA[2014]]></py>
##     <spage><![CDATA[116]]></spage>
##     <epage><![CDATA[124]]></epage>
##     <abstract><![CDATA[The results of successful hacking attacks against commercially available cybersecurity protection tools that had been touted as secure are distilled into a set of concepts that are applicable to many protection planning scenarios. The concepts, which explain why trust in those systems was misplaced, provides a framework for both analyzing known exploits and also evaluating proposed protection systems for predicting likely potential vulnerabilities. The concepts are: 1) differentiating security threats into distinct classes; 2) a five layer model of computing systems; 3) a payload versus protection paradigm; and 4) the nine Ds of cybersecurity, which present practical defensive tactics in an easily remembered scheme. An eavesdropping risk, inherent in many smartphones and notebook computers, is described to motivate improved practices and demonstrate real-world application of the concepts to predicting new vulnerabilities. Additionally, the use of the nine Ds is demonstrated as analysis tool that permits ranking of the expected effectiveness of some potential countermeasures.]]></abstract>
##     <issn><![CDATA[2169-3536]]></issn>
##     <htmlFlag><![CDATA[1]]></htmlFlag>
##     <arnumber><![CDATA[6737236]]></arnumber>
##     <doi><![CDATA[10.1109/ACCESS.2014.2305658]]></doi>
##     <publicationId><![CDATA[6737236]]></publicationId>
##     <mdurl><![CDATA[http://ieeexplore.ieee.org/xpl/articleDetails.jsp?tp=&arnumber=6737236&contentType=Journals+%26+Magazines]]></mdurl>
##     <pdf><![CDATA[http://ieeexplore.ieee.org/stamp/stamp.jsp?arnumber=6737236]]></pdf>
##   </document>
##   <document>
##     <rank>830</rank>
##     <title><![CDATA[Dynamic MR Image Reconstruction&#x2013;Separation From Undersampled (<formula formulatype="inline"> <img src="/images/tex/21500.gif" alt="{\bf k},t"> </formula>)-Space via Low-Rank Plus Sparse Prior]]></title>
##     <authors><![CDATA[Tremoulheac, B.;  Dikaios, N.;  Atkinson, D.;  Arridge, S.R.]]></authors>
##     <affiliations><![CDATA[Centre for Med. Image Comput., Univ. Coll. London, London, UK]]></affiliations>
##     <controlledterms>
##       <term><![CDATA[biomedical MRI]]></term>
##       <term><![CDATA[cardiology]]></term>
##       <term><![CDATA[image reconstruction]]></term>
##       <term><![CDATA[inverse transforms]]></term>
##       <term><![CDATA[medical image processing]]></term>
##       <term><![CDATA[numerical analysis]]></term>
##       <term><![CDATA[optimisation]]></term>
##       <term><![CDATA[phantoms]]></term>
##       <term><![CDATA[principal component analysis]]></term>
##     </controlledterms>
##     <thesaurusterms>
##       <term><![CDATA[Image reconstruction]]></term>
##       <term><![CDATA[Licenses]]></term>
##       <term><![CDATA[Magnetic resonance imaging]]></term>
##       <term><![CDATA[Matrix decomposition]]></term>
##       <term><![CDATA[Robustness]]></term>
##       <term><![CDATA[Sparse matrices]]></term>
##     </thesaurusterms>
##     <pubtitle><![CDATA[Medical Imaging, IEEE Transactions on]]></pubtitle>
##     <punumber><![CDATA[42]]></punumber>
##     <pubtype><![CDATA[Journals & Magazines]]></pubtype>
##     <publisher><![CDATA[IEEE]]></publisher>
##     <volume><![CDATA[33]]></volume>
##     <issue><![CDATA[8]]></issue>
##     <py><![CDATA[2014]]></py>
##     <spage><![CDATA[1689]]></spage>
##     <epage><![CDATA[1701]]></epage>
##     <abstract><![CDATA[Dynamic magnetic resonance imaging (MRI) is used in multiple clinical applications, but can still benefit from higher spatial or temporal resolution. A dynamic MR image reconstruction method from partial ( k, t)-space measurements is introduced that recovers and inherently separates the information in the dynamic scene. The reconstruction model is based on a low-rank plus sparse decomposition prior, which is related to robust principal component analysis. An algorithm is proposed to solve the convex optimization problem based on an alternating direction method of multipliers. The method is validated with numerical phantom simulations and cardiac MRI data against state of the art dynamic MRI reconstruction methods. Results suggest that using the proposed approach as a means of regularizing the inverse problem remains competitive with state of the art reconstruction techniques. Additionally, the decomposition induced by the reconstruction is shown to help in the context of motion estimation in dynamic contrast enhanced MRI.]]></abstract>
##     <issn><![CDATA[0278-0062]]></issn>
##     <htmlFlag><![CDATA[1]]></htmlFlag>
##     <arnumber><![CDATA[6808502]]></arnumber>
##     <doi><![CDATA[10.1109/TMI.2014.2321190]]></doi>
##     <publicationId><![CDATA[6808502]]></publicationId>
##     <mdurl><![CDATA[http://ieeexplore.ieee.org/xpl/articleDetails.jsp?tp=&arnumber=6808502&contentType=Journals+%26+Magazines]]></mdurl>
##     <pdf><![CDATA[http://ieeexplore.ieee.org/stamp/stamp.jsp?arnumber=6808502]]></pdf>
##   </document>
##   <document>
##     <rank>831</rank>
##     <title><![CDATA[Gradient-Orientation-Based PCA Subspace for Novel Face Recognition]]></title>
##     <authors><![CDATA[Ghinea, G.;  Kannan, R.;  Kannaiyan, S.]]></authors>
##     <affiliations><![CDATA[Dept. of Comput. Sci., Brunel Univ., Uxbridge, UK]]></affiliations>
##     <controlledterms>
##       <term><![CDATA[face recognition]]></term>
##       <term><![CDATA[feature extraction]]></term>
##       <term><![CDATA[gradient methods]]></term>
##       <term><![CDATA[image classification]]></term>
##       <term><![CDATA[image matching]]></term>
##       <term><![CDATA[matrix decomposition]]></term>
##       <term><![CDATA[principal component analysis]]></term>
##     </controlledterms>
##     <thesaurusterms>
##       <term><![CDATA[Authentication]]></term>
##       <term><![CDATA[Biometrics]]></term>
##       <term><![CDATA[Cameras]]></term>
##       <term><![CDATA[Computer vision]]></term>
##       <term><![CDATA[Face recognition]]></term>
##       <term><![CDATA[Pattern recognition]]></term>
##       <term><![CDATA[Surveillance]]></term>
##     </thesaurusterms>
##     <pubtitle><![CDATA[Access, IEEE]]></pubtitle>
##     <punumber><![CDATA[6287639]]></punumber>
##     <pubtype><![CDATA[Journals & Magazines]]></pubtype>
##     <publisher><![CDATA[IEEE]]></publisher>
##     <volume><![CDATA[2]]></volume>
##     <py><![CDATA[2014]]></py>
##     <spage><![CDATA[914]]></spage>
##     <epage><![CDATA[920]]></epage>
##     <abstract><![CDATA[Face recognition is an interesting and a challenging problem that has been widely studied in the field of pattern recognition and computer vision. It has many applications such as biometric authentication, video surveillance, and others. In the past decade, several methods for face recognition were proposed. However, these methods suffer from pose and illumination variations. In order to address these problems, this paper proposes a novel methodology to recognize the face images. Since image gradients are invariant to illumination and pose variations, the proposed approach uses gradient orientation to handle these effects. The Schur decomposition is used for matrix decomposition and then Schurvalues and Schurvectors are extracted for subspace projection. We call this subspace projection of face features as Schurfaces, which is numerically stable and have the ability of handling defective matrices. The Hausdorff distance is used with the nearest neighbor classifier to measure the similarity between different faces. Experiments are conducted with Yale face database and ORL face database. The results show that the proposed approach is highly discriminant and achieves a promising accuracy for face recognition than the state-of-the-art approaches.]]></abstract>
##     <issn><![CDATA[2169-3536]]></issn>
##     <htmlFlag><![CDATA[1]]></htmlFlag>
##     <arnumber><![CDATA[6878464]]></arnumber>
##     <doi><![CDATA[10.1109/ACCESS.2014.2348018]]></doi>
##     <publicationId><![CDATA[6878464]]></publicationId>
##     <mdurl><![CDATA[http://ieeexplore.ieee.org/xpl/articleDetails.jsp?tp=&arnumber=6878464&contentType=Journals+%26+Magazines]]></mdurl>
##     <pdf><![CDATA[http://ieeexplore.ieee.org/stamp/stamp.jsp?arnumber=6878464]]></pdf>
##   </document>
##   <document>
##     <rank>832</rank>
##     <title><![CDATA[Energy Distribution of Positive Charges in <formula formulatype="inline"> <img src="/images/tex/21372.gif" alt="{\rm Al}_{2}{\rm O}_{3}{\rm GeO}_{2}/{\rm Ge}"> </formula> pMOSFETs]]></title>
##     <authors><![CDATA[Jigang Ma;  Zhang, J.F.;  Zhigang Ji;  Benbakhti, B.;  Wei Zhang;  Mitard, J.;  Kaczer, B.;  Groeseneken, G.;  Hall, S.;  Robertson, J.;  Chalker, P.]]></authors>
##     <affiliations><![CDATA[Sch. of Eng., Liverpool John Moores Univ., Liverpool, UK]]></affiliations>
##     <controlledterms>
##       <term><![CDATA[MOSFET]]></term>
##       <term><![CDATA[aluminium compounds]]></term>
##       <term><![CDATA[germanium]]></term>
##       <term><![CDATA[germanium compounds]]></term>
##       <term><![CDATA[hole mobility]]></term>
##       <term><![CDATA[interface states]]></term>
##       <term><![CDATA[minimisation]]></term>
##       <term><![CDATA[negative bias temperature instability]]></term>
##     </controlledterms>
##     <thesaurusterms>
##       <term><![CDATA[Aluminum oxide]]></term>
##       <term><![CDATA[Interface states]]></term>
##       <term><![CDATA[Logic gates]]></term>
##       <term><![CDATA[Silicon]]></term>
##       <term><![CDATA[Stress]]></term>
##     </thesaurusterms>
##     <pubtitle><![CDATA[Electron Device Letters, IEEE]]></pubtitle>
##     <punumber><![CDATA[55]]></punumber>
##     <pubtype><![CDATA[Journals & Magazines]]></pubtype>
##     <publisher><![CDATA[IEEE]]></publisher>
##     <volume><![CDATA[35]]></volume>
##     <issue><![CDATA[2]]></issue>
##     <py><![CDATA[2014]]></py>
##     <spage><![CDATA[160]]></spage>
##     <epage><![CDATA[162]]></epage>
##     <abstract><![CDATA[The high hole mobility of Ge makes it a strong candidate for end of roadmap pMOSFETs and low interface states have been achieved for the Al<sub>2</sub>O<sub>3</sub>-GeO<sub>2</sub>-Ge gate-stack. This structure, however, suffers from significant negative bias temperature instability (NBTI), dominated by positive charge (PC) in Al<sub>2</sub>O<sub>3</sub>/GeO<sub>2</sub>. An in-depth understanding of the PCs will assist in the minimization of NBTI and the defect energy distribution will provide valuable information. The energy distribution also provides the effective charge density at a given surface potential, a key parameter required for simulating the impact of NBTI on device and circuit performance. For the first time, this letter reports the energy distribution of the PC in Al<sub>2</sub>O<sub>3</sub>/GeO<sub>2</sub> on Ge. It is found that the energy density of the PC has a clear peak near Ge Ec at the interface and a relatively low level between Ec and Ev. Below Ev at the interface, it increases rapidly and screens 20% of the Vg rise.]]></abstract>
##     <issn><![CDATA[0741-3106]]></issn>
##     <htmlFlag><![CDATA[1]]></htmlFlag>
##     <arnumber><![CDATA[6705621]]></arnumber>
##     <doi><![CDATA[10.1109/LED.2013.2295516]]></doi>
##     <publicationId><![CDATA[6705621]]></publicationId>
##     <mdurl><![CDATA[http://ieeexplore.ieee.org/xpl/articleDetails.jsp?tp=&arnumber=6705621&contentType=Journals+%26+Magazines]]></mdurl>
##     <pdf><![CDATA[http://ieeexplore.ieee.org/stamp/stamp.jsp?arnumber=6705621]]></pdf>
##   </document>
##   <document>
##     <rank>833</rank>
##     <title><![CDATA[Coherent Multimode OAM Superpositions for Multidimensional Modulation]]></title>
##     <authors><![CDATA[Anguita, J.A.;  Herreros, J.;  Djordjevic, I.B.]]></authors>
##     <affiliations><![CDATA[Fac. de Ing. y Cienc. Aplic., Univ. de los Andes, Santiago, Chile]]></affiliations>
##     <controlledterms>
##       <term><![CDATA[angular momentum]]></term>
##       <term><![CDATA[optical communication]]></term>
##       <term><![CDATA[optical vortices]]></term>
##       <term><![CDATA[parity check codes]]></term>
##       <term><![CDATA[spatial light modulators]]></term>
##     </controlledterms>
##     <thesaurusterms>
##       <term><![CDATA[Diffraction]]></term>
##       <term><![CDATA[Gratings]]></term>
##       <term><![CDATA[Laser beams]]></term>
##       <term><![CDATA[Modulation]]></term>
##       <term><![CDATA[Optical beams]]></term>
##       <term><![CDATA[Optical fiber communication]]></term>
##       <term><![CDATA[Optical vortices]]></term>
##     </thesaurusterms>
##     <pubtitle><![CDATA[Photonics Journal, IEEE]]></pubtitle>
##     <punumber><![CDATA[4563994]]></punumber>
##     <pubtype><![CDATA[Journals & Magazines]]></pubtype>
##     <publisher><![CDATA[IEEE]]></publisher>
##     <volume><![CDATA[6]]></volume>
##     <issue><![CDATA[2]]></issue>
##     <py><![CDATA[2014]]></py>
##     <spage><![CDATA[1]]></spage>
##     <epage><![CDATA[11]]></epage>
##     <abstract><![CDATA[The generation, propagation, and detection of high-quality and coherently superimposed optical vortices, carrying two or more orbital angular momentum (OAM) states, is experimentally demonstrated using an optical arrangement based on spatial light modulators. We compare our results with numerical simulations and show that, in the context of turbulence-free wireless optical communication (indoor or satellite), individual OAM state identification at the receiver of an OAM-modulated system can be achieved with good precision, to accommodate for high-dimensional OAM modulation architectures. We apply our results to the simulation of a communication system using low-density parity-check-coded modulation that considers optimal signal constellation design in a channel that includes OAM crosstalk induced by realistic (imperfect) detection.]]></abstract>
##     <issn><![CDATA[1943-0655]]></issn>
##     <htmlFlag><![CDATA[1]]></htmlFlag>
##     <arnumber><![CDATA[6755504]]></arnumber>
##     <doi><![CDATA[10.1109/JPHOT.2014.2309645]]></doi>
##     <publicationId><![CDATA[6755504]]></publicationId>
##     <mdurl><![CDATA[http://ieeexplore.ieee.org/xpl/articleDetails.jsp?tp=&arnumber=6755504&contentType=Journals+%26+Magazines]]></mdurl>
##     <pdf><![CDATA[http://ieeexplore.ieee.org/stamp/stamp.jsp?arnumber=6755504]]></pdf>
##   </document>
##   <document>
##     <rank>834</rank>
##     <title><![CDATA[Nonadiabatically Tapered Microfiber Sensor With Ultrashort Waist]]></title>
##     <authors><![CDATA[Wen Bin Ji;  Yung Chuen Tan;  Bo Lin;  Swee Chuan Tjin;  Kin Kee Chow]]></authors>
##     <affiliations><![CDATA[Dept. of Electr. & Electron. Eng., Nanyang Technol. Univ., Singapore, Singapore]]></affiliations>
##     <controlledterms>
##       <term><![CDATA[fibre optic sensors]]></term>
##       <term><![CDATA[light interferometry]]></term>
##       <term><![CDATA[micro-optics]]></term>
##       <term><![CDATA[microsensors]]></term>
##       <term><![CDATA[optical fibre fabrication]]></term>
##       <term><![CDATA[refractive index]]></term>
##     </controlledterms>
##     <thesaurusterms>
##       <term><![CDATA[Fabrication]]></term>
##       <term><![CDATA[Optical fiber sensors]]></term>
##       <term><![CDATA[Optical fibers]]></term>
##       <term><![CDATA[Optical interferometry]]></term>
##       <term><![CDATA[Optimized production technology]]></term>
##       <term><![CDATA[Refractive index]]></term>
##       <term><![CDATA[Sensitivity]]></term>
##     </thesaurusterms>
##     <pubtitle><![CDATA[Photonics Technology Letters, IEEE]]></pubtitle>
##     <punumber><![CDATA[68]]></punumber>
##     <pubtype><![CDATA[Journals & Magazines]]></pubtype>
##     <publisher><![CDATA[IEEE]]></publisher>
##     <volume><![CDATA[26]]></volume>
##     <issue><![CDATA[22]]></issue>
##     <py><![CDATA[2014]]></py>
##     <spage><![CDATA[2303]]></spage>
##     <epage><![CDATA[2306]]></epage>
##     <abstract><![CDATA[We investigated a nonadiabatically tapered microfiber refractive index sensor with an ultrashort waist length. Free spectral range (FSR) is one common issue in the interferometry-based sensors. Such sensors will not work properly once the resonant wavelengths shift over one cycle. It was found that the temperature and pulling speed are the crucial parameters to achieve a desired tapering profile. By setting a cooler tapering temperature and fast pulling speed, we successfully fabricated ultrashort microfibers with a waist length of only 2.4 mm and total length of 5.5 mm. The resulting FSR is ~80 nm and the maximum sensitivity achieved for such short microfiber is 25 667 nm/RIU.]]></abstract>
##     <issn><![CDATA[1041-1135]]></issn>
##     <htmlFlag><![CDATA[1]]></htmlFlag>
##     <arnumber><![CDATA[6891196]]></arnumber>
##     <doi><![CDATA[10.1109/LPT.2014.2353931]]></doi>
##     <publicationId><![CDATA[6891196]]></publicationId>
##     <mdurl><![CDATA[http://ieeexplore.ieee.org/xpl/articleDetails.jsp?tp=&arnumber=6891196&contentType=Journals+%26+Magazines]]></mdurl>
##     <pdf><![CDATA[http://ieeexplore.ieee.org/stamp/stamp.jsp?arnumber=6891196]]></pdf>
##   </document>
##   <document>
##     <rank>835</rank>
##     <title><![CDATA[Optimizing Time-Multiplexing Auto-Stereoscopic Displays With a Genetic Algorithm]]></title>
##     <authors><![CDATA[Haowen Liang;  Senzhong An;  Jiahui Wang;  Yangui Zhou;  Hang Fan;  Krebs, P.;  Jianying Zhou]]></authors>
##     <affiliations><![CDATA[State Key Lab. of Optoelectron. Mater. & Technol., Sun Yat-Sen Univ., Guangzhou, China]]></affiliations>
##     <controlledterms>
##       <term><![CDATA[adjacent channel interference]]></term>
##       <term><![CDATA[genetic algorithms]]></term>
##       <term><![CDATA[optical crosstalk]]></term>
##       <term><![CDATA[three-dimensional displays]]></term>
##       <term><![CDATA[time division multiplexing]]></term>
##     </controlledterms>
##     <thesaurusterms>
##       <term><![CDATA[Arrays]]></term>
##       <term><![CDATA[Brightness]]></term>
##       <term><![CDATA[Crosstalk]]></term>
##       <term><![CDATA[Lenses]]></term>
##       <term><![CDATA[Optimization]]></term>
##       <term><![CDATA[Shape]]></term>
##       <term><![CDATA[Three-dimensional displays]]></term>
##     </thesaurusterms>
##     <pubtitle><![CDATA[Display Technology, Journal of]]></pubtitle>
##     <punumber><![CDATA[9425]]></punumber>
##     <pubtype><![CDATA[Journals & Magazines]]></pubtype>
##     <publisher><![CDATA[IEEE]]></publisher>
##     <volume><![CDATA[10]]></volume>
##     <issue><![CDATA[8]]></issue>
##     <py><![CDATA[2014]]></py>
##     <spage><![CDATA[695]]></spage>
##     <epage><![CDATA[699]]></epage>
##     <abstract><![CDATA[A figure-of-merit (FOM) of an auto-stereoscopic display system is introduced and adopted to characterize the system performance. This FOM takes into account of the ratio of the signal to the noise arising from the crosstalk from the adjacent channels as well as the brightness uniformity of viewing areas; hence, it is directly related to the glasses-free 3D viewing comfort. With a steadily improving FOM as a target, the genetic algorithm is applied to optimize the optical system, giving rise to substantially improved characteristics of an auto-stereoscopic display system. The numerical simulation is verified with an experiment of a multi-view auto-stereoscopic display unit. It is shown that the system can provide a high fidelity of the display effect with the crosstalk ratio being reduced from around 5% to nearly 1%, which is a very low value obtainable for an auto-stereoscopic system.]]></abstract>
##     <issn><![CDATA[1551-319X]]></issn>
##     <htmlFlag><![CDATA[1]]></htmlFlag>
##     <arnumber><![CDATA[6780603]]></arnumber>
##     <doi><![CDATA[10.1109/JDT.2014.2314138]]></doi>
##     <publicationId><![CDATA[6780603]]></publicationId>
##     <mdurl><![CDATA[http://ieeexplore.ieee.org/xpl/articleDetails.jsp?tp=&arnumber=6780603&contentType=Journals+%26+Magazines]]></mdurl>
##     <pdf><![CDATA[http://ieeexplore.ieee.org/stamp/stamp.jsp?arnumber=6780603]]></pdf>
##   </document>
##   <document>
##     <rank>836</rank>
##     <title><![CDATA[The Stationary Phase Approximation, Time-Frequency Decomposition and Auditory Processing]]></title>
##     <authors><![CDATA[Mulgrew, B.]]></authors>
##     <affiliations><![CDATA[Institute for Digital Communications, School of Engineering, The University of Edinburgh, Edinburgh, United Kingdom]]></affiliations>
##     <thesaurusterms>
##       <term><![CDATA[Approximation methods]]></term>
##       <term><![CDATA[Auditory system]]></term>
##       <term><![CDATA[Bandwidth]]></term>
##       <term><![CDATA[Chirp]]></term>
##       <term><![CDATA[Frequency response]]></term>
##       <term><![CDATA[Prototypes]]></term>
##       <term><![CDATA[Time-frequency analysis]]></term>
##     </thesaurusterms>
##     <pubtitle><![CDATA[Signal Processing, IEEE Transactions on]]></pubtitle>
##     <punumber><![CDATA[78]]></punumber>
##     <pubtype><![CDATA[Journals & Magazines]]></pubtype>
##     <publisher><![CDATA[IEEE]]></publisher>
##     <volume><![CDATA[62]]></volume>
##     <issue><![CDATA[1]]></issue>
##     <py><![CDATA[2014]]></py>
##     <spage><![CDATA[56]]></spage>
##     <epage><![CDATA[68]]></epage>
##     <abstract><![CDATA[The principle of stationary phase (PSP) is re-examined in the context of linear time-frequency (TF) decomposition using Gaussian, gammatone and gamma chirp filters at uniform, logarithmic and cochlear spacings in frequency. This necessitates consideration of the use the PSP on non-asymptotic integrals and leads to the introduction of a test for phase rate dominance. Regions of the TF plane that pass the test and do not contain stationary phase points contribute little or nothing to the final output. Analysis values that lie in these regions can thus be set to zero, i.e., sparsity. In regions of the TF plane that fail the test or are in the vicinity of stationary phase points, synthesis is performed in the usual way. A new interpretation of the location parameters associated with the synthesis filters leads to: i) a new method for locating stationary phase points in the TF plane and ii) a test for phase rate dominance in that plane. Together this is a TF stationary phase approximation (TFSPA) for both analysis and synthesis. The stationary phase regions of several elementary signals are identified theoretically and examples of reconstruction given. An analysis of the TF phase rate characteristics for the case of two simultaneous tones predicts and quantifies a form of simultaneous masking similar to that which characterizes the auditory system.]]></abstract>
##     <issn><![CDATA[1053-587X]]></issn>
##     <htmlFlag><![CDATA[1]]></htmlFlag>
##     <arnumber><![CDATA[6620957]]></arnumber>
##     <doi><![CDATA[10.1109/TSP.2013.2284479]]></doi>
##     <publicationId><![CDATA[6620957]]></publicationId>
##     <mdurl><![CDATA[http://ieeexplore.ieee.org/xpl/articleDetails.jsp?tp=&arnumber=6620957&contentType=Journals+%26+Magazines]]></mdurl>
##     <pdf><![CDATA[http://ieeexplore.ieee.org/stamp/stamp.jsp?arnumber=6620957]]></pdf>
##   </document>
##   <document>
##     <rank>837</rank>
##     <title><![CDATA[Convex Relaxation of Optimal Power Flow&#x2014;Part II: Exactness]]></title>
##     <authors><![CDATA[Low, S.H.]]></authors>
##     <affiliations><![CDATA[Eng. & Appl. Sci. (EAS), California Inst. of Technol. (Caltech), Pasadena, CA, USA]]></affiliations>
##     <controlledterms>
##       <term><![CDATA[load flow]]></term>
##     </controlledterms>
##     <thesaurusterms>
##       <term><![CDATA[Control systems]]></term>
##       <term><![CDATA[Cost function]]></term>
##       <term><![CDATA[Mesh networks]]></term>
##       <term><![CDATA[Phase shifters]]></term>
##       <term><![CDATA[Tutorials]]></term>
##       <term><![CDATA[Upper bound]]></term>
##       <term><![CDATA[Voltage control]]></term>
##     </thesaurusterms>
##     <pubtitle><![CDATA[Control of Network Systems, IEEE Transactions on]]></pubtitle>
##     <punumber><![CDATA[6509490]]></punumber>
##     <pubtype><![CDATA[Journals & Magazines]]></pubtype>
##     <publisher><![CDATA[IEEE]]></publisher>
##     <volume><![CDATA[1]]></volume>
##     <issue><![CDATA[2]]></issue>
##     <py><![CDATA[2014]]></py>
##     <spage><![CDATA[177]]></spage>
##     <epage><![CDATA[189]]></epage>
##     <abstract><![CDATA[This tutorial summarizes recent advances in the convex relaxation of the optimal power flow (OPF) problem, focusing on structural properties rather than algorithms. Part I presents two power flow models, formulates OPF and their relaxations in each model, and proves equivalence relations among them. Part II presents sufficient conditions under which the convex relaxations are exact.]]></abstract>
##     <issn><![CDATA[2325-5870]]></issn>
##     <htmlFlag><![CDATA[1]]></htmlFlag>
##     <arnumber><![CDATA[6815671]]></arnumber>
##     <doi><![CDATA[10.1109/TCNS.2014.2323634]]></doi>
##     <publicationId><![CDATA[6815671]]></publicationId>
##     <mdurl><![CDATA[http://ieeexplore.ieee.org/xpl/articleDetails.jsp?tp=&arnumber=6815671&contentType=Journals+%26+Magazines]]></mdurl>
##     <pdf><![CDATA[http://ieeexplore.ieee.org/stamp/stamp.jsp?arnumber=6815671]]></pdf>
##   </document>
##   <document>
##     <rank>838</rank>
##     <title><![CDATA[Zenith/Nadir Pointing mm-Wave Radars: Linear or Circular Polarization?]]></title>
##     <authors><![CDATA[Galletti, M.;  Dong Huang;  Kollias, P.]]></authors>
##     <affiliations><![CDATA[Radar Sci. Group, Brookhaven Nat. Lab., Upton, NY, USA]]></affiliations>
##     <controlledterms>
##       <term><![CDATA[atmospheric techniques]]></term>
##       <term><![CDATA[meteorological radar]]></term>
##       <term><![CDATA[millimetre wave radar]]></term>
##       <term><![CDATA[radar polarimetry]]></term>
##       <term><![CDATA[reflectivity]]></term>
##     </controlledterms>
##     <thesaurusterms>
##       <term><![CDATA[Coherence]]></term>
##       <term><![CDATA[Correlation]]></term>
##       <term><![CDATA[Covariance matrix]]></term>
##       <term><![CDATA[Polarization]]></term>
##       <term><![CDATA[Radar antennas]]></term>
##       <term><![CDATA[Radar polarimetry]]></term>
##     </thesaurusterms>
##     <pubtitle><![CDATA[Geoscience and Remote Sensing, IEEE Transactions on]]></pubtitle>
##     <punumber><![CDATA[36]]></punumber>
##     <pubtype><![CDATA[Journals & Magazines]]></pubtype>
##     <publisher><![CDATA[IEEE]]></publisher>
##     <volume><![CDATA[52]]></volume>
##     <issue><![CDATA[1]]></issue>
##     <part><![CDATA[2]]></part>
##     <py><![CDATA[2014]]></py>
##     <spage><![CDATA[628]]></spage>
##     <epage><![CDATA[639]]></epage>
##     <abstract><![CDATA[We consider zenith/nadir pointing atmospheric radars and explore the effects of different dual-polarization architectures on the retrieved variables: reflectivity, depolarization ratio, cross-polar coherence, and degree of polarization. Under the assumption of azimuthal symmetry, when the linear depolarization ratio (LDR) and circular depolarization ratio (CDR) modes are compared, it is found that for most atmospheric scatterers reflectivity is comparable, whereas the depolarization ratio dynamic range is maximized at CDR mode by at least 3 dB. In the presence of anisotropic (aligned) scatterers, that is, when azimuthal symmetry is broken, polarimetric variables at CDR mode do have the desirable property of rotational invariance and, further, the dynamic range of CDR can be significantly larger than the dynamic range of LDR. The physical meaning of the cross-polar coherence is revisited in terms of scattering symmetries, that is, departure from reflection symmetry for the LDR mode and departure from rotation symmetry for the CDR mode. The Simultaneous Transmission and Simultaneous Reception mode (STSR mode or hybrid mode or ZDR mode) is also theoretically analyzed for the case of zenith/nadir pointing radars and, under the assumption of azimuthal symmetry, relations are given to compare measurements obtained at hybrid mode with measurements obtained from orthogonal (LDR and CDR) modes.]]></abstract>
##     <issn><![CDATA[0196-2892]]></issn>
##     <htmlFlag><![CDATA[1]]></htmlFlag>
##     <arnumber><![CDATA[6472767]]></arnumber>
##     <doi><![CDATA[10.1109/TGRS.2013.2243155]]></doi>
##     <publicationId><![CDATA[6472767]]></publicationId>
##     <mdurl><![CDATA[http://ieeexplore.ieee.org/xpl/articleDetails.jsp?tp=&arnumber=6472767&contentType=Journals+%26+Magazines]]></mdurl>
##     <pdf><![CDATA[http://ieeexplore.ieee.org/stamp/stamp.jsp?arnumber=6472767]]></pdf>
##   </document>
##   <document>
##     <rank>839</rank>
##     <title><![CDATA[Design of Multi-Band Digital Filters and Full-Band Digital Differentiators Without Frequency Sampling and Iterative Optimization]]></title>
##     <authors><![CDATA[Nakamoto, M.;  Ohno, S.]]></authors>
##     <affiliations><![CDATA[Dept. of Syst. Cybern., Hiroshima Univ., Higashi-Hiroshima, Japan]]></affiliations>
##     <controlledterms>
##       <term><![CDATA[digital filters]]></term>
##       <term><![CDATA[high-pass filters]]></term>
##       <term><![CDATA[iterative methods]]></term>
##       <term><![CDATA[low-pass filters]]></term>
##       <term><![CDATA[optimisation]]></term>
##       <term><![CDATA[signal sampling]]></term>
##       <term><![CDATA[stability]]></term>
##     </controlledterms>
##     <pubtitle><![CDATA[Industrial Electronics, IEEE Transactions on]]></pubtitle>
##     <punumber><![CDATA[41]]></punumber>
##     <pubtype><![CDATA[Journals & Magazines]]></pubtype>
##     <publisher><![CDATA[IEEE]]></publisher>
##     <volume><![CDATA[61]]></volume>
##     <issue><![CDATA[9]]></issue>
##     <py><![CDATA[2014]]></py>
##     <spage><![CDATA[4857]]></spage>
##     <epage><![CDATA[4866]]></epage>
##     <abstract><![CDATA[Most design problems of digital filters (or differentiators) are formulated with a set of grid point in the frequency region (frequency sampling). These problems are usually difficult to solve, and often require iterative optimization. The objective of this paper is to provide an efficient and simplified design approach to multi-band filters (including low-pass filters or high-pass filters) as well as full-band differentiators. The proposed method does not require frequency sampling and iterative optimization to compute the coefficients of the filters or that of the differentiators. The magnitude and phase specifications are simultaneously approximated, and the errors in the specified frequency bands are controlled by using frequency-weighting factors. In addition, a maximum pole radius, which corresponds to a stability margin, can be specified to robustly ensure the stability of the filters or the differentiators. To evaluate the efficiency of proposed method, we compare the proposed method with several established methods. Simulation results show that, although the propose method does not utilize frequency sampling and iterative optimization, the designed filters and differentiators have sufficient performance.]]></abstract>
##     <issn><![CDATA[0278-0046]]></issn>
##     <htmlFlag><![CDATA[1]]></htmlFlag>
##     <arnumber><![CDATA[6663628]]></arnumber>
##     <doi><![CDATA[10.1109/TIE.2013.2290765]]></doi>
##     <publicationId><![CDATA[6663628]]></publicationId>
##     <mdurl><![CDATA[http://ieeexplore.ieee.org/xpl/articleDetails.jsp?tp=&arnumber=6663628&contentType=Journals+%26+Magazines]]></mdurl>
##     <pdf><![CDATA[http://ieeexplore.ieee.org/stamp/stamp.jsp?arnumber=6663628]]></pdf>
##   </document>
##   <document>
##     <rank>840</rank>
##     <title><![CDATA[Large-Scale Deep Belief Nets With MapReduce]]></title>
##     <authors><![CDATA[Kunlei Zhang;  Xue-Wen Chen]]></authors>
##     <affiliations><![CDATA[Dept. of Comput. Sci., Wayne State Univ., Detroit, MI, USA]]></affiliations>
##     <controlledterms>
##       <term><![CDATA[Big Data]]></term>
##       <term><![CDATA[Boltzmann machines]]></term>
##       <term><![CDATA[backpropagation]]></term>
##       <term><![CDATA[parallel programming]]></term>
##     </controlledterms>
##     <thesaurusterms>
##       <term><![CDATA[Belief networks]]></term>
##       <term><![CDATA[Boltzmann machines]]></term>
##       <term><![CDATA[Computational modeling]]></term>
##       <term><![CDATA[Data handling]]></term>
##       <term><![CDATA[Data storage systems]]></term>
##       <term><![CDATA[Distributed computing]]></term>
##       <term><![CDATA[Information management]]></term>
##       <term><![CDATA[Parallel programming]]></term>
##     </thesaurusterms>
##     <pubtitle><![CDATA[Access, IEEE]]></pubtitle>
##     <punumber><![CDATA[6287639]]></punumber>
##     <pubtype><![CDATA[Journals & Magazines]]></pubtype>
##     <publisher><![CDATA[IEEE]]></publisher>
##     <volume><![CDATA[2]]></volume>
##     <py><![CDATA[2014]]></py>
##     <spage><![CDATA[395]]></spage>
##     <epage><![CDATA[403]]></epage>
##     <abstract><![CDATA[Deep belief nets (DBNs) with restricted Boltzmann machines (RBMs) as the building block have recently attracted wide attention due to their great performance in various applications. The learning of a DBN starts with pretraining a series of the RBMs followed by fine-tuning the whole net using backpropagation. Generally, the sequential implementation of both RBMs and backpropagation algorithm takes significant amount of computational time to process massive data sets. The emerging big data learning requires distributed computing for the DBNs. In this paper, we present a distributed learning paradigm for the RBMs and the backpropagation algorithm using MapReduce, a popular parallel programming model. Thus, the DBNs can be trained in a distributed way by stacking a series of distributed RBMs for pretraining and a distributed backpropagation for fine-tuning. Through validation on the benchmark data sets of various practical problems, the experimental results demonstrate that the distributed RBMs and DBNs are amenable to large-scale data with a good performance in terms of accuracy and efficiency.]]></abstract>
##     <issn><![CDATA[2169-3536]]></issn>
##     <htmlFlag><![CDATA[1]]></htmlFlag>
##     <arnumber><![CDATA[6804632]]></arnumber>
##     <doi><![CDATA[10.1109/ACCESS.2014.2319813]]></doi>
##     <publicationId><![CDATA[6804632]]></publicationId>
##     <mdurl><![CDATA[http://ieeexplore.ieee.org/xpl/articleDetails.jsp?tp=&arnumber=6804632&contentType=Journals+%26+Magazines]]></mdurl>
##     <pdf><![CDATA[http://ieeexplore.ieee.org/stamp/stamp.jsp?arnumber=6804632]]></pdf>
##   </document>
##   <document>
##     <rank>841</rank>
##     <title><![CDATA[Noise Resistant Generalized Parametric Validity Index of Clustering for Gene Expression Data]]></title>
##     <authors><![CDATA[Rui Fa;  Nandi, A.K.]]></authors>
##     <affiliations><![CDATA[Dept. of Electron. & Comput. Eng., Brunel Univ., Uxbridge, UK]]></affiliations>
##     <controlledterms>
##       <term><![CDATA[genetics]]></term>
##       <term><![CDATA[pattern clustering]]></term>
##     </controlledterms>
##     <thesaurusterms>
##       <term><![CDATA[Bioinformatics]]></term>
##       <term><![CDATA[Clustering algorithms]]></term>
##       <term><![CDATA[Computational biology]]></term>
##       <term><![CDATA[Gene expression]]></term>
##       <term><![CDATA[Noise measurement]]></term>
##     </thesaurusterms>
##     <pubtitle><![CDATA[Computational Biology and Bioinformatics, IEEE/ACM Transactions on]]></pubtitle>
##     <punumber><![CDATA[8857]]></punumber>
##     <pubtype><![CDATA[Journals & Magazines]]></pubtype>
##     <publisher><![CDATA[IEEE]]></publisher>
##     <volume><![CDATA[11]]></volume>
##     <issue><![CDATA[4]]></issue>
##     <py><![CDATA[2014]]></py>
##     <spage><![CDATA[741]]></spage>
##     <epage><![CDATA[752]]></epage>
##     <abstract><![CDATA[Validity indices have been investigated for decades. However, since there is no study of noise-resistance performance of these indices in the literature, there is no guideline for determining the best clustering in noisy data sets, especially microarray data sets. In this paper, we propose a generalized parametric validity (GPV) index which employs two tunable parameters &#x03B1; and &#x03B2; to control the proportions of objects being considered to calculate the dissimilarities. The greatest advantage of the proposed GPV index is its noise-resistance ability, which results from the flexibility of tuning the parameters. Several rules are set to guide the selection of parameter values. To illustrate the noise-resistance performance of the proposed index, we evaluate the GPV index for assessing five clustering algorithms in two gene expression data simulation models with different noise levels and compare the ability of determining the number of clusters with eight existing indices. We also test the GPV in three groups of real gene expression data sets. The experimental results suggest that the proposed GPV index has superior noise-resistance ability and provides fairly accurate judgements.]]></abstract>
##     <issn><![CDATA[1545-5963]]></issn>
##     <htmlFlag><![CDATA[1]]></htmlFlag>
##     <arnumber><![CDATA[6774444]]></arnumber>
##     <doi><![CDATA[10.1109/TCBB.2014.2312006]]></doi>
##     <publicationId><![CDATA[6774444]]></publicationId>
##     <mdurl><![CDATA[http://ieeexplore.ieee.org/xpl/articleDetails.jsp?tp=&arnumber=6774444&contentType=Journals+%26+Magazines]]></mdurl>
##     <pdf><![CDATA[http://ieeexplore.ieee.org/stamp/stamp.jsp?arnumber=6774444]]></pdf>
##   </document>
##   <document>
##     <rank>842</rank>
##     <title><![CDATA[Vertical-Cavity Surface-Emitting Laser With a Chiral Nematic Liquid Crystal Overlay]]></title>
##     <authors><![CDATA[Xie, Y.;  Beeckman, J.;  Panajotov, K.;  Neyts, K.]]></authors>
##     <affiliations><![CDATA[Dept. of Electron. & Inf. Syst., Ghent Univ., Ghent, Belgium]]></affiliations>
##     <controlledterms>
##       <term><![CDATA[chirality]]></term>
##       <term><![CDATA[light polarisation]]></term>
##       <term><![CDATA[nematic liquid crystals]]></term>
##       <term><![CDATA[optical materials]]></term>
##       <term><![CDATA[surface emitting lasers]]></term>
##     </controlledterms>
##     <thesaurusterms>
##       <term><![CDATA[Optical polarization]]></term>
##       <term><![CDATA[Optical reflection]]></term>
##       <term><![CDATA[Polarization]]></term>
##       <term><![CDATA[Stimulated emission]]></term>
##       <term><![CDATA[Temperature measurement]]></term>
##       <term><![CDATA[Vertical cavity surface emitting lasers]]></term>
##     </thesaurusterms>
##     <pubtitle><![CDATA[Photonics Journal, IEEE]]></pubtitle>
##     <punumber><![CDATA[4563994]]></punumber>
##     <pubtype><![CDATA[Journals & Magazines]]></pubtype>
##     <publisher><![CDATA[IEEE]]></publisher>
##     <volume><![CDATA[6]]></volume>
##     <issue><![CDATA[1]]></issue>
##     <py><![CDATA[2014]]></py>
##     <spage><![CDATA[1]]></spage>
##     <epage><![CDATA[10]]></epage>
##     <abstract><![CDATA[We demonstrate measurement results of the light emission of vertical-cavity surface-emitting lasers (VCSELs) with chiral liquid crystal (CLC) overlay (CLC-VCSEL). Two different CLC materials with different chiral-phase temperature range are presented and compared. The CLC layer acts as an external cavity for the VCSEL for one circular polarization state that results in complex polarization behavior inside the VCSEL cavity. We find that the emission of the device is purely circularly polarized with a degree of circular polarization higher than the degree of linear polarization of the stand-alone VCSEL. The threshold current of the CLC-VCSEL is decreased, and the emission wavelength of the CLC-VCSEL can be temperature tuned over a larger range compared with that of the stand-alone VCSEL. Numerical results reveal that the small reflection at the top CLC has a much larger influence on the laser dynamics than the stand-alone configuration.]]></abstract>
##     <issn><![CDATA[1943-0655]]></issn>
##     <htmlFlag><![CDATA[1]]></htmlFlag>
##     <arnumber><![CDATA[6680653]]></arnumber>
##     <doi><![CDATA[10.1109/JPHOT.2013.2294632]]></doi>
##     <publicationId><![CDATA[6680653]]></publicationId>
##     <mdurl><![CDATA[http://ieeexplore.ieee.org/xpl/articleDetails.jsp?tp=&arnumber=6680653&contentType=Journals+%26+Magazines]]></mdurl>
##     <pdf><![CDATA[http://ieeexplore.ieee.org/stamp/stamp.jsp?arnumber=6680653]]></pdf>
##   </document>
##   <document>
##     <rank>843</rank>
##     <title><![CDATA[Reluctance Accelerator Efficiency Optimization via Pulse Shaping]]></title>
##     <authors><![CDATA[Cooper, L.M.;  Van Cleef, A.R.;  Bristoll, B.T.;  Bartlett, P.A.]]></authors>
##     <affiliations><![CDATA[Dept. of Phys. & Astron., Univ. Coll. London, London, UK]]></affiliations>
##     <controlledterms>
##       <term><![CDATA[MOSFET]]></term>
##       <term><![CDATA[PWM invertors]]></term>
##       <term><![CDATA[capacitors]]></term>
##       <term><![CDATA[electric current control]]></term>
##       <term><![CDATA[linear motors]]></term>
##       <term><![CDATA[machine control]]></term>
##       <term><![CDATA[optimisation]]></term>
##       <term><![CDATA[pulse shaping]]></term>
##       <term><![CDATA[reluctance motors]]></term>
##       <term><![CDATA[solenoids]]></term>
##     </controlledterms>
##     <thesaurusterms>
##       <term><![CDATA[Capacitors]]></term>
##       <term><![CDATA[Discharges (electric)]]></term>
##       <term><![CDATA[Electromagnetic scattering]]></term>
##       <term><![CDATA[Ferroelectric materials]]></term>
##       <term><![CDATA[MOSFET]]></term>
##       <term><![CDATA[Particle accelerators]]></term>
##       <term><![CDATA[Pulse width modulation]]></term>
##       <term><![CDATA[Reluctance machines]]></term>
##       <term><![CDATA[Solenoids]]></term>
##     </thesaurusterms>
##     <pubtitle><![CDATA[Access, IEEE]]></pubtitle>
##     <punumber><![CDATA[6287639]]></punumber>
##     <pubtype><![CDATA[Journals & Magazines]]></pubtype>
##     <publisher><![CDATA[IEEE]]></publisher>
##     <volume><![CDATA[2]]></volume>
##     <py><![CDATA[2014]]></py>
##     <spage><![CDATA[1143]]></spage>
##     <epage><![CDATA[1148]]></epage>
##     <abstract><![CDATA[Reluctance accelerators are used to apply linear forces to ferromagnetic projectiles via solenoids. Efficiency increases for a single-stage reluctance accelerator were produced by manipulating the input current pulse supplied by a discharging capacitor. The development of a theoretical model allowed for the calculation of optimized pulse shapes. A digital pulsewidth modulated switching method was used to control the current pulse shape using an Arduino Uno microcontroller, which supplied signals to the gate of a MOSFET transistor that controlled the current to the system solenoid. An efficiency increase of 5.7% was obtained for a reluctance accelerator with an optimized current pulse shape in comparison to a capacitor discharge with no pulse shaping.]]></abstract>
##     <issn><![CDATA[2169-3536]]></issn>
##     <htmlFlag><![CDATA[1]]></htmlFlag>
##     <arnumber><![CDATA[6907938]]></arnumber>
##     <doi><![CDATA[10.1109/ACCESS.2014.2359996]]></doi>
##     <publicationId><![CDATA[6907938]]></publicationId>
##     <mdurl><![CDATA[http://ieeexplore.ieee.org/xpl/articleDetails.jsp?tp=&arnumber=6907938&contentType=Journals+%26+Magazines]]></mdurl>
##     <pdf><![CDATA[http://ieeexplore.ieee.org/stamp/stamp.jsp?arnumber=6907938]]></pdf>
##   </document>
##   <document>
##     <rank>844</rank>
##     <title><![CDATA[Advanced DSP for 400 Gb/s and Beyond Optical Networks]]></title>
##     <authors><![CDATA[Xiang Zhou;  Nelson, L.]]></authors>
##     <affiliations><![CDATA[AT&T Labs., Middletown, NJ, USA]]></affiliations>
##     <controlledterms>
##       <term><![CDATA[equalisers]]></term>
##       <term><![CDATA[optical fibre dispersion]]></term>
##       <term><![CDATA[optical fibre networks]]></term>
##       <term><![CDATA[optical filters]]></term>
##       <term><![CDATA[optical modulation]]></term>
##       <term><![CDATA[optical transmitters]]></term>
##       <term><![CDATA[oscillators]]></term>
##       <term><![CDATA[phase noise]]></term>
##       <term><![CDATA[quadrature amplitude modulation]]></term>
##       <term><![CDATA[signal denoising]]></term>
##       <term><![CDATA[telecommunication channels]]></term>
##       <term><![CDATA[time-domain analysis]]></term>
##       <term><![CDATA[wavelength division multiplexing]]></term>
##     </controlledterms>
##     <thesaurusterms>
##       <term><![CDATA[Digital signal processing]]></term>
##       <term><![CDATA[Optical filters]]></term>
##       <term><![CDATA[Optical transmitters]]></term>
##       <term><![CDATA[Quadrature amplitude modulation]]></term>
##       <term><![CDATA[Time-domain analysis]]></term>
##       <term><![CDATA[Training]]></term>
##     </thesaurusterms>
##     <pubtitle><![CDATA[Lightwave Technology, Journal of]]></pubtitle>
##     <punumber><![CDATA[50]]></punumber>
##     <pubtype><![CDATA[Journals & Magazines]]></pubtype>
##     <publisher><![CDATA[IEEE]]></publisher>
##     <volume><![CDATA[32]]></volume>
##     <issue><![CDATA[16]]></issue>
##     <py><![CDATA[2014]]></py>
##     <spage><![CDATA[2716]]></spage>
##     <epage><![CDATA[2725]]></epage>
##     <abstract><![CDATA[This paper presents a systematic review of several digital signal processing (DSP)-enabled technologies recently proposed and demonstrated for high spectral efficiency (SE) 400 Gb/s-class and beyond optical networks. These include 1) a newly proposed SE-adaptable optical modulation technology-time-domain hybrid quadrature amplitude modulation (QAM), 2) two advanced transmitter side digital spectral shaping technologies-Nyquist signaling (for spectrally-efficient multiplexing) and digital pre-equalization (for improving tolerance toward channel narrowing effects), and 3) a newly proposed training-assisted two-stage carrier phase recovery algorithm that is designed to address the detrimental cyclic phase slipping problem with minimal training overhead. Additionally, this paper presents a novel DSP-based method for mitigation of equalizer-enhanced phase noise impairments. It is shown that performance degradation caused by the interaction between the long-memory chromatic dispersion compensating filter/equalizer and local oscillator laser phase noise can be effectively mitigated by replacing the commonly used fast single-tap phase-rotation-based equalizer (for typical carrier phase recovery) with a fast multi-tap linear equalizer. Finally, brief reviews of two high-SE 400 Gb/s-class WDM transmission experiments employing these advanced DSP algorithms are presented.]]></abstract>
##     <issn><![CDATA[0733-8724]]></issn>
##     <htmlFlag><![CDATA[1]]></htmlFlag>
##     <arnumber><![CDATA[6810180]]></arnumber>
##     <doi><![CDATA[10.1109/JLT.2014.2321135]]></doi>
##     <publicationId><![CDATA[6810180]]></publicationId>
##     <mdurl><![CDATA[http://ieeexplore.ieee.org/xpl/articleDetails.jsp?tp=&arnumber=6810180&contentType=Journals+%26+Magazines]]></mdurl>
##     <pdf><![CDATA[http://ieeexplore.ieee.org/stamp/stamp.jsp?arnumber=6810180]]></pdf>
##   </document>
##   <document>
##     <rank>845</rank>
##     <title><![CDATA[Co-Channel Interference Modeling in Cognitive Wireless Networks]]></title>
##     <authors><![CDATA[Jian Li;  Shenghong Li;  Feng Zhao;  Rong Du]]></authors>
##     <affiliations><![CDATA[Dept. of Electron. Eng., Shanghai Jiao Tong Univ., Shanghai, China]]></affiliations>
##     <controlledterms>
##       <term><![CDATA[cochannel interference]]></term>
##       <term><![CDATA[cognitive radio]]></term>
##       <term><![CDATA[mathematical analysis]]></term>
##       <term><![CDATA[probability]]></term>
##       <term><![CDATA[signal detection]]></term>
##     </controlledterms>
##     <thesaurusterms>
##       <term><![CDATA[Cognitive radio]]></term>
##       <term><![CDATA[Distribution functions]]></term>
##       <term><![CDATA[Fading]]></term>
##       <term><![CDATA[Interference]]></term>
##       <term><![CDATA[Mathematical model]]></term>
##       <term><![CDATA[Sensors]]></term>
##       <term><![CDATA[Shape]]></term>
##     </thesaurusterms>
##     <pubtitle><![CDATA[Communications, IEEE Transactions on]]></pubtitle>
##     <punumber><![CDATA[26]]></punumber>
##     <pubtype><![CDATA[Journals & Magazines]]></pubtype>
##     <publisher><![CDATA[IEEE]]></publisher>
##     <volume><![CDATA[62]]></volume>
##     <issue><![CDATA[9]]></issue>
##     <py><![CDATA[2014]]></py>
##     <spage><![CDATA[3114]]></spage>
##     <epage><![CDATA[3128]]></epage>
##     <abstract><![CDATA[Cognitive radio is a promising technology for sharing the underutilized frequency bands that have been licensed to primary users. However, due to the uncertainty in detecting the existence of the primary user, the secondary user may interfere with the primary users when both primary and secondary users are active simultaneously. Therefore, understanding the interference and its consequences on the cognitive network is critical. Unlike the statistical models previously reported in the literature that aim at approximation of the interference, based on the solid mathematical analysis, we propose an accurate model for describing the co-channel interference with probability density function, cumulative distribution function, mean, and variance of the interference suffered by the primary users. The proposed model not only takes into account a number of factors, such as the spectrum-sensing scheme, the spatial distribution of secondary users, and the channel conditions, including shadowing and Nakagami fading, but also gives an exact mathematical expression of the influences from these factors. The developed framework supports practical applications such as evaluating the cognitive network of any spatial shape and density of the secondary users and the methods of power control and spectrum sensing used by the secondary users. Simulation results are provided to verify the effectiveness of the analytical model.]]></abstract>
##     <issn><![CDATA[0090-6778]]></issn>
##     <htmlFlag><![CDATA[1]]></htmlFlag>
##     <arnumber><![CDATA[6862033]]></arnumber>
##     <doi><![CDATA[10.1109/TCOMM.2014.2341628]]></doi>
##     <publicationId><![CDATA[6862033]]></publicationId>
##     <mdurl><![CDATA[http://ieeexplore.ieee.org/xpl/articleDetails.jsp?tp=&arnumber=6862033&contentType=Journals+%26+Magazines]]></mdurl>
##     <pdf><![CDATA[http://ieeexplore.ieee.org/stamp/stamp.jsp?arnumber=6862033]]></pdf>
##   </document>
##   <document>
##     <rank>846</rank>
##     <title><![CDATA[Laser-Induced Fluorescence Detection of OH Radicals Generated by Atmospheric-Pressure Nonequilibrium DC Pulse Discharge Plasma Jets]]></title>
##     <authors><![CDATA[Yuji, T.;  Mungkung, N.;  Kawano, H.;  Kanazawa, S.;  Ohkubo, T.;  Akatsuka, H.]]></authors>
##     <affiliations><![CDATA[Fac. of Educ. & Culture, Univ. of Miyazaki, Miyazaki, Japan]]></affiliations>
##     <controlledterms>
##       <term><![CDATA[free radicals]]></term>
##       <term><![CDATA[glow discharges]]></term>
##       <term><![CDATA[plasma diagnostics]]></term>
##       <term><![CDATA[plasma jets]]></term>
##       <term><![CDATA[plasma materials processing]]></term>
##       <term><![CDATA[surface treatment]]></term>
##     </controlledterms>
##     <thesaurusterms>
##       <term><![CDATA[Atmospheric measurements]]></term>
##       <term><![CDATA[Discharges (electric)]]></term>
##       <term><![CDATA[Educational institutions]]></term>
##       <term><![CDATA[Measurement by laser beam]]></term>
##       <term><![CDATA[Plasma measurements]]></term>
##       <term><![CDATA[Plasmas]]></term>
##       <term><![CDATA[Surface treatment]]></term>
##     </thesaurusterms>
##     <pubtitle><![CDATA[Plasma Science, IEEE Transactions on]]></pubtitle>
##     <punumber><![CDATA[27]]></punumber>
##     <pubtype><![CDATA[Journals & Magazines]]></pubtype>
##     <publisher><![CDATA[IEEE]]></publisher>
##     <volume><![CDATA[42]]></volume>
##     <issue><![CDATA[4]]></issue>
##     <py><![CDATA[2014]]></py>
##     <spage><![CDATA[960]]></spage>
##     <epage><![CDATA[964]]></epage>
##     <abstract><![CDATA[It has been proven that many types of radicals released from atmospheric-pressure plasma can provide effective surface treatment and modification of materials. However, the method for measuring radicals generated with atmospheric-pressure plasma and their reaction mechanisms have not become clear in material surface processing. The OH radical distribution was measured successfully in nonequilibrium atmospheric-pressure dc pulse discharge plasma jet by use of the laser-induced fluorescence system. The OH transition [A<sup>2</sup>&#x03A3;<sup>+</sup>(v' = 1) &#x2190; X<sup>2</sup>&#x03A0;(v" = 0)] at 282 nm was used to monitor the ground-state OH radicals.]]></abstract>
##     <issn><![CDATA[0093-3813]]></issn>
##     <arnumber><![CDATA[6766736]]></arnumber>
##     <doi><![CDATA[10.1109/TPS.2014.2306199]]></doi>
##     <publicationId><![CDATA[6766736]]></publicationId>
##     <mdurl><![CDATA[http://ieeexplore.ieee.org/xpl/articleDetails.jsp?tp=&arnumber=6766736&contentType=Journals+%26+Magazines]]></mdurl>
##     <pdf><![CDATA[http://ieeexplore.ieee.org/stamp/stamp.jsp?arnumber=6766736]]></pdf>
##   </document>
##   <document>
##     <rank>847</rank>
##     <title><![CDATA[Analysis of Hand Contact Areas and Interaction Capabilities During Manipulation and Exploration]]></title>
##     <authors><![CDATA[Gonzalez, F.;  Gosselin, F.;  Bachta, W.]]></authors>
##     <affiliations><![CDATA[Interactive Robot. Lab., CEA, Gif-sur-Yvette, France]]></affiliations>
##     <controlledterms>
##       <term><![CDATA[control engineering computing]]></term>
##       <term><![CDATA[dexterous manipulators]]></term>
##       <term><![CDATA[haptic interfaces]]></term>
##       <term><![CDATA[human computer interaction]]></term>
##       <term><![CDATA[human factors]]></term>
##       <term><![CDATA[trees (mathematics)]]></term>
##       <term><![CDATA[virtual reality]]></term>
##     </controlledterms>
##     <thesaurusterms>
##       <term><![CDATA[Grasping]]></term>
##       <term><![CDATA[Haptic interfaces]]></term>
##       <term><![CDATA[Human-computer interfaces]]></term>
##       <term><![CDATA[Robots]]></term>
##     </thesaurusterms>
##     <pubtitle><![CDATA[Haptics, IEEE Transactions on]]></pubtitle>
##     <punumber><![CDATA[4543165]]></punumber>
##     <pubtype><![CDATA[Journals & Magazines]]></pubtype>
##     <publisher><![CDATA[IEEE]]></publisher>
##     <volume><![CDATA[7]]></volume>
##     <issue><![CDATA[4]]></issue>
##     <py><![CDATA[2014]]></py>
##     <spage><![CDATA[415]]></spage>
##     <epage><![CDATA[429]]></epage>
##     <abstract><![CDATA[Manual human-computer interfaces for virtual reality are designed to allow an operator interacting with a computer simulation as naturally as possible. Dexterous haptic interfaces are the best suited for this goal. They give intuitive and efficient control on the environment with haptic and tactile feedback. This paper is aimed at helping in the choice of the interaction areas to be taken into account in the design of such interfaces. The literature dealing with hand interactions is first reviewed in order to point out the contact areas involved in exploration and manipulation tasks. Their frequencies of use are then extracted from existing recordings. The results are gathered in an original graphical interaction map allowing for a simple visualization of the way the hand is used, and compared with a map of mechanoreceptors densities. Then an interaction tree, mapping the relative amount of actions made available through the use of a given contact area, is built and correlated with the losses of hand function induced by amputations. A rating of some existing haptic interfaces and guidelines for their design are finally achieved to illustrate a possible use of the developed graphical tools.]]></abstract>
##     <issn><![CDATA[1939-1412]]></issn>
##     <htmlFlag><![CDATA[1]]></htmlFlag>
##     <arnumber><![CDATA[6809988]]></arnumber>
##     <doi><![CDATA[10.1109/TOH.2014.2321395]]></doi>
##     <publicationId><![CDATA[6809988]]></publicationId>
##     <mdurl><![CDATA[http://ieeexplore.ieee.org/xpl/articleDetails.jsp?tp=&arnumber=6809988&contentType=Journals+%26+Magazines]]></mdurl>
##     <pdf><![CDATA[http://ieeexplore.ieee.org/stamp/stamp.jsp?arnumber=6809988]]></pdf>
##   </document>
##   <document>
##     <rank>848</rank>
##     <title><![CDATA[Public Physical Unclonable Functions]]></title>
##     <authors><![CDATA[Potkonjak, M.;  Goudar, V.]]></authors>
##     <affiliations><![CDATA[Comput. Sci. Dept., Univ. of California Los Angeles, Los Angeles, CA, USA]]></affiliations>
##     <controlledterms>
##       <term><![CDATA[cryptographic protocols]]></term>
##       <term><![CDATA[private key cryptography]]></term>
##     </controlledterms>
##     <thesaurusterms>
##       <term><![CDATA[Cryptography]]></term>
##       <term><![CDATA[Hardware]]></term>
##       <term><![CDATA[Integrated circuit modeling]]></term>
##       <term><![CDATA[Logic gates]]></term>
##       <term><![CDATA[Protocols]]></term>
##       <term><![CDATA[Public key]]></term>
##     </thesaurusterms>
##     <pubtitle><![CDATA[Proceedings of the IEEE]]></pubtitle>
##     <punumber><![CDATA[5]]></punumber>
##     <pubtype><![CDATA[Journals & Magazines]]></pubtype>
##     <publisher><![CDATA[IEEE]]></publisher>
##     <volume><![CDATA[102]]></volume>
##     <issue><![CDATA[8]]></issue>
##     <py><![CDATA[2014]]></py>
##     <spage><![CDATA[1142]]></spage>
##     <epage><![CDATA[1156]]></epage>
##     <abstract><![CDATA[A physical unclonable function (PUF) is an integrated circuit (IC) that serves as a hardware security primitive due to its complexity and the unpredictability between its outputs and the applied inputs. PUFs have received a great deal of research interest and significant commercial activity. Public PUFs (PPUFs) address the crucial PUF limitation of being a secret-key technology. To some extent, the first generation of PPUFs are similar to SIMulation Possible, but Laborious (SIMPL) systems and one-time hardware pads, and employ the time gap between direct execution and simulation. The second PPUF generation employs both process variation and device aging which results in matched devices that are excessively difficult to replicate. The third generation leaves the analog domain and employs reconfigurability and device aging to produce digital PPUFs. We survey representative PPUF architectures, related public protocols and trusted information flows, and related testing issues. We conclude by identifying the most important, challenging, and open PPUF-related problems.]]></abstract>
##     <issn><![CDATA[0018-9219]]></issn>
##     <htmlFlag><![CDATA[1]]></htmlFlag>
##     <arnumber><![CDATA[6856138]]></arnumber>
##     <doi><![CDATA[10.1109/JPROC.2014.2331553]]></doi>
##     <publicationId><![CDATA[6856138]]></publicationId>
##     <mdurl><![CDATA[http://ieeexplore.ieee.org/xpl/articleDetails.jsp?tp=&arnumber=6856138&contentType=Journals+%26+Magazines]]></mdurl>
##     <pdf><![CDATA[http://ieeexplore.ieee.org/stamp/stamp.jsp?arnumber=6856138]]></pdf>
##   </document>
##   <document>
##     <rank>849</rank>
##     <title><![CDATA[Spatial Superchannel Routing in a Two-Span ROADM System for Space Division Multiplexing]]></title>
##     <authors><![CDATA[Nelson, L.E.;  Feuer, M.D.;  Abedin, K.;  Zhou, X.;  Taunay, T.F.;  Fini, J.M.;  Zhu, B.;  Isaac, R.;  Harel, R.;  Cohen, G.;  Marom, D.M.]]></authors>
##     <affiliations><![CDATA[AT&T Labs., Middletown, NJ, USA]]></affiliations>
##     <controlledterms>
##       <term><![CDATA[optical communication equipment]]></term>
##       <term><![CDATA[optical fibre amplifiers]]></term>
##       <term><![CDATA[optical fibre communication]]></term>
##       <term><![CDATA[phase shift keying]]></term>
##       <term><![CDATA[space division multiplexing]]></term>
##       <term><![CDATA[telecommunication network routing]]></term>
##       <term><![CDATA[wavelength division multiplexing]]></term>
##     </controlledterms>
##     <thesaurusterms>
##       <term><![CDATA[Bit error rate]]></term>
##       <term><![CDATA[Frequency measurement]]></term>
##       <term><![CDATA[Mirrors]]></term>
##       <term><![CDATA[Multicore processing]]></term>
##       <term><![CDATA[Optical fiber amplifiers]]></term>
##       <term><![CDATA[Optical noise]]></term>
##       <term><![CDATA[Signal to noise ratio]]></term>
##     </thesaurusterms>
##     <pubtitle><![CDATA[Lightwave Technology, Journal of]]></pubtitle>
##     <punumber><![CDATA[50]]></punumber>
##     <pubtype><![CDATA[Journals & Magazines]]></pubtype>
##     <publisher><![CDATA[IEEE]]></publisher>
##     <volume><![CDATA[32]]></volume>
##     <issue><![CDATA[4]]></issue>
##     <py><![CDATA[2014]]></py>
##     <spage><![CDATA[783]]></spage>
##     <epage><![CDATA[789]]></epage>
##     <abstract><![CDATA[We report a two-span, 67-km space-division-multiplexed (SDM) wavelength-division-multiplexed (WDM) system incorporating the first reconfigurable optical add-drop multiplexer (ROADM) supporting spatial superchannels and the first cladding-pumped multicore erbium-doped fiber amplifier directly spliced to multicore transmission fiber. The ROADM subsystem utilizes two conventional 1 &#x00D7; 20 wavelength selective switches (WSS) each configured to implement a 7 &#x00D7; (1 &#x00D7; 2) WSS. ROADM performance tests indicate that the subchannel insertion losses, attenuation accuracies, and passband widths are well matched to each other and show no significant penalty, compared to the conventional operating mode for the WSS. For 6 &#x00D7; 40 &#x00D7; 128-Gb/s SDM-WDM polarization-multiplexed quadrature phase-shift-keyed (PM-QPSK) transmission on 50 GHz spacing, optical signal-to-noise ratio penalties are less than 1.6 dB in Add, Drop, and Express paths. In addition, we demonstrate the feasibility of utilizing joint signal processing of subchannels in this two-span, ROADM system.]]></abstract>
##     <issn><![CDATA[0733-8724]]></issn>
##     <htmlFlag><![CDATA[1]]></htmlFlag>
##     <arnumber><![CDATA[6623116]]></arnumber>
##     <doi><![CDATA[10.1109/JLT.2013.2283912]]></doi>
##     <publicationId><![CDATA[6623116]]></publicationId>
##     <mdurl><![CDATA[http://ieeexplore.ieee.org/xpl/articleDetails.jsp?tp=&arnumber=6623116&contentType=Journals+%26+Magazines]]></mdurl>
##     <pdf><![CDATA[http://ieeexplore.ieee.org/stamp/stamp.jsp?arnumber=6623116]]></pdf>
##   </document>
##   <document>
##     <rank>850</rank>
##     <title><![CDATA[Caenorhabditis Elegans Segmentation Using Texture-Based Models for Motility Phenotyping]]></title>
##     <authors><![CDATA[Greenblum, A.;  Sznitman, R.;  Fua, P.;  Arratia, P.E.;  Sznitman, J.]]></authors>
##     <affiliations><![CDATA[Dept. of Biomed. Eng., Technion - Israel Inst. of Technol., Haifa, Israel]]></affiliations>
##     <controlledterms>
##       <term><![CDATA[Markov processes]]></term>
##       <term><![CDATA[biological techniques]]></term>
##       <term><![CDATA[biology computing]]></term>
##       <term><![CDATA[image segmentation]]></term>
##       <term><![CDATA[image sequences]]></term>
##       <term><![CDATA[image texture]]></term>
##       <term><![CDATA[microorganisms]]></term>
##     </controlledterms>
##     <thesaurusterms>
##       <term><![CDATA[Computational modeling]]></term>
##       <term><![CDATA[Feature extraction]]></term>
##       <term><![CDATA[Image segmentation]]></term>
##       <term><![CDATA[Image sequences]]></term>
##       <term><![CDATA[Licenses]]></term>
##       <term><![CDATA[Training]]></term>
##       <term><![CDATA[Vectors]]></term>
##     </thesaurusterms>
##     <pubtitle><![CDATA[Biomedical Engineering, IEEE Transactions on]]></pubtitle>
##     <punumber><![CDATA[10]]></punumber>
##     <pubtype><![CDATA[Journals & Magazines]]></pubtype>
##     <publisher><![CDATA[IEEE]]></publisher>
##     <volume><![CDATA[61]]></volume>
##     <issue><![CDATA[8]]></issue>
##     <py><![CDATA[2014]]></py>
##     <spage><![CDATA[2278]]></spage>
##     <epage><![CDATA[2289]]></epage>
##     <abstract><![CDATA[With widening interests in using model organisms for reverse genetic approaches and biomimmetic microrobotics, motility phenotyping of the nematode Caenorhabditis elegans is expanding across a growing array of locomotive environments. One ongoing bottleneck lies in providing users with automatic nematode segmentations of C. elegans in image sequences featuring complex and dynamic visual cues, a first and necessary step prior to extracting motility phenotypes. Here, we propose to tackle such automatic segmentation challenges by introducing a novel texture factor model (TFM). Our approach revolves around the use of combined intensity- and texture-based features integrated within a probabilistic framework. This strategy first provides a coarse nematode segmentation from which a Markov random field model is used to refine the segmentation by inferring pixels belonging to the nematode using an approximate inference technique. Finally, informative priors can then be estimated and integrated in our framework to provide coherent segmentations across image sequences. We validate our TFM method across a wide range of motility environments. Not only does TFM assure comparative performances to existing segmentation methods on traditional environments featuring static backgrounds, it importantly provides state-of-the-art C. elegans segmentations for dynamic environments such as the recently introduced wet granular media. We show how such segmentations may be used to compute nematode &#x201C;skeletons&#x201D; from which motility phenotypes can then be extracted. Overall, our TFM method provides users with a tangible solution to tackle the growing needs of C. elegans segmentation in challenging motility environments.]]></abstract>
##     <issn><![CDATA[0018-9294]]></issn>
##     <htmlFlag><![CDATA[1]]></htmlFlag>
##     <arnumber><![CDATA[6704735]]></arnumber>
##     <doi><![CDATA[10.1109/TBME.2014.2298612]]></doi>
##     <publicationId><![CDATA[6704735]]></publicationId>
##     <mdurl><![CDATA[http://ieeexplore.ieee.org/xpl/articleDetails.jsp?tp=&arnumber=6704735&contentType=Journals+%26+Magazines]]></mdurl>
##     <pdf><![CDATA[http://ieeexplore.ieee.org/stamp/stamp.jsp?arnumber=6704735]]></pdf>
##   </document>
##   <document>
##     <rank>851</rank>
##     <title><![CDATA[Cognitive Dynamics: From Attractors to Active Inference]]></title>
##     <authors><![CDATA[Friston, K.;  SenGupta, B.;  Auletta, G.]]></authors>
##     <affiliations><![CDATA[Wellcome Trust Centre for Neuroimaging, Inst. of Neurology, London, UK]]></affiliations>
##     <controlledterms>
##       <term><![CDATA[Markov processes]]></term>
##       <term><![CDATA[belief networks]]></term>
##       <term><![CDATA[cognitive systems]]></term>
##       <term><![CDATA[inference mechanisms]]></term>
##     </controlledterms>
##     <thesaurusterms>
##       <term><![CDATA[Active filters]]></term>
##       <term><![CDATA[Bayes methods]]></term>
##       <term><![CDATA[Biological systems]]></term>
##       <term><![CDATA[Cognitive science]]></term>
##       <term><![CDATA[Markov processes]]></term>
##       <term><![CDATA[Mathematical model]]></term>
##       <term><![CDATA[Self-organizing networks]]></term>
##     </thesaurusterms>
##     <pubtitle><![CDATA[Proceedings of the IEEE]]></pubtitle>
##     <punumber><![CDATA[5]]></punumber>
##     <pubtype><![CDATA[Journals & Magazines]]></pubtype>
##     <publisher><![CDATA[IEEE]]></publisher>
##     <volume><![CDATA[102]]></volume>
##     <issue><![CDATA[4]]></issue>
##     <py><![CDATA[2014]]></py>
##     <spage><![CDATA[427]]></spage>
##     <epage><![CDATA[445]]></epage>
##     <abstract><![CDATA[This paper combines recent formulations of self-organization and neuronal processing to provide an account of cognitive dynamics from basic principles. We start by showing that inference (and autopoiesis) are emergent features of any (weakly mixing) ergodic random dynamical system. We then apply the emergent dynamics to action and perception in a way that casts action as the fulfillment of (Bayesian) beliefs about the causes of sensations. More formally, we formulate ergodic flows on global random attractors as a generalized descent on a free energy functional of the internal states of a system. This formulation rests on a partition of states based on a Markov blanket that separates internal states from hidden states in the external milieu. This separation means that the internal states effectively represent external states probabilistically. The generalized descent is then related to classical Bayesian (e.g., Kalman-Bucy) filtering and predictive coding-of the sort that might be implemented in the brain. Finally, we present two simulations. The first simulates a primordial soup to illustrate the emergence of a Markov blanket and (active) inference about hidden states. The second uses the same emergent dynamics to simulate action and action observation.]]></abstract>
##     <issn><![CDATA[0018-9219]]></issn>
##     <arnumber><![CDATA[6767058]]></arnumber>
##     <doi><![CDATA[10.1109/JPROC.2014.2306251]]></doi>
##     <publicationId><![CDATA[6767058]]></publicationId>
##     <mdurl><![CDATA[http://ieeexplore.ieee.org/xpl/articleDetails.jsp?tp=&arnumber=6767058&contentType=Journals+%26+Magazines]]></mdurl>
##     <pdf><![CDATA[http://ieeexplore.ieee.org/stamp/stamp.jsp?arnumber=6767058]]></pdf>
##   </document>
##   <document>
##     <rank>852</rank>
##     <title><![CDATA[Online Tests of an Optical Fiber Long-Period Grating Subjected to Gamma Irradiation]]></title>
##     <authors><![CDATA[Sporea, D.;  Stancalie, A.;  Negut, D.;  Pilorget, G.;  Delepine-Lesoille, S.;  Lablonde, L.]]></authors>
##     <affiliations><![CDATA[Nat. Inst. Laser, Plasma & Radiat. Phys., Ma&#x0306;gurele, Romania]]></affiliations>
##     <controlledterms>
##       <term><![CDATA[backscatter]]></term>
##       <term><![CDATA[diffraction gratings]]></term>
##       <term><![CDATA[fibre optic sensors]]></term>
##       <term><![CDATA[gamma-ray effects]]></term>
##       <term><![CDATA[optical fibre testing]]></term>
##       <term><![CDATA[optical variables measurement]]></term>
##       <term><![CDATA[radiation hardening]]></term>
##       <term><![CDATA[spectral analysers]]></term>
##     </controlledterms>
##     <thesaurusterms>
##       <term><![CDATA[Fiber gratings]]></term>
##       <term><![CDATA[Gamma-ray effects]]></term>
##       <term><![CDATA[Optical fiber sensors]]></term>
##       <term><![CDATA[Optical fibers]]></term>
##       <term><![CDATA[Radiation effects]]></term>
##       <term><![CDATA[Temperature measurement]]></term>
##       <term><![CDATA[Temperature sensors]]></term>
##     </thesaurusterms>
##     <pubtitle><![CDATA[Photonics Journal, IEEE]]></pubtitle>
##     <punumber><![CDATA[4563994]]></punumber>
##     <pubtype><![CDATA[Journals & Magazines]]></pubtype>
##     <publisher><![CDATA[IEEE]]></publisher>
##     <volume><![CDATA[6]]></volume>
##     <issue><![CDATA[6]]></issue>
##     <py><![CDATA[2014]]></py>
##     <spage><![CDATA[1]]></spage>
##     <epage><![CDATA[9]]></epage>
##     <abstract><![CDATA[This paper reports the outcomes of the tests that we conducted as online measurements for the evaluation of one optical fiber long-period grating produced by a fusion technique in a single-mode radiation-hardened optical fiber, and subjected to gamma irradiation. During the irradiation, the grating temperature was monitored. Before the irradiation, the temperature sensitivity of the grating was 27.7 pm/&#x00B0;C, while the value of this parameter postirradiation was found to be 29.3 pm/&#x00B0;C. The spectral characteristics of the grating were measured (i) in the laboratory with an ANDO AQ6317C optical spectrum analyzer and (ii) online, for the first time, with a LUNA OBR 4600 backscatter reflectometer, operating in the frequency acquisition mode. Such online measurement enables the study of recovery effects during the irradiation. The wavelength dip of the grating shifted under gamma irradiation, with 16 pm/kGy for the maximum total dose of 45 kGy. At room temperature, the recovery of the irradiation-induced shift of the wavelength dip was almost complete in about 120 h, at a rate of 6.7 pm/h. Postirradiation heating of the sensor produced the reversing of the recovery effect. The investigation indicated that, up to 45 kGy, the grating is more sensitive to radiation than other optical fiber sensors.]]></abstract>
##     <issn><![CDATA[1943-0655]]></issn>
##     <htmlFlag><![CDATA[1]]></htmlFlag>
##     <arnumber><![CDATA[6851848]]></arnumber>
##     <doi><![CDATA[10.1109/JPHOT.2014.2337877]]></doi>
##     <publicationId><![CDATA[6851848]]></publicationId>
##     <mdurl><![CDATA[http://ieeexplore.ieee.org/xpl/articleDetails.jsp?tp=&arnumber=6851848&contentType=Journals+%26+Magazines]]></mdurl>
##     <pdf><![CDATA[http://ieeexplore.ieee.org/stamp/stamp.jsp?arnumber=6851848]]></pdf>
##   </document>
##   <document>
##     <rank>853</rank>
##     <title><![CDATA[Photon-Counting Security Tagging and Verification Using Optically Encoded QR Codes]]></title>
##     <authors><![CDATA[Markman, A.;  Javidi, B.;  Tehranipoor, M.]]></authors>
##     <affiliations><![CDATA[Dept. of Electr. & Comput. Eng., Univ. of Connecticut, Storrs, CT, USA]]></affiliations>
##     <controlledterms>
##       <term><![CDATA[Huffman codes]]></term>
##       <term><![CDATA[cryptography]]></term>
##       <term><![CDATA[image coding]]></term>
##       <term><![CDATA[iterative methods]]></term>
##       <term><![CDATA[masks]]></term>
##       <term><![CDATA[phase coding]]></term>
##       <term><![CDATA[photon counting]]></term>
##       <term><![CDATA[smart phones]]></term>
##       <term><![CDATA[speckle]]></term>
##       <term><![CDATA[statistical analysis]]></term>
##     </controlledterms>
##     <thesaurusterms>
##       <term><![CDATA[Adaptive optics]]></term>
##       <term><![CDATA[Cryptography]]></term>
##       <term><![CDATA[Nonlinear optics]]></term>
##       <term><![CDATA[Optical filters]]></term>
##       <term><![CDATA[Optical imaging]]></term>
##       <term><![CDATA[Optical polarization]]></term>
##       <term><![CDATA[Photonics]]></term>
##     </thesaurusterms>
##     <pubtitle><![CDATA[Photonics Journal, IEEE]]></pubtitle>
##     <punumber><![CDATA[4563994]]></punumber>
##     <pubtype><![CDATA[Journals & Magazines]]></pubtype>
##     <publisher><![CDATA[IEEE]]></publisher>
##     <volume><![CDATA[6]]></volume>
##     <issue><![CDATA[1]]></issue>
##     <py><![CDATA[2014]]></py>
##     <spage><![CDATA[1]]></spage>
##     <epage><![CDATA[9]]></epage>
##     <abstract><![CDATA[We propose an optical security method for object authentication using photon-counting encryption implemented with phase encoded QR codes. By combining the full phase double-random-phase encryption with photon-counting imaging method and applying an iterative Huffman coding technique, we are able to encrypt and compress an image containing primary information about the object. This data can then be stored inside of an optically phase encoded QR code for robust read out, decryption, and authentication. The optically encoded QR code is verified by examining the speckle signature of the optical masks using statistical analysis. Optical experimental results are presented to demonstrate the performance of the system. In addition, experiments with a commercial Smartphone to read the optically encoded QR code are presented. To the best of our knowledge, this is the first report on integrating photon-counting security with optically phase encoded QR codes.]]></abstract>
##     <issn><![CDATA[1943-0655]]></issn>
##     <htmlFlag><![CDATA[1]]></htmlFlag>
##     <arnumber><![CDATA[6685832]]></arnumber>
##     <doi><![CDATA[10.1109/JPHOT.2013.2294625]]></doi>
##     <publicationId><![CDATA[6685832]]></publicationId>
##     <mdurl><![CDATA[http://ieeexplore.ieee.org/xpl/articleDetails.jsp?tp=&arnumber=6685832&contentType=Journals+%26+Magazines]]></mdurl>
##     <pdf><![CDATA[http://ieeexplore.ieee.org/stamp/stamp.jsp?arnumber=6685832]]></pdf>
##   </document>
##   <document>
##     <rank>854</rank>
##     <title><![CDATA[Evaluating R&amp;D Projects as Investments by Using an Overall Ranking From Four New Fuzzy Similarity Measure-Based TOPSIS Variants]]></title>
##     <authors><![CDATA[Collan, M.;  Luukka, P.]]></authors>
##     <affiliations><![CDATA[Sch. of Bus., Lappeenranta Univ. of Technol., Lappeenranta, Finland]]></affiliations>
##     <controlledterms>
##       <term><![CDATA[fuzzy set theory]]></term>
##       <term><![CDATA[investment]]></term>
##       <term><![CDATA[research and development]]></term>
##     </controlledterms>
##     <pubtitle><![CDATA[Fuzzy Systems, IEEE Transactions on]]></pubtitle>
##     <punumber><![CDATA[91]]></punumber>
##     <pubtype><![CDATA[Journals & Magazines]]></pubtype>
##     <publisher><![CDATA[IEEE]]></publisher>
##     <volume><![CDATA[22]]></volume>
##     <issue><![CDATA[3]]></issue>
##     <py><![CDATA[2014]]></py>
##     <spage><![CDATA[505]]></spage>
##     <epage><![CDATA[515]]></epage>
##     <abstract><![CDATA[Research and development (R&amp;D) project ranking as investments is a well-known problem that is made difficult by incomplete and imprecise information about future project profitability. This paper shows how profitability results of R&amp;D project evaluation with the fuzzy pay-off method can be ranked with four new variants of fuzzy TOPSIS each using a different fuzzy similarity measure. An overall project ranking that incorporates the four new variants' rankings with three different ideal solutions totaling 12 subrankings is presented. The implementation of the created methods is illustrated with a numerical example.]]></abstract>
##     <issn><![CDATA[1063-6706]]></issn>
##     <htmlFlag><![CDATA[1]]></htmlFlag>
##     <arnumber><![CDATA[6512571]]></arnumber>
##     <doi><![CDATA[10.1109/TFUZZ.2013.2260758]]></doi>
##     <publicationId><![CDATA[6512571]]></publicationId>
##     <mdurl><![CDATA[http://ieeexplore.ieee.org/xpl/articleDetails.jsp?tp=&arnumber=6512571&contentType=Journals+%26+Magazines]]></mdurl>
##     <pdf><![CDATA[http://ieeexplore.ieee.org/stamp/stamp.jsp?arnumber=6512571]]></pdf>
##   </document>
##   <document>
##     <rank>855</rank>
##     <title><![CDATA[Optical Interaction of a Pair of Nanoholes in Au Film via Surface Plasmon Polaritons]]></title>
##     <authors><![CDATA[Janipour, M.;  Pakizeh, T.;  Hodjat-Kashani, F.]]></authors>
##     <affiliations><![CDATA[Dept. of Electr. Eng., Iran Univ. of Sci. & Technol., Tehran, Iran]]></affiliations>
##     <controlledterms>
##       <term><![CDATA[gold]]></term>
##       <term><![CDATA[light polarisation]]></term>
##       <term><![CDATA[metallic thin films]]></term>
##       <term><![CDATA[nanophotonics]]></term>
##       <term><![CDATA[polaritons]]></term>
##       <term><![CDATA[surface plasmons]]></term>
##     </controlledterms>
##     <thesaurusterms>
##       <term><![CDATA[Gold]]></term>
##       <term><![CDATA[Magnetic separation]]></term>
##       <term><![CDATA[Optical films]]></term>
##       <term><![CDATA[Optical polarization]]></term>
##       <term><![CDATA[Optical surface waves]]></term>
##     </thesaurusterms>
##     <pubtitle><![CDATA[Photonics Journal, IEEE]]></pubtitle>
##     <punumber><![CDATA[4563994]]></punumber>
##     <pubtype><![CDATA[Journals & Magazines]]></pubtype>
##     <publisher><![CDATA[IEEE]]></publisher>
##     <volume><![CDATA[6]]></volume>
##     <issue><![CDATA[3]]></issue>
##     <py><![CDATA[2014]]></py>
##     <spage><![CDATA[1]]></spage>
##     <epage><![CDATA[13]]></epage>
##     <abstract><![CDATA[The optical interaction mechanism of a nanohole pair milled in an opaque gold film by means of surface plasmon polaritons (SPPs) propagation is investigated. This interaction depends on the polarization direction of the incident light and the separation distance between the nanoholes. It is found that when the nanoholes are illuminated by a plane wave incident light polarized parallel to the axis of the nanohole pair, the SPP waves can propagate between the nanoholes, and therefore, the nanoholes can interact through SPPs. In contrast, it is shown that for the incident plane wave polarized perpendicular to the pair axis, the propagation direction of the SPPs is normal to the pair axis, and thus, a weak interaction through the SPPs can occur between the nanoholes. It is also shown that in order to investigate the interaction of the nanoholes through the SPP waves, each nanohole can be modeled by a magnetic dipole, which propagates SPPs. Thus, the optical properties of the interacting nanoholes can be modeled using a magnetic-coupled dipole approximation method accounting SPPs to confirm the simulation results.]]></abstract>
##     <issn><![CDATA[1943-0655]]></issn>
##     <htmlFlag><![CDATA[1]]></htmlFlag>
##     <arnumber><![CDATA[6820732]]></arnumber>
##     <doi><![CDATA[10.1109/JPHOT.2014.2326653]]></doi>
##     <publicationId><![CDATA[6820732]]></publicationId>
##     <mdurl><![CDATA[http://ieeexplore.ieee.org/xpl/articleDetails.jsp?tp=&arnumber=6820732&contentType=Journals+%26+Magazines]]></mdurl>
##     <pdf><![CDATA[http://ieeexplore.ieee.org/stamp/stamp.jsp?arnumber=6820732]]></pdf>
##   </document>
##   <document>
##     <rank>856</rank>
##     <title><![CDATA[Ultra-Wideband Dual-Polarized Patch Antenna With Four Capacitively Coupled Feeds]]></title>
##     <authors><![CDATA[Fuguo Zhu;  Gao, S.;  Ho, A.T.S.;  Abd-Alhameed, R.A.;  See, C.H.;  Brown, T.W.C.;  Jianzhou Li;  Gao Wei;  Jiadong Xu]]></authors>
##     <affiliations><![CDATA[Sch. of Eng. & Digital Arts, Univ. of Kent, Canterbury, UK]]></affiliations>
##     <controlledterms>
##       <term><![CDATA[antenna feeds]]></term>
##       <term><![CDATA[baluns]]></term>
##       <term><![CDATA[microstrip antennas]]></term>
##       <term><![CDATA[microstrip lines]]></term>
##       <term><![CDATA[surface mount technology]]></term>
##       <term><![CDATA[ultra wideband antennas]]></term>
##     </controlledterms>
##     <thesaurusterms>
##       <term><![CDATA[Antenna feeds]]></term>
##       <term><![CDATA[Antenna measurements]]></term>
##       <term><![CDATA[Bandwidth]]></term>
##       <term><![CDATA[Impedance matching]]></term>
##       <term><![CDATA[Patch antennas]]></term>
##       <term><![CDATA[Ultra wideband antennas]]></term>
##     </thesaurusterms>
##     <pubtitle><![CDATA[Antennas and Propagation, IEEE Transactions on]]></pubtitle>
##     <punumber><![CDATA[8]]></punumber>
##     <pubtype><![CDATA[Journals & Magazines]]></pubtype>
##     <publisher><![CDATA[IEEE]]></publisher>
##     <volume><![CDATA[62]]></volume>
##     <issue><![CDATA[5]]></issue>
##     <py><![CDATA[2014]]></py>
##     <spage><![CDATA[2440]]></spage>
##     <epage><![CDATA[2449]]></epage>
##     <abstract><![CDATA[A novel dual-polarized patch antenna for ultra-wideband (UWB) applications is presented. The antenna consists of a square patch and four capacitively coupled feeds to enhance the impedance bandwidth. Each feed is formed by a vertical isosceles trapezoidal patch and a horizontal isosceles triangular patch. The four feeds are connected to the microstrip lines that are printed on the bottom layer of the grounded FR4 substrate. Two tapered baluns are utilized to excite the antenna to achieve high isolation between the ports and reduce the cross-polarization levels. In order to increase the antenna gain and reduce the backward radiation, a compact surface mounted cavity is integrated with the antenna. The antenna prototype has achieved an impedance bandwidth of 112% at (|S11 | &#x2264; -10 dB) whereas the coupling between the two ports is below -28 dB across the operating frequency range. The measured antenna gain varies from 3.91 to 10.2 dBi for port 1 and from 3.38 to 9.21 dBi for port 2, with a 3-dB gain bandwidth of 107%.]]></abstract>
##     <issn><![CDATA[0018-926X]]></issn>
##     <htmlFlag><![CDATA[1]]></htmlFlag>
##     <arnumber><![CDATA[6748899]]></arnumber>
##     <doi><![CDATA[10.1109/TAP.2014.2308524]]></doi>
##     <publicationId><![CDATA[6748899]]></publicationId>
##     <mdurl><![CDATA[http://ieeexplore.ieee.org/xpl/articleDetails.jsp?tp=&arnumber=6748899&contentType=Journals+%26+Magazines]]></mdurl>
##     <pdf><![CDATA[http://ieeexplore.ieee.org/stamp/stamp.jsp?arnumber=6748899]]></pdf>
##   </document>
##   <document>
##     <rank>857</rank>
##     <title><![CDATA[A New Unconditionally Stable Scheme for FDTD Method Using Associated Hermite Orthogonal Functions]]></title>
##     <authors><![CDATA[Zheng-Yu Huang;  Li-Hua Shi;  Bin Chen;  Ying-Hui Zhou]]></authors>
##     <affiliations><![CDATA[Nat. Key Lab. on Electromagn. Environ. Effects & Electro-Opt. Eng., PLA Univ. of Sci. & Technol., Nanjing, China]]></affiliations>
##     <controlledterms>
##       <term><![CDATA[Galerkin method]]></term>
##       <term><![CDATA[Maxwell equations]]></term>
##       <term><![CDATA[computational electromagnetics]]></term>
##       <term><![CDATA[electromagnetic fields]]></term>
##       <term><![CDATA[finite difference time-domain analysis]]></term>
##     </controlledterms>
##     <thesaurusterms>
##       <term><![CDATA[Equations]]></term>
##       <term><![CDATA[Finite difference methods]]></term>
##       <term><![CDATA[Magnetic fields]]></term>
##       <term><![CDATA[Mathematical model]]></term>
##       <term><![CDATA[Matrix decomposition]]></term>
##       <term><![CDATA[Time-domain analysis]]></term>
##       <term><![CDATA[Vectors]]></term>
##     </thesaurusterms>
##     <pubtitle><![CDATA[Antennas and Propagation, IEEE Transactions on]]></pubtitle>
##     <punumber><![CDATA[8]]></punumber>
##     <pubtype><![CDATA[Journals & Magazines]]></pubtype>
##     <publisher><![CDATA[IEEE]]></publisher>
##     <volume><![CDATA[62]]></volume>
##     <issue><![CDATA[9]]></issue>
##     <py><![CDATA[2014]]></py>
##     <spage><![CDATA[4804]]></spage>
##     <epage><![CDATA[4809]]></epage>
##     <abstract><![CDATA[An unconditionally stable solution using associated Hermite (AH) functions is proposed for the finite-difference time-domain (FDTD) method. The electromagnetic fields and their time derivatives in time-domain Maxwell's equations are expanded by these orthonormal basis functions. By applying Galerkin temporal testing procedure to these expanded equations the time variable can be eliminated from the calculations. A set of implicit equations is derived to calculate the magnetic filed expansion coefficients of all orders of AH functions for the temporal variable. And the electrical field coefficients can be obtained respectively. With the appropriate translation and scale parameters, we can find a minimum-order basis functions subspace to approach a particular electromagnetic field. The numerical results have shown that the proposed method can reduce the CPU time to 0.59% of the traditional FDTD method while maintaining good accuracy.]]></abstract>
##     <issn><![CDATA[0018-926X]]></issn>
##     <htmlFlag><![CDATA[1]]></htmlFlag>
##     <arnumber><![CDATA[6822549]]></arnumber>
##     <doi><![CDATA[10.1109/TAP.2014.2327141]]></doi>
##     <publicationId><![CDATA[6822549]]></publicationId>
##     <mdurl><![CDATA[http://ieeexplore.ieee.org/xpl/articleDetails.jsp?tp=&arnumber=6822549&contentType=Journals+%26+Magazines]]></mdurl>
##     <pdf><![CDATA[http://ieeexplore.ieee.org/stamp/stamp.jsp?arnumber=6822549]]></pdf>
##   </document>
##   <document>
##     <rank>858</rank>
##     <title><![CDATA[High-Gain, High Transmissibility PZT Displacement Amplification Using a Rolling-Contact Buckling Mechanism and Preload Compensation Springs]]></title>
##     <authors><![CDATA[Torres, J.;  Asada, H.H.]]></authors>
##     <affiliations><![CDATA[Dept. of Mech. Eng., Massachusetts Inst. of Technol., Cambridge, MA, USA]]></affiliations>
##     <controlledterms>
##       <term><![CDATA[buckling]]></term>
##       <term><![CDATA[elasticity]]></term>
##       <term><![CDATA[mechanical contact]]></term>
##       <term><![CDATA[piezoelectric actuators]]></term>
##       <term><![CDATA[power transmission (mechanical)]]></term>
##       <term><![CDATA[rolling friction]]></term>
##       <term><![CDATA[springs (mechanical)]]></term>
##     </controlledterms>
##     <thesaurusterms>
##       <term><![CDATA[Actuators]]></term>
##       <term><![CDATA[Force]]></term>
##       <term><![CDATA[Friction]]></term>
##       <term><![CDATA[Joints]]></term>
##       <term><![CDATA[Robots]]></term>
##       <term><![CDATA[Springs]]></term>
##       <term><![CDATA[Torque]]></term>
##     </thesaurusterms>
##     <pubtitle><![CDATA[Robotics, IEEE Transactions on]]></pubtitle>
##     <punumber><![CDATA[8860]]></punumber>
##     <pubtype><![CDATA[Journals & Magazines]]></pubtype>
##     <publisher><![CDATA[IEEE]]></publisher>
##     <volume><![CDATA[30]]></volume>
##     <issue><![CDATA[4]]></issue>
##     <py><![CDATA[2014]]></py>
##     <spage><![CDATA[781]]></spage>
##     <epage><![CDATA[791]]></epage>
##     <abstract><![CDATA[A novel design concept of piezoelectric actuators producing large displacement while transmitting a significant amount of energy is presented. A rolling-contact buckling mechanism with a novel preload mechanism can amplify the PZT stack's displacement on the order of 100 times while transmitting several times larger work output than conventional flexure-type displacement amplification mechanisms. Existing displacement amplification mechanisms are analyzed in terms of transmissibility and are characterized with two lumped-parameter elements: serial and parallel compliances. The maximum transmissibility is attained when the parallel stiffness and the serial compliance are zero. An existing flexure mechanism using structural buckling, that produces a large displacement but a low transmissibility, is replaced by a rolling-contact mechanism that approaches the maximum criterion. Furthermore, a mechanism is presented to apply a constant preload to each PZT stack despite their movement. A prototype has been built to implement the design concept and verify the theoretical results. Experiments using the prototype demonstrate that it produces a 4.2 mm free displacement with over 60% transmissibility.]]></abstract>
##     <issn><![CDATA[1552-3098]]></issn>
##     <htmlFlag><![CDATA[1]]></htmlFlag>
##     <arnumber><![CDATA[6750118]]></arnumber>
##     <doi><![CDATA[10.1109/TRO.2014.2301535]]></doi>
##     <publicationId><![CDATA[6750118]]></publicationId>
##     <mdurl><![CDATA[http://ieeexplore.ieee.org/xpl/articleDetails.jsp?tp=&arnumber=6750118&contentType=Journals+%26+Magazines]]></mdurl>
##     <pdf><![CDATA[http://ieeexplore.ieee.org/stamp/stamp.jsp?arnumber=6750118]]></pdf>
##   </document>
##   <document>
##     <rank>859</rank>
##     <title><![CDATA[Optimal Battery Dimensioning and Control of a CVT PHEV Powertrain]]></title>
##     <authors><![CDATA[Murgovski, N.;  Johannesson, L.M.;  Egardt, B.]]></authors>
##     <affiliations><![CDATA[Dept. of Signals & Syst., Chalmers Univ. of Technol., Gothenburg, Sweden]]></affiliations>
##     <controlledterms>
##       <term><![CDATA[battery powered vehicles]]></term>
##       <term><![CDATA[convex programming]]></term>
##       <term><![CDATA[hybrid electric vehicles]]></term>
##       <term><![CDATA[internal combustion engines]]></term>
##       <term><![CDATA[optimal control]]></term>
##       <term><![CDATA[power transmission (mechanical)]]></term>
##     </controlledterms>
##     <thesaurusterms>
##       <term><![CDATA[Batteries]]></term>
##       <term><![CDATA[Gears]]></term>
##       <term><![CDATA[Ice]]></term>
##       <term><![CDATA[Mechanical power transmission]]></term>
##       <term><![CDATA[Optimization]]></term>
##       <term><![CDATA[Torque]]></term>
##       <term><![CDATA[Vehicles]]></term>
##     </thesaurusterms>
##     <pubtitle><![CDATA[Vehicular Technology, IEEE Transactions on]]></pubtitle>
##     <punumber><![CDATA[25]]></punumber>
##     <pubtype><![CDATA[Journals & Magazines]]></pubtype>
##     <publisher><![CDATA[IEEE]]></publisher>
##     <volume><![CDATA[63]]></volume>
##     <issue><![CDATA[5]]></issue>
##     <py><![CDATA[2014]]></py>
##     <spage><![CDATA[2151]]></spage>
##     <epage><![CDATA[2161]]></epage>
##     <abstract><![CDATA[This paper presents convex modeling steps for the problem of optimal battery dimensioning and control of a plug-in hybrid electric vehicle with a continuous variable transmission. The power limits of the internal combustion engine and the electric machine are approximated as convex/concave functions in kinetic energy, whereas their losses are approximated as convex in both kinetic energy and power. An example of minimizing the total cost of ownership of a city bus including a battery wear model is presented. The proposed method is also used to obtain optimal charging power from an infrastructure that is to be designed at the same time the bus is dimensioned.]]></abstract>
##     <issn><![CDATA[0018-9545]]></issn>
##     <htmlFlag><![CDATA[1]]></htmlFlag>
##     <arnumber><![CDATA[6662478]]></arnumber>
##     <doi><![CDATA[10.1109/TVT.2013.2290601]]></doi>
##     <publicationId><![CDATA[6662478]]></publicationId>
##     <mdurl><![CDATA[http://ieeexplore.ieee.org/xpl/articleDetails.jsp?tp=&arnumber=6662478&contentType=Journals+%26+Magazines]]></mdurl>
##     <pdf><![CDATA[http://ieeexplore.ieee.org/stamp/stamp.jsp?arnumber=6662478]]></pdf>
##   </document>
##   <document>
##     <rank>860</rank>
##     <title><![CDATA[Investigation of Efficiency and Droop Behavior Comparison for InGaN/GaN Super Wide-Well Light Emitting Diodes Grown on Different Substrates]]></title>
##     <authors><![CDATA[Tongbo Wei;  Lian Zhang;  Xiaoli Ji;  Junxi Wang;  Ziqiang Huo;  Baojun Sun;  Qiang Hu;  Xuecheng Wei;  Ruifei Duan;  Lixia Zhao;  Yiping Zeng;  Jinmin Li]]></authors>
##     <affiliations><![CDATA[State Key Lab. of Solid-State Lighting, Inst. of Semicond., Beijing, China]]></affiliations>
##     <controlledterms>
##       <term><![CDATA[III-V semiconductors]]></term>
##       <term><![CDATA[current density]]></term>
##       <term><![CDATA[gallium compounds]]></term>
##       <term><![CDATA[indium compounds]]></term>
##       <term><![CDATA[light emitting diodes]]></term>
##       <term><![CDATA[quantum well devices]]></term>
##       <term><![CDATA[sapphire]]></term>
##       <term><![CDATA[stress relaxation]]></term>
##       <term><![CDATA[substrates]]></term>
##       <term><![CDATA[wide band gap semiconductors]]></term>
##     </controlledterms>
##     <thesaurusterms>
##       <term><![CDATA[Gallium nitride]]></term>
##       <term><![CDATA[Light emitting diodes]]></term>
##       <term><![CDATA[Lighting]]></term>
##       <term><![CDATA[Quantum well devices]]></term>
##       <term><![CDATA[Strain]]></term>
##       <term><![CDATA[Substrates]]></term>
##       <term><![CDATA[Surface morphology]]></term>
##     </thesaurusterms>
##     <pubtitle><![CDATA[Photonics Journal, IEEE]]></pubtitle>
##     <punumber><![CDATA[4563994]]></punumber>
##     <pubtype><![CDATA[Journals & Magazines]]></pubtype>
##     <publisher><![CDATA[IEEE]]></publisher>
##     <volume><![CDATA[6]]></volume>
##     <issue><![CDATA[6]]></issue>
##     <py><![CDATA[2014]]></py>
##     <spage><![CDATA[1]]></spage>
##     <epage><![CDATA[10]]></epage>
##     <abstract><![CDATA[In this work, efficiency droop of InGaN/GaN multiple-quantum-well LEDs with super wide well (WW) is discussed by comparing the external quantum efficiency (EQE) of GaN grown on sapphire and FS-GaN substrates. The luminescence and electrical characteristics of these WW LEDs are also experimentally and theoretically analyzed. With the increase of well width from 3 nm to 6 nm, high V-pits density and more strain relaxation are found in WW LED on sapphire, which exhibits greatly reduced peak efficiency but almost negligible droop behavior. In contrast, despite a larger polarization field, WW LED on FS-GaN shows obviously enhanced peak efficiency and comparable droop compared to the counterpart with 3-nm well. The Auger recombination probably dominates the mechanism of efficiency droop rather than defect-related nonradiative recombination or polarization effect in the WW LED on both sapphire and FS-GaN, especially at high current density.]]></abstract>
##     <issn><![CDATA[1943-0655]]></issn>
##     <htmlFlag><![CDATA[1]]></htmlFlag>
##     <arnumber><![CDATA[6939631]]></arnumber>
##     <doi><![CDATA[10.1109/JPHOT.2014.2363428]]></doi>
##     <publicationId><![CDATA[6939631]]></publicationId>
##     <mdurl><![CDATA[http://ieeexplore.ieee.org/xpl/articleDetails.jsp?tp=&arnumber=6939631&contentType=Journals+%26+Magazines]]></mdurl>
##     <pdf><![CDATA[http://ieeexplore.ieee.org/stamp/stamp.jsp?arnumber=6939631]]></pdf>
##   </document>
##   <document>
##     <rank>861</rank>
##     <title><![CDATA[Tunable Erbium-Doped Fiber Laser Based on Random Distributed Feedback]]></title>
##     <authors><![CDATA[Lulu Wang;  Xinyong Dong;  Shum, P.P.;  Haibin Su]]></authors>
##     <affiliations><![CDATA[Inst. of Optoelectron. Technol., China Jiliang Univ., Hangzhou, China]]></affiliations>
##     <controlledterms>
##       <term><![CDATA[Fabry-Perot interferometers]]></term>
##       <term><![CDATA[Rayleigh scattering]]></term>
##       <term><![CDATA[erbium]]></term>
##       <term><![CDATA[fibre lasers]]></term>
##       <term><![CDATA[interference filters]]></term>
##       <term><![CDATA[laser cavity resonators]]></term>
##       <term><![CDATA[laser feedback]]></term>
##       <term><![CDATA[laser tuning]]></term>
##       <term><![CDATA[optical pumping]]></term>
##     </controlledterms>
##     <thesaurusterms>
##       <term><![CDATA[Distributed feedback devices]]></term>
##       <term><![CDATA[Erbium-doped fiber lasers]]></term>
##       <term><![CDATA[Laser excitation]]></term>
##       <term><![CDATA[Laser feedback]]></term>
##       <term><![CDATA[Optical fibers]]></term>
##       <term><![CDATA[Power lasers]]></term>
##       <term><![CDATA[Pump lasers]]></term>
##     </thesaurusterms>
##     <pubtitle><![CDATA[Photonics Journal, IEEE]]></pubtitle>
##     <punumber><![CDATA[4563994]]></punumber>
##     <pubtype><![CDATA[Journals & Magazines]]></pubtype>
##     <publisher><![CDATA[IEEE]]></publisher>
##     <volume><![CDATA[6]]></volume>
##     <issue><![CDATA[5]]></issue>
##     <py><![CDATA[2014]]></py>
##     <spage><![CDATA[1]]></spage>
##     <epage><![CDATA[5]]></epage>
##     <abstract><![CDATA[A tunable erbium-doped fiber (EDF) laser based on random distributed feedback through backward Rayleigh scattering in a 20-km-long single-mode fiber and a tunable fiber Fabry-Perot interferometer filter is demonstrated. It has a broad wavelength tuning range up to 40 nm (from 1525 to 1565 nm) with a narrow linewidth of ~0.04 nm. Due to the half-opened laser cavity design and the efficient gain from the pumped EDF, the measured threshold power is only 13 mW, and the pump efficiency is up to 14%.]]></abstract>
##     <issn><![CDATA[1943-0655]]></issn>
##     <htmlFlag><![CDATA[1]]></htmlFlag>
##     <arnumber><![CDATA[6884783]]></arnumber>
##     <doi><![CDATA[10.1109/JPHOT.2014.2352623]]></doi>
##     <publicationId><![CDATA[6884783]]></publicationId>
##     <mdurl><![CDATA[http://ieeexplore.ieee.org/xpl/articleDetails.jsp?tp=&arnumber=6884783&contentType=Journals+%26+Magazines]]></mdurl>
##     <pdf><![CDATA[http://ieeexplore.ieee.org/stamp/stamp.jsp?arnumber=6884783]]></pdf>
##   </document>
##   <document>
##     <rank>862</rank>
##     <title><![CDATA[Semantic Link Network-Based Model for Organizing Multimedia Big Data]]></title>
##     <authors><![CDATA[Chuanping Hu;  Zheng Xu;  Yunhuai Liu;  Lin Mei;  Lan Chen;  Xiangfeng Luo]]></authors>
##     <affiliations><![CDATA[Minist. of Public Security, Third Res. Inst., Shanghai, China]]></affiliations>
##     <controlledterms>
##       <term><![CDATA[Big Data]]></term>
##       <term><![CDATA[multimedia computing]]></term>
##       <term><![CDATA[semantic networks]]></term>
##     </controlledterms>
##     <thesaurusterms>
##       <term><![CDATA[Big data]]></term>
##       <term><![CDATA[Computational modeling]]></term>
##       <term><![CDATA[Data models]]></term>
##       <term><![CDATA[Multimedia communication]]></term>
##       <term><![CDATA[Ontologies]]></term>
##       <term><![CDATA[Semantics]]></term>
##       <term><![CDATA[Streaming media]]></term>
##     </thesaurusterms>
##     <pubtitle><![CDATA[Emerging Topics in Computing, IEEE Transactions on]]></pubtitle>
##     <punumber><![CDATA[6245516]]></punumber>
##     <pubtype><![CDATA[Journals & Magazines]]></pubtype>
##     <publisher><![CDATA[IEEE]]></publisher>
##     <volume><![CDATA[2]]></volume>
##     <issue><![CDATA[3]]></issue>
##     <py><![CDATA[2014]]></py>
##     <spage><![CDATA[376]]></spage>
##     <epage><![CDATA[387]]></epage>
##     <abstract><![CDATA[Recent research shows that multimedia resources in the wild are growing at a staggering rate. The rapid increase number of multimedia resources has brought an urgent need to develop intelligent methods to organize and process them. In this paper, the semantic link network model is used for organizing multimedia resources. A whole model for generating the association relation between multimedia resources using semantic link network model is proposed. The definitions, modules, and mechanisms of the semantic link network are used in the proposed method. The integration between the semantic link network and multimedia resources provides a new prospect for organizing them with their semantics. The tags and the surrounding texts of multimedia resources are used to measure their semantic association. The hierarchical semantic of multimedia resources is defined by their annotated tags and surrounding texts. The semantics of tags and surrounding texts are different in the proposed framework. The modules of semantic link network model are implemented to measure association relations. A real data set including 100 thousand images with social tags from Flickr is used in our experiments. Two evaluation methods, including clustering and retrieval, are performed, which shows the proposed method can measure the semantic relatedness between Flickr images accurately and robustly.]]></abstract>
##     <issn><![CDATA[2168-6750]]></issn>
##     <htmlFlag><![CDATA[1]]></htmlFlag>
##     <arnumber><![CDATA[6786371]]></arnumber>
##     <doi><![CDATA[10.1109/TETC.2014.2316525]]></doi>
##     <publicationId><![CDATA[6786371]]></publicationId>
##     <mdurl><![CDATA[http://ieeexplore.ieee.org/xpl/articleDetails.jsp?tp=&arnumber=6786371&contentType=Journals+%26+Magazines]]></mdurl>
##     <pdf><![CDATA[http://ieeexplore.ieee.org/stamp/stamp.jsp?arnumber=6786371]]></pdf>
##   </document>
##   <document>
##     <rank>863</rank>
##     <title><![CDATA[Successive Developmental Levels of Autobiographical Memory for Learning Through Social Interaction]]></title>
##     <authors><![CDATA[Pointeau, G.;  Petit, M.;  Dominey, P.F.]]></authors>
##     <affiliations><![CDATA[Stem Cell & Brain Res. Inst., Robot Cognition Lab., INSERM, Bron, France]]></affiliations>
##     <controlledterms>
##       <term><![CDATA[control engineering computing]]></term>
##       <term><![CDATA[human-robot interaction]]></term>
##       <term><![CDATA[humanoid robots]]></term>
##       <term><![CDATA[learning (artificial intelligence)]]></term>
##     </controlledterms>
##     <thesaurusterms>
##       <term><![CDATA[Cognition]]></term>
##       <term><![CDATA[Context]]></term>
##       <term><![CDATA[Dispersion]]></term>
##       <term><![CDATA[Robot kinematics]]></term>
##       <term><![CDATA[Semantics]]></term>
##       <term><![CDATA[Vectors]]></term>
##     </thesaurusterms>
##     <pubtitle><![CDATA[Autonomous Mental Development, IEEE Transactions on]]></pubtitle>
##     <punumber><![CDATA[4563672]]></punumber>
##     <pubtype><![CDATA[Journals & Magazines]]></pubtype>
##     <publisher><![CDATA[IEEE]]></publisher>
##     <volume><![CDATA[6]]></volume>
##     <issue><![CDATA[3]]></issue>
##     <py><![CDATA[2014]]></py>
##     <spage><![CDATA[200]]></spage>
##     <epage><![CDATA[212]]></epage>
##     <abstract><![CDATA[A developing cognitive system will ideally acquire knowledge of its interaction in the world, and will be able to use that knowledge to construct a scaffolding for progressively structured levels of behavior. The current research implements and tests an autobiographical memory system by which a humanoid robot, the iCub, can accumulate its experience in interacting with humans, and extract regularities that characterize this experience. This knowledge is then used in order to form composite representations of common experiences. We first apply this to the development of knowledge of spatial locations, and relations between objects in space. We then demonstrate how this can be extended to temporal relations between events, including &#x201C;before&#x201D; and &#x201C;after,&#x201D; which structure the occurrence of events in time. In the system, after extended sessions of interaction with a human, the resulting accumulated experience is processed in an offline manner, in a form of consolidation, during which common elements of different experiences are generalized in order to generate new meanings. These learned meanings then form the basis for simple behaviors that, when encoded in the autobiographical memory, can form the basis for memories of shared experiences with the human, and which can then be reused as a form of game playing or shared plan execution.]]></abstract>
##     <issn><![CDATA[1943-0604]]></issn>
##     <htmlFlag><![CDATA[1]]></htmlFlag>
##     <arnumber><![CDATA[6784467]]></arnumber>
##     <doi><![CDATA[10.1109/TAMD.2014.2307342]]></doi>
##     <publicationId><![CDATA[6784467]]></publicationId>
##     <mdurl><![CDATA[http://ieeexplore.ieee.org/xpl/articleDetails.jsp?tp=&arnumber=6784467&contentType=Journals+%26+Magazines]]></mdurl>
##     <pdf><![CDATA[http://ieeexplore.ieee.org/stamp/stamp.jsp?arnumber=6784467]]></pdf>
##   </document>
##   <document>
##     <rank>864</rank>
##     <title><![CDATA[A Carpooling Recommendation System for Taxicab Services]]></title>
##     <authors><![CDATA[Desheng Zhang;  Tian He;  Yunhuai Liu;  Shan Lin;  Stankovic, J.A.]]></authors>
##     <affiliations><![CDATA[Dept. of Comput. Sci. & Eng., Univ. of Minnesota, Minneapolis, MN, USA]]></affiliations>
##     <controlledterms>
##       <term><![CDATA[recommender systems]]></term>
##       <term><![CDATA[traffic engineering computing]]></term>
##       <term><![CDATA[ubiquitous computing]]></term>
##     </controlledterms>
##     <thesaurusterms>
##       <term><![CDATA[Cities and towns]]></term>
##       <term><![CDATA[Dispatching]]></term>
##       <term><![CDATA[Global Positioning System]]></term>
##       <term><![CDATA[Public transportation]]></term>
##       <term><![CDATA[Real-time systems]]></term>
##       <term><![CDATA[Transportation]]></term>
##       <term><![CDATA[Urban areas]]></term>
##     </thesaurusterms>
##     <pubtitle><![CDATA[Emerging Topics in Computing, IEEE Transactions on]]></pubtitle>
##     <punumber><![CDATA[6245516]]></punumber>
##     <pubtype><![CDATA[Journals & Magazines]]></pubtype>
##     <publisher><![CDATA[IEEE]]></publisher>
##     <volume><![CDATA[2]]></volume>
##     <issue><![CDATA[3]]></issue>
##     <py><![CDATA[2014]]></py>
##     <spage><![CDATA[254]]></spage>
##     <epage><![CDATA[266]]></epage>
##     <abstract><![CDATA[Carpooling taxicab services hold the promise of providing additional transportation supply, especially in the extreme weather or rush hour when regular taxicab services are insufficient. Although many recommendation systems about regular taxicab services have been proposed recently, little research, if any, has been done to assist passengers to find a successful taxicab ride with carpooling. In this paper, we present the first systematic work to design a unified recommendation system for both the regular and carpooling services, called CallCab, based on a data-driven approach. In response to a passenger's real-time request, CallCab aims to recommend either: 1) a vacant taxicab for a regular service with no detour or 2) an occupied taxicab heading to the similar direction for a carpooling service with the minimum detour, yet without assuming any knowledge of destinations of passengers already in taxicabs. To analyze these unknown destinations of occupied taxicabs, CallCab generates and refines taxicab trip distributions based on GPS data sets and context information collected in the existing taxicab infrastructure. To improve CallCab's efficiency to process such a big data set, we augment the efficient MapReduce model with a Measure phase tailored for our CallCab. Finally, we design a reciprocal price mechanism to facilitate the taxicab carpooling implementation in the real world. We evaluate CallCab with a real-world data set of 14000 taxicabs, and results show that compared with the ground truth, CallCab reduces 60% of the total mileage to deliver all passengers and 41% of passenger's waiting time. Our price mechanism reduces 23% of passengers' fares and increases 28% of drivers' profits simultaneously.]]></abstract>
##     <issn><![CDATA[2168-6750]]></issn>
##     <htmlFlag><![CDATA[1]]></htmlFlag>
##     <arnumber><![CDATA[6894613]]></arnumber>
##     <doi><![CDATA[10.1109/TETC.2014.2356493]]></doi>
##     <publicationId><![CDATA[6894613]]></publicationId>
##     <mdurl><![CDATA[http://ieeexplore.ieee.org/xpl/articleDetails.jsp?tp=&arnumber=6894613&contentType=Journals+%26+Magazines]]></mdurl>
##     <pdf><![CDATA[http://ieeexplore.ieee.org/stamp/stamp.jsp?arnumber=6894613]]></pdf>
##   </document>
##   <document>
##     <rank>865</rank>
##     <title><![CDATA[Parasitic Stimulated Amplification in High-Peak-Power and Diode-Seeded Nanosecond Fiber Amplifiers]]></title>
##     <authors><![CDATA[Chang, C.L.;  Lai, P.Y.;  Li, Y.Y.;  Lai, Y.P.;  Huang, C.W.;  Chen, S.H.;  Lee, Y.W.;  Huang, S.L.]]></authors>
##     <affiliations><![CDATA[Inst. of Photonics & Optoelectron., Nat. Taiwan Univ., Taipei, Taiwan]]></affiliations>
##     <controlledterms>
##       <term><![CDATA[optical fibre amplifiers]]></term>
##       <term><![CDATA[optical filters]]></term>
##       <term><![CDATA[semiconductor lasers]]></term>
##       <term><![CDATA[ytterbium]]></term>
##     </controlledterms>
##     <thesaurusterms>
##       <term><![CDATA[Bandwidth]]></term>
##       <term><![CDATA[Fiber lasers]]></term>
##       <term><![CDATA[Mathematical model]]></term>
##       <term><![CDATA[Optical fiber amplifiers]]></term>
##       <term><![CDATA[Optical fiber polarization]]></term>
##       <term><![CDATA[Optical fiber theory]]></term>
##       <term><![CDATA[Semiconductor lasers]]></term>
##     </thesaurusterms>
##     <pubtitle><![CDATA[Photonics Journal, IEEE]]></pubtitle>
##     <punumber><![CDATA[4563994]]></punumber>
##     <pubtype><![CDATA[Journals & Magazines]]></pubtype>
##     <publisher><![CDATA[IEEE]]></publisher>
##     <volume><![CDATA[6]]></volume>
##     <issue><![CDATA[3]]></issue>
##     <py><![CDATA[2014]]></py>
##     <spage><![CDATA[1]]></spage>
##     <epage><![CDATA[8]]></epage>
##     <abstract><![CDATA[The broadband parasitic amplification in a diode-seeded nanosecond ytterbium-doped fiber laser amplifier system is numerically and experimentally investigated. The amplification is originated from a weak and pulsed parasitic signal associated with the 1064-nm seed diode laser. Although the average power of the parasitic pulse is less than 5% of the total seed laser power, a significant transient spike is observed during the amplification. In agreement with the simulation, nonlinear effects caused by the transient spike limits the scaling of signal peak power in fiber preamplifiers. With the utilization of a narrow bandwidth filter to eliminate the parasitic pulse, the power and energy scalability of a multistage diode-seeded fiber amplifier laser system has been significantly improved. At 1064 nm, pulses with the peak power of 120 kW and energy of 1.2 mJ have been successfully generated in the multistage Yb<sup>3+</sup>-doped fiber amplifier with an energy gain of 63 dB and 56% conversion efficiency. In viewing of the parasitic pulse's 8.8-nm bandwidth, it has the potential to become a novel seed source for high-peak-power fiber amplifiers.]]></abstract>
##     <issn><![CDATA[1943-0655]]></issn>
##     <htmlFlag><![CDATA[1]]></htmlFlag>
##     <arnumber><![CDATA[6804656]]></arnumber>
##     <doi><![CDATA[10.1109/JPHOT.2014.2319090]]></doi>
##     <publicationId><![CDATA[6804656]]></publicationId>
##     <mdurl><![CDATA[http://ieeexplore.ieee.org/xpl/articleDetails.jsp?tp=&arnumber=6804656&contentType=Journals+%26+Magazines]]></mdurl>
##     <pdf><![CDATA[http://ieeexplore.ieee.org/stamp/stamp.jsp?arnumber=6804656]]></pdf>
##   </document>
##   <document>
##     <rank>866</rank>
##     <title><![CDATA[Fault Diagnosis in Hybrid Electric Vehicle Regenerative Braking System]]></title>
##     <authors><![CDATA[Sankavaram, C.;  Pattipati, B.;  Pattipati, K.R.;  Yilu Zhang;  Howell, M.]]></authors>
##     <affiliations><![CDATA[Dept. of Electr. & Comput. Eng., Univ. of Connecticut, Storrs, CT, USA]]></affiliations>
##     <controlledterms>
##       <term><![CDATA[energy conservation]]></term>
##       <term><![CDATA[fault diagnosis]]></term>
##       <term><![CDATA[feature extraction]]></term>
##       <term><![CDATA[hybrid electric vehicles]]></term>
##       <term><![CDATA[regenerative braking]]></term>
##       <term><![CDATA[signal processing]]></term>
##       <term><![CDATA[statistical analysis]]></term>
##     </controlledterms>
##     <thesaurusterms>
##       <term><![CDATA[Batteries]]></term>
##       <term><![CDATA[Brakes]]></term>
##       <term><![CDATA[Fault diagnosis]]></term>
##       <term><![CDATA[Hidden Markov models]]></term>
##       <term><![CDATA[Mechanical power transmission]]></term>
##       <term><![CDATA[System-on-chip]]></term>
##       <term><![CDATA[Torque control]]></term>
##     </thesaurusterms>
##     <pubtitle><![CDATA[Access, IEEE]]></pubtitle>
##     <punumber><![CDATA[6287639]]></punumber>
##     <pubtype><![CDATA[Journals & Magazines]]></pubtype>
##     <publisher><![CDATA[IEEE]]></publisher>
##     <volume><![CDATA[2]]></volume>
##     <py><![CDATA[2014]]></py>
##     <spage><![CDATA[1225]]></spage>
##     <epage><![CDATA[1239]]></epage>
##     <abstract><![CDATA[Regenerative braking is one of the most promising and environmentally friendly technologies used in electric and hybrid electric vehicles to improve energy efficiency and vehicle stability. This paper presents a systematic data-driven process for detecting and diagnosing faults in the regenerative braking system of hybrid electric vehicles. The diagnostic process involves signal processing and statistical techniques for feature extraction, data reduction for implementation in memory-constrained electronic control units, and variety of fault classification methodologies to isolate faults in the regenerative braking system. The results demonstrate that highly accurate fault diagnosis is possible with the classification methodologies. The process can be employed for fault analysis in a wide variety of systems, ranging from automobiles to buildings to aerospace systems.]]></abstract>
##     <issn><![CDATA[2169-3536]]></issn>
##     <htmlFlag><![CDATA[1]]></htmlFlag>
##     <arnumber><![CDATA[6920008]]></arnumber>
##     <doi><![CDATA[10.1109/ACCESS.2014.2362756]]></doi>
##     <publicationId><![CDATA[6920008]]></publicationId>
##     <mdurl><![CDATA[http://ieeexplore.ieee.org/xpl/articleDetails.jsp?tp=&arnumber=6920008&contentType=Journals+%26+Magazines]]></mdurl>
##     <pdf><![CDATA[http://ieeexplore.ieee.org/stamp/stamp.jsp?arnumber=6920008]]></pdf>
##   </document>
##   <document>
##     <rank>867</rank>
##     <title><![CDATA[A Bregman Matrix and the Gradient of Mutual Information for Vector Poisson and Gaussian Channels]]></title>
##     <authors><![CDATA[Liming Wang;  Carlson, D.E.;  Rodrigues, M.R.D.;  Calderbank, R.;  Carin, L.]]></authors>
##     <affiliations><![CDATA[Dept. of Electr. & Comput. Eng., Duke Univ., Durham, NC, USA]]></affiliations>
##     <controlledterms>
##       <term><![CDATA[Gaussian channels]]></term>
##       <term><![CDATA[compressed sensing]]></term>
##       <term><![CDATA[entropy]]></term>
##       <term><![CDATA[least mean squares methods]]></term>
##       <term><![CDATA[matrix algebra]]></term>
##       <term><![CDATA[signal classification]]></term>
##     </controlledterms>
##     <thesaurusterms>
##       <term><![CDATA[Channel models]]></term>
##       <term><![CDATA[Covariance matrices]]></term>
##       <term><![CDATA[Dark current]]></term>
##       <term><![CDATA[Measurement]]></term>
##       <term><![CDATA[Mutual information]]></term>
##       <term><![CDATA[Optimization]]></term>
##       <term><![CDATA[Vectors]]></term>
##     </thesaurusterms>
##     <pubtitle><![CDATA[Information Theory, IEEE Transactions on]]></pubtitle>
##     <punumber><![CDATA[18]]></punumber>
##     <pubtype><![CDATA[Journals & Magazines]]></pubtype>
##     <publisher><![CDATA[IEEE]]></publisher>
##     <volume><![CDATA[60]]></volume>
##     <issue><![CDATA[5]]></issue>
##     <py><![CDATA[2014]]></py>
##     <spage><![CDATA[2611]]></spage>
##     <epage><![CDATA[2629]]></epage>
##     <abstract><![CDATA[A generalization of Bregman divergence is developed and utilized to unify vector Poisson and Gaussian channel models, from the perspective of the gradient of mutual information. The gradient is with respect to the measurement matrix in a compressive-sensing setting, and mutual information is considered for signal recovery and classification. Existing gradient-of-mutual-information results for scalar Poisson models are recovered as special cases, as are known results for the vector Gaussian model. The Bregman-divergence generalization yields a Bregman matrix, and this matrix induces numerous matrix-valued metrics. The metrics associated with the Bregman matrix are detailed, as are its other properties. The Bregman matrix is also utilized to connect the relative entropy and mismatched minimum mean squared error. Two applications are considered: 1) compressive sensing with a Poisson measurement model and 2) compressive topic modeling for analysis of a document corpora (word-count data). In both of these settings, we use the developed theory to optimize the compressive measurement matrix, for signal recovery and classification.]]></abstract>
##     <issn><![CDATA[0018-9448]]></issn>
##     <arnumber><![CDATA[6746209]]></arnumber>
##     <doi><![CDATA[10.1109/TIT.2014.2307068]]></doi>
##     <publicationId><![CDATA[6746209]]></publicationId>
##     <mdurl><![CDATA[http://ieeexplore.ieee.org/xpl/articleDetails.jsp?tp=&arnumber=6746209&contentType=Journals+%26+Magazines]]></mdurl>
##     <pdf><![CDATA[http://ieeexplore.ieee.org/stamp/stamp.jsp?arnumber=6746209]]></pdf>
##   </document>
##   <document>
##     <rank>868</rank>
##     <title><![CDATA[MIMO Characterization on System Level of 5G Microbase Stations Subject to Randomness in LOS]]></title>
##     <authors><![CDATA[Kildal, P.-S.;  Xiaoming Chen;  Gustafsson, M.;  Zhengzhao Shen]]></authors>
##     <affiliations><![CDATA[Dept. of Signals & Syst., Chalmers Univ. of Technol., Gothenburg, Sweden]]></affiliations>
##     <controlledterms>
##       <term><![CDATA[4G mobile communication]]></term>
##       <term><![CDATA[Long Term Evolution]]></term>
##       <term><![CDATA[MIMO communication]]></term>
##       <term><![CDATA[OFDM modulation]]></term>
##       <term><![CDATA[antenna arrays]]></term>
##       <term><![CDATA[antenna radiation patterns]]></term>
##       <term><![CDATA[radio receivers]]></term>
##       <term><![CDATA[reverberation chambers]]></term>
##       <term><![CDATA[smart phones]]></term>
##       <term><![CDATA[transceivers]]></term>
##     </controlledterms>
##     <thesaurusterms>
##       <term><![CDATA[Broadband antennas]]></term>
##       <term><![CDATA[MIMO]]></term>
##       <term><![CDATA[Mobile communication]]></term>
##       <term><![CDATA[Next generation networking]]></term>
##       <term><![CDATA[OFDM]]></term>
##       <term><![CDATA[Reverberation chambers]]></term>
##       <term><![CDATA[Statistical analysis]]></term>
##       <term><![CDATA[Wireless networks]]></term>
##     </thesaurusterms>
##     <pubtitle><![CDATA[Access, IEEE]]></pubtitle>
##     <punumber><![CDATA[6287639]]></punumber>
##     <pubtype><![CDATA[Journals & Magazines]]></pubtype>
##     <publisher><![CDATA[IEEE]]></publisher>
##     <volume><![CDATA[2]]></volume>
##     <py><![CDATA[2014]]></py>
##     <spage><![CDATA[1064]]></spage>
##     <epage><![CDATA[1077]]></epage>
##     <abstract><![CDATA[Wireless systems have become more and more advanced in terms of handling the statistical properties of wireless channels. For example, the 4G long term evolution (LTE) system takes advantage of multiport antennas [multiple-input multiple-output (MIMO) technology] and orthogonal frequency division multiplexing (OFDM) to improve the detection probability of single bitstream by diversity in the spatial and frequency domains, respectively. The 4G system also supports transmission of two bitstreams by appropriate signal processing of the MIMO subchannels. The reverberation chamber emulates according to previous works rich isotropic multipath (RIMP) and has proven to be very useful for characterizing smart phones for LTE systems. The measured throughput can be accurately modeled by the simple digital threshold receiver, accounting accurately for both the MIMO and OFDM functions. The throughput is equivalent to the probability of detection (PoD) of the transmitted bitstream. The purpose of this paper is to introduce a systematic approach to include the statistical properties of the user and his or her terminal, when characterizing the performance. The user statistics will have a larger effect in environments with stronger line-of-sight (LOS), because the angle of arrival and the polarization of the LOS contribution vary due to the user's orientation and practices. These variations are stochastic, and therefore, we introduce the term random-LOS to describe this. This paper elaborates on the characterization of an example antenna in both RIMP and random-LOS. The chosen antenna is a wideband microbase transceiver station (BTS) antenna. We show how to characterize the micro-BTS by the PoD of one and two bitstreams in both RIMP and random-LOS, by considering the user randomly located and oriented within the angular coverage sector. We limit the treatment to a wall-mounted BTS antenna, and assume a desired hemispherical coverage. The angular coverages of both one and two bitst- eams for the random-LOS case are plotted as MIMO-coverage radiation patterns of the whole four-port digital antenna system. Such characterizations in terms of PoD have never been done before on any practical antenna system. The final results are easy to interpret, and they open up a new world of opportunities for designing and optimizing 5G antennas on system level.]]></abstract>
##     <issn><![CDATA[2169-3536]]></issn>
##     <htmlFlag><![CDATA[1]]></htmlFlag>
##     <arnumber><![CDATA[6902753]]></arnumber>
##     <doi><![CDATA[10.1109/ACCESS.2014.2358937]]></doi>
##     <publicationId><![CDATA[6902753]]></publicationId>
##     <mdurl><![CDATA[http://ieeexplore.ieee.org/xpl/articleDetails.jsp?tp=&arnumber=6902753&contentType=Journals+%26+Magazines]]></mdurl>
##     <pdf><![CDATA[http://ieeexplore.ieee.org/stamp/stamp.jsp?arnumber=6902753]]></pdf>
##   </document>
##   <document>
##     <rank>869</rank>
##     <title><![CDATA[Spiral Propagation of Polymer Optical Fiber Fuse Accompanied by Spontaneous Burst and Its Real-Time Monitoring Using Brillouin Scattering]]></title>
##     <authors><![CDATA[Mizuno, Y.;  Hayashi, N.;  Tanaka, H.;  Nakamura, K.]]></authors>
##     <affiliations><![CDATA[Precision & Intell. Lab., Tokyo Inst. of Technol., Yokohama, Japan]]></affiliations>
##     <controlledterms>
##       <term><![CDATA[optical fibre testing]]></term>
##       <term><![CDATA[optical polymers]]></term>
##       <term><![CDATA[polymer fibres]]></term>
##       <term><![CDATA[stimulated Brillouin scattering]]></term>
##     </controlledterms>
##     <thesaurusterms>
##       <term><![CDATA[Density measurement]]></term>
##       <term><![CDATA[Fuses]]></term>
##       <term><![CDATA[Optical fibers]]></term>
##       <term><![CDATA[Power system measurements]]></term>
##       <term><![CDATA[Scattering]]></term>
##       <term><![CDATA[Temperature measurement]]></term>
##     </thesaurusterms>
##     <pubtitle><![CDATA[Photonics Journal, IEEE]]></pubtitle>
##     <punumber><![CDATA[4563994]]></punumber>
##     <pubtype><![CDATA[Journals & Magazines]]></pubtype>
##     <publisher><![CDATA[IEEE]]></publisher>
##     <volume><![CDATA[6]]></volume>
##     <issue><![CDATA[3]]></issue>
##     <py><![CDATA[2014]]></py>
##     <spage><![CDATA[1]]></spage>
##     <epage><![CDATA[7]]></epage>
##     <abstract><![CDATA[We study the propagation behavior of the polymer optical fiber (POF) fuse at a power density up to several tens of kW/cm<sup>2</sup> (corresponding to subwatt power). The propagation velocity is raised in proportion to the power density, reaching 41 mm/s at 67 kW/cm<sup>2</sup>. We also observe spiral oscillation and spontaneous termination of the fuse propagation, with the latter accompanied by a burst. We then develop a new method of detecting the location of the propagating POF fuse remotely and non visually in real time using Brillouin scattering, which can be clearly observed at such a high power density. This method requires neither additional light injection nor signal integration, and it could be used to monitor the propagating fuse in glass fibers.]]></abstract>
##     <issn><![CDATA[1943-0655]]></issn>
##     <htmlFlag><![CDATA[1]]></htmlFlag>
##     <arnumber><![CDATA[6814840]]></arnumber>
##     <doi><![CDATA[10.1109/JPHOT.2014.2323301]]></doi>
##     <publicationId><![CDATA[6814840]]></publicationId>
##     <mdurl><![CDATA[http://ieeexplore.ieee.org/xpl/articleDetails.jsp?tp=&arnumber=6814840&contentType=Journals+%26+Magazines]]></mdurl>
##     <pdf><![CDATA[http://ieeexplore.ieee.org/stamp/stamp.jsp?arnumber=6814840]]></pdf>
##   </document>
##   <document>
##     <rank>870</rank>
##     <title><![CDATA[Impedance-based fault location in transmission networks: theory and application]]></title>
##     <authors><![CDATA[Das, S.;  Santoso, S.;  Gaikwad, A.;  Patel, M.]]></authors>
##     <affiliations><![CDATA[Dept. of Electr. & Comput. Eng., Univ. of Texas at Austin, Austin, TX, USA]]></affiliations>
##     <controlledterms>
##       <term><![CDATA[electric impedance]]></term>
##       <term><![CDATA[fault location]]></term>
##       <term><![CDATA[power transmission faults]]></term>
##     </controlledterms>
##     <thesaurusterms>
##       <term><![CDATA[Algorithm design and analysis]]></term>
##       <term><![CDATA[Estimation]]></term>
##       <term><![CDATA[Fault currents]]></term>
##       <term><![CDATA[Fault location]]></term>
##       <term><![CDATA[Impedance]]></term>
##       <term><![CDATA[Mutual coupling]]></term>
##       <term><![CDATA[Resistance]]></term>
##       <term><![CDATA[Synchronization]]></term>
##       <term><![CDATA[Transmission lines]]></term>
##     </thesaurusterms>
##     <pubtitle><![CDATA[Access, IEEE]]></pubtitle>
##     <punumber><![CDATA[6287639]]></punumber>
##     <pubtype><![CDATA[Journals & Magazines]]></pubtype>
##     <publisher><![CDATA[IEEE]]></publisher>
##     <volume><![CDATA[2]]></volume>
##     <py><![CDATA[2014]]></py>
##     <spage><![CDATA[537]]></spage>
##     <epage><![CDATA[557]]></epage>
##     <abstract><![CDATA[A number of impedance-based fault location algorithms have been developed for estimating the distance to faults in a transmission network. Each algorithm has specific input data requirements and makes certain assumptions that may or may not hold true in a particular fault location scenario. Without a detailed understanding of the principle of each fault-locating method, choosing the most suitable fault location algorithm can be a challenging task. This paper, therefore, presents the theory of one-ended (simple reactance, Takagi, modified Takagi, Eriksson, and Novosel et al.) and two-ended (synchronized, unsynchronized, and current-only) impedance-based fault location algorithms and demonstrates their application in locating real-world faults. The theory details the formulation and input data requirement of each fault-locating algorithm and evaluates the sensitivity of each to the following error sources: 1) load; 2) remote infeed; 3) fault resistance; 4) mutual coupling; 5) inaccurate line impedances; 6) DC offset and CT saturation; 7) three-terminal lines; and 8) tapped radial lines. From the theoretical analysis and field data testing, the following criteria are recommended for choosing the most suitable fault-locating algorithm: 1) data availability and 2) fault location application scenario. Another objective of this paper is to assess what additional information can be gleaned from waveforms recorded by intelligent electronic devices (IEDs) during a fault. Actual fault event data captured in utility networks is exploited to gain valuable feedback about the transmission network upstream from the IED device, and estimate the value of fault resistance.]]></abstract>
##     <issn><![CDATA[2169-3536]]></issn>
##     <htmlFlag><![CDATA[1]]></htmlFlag>
##     <arnumber><![CDATA[6814841]]></arnumber>
##     <doi><![CDATA[10.1109/ACCESS.2014.2323353]]></doi>
##     <publicationId><![CDATA[6814841]]></publicationId>
##     <mdurl><![CDATA[http://ieeexplore.ieee.org/xpl/articleDetails.jsp?tp=&arnumber=6814841&contentType=Journals+%26+Magazines]]></mdurl>
##     <pdf><![CDATA[http://ieeexplore.ieee.org/stamp/stamp.jsp?arnumber=6814841]]></pdf>
##   </document>
##   <document>
##     <rank>871</rank>
##     <title><![CDATA[An Electrically Tunable Polarizer for a Fiber System Based on a Polarization-Dependent Beam Size Derived From a Liquid Crystal Lens]]></title>
##     <authors><![CDATA[Chen, M.;  Chyong-Hua Chen;  Yinchieh Lai;  Yan-qing Lu;  Yi-Hsin Lin]]></authors>
##     <affiliations><![CDATA[Dept. of Photonics, Nat. Chiao Tung Univ., Hsinchu, Taiwan]]></affiliations>
##     <controlledterms>
##       <term><![CDATA[laser beams]]></term>
##       <term><![CDATA[lenses]]></term>
##       <term><![CDATA[liquid crystals]]></term>
##       <term><![CDATA[optical fibres]]></term>
##       <term><![CDATA[optical polarisers]]></term>
##       <term><![CDATA[optical tuning]]></term>
##     </controlledterms>
##     <thesaurusterms>
##       <term><![CDATA[Couplings]]></term>
##       <term><![CDATA[Lenses]]></term>
##       <term><![CDATA[Optical fiber polarization]]></term>
##       <term><![CDATA[Optical fiber sensors]]></term>
##       <term><![CDATA[Optical polarization]]></term>
##     </thesaurusterms>
##     <pubtitle><![CDATA[Photonics Journal, IEEE]]></pubtitle>
##     <punumber><![CDATA[4563994]]></punumber>
##     <pubtype><![CDATA[Journals & Magazines]]></pubtype>
##     <publisher><![CDATA[IEEE]]></publisher>
##     <volume><![CDATA[6]]></volume>
##     <issue><![CDATA[3]]></issue>
##     <py><![CDATA[2014]]></py>
##     <spage><![CDATA[1]]></spage>
##     <epage><![CDATA[8]]></epage>
##     <abstract><![CDATA[A broadband electrically tunable variable polarizer for fiber systems based on a liquid crystal lens is proposed and demonstrated. The polarization selectivity is based on a polarization-sensitive coupling efficiency to the fiber in the fiber system. For an incident ordinary ray, the output beam size remains the same as that of the incident beam, which results in low coupling efficiency. For an incident extraordinary ray (e-ray), the output beam size is close to the size of the fiber core, giving rise to high coupling efficiency because of the lens effect. Moreover, the output beam size of the e-ray can be electrically controlled, thus allowing the optical attenuation to be manipulated. In our experiments, the polarization-dependent loss from the visible to the near-infrared spectral region was approximately 12 dB. Such a broadband electrically variable polarizer may be applied to various designs for fiber-optic sensing devices and polarization-sensitive optical instruments.]]></abstract>
##     <issn><![CDATA[1943-0655]]></issn>
##     <htmlFlag><![CDATA[1]]></htmlFlag>
##     <arnumber><![CDATA[6805132]]></arnumber>
##     <doi><![CDATA[10.1109/JPHOT.2014.2319103]]></doi>
##     <publicationId><![CDATA[6805132]]></publicationId>
##     <mdurl><![CDATA[http://ieeexplore.ieee.org/xpl/articleDetails.jsp?tp=&arnumber=6805132&contentType=Journals+%26+Magazines]]></mdurl>
##     <pdf><![CDATA[http://ieeexplore.ieee.org/stamp/stamp.jsp?arnumber=6805132]]></pdf>
##   </document>
##   <document>
##     <rank>872</rank>
##     <title><![CDATA[Temporal Video Quality Model Accounting for Variable Frame Delay Distortions]]></title>
##     <authors><![CDATA[Pinson, M.H.;  Lark Kwon Choi;  Bovik, A.C.]]></authors>
##     <affiliations><![CDATA[Inst. for Telecommun. Sci., Boulder, CO, USA]]></affiliations>
##     <controlledterms>
##       <term><![CDATA[delays]]></term>
##       <term><![CDATA[distortion]]></term>
##       <term><![CDATA[video signal processing]]></term>
##     </controlledterms>
##     <thesaurusterms>
##       <term><![CDATA[Databases]]></term>
##       <term><![CDATA[Delays]]></term>
##       <term><![CDATA[Image edge detection]]></term>
##       <term><![CDATA[PSNR]]></term>
##       <term><![CDATA[Quality assessment]]></term>
##       <term><![CDATA[Video recording]]></term>
##       <term><![CDATA[Video sequences]]></term>
##     </thesaurusterms>
##     <pubtitle><![CDATA[Broadcasting, IEEE Transactions on]]></pubtitle>
##     <punumber><![CDATA[11]]></punumber>
##     <pubtype><![CDATA[Journals & Magazines]]></pubtype>
##     <publisher><![CDATA[IEEE]]></publisher>
##     <volume><![CDATA[60]]></volume>
##     <issue><![CDATA[4]]></issue>
##     <py><![CDATA[2014]]></py>
##     <spage><![CDATA[637]]></spage>
##     <epage><![CDATA[649]]></epage>
##     <abstract><![CDATA[We announce a new video quality model (VQM) that accounts for the perceptual impact of variable frame delays (VFD) in videos with demonstrated top performance on the laboratory for image and video engineering (LIVE) mobile video quality assessment (VQA) database. This model, called VQM_VFD, uses perceptual features extracted from spatialtemporal blocks spanning fixed angular extents and a long edge detection filter. VQM_VFD predicts video quality by measuring multiple frame delays using perception based parameters to track subjective quality over time. In the performance analysis of VQM_VFD, we evaluated its efficacy at predicting human opinions of visual quality. A detailed correlation analysis and statistical hypothesis testing show that VQM_VFD accurately predicts human subjective judgments and substantially outperforms top-performing image quality assessment and VQA models previously tested on the LIVE mobile VQA database. VQM_VFD achieved the best performance on the mobile and tablet studies of the LIVE mobile VQA database for simulated compression, wireless packet-loss, and rate adaptation, but not for temporal dynamics. These results validate the new model and warrant a hard release of the VQM_VFD algorithm.]]></abstract>
##     <issn><![CDATA[0018-9316]]></issn>
##     <htmlFlag><![CDATA[1]]></htmlFlag>
##     <arnumber><![CDATA[6954510]]></arnumber>
##     <doi><![CDATA[10.1109/TBC.2014.2365260]]></doi>
##     <publicationId><![CDATA[6954510]]></publicationId>
##     <mdurl><![CDATA[http://ieeexplore.ieee.org/xpl/articleDetails.jsp?tp=&arnumber=6954510&contentType=Journals+%26+Magazines]]></mdurl>
##     <pdf><![CDATA[http://ieeexplore.ieee.org/stamp/stamp.jsp?arnumber=6954510]]></pdf>
##   </document>
##   <document>
##     <rank>873</rank>
##     <title><![CDATA[Normalized Correlation-Based Quantization Modulation for Robust Watermarking]]></title>
##     <authors><![CDATA[Xinshan Zhu;  Jie Ding;  Honghui Dong;  Kongfa Hu;  Xiaobin Zhang]]></authors>
##     <affiliations><![CDATA[Sch. of Electr. Eng. & Autom., Tianjin Univ., Tianjin, China]]></affiliations>
##     <controlledterms>
##       <term><![CDATA[AWGN]]></term>
##       <term><![CDATA[correlation methods]]></term>
##       <term><![CDATA[error statistics]]></term>
##       <term><![CDATA[filtering theory]]></term>
##       <term><![CDATA[image watermarking]]></term>
##       <term><![CDATA[modulation]]></term>
##       <term><![CDATA[quantisation (signal)]]></term>
##       <term><![CDATA[wavelet transforms]]></term>
##     </controlledterms>
##     <thesaurusterms>
##       <term><![CDATA[Decoding]]></term>
##       <term><![CDATA[Feature extraction]]></term>
##       <term><![CDATA[Modulation]]></term>
##       <term><![CDATA[Quantization (signal)]]></term>
##       <term><![CDATA[Robustness]]></term>
##       <term><![CDATA[Vectors]]></term>
##       <term><![CDATA[Watermarking]]></term>
##     </thesaurusterms>
##     <pubtitle><![CDATA[Multimedia, IEEE Transactions on]]></pubtitle>
##     <punumber><![CDATA[6046]]></punumber>
##     <pubtype><![CDATA[Journals & Magazines]]></pubtype>
##     <publisher><![CDATA[IEEE]]></publisher>
##     <volume><![CDATA[16]]></volume>
##     <issue><![CDATA[7]]></issue>
##     <py><![CDATA[2014]]></py>
##     <spage><![CDATA[1888]]></spage>
##     <epage><![CDATA[1904]]></epage>
##     <abstract><![CDATA[A novel quantization watermarking method is presented in this paper, which is developed following the established feature modulation watermarking model. In this method, a feature signal is obtained by computing the normalized correlation (NC) between the host signal and a random signal. Information modulation is carried out on the generated NC by selecting a codeword from the codebook associated with the embedded information. In a simple case, the structured codebooks are designed using uniform quantizers for modulation. The watermarked signal is produced to provide the modulated NC in the sense of minimizing the embedding distortion. The performance of the NC-based quantization modulation (NCQM) is analytically investigated, in terms of the embedding distortion and the decoding error probability in the presence of valumetric scaling and additive noise attacks. Numerical simulations on artificial signals confirm the validity of our analyses and exhibit the performance advantage of NCQM over other modulation techniques. The proposed method is also simulated on real images by using the wavelet-based implementations, where the host signal is constructed by the detail coefficients of wavelet decomposition at the third level and transformed into the NC feature signal for the information modulation. Experimental results show that the proposed NCQM not only achieves the improved watermark imperceptibility and a higher embedding capacity in high-noise regimes, but also is more robust to a wide range of attacks, e.g., valumetric scaling, Gaussian filtering, additive noise, Gamma correction, and Gray-level transformations, as compared with the state-of-the-art watermarking methods.]]></abstract>
##     <issn><![CDATA[1520-9210]]></issn>
##     <htmlFlag><![CDATA[1]]></htmlFlag>
##     <arnumber><![CDATA[6860239]]></arnumber>
##     <doi><![CDATA[10.1109/TMM.2014.2340695]]></doi>
##     <publicationId><![CDATA[6860239]]></publicationId>
##     <mdurl><![CDATA[http://ieeexplore.ieee.org/xpl/articleDetails.jsp?tp=&arnumber=6860239&contentType=Journals+%26+Magazines]]></mdurl>
##     <pdf><![CDATA[http://ieeexplore.ieee.org/stamp/stamp.jsp?arnumber=6860239]]></pdf>
##   </document>
##   <document>
##     <rank>874</rank>
##     <title><![CDATA[Application of Crop Model Data Assimilation With a Particle Filter for Estimating Regional Winter Wheat Yields]]></title>
##     <authors><![CDATA[Zhiwei Jiang;  Zhongxin Chen;  Jin Chen;  Jia Liu;  Jianqiang Ren;  Zongnan Li;  Liang Sun;  He Li]]></authors>
##     <affiliations><![CDATA[State Key Lab. of Earth Surface Processes & Resource Ecology, Beijing Normal Univ., Beijing, China]]></affiliations>
##     <controlledterms>
##       <term><![CDATA[crops]]></term>
##       <term><![CDATA[data assimilation]]></term>
##     </controlledterms>
##     <thesaurusterms>
##       <term><![CDATA[Agriculture]]></term>
##       <term><![CDATA[Biological system modeling]]></term>
##       <term><![CDATA[Data assimilation]]></term>
##       <term><![CDATA[Data models]]></term>
##       <term><![CDATA[Remote sensing]]></term>
##       <term><![CDATA[Soil]]></term>
##       <term><![CDATA[Yield estimation]]></term>
##     </thesaurusterms>
##     <pubtitle><![CDATA[Selected Topics in Applied Earth Observations and Remote Sensing, IEEE Journal of]]></pubtitle>
##     <punumber><![CDATA[4609443]]></punumber>
##     <pubtype><![CDATA[Journals & Magazines]]></pubtype>
##     <publisher><![CDATA[IEEE]]></publisher>
##     <volume><![CDATA[7]]></volume>
##     <issue><![CDATA[11]]></issue>
##     <py><![CDATA[2014]]></py>
##     <spage><![CDATA[4422]]></spage>
##     <epage><![CDATA[4431]]></epage>
##     <abstract><![CDATA[To improve the performance of crop models for regional crop yield estimates, a particle filter (PF) was introduced to develop a data assimilation strategy using the Crop Environment Resource Synthesis (CERES)-Wheat model. Two experiments involving winter wheat yield estimations were conducted at a field plot and on a regional scale to test the feasibility of the PF-based data assimilation strategy and to analyze the effects of the PF parameters and spatiotemporal scales of assimilating observations on the performance of the crop model data assimilation. The significant improvements in the yield estimation suggest that PF-based crop model data assimilation is feasible. Winter wheat yields from the field plots were forecasted with a determination coefficient (R<sup>2</sup>) of 0.87, a root-mean-square error (RMSE) of 251 kg/ha, and a relative error (RE) of 2.95%. An acceptable yield at the county scale was estimated with a R<sup>2</sup> of 0.998, a RMSE of 9734 t, and a RE of 4.29%. The optimal yield estimates may be highly dependent on the reasonable spatiotemporal resolution of assimilating observations. A configuration using a particle size of 50, LAI maps with a moderate spatial resolution (e.g., 1 km), and an assimilation interval of 20 d results in a reasonable tradeoff between accuracy and effectiveness in regional applications.]]></abstract>
##     <issn><![CDATA[1939-1404]]></issn>
##     <htmlFlag><![CDATA[1]]></htmlFlag>
##     <arnumber><![CDATA[6805188]]></arnumber>
##     <doi><![CDATA[10.1109/JSTARS.2014.2316012]]></doi>
##     <publicationId><![CDATA[6805188]]></publicationId>
##     <mdurl><![CDATA[http://ieeexplore.ieee.org/xpl/articleDetails.jsp?tp=&arnumber=6805188&contentType=Journals+%26+Magazines]]></mdurl>
##     <pdf><![CDATA[http://ieeexplore.ieee.org/stamp/stamp.jsp?arnumber=6805188]]></pdf>
##   </document>
##   <document>
##     <rank>875</rank>
##     <title><![CDATA[Closed-Form Analysis of Relay-Based Cognitive Radio Networks Over Nakagami- <formula formulatype="inline"> <img src="/images/tex/254.gif" alt="m"> </formula> Fading Channels]]></title>
##     <authors><![CDATA[Hussain, S.;  Fernando, X.N.]]></authors>
##     <affiliations><![CDATA[Dept. of Electr. & Comput. Eng., Ryerson Univ., Toronto, ON, Canada]]></affiliations>
##     <controlledterms>
##       <term><![CDATA[Nakagami channels]]></term>
##       <term><![CDATA[cognitive radio]]></term>
##       <term><![CDATA[radio links]]></term>
##       <term><![CDATA[radio networks]]></term>
##       <term><![CDATA[radio spectrum management]]></term>
##       <term><![CDATA[relay networks (telecommunication)]]></term>
##       <term><![CDATA[statistical analysis]]></term>
##     </controlledterms>
##     <thesaurusterms>
##       <term><![CDATA[Cascading style sheets]]></term>
##       <term><![CDATA[Diversity reception]]></term>
##       <term><![CDATA[Fading]]></term>
##       <term><![CDATA[Performance analysis]]></term>
##       <term><![CDATA[Receivers]]></term>
##       <term><![CDATA[Relays]]></term>
##       <term><![CDATA[Sensors]]></term>
##     </thesaurusterms>
##     <pubtitle><![CDATA[Vehicular Technology, IEEE Transactions on]]></pubtitle>
##     <punumber><![CDATA[25]]></punumber>
##     <pubtype><![CDATA[Journals & Magazines]]></pubtype>
##     <publisher><![CDATA[IEEE]]></publisher>
##     <volume><![CDATA[63]]></volume>
##     <issue><![CDATA[3]]></issue>
##     <py><![CDATA[2014]]></py>
##     <spage><![CDATA[1193]]></spage>
##     <epage><![CDATA[1203]]></epage>
##     <abstract><![CDATA[We propose a general framework for a comprehensive performance analysis of cooperative spectrum sensing (CSS) in cognitive radio (CR) networks. Specifically, we investigate the detection accuracy of a relay-based CR network over independent nonidentical Nakagami- m fading channels. Based on the probability density function (pdf) approach, we derive new exact and approximated closed-form expressions for the average detection probability and the average false alarm probability employing two diversity combining techniques, namely, the maximal ratio combining (MRC) scheme and the selection combining (SC) scheme. We also investigate the convergence rate of infinite series that appears in the derived exact closed-form expressions and propose to use a powerful acceleration algorithm that allows for the series termination with a finite number of terms. The results obtained reveal the importance of including the relaying link statistics and the combination techniques in the performance analysis of CR networks. The derived closed-form expression can be used to determine the energy threshold and the relaying power constraint that meet a given detection accuracy value over nonidentically distributed Nakagami- m fading.]]></abstract>
##     <issn><![CDATA[0018-9545]]></issn>
##     <htmlFlag><![CDATA[1]]></htmlFlag>
##     <arnumber><![CDATA[6606840]]></arnumber>
##     <doi><![CDATA[10.1109/TVT.2013.2283078]]></doi>
##     <publicationId><![CDATA[6606840]]></publicationId>
##     <mdurl><![CDATA[http://ieeexplore.ieee.org/xpl/articleDetails.jsp?tp=&arnumber=6606840&contentType=Journals+%26+Magazines]]></mdurl>
##     <pdf><![CDATA[http://ieeexplore.ieee.org/stamp/stamp.jsp?arnumber=6606840]]></pdf>
##   </document>
##   <document>
##     <rank>876</rank>
##     <title><![CDATA[An Overlay-Based Data Mining Architecture Tolerant to Physical Network Disruptions]]></title>
##     <authors><![CDATA[Suto, K.;  Nishiyama, H.;  Kato, N.;  Mizutani, K.;  Akashi, O.;  Takahara, A.]]></authors>
##     <affiliations><![CDATA[Grad. Sch. of Inf. Sci., Tohoku Univ., Sendai, Japan]]></affiliations>
##     <controlledterms>
##       <term><![CDATA[Big Data]]></term>
##       <term><![CDATA[data mining]]></term>
##       <term><![CDATA[numerical analysis]]></term>
##       <term><![CDATA[parallel processing]]></term>
##       <term><![CDATA[resource allocation]]></term>
##     </controlledterms>
##     <thesaurusterms>
##       <term><![CDATA[Computer architecture]]></term>
##       <term><![CDATA[Data mining]]></term>
##       <term><![CDATA[Electric breakdown]]></term>
##       <term><![CDATA[Overlay networks]]></term>
##       <term><![CDATA[Scalability]]></term>
##       <term><![CDATA[Servers]]></term>
##     </thesaurusterms>
##     <pubtitle><![CDATA[Emerging Topics in Computing, IEEE Transactions on]]></pubtitle>
##     <punumber><![CDATA[6245516]]></punumber>
##     <pubtype><![CDATA[Journals & Magazines]]></pubtype>
##     <publisher><![CDATA[IEEE]]></publisher>
##     <volume><![CDATA[2]]></volume>
##     <issue><![CDATA[3]]></issue>
##     <py><![CDATA[2014]]></py>
##     <spage><![CDATA[292]]></spage>
##     <epage><![CDATA[301]]></epage>
##     <abstract><![CDATA[Management scheme for highly scalable big data mining has not been well studied in spite of the fact that big data mining provides many valuable and important information for us. An overlay-based parallel data mining architecture, which executes fully distributed data management and processing by employing the overlay network, can achieve high scalability. However, the overlay-based parallel mining architecture is not capable of providing data mining services in case of the physical network disruption that is caused by router/communication line breakdowns because numerous nodes are removed from the overlay network. To cope with this issue, this paper proposes an overlay network construction scheme based on node location in physical network, and a distributed task allocation scheme using overlay network technology. The numerical analysis indicates that the proposed schemes considerably outperform the conventional schemes in terms of service availability against physical network disruption.]]></abstract>
##     <issn><![CDATA[2168-6750]]></issn>
##     <htmlFlag><![CDATA[1]]></htmlFlag>
##     <arnumber><![CDATA[6832507]]></arnumber>
##     <doi><![CDATA[10.1109/TETC.2014.2330517]]></doi>
##     <publicationId><![CDATA[6832507]]></publicationId>
##     <mdurl><![CDATA[http://ieeexplore.ieee.org/xpl/articleDetails.jsp?tp=&arnumber=6832507&contentType=Journals+%26+Magazines]]></mdurl>
##     <pdf><![CDATA[http://ieeexplore.ieee.org/stamp/stamp.jsp?arnumber=6832507]]></pdf>
##   </document>
##   <document>
##     <rank>877</rank>
##     <title><![CDATA[Photonic Generation of Microwave Pulse Using a Phase-Modulator-Based Sagnac Interferometer and Wavelength-to-Time Mapping]]></title>
##     <authors><![CDATA[Wei Li;  Wei Yu Wang;  Wen Ting Wang;  Jian Guo Liu;  Ning Hua Zhu]]></authors>
##     <affiliations><![CDATA[State Key Lab. on Integrated Optoelectron., Inst. of Semicond., Beijing, China]]></affiliations>
##     <controlledterms>
##       <term><![CDATA[Sagnac interferometers]]></term>
##       <term><![CDATA[microwave photonics]]></term>
##       <term><![CDATA[optical modulation]]></term>
##       <term><![CDATA[optical pulse generation]]></term>
##       <term><![CDATA[phase modulation]]></term>
##     </controlledterms>
##     <thesaurusterms>
##       <term><![CDATA[Microwave communication]]></term>
##       <term><![CDATA[Microwave filters]]></term>
##       <term><![CDATA[Microwave measurement]]></term>
##       <term><![CDATA[Microwave photonics]]></term>
##       <term><![CDATA[Optical pulses]]></term>
##       <term><![CDATA[Sagnac interferometers]]></term>
##     </thesaurusterms>
##     <pubtitle><![CDATA[Photonics Journal, IEEE]]></pubtitle>
##     <punumber><![CDATA[4563994]]></punumber>
##     <pubtype><![CDATA[Journals & Magazines]]></pubtype>
##     <publisher><![CDATA[IEEE]]></publisher>
##     <volume><![CDATA[6]]></volume>
##     <issue><![CDATA[6]]></issue>
##     <py><![CDATA[2014]]></py>
##     <spage><![CDATA[1]]></spage>
##     <epage><![CDATA[8]]></epage>
##     <abstract><![CDATA[We report a new photonic microwave pulse generator using a phase-modulator-based Sagnac interferometer (PM-SI) and wavelength-to-time mapping in a dispersive element. The key significance of the proposed scheme is that the phase of the generated microwave pulse can be continuously tuned by adjusting the bias voltage applied to the PM. Thanks to the wide bandwidth of the PM, high-speed phase modulation of the microwave pulse can be realized. Moreover, the PM-SI is polarization independent, which means that the PM-SI is insensitive to the state of polarization of the input optical signal, ensuring a stable operation of the system. The proposed scheme is theoretically analyzed and experimentally verified.]]></abstract>
##     <issn><![CDATA[1943-0655]]></issn>
##     <htmlFlag><![CDATA[1]]></htmlFlag>
##     <arnumber><![CDATA[6967770]]></arnumber>
##     <doi><![CDATA[10.1109/JPHOT.2014.2374595]]></doi>
##     <publicationId><![CDATA[6967770]]></publicationId>
##     <mdurl><![CDATA[http://ieeexplore.ieee.org/xpl/articleDetails.jsp?tp=&arnumber=6967770&contentType=Journals+%26+Magazines]]></mdurl>
##     <pdf><![CDATA[http://ieeexplore.ieee.org/stamp/stamp.jsp?arnumber=6967770]]></pdf>
##   </document>
##   <document>
##     <rank>878</rank>
##     <title><![CDATA[Color-Shift Keying and Code-Division Multiple-Access Transmission for RGB-LED Visible Light Communications Using Mobile Phone Camera]]></title>
##     <authors><![CDATA[Shih-Hao Chen;  Chi-Wai Chow]]></authors>
##     <affiliations><![CDATA[Dept. of Photonics & the Inst. of Electro-Opt. Eng., Nat. Chiao Tung Univ., Hsinchu, Taiwan]]></affiliations>
##     <controlledterms>
##       <term><![CDATA[cameras]]></term>
##       <term><![CDATA[code division multiple access]]></term>
##       <term><![CDATA[image sensors]]></term>
##       <term><![CDATA[light emitting diodes]]></term>
##       <term><![CDATA[light interference]]></term>
##       <term><![CDATA[mobile handsets]]></term>
##       <term><![CDATA[optical communication]]></term>
##     </controlledterms>
##     <thesaurusterms>
##       <term><![CDATA[Bit error rate]]></term>
##       <term><![CDATA[CMOS integrated circuits]]></term>
##       <term><![CDATA[Cameras]]></term>
##       <term><![CDATA[Light emitting diodes]]></term>
##       <term><![CDATA[Multiaccess communication]]></term>
##       <term><![CDATA[Optical fiber communication]]></term>
##     </thesaurusterms>
##     <pubtitle><![CDATA[Photonics Journal, IEEE]]></pubtitle>
##     <punumber><![CDATA[4563994]]></punumber>
##     <pubtype><![CDATA[Journals & Magazines]]></pubtype>
##     <publisher><![CDATA[IEEE]]></publisher>
##     <volume><![CDATA[6]]></volume>
##     <issue><![CDATA[6]]></issue>
##     <py><![CDATA[2014]]></py>
##     <spage><![CDATA[1]]></spage>
##     <epage><![CDATA[6]]></epage>
##     <abstract><![CDATA[Light-emitting diodes (LEDs) have been deployed for various applications in our daily lives. Similarly, image sensors or cameras integrated into mobile phones have become common. Hence, visible light communication (VLC) using LED and mobile phone cameras is attractive and provides low-cost wireless communication. In this paper, we propose and demonstrate a VLC system using color-shift-keying (CSK) modulation and code-division multiple-access (CDMA) technology simultaneously for the first time; a mobile phone camera is used as the receiver (Rx). CSK is used to enhance the VLC system capacity and to mitigate the single color light interference, whereas CDMA allows multiple users to access the network. The system design and operation mechanism of the proposed CSK-CDMA VLC system are discussed. A proof-of-concept demonstration is performed, and error-free transmission is achieved for multiple-access users. A 3-dB transmission gain is also obtained in each user when compared with the traditional on-off keying (OOK) modulation.]]></abstract>
##     <issn><![CDATA[1943-0655]]></issn>
##     <htmlFlag><![CDATA[1]]></htmlFlag>
##     <arnumber><![CDATA[6977905]]></arnumber>
##     <doi><![CDATA[10.1109/JPHOT.2014.2374612]]></doi>
##     <publicationId><![CDATA[6977905]]></publicationId>
##     <mdurl><![CDATA[http://ieeexplore.ieee.org/xpl/articleDetails.jsp?tp=&arnumber=6977905&contentType=Journals+%26+Magazines]]></mdurl>
##     <pdf><![CDATA[http://ieeexplore.ieee.org/stamp/stamp.jsp?arnumber=6977905]]></pdf>
##   </document>
##   <document>
##     <rank>879</rank>
##     <title><![CDATA[Capacity Analysis of Interference Alignment With Bounded CSI Uncertainty]]></title>
##     <authors><![CDATA[Guiazon, R.F.;  Kai-Kit Wong;  Wisely, D.]]></authors>
##     <affiliations><![CDATA[Dept. of Electron. & Electr. Eng., Univ. Coll. London, London, UK]]></affiliations>
##     <controlledterms>
##       <term><![CDATA[channel capacity]]></term>
##       <term><![CDATA[radiofrequency interference]]></term>
##       <term><![CDATA[wireless channels]]></term>
##     </controlledterms>
##     <thesaurusterms>
##       <term><![CDATA[Channel estimation]]></term>
##       <term><![CDATA[Heating]]></term>
##       <term><![CDATA[Interference channels]]></term>
##       <term><![CDATA[MIMO]]></term>
##       <term><![CDATA[Signal to noise ratio]]></term>
##       <term><![CDATA[Vectors]]></term>
##     </thesaurusterms>
##     <pubtitle><![CDATA[Wireless Communications Letters, IEEE]]></pubtitle>
##     <punumber><![CDATA[5962382]]></punumber>
##     <pubtype><![CDATA[Journals & Magazines]]></pubtype>
##     <publisher><![CDATA[IEEE]]></publisher>
##     <volume><![CDATA[3]]></volume>
##     <issue><![CDATA[5]]></issue>
##     <py><![CDATA[2014]]></py>
##     <spage><![CDATA[505]]></spage>
##     <epage><![CDATA[508]]></epage>
##     <abstract><![CDATA[Interference alignment (IA) has been demonstrated to achieve the degree-of-freedom (DoF) of an interference channel given perfect global channel state information (CSI). In this letter, we consider the case of imperfect CSI with bounded errors and derive a capacity lower bound of the channel using IA. We show that this lower bound is within 1 bps/Hz of the capacity of the perfect CSI case up to a certain signal-to-noise ratio (SNR) which we refer to it as the saturating SNR. Further, we introduce a new metric called modified DoF (mDoF) in order to characterize the multiplexing performance of IA with imperfect CSI at finite SNR. Simulation results for the 3-user case are provided to illustrate the region within which the actual capacity of IA falls.]]></abstract>
##     <issn><![CDATA[2162-2337]]></issn>
##     <htmlFlag><![CDATA[1]]></htmlFlag>
##     <arnumber><![CDATA[6868967]]></arnumber>
##     <doi><![CDATA[10.1109/LWC.2014.2344656]]></doi>
##     <publicationId><![CDATA[6868967]]></publicationId>
##     <mdurl><![CDATA[http://ieeexplore.ieee.org/xpl/articleDetails.jsp?tp=&arnumber=6868967&contentType=Journals+%26+Magazines]]></mdurl>
##     <pdf><![CDATA[http://ieeexplore.ieee.org/stamp/stamp.jsp?arnumber=6868967]]></pdf>
##   </document>
##   <document>
##     <rank>880</rank>
##     <title><![CDATA[System Design for Geosynchronous Synthetic Aperture Radar Missions]]></title>
##     <authors><![CDATA[Hobbs, S.;  Mitchell, C.;  Forte, B.;  Holley, R.;  Snapir, B.;  Whittaker, P.]]></authors>
##     <affiliations><![CDATA[Space Res. Centre, Cranfield Univ., Cranfield, UK]]></affiliations>
##     <controlledterms>
##       <term><![CDATA[geophysical image processing]]></term>
##       <term><![CDATA[radar imaging]]></term>
##       <term><![CDATA[synthetic aperture radar]]></term>
##     </controlledterms>
##     <thesaurusterms>
##       <term><![CDATA[Atmospheric measurements]]></term>
##       <term><![CDATA[Azimuth]]></term>
##       <term><![CDATA[Low earth orbit satellites]]></term>
##       <term><![CDATA[Orbits]]></term>
##       <term><![CDATA[Synthetic aperture radar]]></term>
##       <term><![CDATA[System analysis and design]]></term>
##     </thesaurusterms>
##     <pubtitle><![CDATA[Geoscience and Remote Sensing, IEEE Transactions on]]></pubtitle>
##     <punumber><![CDATA[36]]></punumber>
##     <pubtype><![CDATA[Journals & Magazines]]></pubtype>
##     <publisher><![CDATA[IEEE]]></publisher>
##     <volume><![CDATA[52]]></volume>
##     <issue><![CDATA[12]]></issue>
##     <py><![CDATA[2014]]></py>
##     <spage><![CDATA[7750]]></spage>
##     <epage><![CDATA[7763]]></epage>
##     <abstract><![CDATA[Geosynchronous synthetic aperture radar (GEO SAR) has been studied for several decades but has not yet been implemented. This paper provides an overview of mission design, describing significant constraints (atmosphere, orbit, temporal stability of the surface and atmosphere, measurement physics, and radar performance) and then uses these to propose an approach to initial system design. The methodology encompasses all GEO SAR mission concepts proposed to date. Important classifications of missions are: 1) those that require atmospheric phase compensation to achieve their design spatial resolution; and 2) those that achieve full spatial resolution without phase compensation. Means of estimating the atmospheric phase screen are noted, including a novel measurement of the mean rate of change of the atmospheric phase delay, which GEO SAR enables. Candidate mission concepts are described. It seems likely that GEO SAR will be feasible in a wide range of situations, although extreme weather and unstable surfaces (e.g., water, tall vegetation) prevent 100% coverage. GEO SAR offers an exciting imaging capability that powerfully complements existing systems.]]></abstract>
##     <issn><![CDATA[0196-2892]]></issn>
##     <htmlFlag><![CDATA[1]]></htmlFlag>
##     <arnumber><![CDATA[6812208]]></arnumber>
##     <doi><![CDATA[10.1109/TGRS.2014.2318171]]></doi>
##     <publicationId><![CDATA[6812208]]></publicationId>
##     <mdurl><![CDATA[http://ieeexplore.ieee.org/xpl/articleDetails.jsp?tp=&arnumber=6812208&contentType=Journals+%26+Magazines]]></mdurl>
##     <pdf><![CDATA[http://ieeexplore.ieee.org/stamp/stamp.jsp?arnumber=6812208]]></pdf>
##   </document>
##   <document>
##     <rank>881</rank>
##     <title><![CDATA[Sub-Wavelength Grating Lenses With a Twist]]></title>
##     <authors><![CDATA[Vo, S.;  Fattal, D.;  Sorin, W.V.;  Zhen Peng;  Tho Tran;  Fiorentino, M.;  Beausoleil, R.G.]]></authors>
##     <affiliations><![CDATA[Hewlett-Packard Labs., Palo Alto, CA, USA]]></affiliations>
##     <controlledterms>
##       <term><![CDATA[diffraction gratings]]></term>
##       <term><![CDATA[lenses]]></term>
##       <term><![CDATA[light polarisation]]></term>
##       <term><![CDATA[light transmission]]></term>
##       <term><![CDATA[optical fabrication]]></term>
##       <term><![CDATA[optical isolators]]></term>
##     </controlledterms>
##     <thesaurusterms>
##       <term><![CDATA[Diffraction gratings]]></term>
##       <term><![CDATA[Gratings]]></term>
##       <term><![CDATA[Indexes]]></term>
##       <term><![CDATA[Lenses]]></term>
##       <term><![CDATA[Optical reflection]]></term>
##       <term><![CDATA[Silicon]]></term>
##     </thesaurusterms>
##     <pubtitle><![CDATA[Photonics Technology Letters, IEEE]]></pubtitle>
##     <punumber><![CDATA[68]]></punumber>
##     <pubtype><![CDATA[Journals & Magazines]]></pubtype>
##     <publisher><![CDATA[IEEE]]></publisher>
##     <volume><![CDATA[26]]></volume>
##     <issue><![CDATA[13]]></issue>
##     <py><![CDATA[2014]]></py>
##     <spage><![CDATA[1375]]></spage>
##     <epage><![CDATA[1378]]></epage>
##     <abstract><![CDATA[Dielectric high-contrast sub-wavelength grating (SWG) structures have received much attention in recent years, offering a new paradigm for the integration of optical systems. Their nanoscale resonant properties can result in a complex and unintuitive far-field behavior that, if carefully crafted, allows the full control of the optical phase front from a thin sub-wavelength planar layer. To date, experimental demonstrations of these new devices have only been realized with polarized light in a reflective mode, greatly limiting their use for practical systems. In this letter, we demonstrate a highly efficient, sub-wavelength thick, transmissive grating lens configuration using symmetrical resonant posts to achieve polarization-independent operation. Our transmissive SWG lenses are easily fabricated using low-cost scalable semiconductor process technology. To illustrate their performance, we demonstrate the generation of high-order orbital angular momentum beams and their use in an optical mode-isolator application that achieves a suppression ratio of over 25 dB.]]></abstract>
##     <issn><![CDATA[1041-1135]]></issn>
##     <htmlFlag><![CDATA[1]]></htmlFlag>
##     <arnumber><![CDATA[6819845]]></arnumber>
##     <doi><![CDATA[10.1109/LPT.2014.2325947]]></doi>
##     <publicationId><![CDATA[6819845]]></publicationId>
##     <mdurl><![CDATA[http://ieeexplore.ieee.org/xpl/articleDetails.jsp?tp=&arnumber=6819845&contentType=Journals+%26+Magazines]]></mdurl>
##     <pdf><![CDATA[http://ieeexplore.ieee.org/stamp/stamp.jsp?arnumber=6819845]]></pdf>
##   </document>
##   <document>
##     <rank>882</rank>
##     <title><![CDATA[A Micromachined Dual-Band Orthomode Transducer]]></title>
##     <authors><![CDATA[Leal-Sevillano, C.A.;  Yingtao Tian;  Lancaster, M.J.;  Ruiz-Cruz, J.A.;  Montejo-Garai, J.R.;  Rebollar, J.M.]]></authors>
##     <affiliations><![CDATA[Dept. de Electromagnetismo y Teor. de Circuitos, Univ. Politec. de Madrid, Madrid, Spain]]></affiliations>
##     <controlledterms>
##       <term><![CDATA[micromachining]]></term>
##       <term><![CDATA[microsensors]]></term>
##       <term><![CDATA[photoresists]]></term>
##       <term><![CDATA[transducers]]></term>
##     </controlledterms>
##     <thesaurusterms>
##       <term><![CDATA[Dual band]]></term>
##       <term><![CDATA[Fabrication]]></term>
##       <term><![CDATA[Micromachining]]></term>
##       <term><![CDATA[Ports (Computers)]]></term>
##       <term><![CDATA[Rectangular waveguides]]></term>
##       <term><![CDATA[Standards]]></term>
##       <term><![CDATA[Transducers]]></term>
##     </thesaurusterms>
##     <pubtitle><![CDATA[Microwave Theory and Techniques, IEEE Transactions on]]></pubtitle>
##     <punumber><![CDATA[22]]></punumber>
##     <pubtype><![CDATA[Journals & Magazines]]></pubtype>
##     <publisher><![CDATA[IEEE]]></publisher>
##     <volume><![CDATA[62]]></volume>
##     <issue><![CDATA[1]]></issue>
##     <py><![CDATA[2014]]></py>
##     <spage><![CDATA[55]]></spage>
##     <epage><![CDATA[63]]></epage>
##     <abstract><![CDATA[In this paper, an orthomode transducer (OMT) for dual-band operation and optimized for stacked micromachined layers implementation is presented. The proposed design avoids the use of septums, irises, pins, or small features and minimizes the number of equal-thickness micromachined layers required. In this way, the micromachining fabrication is simplified, making the proposed design a very attractive candidate for high frequency applications and for low-cost batch production. A W-band dual-band design (one different polarization in each frequency band) with more than 10% fractional bandwidth for each band and 30% separation between bands is presented. In addition, proper routing and layered bends are designed for an optimum standard interfacing with the same orientation of the input/output ports. Two OMTs in a back-to-back configuration are fabricated using a thick SU-8 photo-resist micromachining process. A total of six stacked SU-8 layers, all of them with the same thickness of 635 &#x03BC;m, are used. The experimental results are coherent with the tolerance and misalignment of the process, validating the proposed novel OMT design.]]></abstract>
##     <issn><![CDATA[0018-9480]]></issn>
##     <htmlFlag><![CDATA[1]]></htmlFlag>
##     <arnumber><![CDATA[6678791]]></arnumber>
##     <doi><![CDATA[10.1109/TMTT.2013.2292611]]></doi>
##     <publicationId><![CDATA[6678791]]></publicationId>
##     <mdurl><![CDATA[http://ieeexplore.ieee.org/xpl/articleDetails.jsp?tp=&arnumber=6678791&contentType=Journals+%26+Magazines]]></mdurl>
##     <pdf><![CDATA[http://ieeexplore.ieee.org/stamp/stamp.jsp?arnumber=6678791]]></pdf>
##   </document>
##   <document>
##     <rank>883</rank>
##     <title><![CDATA[Structural Generative Descriptions for Time Series Classification]]></title>
##     <authors><![CDATA[Garcia-Trevino, E.S.;  Barria, J.A.]]></authors>
##     <affiliations><![CDATA[Dept. of Electr. & Electron. Eng., Imperial Coll. London, London, UK]]></affiliations>
##     <controlledterms>
##       <term><![CDATA[data mining]]></term>
##       <term><![CDATA[pattern classification]]></term>
##       <term><![CDATA[probability]]></term>
##       <term><![CDATA[statistical analysis]]></term>
##       <term><![CDATA[time series]]></term>
##     </controlledterms>
##     <thesaurusterms>
##       <term><![CDATA[Equations]]></term>
##       <term><![CDATA[Estimation]]></term>
##       <term><![CDATA[Pattern recognition]]></term>
##       <term><![CDATA[Time series analysis]]></term>
##       <term><![CDATA[Time-domain analysis]]></term>
##       <term><![CDATA[Vectors]]></term>
##       <term><![CDATA[Wavelet transforms]]></term>
##     </thesaurusterms>
##     <pubtitle><![CDATA[Cybernetics, IEEE Transactions on]]></pubtitle>
##     <punumber><![CDATA[6221036]]></punumber>
##     <pubtype><![CDATA[Journals & Magazines]]></pubtype>
##     <publisher><![CDATA[IEEE]]></publisher>
##     <volume><![CDATA[44]]></volume>
##     <issue><![CDATA[10]]></issue>
##     <py><![CDATA[2014]]></py>
##     <spage><![CDATA[1978]]></spage>
##     <epage><![CDATA[1991]]></epage>
##     <abstract><![CDATA[In this paper, we formulate a novel time series representation framework that captures the inherent data dependency of time series and that can be easily incorporated into existing statistical classification algorithms. The impact of the proposed data representation stage in the solution to the generic underlying problem of time series classification is investigated. The proposed framework, which we call structural generative descriptions moves the structural time series representation to the probability domain, and hence is able to combine statistical and structural pattern recognition paradigms in a novel fashion. Two algorithm instantiations based on the proposed framework are developed. The algorithms are tested and compared using different publicly available real-world benchmark data. Results reported in this paper show the potential of the proposed representation framework, which in the experiments investigated, performs better or comparable to state-of-the-art time series description techniques.]]></abstract>
##     <issn><![CDATA[2168-2267]]></issn>
##     <htmlFlag><![CDATA[1]]></htmlFlag>
##     <arnumber><![CDATA[6819447]]></arnumber>
##     <doi><![CDATA[10.1109/TCYB.2014.2322310]]></doi>
##     <publicationId><![CDATA[6819447]]></publicationId>
##     <mdurl><![CDATA[http://ieeexplore.ieee.org/xpl/articleDetails.jsp?tp=&arnumber=6819447&contentType=Journals+%26+Magazines]]></mdurl>
##     <pdf><![CDATA[http://ieeexplore.ieee.org/stamp/stamp.jsp?arnumber=6819447]]></pdf>
##   </document>
##   <document>
##     <rank>884</rank>
##     <title><![CDATA[Breakthroughs in Photonics 2013: Fourier Ptychographic Imaging]]></title>
##     <authors><![CDATA[Guoan Zheng]]></authors>
##     <affiliations><![CDATA[Biomed. Eng., Univ. of Connecticut, Storrs, CT, USA]]></affiliations>
##     <controlledterms>
##       <term><![CDATA[Fourier transform spectra]]></term>
##       <term><![CDATA[X-ray optics]]></term>
##       <term><![CDATA[image processing]]></term>
##       <term><![CDATA[transmission electron microscopy]]></term>
##     </controlledterms>
##     <thesaurusterms>
##       <term><![CDATA[Image reconstruction]]></term>
##       <term><![CDATA[Image resolution]]></term>
##       <term><![CDATA[Lenses]]></term>
##       <term><![CDATA[Light emitting diodes]]></term>
##       <term><![CDATA[Microscopy]]></term>
##       <term><![CDATA[Optical microscopy]]></term>
##     </thesaurusterms>
##     <pubtitle><![CDATA[Photonics Journal, IEEE]]></pubtitle>
##     <punumber><![CDATA[4563994]]></punumber>
##     <pubtype><![CDATA[Journals & Magazines]]></pubtype>
##     <publisher><![CDATA[IEEE]]></publisher>
##     <volume><![CDATA[6]]></volume>
##     <issue><![CDATA[2]]></issue>
##     <py><![CDATA[2014]]></py>
##     <spage><![CDATA[1]]></spage>
##     <epage><![CDATA[7]]></epage>
##     <abstract><![CDATA[Fourier ptychography (FP) is a recently developed computational framework for high-resolution high-throughput imaging. In this paper, we will review the latest development of the Fourier ptychographic imaging scheme. We will demonstrate its applications in wide-field imaging, quantitative phase imaging, and adaptive imaging. We will also discuss its potential applications in X-ray optics and transmission electron microscopy.]]></abstract>
##     <issn><![CDATA[1943-0655]]></issn>
##     <htmlFlag><![CDATA[1]]></htmlFlag>
##     <arnumber><![CDATA[6754138]]></arnumber>
##     <doi><![CDATA[10.1109/JPHOT.2014.2308632]]></doi>
##     <publicationId><![CDATA[6754138]]></publicationId>
##     <mdurl><![CDATA[http://ieeexplore.ieee.org/xpl/articleDetails.jsp?tp=&arnumber=6754138&contentType=Journals+%26+Magazines]]></mdurl>
##     <pdf><![CDATA[http://ieeexplore.ieee.org/stamp/stamp.jsp?arnumber=6754138]]></pdf>
##   </document>
##   <document>
##     <rank>885</rank>
##     <title><![CDATA[Diagnosis of Active Systems by Semantic Patterns]]></title>
##     <authors><![CDATA[Lamperti, G.;  XiangFu Zhao]]></authors>
##     <affiliations><![CDATA[Dept. of Inf. Eng., Univ. of Brescia, Brescia, Italy]]></affiliations>
##     <controlledterms>
##       <term><![CDATA[discrete event systems]]></term>
##       <term><![CDATA[failure analysis]]></term>
##       <term><![CDATA[fault diagnosis]]></term>
##       <term><![CDATA[formal languages]]></term>
##       <term><![CDATA[reliability theory]]></term>
##     </controlledterms>
##     <thesaurusterms>
##       <term><![CDATA[Automata]]></term>
##       <term><![CDATA[Circuit faults]]></term>
##       <term><![CDATA[Context]]></term>
##       <term><![CDATA[History]]></term>
##       <term><![CDATA[Monitoring]]></term>
##       <term><![CDATA[Semantics]]></term>
##       <term><![CDATA[Syntactics]]></term>
##     </thesaurusterms>
##     <pubtitle><![CDATA[Systems, Man, and Cybernetics: Systems, IEEE Transactions on]]></pubtitle>
##     <punumber><![CDATA[6221021]]></punumber>
##     <pubtype><![CDATA[Journals & Magazines]]></pubtype>
##     <publisher><![CDATA[IEEE]]></publisher>
##     <volume><![CDATA[44]]></volume>
##     <issue><![CDATA[8]]></issue>
##     <py><![CDATA[2014]]></py>
##     <spage><![CDATA[1028]]></spage>
##     <epage><![CDATA[1043]]></epage>
##     <abstract><![CDATA[A gap still exists between complex discrete-event systems (DESs) and the effectiveness of the state-of-the-art diagnosis techniques, where faults are defined at component levels and diagnoses incorporate the occurrences of component faults. All these approaches to diagnosis are context-free, in as much diagnosis is anchored to components, irrespective of the context in which they are embedded. By contrast, since complex DESs are naturally organized in hierarchies of contexts, different diagnosis rules are to be defined for different contexts. Diagnosis rules are specified based on associations between context-sensitive faults and regular expressions, called semantic patterns. Since the alphabets of such regular expressions are stratified, so that the semantic patterns of a context are defined based on the interface symbols of its subcontexts only, separation of concerns is achieved, and the expressive power of diagnosis is enhanced. This new approach to diagnosis is bound to seemingly contradictory but nonetheless possible scenarios: a DES can be normal despite the faulty behavior of a number of its components; also, it can be faulty despite the normal behavior of all its components.]]></abstract>
##     <issn><![CDATA[2168-2216]]></issn>
##     <htmlFlag><![CDATA[1]]></htmlFlag>
##     <arnumber><![CDATA[6725692]]></arnumber>
##     <doi><![CDATA[10.1109/TSMC.2013.2296277]]></doi>
##     <publicationId><![CDATA[6725692]]></publicationId>
##     <mdurl><![CDATA[http://ieeexplore.ieee.org/xpl/articleDetails.jsp?tp=&arnumber=6725692&contentType=Journals+%26+Magazines]]></mdurl>
##     <pdf><![CDATA[http://ieeexplore.ieee.org/stamp/stamp.jsp?arnumber=6725692]]></pdf>
##   </document>
##   <document>
##     <rank>886</rank>
##     <title><![CDATA[Power-Efficient Tomlinson-Harashima Precoding for the Downlink of Multi-User MISO Systems]]></title>
##     <authors><![CDATA[Garcia-Rodriguez, A.;  Masouros, C.]]></authors>
##     <affiliations><![CDATA[Dept. of Electron. & Electr. Eng., Univ. Coll. London, London, UK]]></affiliations>
##     <controlledterms>
##       <term><![CDATA[computational complexity]]></term>
##       <term><![CDATA[energy conservation]]></term>
##       <term><![CDATA[mobile radio]]></term>
##       <term><![CDATA[multi-access systems]]></term>
##       <term><![CDATA[optimisation]]></term>
##       <term><![CDATA[precoding]]></term>
##       <term><![CDATA[radio transmitters]]></term>
##       <term><![CDATA[radiofrequency interference]]></term>
##       <term><![CDATA[signal processing]]></term>
##       <term><![CDATA[telecommunication power management]]></term>
##     </controlledterms>
##     <thesaurusterms>
##       <term><![CDATA[Interference]]></term>
##       <term><![CDATA[Licenses]]></term>
##       <term><![CDATA[Modulation]]></term>
##       <term><![CDATA[Optimization]]></term>
##       <term><![CDATA[Receivers]]></term>
##       <term><![CDATA[Signal to noise ratio]]></term>
##       <term><![CDATA[Transmitters]]></term>
##     </thesaurusterms>
##     <pubtitle><![CDATA[Communications, IEEE Transactions on]]></pubtitle>
##     <punumber><![CDATA[26]]></punumber>
##     <pubtype><![CDATA[Journals & Magazines]]></pubtype>
##     <publisher><![CDATA[IEEE]]></publisher>
##     <volume><![CDATA[62]]></volume>
##     <issue><![CDATA[6]]></issue>
##     <py><![CDATA[2014]]></py>
##     <spage><![CDATA[1884]]></spage>
##     <epage><![CDATA[1896]]></epage>
##     <abstract><![CDATA[We propose a power-efficient Tomlinson-Harashima precoder (THP) in the downlink of multi-user multiple-input single-output (MU-MISO) systems, where a transmit power reduction is achieved by means of interference optimization. The adopted approach is based on adaptively scaling the symbols of a number of users whose received signal-to-noise ratio (SNR) thresholds are known to the transmitter. By doing this, the interference can be better aligned to the symbols of interest, thus reducing the power required to cancel it. The scaling is performed by forming a constrained optimization problem, solved with existing well-known techniques, which entails an increase in the computational complexity at the base station. To quantify this trade-off in performance and complexity, a study of the impact in the signal processing load is carried out by means of a power efficiency analysis. The presented analytical and simulation results in this paper confirm that the proposed technique increases the power efficiency up to 100% with respect to previous THP-based approaches while, at the same time, maintaining the same average performance.]]></abstract>
##     <issn><![CDATA[0090-6778]]></issn>
##     <htmlFlag><![CDATA[1]]></htmlFlag>
##     <arnumber><![CDATA[6797864]]></arnumber>
##     <doi><![CDATA[10.1109/TCOMM.2014.2317189]]></doi>
##     <publicationId><![CDATA[6797864]]></publicationId>
##     <mdurl><![CDATA[http://ieeexplore.ieee.org/xpl/articleDetails.jsp?tp=&arnumber=6797864&contentType=Journals+%26+Magazines]]></mdurl>
##     <pdf><![CDATA[http://ieeexplore.ieee.org/stamp/stamp.jsp?arnumber=6797864]]></pdf>
##   </document>
##   <document>
##     <rank>887</rank>
##     <title><![CDATA[Particle Filter for Fault Diagnosis and Robust Navigation of Underwater Robot]]></title>
##     <authors><![CDATA[Bo Zhao;  Skjetne, R.;  Blanke, M.;  Dukan, F.]]></authors>
##     <affiliations><![CDATA[Dept. of Marine Technol., Norwegian Univ. of Sci. & Technol., Trondheim, Norway]]></affiliations>
##     <controlledterms>
##       <term><![CDATA[autonomous underwater vehicles]]></term>
##       <term><![CDATA[fault diagnosis]]></term>
##       <term><![CDATA[fault tolerant control]]></term>
##       <term><![CDATA[hidden Markov models]]></term>
##       <term><![CDATA[particle filtering (numerical methods)]]></term>
##       <term><![CDATA[path planning]]></term>
##       <term><![CDATA[robust control]]></term>
##       <term><![CDATA[sensors]]></term>
##       <term><![CDATA[state estimation]]></term>
##     </controlledterms>
##     <thesaurusterms>
##       <term><![CDATA[Fault diagnosis]]></term>
##       <term><![CDATA[Fault tolerance]]></term>
##       <term><![CDATA[Hidden Markov models]]></term>
##       <term><![CDATA[Navigation]]></term>
##       <term><![CDATA[Particle filters]]></term>
##       <term><![CDATA[Robustness]]></term>
##       <term><![CDATA[Underwater vehicles]]></term>
##     </thesaurusterms>
##     <pubtitle><![CDATA[Control Systems Technology, IEEE Transactions on]]></pubtitle>
##     <punumber><![CDATA[87]]></punumber>
##     <pubtype><![CDATA[Journals & Magazines]]></pubtype>
##     <publisher><![CDATA[IEEE]]></publisher>
##     <volume><![CDATA[22]]></volume>
##     <issue><![CDATA[6]]></issue>
##     <py><![CDATA[2014]]></py>
##     <spage><![CDATA[2399]]></spage>
##     <epage><![CDATA[2407]]></epage>
##     <abstract><![CDATA[A particle filter (PF)-based robust navigation with fault diagnosis (FD) is designed for an underwater robot, where 10 failure modes of sensors and thrusters are considered. The nominal underwater robot and its anomaly are described by a switching-mode hidden Markov model. By extensively running a PF on the model, the FD and robust navigation are achieved. Closed-loop full-scale experimental results show that the proposed method is robust, can diagnose faults effectively, and can provide good state estimation even in cases where multiple faults occur. Comparing with other methods, the proposed method can diagnose all faults within a single structure, it can diagnose simultaneous faults, and it is easily implemented.]]></abstract>
##     <issn><![CDATA[1063-6536]]></issn>
##     <htmlFlag><![CDATA[1]]></htmlFlag>
##     <arnumber><![CDATA[6736066]]></arnumber>
##     <doi><![CDATA[10.1109/TCST.2014.2300815]]></doi>
##     <publicationId><![CDATA[6736066]]></publicationId>
##     <mdurl><![CDATA[http://ieeexplore.ieee.org/xpl/articleDetails.jsp?tp=&arnumber=6736066&contentType=Journals+%26+Magazines]]></mdurl>
##     <pdf><![CDATA[http://ieeexplore.ieee.org/stamp/stamp.jsp?arnumber=6736066]]></pdf>
##   </document>
##   <document>
##     <rank>888</rank>
##     <title><![CDATA[Event-Based Mobile Social Networks: Services, Technologies, and Applications]]></title>
##     <authors><![CDATA[Ahmed, A.M.;  Tie Qiu;  Feng Xia;  Jedari, B.;  Abolfazli, S.]]></authors>
##     <affiliations><![CDATA[Sch. of Software, Dalian Univ. of Technol., Dalian, China]]></affiliations>
##     <controlledterms>
##       <term><![CDATA[Internet of Things]]></term>
##       <term><![CDATA[mobile computing]]></term>
##       <term><![CDATA[multimedia systems]]></term>
##       <term><![CDATA[smart phones]]></term>
##       <term><![CDATA[social networking (online)]]></term>
##     </controlledterms>
##     <thesaurusterms>
##       <term><![CDATA[Computer applications]]></term>
##       <term><![CDATA[Event detection]]></term>
##       <term><![CDATA[Mobile communication]]></term>
##       <term><![CDATA[Smart phones]]></term>
##       <term><![CDATA[Social network services]]></term>
##     </thesaurusterms>
##     <pubtitle><![CDATA[Access, IEEE]]></pubtitle>
##     <punumber><![CDATA[6287639]]></punumber>
##     <pubtype><![CDATA[Journals & Magazines]]></pubtype>
##     <publisher><![CDATA[IEEE]]></publisher>
##     <volume><![CDATA[2]]></volume>
##     <py><![CDATA[2014]]></py>
##     <spage><![CDATA[500]]></spage>
##     <epage><![CDATA[513]]></epage>
##     <abstract><![CDATA[Event-based mobile social networks (MSNs) are a special type of MSN that has an immanently temporal common feature, which allows any smart phone user to create events to share group messaging, locations, photos, and insights among participants. The emergence of Internet of Things and event-based social applications integrated with context-awareness ability can be helpful in planning and organizing social events like meetings, conferences, and tradeshows. This paper first provides review of the event-based social networks and the basic principles and architecture of event-based MSNs. Next, event-based MSNs with smartphone contained technology elements, such as context-aware mobility and multimedia sharing, are presented. By combining the feature of context-aware mobility with multimedia sharing in event-based MSNs, event organizers, and planners with the service providers optimize their capability to recognize value for the multimedia services they deliver. The unique features of the current event-based MSNs give rise to the major technology trends to watch for designing applications. These mobile applications and their main features are described. At the end, discussions on the evaluation of the event-based mobile applications based on their main features are presented. Some open research issues and challenges in this important area of research are also outlined.]]></abstract>
##     <issn><![CDATA[2169-3536]]></issn>
##     <htmlFlag><![CDATA[1]]></htmlFlag>
##     <arnumber><![CDATA[6805126]]></arnumber>
##     <doi><![CDATA[10.1109/ACCESS.2014.2319823]]></doi>
##     <publicationId><![CDATA[6805126]]></publicationId>
##     <mdurl><![CDATA[http://ieeexplore.ieee.org/xpl/articleDetails.jsp?tp=&arnumber=6805126&contentType=Journals+%26+Magazines]]></mdurl>
##     <pdf><![CDATA[http://ieeexplore.ieee.org/stamp/stamp.jsp?arnumber=6805126]]></pdf>
##   </document>
##   <document>
##     <rank>889</rank>
##     <title><![CDATA[Millimeter-Wave Cellular Wireless Networks: Potentials and Challenges]]></title>
##     <authors><![CDATA[Rangan, S.;  Rappaport, T.S.;  Erkip, E.]]></authors>
##     <affiliations><![CDATA[NYU WIRELESS Center, New York Univ., New York, NY, USA]]></affiliations>
##     <controlledterms>
##       <term><![CDATA[4G mobile communication]]></term>
##       <term><![CDATA[antenna arrays]]></term>
##       <term><![CDATA[array signal processing]]></term>
##       <term><![CDATA[cellular radio]]></term>
##       <term><![CDATA[channel capacity]]></term>
##       <term><![CDATA[space division multiplexing]]></term>
##     </controlledterms>
##     <thesaurusterms>
##       <term><![CDATA[Antennas]]></term>
##       <term><![CDATA[Atmospheric measurements]]></term>
##       <term><![CDATA[Bandwidth]]></term>
##       <term><![CDATA[Cellular networks]]></term>
##       <term><![CDATA[Millimeter wave communication]]></term>
##       <term><![CDATA[Radio spectrum management]]></term>
##       <term><![CDATA[Shadow mapping]]></term>
##     </thesaurusterms>
##     <pubtitle><![CDATA[Proceedings of the IEEE]]></pubtitle>
##     <punumber><![CDATA[5]]></punumber>
##     <pubtype><![CDATA[Journals & Magazines]]></pubtype>
##     <publisher><![CDATA[IEEE]]></publisher>
##     <volume><![CDATA[102]]></volume>
##     <issue><![CDATA[3]]></issue>
##     <py><![CDATA[2014]]></py>
##     <spage><![CDATA[366]]></spage>
##     <epage><![CDATA[385]]></epage>
##     <abstract><![CDATA[Millimeter-wave (mmW) frequencies between 30 and 300 GHz are a new frontier for cellular communication that offers the promise of orders of magnitude greater bandwidths combined with further gains via beamforming and spatial multiplexing from multielement antenna arrays. This paper surveys measurements and capacity studies to assess this technology with a focus on small cell deployments in urban environments. The conclusions are extremely encouraging; measurements in New York City at 28 and 73 GHz demonstrate that, even in an urban canyon environment, significant non-line-of-sight (NLOS) outdoor, street-level coverage is possible up to approximately 200 m from a potential low-power microcell or picocell base station. In addition, based on statistical channel models from these measurements, it is shown that mmW systems can offer more than an order of magnitude increase in capacity over current state-of-the-art 4G cellular networks at current cell densities. Cellular systems, however, will need to be significantly redesigned to fully achieve these gains. Specifically, the requirement of highly directional and adaptive transmissions, directional isolation between links, and significant possibilities of outage have strong implications on multiple access, channel structure, synchronization, and receiver design. To address these challenges, the paper discusses how various technologies including adaptive beamforming, multihop relaying, heterogeneous network architectures, and carrier aggregation can be leveraged in the mmW context.]]></abstract>
##     <issn><![CDATA[0018-9219]]></issn>
##     <htmlFlag><![CDATA[1]]></htmlFlag>
##     <arnumber><![CDATA[6732923]]></arnumber>
##     <doi><![CDATA[10.1109/JPROC.2014.2299397]]></doi>
##     <publicationId><![CDATA[6732923]]></publicationId>
##     <mdurl><![CDATA[http://ieeexplore.ieee.org/xpl/articleDetails.jsp?tp=&arnumber=6732923&contentType=Journals+%26+Magazines]]></mdurl>
##     <pdf><![CDATA[http://ieeexplore.ieee.org/stamp/stamp.jsp?arnumber=6732923]]></pdf>
##   </document>
##   <document>
##     <rank>890</rank>
##     <title><![CDATA[Monolithically Integrated Optical Phase Lock Loop for Microwave Photonics]]></title>
##     <authors><![CDATA[Balakier, K.;  Fice, M.J.;  Ponnampalam, L.;  Seeds, A.J.;  Renaud, C.C.]]></authors>
##     <affiliations><![CDATA[Dept. of Electron. & Electr. Eng., Univ. Coll. London, London, UK]]></affiliations>
##     <controlledterms>
##       <term><![CDATA[closed loop systems]]></term>
##       <term><![CDATA[heterodyne detection]]></term>
##       <term><![CDATA[integrated optoelectronics]]></term>
##       <term><![CDATA[laser mode locking]]></term>
##       <term><![CDATA[laser noise]]></term>
##       <term><![CDATA[microwave generation]]></term>
##       <term><![CDATA[microwave photonics]]></term>
##       <term><![CDATA[optical delay lines]]></term>
##       <term><![CDATA[optical design techniques]]></term>
##       <term><![CDATA[optical phase locked loops]]></term>
##       <term><![CDATA[phase noise]]></term>
##       <term><![CDATA[reviews]]></term>
##       <term><![CDATA[semiconductor lasers]]></term>
##     </controlledterms>
##     <thesaurusterms>
##       <term><![CDATA[Laser feedback]]></term>
##       <term><![CDATA[Laser tuning]]></term>
##       <term><![CDATA[Measurement by laser beam]]></term>
##       <term><![CDATA[Optical filters]]></term>
##       <term><![CDATA[Optical mixing]]></term>
##       <term><![CDATA[Photonics]]></term>
##     </thesaurusterms>
##     <pubtitle><![CDATA[Lightwave Technology, Journal of]]></pubtitle>
##     <punumber><![CDATA[50]]></punumber>
##     <pubtype><![CDATA[Journals & Magazines]]></pubtype>
##     <publisher><![CDATA[IEEE]]></publisher>
##     <volume><![CDATA[32]]></volume>
##     <issue><![CDATA[20]]></issue>
##     <py><![CDATA[2014]]></py>
##     <spage><![CDATA[3893]]></spage>
##     <epage><![CDATA[3900]]></epage>
##     <abstract><![CDATA[We present a review of the critical design aspects of monolithically integrated optical phase lock loops (OPLLs). OPLL design procedures and OPLL parameters are discussed. A technique to evaluate the gain of the closed loop operating system is introduced and experimentally validated for the first time. A dual-OPLL system, when synchronised to an optical frequency comb generator without any prior filtering of the comb lines, allows generation of high spectral purity signals at any desired frequency from several GHz up to THz range. Heterodyne phase locking was achieved at a continuously tuneable offset frequency between 2 and 6 GHz. Thanks to the photonic integration, small dimensions, and custom-made electronics, the propagation delay in the loop was less than 1.8 ns, allowing good phase noise performance with OPLLs based on lasers with linewidths less than a few MHz. The system demonstrates the potential for photonic integration to be applied in various microwave photonics applications where narrow-bandwidth tuneable optical filters with amplification functionality are required.]]></abstract>
##     <issn><![CDATA[0733-8724]]></issn>
##     <htmlFlag><![CDATA[1]]></htmlFlag>
##     <arnumber><![CDATA[6800012]]></arnumber>
##     <doi><![CDATA[10.1109/JLT.2014.2317941]]></doi>
##     <publicationId><![CDATA[6800012]]></publicationId>
##     <mdurl><![CDATA[http://ieeexplore.ieee.org/xpl/articleDetails.jsp?tp=&arnumber=6800012&contentType=Journals+%26+Magazines]]></mdurl>
##     <pdf><![CDATA[http://ieeexplore.ieee.org/stamp/stamp.jsp?arnumber=6800012]]></pdf>
##   </document>
##   <document>
##     <rank>891</rank>
##     <title><![CDATA[De-Polarization of On-Body Channels and Polarization Diversity at 60 GHz]]></title>
##     <authors><![CDATA[Nechayev, Y.I.;  Constantinou, C.C.;  Xianyue Wu;  Hall, P.S.]]></authors>
##     <affiliations><![CDATA[Sch. of Electron., Electr. & Comput. Eng., Univ. of Birmingham, Birmingham, UK]]></affiliations>
##     <controlledterms>
##       <term><![CDATA[MIMO communication]]></term>
##       <term><![CDATA[body area networks]]></term>
##       <term><![CDATA[diversity reception]]></term>
##       <term><![CDATA[electromagnetic wave polarisation]]></term>
##       <term><![CDATA[horn antennas]]></term>
##       <term><![CDATA[millimetre wave antennas]]></term>
##       <term><![CDATA[millimetre wave propagation]]></term>
##       <term><![CDATA[radio links]]></term>
##       <term><![CDATA[wireless channels]]></term>
##     </controlledterms>
##     <thesaurusterms>
##       <term><![CDATA[Antenna measurements]]></term>
##       <term><![CDATA[Diversity methods]]></term>
##       <term><![CDATA[Diversity reception]]></term>
##       <term><![CDATA[Horn antennas]]></term>
##       <term><![CDATA[MIMO]]></term>
##     </thesaurusterms>
##     <pubtitle><![CDATA[Antennas and Propagation, IEEE Transactions on]]></pubtitle>
##     <punumber><![CDATA[8]]></punumber>
##     <pubtype><![CDATA[Journals & Magazines]]></pubtype>
##     <publisher><![CDATA[IEEE]]></publisher>
##     <volume><![CDATA[62]]></volume>
##     <issue><![CDATA[12]]></issue>
##     <py><![CDATA[2014]]></py>
##     <spage><![CDATA[6519]]></spage>
##     <epage><![CDATA[6523]]></epage>
##     <abstract><![CDATA[Measurements of on-body dynamic propagation channels have been performed at 60 GHz using dual-polarized scalar horn antennas. Comparison of the statistics of the measured signals showed that the choice of polarization (vertical or horizontal) does not affect the path losses significantly, and the relative polarization of the two antennas depends on their placement on the body. In volatile links, such as those from the wrist to other parts of the body, random movements equalize the differences between different polarization configurations. These depolarization effects can be exploited to improve link performance through the use of receive diversity with maximal ratio combining. More advanced multiple-input multiple-output diversity methods are found to produce only marginally better performance compared to receive maximal ratio combining.]]></abstract>
##     <issn><![CDATA[0018-926X]]></issn>
##     <htmlFlag><![CDATA[1]]></htmlFlag>
##     <arnumber><![CDATA[6915677]]></arnumber>
##     <doi><![CDATA[10.1109/TAP.2014.2361140]]></doi>
##     <publicationId><![CDATA[6915677]]></publicationId>
##     <mdurl><![CDATA[http://ieeexplore.ieee.org/xpl/articleDetails.jsp?tp=&arnumber=6915677&contentType=Journals+%26+Magazines]]></mdurl>
##     <pdf><![CDATA[http://ieeexplore.ieee.org/stamp/stamp.jsp?arnumber=6915677]]></pdf>
##   </document>
##   <document>
##     <rank>892</rank>
##     <title><![CDATA[Simple and Causal Copper Cable Model Suitable for G.fast Frequencies]]></title>
##     <authors><![CDATA[Acatauassu, D.;  Host, S.;  Chenguang Lu;  Berg, M.;  Klautau, A.;  Borjesson, P.O.]]></authors>
##     <affiliations><![CDATA[Signal Process. Lab., Fed. Univ. of Para, Belem, Brazil]]></affiliations>
##     <controlledterms>
##       <term><![CDATA[copper]]></term>
##       <term><![CDATA[digital subscriber lines]]></term>
##       <term><![CDATA[transmission line theory]]></term>
##       <term><![CDATA[twisted pair cables]]></term>
##     </controlledterms>
##     <thesaurusterms>
##       <term><![CDATA[Communication cables]]></term>
##       <term><![CDATA[Copper]]></term>
##       <term><![CDATA[Crosstalk]]></term>
##       <term><![CDATA[Impedance]]></term>
##       <term><![CDATA[Mathematical model]]></term>
##       <term><![CDATA[Power cables]]></term>
##       <term><![CDATA[Propagation constant]]></term>
##     </thesaurusterms>
##     <pubtitle><![CDATA[Communications, IEEE Transactions on]]></pubtitle>
##     <punumber><![CDATA[26]]></punumber>
##     <pubtype><![CDATA[Journals & Magazines]]></pubtype>
##     <publisher><![CDATA[IEEE]]></publisher>
##     <volume><![CDATA[62]]></volume>
##     <issue><![CDATA[11]]></issue>
##     <py><![CDATA[2014]]></py>
##     <spage><![CDATA[4040]]></spage>
##     <epage><![CDATA[4051]]></epage>
##     <abstract><![CDATA[G.fast is a new standard from the International Telecommunication Union, which targets 1 Gb/s over short copper loops using frequencies up to 212 MHz. This new technology requires accurate parametric cable models for simulation, design, and performance evaluation tests. Some existing copper cable models were designed for the very high speed digital subscriber line spectra, i.e., frequencies up to 30 MHz, and adopt assumptions that are violated when the frequency range is extended to G.fast frequencies. This paper introduces a simple and causal cable model that is able to accurately characterize copper loops composed by single or multiple segments, in both frequency and time domains. Results using G.fast topologies show that, apart from being accurate, the new model is attractive due to its low computational cost and closed-form expressions for fitting its parameters to measurement data.]]></abstract>
##     <issn><![CDATA[0090-6778]]></issn>
##     <htmlFlag><![CDATA[1]]></htmlFlag>
##     <arnumber><![CDATA[6934987]]></arnumber>
##     <doi><![CDATA[10.1109/TCOMM.2014.2364585]]></doi>
##     <publicationId><![CDATA[6934987]]></publicationId>
##     <mdurl><![CDATA[http://ieeexplore.ieee.org/xpl/articleDetails.jsp?tp=&arnumber=6934987&contentType=Journals+%26+Magazines]]></mdurl>
##     <pdf><![CDATA[http://ieeexplore.ieee.org/stamp/stamp.jsp?arnumber=6934987]]></pdf>
##   </document>
##   <document>
##     <rank>893</rank>
##     <title><![CDATA[Biconditional Binary Decision Diagrams: A Novel Canonical Logic Representation Form]]></title>
##     <authors><![CDATA[Amaru, L.;  Gaillardon, P.-E.;  De Micheli, G.]]></authors>
##     <affiliations><![CDATA[Integrated Syst. Lab., Swiss Fed. Inst. of Technol.-Lausanne-EPFL, Lausanne, Switzerland]]></affiliations>
##     <controlledterms>
##       <term><![CDATA[Boolean functions]]></term>
##       <term><![CDATA[binary decision diagrams]]></term>
##       <term><![CDATA[comparators (circuits)]]></term>
##       <term><![CDATA[electronic design automation]]></term>
##       <term><![CDATA[software packages]]></term>
##     </controlledterms>
##     <thesaurusterms>
##       <term><![CDATA[Binary decision diagrams]]></term>
##       <term><![CDATA[Boolean functions]]></term>
##       <term><![CDATA[Design automation]]></term>
##       <term><![CDATA[Digital circuits]]></term>
##     </thesaurusterms>
##     <pubtitle><![CDATA[Emerging and Selected Topics in Circuits and Systems, IEEE Journal on]]></pubtitle>
##     <punumber><![CDATA[5503868]]></punumber>
##     <pubtype><![CDATA[Journals & Magazines]]></pubtype>
##     <publisher><![CDATA[IEEE]]></publisher>
##     <volume><![CDATA[4]]></volume>
##     <issue><![CDATA[4]]></issue>
##     <py><![CDATA[2014]]></py>
##     <spage><![CDATA[487]]></spage>
##     <epage><![CDATA[500]]></epage>
##     <abstract><![CDATA[In this paper, we present biconditional binary decision diagrams (BBDDs), a novel canonical representation form for Boolean functions. BBDDs are binary decision diagrams where the branching condition, and its associated logic expansion, is biconditional on two variables. Empowered by reduction and ordering rules, BBDDs are remarkably compact and unique for a Boolean function. The interest of such representation form in modern electronic design automation (EDA) is twofold. On the one hand, BBDDs improve the efficiency of traditional EDA tasks based on decision diagrams, especially for arithmetic intensive designs. On the other hand, BBDDs represent the natural and native design abstraction for emerging technologies where the circuit primitive is a comparator, rather than a simple switch. We provide, in this paper, a solid ground for BBDDs by studying their underlying theory and manipulation properties. Thanks to an efficient BBDD software package implementation, we validate (1) speed-up in traditional decision diagrams applications with up to 4.4 &#x00D7; gain with respect to other DDs, and (2) improved synthesis of circuits in emerging technologies, with about 32% shorter critical path than state-of-art synthesis techniques.]]></abstract>
##     <issn><![CDATA[2156-3357]]></issn>
##     <htmlFlag><![CDATA[1]]></htmlFlag>
##     <arnumber><![CDATA[6924810]]></arnumber>
##     <doi><![CDATA[10.1109/JETCAS.2014.2361058]]></doi>
##     <publicationId><![CDATA[6924810]]></publicationId>
##     <mdurl><![CDATA[http://ieeexplore.ieee.org/xpl/articleDetails.jsp?tp=&arnumber=6924810&contentType=Journals+%26+Magazines]]></mdurl>
##     <pdf><![CDATA[http://ieeexplore.ieee.org/stamp/stamp.jsp?arnumber=6924810]]></pdf>
##   </document>
##   <document>
##     <rank>894</rank>
##     <title><![CDATA[Unsupervised content classification based nonrigid registration of differently stained histology images]]></title>
##     <authors><![CDATA[Song, Y.;  Treanor, D.;  Bulpitt, A.J.;  Wijayathunga, N.;  Roberts, N.;  Wilcox, R.;  Magee, D.R.]]></authors>
##     <affiliations><![CDATA[Sch. of Comput., Univ. of Leeds, Leeds, UK]]></affiliations>
##     <controlledterms>
##       <term><![CDATA[biological tissues]]></term>
##       <term><![CDATA[biomedical MRI]]></term>
##       <term><![CDATA[diseases]]></term>
##       <term><![CDATA[image classification]]></term>
##       <term><![CDATA[image registration]]></term>
##       <term><![CDATA[image sequences]]></term>
##       <term><![CDATA[liver]]></term>
##       <term><![CDATA[medical image processing]]></term>
##       <term><![CDATA[probability]]></term>
##     </controlledterms>
##     <thesaurusterms>
##       <term><![CDATA[Classification algorithms]]></term>
##       <term><![CDATA[Educational institutions]]></term>
##       <term><![CDATA[Image color analysis]]></term>
##       <term><![CDATA[Liver]]></term>
##       <term><![CDATA[Microscopy]]></term>
##       <term><![CDATA[Support vector machine classification]]></term>
##       <term><![CDATA[Vectors]]></term>
##     </thesaurusterms>
##     <pubtitle><![CDATA[Biomedical Engineering, IEEE Transactions on]]></pubtitle>
##     <punumber><![CDATA[10]]></punumber>
##     <pubtype><![CDATA[Journals & Magazines]]></pubtype>
##     <publisher><![CDATA[IEEE]]></publisher>
##     <volume><![CDATA[61]]></volume>
##     <issue><![CDATA[1]]></issue>
##     <py><![CDATA[2014]]></py>
##     <spage><![CDATA[96]]></spage>
##     <epage><![CDATA[108]]></epage>
##     <abstract><![CDATA[Registration of histopathology images of consecutive tissue sections stained with different histochemical or immunohistochemical stains is an important step in a number of application areas, such as the investigation of the pathology of a disease, validation of MRI sequences against tissue images, multiscale physical modeling, etc. In each case, information from each stain needs to be spatially aligned and combined to ascertain physical or functional properties of the tissue. However, in addition to the gigabyte-size images and nonrigid distortions present in the tissue, a major challenge for registering differently stained histology image pairs is the dissimilar structural appearance due to different stains highlighting different substances in tissues. In this paper, we address this challenge by developing an unsupervised content classification method that generates multichannel probability images from a roughly aligned image pair. Each channel corresponds to one automatically identified content class. The probability images enhance the structural similarity between image pairs. By integrating the classification method into a multiresolution-block-matching-based nonrigid registration scheme (N. Roberts, D. Magee, Y. Song, K. Brabazon, M. Shires, D. Crellin, N. Orsi, P. Quirke, and D. Treanor, &#x201C;Toward routine use of 3D histopathology as a research tool,&#x201D; Amer. J. Pathology, vol. 180, no. 5, 2012.), we improve the performance of registering multistained histology images. Evaluation was conducted on 77 histological image pairs taken from three liver specimens and one intervertebral disc specimen. In total, six types of histochemical stains were tested. We evaluated our method against the same registration method implemented without applying the classification algorithm (intensity-based registration) and the state-of-the-art mutual information based registration. Superior results are obtained with the proposed method.]]></abstract>
##     <issn><![CDATA[0018-9294]]></issn>
##     <htmlFlag><![CDATA[1]]></htmlFlag>
##     <arnumber><![CDATA[6576835]]></arnumber>
##     <doi><![CDATA[10.1109/TBME.2013.2277777]]></doi>
##     <publicationId><![CDATA[6576835]]></publicationId>
##     <mdurl><![CDATA[http://ieeexplore.ieee.org/xpl/articleDetails.jsp?tp=&arnumber=6576835&contentType=Journals+%26+Magazines]]></mdurl>
##     <pdf><![CDATA[http://ieeexplore.ieee.org/stamp/stamp.jsp?arnumber=6576835]]></pdf>
##   </document>
##   <document>
##     <rank>895</rank>
##     <title><![CDATA[Coupling Effects Among Degenerate Modes in Multimode Optical Fibers]]></title>
##     <authors><![CDATA[Palmieri, L.;  Galtarossa, A.]]></authors>
##     <affiliations><![CDATA[Dept. of Inf. Eng., Univ. of Padova, Padua, Italy]]></affiliations>
##     <controlledterms>
##       <term><![CDATA[optical design techniques]]></term>
##       <term><![CDATA[optical fibre couplers]]></term>
##       <term><![CDATA[space division multiplexing]]></term>
##     </controlledterms>
##     <thesaurusterms>
##       <term><![CDATA[Azimuth]]></term>
##       <term><![CDATA[Birefringence]]></term>
##       <term><![CDATA[Manifolds]]></term>
##       <term><![CDATA[Optical coupling]]></term>
##       <term><![CDATA[Propagation constant]]></term>
##       <term><![CDATA[Silicon compounds]]></term>
##     </thesaurusterms>
##     <pubtitle><![CDATA[Photonics Journal, IEEE]]></pubtitle>
##     <punumber><![CDATA[4563994]]></punumber>
##     <pubtype><![CDATA[Journals & Magazines]]></pubtype>
##     <publisher><![CDATA[IEEE]]></publisher>
##     <volume><![CDATA[6]]></volume>
##     <issue><![CDATA[6]]></issue>
##     <py><![CDATA[2014]]></py>
##     <spage><![CDATA[1]]></spage>
##     <epage><![CDATA[8]]></epage>
##     <abstract><![CDATA[Multimode optical fibers have recently received revived attention in the framework of space-division multiplexed systems, where the spatial diversity of fiber modes is exploited to increase transmission capacity. The complexity of these systems strongly depends on the coupling characteristic of the fiber. Therefore, a better description of coupling effects may lead to the more accurate modeling of the system and to an optimized design of multimode fibers. In this paper, we analyze coupling among (quasi) degenerate modes as a consequence of different kinds of coupling sources.]]></abstract>
##     <issn><![CDATA[1943-0655]]></issn>
##     <htmlFlag><![CDATA[1]]></htmlFlag>
##     <arnumber><![CDATA[6867306]]></arnumber>
##     <doi><![CDATA[10.1109/JPHOT.2014.2343998]]></doi>
##     <publicationId><![CDATA[6867306]]></publicationId>
##     <mdurl><![CDATA[http://ieeexplore.ieee.org/xpl/articleDetails.jsp?tp=&arnumber=6867306&contentType=Journals+%26+Magazines]]></mdurl>
##     <pdf><![CDATA[http://ieeexplore.ieee.org/stamp/stamp.jsp?arnumber=6867306]]></pdf>
##   </document>
##   <document>
##     <rank>896</rank>
##     <title><![CDATA[Marginal Likelihoods for Distributed Parameter Estimation of Gaussian Graphical Models]]></title>
##     <authors><![CDATA[Zhaoshi Meng;  Wei, D.;  Wiesel, A.;  Hero, A.O.]]></authors>
##     <affiliations><![CDATA[Dept. of Electr. Eng. & Comput. Sci., Univ. of Michigan, Ann Arbor, MI, USA]]></affiliations>
##     <controlledterms>
##       <term><![CDATA[Gaussian processes]]></term>
##       <term><![CDATA[covariance matrices]]></term>
##       <term><![CDATA[graph theory]]></term>
##       <term><![CDATA[iterative methods]]></term>
##       <term><![CDATA[maximum likelihood estimation]]></term>
##       <term><![CDATA[message passing]]></term>
##     </controlledterms>
##     <thesaurusterms>
##       <term><![CDATA[Covariance matrices]]></term>
##       <term><![CDATA[Graphical models]]></term>
##       <term><![CDATA[Indexes]]></term>
##       <term><![CDATA[Maximum likelihood estimation]]></term>
##       <term><![CDATA[Nickel]]></term>
##       <term><![CDATA[Parameter estimation]]></term>
##     </thesaurusterms>
##     <pubtitle><![CDATA[Signal Processing, IEEE Transactions on]]></pubtitle>
##     <punumber><![CDATA[78]]></punumber>
##     <pubtype><![CDATA[Journals & Magazines]]></pubtype>
##     <publisher><![CDATA[IEEE]]></publisher>
##     <volume><![CDATA[62]]></volume>
##     <issue><![CDATA[20]]></issue>
##     <py><![CDATA[2014]]></py>
##     <spage><![CDATA[5425]]></spage>
##     <epage><![CDATA[5438]]></epage>
##     <abstract><![CDATA[We consider distributed estimation of the inverse covariance matrix, also called the concentration or precision matrix, in Gaussian graphical models. Traditional centralized estimation often requires global inference of the covariance matrix, which can be computationally intensive in large dimensions. Approximate inference based on message-passing algorithms, on the other hand, can lead to unstable and biased estimation in loopy graphical models. Here, we propose a general framework for distributed estimation based on a maximum marginal likelihood (MML) approach. This approach computes local parameter estimates by maximizing marginal likelihoods defined with respect to data collected from local neighborhoods. Due to the non-convexity of the MML problem, we introduce and solve a convex relaxation. The local estimates are then combined into a global estimate without the need for iterative message-passing between neighborhoods. The proposed algorithm is naturally parallelizable and computationally efficient, thereby making it suitable for high-dimensional problems. In the classical regime where the number of variables p is fixed and the number of samples T increases to infinity, the proposed estimator is shown to be asymptotically consistent and to improve monotonically as the local neighborhood size increases. In the high-dimensional scaling regime where both p and T increase to infinity, the convergence rate to the true parameters is derived and is seen to be comparable to centralized maximum-likelihood estimation. Extensive numerical experiments demonstrate the improved performance of the two-hop version of the proposed estimator, which suffices to almost close the gap to the centralized maximum likelihood estimator at a reduced computational cost.]]></abstract>
##     <issn><![CDATA[1053-587X]]></issn>
##     <htmlFlag><![CDATA[1]]></htmlFlag>
##     <arnumber><![CDATA[6882196]]></arnumber>
##     <doi><![CDATA[10.1109/TSP.2014.2350956]]></doi>
##     <publicationId><![CDATA[6882196]]></publicationId>
##     <mdurl><![CDATA[http://ieeexplore.ieee.org/xpl/articleDetails.jsp?tp=&arnumber=6882196&contentType=Journals+%26+Magazines]]></mdurl>
##     <pdf><![CDATA[http://ieeexplore.ieee.org/stamp/stamp.jsp?arnumber=6882196]]></pdf>
##   </document>
##   <document>
##     <rank>897</rank>
##     <title><![CDATA[Characterizing the Spectral Properties and Time Variation of the In-Vehicle Wireless Communication Channel]]></title>
##     <authors><![CDATA[Herbert, S.;  Wassell, I.;  Tian-Hong Loh;  Rigelsford, J.]]></authors>
##     <affiliations><![CDATA[Comput. Lab., Univ. of Cambridge, Cambridge, UK]]></affiliations>
##     <controlledterms>
##       <term><![CDATA[mobile radio]]></term>
##       <term><![CDATA[radio receivers]]></term>
##       <term><![CDATA[wireless channels]]></term>
##     </controlledterms>
##     <thesaurusterms>
##       <term><![CDATA[Antenna measurements]]></term>
##       <term><![CDATA[Cavity resonators]]></term>
##       <term><![CDATA[Doppler effect]]></term>
##       <term><![CDATA[Frequency measurement]]></term>
##       <term><![CDATA[Receivers]]></term>
##       <term><![CDATA[Reverberation chambers]]></term>
##       <term><![CDATA[Vehicles]]></term>
##     </thesaurusterms>
##     <pubtitle><![CDATA[Communications, IEEE Transactions on]]></pubtitle>
##     <punumber><![CDATA[26]]></punumber>
##     <pubtype><![CDATA[Journals & Magazines]]></pubtype>
##     <publisher><![CDATA[IEEE]]></publisher>
##     <volume><![CDATA[62]]></volume>
##     <issue><![CDATA[7]]></issue>
##     <py><![CDATA[2014]]></py>
##     <spage><![CDATA[2390]]></spage>
##     <epage><![CDATA[2399]]></epage>
##     <abstract><![CDATA[To deploy effective communication systems in vehicle cavities, it is critical to understand the time variation of the in-vehicle channel. Initially, rapid channel variation is addressed, which is characterized in the frequency domain as a Doppler spread. It is then shown that, for typical Doppler spreads, the in-vehicle channel is underspread, and therefore, the information capacity approaches the capacity achieved with perfect receiver channel state information in the infinite bandwidth limit. Measurements are performed for a number of channel variation scenarios (e.g., absorptive motion, reflective motion, one antenna moving, and both antennas moving) at a number of carrier frequencies and for a number of cavity loading scenarios. It is found that the Doppler spread increases with carrier frequency; however, the type of channel variation and loading appear to have little effect. Channel variation over a longer time period is also measured to characterize the slower channel variation. Channel variation is a function of the cavity occupant motion, which is difficult to model theoretically; therefore, an empirical model for the slow channel variation is proposed, which leads to an improved estimate of the channel state.]]></abstract>
##     <issn><![CDATA[0090-6778]]></issn>
##     <htmlFlag><![CDATA[1]]></htmlFlag>
##     <arnumber><![CDATA[6825811]]></arnumber>
##     <doi><![CDATA[10.1109/TCOMM.2014.2328635]]></doi>
##     <publicationId><![CDATA[6825811]]></publicationId>
##     <mdurl><![CDATA[http://ieeexplore.ieee.org/xpl/articleDetails.jsp?tp=&arnumber=6825811&contentType=Journals+%26+Magazines]]></mdurl>
##     <pdf><![CDATA[http://ieeexplore.ieee.org/stamp/stamp.jsp?arnumber=6825811]]></pdf>
##   </document>
##   <document>
##     <rank>898</rank>
##     <title><![CDATA[A Variational Method to Solve Waveguide Problems by Employing Dynamic Programming Technique]]></title>
##     <authors><![CDATA[Yan Ju Chiang;  Likarn Wang]]></authors>
##     <affiliations><![CDATA[Dept. of Electron. Eng., Oriental Inst. of Technol., Taipei, Taiwan]]></affiliations>
##     <controlledterms>
##       <term><![CDATA[dynamic programming]]></term>
##       <term><![CDATA[finite difference methods]]></term>
##       <term><![CDATA[optical waveguides]]></term>
##       <term><![CDATA[rib waveguides]]></term>
##       <term><![CDATA[silicon-on-insulator]]></term>
##     </controlledterms>
##     <thesaurusterms>
##       <term><![CDATA[Dynamic programming]]></term>
##       <term><![CDATA[Equations]]></term>
##       <term><![CDATA[Mathematical model]]></term>
##       <term><![CDATA[Optical waveguides]]></term>
##       <term><![CDATA[Permittivity]]></term>
##       <term><![CDATA[Slabs]]></term>
##       <term><![CDATA[Transmission line matrix methods]]></term>
##     </thesaurusterms>
##     <pubtitle><![CDATA[Photonics Journal, IEEE]]></pubtitle>
##     <punumber><![CDATA[4563994]]></punumber>
##     <pubtype><![CDATA[Journals & Magazines]]></pubtype>
##     <publisher><![CDATA[IEEE]]></publisher>
##     <volume><![CDATA[6]]></volume>
##     <issue><![CDATA[5]]></issue>
##     <py><![CDATA[2014]]></py>
##     <spage><![CDATA[1]]></spage>
##     <epage><![CDATA[22]]></epage>
##     <abstract><![CDATA[A new variational method for the modal analysis of 2-D waveguide structures is proposed in this paper. Maxwell's equations describing the modal properties of scalar and semivectorial guided waves in the 2-D waveguides are expressed as variational/minimization problems, which are then discretized in terms of a finite-difference scheme on a finite rectangular computational window. By applying a dynamic programming technique to solve such variational problems, a 1-D equation representing the relation between the modal fields on any pair of adjacent columns in the computational window can be derived. By using such 1-D equation in a stepwise fashion from one boundary column toward the other boundary column, a system of linear equations with the unknown column modal fields can be derived and then solved to give both the accurate modal indexes and the discrete modal fields. In the examples of one weakly guiding rib-type dielectric waveguide and another strongly guiding silicon-on-insulator waveguide, computational results show that a small size of the coefficient matrix for such a system of linear equations is adequate to cause a relative error of 10<sup>-5</sup>-10<sup>-6</sup> in the evaluation of the modal indexes reachable in an efficient manner. The results of the convergence tests show that the proposed method is at least an order of magnitude faster than the conventional finite-difference beam propagation method because of the transformation of a 2-D problem into a 1-D problem. Moreover, the proposed method is applied to investigate the modal properties of the conductor-gap-silicon plasmonic waveguide. The feature of the hybrid guided-mode profile is also observable from the modal field calculated by the proposed method.]]></abstract>
##     <issn><![CDATA[1943-0655]]></issn>
##     <htmlFlag><![CDATA[1]]></htmlFlag>
##     <arnumber><![CDATA[6892923]]></arnumber>
##     <doi><![CDATA[10.1109/JPHOT.2014.2353627]]></doi>
##     <publicationId><![CDATA[6892923]]></publicationId>
##     <mdurl><![CDATA[http://ieeexplore.ieee.org/xpl/articleDetails.jsp?tp=&arnumber=6892923&contentType=Journals+%26+Magazines]]></mdurl>
##     <pdf><![CDATA[http://ieeexplore.ieee.org/stamp/stamp.jsp?arnumber=6892923]]></pdf>
##   </document>
##   <document>
##     <rank>899</rank>
##     <title><![CDATA[A Current-Shared Cascade Structure With an Auxiliary Power Regulator for Switching Mode RF Power Amplifiers]]></title>
##     <authors><![CDATA[Hoyong Hwang;  Changhyun Lee;  Jonghoon Park;  Changkun Park]]></authors>
##     <affiliations><![CDATA[Sch. of Electron. Eng., Soongsil Univ., Seoul, South Korea]]></affiliations>
##     <controlledterms>
##       <term><![CDATA[CMOS integrated circuits]]></term>
##       <term><![CDATA[MMIC power amplifiers]]></term>
##       <term><![CDATA[UHF power amplifiers]]></term>
##       <term><![CDATA[differential amplifiers]]></term>
##       <term><![CDATA[driver circuits]]></term>
##       <term><![CDATA[field effect MMIC]]></term>
##     </controlledterms>
##     <thesaurusterms>
##       <term><![CDATA[CMOS integrated circuits]]></term>
##       <term><![CDATA[Power demand]]></term>
##       <term><![CDATA[Power dissipation]]></term>
##       <term><![CDATA[Power generation]]></term>
##       <term><![CDATA[Regulators]]></term>
##       <term><![CDATA[Transistors]]></term>
##       <term><![CDATA[Voltage control]]></term>
##     </thesaurusterms>
##     <pubtitle><![CDATA[Microwave Theory and Techniques, IEEE Transactions on]]></pubtitle>
##     <punumber><![CDATA[22]]></punumber>
##     <pubtype><![CDATA[Journals & Magazines]]></pubtype>
##     <publisher><![CDATA[IEEE]]></publisher>
##     <volume><![CDATA[62]]></volume>
##     <issue><![CDATA[11]]></issue>
##     <py><![CDATA[2014]]></py>
##     <spage><![CDATA[2711]]></spage>
##     <epage><![CDATA[2722]]></epage>
##     <abstract><![CDATA[In this study, we propose an auxiliary power regulator (APR) for the current-shared cascade (CSC) structure of the driver stages of RF CMOS power. Although the CSC structure provides a method to reduce the power consumption at the driver stages of a differential power amplifier (PA), it is difficult to obtain optimum levels of effective supply voltage for the first and second driver stages. Additionally, the transistor sizes of the first and second driver stages of the CSC structure must be identical to ensure the proper operation of the PA. Thus, in this work, we propose an APR structure that ensures the proper operation of a PA with different transistor sizes in its first and second driver stages. To prove the feasibility of the proposed technique, we designed the PA with a CSC structure using an APR. From the measured results, we successfully verify the feasibility of the proposed structure. Additionally, we provide experimental results for a typical CMOS PA to determine the optimum supply voltage for the second driver stage.]]></abstract>
##     <issn><![CDATA[0018-9480]]></issn>
##     <htmlFlag><![CDATA[1]]></htmlFlag>
##     <arnumber><![CDATA[6908029]]></arnumber>
##     <doi><![CDATA[10.1109/TMTT.2014.2356974]]></doi>
##     <publicationId><![CDATA[6908029]]></publicationId>
##     <mdurl><![CDATA[http://ieeexplore.ieee.org/xpl/articleDetails.jsp?tp=&arnumber=6908029&contentType=Journals+%26+Magazines]]></mdurl>
##     <pdf><![CDATA[http://ieeexplore.ieee.org/stamp/stamp.jsp?arnumber=6908029]]></pdf>
##   </document>
##   <document>
##     <rank>900</rank>
##     <title><![CDATA[Individual Decision Making Can Drive Epidemics: A Fuzzy Cognitive Map Study]]></title>
##     <authors><![CDATA[Shan Mei;  Yifan Zhu;  Xiaogang Qiu;  Xuan Zhou;  Zhenghu Zu;  Boukhanovsky, A.V.;  Sloot, P.M.A.]]></authors>
##     <affiliations><![CDATA[Inst. of Simulation Eng., Nat. Univ. of Defense Technol., Changsha, China]]></affiliations>
##     <controlledterms>
##       <term><![CDATA[decision making]]></term>
##       <term><![CDATA[diseases]]></term>
##       <term><![CDATA[epidemics]]></term>
##       <term><![CDATA[fuzzy systems]]></term>
##       <term><![CDATA[matrix algebra]]></term>
##     </controlledterms>
##     <thesaurusterms>
##       <term><![CDATA[Cognition]]></term>
##       <term><![CDATA[Computational modeling]]></term>
##       <term><![CDATA[Decision making]]></term>
##       <term><![CDATA[Diseases]]></term>
##       <term><![CDATA[Humans]]></term>
##       <term><![CDATA[Mood]]></term>
##     </thesaurusterms>
##     <pubtitle><![CDATA[Fuzzy Systems, IEEE Transactions on]]></pubtitle>
##     <punumber><![CDATA[91]]></punumber>
##     <pubtype><![CDATA[Journals & Magazines]]></pubtype>
##     <publisher><![CDATA[IEEE]]></publisher>
##     <volume><![CDATA[22]]></volume>
##     <issue><![CDATA[2]]></issue>
##     <py><![CDATA[2014]]></py>
##     <spage><![CDATA[264]]></spage>
##     <epage><![CDATA[273]]></epage>
##     <abstract><![CDATA[Existing studies on the propagation of infectious diseases have not sufficiently considered the uncertainties that are related to individual behavior and its influence on individual decision making to prevent infections, even though it is well known that changes in behavior can lead to variations in the macrodynamics of the spread of infectious diseases. These influencing factors can be categorized into emotion-related and cognition-related components. We present a fuzzy cognitive map (FCM) denotative model to describe how the factors of individual emotions and cognition influence each other. We adjust the weight matrix of causal relationships between these factors by using a so-called nonlinear Hebbian learning method. Based on this FCM model, we can implement individual decision rules against possible infections for disease propagation studies. We take the simulation of influenza A [H1N1] spreading on a campus as an example. We find that individual decision making against infections (frequent washing, respirator usage, and crowd contact avoidance) can significantly decrease the at-peak number of infected patients, even when common policies, such as isolation and vaccination, are not deployed.]]></abstract>
##     <issn><![CDATA[1063-6706]]></issn>
##     <htmlFlag><![CDATA[1]]></htmlFlag>
##     <arnumber><![CDATA[6475999]]></arnumber>
##     <doi><![CDATA[10.1109/TFUZZ.2013.2251638]]></doi>
##     <publicationId><![CDATA[6475999]]></publicationId>
##     <mdurl><![CDATA[http://ieeexplore.ieee.org/xpl/articleDetails.jsp?tp=&arnumber=6475999&contentType=Journals+%26+Magazines]]></mdurl>
##     <pdf><![CDATA[http://ieeexplore.ieee.org/stamp/stamp.jsp?arnumber=6475999]]></pdf>
##   </document>
##   <document>
##     <rank>901</rank>
##     <title><![CDATA[Hybridly Polarized Nanofocusing of Metallic Helical Nanocone]]></title>
##     <authors><![CDATA[Dengfeng Kuang;  Delun Zhang;  Sheng Ouyang]]></authors>
##     <affiliations><![CDATA[Key Lab. of Opt. Inf. Sci. & Technol. of the Educ. Minist. of China, Nankai Univ., Tianjin, China]]></affiliations>
##     <controlledterms>
##       <term><![CDATA[finite difference time-domain analysis]]></term>
##       <term><![CDATA[nanophotonics]]></term>
##       <term><![CDATA[nanostructured materials]]></term>
##       <term><![CDATA[optical focusing]]></term>
##       <term><![CDATA[polaritons]]></term>
##       <term><![CDATA[surface plasmons]]></term>
##     </controlledterms>
##     <thesaurusterms>
##       <term><![CDATA[Electric fields]]></term>
##       <term><![CDATA[Finite difference methods]]></term>
##       <term><![CDATA[Nanostructures]]></term>
##       <term><![CDATA[Optical polarization]]></term>
##       <term><![CDATA[Optical sensors]]></term>
##       <term><![CDATA[Plasmons]]></term>
##       <term><![CDATA[Silver]]></term>
##     </thesaurusterms>
##     <pubtitle><![CDATA[Photonics Journal, IEEE]]></pubtitle>
##     <punumber><![CDATA[4563994]]></punumber>
##     <pubtype><![CDATA[Journals & Magazines]]></pubtype>
##     <publisher><![CDATA[IEEE]]></publisher>
##     <volume><![CDATA[6]]></volume>
##     <issue><![CDATA[2]]></issue>
##     <py><![CDATA[2014]]></py>
##     <spage><![CDATA[1]]></spage>
##     <epage><![CDATA[9]]></epage>
##     <abstract><![CDATA[We present a metallic helical nanocone, which is a combination of a helical structure and a conic structure. Surface plasmon poloritons are excited at the boundary of the bottom of the helical nanocone, then guided and rotated by the groove provide by the helical structure on the nanocone, and finally, squeezed to the apex of the tip to form a highly localized strong electromagnetic field. The intensity distributions and polarization of the electric fields generated by the metallic helical nanocone are simulated with 3-D finite-difference time-domain method, which indicate that highly confined and hybridly polarized optical fields can be produced at the apex of the metallic helical nanocone.]]></abstract>
##     <issn><![CDATA[1943-0655]]></issn>
##     <htmlFlag><![CDATA[1]]></htmlFlag>
##     <arnumber><![CDATA[6744628]]></arnumber>
##     <doi><![CDATA[10.1109/JPHOT.2014.2306822]]></doi>
##     <publicationId><![CDATA[6744628]]></publicationId>
##     <mdurl><![CDATA[http://ieeexplore.ieee.org/xpl/articleDetails.jsp?tp=&arnumber=6744628&contentType=Journals+%26+Magazines]]></mdurl>
##     <pdf><![CDATA[http://ieeexplore.ieee.org/stamp/stamp.jsp?arnumber=6744628]]></pdf>
##   </document>
##   <document>
##     <rank>902</rank>
##     <title><![CDATA[Device-to-Device Communications for National Security and Public Safety]]></title>
##     <authors><![CDATA[Fodor, G.;  Parkvall, S.;  Sorrentino, S.;  Wallentin, P.;  Qianxi Lu;  Brahmi, N.]]></authors>
##     <affiliations><![CDATA[Autom. Control Lab., KTH R. Inst. of Technol., Stockholm, Sweden]]></affiliations>
##     <controlledterms>
##       <term><![CDATA[Long Term Evolution]]></term>
##       <term><![CDATA[radio equipment]]></term>
##     </controlledterms>
##     <thesaurusterms>
##       <term><![CDATA[5G mobile communication]]></term>
##       <term><![CDATA[Broadband communication]]></term>
##       <term><![CDATA[Disasters]]></term>
##       <term><![CDATA[Emergency services]]></term>
##       <term><![CDATA[Long Term Evolution]]></term>
##       <term><![CDATA[Mobile communication]]></term>
##       <term><![CDATA[National security]]></term>
##       <term><![CDATA[Safety]]></term>
##     </thesaurusterms>
##     <pubtitle><![CDATA[Access, IEEE]]></pubtitle>
##     <punumber><![CDATA[6287639]]></punumber>
##     <pubtype><![CDATA[Journals & Magazines]]></pubtype>
##     <publisher><![CDATA[IEEE]]></publisher>
##     <volume><![CDATA[2]]></volume>
##     <py><![CDATA[2014]]></py>
##     <spage><![CDATA[1510]]></spage>
##     <epage><![CDATA[1520]]></epage>
##     <abstract><![CDATA[Device-to-device (D2D) communications have been proposed as an underlay to long-term evolution (LTE) networks as a means of harvesting the proximity, reuse, and hop gains. However, D2D communications can also serve as a technology component for providing public protection and disaster relief (PPDR) and national security and public safety (NSPS) services. In the United States, for example, spectrum has been reserved in the 700-MHz band for an LTE-based public safety network. The key requirement for the evolving broadband PPDR and NSPS services capable systems is to provide access to cellular services when the infrastructure is available and to efficiently support local services even if a subset or all of the network nodes become dysfunctional due to public disaster or emergency situations. This paper reviews some of the key requirements, technology challenges, and solution approaches that must be in place in order to enable LTE networks and, in particular, D2D communications, to meet PPDR and NSPS-related requirements. In particular, we propose a clustering-procedure-based approach to the design of a system that integrates cellular and ad hoc operation modes depending on the availability of infrastructure nodes. System simulations demonstrate the viability of the proposed design. The proposed scheme is currently considered as a technology component of the evolving 5G concept developed by the European 5G research project METIS.]]></abstract>
##     <issn><![CDATA[2169-3536]]></issn>
##     <htmlFlag><![CDATA[1]]></htmlFlag>
##     <arnumber><![CDATA[6985517]]></arnumber>
##     <doi><![CDATA[10.1109/ACCESS.2014.2379938]]></doi>
##     <publicationId><![CDATA[6985517]]></publicationId>
##     <mdurl><![CDATA[http://ieeexplore.ieee.org/xpl/articleDetails.jsp?tp=&arnumber=6985517&contentType=Journals+%26+Magazines]]></mdurl>
##     <pdf><![CDATA[http://ieeexplore.ieee.org/stamp/stamp.jsp?arnumber=6985517]]></pdf>
##   </document>
##   <document>
##     <rank>903</rank>
##     <title><![CDATA[Control for Safety Specifications of Systems With Imperfect Information on a Partial Order]]></title>
##     <authors><![CDATA[Ghaemi, R.;  Del Vecchio, D.]]></authors>
##     <affiliations><![CDATA[Gen. Electr. Global Res., Schenectady, NY, USA]]></affiliations>
##     <controlledterms>
##       <term><![CDATA[closed loop systems]]></term>
##       <term><![CDATA[open loop systems]]></term>
##       <term><![CDATA[state-space methods]]></term>
##       <term><![CDATA[uncertain systems]]></term>
##     </controlledterms>
##     <thesaurusterms>
##       <term><![CDATA[Aerospace electronics]]></term>
##       <term><![CDATA[Approximation methods]]></term>
##       <term><![CDATA[Atmospheric measurements]]></term>
##       <term><![CDATA[Open loop systems]]></term>
##       <term><![CDATA[Particle measurements]]></term>
##       <term><![CDATA[Safety]]></term>
##       <term><![CDATA[Trajectory]]></term>
##     </thesaurusterms>
##     <pubtitle><![CDATA[Automatic Control, IEEE Transactions on]]></pubtitle>
##     <punumber><![CDATA[9]]></punumber>
##     <pubtype><![CDATA[Journals & Magazines]]></pubtype>
##     <publisher><![CDATA[IEEE]]></publisher>
##     <volume><![CDATA[59]]></volume>
##     <issue><![CDATA[4]]></issue>
##     <py><![CDATA[2014]]></py>
##     <spage><![CDATA[982]]></spage>
##     <epage><![CDATA[995]]></epage>
##     <abstract><![CDATA[In this paper, we consider the control problem for uncertain systems with imperfect information, in which an output of interest must be kept outside an undesired region (the bad set) in the output space. The state, input, output, and disturbance spaces are equipped with partial orders. The system dynamics are either input/output order preserving with output in R<sup>2</sup> or given by the parallel composition of input/output order preserving dynamics each with scalar output. We provide necessary and sufficient conditions under which an initial set of possible system states is safe, that is, the corresponding outputs are steerable away from the bad set with open loop controls. A closed loop control strategy is explicitly constructed, which guarantees that the current set of possible system states, as obtained from an estimator, generates outputs that never enter the bad set. The complexity of algorithms that check safety of an initial set of states and implement the control map is quadratic with the dimension of the state space. The algorithms are illustrated on two application examples: a ship maneuver to avoid an obstacle and safe navigation of an helicopter among buildings.]]></abstract>
##     <issn><![CDATA[0018-9286]]></issn>
##     <htmlFlag><![CDATA[1]]></htmlFlag>
##     <arnumber><![CDATA[6716989]]></arnumber>
##     <doi><![CDATA[10.1109/TAC.2014.2301563]]></doi>
##     <publicationId><![CDATA[6716989]]></publicationId>
##     <mdurl><![CDATA[http://ieeexplore.ieee.org/xpl/articleDetails.jsp?tp=&arnumber=6716989&contentType=Journals+%26+Magazines]]></mdurl>
##     <pdf><![CDATA[http://ieeexplore.ieee.org/stamp/stamp.jsp?arnumber=6716989]]></pdf>
##   </document>
##   <document>
##     <rank>904</rank>
##     <title><![CDATA[Phase-Stabilized Delivery for Multiple Local Oscillator Signals via Optical Fiber]]></title>
##     <authors><![CDATA[Anxu Zhang;  Yitang Dai;  Feifei Yin;  Tianpeng Ren;  Kun Xu;  Jianqiang Li;  Jintong Lin;  Geshi Tang]]></authors>
##     <affiliations><![CDATA[State Key Lab. of Inf. Photonics & Opt. Commun., Beijing Univ. of Posts & Telecommun., Beijing, China]]></affiliations>
##     <controlledterms>
##       <term><![CDATA[delays]]></term>
##       <term><![CDATA[microwave photonics]]></term>
##       <term><![CDATA[optical fibre communication]]></term>
##       <term><![CDATA[optical links]]></term>
##       <term><![CDATA[oscillators]]></term>
##       <term><![CDATA[stability]]></term>
##     </controlledterms>
##     <thesaurusterms>
##       <term><![CDATA[Delays]]></term>
##       <term><![CDATA[Fluctuations]]></term>
##       <term><![CDATA[Optical fiber communication]]></term>
##       <term><![CDATA[Optical fiber dispersion]]></term>
##       <term><![CDATA[Optical fibers]]></term>
##       <term><![CDATA[Radio frequency]]></term>
##     </thesaurusterms>
##     <pubtitle><![CDATA[Photonics Journal, IEEE]]></pubtitle>
##     <punumber><![CDATA[4563994]]></punumber>
##     <pubtype><![CDATA[Journals & Magazines]]></pubtype>
##     <publisher><![CDATA[IEEE]]></publisher>
##     <volume><![CDATA[6]]></volume>
##     <issue><![CDATA[3]]></issue>
##     <py><![CDATA[2014]]></py>
##     <spage><![CDATA[1]]></spage>
##     <epage><![CDATA[8]]></epage>
##     <abstract><![CDATA[In this paper, we propose and demonstrate a multiple local oscillator (LO) signal phase stabilization technique for long-distance fiber delivery. One of the LO signals, which acts as a reference single, is round-trip transferred between the central station and the remote end, carrying the phase variation arises from perturbations of the fiber link. The wavelength of the optical carrier is adjusted according to the phase variation of the reference LO, to stabilize the delay of the fiber link. Once the delay of the link is stabilized, the phases of other LO signals that transferred through the same fiber link will all be stabilized. Meanwhile, the delay tunable range is in proportion to the fiber length, which means a very long delivery distance can be expected. Experimentally, LOs at frequencies of 2.46 GHz and 8 GHz are transferred through a 30-km fiber link, and significant phase drift compression is observed at both frequencies.]]></abstract>
##     <issn><![CDATA[1943-0655]]></issn>
##     <htmlFlag><![CDATA[1]]></htmlFlag>
##     <arnumber><![CDATA[6814838]]></arnumber>
##     <doi><![CDATA[10.1109/JPHOT.2014.2323298]]></doi>
##     <publicationId><![CDATA[6814838]]></publicationId>
##     <mdurl><![CDATA[http://ieeexplore.ieee.org/xpl/articleDetails.jsp?tp=&arnumber=6814838&contentType=Journals+%26+Magazines]]></mdurl>
##     <pdf><![CDATA[http://ieeexplore.ieee.org/stamp/stamp.jsp?arnumber=6814838]]></pdf>
##   </document>
##   <document>
##     <rank>905</rank>
##     <title><![CDATA[Improved VIIRS Day/Night Band Imagery With Near-Constant Contrast]]></title>
##     <authors><![CDATA[Liang, C.K.;  Mills, S.;  Hauss, B.I.;  Miller, S.D.]]></authors>
##     <affiliations><![CDATA[Northrop Grumman Corp., Redondo Beach, CA, USA]]></affiliations>
##     <controlledterms>
##       <term><![CDATA[airglow]]></term>
##       <term><![CDATA[clouds]]></term>
##       <term><![CDATA[infrared imaging]]></term>
##       <term><![CDATA[optical variables measurement]]></term>
##       <term><![CDATA[quantisation (signal)]]></term>
##       <term><![CDATA[radiometers]]></term>
##       <term><![CDATA[stray light]]></term>
##     </controlledterms>
##     <thesaurusterms>
##       <term><![CDATA[Clouds]]></term>
##       <term><![CDATA[Dynamic range]]></term>
##       <term><![CDATA[Lighting]]></term>
##       <term><![CDATA[Moon]]></term>
##       <term><![CDATA[Signal processing algorithms]]></term>
##       <term><![CDATA[Stray light]]></term>
##       <term><![CDATA[Table lookup]]></term>
##     </thesaurusterms>
##     <pubtitle><![CDATA[Geoscience and Remote Sensing, IEEE Transactions on]]></pubtitle>
##     <punumber><![CDATA[36]]></punumber>
##     <pubtype><![CDATA[Journals & Magazines]]></pubtype>
##     <publisher><![CDATA[IEEE]]></publisher>
##     <volume><![CDATA[52]]></volume>
##     <issue><![CDATA[11]]></issue>
##     <py><![CDATA[2014]]></py>
##     <spage><![CDATA[6964]]></spage>
##     <epage><![CDATA[6971]]></epage>
##     <abstract><![CDATA[The Suomi-NPP Visible Infrared Imager Radiometer Suite (VIIRS) instrument provides the next generation of visible/infrared imaging including the day/night band (DNB) with nominal bandwidth from 500 to 900 nm. Previous to VIIRS, the Defense Meteorological Satellite Program Operational Linescan System (OLS) measured radiances that spanned over seven orders of magnitude, using an onboard gain adjustment to provide the capability to image atmospheric features across the solar terminator, to observe nighttime light emissions over the globe, and to monitor the global distribution of clouds. The VIIRS DNB detects radiances that span over eight orders of magnitude, and because it has 13-14-b quantization (compared with 6 b for OLS) with three gain stages, the DNB has its full dynamic range at every part of the scan. One process that is applied to the VIIRS DNB radiances is a solar/lunar zenith angle dependent gain adjustment to create near-constant contrast (NCC) imagery. The at-launch NCC algorithm was designed to reproduce the OLS capability and, thus, was constrained to solar and lunar angles from 0&#x00B0; to 105&#x00B0;. This limitation has, in part, lead to suboptimal imagery due to the assumption that DNB radiances fall off exponentially beyond twilight. The VIIRS DNB ultrasensitivity in low-light conditions enables it to detect faint emissions from a phenomenon called airglow, thus invalidating the exponential fall-off assumption. Another complication to the NCC imagery algorithm is the stray light contamination that contaminates the DNB radiances in the astronomical twilight region. We address these issues and develop a solution that leads to high-quality imagery for all solar and lunar conditions.]]></abstract>
##     <issn><![CDATA[0196-2892]]></issn>
##     <htmlFlag><![CDATA[1]]></htmlFlag>
##     <arnumber><![CDATA[6762963]]></arnumber>
##     <doi><![CDATA[10.1109/TGRS.2014.2306132]]></doi>
##     <publicationId><![CDATA[6762963]]></publicationId>
##     <mdurl><![CDATA[http://ieeexplore.ieee.org/xpl/articleDetails.jsp?tp=&arnumber=6762963&contentType=Journals+%26+Magazines]]></mdurl>
##     <pdf><![CDATA[http://ieeexplore.ieee.org/stamp/stamp.jsp?arnumber=6762963]]></pdf>
##   </document>
##   <document>
##     <rank>906</rank>
##     <title><![CDATA[Power Consumption of Integrated Low-Power Receivers]]></title>
##     <authors><![CDATA[Nilsson, E.;  Svensson, C.]]></authors>
##     <affiliations><![CDATA[Lab. of Math., Phys., & Electr. Eng., Halmstad Univ., Halmstad, Sweden]]></affiliations>
##     <controlledterms>
##       <term><![CDATA[CMOS integrated circuits]]></term>
##       <term><![CDATA[Internet of Things]]></term>
##       <term><![CDATA[optimisation]]></term>
##       <term><![CDATA[power consumption]]></term>
##       <term><![CDATA[radio receivers]]></term>
##     </controlledterms>
##     <thesaurusterms>
##       <term><![CDATA[Bandwidth]]></term>
##       <term><![CDATA[Baseband]]></term>
##       <term><![CDATA[Detectors]]></term>
##       <term><![CDATA[Noise]]></term>
##       <term><![CDATA[Radio frequency]]></term>
##       <term><![CDATA[Receivers]]></term>
##       <term><![CDATA[Sensitivity]]></term>
##     </thesaurusterms>
##     <pubtitle><![CDATA[Emerging and Selected Topics in Circuits and Systems, IEEE Journal on]]></pubtitle>
##     <punumber><![CDATA[5503868]]></punumber>
##     <pubtype><![CDATA[Journals & Magazines]]></pubtype>
##     <publisher><![CDATA[IEEE]]></publisher>
##     <volume><![CDATA[4]]></volume>
##     <issue><![CDATA[3]]></issue>
##     <py><![CDATA[2014]]></py>
##     <spage><![CDATA[273]]></spage>
##     <epage><![CDATA[283]]></epage>
##     <abstract><![CDATA[With the advent of Internet of Things (IoT) it has become clear that radio-frequency (RF) designers have to be aware of power constraints, e.g., in the design of simplistic ultra-low power receivers often used as wake-up radios (WuRs). The objective of this work, one of the first systematic studies of power bounds for RF-systems, is to provide an overview and intuitive feel for how power consumption and sensitivity relates for low-power receivers. This was done by setting up basic circuit schematics for different radio receiver architectures to find analytical expressions for their output signal-to-noise ratio including power consumption, bandwidth, sensitivity, and carrier frequency. The analytical expressions and optimizations of the circuits give us relations between dc-energy-per-bit and receiver sensitivity, which can be compared to recent published low-power receivers. The parameter set used in the analysis is meant to reflect typical values for an integrated 90 nm complementary metal-oxide-semiconductor fabrication processes, and typical small sized RF lumped components.]]></abstract>
##     <issn><![CDATA[2156-3357]]></issn>
##     <htmlFlag><![CDATA[1]]></htmlFlag>
##     <arnumber><![CDATA[6866244]]></arnumber>
##     <doi><![CDATA[10.1109/JETCAS.2014.2337151]]></doi>
##     <publicationId><![CDATA[6866244]]></publicationId>
##     <mdurl><![CDATA[http://ieeexplore.ieee.org/xpl/articleDetails.jsp?tp=&arnumber=6866244&contentType=Journals+%26+Magazines]]></mdurl>
##     <pdf><![CDATA[http://ieeexplore.ieee.org/stamp/stamp.jsp?arnumber=6866244]]></pdf>
##   </document>
##   <document>
##     <rank>907</rank>
##     <title><![CDATA[State-of-the-Art Database of Terahertz Spectroscopy Based on Modern Web Technology]]></title>
##     <authors><![CDATA[Notake, T.;  Endo, R.;  Fukunaga, K.;  Hosako, I.;  Otani, C.;  Minamide, H.]]></authors>
##     <affiliations><![CDATA[Center for Adv. Ohotonics, RIKEN, Sendai, Japan]]></affiliations>
##     <controlledterms>
##       <term><![CDATA[Internet]]></term>
##       <term><![CDATA[electrical engineering computing]]></term>
##       <term><![CDATA[hypermedia markup languages]]></term>
##       <term><![CDATA[information services]]></term>
##       <term><![CDATA[physics computing]]></term>
##       <term><![CDATA[terahertz spectroscopy]]></term>
##     </controlledterms>
##     <thesaurusterms>
##       <term><![CDATA[Browsers]]></term>
##       <term><![CDATA[Databases]]></term>
##       <term><![CDATA[Educational institutions]]></term>
##       <term><![CDATA[Electronic mail]]></term>
##       <term><![CDATA[Internet]]></term>
##       <term><![CDATA[Spectroscopy]]></term>
##     </thesaurusterms>
##     <pubtitle><![CDATA[Terahertz Science and Technology, IEEE Transactions on]]></pubtitle>
##     <punumber><![CDATA[5503871]]></punumber>
##     <pubtype><![CDATA[Journals & Magazines]]></pubtype>
##     <publisher><![CDATA[IEEE]]></publisher>
##     <volume><![CDATA[4]]></volume>
##     <issue><![CDATA[1]]></issue>
##     <py><![CDATA[2014]]></py>
##     <spage><![CDATA[110]]></spage>
##     <epage><![CDATA[115]]></epage>
##     <abstract><![CDATA[We present a review of a pioneering database on terahertz spectroscopy established by the National Institute of Information and Communication Technology and RIKEN. The database is universally available on the World Wide Web via the Internet since 2008 and was recently restructured on the basis of contemporary HTML5 technology. Flexibility and convenience to browse spectroscopic data are much improved. In the database, no less than 1500 spectroscopic data have been consolidated and provided for various materials under different conditions. Furthermore, a data upload system from general users has been prepared to enrich the database further.]]></abstract>
##     <issn><![CDATA[2156-342X]]></issn>
##     <htmlFlag><![CDATA[1]]></htmlFlag>
##     <arnumber><![CDATA[6654317]]></arnumber>
##     <doi><![CDATA[10.1109/TTHZ.2013.2284862]]></doi>
##     <publicationId><![CDATA[6654317]]></publicationId>
##     <mdurl><![CDATA[http://ieeexplore.ieee.org/xpl/articleDetails.jsp?tp=&arnumber=6654317&contentType=Journals+%26+Magazines]]></mdurl>
##     <pdf><![CDATA[http://ieeexplore.ieee.org/stamp/stamp.jsp?arnumber=6654317]]></pdf>
##   </document>
##   <document>
##     <rank>908</rank>
##     <title><![CDATA[Characterization of Optical Coherence Tomography Images Acquired at Large Distances With Large-Diameter Beams]]></title>
##     <authors><![CDATA[Popescu, D.P.;  Smith, M.S.D.;  Sowa, M.G.]]></authors>
##     <affiliations><![CDATA[Med. Devices Portfolio, Nat. Res. Council of Canada, Winnipeg, MB, Canada]]></affiliations>
##     <controlledterms>
##       <term><![CDATA[image processing]]></term>
##       <term><![CDATA[optical collimators]]></term>
##       <term><![CDATA[optical fibres]]></term>
##       <term><![CDATA[optical tomography]]></term>
##     </controlledterms>
##     <thesaurusterms>
##       <term><![CDATA[Biomedical optical imaging]]></term>
##       <term><![CDATA[Coherence]]></term>
##       <term><![CDATA[Collimators]]></term>
##       <term><![CDATA[Optical fibers]]></term>
##       <term><![CDATA[Optical imaging]]></term>
##       <term><![CDATA[Optical variables measurement]]></term>
##     </thesaurusterms>
##     <pubtitle><![CDATA[Photonics Journal, IEEE]]></pubtitle>
##     <punumber><![CDATA[4563994]]></punumber>
##     <pubtype><![CDATA[Journals & Magazines]]></pubtype>
##     <publisher><![CDATA[IEEE]]></publisher>
##     <volume><![CDATA[6]]></volume>
##     <issue><![CDATA[3]]></issue>
##     <py><![CDATA[2014]]></py>
##     <spage><![CDATA[1]]></spage>
##     <epage><![CDATA[11]]></epage>
##     <abstract><![CDATA[We tested the imaging capabilities for variants of a 1550-nm swept-source fiber-based optical coherence tomography system with a telecentric system incorporated at the end of its sample arm. The system was designed for in vivo imaging of burns; therefore, we acquired images from samples located at distances greater than 24 cm from the exit of the telecentric system. Each system variation had a specific combination of diameters for the reference and sample beams. In the reference arm, we used, alternately, two collimated beams with diameters of 1.5 and 14 mm, respectively. In the sample arm, we tested collimated beams with the following diameters: 1.5, 3.5, 5.7, 8.4, and 14 mm. A galvanometric mirror system scanned the collimated sample beam across the entrance pupil of the telecentric system. The sample beam exited the telecentric system parallel with its optical axis and convergent onto the sample. Depending on the collimator used in the sample arm, images were acquired with beams focused to waist diameters ranging from 40 to 240 &#x03BC;m. We acquired images with the sample at different locations within a &#x00B1;30 mm range centered about the sample beam waist. Furthermore, we used the signal-to-noise ratio, the detected signal intensity, and the visual appearance to compare images acquired with different sample/reference beam configurations.]]></abstract>
##     <issn><![CDATA[1943-0655]]></issn>
##     <htmlFlag><![CDATA[1]]></htmlFlag>
##     <arnumber><![CDATA[6815693]]></arnumber>
##     <doi><![CDATA[10.1109/JPHOT.2014.2321754]]></doi>
##     <publicationId><![CDATA[6815693]]></publicationId>
##     <mdurl><![CDATA[http://ieeexplore.ieee.org/xpl/articleDetails.jsp?tp=&arnumber=6815693&contentType=Journals+%26+Magazines]]></mdurl>
##     <pdf><![CDATA[http://ieeexplore.ieee.org/stamp/stamp.jsp?arnumber=6815693]]></pdf>
##   </document>
##   <document>
##     <rank>909</rank>
##     <title><![CDATA[Understanding the Nature of Social Mobile Instant Messaging in Cellular Networks]]></title>
##     <authors><![CDATA[Xuan Zhou;  Zhifeng Zhao;  Rongpeng Li;  Yifan Zhou;  Palicot, J.;  Honggang Zhang]]></authors>
##     <affiliations><![CDATA[Zhejiang Univ., Hangzhou, China]]></affiliations>
##     <controlledterms>
##       <term><![CDATA[3G mobile communication]]></term>
##       <term><![CDATA[cellular radio]]></term>
##       <term><![CDATA[computer network performance evaluation]]></term>
##       <term><![CDATA[electronic messaging]]></term>
##       <term><![CDATA[mobile computing]]></term>
##       <term><![CDATA[social networking (online)]]></term>
##       <term><![CDATA[telecommunication traffic]]></term>
##     </controlledterms>
##     <thesaurusterms>
##       <term><![CDATA[Instant messaging]]></term>
##       <term><![CDATA[Joints]]></term>
##       <term><![CDATA[Mobile communication]]></term>
##       <term><![CDATA[Mobile computing]]></term>
##       <term><![CDATA[Servers]]></term>
##       <term><![CDATA[Smart phones]]></term>
##     </thesaurusterms>
##     <pubtitle><![CDATA[Communications Letters, IEEE]]></pubtitle>
##     <punumber><![CDATA[4234]]></punumber>
##     <pubtype><![CDATA[Journals & Magazines]]></pubtype>
##     <publisher><![CDATA[IEEE]]></publisher>
##     <volume><![CDATA[18]]></volume>
##     <issue><![CDATA[3]]></issue>
##     <py><![CDATA[2014]]></py>
##     <spage><![CDATA[389]]></spage>
##     <epage><![CDATA[392]]></epage>
##     <abstract><![CDATA[Social mobile instant messaging (MIM) applications, running on portable devices such as smartphones and tablets, have become increasingly popular around the world, and have generated significant traffic demands on cellular networks. Despite its huge user base and popularity, little work has been done to characterize the traffic patterns of MIM. Compared with traditional cellular network service, MIM traffic embodies several specific attributes such as non-Poisson arrivals, keep-alive (KA) mechanism and heavy-tailed message length, which consumes even small amount of core network bandwidth but considerable radio resources of mobile access network. This letter investigates user behavior patterns and traffic characteristics of MIM applications, based on real traffic measurements within a large-scale cellular network covering 7 million subscribers. Moreover, we propose a joint ON/OFF model to describe the traffic characteristics of MIM, and evaluate the performance of cellular network running MIM service in various scenarios. Comparing with the MIM service models of 3GPP, our results are more realistic to estimate the networks performance.]]></abstract>
##     <issn><![CDATA[1089-7798]]></issn>
##     <htmlFlag><![CDATA[1]]></htmlFlag>
##     <arnumber><![CDATA[6784528]]></arnumber>
##     <doi><![CDATA[10.1109/LCOMM.2014.012014.132592]]></doi>
##     <publicationId><![CDATA[6784528]]></publicationId>
##     <mdurl><![CDATA[http://ieeexplore.ieee.org/xpl/articleDetails.jsp?tp=&arnumber=6784528&contentType=Journals+%26+Magazines]]></mdurl>
##     <pdf><![CDATA[http://ieeexplore.ieee.org/stamp/stamp.jsp?arnumber=6784528]]></pdf>
##   </document>
##   <document>
##     <rank>910</rank>
##     <title><![CDATA[On-Line Nonintrusive Detection of Wood Pellets in Pneumatic Conveying Pipelines Using Vibration and Acoustic Sensors]]></title>
##     <authors><![CDATA[Duo Sun;  Yong Yan;  Carter, R.M.;  Lingjun Gao;  Gang Lu;  Riley, G.;  Wood, M.]]></authors>
##     <affiliations><![CDATA[Instrum., Control & Embedded Syst. Res. Group, Univ. of Kent, Canterbury, UK]]></affiliations>
##     <controlledterms>
##       <term><![CDATA[acoustic generators]]></term>
##       <term><![CDATA[acoustic transducers]]></term>
##       <term><![CDATA[conveyors]]></term>
##       <term><![CDATA[noise (working environment)]]></term>
##       <term><![CDATA[noise abatement]]></term>
##       <term><![CDATA[pipelines]]></term>
##       <term><![CDATA[pneumatic systems]]></term>
##       <term><![CDATA[renewable materials]]></term>
##       <term><![CDATA[time-frequency analysis]]></term>
##       <term><![CDATA[vibration measurement]]></term>
##       <term><![CDATA[wood]]></term>
##     </controlledterms>
##     <thesaurusterms>
##       <term><![CDATA[Acoustic sensors]]></term>
##       <term><![CDATA[Noise reduction]]></term>
##       <term><![CDATA[Pipelines]]></term>
##       <term><![CDATA[Vibrations]]></term>
##       <term><![CDATA[Wavelet transforms]]></term>
##     </thesaurusterms>
##     <pubtitle><![CDATA[Instrumentation and Measurement, IEEE Transactions on]]></pubtitle>
##     <punumber><![CDATA[19]]></punumber>
##     <pubtype><![CDATA[Journals & Magazines]]></pubtype>
##     <publisher><![CDATA[IEEE]]></publisher>
##     <volume><![CDATA[63]]></volume>
##     <issue><![CDATA[5]]></issue>
##     <py><![CDATA[2014]]></py>
##     <spage><![CDATA[993]]></spage>
##     <epage><![CDATA[1001]]></epage>
##     <abstract><![CDATA[This paper presents a novel instrumentation system for on-line nonintrusive detection of wood pellets in pneumatic conveying pipelines using vibration and acoustic sensors. The system captures the vibration and sound generated by the collisions between biomass particles and the pipe wall. Time-frequency analysis technique is used to eliminate environmental noise from the signal, extract information about the collisions, and identify the presence of wood pellets. Experiments were carried out on an industrial pneumatic conveying pipeline to assess effectiveness and operability. The impacts of various factors on the performance of the detection system are compared and discussed, including different sensing (vibration sensor versus acoustic sensor), different time-frequency analysis methods (wavelet-based denoising versus classic filtering), and different system installation locations.]]></abstract>
##     <issn><![CDATA[0018-9456]]></issn>
##     <htmlFlag><![CDATA[1]]></htmlFlag>
##     <arnumber><![CDATA[6679243]]></arnumber>
##     <doi><![CDATA[10.1109/TIM.2013.2292284]]></doi>
##     <publicationId><![CDATA[6679243]]></publicationId>
##     <mdurl><![CDATA[http://ieeexplore.ieee.org/xpl/articleDetails.jsp?tp=&arnumber=6679243&contentType=Journals+%26+Magazines]]></mdurl>
##     <pdf><![CDATA[http://ieeexplore.ieee.org/stamp/stamp.jsp?arnumber=6679243]]></pdf>
##   </document>
##   <document>
##     <rank>911</rank>
##     <title><![CDATA[Example of a Complementary Use of Model Checking and Human Performance Simulation]]></title>
##     <authors><![CDATA[Gelman, G.;  Feigh, K.M.;  Rushby, J.]]></authors>
##     <affiliations><![CDATA[Sch. of Aerosp. Eng., Georgia Inst. of Technol., Atlanta, GA, USA]]></affiliations>
##     <controlledterms>
##       <term><![CDATA[aerospace control]]></term>
##       <term><![CDATA[aerospace engineering]]></term>
##       <term><![CDATA[aircraft]]></term>
##       <term><![CDATA[formal verification]]></term>
##       <term><![CDATA[human computer interaction]]></term>
##       <term><![CDATA[human factors]]></term>
##     </controlledterms>
##     <thesaurusterms>
##       <term><![CDATA[Aircraft]]></term>
##       <term><![CDATA[Analytical models]]></term>
##       <term><![CDATA[Atmospheric modeling]]></term>
##       <term><![CDATA[Automation]]></term>
##       <term><![CDATA[Cognitive science]]></term>
##       <term><![CDATA[Mathematical model]]></term>
##       <term><![CDATA[Model checking]]></term>
##     </thesaurusterms>
##     <pubtitle><![CDATA[Human-Machine Systems, IEEE Transactions on]]></pubtitle>
##     <punumber><![CDATA[6221037]]></punumber>
##     <pubtype><![CDATA[Journals & Magazines]]></pubtype>
##     <publisher><![CDATA[IEEE]]></publisher>
##     <volume><![CDATA[44]]></volume>
##     <issue><![CDATA[5]]></issue>
##     <py><![CDATA[2014]]></py>
##     <spage><![CDATA[576]]></spage>
##     <epage><![CDATA[590]]></epage>
##     <abstract><![CDATA[Aircraft automation designers are faced with the challenge to develop and improve automation such that it is transparent to the pilots using it. To identify problems that may arise between pilots and automation, methods are needed that can uncover potential problems with automation early in the design process. In this paper, simulation and model checking are combined and their respective advantages leveraged to find problematic human-automation interaction using methods that would be available early in the design process. A particular problem of interest is automation surprises, which describe events when pilots are surprised by the actions of the automation. The Tarom flight 381 incident involving the former Airbus automatic speed protection logic, leading to an automation surprise, is used as a common case study. Results of this case study indicate that both methods identified the automation surprise found in the Tarom flight 381 incident, and that the simulation identified additional automation surprises associated with that flight logic. The work shows that the methods can be symbiotically combined, and the joint method is suitable to identify problematic human-automation interaction such as automation surprise.]]></abstract>
##     <issn><![CDATA[2168-2291]]></issn>
##     <htmlFlag><![CDATA[1]]></htmlFlag>
##     <arnumber><![CDATA[6861991]]></arnumber>
##     <doi><![CDATA[10.1109/THMS.2014.2331034]]></doi>
##     <publicationId><![CDATA[6861991]]></publicationId>
##     <mdurl><![CDATA[http://ieeexplore.ieee.org/xpl/articleDetails.jsp?tp=&arnumber=6861991&contentType=Journals+%26+Magazines]]></mdurl>
##     <pdf><![CDATA[http://ieeexplore.ieee.org/stamp/stamp.jsp?arnumber=6861991]]></pdf>
##   </document>
##   <document>
##     <rank>912</rank>
##     <title><![CDATA[Low-Frequency Noise and Passive Imaging With 670 GHz HEMT Low-Noise Amplifiers]]></title>
##     <authors><![CDATA[Grossman, E.N.;  Leong, K.;  Xiaobing Mei;  Deal, W.]]></authors>
##     <affiliations><![CDATA[Nat. Inst. of Stand. & Technol., Boulder, CO, USA]]></affiliations>
##     <controlledterms>
##       <term><![CDATA[1/f noise]]></term>
##       <term><![CDATA[III-V semiconductors]]></term>
##       <term><![CDATA[high electron mobility transistors]]></term>
##       <term><![CDATA[indium compounds]]></term>
##       <term><![CDATA[low noise amplifiers]]></term>
##       <term><![CDATA[semiconductor diodes]]></term>
##       <term><![CDATA[wide band gap semiconductors]]></term>
##     </controlledterms>
##     <thesaurusterms>
##       <term><![CDATA[1f noise]]></term>
##       <term><![CDATA[HEMTs]]></term>
##       <term><![CDATA[Imaging]]></term>
##       <term><![CDATA[Indium phosphide]]></term>
##       <term><![CDATA[Radar imaging]]></term>
##       <term><![CDATA[Radiometry]]></term>
##       <term><![CDATA[Submillimeter wave technology]]></term>
##       <term><![CDATA[Temperature measurement]]></term>
##     </thesaurusterms>
##     <pubtitle><![CDATA[Terahertz Science and Technology, IEEE Transactions on]]></pubtitle>
##     <punumber><![CDATA[5503871]]></punumber>
##     <pubtype><![CDATA[Journals & Magazines]]></pubtype>
##     <publisher><![CDATA[IEEE]]></publisher>
##     <volume><![CDATA[4]]></volume>
##     <issue><![CDATA[6]]></issue>
##     <py><![CDATA[2014]]></py>
##     <spage><![CDATA[749]]></spage>
##     <epage><![CDATA[752]]></epage>
##     <abstract><![CDATA[We combine newly developed InP HEMT amplifiers operating at 670 GHz with a zero-bias diode (ZBD) in order to investigate limits on passive imaging performance possible with fully uncooled, direct-detection technology. Noise-equivalent temperature difference (NETD) values under 2 K are found for reference conditions (30 Hz modulation). However, noise spectra continue to fall approximately as 1/f out to &#x201C;knee&#x201D; frequencies of several hundred hertz (Hz) and spectral densities of ~ 0.12 K/Hz<sup>1/2</sup> for the current, somewhat gain-starved, amplifiers. The amplifier alone contributes 0.043 K/Hz<sup>1/2</sup>. These results indicate that modest-sized, rapidly scanned arrays should provide real-time, passive imaging-desired for standoff security screening and other applications-with the same image quality as cryogenic bolometer arrays.]]></abstract>
##     <issn><![CDATA[2156-342X]]></issn>
##     <htmlFlag><![CDATA[1]]></htmlFlag>
##     <arnumber><![CDATA[6891365]]></arnumber>
##     <doi><![CDATA[10.1109/TTHZ.2014.2352035]]></doi>
##     <publicationId><![CDATA[6891365]]></publicationId>
##     <mdurl><![CDATA[http://ieeexplore.ieee.org/xpl/articleDetails.jsp?tp=&arnumber=6891365&contentType=Journals+%26+Magazines]]></mdurl>
##     <pdf><![CDATA[http://ieeexplore.ieee.org/stamp/stamp.jsp?arnumber=6891365]]></pdf>
##   </document>
##   <document>
##     <rank>913</rank>
##     <title><![CDATA[Differential Evolution With Two-Level Parameter Adaptation]]></title>
##     <authors><![CDATA[Wei-Jie Yu;  Meie Shen;  Wei-Neng Chen;  Zhi-Hui Zhan;  Yue-Jiao Gong;  Ying Lin;  Ou Liu;  Jun Zhang]]></authors>
##     <affiliations><![CDATA[Sun Yat-Sen Univ., Guangzhou, China]]></affiliations>
##     <controlledterms>
##       <term><![CDATA[adaptive control]]></term>
##       <term><![CDATA[benchmark testing]]></term>
##       <term><![CDATA[convergence]]></term>
##       <term><![CDATA[evolutionary computation]]></term>
##       <term><![CDATA[greedy algorithms]]></term>
##       <term><![CDATA[optimisation]]></term>
##     </controlledterms>
##     <thesaurusterms>
##       <term><![CDATA[Convergence]]></term>
##       <term><![CDATA[Diversity reception]]></term>
##       <term><![CDATA[Optimization]]></term>
##       <term><![CDATA[Process control]]></term>
##       <term><![CDATA[Sociology]]></term>
##       <term><![CDATA[Statistics]]></term>
##       <term><![CDATA[Vectors]]></term>
##     </thesaurusterms>
##     <pubtitle><![CDATA[Cybernetics, IEEE Transactions on]]></pubtitle>
##     <punumber><![CDATA[6221036]]></punumber>
##     <pubtype><![CDATA[Journals & Magazines]]></pubtype>
##     <publisher><![CDATA[IEEE]]></publisher>
##     <volume><![CDATA[44]]></volume>
##     <issue><![CDATA[7]]></issue>
##     <py><![CDATA[2014]]></py>
##     <spage><![CDATA[1080]]></spage>
##     <epage><![CDATA[1099]]></epage>
##     <abstract><![CDATA[The performance of differential evolution (DE) largely depends on its mutation strategy and control parameters. In this paper, we propose an adaptive DE (ADE) algorithm with a new mutation strategy DE/lbest/1 and a two-level adaptive parameter control scheme. The DE/lbest/1 strategy is a variant of the greedy DE/best/1 strategy. However, the population is mutated under the guide of multiple locally best individuals in DE/lbest/1 instead of one globally best individual in DE/best/1. This strategy is beneficial to the balance between fast convergence and population diversity. The two-level adaptive parameter control scheme is implemented mainly in two steps. In the first step, the population-level parameters F<sub>p</sub> and CR<sub>p</sub> for the whole population are adaptively controlled according to the optimization states, namely, the exploration state and the exploitation state in each generation. These optimization states are estimated by measuring the population distribution. Then, the individual-level parameters F<sub>i</sub> and CR<sub>i</sub> for each individual are generated by adjusting the population-level parameters. The adjustment is based on considering the individual's fitness value and its distance from the globally best individual. This way, the parameters can be adapted to not only the overall state of the population but also the characteristics of different individuals. The performance of the proposed ADE is evaluated on a suite of benchmark functions. Experimental results show that ADE generally outperforms four state-of-the-art DE variants on different kinds of optimization problems. The effects of ADE components, parameter properties of ADE, search behavior of ADE, and parameter sensitivity of ADE are also studied. Finally, we investigate the capability of ADE for solving three real-world optimization problems.]]></abstract>
##     <issn><![CDATA[2168-2267]]></issn>
##     <htmlFlag><![CDATA[1]]></htmlFlag>
##     <arnumber><![CDATA[6588590]]></arnumber>
##     <doi><![CDATA[10.1109/TCYB.2013.2279211]]></doi>
##     <publicationId><![CDATA[6588590]]></publicationId>
##     <mdurl><![CDATA[http://ieeexplore.ieee.org/xpl/articleDetails.jsp?tp=&arnumber=6588590&contentType=Journals+%26+Magazines]]></mdurl>
##     <pdf><![CDATA[http://ieeexplore.ieee.org/stamp/stamp.jsp?arnumber=6588590]]></pdf>
##   </document>
##   <document>
##     <rank>914</rank>
##     <title><![CDATA[Discrete Anamorphic Transform for Image Compression]]></title>
##     <authors><![CDATA[Asghari, M.H.;  Jalali, B.]]></authors>
##     <affiliations><![CDATA[Electr. Eng. Dept., Univ. of California, Los Angeles, Los Angeles, CA, USA]]></affiliations>
##     <controlledterms>
##       <term><![CDATA[data compression]]></term>
##       <term><![CDATA[discrete transforms]]></term>
##       <term><![CDATA[image coding]]></term>
##     </controlledterms>
##     <thesaurusterms>
##       <term><![CDATA[Bandwidth]]></term>
##       <term><![CDATA[Brightness]]></term>
##       <term><![CDATA[Image coding]]></term>
##       <term><![CDATA[Kernel]]></term>
##       <term><![CDATA[PSNR]]></term>
##       <term><![CDATA[Transform coding]]></term>
##       <term><![CDATA[Transforms]]></term>
##     </thesaurusterms>
##     <pubtitle><![CDATA[Signal Processing Letters, IEEE]]></pubtitle>
##     <punumber><![CDATA[97]]></punumber>
##     <pubtype><![CDATA[Journals & Magazines]]></pubtype>
##     <publisher><![CDATA[IEEE]]></publisher>
##     <volume><![CDATA[21]]></volume>
##     <issue><![CDATA[7]]></issue>
##     <py><![CDATA[2014]]></py>
##     <spage><![CDATA[829]]></spage>
##     <epage><![CDATA[833]]></epage>
##     <abstract><![CDATA[To deal with the exponential increase of digital data, new compression technologies are needed for more efficient representation of information. We introduce a physics-based transform that enables image compression by increasing the spatial coherency. We also present the Stretched Modulation Distribution, a new density function that provides the recipe for the proposed image compression. Experimental results show pre-compression using our method can improve the performance of JPEG 2000 format.]]></abstract>
##     <issn><![CDATA[1070-9908]]></issn>
##     <htmlFlag><![CDATA[1]]></htmlFlag>
##     <arnumber><![CDATA[6804641]]></arnumber>
##     <doi><![CDATA[10.1109/LSP.2014.2319586]]></doi>
##     <publicationId><![CDATA[6804641]]></publicationId>
##     <mdurl><![CDATA[http://ieeexplore.ieee.org/xpl/articleDetails.jsp?tp=&arnumber=6804641&contentType=Journals+%26+Magazines]]></mdurl>
##     <pdf><![CDATA[http://ieeexplore.ieee.org/stamp/stamp.jsp?arnumber=6804641]]></pdf>
##   </document>
##   <document>
##     <rank>915</rank>
##     <title><![CDATA[Simplified Model of Shielding Effectiveness of a Cavity with Apertures on Different Sides]]></title>
##     <authors><![CDATA[Cui Hao;  Denghua Li]]></authors>
##     <affiliations><![CDATA[Sch. of Electron. & Inf. Eng., Beijing Jiaotong Univ., Beijing, China]]></affiliations>
##     <controlledterms>
##       <term><![CDATA[coplanar waveguides]]></term>
##       <term><![CDATA[durability]]></term>
##       <term><![CDATA[electromagnetic shielding]]></term>
##       <term><![CDATA[electromagnetic wave scattering]]></term>
##       <term><![CDATA[rectangular waveguides]]></term>
##       <term><![CDATA[strip lines]]></term>
##       <term><![CDATA[transmission line theory]]></term>
##     </controlledterms>
##     <thesaurusterms>
##       <term><![CDATA[Apertures]]></term>
##       <term><![CDATA[Cavity resonators]]></term>
##       <term><![CDATA[Couplings]]></term>
##       <term><![CDATA[Electromagnetic scattering]]></term>
##       <term><![CDATA[Impedance]]></term>
##       <term><![CDATA[Propagation constant]]></term>
##       <term><![CDATA[Strips]]></term>
##     </thesaurusterms>
##     <pubtitle><![CDATA[Electromagnetic Compatibility, IEEE Transactions on]]></pubtitle>
##     <punumber><![CDATA[15]]></punumber>
##     <pubtype><![CDATA[Journals & Magazines]]></pubtype>
##     <publisher><![CDATA[IEEE]]></publisher>
##     <volume><![CDATA[56]]></volume>
##     <issue><![CDATA[2]]></issue>
##     <py><![CDATA[2014]]></py>
##     <spage><![CDATA[335]]></spage>
##     <epage><![CDATA[342]]></epage>
##     <abstract><![CDATA[Because of the importance of durability against electromagnetic interference in modern life, the shielding effectiveness of a rectangular cavity with apertures on different sides irradiated by the plane electromagnetic wave is modeled in this paper. In the modeling process, the incident angle and the polarization angle of the plane electromagnetic wave are introduced, and the scattering voltage produced on the outer aperture is taken as the radiation source. The aperture is taken as the asymmetrical coplanar stripline and the rectangular cavity is taken as a rectangular waveguide with one end completely open and another end completely closed. The shielding effectiveness of any point in the cavity is calculated using transmission line theory. Compared to other authors before, this method improves the completeness of the model and computation speed which supplies good reference for engineering practice.]]></abstract>
##     <issn><![CDATA[0018-9375]]></issn>
##     <htmlFlag><![CDATA[1]]></htmlFlag>
##     <arnumber><![CDATA[6714607]]></arnumber>
##     <doi><![CDATA[10.1109/TEMC.2013.2280152]]></doi>
##     <publicationId><![CDATA[6714607]]></publicationId>
##     <mdurl><![CDATA[http://ieeexplore.ieee.org/xpl/articleDetails.jsp?tp=&arnumber=6714607&contentType=Journals+%26+Magazines]]></mdurl>
##     <pdf><![CDATA[http://ieeexplore.ieee.org/stamp/stamp.jsp?arnumber=6714607]]></pdf>
##   </document>
##   <document>
##     <rank>916</rank>
##     <title><![CDATA[Non-Rigid Object Detection with LocalInterleaved Sequential Alignment (LISA)]]></title>
##     <authors><![CDATA[Zimmermann, K.;  Hurych, D.;  Svoboda, T.]]></authors>
##     <affiliations><![CDATA[Dept. of Cybern., Czech Tech. Univ. in Prague, Prague, Czech Republic]]></affiliations>
##     <controlledterms>
##       <term><![CDATA[feature extraction]]></term>
##       <term><![CDATA[object detection]]></term>
##       <term><![CDATA[regression analysis]]></term>
##     </controlledterms>
##     <thesaurusterms>
##       <term><![CDATA[Computational modeling]]></term>
##       <term><![CDATA[Deformable models]]></term>
##       <term><![CDATA[Detectors]]></term>
##       <term><![CDATA[Estimation]]></term>
##       <term><![CDATA[Feature extraction]]></term>
##       <term><![CDATA[Object detection]]></term>
##       <term><![CDATA[Training]]></term>
##     </thesaurusterms>
##     <pubtitle><![CDATA[Pattern Analysis and Machine Intelligence, IEEE Transactions on]]></pubtitle>
##     <punumber><![CDATA[34]]></punumber>
##     <pubtype><![CDATA[Journals & Magazines]]></pubtype>
##     <publisher><![CDATA[IEEE]]></publisher>
##     <volume><![CDATA[36]]></volume>
##     <issue><![CDATA[4]]></issue>
##     <py><![CDATA[2014]]></py>
##     <spage><![CDATA[731]]></spage>
##     <epage><![CDATA[743]]></epage>
##     <abstract><![CDATA[This paper shows that the successively evaluated features used in a sliding window detection process to decide about object presence/absence also contain knowledge about object deformation. We exploit these detection features to estimate the object deformation. Estimated deformation is then immediately applied to not yet evaluated features to align them with the observed image data. In our approach, the alignment estimators are jointly learned with the detector. The joint process allows for the learning of each detection stage from less deformed training samples than in the previous stage. For the alignment estimation we propose regressors that approximate non-linear regression functions and compute the alignment parameters extremely fast.]]></abstract>
##     <issn><![CDATA[0162-8828]]></issn>
##     <htmlFlag><![CDATA[1]]></htmlFlag>
##     <arnumber><![CDATA[6778003]]></arnumber>
##     <doi><![CDATA[10.1109/TPAMI.2013.171]]></doi>
##     <publicationId><![CDATA[6778003]]></publicationId>
##     <mdurl><![CDATA[http://ieeexplore.ieee.org/xpl/articleDetails.jsp?tp=&arnumber=6778003&contentType=Journals+%26+Magazines]]></mdurl>
##     <pdf><![CDATA[http://ieeexplore.ieee.org/stamp/stamp.jsp?arnumber=6778003]]></pdf>
##   </document>
##   <document>
##     <rank>917</rank>
##     <title><![CDATA[Low-Complexity Soft-Output Quantum-Assisted Multiuser Detection for Direct-Sequence Spreading and Slow Subcarrier-Hopping Aided SDMA-OFDM Systems]]></title>
##     <authors><![CDATA[Botsinis, P.;  Alanis, D.;  Ng, S.X.;  Hanzo, L.]]></authors>
##     <affiliations><![CDATA[Sch. of Electron. & Comput. Sci., Univ. of Southampton, Southampton, UK]]></affiliations>
##     <controlledterms>
##       <term><![CDATA[OFDM modulation]]></term>
##       <term><![CDATA[ant colony optimisation]]></term>
##       <term><![CDATA[computational complexity]]></term>
##       <term><![CDATA[error statistics]]></term>
##       <term><![CDATA[maximum likelihood detection]]></term>
##       <term><![CDATA[multiuser detection]]></term>
##       <term><![CDATA[quantum computing]]></term>
##       <term><![CDATA[space division multiple access]]></term>
##       <term><![CDATA[spread spectrum communication]]></term>
##     </controlledterms>
##     <thesaurusterms>
##       <term><![CDATA[Communication systems]]></term>
##       <term><![CDATA[Computational complexity]]></term>
##       <term><![CDATA[Decision support systems]]></term>
##       <term><![CDATA[Detectors]]></term>
##       <term><![CDATA[Multiuser detection]]></term>
##       <term><![CDATA[OFDM]]></term>
##     </thesaurusterms>
##     <pubtitle><![CDATA[Access, IEEE]]></pubtitle>
##     <punumber><![CDATA[6287639]]></punumber>
##     <pubtype><![CDATA[Journals & Magazines]]></pubtype>
##     <publisher><![CDATA[IEEE]]></publisher>
##     <volume><![CDATA[2]]></volume>
##     <py><![CDATA[2014]]></py>
##     <spage><![CDATA[451]]></spage>
##     <epage><![CDATA[472]]></epage>
##     <abstract><![CDATA[Low-complexity suboptimal multiuser detectors (MUDs) are widely used in multiple access communication systems for separating users, since the computational complexity of the maximum likelihood (ML) detector is potentially excessive for practical implementation. Quantum computing may be invoked in the detection procedure, by exploiting its inherent parallelism for approaching the ML MUDs performance at a substantially reduced number of cost function evaluations. In this contribution, we propose a soft-output (SO) quantum-assisted MUD achieving a near-ML performance and compare it to the corresponding SO ant colony optimization MUD. We investigate rank deficient direct-sequence spreading (DSS) and slow subcarrier-hopping aided (SSCH) spatial division multiple access orthogonal frequency division multiplexing systems, where the number of users to be detected is higher than the number of receive antenna elements used. We show that for a given complexity budget, the proposed SO-Du&#x0308;rr-H&#x00F8;yer algorithm (DHA) QMUD achieves a better performance. We also propose an adaptive hybrid SO-ML/SO-DHA MUD, which adapts itself to the number of users equipped with the same spreading sequence and transmitting on the same subcarrier. Finally, we propose a DSS-based uniform SSCH scheme, which improves the system's performance by 0.5 dB at a BER of 10<sup>-5</sup>, despite reducing the complexity required by the MUDs employed.]]></abstract>
##     <issn><![CDATA[2169-3536]]></issn>
##     <htmlFlag><![CDATA[1]]></htmlFlag>
##     <arnumber><![CDATA[6812127]]></arnumber>
##     <doi><![CDATA[10.1109/ACCESS.2014.2322013]]></doi>
##     <publicationId><![CDATA[6812127]]></publicationId>
##     <mdurl><![CDATA[http://ieeexplore.ieee.org/xpl/articleDetails.jsp?tp=&arnumber=6812127&contentType=Journals+%26+Magazines]]></mdurl>
##     <pdf><![CDATA[http://ieeexplore.ieee.org/stamp/stamp.jsp?arnumber=6812127]]></pdf>
##   </document>
##   <document>
##     <rank>918</rank>
##     <title><![CDATA[Energy-Efficient Design in Heterogeneous Cellular Networks Based on Large-Scale User Behavior Constraints]]></title>
##     <authors><![CDATA[Yu Huang;  Xing Zhang;  Jiaxin Zhang;  Jian Tang;  Zhuowen Su;  Wenbo Wang]]></authors>
##     <affiliations><![CDATA[Key Lab. of Universal Wireless Commun., Beijing Univ. of Posts & Telecommun., Beijing, China]]></affiliations>
##     <controlledterms>
##       <term><![CDATA[cellular radio]]></term>
##       <term><![CDATA[energy conservation]]></term>
##       <term><![CDATA[optimisation]]></term>
##       <term><![CDATA[telecommunication power management]]></term>
##     </controlledterms>
##     <thesaurusterms>
##       <term><![CDATA[Interference]]></term>
##       <term><![CDATA[Numerical models]]></term>
##       <term><![CDATA[Optimization]]></term>
##       <term><![CDATA[Power demand]]></term>
##       <term><![CDATA[Resource management]]></term>
##       <term><![CDATA[Signal to noise ratio]]></term>
##       <term><![CDATA[Wireless communication]]></term>
##     </thesaurusterms>
##     <pubtitle><![CDATA[Wireless Communications, IEEE Transactions on]]></pubtitle>
##     <punumber><![CDATA[7693]]></punumber>
##     <pubtype><![CDATA[Journals & Magazines]]></pubtype>
##     <publisher><![CDATA[IEEE]]></publisher>
##     <volume><![CDATA[13]]></volume>
##     <issue><![CDATA[9]]></issue>
##     <py><![CDATA[2014]]></py>
##     <spage><![CDATA[4746]]></spage>
##     <epage><![CDATA[4757]]></epage>
##     <abstract><![CDATA[Large-scale user behavior can be used as the guidance for deployment, configuration, and service control in heterogeneous cellular networks (HCNs). However, in wireless networks, large-scale user behavior (in terms of traffic fluctuation in spatial domain) follows inhomogeneous distribution, which brings enormous challenges to energy-efficient design of HCNs. In this paper, the heterogeneity of large-scale user behavior is quantitatively characterized and exploited to study the energy efficiency (EE) in HCNs. An optimization problem is formulated for energy-efficient two-tier deployment and configuration, where the base station (BS) density, BS transmit power, BS static power, and quality of service are taken into account. We present closed-form formulas that establish the quantitative relationship between large-scale user behavior and energy-efficient HCN configuration. These results can be used to determine BS density and BS transmit power with the objective of achieving optimal EE. Furthermore, we present three energy-efficient control strategies of micro BSs, including micro BS sleep control, coverage expansion control, and coverage shrinking control. Simulation results validate our theoretical analysis and demonstrate that the proposed control strategies can potentially lead to significant power savings.]]></abstract>
##     <issn><![CDATA[1536-1276]]></issn>
##     <htmlFlag><![CDATA[1]]></htmlFlag>
##     <arnumber><![CDATA[6832633]]></arnumber>
##     <doi><![CDATA[10.1109/TWC.2014.2330334]]></doi>
##     <publicationId><![CDATA[6832633]]></publicationId>
##     <mdurl><![CDATA[http://ieeexplore.ieee.org/xpl/articleDetails.jsp?tp=&arnumber=6832633&contentType=Journals+%26+Magazines]]></mdurl>
##     <pdf><![CDATA[http://ieeexplore.ieee.org/stamp/stamp.jsp?arnumber=6832633]]></pdf>
##   </document>
##   <document>
##     <rank>919</rank>
##     <title><![CDATA[An Efficient and Scalable Semiconductor Architecture for Parallel Automata Processing]]></title>
##     <authors><![CDATA[Dlugosch, P.;  Brown, D.;  Glendenning, P.;  Leventhal, M.;  Noyes, H.]]></authors>
##     <affiliations><![CDATA[Micron Technol., DRAM Solutions Group, Boise, ID, USA]]></affiliations>
##     <controlledterms>
##       <term><![CDATA[XML]]></term>
##       <term><![CDATA[field programmable gate arrays]]></term>
##       <term><![CDATA[finite automata]]></term>
##       <term><![CDATA[parallel processing]]></term>
##       <term><![CDATA[pattern matching]]></term>
##     </controlledterms>
##     <thesaurusterms>
##       <term><![CDATA[Arrays]]></term>
##       <term><![CDATA[Automata]]></term>
##       <term><![CDATA[Complexity theory]]></term>
##       <term><![CDATA[Hardware]]></term>
##       <term><![CDATA[Radiation detectors]]></term>
##       <term><![CDATA[Routing]]></term>
##     </thesaurusterms>
##     <pubtitle><![CDATA[Parallel and Distributed Systems, IEEE Transactions on]]></pubtitle>
##     <punumber><![CDATA[71]]></punumber>
##     <pubtype><![CDATA[Journals & Magazines]]></pubtype>
##     <publisher><![CDATA[IEEE]]></publisher>
##     <volume><![CDATA[25]]></volume>
##     <issue><![CDATA[12]]></issue>
##     <py><![CDATA[2014]]></py>
##     <spage><![CDATA[3088]]></spage>
##     <epage><![CDATA[3098]]></epage>
##     <abstract><![CDATA[We present the design and development of the automata processor, a massively parallel non-von Neumann semiconductor architecture that is purpose-built for automata processing. This architecture can directly implement non-deterministic finite automata in hardware and can be used to implement complex regular expressions, as well as other types of automata which cannot be expressed as regular expressions. We demonstrate that this architecture exceeds the capabilities of high-performance FPGA-based implementations of regular expression processors. We report on the development of an XML-based language for describing automata for easy compilation targeted to the hardware. The automata processor can be effectively utilized in a diverse array of applications driven by pattern matching, such as cyber security and computational biology.]]></abstract>
##     <issn><![CDATA[1045-9219]]></issn>
##     <htmlFlag><![CDATA[1]]></htmlFlag>
##     <arnumber><![CDATA[6719386]]></arnumber>
##     <doi><![CDATA[10.1109/TPDS.2014.8]]></doi>
##     <publicationId><![CDATA[6719386]]></publicationId>
##     <mdurl><![CDATA[http://ieeexplore.ieee.org/xpl/articleDetails.jsp?tp=&arnumber=6719386&contentType=Journals+%26+Magazines]]></mdurl>
##     <pdf><![CDATA[http://ieeexplore.ieee.org/stamp/stamp.jsp?arnumber=6719386]]></pdf>
##   </document>
##   <document>
##     <rank>920</rank>
##     <title><![CDATA[Unidirectional Transmission Based on a Passive PT Symmetric Grating With a Nonlinear Silicon Distributed Bragg Reflector Cavity]]></title>
##     <authors><![CDATA[Ye-Long Xu;  Liang Feng;  Ming-Hui Lu;  Yan-eng Chen]]></authors>
##     <affiliations><![CDATA[Dept. of Mater. Sci. & Eng., Nanjing Univ., Nanjing, China]]></affiliations>
##     <controlledterms>
##       <term><![CDATA[diffraction gratings]]></term>
##       <term><![CDATA[distributed Bragg reflectors]]></term>
##       <term><![CDATA[elemental semiconductors]]></term>
##       <term><![CDATA[finite difference time-domain analysis]]></term>
##       <term><![CDATA[optical waveguides]]></term>
##       <term><![CDATA[silicon]]></term>
##     </controlledterms>
##     <thesaurusterms>
##       <term><![CDATA[Cavity resonators]]></term>
##       <term><![CDATA[Distributed Bragg reflectors]]></term>
##       <term><![CDATA[Modulation]]></term>
##       <term><![CDATA[Nonlinear optics]]></term>
##       <term><![CDATA[Optical resonators]]></term>
##       <term><![CDATA[Optical waveguides]]></term>
##       <term><![CDATA[Silicon]]></term>
##     </thesaurusterms>
##     <pubtitle><![CDATA[Photonics Journal, IEEE]]></pubtitle>
##     <punumber><![CDATA[4563994]]></punumber>
##     <pubtype><![CDATA[Journals & Magazines]]></pubtype>
##     <publisher><![CDATA[IEEE]]></publisher>
##     <volume><![CDATA[6]]></volume>
##     <issue><![CDATA[1]]></issue>
##     <py><![CDATA[2014]]></py>
##     <spage><![CDATA[1]]></spage>
##     <epage><![CDATA[7]]></epage>
##     <abstract><![CDATA[A silicon waveguide consisting of passive parity-time (PT) symmetry optical potentials connected with a distributed Bragg reflector (DBR)-based resonator is proposed to achieve nonreciprocal waveguide transmission. The unidirectional reflectionless waveguide implementing PT symmetry is discussed by consistent coupling mode theory and finite-difference time-domain (FDTD) simulation results. Due to the high field confinement in the DBR cavity, transmission of a light pulse through the device is analyzed in the nonlinear optical regime using FDTD simulations. The combination of the nonlinear DBR cavity and the passive PT symmetric grating results in unidirectional transmission in the silicon waveguide, while still keeping the reflection minimum in the desired direction. The numerical simulation shows an extinction ratio of about 20 dB at the telecom wavelength of around 1550 nm.]]></abstract>
##     <issn><![CDATA[1943-0655]]></issn>
##     <htmlFlag><![CDATA[1]]></htmlFlag>
##     <arnumber><![CDATA[6690216]]></arnumber>
##     <doi><![CDATA[10.1109/JPHOT.2013.2295462]]></doi>
##     <publicationId><![CDATA[6690216]]></publicationId>
##     <mdurl><![CDATA[http://ieeexplore.ieee.org/xpl/articleDetails.jsp?tp=&arnumber=6690216&contentType=Journals+%26+Magazines]]></mdurl>
##     <pdf><![CDATA[http://ieeexplore.ieee.org/stamp/stamp.jsp?arnumber=6690216]]></pdf>
##   </document>
##   <document>
##     <rank>921</rank>
##     <title><![CDATA[A Simple but Powerful Heuristic Method for Accelerating <formula formulatype="inline"> <img src="/images/tex/348.gif" alt="k"> </formula> -Means Clustering of Large-Scale Data in Life Science]]></title>
##     <authors><![CDATA[Ichikawa, K.;  Morishita, S.]]></authors>
##     <affiliations><![CDATA[Dept. of Comput. Biol., Univ. of Tokyo, Kashiwa, Japan]]></affiliations>
##     <controlledterms>
##       <term><![CDATA[bioinformatics]]></term>
##       <term><![CDATA[data mining]]></term>
##       <term><![CDATA[optimisation]]></term>
##       <term><![CDATA[pattern clustering]]></term>
##     </controlledterms>
##     <thesaurusterms>
##       <term><![CDATA[Bioinformatics]]></term>
##       <term><![CDATA[Clustering algorithms]]></term>
##       <term><![CDATA[Computational biology]]></term>
##       <term><![CDATA[Correlation]]></term>
##       <term><![CDATA[Correlation coefficient]]></term>
##       <term><![CDATA[Euclidean distance]]></term>
##     </thesaurusterms>
##     <pubtitle><![CDATA[Computational Biology and Bioinformatics, IEEE/ACM Transactions on]]></pubtitle>
##     <punumber><![CDATA[8857]]></punumber>
##     <pubtype><![CDATA[Journals & Magazines]]></pubtype>
##     <publisher><![CDATA[IEEE]]></publisher>
##     <volume><![CDATA[11]]></volume>
##     <issue><![CDATA[4]]></issue>
##     <py><![CDATA[2014]]></py>
##     <spage><![CDATA[681]]></spage>
##     <epage><![CDATA[692]]></epage>
##     <abstract><![CDATA[K-means clustering has been widely used to gain insight into biological systems from large-scale life science data. To quantify the similarities among biological data sets, Pearson correlation distance and standardized Euclidean distance are used most frequently; however, optimization methods have been largely unexplored. These two distance measurements are equivalent in the sense that they yield the same k-means clustering result for identical sets of k initial centroids. Thus, an efficient algorithm used for one is applicable to the other. Several optimization methods are available for the Euclidean distance and can be used for processing the standardized Euclidean distance; however, they are not customized for this context. We instead approached the problem by studying the properties of the Pearson correlation distance, and we invented a simple but powerful heuristic method for markedly pruning unnecessary computation while retaining the final solution. Tests using real biological data sets with 50-60K vectors of dimensions 10-2001 (~400 MB in size) demonstrated marked reduction in computation time for k = 10-500 in comparison with other state-of-the-art pruning methods such as Elkan's and Hamerly's algorithms. The BoostKCP software is available at http://mlab.cb.k.u-tokyo.ac.jp/~ichikawa/boostKCP/.]]></abstract>
##     <issn><![CDATA[1545-5963]]></issn>
##     <htmlFlag><![CDATA[1]]></htmlFlag>
##     <arnumber><![CDATA[6739991]]></arnumber>
##     <doi><![CDATA[10.1109/TCBB.2014.2306200]]></doi>
##     <publicationId><![CDATA[6739991]]></publicationId>
##     <mdurl><![CDATA[http://ieeexplore.ieee.org/xpl/articleDetails.jsp?tp=&arnumber=6739991&contentType=Journals+%26+Magazines]]></mdurl>
##     <pdf><![CDATA[http://ieeexplore.ieee.org/stamp/stamp.jsp?arnumber=6739991]]></pdf>
##   </document>
##   <document>
##     <rank>922</rank>
##     <title><![CDATA[Power Minimization Based Resource Allocation for Interference Mitigation in OFDMA Femtocell Networks]]></title>
##     <authors><![CDATA[Lopez-Perez, D.;  Xiaoli Chu;  Vasilakos, A.V.;  Claussen, H.]]></authors>
##     <affiliations><![CDATA[Bell Labs., Alcatel-Lucent, Dublin, Ireland]]></affiliations>
##     <controlledterms>
##       <term><![CDATA[OFDM modulation]]></term>
##       <term><![CDATA[channel allocation]]></term>
##       <term><![CDATA[femtocellular radio]]></term>
##       <term><![CDATA[frequency division multiple access]]></term>
##       <term><![CDATA[interference suppression]]></term>
##       <term><![CDATA[telecommunication network topology]]></term>
##     </controlledterms>
##     <thesaurusterms>
##       <term><![CDATA[Computer architecture]]></term>
##       <term><![CDATA[Frequency modulation]]></term>
##       <term><![CDATA[Interference]]></term>
##       <term><![CDATA[Minimization]]></term>
##       <term><![CDATA[Resource management]]></term>
##       <term><![CDATA[Signal to noise ratio]]></term>
##       <term><![CDATA[Throughput]]></term>
##     </thesaurusterms>
##     <pubtitle><![CDATA[Selected Areas in Communications, IEEE Journal on]]></pubtitle>
##     <punumber><![CDATA[49]]></punumber>
##     <pubtype><![CDATA[Journals & Magazines]]></pubtype>
##     <publisher><![CDATA[IEEE]]></publisher>
##     <volume><![CDATA[32]]></volume>
##     <issue><![CDATA[2]]></issue>
##     <py><![CDATA[2014]]></py>
##     <spage><![CDATA[333]]></spage>
##     <epage><![CDATA[344]]></epage>
##     <abstract><![CDATA[With the introduction of femtocells, cellular networks are moving from the conventional centralized network architecture to a distributed one, where each network cell should make its own radio resource allocation decisions, while providing inter-cell interference mitigation. However, realizing such distributed network architecture is not a trivial task. In this paper, we first introduce a simple self-organization rule, based on minimizing cell transmit power, following which a distributed cellular network is able to converge into an efficient resource reuse pattern. Based on such self-organization rule and taking realistic resource allocation constraints into account, we also propose two novel resource allocation algorithms, being autonomous and coordinated, respectively. Performance of the proposed self-organization rule and resource allocation algorithms are evaluated using system-level simulations, and show that power efficiency is not necessarily in conflict with capacity improvements at the network level. The proposed resource allocation algorithms provide significant performance improvements in terms of user outages and network capacity over cutting-edge resource allocation algorithms proposed in the literature.]]></abstract>
##     <issn><![CDATA[0733-8716]]></issn>
##     <htmlFlag><![CDATA[1]]></htmlFlag>
##     <arnumber><![CDATA[6514950]]></arnumber>
##     <doi><![CDATA[10.1109/JSAC.2014.141213]]></doi>
##     <publicationId><![CDATA[6514950]]></publicationId>
##     <mdurl><![CDATA[http://ieeexplore.ieee.org/xpl/articleDetails.jsp?tp=&arnumber=6514950&contentType=Journals+%26+Magazines]]></mdurl>
##     <pdf><![CDATA[http://ieeexplore.ieee.org/stamp/stamp.jsp?arnumber=6514950]]></pdf>
##   </document>
##   <document>
##     <rank>923</rank>
##     <title><![CDATA[A Subwavelength Focusing Structure Composite of Nanoscale Metallic Slits Array With Patterned Dielectric Substrate]]></title>
##     <authors><![CDATA[Sen Jia;  Yiming Wu;  Xianhua Wang;  Ning Wang]]></authors>
##     <affiliations><![CDATA[Opt. Directional & Pointing Tech. Res. Center, Xi'an Inst. of Opt. & Precision Mech., Xi'an, China]]></affiliations>
##     <controlledterms>
##       <term><![CDATA[lenses]]></term>
##       <term><![CDATA[optical focusing]]></term>
##       <term><![CDATA[plasmonics]]></term>
##       <term><![CDATA[polaritons]]></term>
##       <term><![CDATA[surface plasmons]]></term>
##     </controlledterms>
##     <thesaurusterms>
##       <term><![CDATA[Dielectric substrates]]></term>
##       <term><![CDATA[Filling]]></term>
##       <term><![CDATA[Focusing]]></term>
##       <term><![CDATA[Lenses]]></term>
##       <term><![CDATA[Materials]]></term>
##       <term><![CDATA[Metals]]></term>
##       <term><![CDATA[Plasmons]]></term>
##     </thesaurusterms>
##     <pubtitle><![CDATA[Photonics Journal, IEEE]]></pubtitle>
##     <punumber><![CDATA[4563994]]></punumber>
##     <pubtype><![CDATA[Journals & Magazines]]></pubtype>
##     <publisher><![CDATA[IEEE]]></publisher>
##     <volume><![CDATA[6]]></volume>
##     <issue><![CDATA[1]]></issue>
##     <py><![CDATA[2014]]></py>
##     <spage><![CDATA[1]]></spage>
##     <epage><![CDATA[8]]></epage>
##     <abstract><![CDATA[A novel plasmonic lens consisting of metallic nanoslits with a patterned dielectric substrate is proposed. In the structure, all metallic nanoslits have the same geometrical parameters and interspacing. The phase of incident light is modulated beforehand by the substrate before they bump on the metal slits. The surface plasmon polaritons (SPPs) wave excited by the modulated light will be focused after passing through the slits. Numerical simulation demonstrates that the size of the generated focal spot is very close to half wavelength. The overriding advantage of the proposed structure is that it significantly reduces the difficulty in fabrication while the focusing properties are comparable to the ones with filling materials in the slits.]]></abstract>
##     <issn><![CDATA[1943-0655]]></issn>
##     <htmlFlag><![CDATA[1]]></htmlFlag>
##     <arnumber><![CDATA[6704768]]></arnumber>
##     <doi><![CDATA[10.1109/JPHOT.2014.2298406]]></doi>
##     <publicationId><![CDATA[6704768]]></publicationId>
##     <mdurl><![CDATA[http://ieeexplore.ieee.org/xpl/articleDetails.jsp?tp=&arnumber=6704768&contentType=Journals+%26+Magazines]]></mdurl>
##     <pdf><![CDATA[http://ieeexplore.ieee.org/stamp/stamp.jsp?arnumber=6704768]]></pdf>
##   </document>
##   <document>
##     <rank>924</rank>
##     <title><![CDATA[A Novel MPSoC Interface and Control Architecture for Multistandard RF Transceivers]]></title>
##     <authors><![CDATA[Brandstatter, S.;  Huemer, M.]]></authors>
##     <affiliations><![CDATA[Co., KG., Danube Mobile Commun. Eng. GmbH, Linz, Austria]]></affiliations>
##     <controlledterms>
##       <term><![CDATA[data communication]]></term>
##       <term><![CDATA[mobile radio]]></term>
##       <term><![CDATA[multiprocessing systems]]></term>
##       <term><![CDATA[radio transceivers]]></term>
##       <term><![CDATA[system-on-chip]]></term>
##       <term><![CDATA[telecommunication power management]]></term>
##     </controlledterms>
##     <thesaurusterms>
##       <term><![CDATA[Baseband]]></term>
##       <term><![CDATA[Mobile communication]]></term>
##       <term><![CDATA[Process control]]></term>
##       <term><![CDATA[Radio frequency]]></term>
##       <term><![CDATA[Transceivers]]></term>
##     </thesaurusterms>
##     <pubtitle><![CDATA[Access, IEEE]]></pubtitle>
##     <punumber><![CDATA[6287639]]></punumber>
##     <pubtype><![CDATA[Journals & Magazines]]></pubtype>
##     <publisher><![CDATA[IEEE]]></publisher>
##     <volume><![CDATA[2]]></volume>
##     <py><![CDATA[2014]]></py>
##     <spage><![CDATA[771]]></spage>
##     <epage><![CDATA[787]]></epage>
##     <abstract><![CDATA[The introduction of new mobile communication standards, enabling the ever growing amount of data transmitted in mobile communication networks, continuously increases the complexity of control processing within radio frequency (RF) transceivers. Since this complexity cannot be handled by traditional approaches, this paper focuses on the partitioning of RF transceiver systems and on the implementation of application-specific components to introduce an advanced multiprocessor system-on-chip interface and control architecture which is able to fulfill the requirements of future RF transceiver integrations. The proposed framework demonstrates a high degree of scalability, flexibility, and reusability. Consequently, the time to market for products can be reduced and fast adaptations to the requirements of the market are feasible. In addition, the developed application-specific components achieve improved or at least equivalent performance results compared with common architectures while the silicon area can be reduced. This characteristic has positive effects on the costs as well as on the power consumption of the RF transceiver.]]></abstract>
##     <issn><![CDATA[2169-3536]]></issn>
##     <htmlFlag><![CDATA[1]]></htmlFlag>
##     <arnumber><![CDATA[6870422]]></arnumber>
##     <doi><![CDATA[10.1109/ACCESS.2014.2345194]]></doi>
##     <publicationId><![CDATA[6870422]]></publicationId>
##     <mdurl><![CDATA[http://ieeexplore.ieee.org/xpl/articleDetails.jsp?tp=&arnumber=6870422&contentType=Journals+%26+Magazines]]></mdurl>
##     <pdf><![CDATA[http://ieeexplore.ieee.org/stamp/stamp.jsp?arnumber=6870422]]></pdf>
##   </document>
##   <document>
##     <rank>925</rank>
##     <title><![CDATA[Ultrafast imaging in biomedical ultrasound]]></title>
##     <authors><![CDATA[Tanter, M.;  Fink, M.]]></authors>
##     <affiliations><![CDATA[Inst. Langevin, Ecole Super. de Phys. et de Chim. Industrielles de la Ville de Paris (ESPCI), Paris, France]]></affiliations>
##     <controlledterms>
##       <term><![CDATA[biomedical ultrasonics]]></term>
##       <term><![CDATA[graphics processing units]]></term>
##       <term><![CDATA[patient monitoring]]></term>
##       <term><![CDATA[reviews]]></term>
##       <term><![CDATA[ultrasonic transmission]]></term>
##     </controlledterms>
##     <thesaurusterms>
##       <term><![CDATA[Acoustics]]></term>
##       <term><![CDATA[Biomedical imaging]]></term>
##       <term><![CDATA[Holography]]></term>
##       <term><![CDATA[Optical imaging]]></term>
##       <term><![CDATA[Ultrafast optics]]></term>
##       <term><![CDATA[Ultrasonic imaging]]></term>
##     </thesaurusterms>
##     <pubtitle><![CDATA[Ultrasonics, Ferroelectrics, and Frequency Control, IEEE Transactions on]]></pubtitle>
##     <punumber><![CDATA[58]]></punumber>
##     <pubtype><![CDATA[Journals & Magazines]]></pubtype>
##     <publisher><![CDATA[IEEE]]></publisher>
##     <volume><![CDATA[61]]></volume>
##     <issue><![CDATA[1]]></issue>
##     <py><![CDATA[2014]]></py>
##     <spage><![CDATA[102]]></spage>
##     <epage><![CDATA[119]]></epage>
##     <abstract><![CDATA[Although the use of ultrasonic plane-wave transmissions rather than line-per-line focused beam transmissions has been long studied in research, clinical application of this technology was only recently made possible through developments in graphical processing unit (GPU)-based platforms. Far beyond a technological breakthrough, the use of plane or diverging wave transmissions enables attainment of ultrafast frame rates (typically faster than 1000 frames per second) over a large field of view. This concept has also inspired the emergence of completely novel imaging modes which are valuable for ultrasound-based screening, diagnosis, and therapeutic monitoring. In this review article, we present the basic principles and implementation of ultrafast imaging. In particular, present and future applications of ultrafast imaging in biomedical ultrasound are illustrated and discussed.]]></abstract>
##     <issn><![CDATA[0885-3010]]></issn>
##     <htmlFlag><![CDATA[1]]></htmlFlag>
##     <arnumber><![CDATA[6689779]]></arnumber>
##     <doi><![CDATA[10.1109/TUFFC.2014.2882]]></doi>
##     <publicationId><![CDATA[6689779]]></publicationId>
##     <mdurl><![CDATA[http://ieeexplore.ieee.org/xpl/articleDetails.jsp?tp=&arnumber=6689779&contentType=Journals+%26+Magazines]]></mdurl>
##     <pdf><![CDATA[http://ieeexplore.ieee.org/stamp/stamp.jsp?arnumber=6689779]]></pdf>
##   </document>
##   <document>
##     <rank>926</rank>
##     <title><![CDATA[Supercapacitor Energy Storage for Magnetic Resonance Imaging Systems]]></title>
##     <authors><![CDATA[Ristic, M.;  Gryska, Y.;  McGinley, J.V.M.;  Yufit, V.]]></authors>
##     <affiliations><![CDATA[Dept. of Mech. Eng., Imperial Coll. London, London, UK]]></affiliations>
##     <controlledterms>
##       <term><![CDATA[biomedical MRI]]></term>
##       <term><![CDATA[iron compounds]]></term>
##       <term><![CDATA[lithium compounds]]></term>
##       <term><![CDATA[phosphorus compounds]]></term>
##       <term><![CDATA[power supplies to apparatus]]></term>
##       <term><![CDATA[secondary cells]]></term>
##       <term><![CDATA[supercapacitors]]></term>
##     </controlledterms>
##     <thesaurusterms>
##       <term><![CDATA[Batteries]]></term>
##       <term><![CDATA[Capacitance]]></term>
##       <term><![CDATA[Coils]]></term>
##       <term><![CDATA[Magnetic resonance imaging]]></term>
##       <term><![CDATA[Supercapacitors]]></term>
##     </thesaurusterms>
##     <pubtitle><![CDATA[Industrial Electronics, IEEE Transactions on]]></pubtitle>
##     <punumber><![CDATA[41]]></punumber>
##     <pubtype><![CDATA[Journals & Magazines]]></pubtype>
##     <publisher><![CDATA[IEEE]]></publisher>
##     <volume><![CDATA[61]]></volume>
##     <issue><![CDATA[8]]></issue>
##     <py><![CDATA[2014]]></py>
##     <spage><![CDATA[4255]]></spage>
##     <epage><![CDATA[4264]]></epage>
##     <abstract><![CDATA[Magnetic resonance imaging (MRI) involves very short pulses of very high current. Substantial savings in the high cost of MRI installations may be realized by employing suitable electrical energy storage, for which supercapacitors are strong candidates in view of high specific power and long cycle life. A key question is whether the well-known capacitance degradation with increased frequency is compatible with the complex and highly variable duty cycles of various MRI sequences. Compatibility of the supercapacitor voltage range with the MRI system must also be considered. We present a detailed analysis of power duty profiles in MRI, using actual imaging sequences, that has not been reported previously. We also propose and validate a simplified supercapacitor model that can accurately simulate its performance in the MRI system, involving pulses that are several orders of magnitude shorter than those considered previously. Results of equivalent experiments involving lithium-ion iron phosphate ( LiFePO<sub>4</sub>) batteries are also reported. Finally, we present a detailed analysis of the overall energy storage performance in a realistic neurological examination. The study is based on a specific system of our own design, and we fully disclose its relevant parameters, so that the results would be of direct practical value to the wider community, including developers of MRI.]]></abstract>
##     <issn><![CDATA[0278-0046]]></issn>
##     <htmlFlag><![CDATA[1]]></htmlFlag>
##     <arnumber><![CDATA[6648426]]></arnumber>
##     <doi><![CDATA[10.1109/TIE.2013.2282596]]></doi>
##     <publicationId><![CDATA[6648426]]></publicationId>
##     <mdurl><![CDATA[http://ieeexplore.ieee.org/xpl/articleDetails.jsp?tp=&arnumber=6648426&contentType=Journals+%26+Magazines]]></mdurl>
##     <pdf><![CDATA[http://ieeexplore.ieee.org/stamp/stamp.jsp?arnumber=6648426]]></pdf>
##   </document>
##   <document>
##     <rank>927</rank>
##     <title><![CDATA[Direct-Estimation Algorithm for Mapping Daily Land-Surface Broadband Albedo From MODIS Data]]></title>
##     <authors><![CDATA[Ying Qu;  Qiang Liu;  Shunlin Liang;  Lizhao Wang;  Nanfeng Liu;  Suhong Liu]]></authors>
##     <affiliations><![CDATA[Sch. of Geogr., Beijing Normal Univ., Beijing, China]]></affiliations>
##     <controlledterms>
##       <term><![CDATA[atmospheric radiation]]></term>
##       <term><![CDATA[geophysical techniques]]></term>
##       <term><![CDATA[remote sensing]]></term>
##     </controlledterms>
##     <pubtitle><![CDATA[Geoscience and Remote Sensing, IEEE Transactions on]]></pubtitle>
##     <punumber><![CDATA[36]]></punumber>
##     <pubtype><![CDATA[Journals & Magazines]]></pubtype>
##     <publisher><![CDATA[IEEE]]></publisher>
##     <volume><![CDATA[52]]></volume>
##     <issue><![CDATA[2]]></issue>
##     <py><![CDATA[2014]]></py>
##     <spage><![CDATA[907]]></spage>
##     <epage><![CDATA[919]]></epage>
##     <abstract><![CDATA[Land surface albedo is a critical parameter in surface-energy budget studies. Over the past several decades, many albedo products are generated from remote-sensing data sets. The Moderate Resolution Imaging Spectroradiometer (MODIS) bidirectional reflectance distribution function (BRDF)/Albedo algorithm is used to routinely produce eight day (16-day composite), 1-km resolution MODIS albedo products. When some natural processes or human activities occur, the land-surface broadband albedo can change rapidly, so it is necessary to enhance the temporal resolution of albedo product. We present a direct-estimation algorithm for mapping daily land-surface broadband albedo from MODIS data. The polarization and directionality of the Earth's reflectance-3/polarization and anisotropy of reflectances for atmospheric sciences coupled with observations from a Lidar BRDF database is employed as a training data set, and the 6S atmospheric radiative transfer code is used to simulate the top-of-atmosphere (TOA) reflectances. Then a relationship between TOA reflectances and land-surface broadband albedos is developed using an angular bin regression method. The robustness of this method for different angular bins, aerosol conditions, and land-cover types is analyzed. Simulation results show that the absolute error of this algorithm is ~ 0.009 for vegetation, 0.012 for soil, and 0.030 for snow/ice. Validation of the direct-estimation algorithm against in situ measurement data shows that the proposed method is capable of characterizing the temporal variation of albedo, especially when the land-surface BRDF changes rapidly.]]></abstract>
##     <issn><![CDATA[0196-2892]]></issn>
##     <htmlFlag><![CDATA[1]]></htmlFlag>
##     <arnumber><![CDATA[6514500]]></arnumber>
##     <doi><![CDATA[10.1109/TGRS.2013.2245670]]></doi>
##     <publicationId><![CDATA[6514500]]></publicationId>
##     <mdurl><![CDATA[http://ieeexplore.ieee.org/xpl/articleDetails.jsp?tp=&arnumber=6514500&contentType=Journals+%26+Magazines]]></mdurl>
##     <pdf><![CDATA[http://ieeexplore.ieee.org/stamp/stamp.jsp?arnumber=6514500]]></pdf>
##   </document>
##   <document>
##     <rank>928</rank>
##     <title><![CDATA[Joint Audiovisual Hidden Semi-Markov Model-Based Speech Synthesis]]></title>
##     <authors><![CDATA[Schabus, D.;  Pucher, M.;  Hofer, G.]]></authors>
##     <affiliations><![CDATA[Telecommun. Res. Center Vienna (FTW), Vienna, Austria]]></affiliations>
##     <controlledterms>
##       <term><![CDATA[audio-visual systems]]></term>
##       <term><![CDATA[hidden Markov models]]></term>
##       <term><![CDATA[image recognition]]></term>
##       <term><![CDATA[natural language processing]]></term>
##       <term><![CDATA[natural languages]]></term>
##       <term><![CDATA[speech intelligibility]]></term>
##       <term><![CDATA[speech synthesis]]></term>
##       <term><![CDATA[synchronisation]]></term>
##       <term><![CDATA[training]]></term>
##     </controlledterms>
##     <thesaurusterms>
##       <term><![CDATA[Acoustics]]></term>
##       <term><![CDATA[Hidden Markov models]]></term>
##       <term><![CDATA[Joints]]></term>
##       <term><![CDATA[Speech]]></term>
##       <term><![CDATA[Synchronization]]></term>
##       <term><![CDATA[Training]]></term>
##       <term><![CDATA[Visualization]]></term>
##     </thesaurusterms>
##     <pubtitle><![CDATA[Selected Topics in Signal Processing, IEEE Journal of]]></pubtitle>
##     <punumber><![CDATA[4200690]]></punumber>
##     <pubtype><![CDATA[Journals & Magazines]]></pubtype>
##     <publisher><![CDATA[IEEE]]></publisher>
##     <volume><![CDATA[8]]></volume>
##     <issue><![CDATA[2]]></issue>
##     <py><![CDATA[2014]]></py>
##     <spage><![CDATA[336]]></spage>
##     <epage><![CDATA[347]]></epage>
##     <abstract><![CDATA[This paper investigates joint speaker-dependent audiovisual Hidden Semi-Markov Models (HSMM) where the visual models produce a sequence of 3D motion tracking data that is used to animate a talking head and the acoustic models are used for speech synthesis. Different acoustic, visual, and joint audiovisual models for four different Austrian German speakers were trained and we show that the joint models perform better compared to other approaches in terms of synchronization quality of the synthesized visual speech. In addition, a detailed analysis of the acoustic and visual alignment is provided for the different models. Importantly, the joint audiovisual modeling does not decrease the acoustic synthetic speech quality compared to acoustic-only modeling so that there is a clear advantage in the common duration model of the joint audiovisual modeling approach that is used for synchronizing acoustic and visual parameter sequences. Finally, it provides a model that integrates the visual and acoustic speech dynamics.]]></abstract>
##     <issn><![CDATA[1932-4553]]></issn>
##     <arnumber><![CDATA[6589946]]></arnumber>
##     <doi><![CDATA[10.1109/JSTSP.2013.2281036]]></doi>
##     <publicationId><![CDATA[6589946]]></publicationId>
##     <mdurl><![CDATA[http://ieeexplore.ieee.org/xpl/articleDetails.jsp?tp=&arnumber=6589946&contentType=Journals+%26+Magazines]]></mdurl>
##     <pdf><![CDATA[http://ieeexplore.ieee.org/stamp/stamp.jsp?arnumber=6589946]]></pdf>
##   </document>
##   <document>
##     <rank>929</rank>
##     <title><![CDATA[Direct Machining of Low-Loss THz Waveguide Components With an RF Choke]]></title>
##     <authors><![CDATA[Lewis, S.M.;  Nanni, E.A.;  Temkin, R.J.]]></authors>
##     <affiliations><![CDATA[Dept. of Nucl. Sci. & Eng., Massachusetts Inst. of Technol. (MIT), Cambridge, MA, USA]]></affiliations>
##     <controlledterms>
##       <term><![CDATA[copper]]></term>
##       <term><![CDATA[inductors]]></term>
##       <term><![CDATA[machining]]></term>
##       <term><![CDATA[millimetre wave directional couplers]]></term>
##       <term><![CDATA[parallel plate waveguides]]></term>
##     </controlledterms>
##     <thesaurusterms>
##       <term><![CDATA[Directional couplers]]></term>
##       <term><![CDATA[Fabrication]]></term>
##       <term><![CDATA[Machining]]></term>
##       <term><![CDATA[Millimeter wave technology]]></term>
##       <term><![CDATA[Optical device fabrication]]></term>
##       <term><![CDATA[Waveguide components]]></term>
##     </thesaurusterms>
##     <pubtitle><![CDATA[Microwave and Wireless Components Letters, IEEE]]></pubtitle>
##     <punumber><![CDATA[7260]]></punumber>
##     <pubtype><![CDATA[Journals & Magazines]]></pubtype>
##     <publisher><![CDATA[IEEE]]></publisher>
##     <volume><![CDATA[24]]></volume>
##     <issue><![CDATA[12]]></issue>
##     <py><![CDATA[2014]]></py>
##     <spage><![CDATA[842]]></spage>
##     <epage><![CDATA[844]]></epage>
##     <abstract><![CDATA[We present results for the successful fabrication of low-loss THz metallic waveguide components using direct machining with a CNC end mill. The approach uses a split-block machining process with the addition of an RF choke running parallel to the waveguide. The choke greatly reduces coupling to the parasitic TM0 mode of the parallel-plate waveguide produced by the split-block. This method has demonstrated loss as low as 0.2 dB/cm at 280 GHz for a copper WR-3 waveguide. It has also been used in the fabrication of 3 and 10 dB directional couplers in brass, demonstrating excellent agreement with design simulations from 240-260 GHz. The method may be adapted to structures with features on the order of 200 &#x03BC;m.]]></abstract>
##     <issn><![CDATA[1531-1309]]></issn>
##     <htmlFlag><![CDATA[1]]></htmlFlag>
##     <arnumber><![CDATA[6740086]]></arnumber>
##     <doi><![CDATA[10.1109/LMWC.2014.2303161]]></doi>
##     <publicationId><![CDATA[6740086]]></publicationId>
##     <mdurl><![CDATA[http://ieeexplore.ieee.org/xpl/articleDetails.jsp?tp=&arnumber=6740086&contentType=Journals+%26+Magazines]]></mdurl>
##     <pdf><![CDATA[http://ieeexplore.ieee.org/stamp/stamp.jsp?arnumber=6740086]]></pdf>
##   </document>
##   <document>
##     <rank>930</rank>
##     <title><![CDATA[Representation and Evolution of the Angular Momentum of the Light]]></title>
##     <authors><![CDATA[Martelli, P.;  Martinelli, M.]]></authors>
##     <affiliations><![CDATA[Dipt. di Elettron. Inf. e Bioingegneria, Politec. di Milano, Milan, Italy]]></affiliations>
##     <controlledterms>
##       <term><![CDATA[angular momentum]]></term>
##       <term><![CDATA[quantum optics]]></term>
##     </controlledterms>
##     <thesaurusterms>
##       <term><![CDATA[Dielectrics]]></term>
##       <term><![CDATA[Mirrors]]></term>
##       <term><![CDATA[Optical polarization]]></term>
##       <term><![CDATA[Optical propagation]]></term>
##       <term><![CDATA[Photonics]]></term>
##       <term><![CDATA[Polarization]]></term>
##       <term><![CDATA[Reflection]]></term>
##     </thesaurusterms>
##     <pubtitle><![CDATA[Photonics Journal, IEEE]]></pubtitle>
##     <punumber><![CDATA[4563994]]></punumber>
##     <pubtype><![CDATA[Journals & Magazines]]></pubtype>
##     <publisher><![CDATA[IEEE]]></publisher>
##     <volume><![CDATA[6]]></volume>
##     <issue><![CDATA[4]]></issue>
##     <py><![CDATA[2014]]></py>
##     <spage><![CDATA[1]]></spage>
##     <epage><![CDATA[9]]></epage>
##     <abstract><![CDATA[A compact formalism for representing the angular momentum modes in paraxial optics is considered. The evolution of both spin and orbital components of angular momentum is described within such representation. A thorough analysis of the effect of reflection is carried out, and different kinds of reflective interfaces are analyzed. A general method for building the matrices representing the transformations of the spin and orbital angular momenta is introduced and applied to some meaningful examples concerning wave plates, mirrors, and prisms.]]></abstract>
##     <issn><![CDATA[1943-0655]]></issn>
##     <htmlFlag><![CDATA[1]]></htmlFlag>
##     <arnumber><![CDATA[6842668]]></arnumber>
##     <doi><![CDATA[10.1109/JPHOT.2014.2332555]]></doi>
##     <publicationId><![CDATA[6842668]]></publicationId>
##     <mdurl><![CDATA[http://ieeexplore.ieee.org/xpl/articleDetails.jsp?tp=&arnumber=6842668&contentType=Journals+%26+Magazines]]></mdurl>
##     <pdf><![CDATA[http://ieeexplore.ieee.org/stamp/stamp.jsp?arnumber=6842668]]></pdf>
##   </document>
##   <document>
##     <rank>931</rank>
##     <title><![CDATA[The Structure and Quantum Capacity of a Partially Degradable Quantum Channel]]></title>
##     <authors><![CDATA[Gyongyosi, L.]]></authors>
##     <affiliations><![CDATA[Dept. of Telecommun., Budapest Univ. of Technol. & Econ., Budapest, Hungary]]></affiliations>
##     <controlledterms>
##       <term><![CDATA[channel capacity]]></term>
##       <term><![CDATA[quantum communication]]></term>
##     </controlledterms>
##     <thesaurusterms>
##       <term><![CDATA[Capacity planning]]></term>
##       <term><![CDATA[Channel capacity]]></term>
##       <term><![CDATA[Degradation]]></term>
##       <term><![CDATA[Quantum mechanics]]></term>
##       <term><![CDATA[Shannon theory]]></term>
##       <term><![CDATA[Simulation]]></term>
##     </thesaurusterms>
##     <pubtitle><![CDATA[Access, IEEE]]></pubtitle>
##     <punumber><![CDATA[6287639]]></punumber>
##     <pubtype><![CDATA[Journals & Magazines]]></pubtype>
##     <publisher><![CDATA[IEEE]]></publisher>
##     <volume><![CDATA[2]]></volume>
##     <py><![CDATA[2014]]></py>
##     <spage><![CDATA[333]]></spage>
##     <epage><![CDATA[355]]></epage>
##     <abstract><![CDATA[The quantum capacity of degradable quantum channels has been proven to be additive. On the other hand, there is no general rule for the behavior of quantum capacity for antidegradable quantum channels. We introduce the set of partially degradable (PD) quantum channels to answer the question of additivity of quantum capacity for a well-separable subset of antidegradable channels. A quantum channel is PD if the channel output can be used to simulate the degraded environment state. The PD channels could exist both in the degradable, antidegradable and conjugate degradable family. We define the term partial simulation, which is a clear benefit that arises from the structure of the complementary channel of a PD channel. We prove that the quantum capacity of an arbitrary dimensional PD channel is additive. We also demonstrate that better quantum data rates can be achieved over a PD channel in comparison with standard (non-PD) channels. Our results indicate that the partial degradability property can be exploited and yet still hold many benefits for quantum communications.]]></abstract>
##     <issn><![CDATA[2169-3536]]></issn>
##     <htmlFlag><![CDATA[1]]></htmlFlag>
##     <arnumber><![CDATA[6798658]]></arnumber>
##     <doi><![CDATA[10.1109/ACCESS.2014.2317652]]></doi>
##     <publicationId><![CDATA[6798658]]></publicationId>
##     <mdurl><![CDATA[http://ieeexplore.ieee.org/xpl/articleDetails.jsp?tp=&arnumber=6798658&contentType=Journals+%26+Magazines]]></mdurl>
##     <pdf><![CDATA[http://ieeexplore.ieee.org/stamp/stamp.jsp?arnumber=6798658]]></pdf>
##   </document>
##   <document>
##     <rank>932</rank>
##     <title><![CDATA[An Equivalent-Asymmetric Coupling Coefficient DFB Laser With High Output Efficiency and Stable Single Longitudinal Mode Operation]]></title>
##     <authors><![CDATA[Junshou Zheng;  Nannan Song;  Yunshan Zhang;  Yuechun Shi;  Song Tang;  Lianyan Li;  Renjia Guo;  Xiangfei Chen]]></authors>
##     <affiliations><![CDATA[Microwave Photonics Technol. Lab., Nanjing Univ., Nanjing, China]]></affiliations>
##     <controlledterms>
##       <term><![CDATA[Bragg gratings]]></term>
##       <term><![CDATA[diffraction gratings]]></term>
##       <term><![CDATA[distributed feedback lasers]]></term>
##       <term><![CDATA[laser modes]]></term>
##       <term><![CDATA[laser stability]]></term>
##       <term><![CDATA[optical couplers]]></term>
##       <term><![CDATA[optical hole burning]]></term>
##       <term><![CDATA[semiconductor lasers]]></term>
##     </controlledterms>
##     <thesaurusterms>
##       <term><![CDATA[Distributed feedback devices]]></term>
##       <term><![CDATA[Semiconductor lasers]]></term>
##     </thesaurusterms>
##     <pubtitle><![CDATA[Photonics Journal, IEEE]]></pubtitle>
##     <punumber><![CDATA[4563994]]></punumber>
##     <pubtype><![CDATA[Journals & Magazines]]></pubtype>
##     <publisher><![CDATA[IEEE]]></publisher>
##     <volume><![CDATA[6]]></volume>
##     <issue><![CDATA[6]]></issue>
##     <py><![CDATA[2014]]></py>
##     <spage><![CDATA[1]]></spage>
##     <epage><![CDATA[9]]></epage>
##     <abstract><![CDATA[An equivalent-asymmetric coupling coefficient (EACC) distributed feedback laser (DFB) semiconductor laser with equivalent-half apodization grating (EHAG) structure is proposed and experimentally demonstrated for the first time; the EHAG profile is equivalently realized by linearly changing the duty cycle of a sampled Bragg grating along the one half cavity, while that of the other half cavity is kept uniform. Compared with the equivalent-symmetric coupling coefficient DFB laser, the simulated intensity distribution of the EACC DFB laser shows that the light power is concentrated not only on near the phase-shift region and that the power at the front end is enlarged. Therefore, the longitudinal spatial hole burning may be reduced; the output efficiency and the single-mode stability may be improved. The experimental results show that the output power ratio between the front and rear facets is about 2.17 and the side-mode suppression ratios are over 50 dB when the injection current is in the range from 60 to 200 mA.]]></abstract>
##     <issn><![CDATA[1943-0655]]></issn>
##     <htmlFlag><![CDATA[1]]></htmlFlag>
##     <arnumber><![CDATA[6951415]]></arnumber>
##     <doi><![CDATA[10.1109/JPHOT.2014.2368776]]></doi>
##     <publicationId><![CDATA[6951415]]></publicationId>
##     <mdurl><![CDATA[http://ieeexplore.ieee.org/xpl/articleDetails.jsp?tp=&arnumber=6951415&contentType=Journals+%26+Magazines]]></mdurl>
##     <pdf><![CDATA[http://ieeexplore.ieee.org/stamp/stamp.jsp?arnumber=6951415]]></pdf>
##   </document>
##   <document>
##     <rank>933</rank>
##     <title><![CDATA[Design of a Disturbance Observer for a Two-Link Manipulator With Flexible Joints]]></title>
##     <authors><![CDATA[Jong Nam Yun;  Jian-Bo Su]]></authors>
##     <affiliations><![CDATA[Dept. of Autom., Shanghai Jiao Tong Univ., Shanghai, China]]></affiliations>
##     <controlledterms>
##       <term><![CDATA[control nonlinearities]]></term>
##       <term><![CDATA[flexible manipulators]]></term>
##       <term><![CDATA[manipulator dynamics]]></term>
##       <term><![CDATA[observers]]></term>
##       <term><![CDATA[optimal control]]></term>
##       <term><![CDATA[robust control]]></term>
##       <term><![CDATA[uncertain systems]]></term>
##     </controlledterms>
##     <pubtitle><![CDATA[Control Systems Technology, IEEE Transactions on]]></pubtitle>
##     <punumber><![CDATA[87]]></punumber>
##     <pubtype><![CDATA[Journals & Magazines]]></pubtype>
##     <publisher><![CDATA[IEEE]]></publisher>
##     <volume><![CDATA[22]]></volume>
##     <issue><![CDATA[2]]></issue>
##     <py><![CDATA[2014]]></py>
##     <spage><![CDATA[809]]></spage>
##     <epage><![CDATA[815]]></epage>
##     <abstract><![CDATA[This brief proposes a method for designing a disturbance observer (DOB) to decouple joint interactions in robot dynamics with nonlinearity. The traditional DOB based on filter design theory has limited performance since the cut-off frequency of its Q-filter is the only tunable parameter to deal with disturbance suppression and model uncertainty. In this brief, a robust optimal design method is developed for the DOB, which can achieve optimal performance of suppressing disturbance by systematically shaping its Q-filter. Simulation results of application to a two-link manipulator with flexible joints show the improvements in disturbance suppression, which illustrates the validity of the proposed method.]]></abstract>
##     <issn><![CDATA[1063-6536]]></issn>
##     <htmlFlag><![CDATA[1]]></htmlFlag>
##     <arnumber><![CDATA[6517273]]></arnumber>
##     <doi><![CDATA[10.1109/TCST.2013.2248733]]></doi>
##     <publicationId><![CDATA[6517273]]></publicationId>
##     <mdurl><![CDATA[http://ieeexplore.ieee.org/xpl/articleDetails.jsp?tp=&arnumber=6517273&contentType=Journals+%26+Magazines]]></mdurl>
##     <pdf><![CDATA[http://ieeexplore.ieee.org/stamp/stamp.jsp?arnumber=6517273]]></pdf>
##   </document>
##   <document>
##     <rank>934</rank>
##     <title><![CDATA[Neural Implementation of Shape-Invariant Touch Counter Based on Euler Calculus]]></title>
##     <authors><![CDATA[Miura, K.;  Nakada, K.]]></authors>
##     <affiliations><![CDATA[Grad. Sch. of Inf. Sci., Tohoku Univ., Sendai, Japan]]></affiliations>
##     <controlledterms>
##       <term><![CDATA[brain]]></term>
##       <term><![CDATA[calculus]]></term>
##       <term><![CDATA[field programmable gate arrays]]></term>
##       <term><![CDATA[integral equations]]></term>
##       <term><![CDATA[medical image processing]]></term>
##       <term><![CDATA[neurophysiology]]></term>
##       <term><![CDATA[parallel algorithms]]></term>
##       <term><![CDATA[touch sensitive screens]]></term>
##     </controlledterms>
##     <thesaurusterms>
##       <term><![CDATA[Algorithm design and analysis]]></term>
##       <term><![CDATA[Brain modeling]]></term>
##       <term><![CDATA[Calculus]]></term>
##       <term><![CDATA[Neuromorphic engineering]]></term>
##       <term><![CDATA[Object detection]]></term>
##       <term><![CDATA[Radiation detectors]]></term>
##       <term><![CDATA[Shape analysis]]></term>
##       <term><![CDATA[Sociology]]></term>
##       <term><![CDATA[Visualization]]></term>
##     </thesaurusterms>
##     <pubtitle><![CDATA[Access, IEEE]]></pubtitle>
##     <punumber><![CDATA[6287639]]></punumber>
##     <pubtype><![CDATA[Journals & Magazines]]></pubtype>
##     <publisher><![CDATA[IEEE]]></publisher>
##     <volume><![CDATA[2]]></volume>
##     <py><![CDATA[2014]]></py>
##     <spage><![CDATA[960]]></spage>
##     <epage><![CDATA[970]]></epage>
##     <abstract><![CDATA[One of the goals of neuromorphic engineering is to imitate the brain's ability to recognize and count the number of individual objects as entities based on the global consistency of the information from the population of activated tactile (or visual) sensory neurons whatever the objects' shapes are. To achieve this flexibility, it may be worth examining an unconventional algorithm such as topological methods. Here, we propose a fully parallelized algorithm for a shape-invariant touch counter for 2-D pixels. The number of touches is counted by the Euler integral, a generalized integral, in which a connected component counter (Betti number) for the binary image was used as elemental module. Through examples of touches, we demonstrate transparently how the proposed circuit architecture embodies the Euler integral in the form of recurrent neural networks for iterative vector operations. Our parallelization can lead the way to Field-Programmable Gate Array or Digital Signal Processor implementations of topological algorithms with scalability to high resolutions of pixels.]]></abstract>
##     <issn><![CDATA[2169-3536]]></issn>
##     <htmlFlag><![CDATA[1]]></htmlFlag>
##     <arnumber><![CDATA[6883116]]></arnumber>
##     <doi><![CDATA[10.1109/ACCESS.2014.2351832]]></doi>
##     <publicationId><![CDATA[6883116]]></publicationId>
##     <mdurl><![CDATA[http://ieeexplore.ieee.org/xpl/articleDetails.jsp?tp=&arnumber=6883116&contentType=Journals+%26+Magazines]]></mdurl>
##     <pdf><![CDATA[http://ieeexplore.ieee.org/stamp/stamp.jsp?arnumber=6883116]]></pdf>
##   </document>
##   <document>
##     <rank>935</rank>
##     <title><![CDATA[Effective Results Ranking for Mobile Query by Singing/Humming Using a Hybrid Recommendation Mechanism]]></title>
##     <authors><![CDATA[Ning-Han Liu]]></authors>
##     <affiliations><![CDATA[Manage. Inf. Syst. Dept., Nat. Pingtung Univ. of Sci. & Technol., Pingtung, Taiwan]]></affiliations>
##     <controlledterms>
##       <term><![CDATA[audio signal processing]]></term>
##       <term><![CDATA[mobile computing]]></term>
##       <term><![CDATA[music]]></term>
##       <term><![CDATA[query processing]]></term>
##       <term><![CDATA[recommender systems]]></term>
##     </controlledterms>
##     <thesaurusterms>
##       <term><![CDATA[Accuracy]]></term>
##       <term><![CDATA[Collaboration]]></term>
##       <term><![CDATA[Databases]]></term>
##       <term><![CDATA[History]]></term>
##       <term><![CDATA[Licenses]]></term>
##       <term><![CDATA[Recommender systems]]></term>
##     </thesaurusterms>
##     <pubtitle><![CDATA[Multimedia, IEEE Transactions on]]></pubtitle>
##     <punumber><![CDATA[6046]]></punumber>
##     <pubtype><![CDATA[Journals & Magazines]]></pubtype>
##     <publisher><![CDATA[IEEE]]></publisher>
##     <volume><![CDATA[16]]></volume>
##     <issue><![CDATA[5]]></issue>
##     <py><![CDATA[2014]]></py>
##     <spage><![CDATA[1407]]></spage>
##     <epage><![CDATA[1420]]></epage>
##     <abstract><![CDATA[When a user cannot remember the title of a song, or its related details, the most direct and convenient method to search for the song is by humming a section of it. This search method is particularly important when a user does not have access to operate the audio device. The design methodology used in conventional search mechanisms that query by singing/humming, commonly emphasize signal processing or music comparison. The background of the user often influences the genres of the songs being searched, and this is an area of research seldom studied. In our study, we use the information from a user's search history, as well as the properties of genres common to users with similar backgrounds, to estimate the genre or style the current user may be interested in based on a probability calculation. The accuracy from querying by singing/humming is improved. Our method can be divided into two phases. In the first phase, we find the possible search results. This is similar to the conventional singing/humming query process. During the second phase, the musical preference of the user is utilized to rank the possible search results again. Songs that are most likely to be queried would be positioned at the front of the list in the search results. Through our experiments, significant improvement is demonstrated with our method.]]></abstract>
##     <issn><![CDATA[1520-9210]]></issn>
##     <htmlFlag><![CDATA[1]]></htmlFlag>
##     <arnumber><![CDATA[6762948]]></arnumber>
##     <doi><![CDATA[10.1109/TMM.2014.2311326]]></doi>
##     <publicationId><![CDATA[6762948]]></publicationId>
##     <mdurl><![CDATA[http://ieeexplore.ieee.org/xpl/articleDetails.jsp?tp=&arnumber=6762948&contentType=Journals+%26+Magazines]]></mdurl>
##     <pdf><![CDATA[http://ieeexplore.ieee.org/stamp/stamp.jsp?arnumber=6762948]]></pdf>
##   </document>
##   <document>
##     <rank>936</rank>
##     <title><![CDATA[Non-Imaging Winston Cone Concentrators for Submillimeter-Wave, Overmoded Waveguide]]></title>
##     <authors><![CDATA[Grossman, E.N.;  Friedman, O.D.;  Nelson, A.O.]]></authors>
##     <affiliations><![CDATA[Nat. Inst. of Stand. & Technol., Boulder, CO, USA]]></affiliations>
##     <controlledterms>
##       <term><![CDATA[geometrical optics]]></term>
##       <term><![CDATA[laser beams]]></term>
##       <term><![CDATA[optical couplers]]></term>
##       <term><![CDATA[optical design techniques]]></term>
##       <term><![CDATA[optical waveguides]]></term>
##       <term><![CDATA[submillimetre wave devices]]></term>
##     </controlledterms>
##     <thesaurusterms>
##       <term><![CDATA[Apertures]]></term>
##       <term><![CDATA[Laser beams]]></term>
##       <term><![CDATA[Measurement by laser beam]]></term>
##       <term><![CDATA[Optical waveguides]]></term>
##       <term><![CDATA[Prototypes]]></term>
##       <term><![CDATA[Waveguide lasers]]></term>
##     </thesaurusterms>
##     <pubtitle><![CDATA[Terahertz Science and Technology, IEEE Transactions on]]></pubtitle>
##     <punumber><![CDATA[5503871]]></punumber>
##     <pubtype><![CDATA[Journals & Magazines]]></pubtype>
##     <publisher><![CDATA[IEEE]]></publisher>
##     <volume><![CDATA[4]]></volume>
##     <issue><![CDATA[1]]></issue>
##     <py><![CDATA[2014]]></py>
##     <spage><![CDATA[65]]></spage>
##     <epage><![CDATA[74]]></epage>
##     <abstract><![CDATA[We describe the design, simulation, and measured performance of concentrators designed to couple submillimeter wavelength radiation from free space into highly overmoded, rectangular, WR-10 waveguide. They consist of a combination of a Winston cone (also called a compound parabolic concentrator or CPC) with an adiabatic circular to rectangular transition. They are intended for use as adapters, between instruments using overmoded WR-10 waveguide as input or output and sources propagating through free space. Unlike conventional waveguide-coupled antennas, a geometric optics analysis is more appropriate than a mode-by-mode electromagnetic calculation of impedance and far-field pattern. Six separate designs were studied, with input diameters from 5 to 16 mm, and &#x201C;throat&#x201D; diameters (i.e., diameters at the circular interface between cone and transition sections) of 1 to 4 mm. Measurements at 394 &#x03BC;m wavelength (760 GHz) using a far-IR waveguide laser beam indicate efficiencies of 40%-55%. The angular response is primarily determined by the Winston cone, and is well predicted by geometric optics theory, i.e., approximately constant out to an angle determined by the ratio of input to throat diameters. The efficiencies are primarily determined by the transition section, and for all concentrators are consistent with an average reflectance of 94% from the gold-plated, electroformed, interior surfaces. For each individual concentrator, efficiency variations with polarization, angular orientation and beamsize are below the measurement uncertainty.]]></abstract>
##     <issn><![CDATA[2156-342X]]></issn>
##     <htmlFlag><![CDATA[1]]></htmlFlag>
##     <arnumber><![CDATA[6648689]]></arnumber>
##     <doi><![CDATA[10.1109/TTHZ.2013.2283371]]></doi>
##     <publicationId><![CDATA[6648689]]></publicationId>
##     <mdurl><![CDATA[http://ieeexplore.ieee.org/xpl/articleDetails.jsp?tp=&arnumber=6648689&contentType=Journals+%26+Magazines]]></mdurl>
##     <pdf><![CDATA[http://ieeexplore.ieee.org/stamp/stamp.jsp?arnumber=6648689]]></pdf>
##   </document>
##   <document>
##     <rank>937</rank>
##     <title><![CDATA[A Method for Attenuating the Spurious Responses of Aluminum Nitride Micromechanical Filters]]></title>
##     <authors><![CDATA[Olsson, R.H.;  Nguyen, J.;  Pluym, T.;  Hietala, V.M.]]></authors>
##     <affiliations><![CDATA[Sandia Nat. Labs., MEMS Technol., Albuquerque, NM, USA]]></affiliations>
##     <controlledterms>
##       <term><![CDATA[III-V semiconductors]]></term>
##       <term><![CDATA[UHF filters]]></term>
##       <term><![CDATA[acoustic resonator filters]]></term>
##       <term><![CDATA[aluminium compounds]]></term>
##       <term><![CDATA[micromechanical resonators]]></term>
##     </controlledterms>
##     <thesaurusterms>
##       <term><![CDATA[Aluminum nitride]]></term>
##       <term><![CDATA[Bandwidth]]></term>
##       <term><![CDATA[Electrodes]]></term>
##       <term><![CDATA[III-V semiconductor materials]]></term>
##       <term><![CDATA[Impedance]]></term>
##       <term><![CDATA[Microcavities]]></term>
##       <term><![CDATA[Resonant frequency]]></term>
##     </thesaurusterms>
##     <pubtitle><![CDATA[Microelectromechanical Systems, Journal of]]></pubtitle>
##     <punumber><![CDATA[84]]></punumber>
##     <pubtype><![CDATA[Journals & Magazines]]></pubtype>
##     <publisher><![CDATA[IEEE]]></publisher>
##     <volume><![CDATA[23]]></volume>
##     <issue><![CDATA[5]]></issue>
##     <py><![CDATA[2014]]></py>
##     <spage><![CDATA[1198]]></spage>
##     <epage><![CDATA[1207]]></epage>
##     <abstract><![CDATA[We present a method for attenuating the spurious responses in aluminum nitride micromechanical filters and demonstrate the technique in a 4-pole self-coupled filter operating at 494 MHz. In the standard implementation of a 4-pole self-coupled filter, each filter pole is realized using physically identical resonators. The spur mitigation approach reported here realizes the four poles of the filter using two different physical implementations of the resonator. Both resonators are designed to have identical responses at the desired resonant frequency of 494 MHz, while many of the spurious responses of the two resonators appear at nonidentical frequencies and do not add constructively at the filter output. Using the reported method, the measured attenuation of the largest filter spur is increased by 47.5 dB when compared with a 4-pole filter realized using identical resonators (standard approach) to form each filter pole. The filter realized using the reported spur attenuation approach has &gt;59.6 dBc of stopband and spurious response rejection over nearly a 2-GHz frequency span.]]></abstract>
##     <issn><![CDATA[1057-7157]]></issn>
##     <htmlFlag><![CDATA[1]]></htmlFlag>
##     <arnumber><![CDATA[6774862]]></arnumber>
##     <doi><![CDATA[10.1109/JMEMS.2014.2308544]]></doi>
##     <publicationId><![CDATA[6774862]]></publicationId>
##     <mdurl><![CDATA[http://ieeexplore.ieee.org/xpl/articleDetails.jsp?tp=&arnumber=6774862&contentType=Journals+%26+Magazines]]></mdurl>
##     <pdf><![CDATA[http://ieeexplore.ieee.org/stamp/stamp.jsp?arnumber=6774862]]></pdf>
##   </document>
##   <document>
##     <rank>938</rank>
##     <title><![CDATA[Introduction of Si PERC Rear Contacting Design to Boost Efficiency of Cu(In,Ga)Se<inline-formula> <img src="/images/tex/996.gif" alt="_{\bf 2}"> </inline-formula> Solar Cells]]></title>
##     <authors><![CDATA[Vermang, B.;  Watjen, J.T.;  Frisk, C.;  Fjallstrom, V.;  Rostvall, F.;  Edoff, M.;  Salome, P.;  Borme, J.;  Nicoara, N.;  Sadewasser, S.]]></authors>
##     <affiliations><![CDATA[Dept. of Eng. Sci., Uppsala Univ., Uppsala, Sweden]]></affiliations>
##     <controlledterms>
##       <term><![CDATA[aluminium compounds]]></term>
##       <term><![CDATA[copper compounds]]></term>
##       <term><![CDATA[electron beam lithography]]></term>
##       <term><![CDATA[gallium compounds]]></term>
##       <term><![CDATA[indium compounds]]></term>
##       <term><![CDATA[passivation]]></term>
##       <term><![CDATA[point contacts]]></term>
##       <term><![CDATA[short-circuit currents]]></term>
##       <term><![CDATA[solar cells]]></term>
##       <term><![CDATA[ternary semiconductors]]></term>
##     </controlledterms>
##     <thesaurusterms>
##       <term><![CDATA[Aluminum oxide]]></term>
##       <term><![CDATA[Lithography]]></term>
##       <term><![CDATA[Passivation]]></term>
##       <term><![CDATA[Photovoltaic cells]]></term>
##       <term><![CDATA[Photovoltaic systems]]></term>
##       <term><![CDATA[Silicon]]></term>
##       <term><![CDATA[Thin film devices]]></term>
##     </thesaurusterms>
##     <pubtitle><![CDATA[Photovoltaics, IEEE Journal of]]></pubtitle>
##     <punumber><![CDATA[5503869]]></punumber>
##     <pubtype><![CDATA[Journals & Magazines]]></pubtype>
##     <publisher><![CDATA[IEEE]]></publisher>
##     <volume><![CDATA[4]]></volume>
##     <issue><![CDATA[6]]></issue>
##     <py><![CDATA[2014]]></py>
##     <spage><![CDATA[1644]]></spage>
##     <epage><![CDATA[1649]]></epage>
##     <abstract><![CDATA[Recently, Cu(In,Ga)Se<sub>2</sub> (CIGS) solar cells have achieved 21% world-record efficiency, partly due to the introduction of a postdeposition potassium treatment to improve the front interface of CIGS absorber layers. However, as high-efficiency CIGS solar cells essentially require long diffusion lengths, the highly recombinative rear of these devices also deserves attention. In this paper, an Al<sub>2</sub>O<sub>3</sub> rear surface passivation layer with nanosized local point contacts is studied to reduce recombination at the standard Mo/CIGS rear interface. First, passivation layers with well-controlled grids of nanosized point openings are established by use of electron beam lithography. Next, rear-passivated CIGS solar cells with 240-nm-thick absorber layers are fabricated as study devices. These cells show an increase in open-circuit voltage (+57 mV), short-circuit current (+3.8 mA/cm<sup>2</sup>), and fill factor [9.5% (abs.)], compared with corresponding unpassivated reference cells, mainly due to improvements in rear surface passivation and rear internal reflection. Finally, solar cell capacitance simulator (SCAPS) modeling is used to calculate the effect of reduced back contact recombination on high-efficiency solar cells with standard absorber layer thickness. The modeling shows that up to 50-mV increase in open-circuit voltage is anticipated.]]></abstract>
##     <issn><![CDATA[2156-3381]]></issn>
##     <htmlFlag><![CDATA[1]]></htmlFlag>
##     <arnumber><![CDATA[6891123]]></arnumber>
##     <doi><![CDATA[10.1109/JPHOTOV.2014.2350696]]></doi>
##     <publicationId><![CDATA[6891123]]></publicationId>
##     <mdurl><![CDATA[http://ieeexplore.ieee.org/xpl/articleDetails.jsp?tp=&arnumber=6891123&contentType=Journals+%26+Magazines]]></mdurl>
##     <pdf><![CDATA[http://ieeexplore.ieee.org/stamp/stamp.jsp?arnumber=6891123]]></pdf>
##   </document>
##   <document>
##     <rank>939</rank>
##     <title><![CDATA[Fiber Optic Liquid Level Sensor Based on Integration of Lever Principle and Optical Interferometry]]></title>
##     <authors><![CDATA[Xinpu Zhang;  Wei Peng;  Zigeng Liu;  Zhenfeng Gong]]></authors>
##     <affiliations><![CDATA[Coll. of Phys. & Optoelectron. Eng., Dalian Univ. of Technol., Dalian, China]]></affiliations>
##     <controlledterms>
##       <term><![CDATA[Mach-Zehnder interferometers]]></term>
##       <term><![CDATA[fibre optic sensors]]></term>
##       <term><![CDATA[holey fibres]]></term>
##       <term><![CDATA[level measurement]]></term>
##       <term><![CDATA[organic compounds]]></term>
##       <term><![CDATA[photonic crystals]]></term>
##     </controlledterms>
##     <thesaurusterms>
##       <term><![CDATA[Liquids]]></term>
##       <term><![CDATA[Optical fiber sensors]]></term>
##       <term><![CDATA[Optical fibers]]></term>
##       <term><![CDATA[Sensitivity]]></term>
##       <term><![CDATA[Splicing]]></term>
##     </thesaurusterms>
##     <pubtitle><![CDATA[Photonics Journal, IEEE]]></pubtitle>
##     <punumber><![CDATA[4563994]]></punumber>
##     <pubtype><![CDATA[Journals & Magazines]]></pubtype>
##     <publisher><![CDATA[IEEE]]></publisher>
##     <volume><![CDATA[6]]></volume>
##     <issue><![CDATA[2]]></issue>
##     <py><![CDATA[2014]]></py>
##     <spage><![CDATA[1]]></spage>
##     <epage><![CDATA[7]]></epage>
##     <abstract><![CDATA[We present a fiber-optic liquid level sensor that is conducted by a combination of optical interferometry and lever principle. The sensing unit is a Mach-Zehnder interferometer (MZI), which is formed by sandwiching a piece of photonic crystal fiber (PCF) between two single-mode fibers (SMFs). The measuring equipment is composed of a rotatable lever and a fixed link. The rotatable lever includes two different length arms, i.e., L<sub>1</sub> and L<sub>2</sub>. Both ends of the MZI are glued on the tip of the L<sub>2</sub> arm and the fixed link using ethoxyline, respectively. A hanging stick, which is dipped into a liquid tank, is directly mounted on the other end of the rotatable lever. The buoyancy will increase as the stick depth of immersion into the liquid increases. The tension the MZI subjected will increase according to the proportion of L<sub>1</sub>/L<sub>2</sub> on account of the lever principle. The sensitivity of the sensor could be regulated with different ratios of lever arms. In our experiment, a maximum sensitivity of 111.27 pm/mm was obtained with a 1 : 7.8 ratio of two lever arms L<sub>1</sub>/L<sub>2</sub>. The demonstrated liquid level sensor has the advantages of simple structure, easy fabrication, low cost, and high sensitivity.]]></abstract>
##     <issn><![CDATA[1943-0655]]></issn>
##     <htmlFlag><![CDATA[1]]></htmlFlag>
##     <arnumber><![CDATA[6758340]]></arnumber>
##     <doi><![CDATA[10.1109/JPHOT.2014.2310208]]></doi>
##     <publicationId><![CDATA[6758340]]></publicationId>
##     <mdurl><![CDATA[http://ieeexplore.ieee.org/xpl/articleDetails.jsp?tp=&arnumber=6758340&contentType=Journals+%26+Magazines]]></mdurl>
##     <pdf><![CDATA[http://ieeexplore.ieee.org/stamp/stamp.jsp?arnumber=6758340]]></pdf>
##   </document>
##   <document>
##     <rank>940</rank>
##     <title><![CDATA[Joint Optimization of Transmission-Order Selection and Channel Allocation for Bidirectional Wireless Links&#x2014;Part II: Algorithms]]></title>
##     <authors><![CDATA[Uykan, Z.;  Jantti, R.]]></authors>
##     <affiliations><![CDATA[Dept. of Control & Autom. Eng., Dogus Univ., Istanbul, Turkey]]></affiliations>
##     <controlledterms>
##       <term><![CDATA[cellular radio]]></term>
##       <term><![CDATA[channel allocation]]></term>
##       <term><![CDATA[distributed algorithms]]></term>
##       <term><![CDATA[optimisation]]></term>
##     </controlledterms>
##     <thesaurusterms>
##       <term><![CDATA[Algorithm design and analysis]]></term>
##       <term><![CDATA[Games]]></term>
##       <term><![CDATA[Indexes]]></term>
##       <term><![CDATA[Interference]]></term>
##       <term><![CDATA[Joints]]></term>
##       <term><![CDATA[Optimization]]></term>
##       <term><![CDATA[Wireless communication]]></term>
##     </thesaurusterms>
##     <pubtitle><![CDATA[Wireless Communications, IEEE Transactions on]]></pubtitle>
##     <punumber><![CDATA[7693]]></punumber>
##     <pubtype><![CDATA[Journals & Magazines]]></pubtype>
##     <publisher><![CDATA[IEEE]]></publisher>
##     <volume><![CDATA[13]]></volume>
##     <issue><![CDATA[7]]></issue>
##     <py><![CDATA[2014]]></py>
##     <spage><![CDATA[3991]]></spage>
##     <epage><![CDATA[4002]]></epage>
##     <abstract><![CDATA[This is the second in a two-part series of papers on transmission order (TO) optimization in the presence of channel allocation (CA), i.e., joint optimization of the TO selection and CA problem, for interfering bidirectional wireless links. Part I of this paper thoroughly analyzes the joint optimization problem from a game theoretic perspective for a general deterministic setting. Here in Part II, we present novel distributed and centralized CA-TO algorithms, together with their performance analysis, for Device-to-Device (D2D) communications underlaying cellular networks based on the findings in Part I of this paper. Here, TO is a novel dimension for optimization. In Part II, we propose and analyze novel two distributed and one centralized joint CA-TO algorithms. Our investigations show that: i) our algorithms contain many of the existing TO algorithms and CA algorithms as its special cases and can thus be considered as a general framework for the joint CA and TO optimization. The computer simulations for TDD-based D2D communications underlaying cellular network show that the proposed distributed and centralized joint CA-TO algorithms remarkably outperform the reference algorithms.]]></abstract>
##     <issn><![CDATA[1536-1276]]></issn>
##     <htmlFlag><![CDATA[1]]></htmlFlag>
##     <arnumber><![CDATA[6783729]]></arnumber>
##     <doi><![CDATA[10.1109/TWC.2014.2314639]]></doi>
##     <publicationId><![CDATA[6783729]]></publicationId>
##     <mdurl><![CDATA[http://ieeexplore.ieee.org/xpl/articleDetails.jsp?tp=&arnumber=6783729&contentType=Journals+%26+Magazines]]></mdurl>
##     <pdf><![CDATA[http://ieeexplore.ieee.org/stamp/stamp.jsp?arnumber=6783729]]></pdf>
##   </document>
##   <document>
##     <rank>941</rank>
##     <title><![CDATA[Bi-Velocity Discrete Particle Swarm Optimization and Its Application to Multicast Routing Problem in Communication Networks]]></title>
##     <authors><![CDATA[Meie Shen;  Zhi-Hui Zhan;  Wei-Neng Chen;  Yue-Jiao Gong;  Jun Zhang;  Yun Li]]></authors>
##     <affiliations><![CDATA[Sch. of Comput. Sci., Beijing Inf. Sci. & Technol. Univ., Beijing, China]]></affiliations>
##     <controlledterms>
##       <term><![CDATA[computational complexity]]></term>
##       <term><![CDATA[multicast communication]]></term>
##       <term><![CDATA[particle swarm optimisation]]></term>
##       <term><![CDATA[polynomials]]></term>
##       <term><![CDATA[telecommunication network routing]]></term>
##       <term><![CDATA[trees (mathematics)]]></term>
##     </controlledterms>
##     <thesaurusterms>
##       <term><![CDATA[Communication networks]]></term>
##       <term><![CDATA[Multicast communication]]></term>
##       <term><![CDATA[Particle swarm optimization]]></term>
##       <term><![CDATA[Routing]]></term>
##       <term><![CDATA[Steiner trees]]></term>
##     </thesaurusterms>
##     <pubtitle><![CDATA[Industrial Electronics, IEEE Transactions on]]></pubtitle>
##     <punumber><![CDATA[41]]></punumber>
##     <pubtype><![CDATA[Journals & Magazines]]></pubtype>
##     <publisher><![CDATA[IEEE]]></publisher>
##     <volume><![CDATA[61]]></volume>
##     <issue><![CDATA[12]]></issue>
##     <py><![CDATA[2014]]></py>
##     <spage><![CDATA[7141]]></spage>
##     <epage><![CDATA[7151]]></epage>
##     <abstract><![CDATA[This paper proposes a novel bi-velocity discrete particle swarm optimization (BVDPSO) approach and extends its application to the nondeterministic polynomial (NP) complete multicast routing problem (MRP). The main contribution is the extension of particle swarm optimization (PSO) from the continuous domain to the binary or discrete domain. First, a novel bi-velocity strategy is developed to represent the possibilities of each dimension being 1 and 0. This strategy is suitable to describe the binary characteristic of the MRP, where 1 stands for a node being selected to construct the multicast tree, whereas 0 stands for being otherwise. Second, BVDPSO updates the velocity and position according to the learning mechanism of the original PSO in the continuous domain. This maintains the fast convergence speed and global search ability of the original PSO. Experiments are comprehensively conducted on all of the 58 instances with small, medium, and large scales in the Operation Research Library (OR-library). The results confirm that BVDPSO can obtain optimal or near-optimal solutions rapidly since it only needs to generate a few multicast trees. BVDPSO outperforms not only several state-of-the-art and recent heuristic algorithms for the MRP problems, but also algorithms based on genetic algorithms, ant colony optimization, and PSO.]]></abstract>
##     <issn><![CDATA[0278-0046]]></issn>
##     <htmlFlag><![CDATA[1]]></htmlFlag>
##     <arnumber><![CDATA[6779598]]></arnumber>
##     <doi><![CDATA[10.1109/TIE.2014.2314075]]></doi>
##     <publicationId><![CDATA[6779598]]></publicationId>
##     <mdurl><![CDATA[http://ieeexplore.ieee.org/xpl/articleDetails.jsp?tp=&arnumber=6779598&contentType=Journals+%26+Magazines]]></mdurl>
##     <pdf><![CDATA[http://ieeexplore.ieee.org/stamp/stamp.jsp?arnumber=6779598]]></pdf>
##   </document>
##   <document>
##     <rank>942</rank>
##     <title><![CDATA[A High-Speed Second-Order Photonic Differentiator Based on Two-Stage Silicon Self-Coupled Optical Waveguide]]></title>
##     <authors><![CDATA[Liang Zhang;  Jiayang Wu;  Xi Yin;  Xiaowen Sun;  Pan Cao;  Xinhong Jiang;  Yikai Su]]></authors>
##     <affiliations><![CDATA[Dept. of Electron. Eng., Shanghai Jiao Tong Univ., Shanghai, China]]></affiliations>
##     <controlledterms>
##       <term><![CDATA[elemental semiconductors]]></term>
##       <term><![CDATA[high-speed optical techniques]]></term>
##       <term><![CDATA[integrated optics]]></term>
##       <term><![CDATA[integrated optoelectronics]]></term>
##       <term><![CDATA[multiplexing]]></term>
##       <term><![CDATA[notch filters]]></term>
##       <term><![CDATA[optical fabrication]]></term>
##       <term><![CDATA[optical filters]]></term>
##       <term><![CDATA[optical waveguides]]></term>
##       <term><![CDATA[silicon-on-insulator]]></term>
##     </controlledterms>
##     <thesaurusterms>
##       <term><![CDATA[High-speed optical techniques]]></term>
##       <term><![CDATA[Optical amplifiers]]></term>
##       <term><![CDATA[Optical fibers]]></term>
##       <term><![CDATA[Optical filters]]></term>
##       <term><![CDATA[Optical resonators]]></term>
##       <term><![CDATA[Photonics]]></term>
##     </thesaurusterms>
##     <pubtitle><![CDATA[Photonics Journal, IEEE]]></pubtitle>
##     <punumber><![CDATA[4563994]]></punumber>
##     <pubtype><![CDATA[Journals & Magazines]]></pubtype>
##     <publisher><![CDATA[IEEE]]></publisher>
##     <volume><![CDATA[6]]></volume>
##     <issue><![CDATA[2]]></issue>
##     <py><![CDATA[2014]]></py>
##     <spage><![CDATA[1]]></spage>
##     <epage><![CDATA[5]]></epage>
##     <abstract><![CDATA[In this paper, we propose and demonstrate an all-optical second-order differentiator based on a two-stage self-coupled optical waveguide on a silicon-on-insulator platform. The transmission spectrum of the fabricated device has a parabola-like filtering notch with a 3-dB bandwidth of up to ~ 100 GHz and a depth of ~ 12 dB. Experiments are carried out for 10-, 20-, and 40-Gb/s optical time-domain multiplexing (OTDM) picosecond pulse trains, and the second-order differentiations are achieved using the fabricated device.]]></abstract>
##     <issn><![CDATA[1943-0655]]></issn>
##     <htmlFlag><![CDATA[1]]></htmlFlag>
##     <arnumber><![CDATA[6728605]]></arnumber>
##     <doi><![CDATA[10.1109/JPHOT.2014.2302806]]></doi>
##     <publicationId><![CDATA[6728605]]></publicationId>
##     <mdurl><![CDATA[http://ieeexplore.ieee.org/xpl/articleDetails.jsp?tp=&arnumber=6728605&contentType=Journals+%26+Magazines]]></mdurl>
##     <pdf><![CDATA[http://ieeexplore.ieee.org/stamp/stamp.jsp?arnumber=6728605]]></pdf>
##   </document>
##   <document>
##     <rank>943</rank>
##     <title><![CDATA[Unconstrained Video Monitoring of Breathing Behavior and Application to Diagnosis of Sleep Apnea]]></title>
##     <authors><![CDATA[Ching-Wei Wang;  Hunter, A.;  Gravill, N.;  Matusiewicz, S.]]></authors>
##     <affiliations><![CDATA[Grad. Inst. of Biomed. Eng., Nat. Taiwan Univ. of Sci. & Technol., Taipei, Taiwan]]></affiliations>
##     <controlledterms>
##       <term><![CDATA[infrared imaging]]></term>
##       <term><![CDATA[medical disorders]]></term>
##       <term><![CDATA[patient diagnosis]]></term>
##       <term><![CDATA[patient monitoring]]></term>
##       <term><![CDATA[pneumodynamics]]></term>
##     </controlledterms>
##     <thesaurusterms>
##       <term><![CDATA[Adaptation models]]></term>
##       <term><![CDATA[Cameras]]></term>
##       <term><![CDATA[Monitoring]]></term>
##       <term><![CDATA[Motion detection]]></term>
##       <term><![CDATA[Noise]]></term>
##       <term><![CDATA[Sensors]]></term>
##       <term><![CDATA[Sleep apnea]]></term>
##     </thesaurusterms>
##     <pubtitle><![CDATA[Biomedical Engineering, IEEE Transactions on]]></pubtitle>
##     <punumber><![CDATA[10]]></punumber>
##     <pubtype><![CDATA[Journals & Magazines]]></pubtype>
##     <publisher><![CDATA[IEEE]]></publisher>
##     <volume><![CDATA[61]]></volume>
##     <issue><![CDATA[2]]></issue>
##     <py><![CDATA[2014]]></py>
##     <spage><![CDATA[396]]></spage>
##     <epage><![CDATA[404]]></epage>
##     <abstract><![CDATA[This paper presents a new real-time automated infrared video monitoring technique for detection of breathing anomalies, and its application in the diagnosis of obstructive sleep apnea. We introduce a novel motion model to detect subtle, cyclical breathing signals from video, a new 3-D unsupervised self-adaptive breathing template to learn individuals' normal breathing patterns online, and a robust action classification method to recognize abnormal breathing activities and limb movements. This technique avoids imposing positional constraints on the patient, allowing patients to sleep on their back or side, with or without facing the camera, fully or partially occluded by the bed clothes. Moreover, shallow and abdominal breathing patterns do not adversely affect the performance of the method, and it is insensitive to environmental settings such as infrared lighting levels and camera view angles. The experimental results show that the technique achieves high accuracy (94% for the clinical data) in recognizing apnea episodes and body movements and is robust to various occlusion levels, body poses, body movements (i.e., minor head movement, limb movement, body rotation, and slight torso movement), and breathing behavior (e.g., shallow versus heavy breathing, mouth breathing, chest breathing, and abdominal breathing).]]></abstract>
##     <issn><![CDATA[0018-9294]]></issn>
##     <htmlFlag><![CDATA[1]]></htmlFlag>
##     <arnumber><![CDATA[6587794]]></arnumber>
##     <doi><![CDATA[10.1109/TBME.2013.2280132]]></doi>
##     <publicationId><![CDATA[6587794]]></publicationId>
##     <mdurl><![CDATA[http://ieeexplore.ieee.org/xpl/articleDetails.jsp?tp=&arnumber=6587794&contentType=Journals+%26+Magazines]]></mdurl>
##     <pdf><![CDATA[http://ieeexplore.ieee.org/stamp/stamp.jsp?arnumber=6587794]]></pdf>
##   </document>
##   <document>
##     <rank>944</rank>
##     <title><![CDATA[Semi-supervised Domain Adaptation on Manifolds]]></title>
##     <authors><![CDATA[Li Cheng;  Sinno Jialin Pan]]></authors>
##     <affiliations><![CDATA[Bioinf. Inst., A*STAR, Singapore, Singapore]]></affiliations>
##     <controlledterms>
##       <term><![CDATA[data handling]]></term>
##       <term><![CDATA[iterative methods]]></term>
##       <term><![CDATA[learning (artificial intelligence)]]></term>
##       <term><![CDATA[matrix algebra]]></term>
##       <term><![CDATA[optimisation]]></term>
##     </controlledterms>
##     <thesaurusterms>
##       <term><![CDATA[Covariance matrices]]></term>
##       <term><![CDATA[Laplace equations]]></term>
##       <term><![CDATA[Learning systems]]></term>
##       <term><![CDATA[Manifolds]]></term>
##       <term><![CDATA[Object recognition]]></term>
##       <term><![CDATA[Optimization]]></term>
##       <term><![CDATA[Vectors]]></term>
##     </thesaurusterms>
##     <pubtitle><![CDATA[Neural Networks and Learning Systems, IEEE Transactions on]]></pubtitle>
##     <punumber><![CDATA[5962385]]></punumber>
##     <pubtype><![CDATA[Journals & Magazines]]></pubtype>
##     <publisher><![CDATA[IEEE]]></publisher>
##     <volume><![CDATA[25]]></volume>
##     <issue><![CDATA[12]]></issue>
##     <py><![CDATA[2014]]></py>
##     <spage><![CDATA[2240]]></spage>
##     <epage><![CDATA[2249]]></epage>
##     <abstract><![CDATA[In real-life problems, the following semi-supervised domain adaptation scenario is often encountered: we have full access to some source data, which is usually very large; the target data distribution is under certain unknown transformation of the source data distribution; meanwhile, only a small fraction of the target instances come with labels. The goal is to learn a prediction model by incorporating information from the source domain that is able to generalize well on the target test instances. We consider an explicit form of transformation functions and especially linear transformations that maps examples from the source to the target domain, and we argue that by proper preprocessing of the data from both source and target domains, the feasible transformation functions can be characterized by a set of rotation matrices. This naturally leads to an optimization formulation under the special orthogonal group constraints. We present an iterative coordinate descent solver that is able to jointly learn the transformation as well as the model parameters, while the geodesic update ensures the manifold constraints are always satisfied. Our framework is sufficiently general to work with a variety of loss functions and prediction problems. Empirical evaluations on synthetic and real-world experiments demonstrate the competitive performance of our method with respect to the state-of-the-art.]]></abstract>
##     <issn><![CDATA[2162-237X]]></issn>
##     <htmlFlag><![CDATA[1]]></htmlFlag>
##     <arnumber><![CDATA[6774431]]></arnumber>
##     <doi><![CDATA[10.1109/TNNLS.2014.2308325]]></doi>
##     <publicationId><![CDATA[6774431]]></publicationId>
##     <mdurl><![CDATA[http://ieeexplore.ieee.org/xpl/articleDetails.jsp?tp=&arnumber=6774431&contentType=Journals+%26+Magazines]]></mdurl>
##     <pdf><![CDATA[http://ieeexplore.ieee.org/stamp/stamp.jsp?arnumber=6774431]]></pdf>
##   </document>
##   <document>
##     <rank>945</rank>
##     <title><![CDATA[Graph-Based Multicell Scheduling in OFDMA-Based Small Cell Networks]]></title>
##     <authors><![CDATA[Pateromichelakis, E.;  Shariat, M.;  Quddus, A.U.;  Tafazolli, R.]]></authors>
##     <affiliations><![CDATA[Dept. of Electron. Eng., Univ. of Surrey, Guildford, UK]]></affiliations>
##     <controlledterms>
##       <term><![CDATA[cellular radio]]></term>
##       <term><![CDATA[frequency division multiple access]]></term>
##       <term><![CDATA[graph theory]]></term>
##       <term><![CDATA[optimisation]]></term>
##       <term><![CDATA[radiofrequency interference]]></term>
##       <term><![CDATA[resource allocation]]></term>
##       <term><![CDATA[scheduling]]></term>
##       <term><![CDATA[telecommunication network reliability]]></term>
##     </controlledterms>
##     <thesaurusterms>
##       <term><![CDATA[Clustering algorithms]]></term>
##       <term><![CDATA[Complexity theory]]></term>
##       <term><![CDATA[Downlink]]></term>
##       <term><![CDATA[Graph theory]]></term>
##       <term><![CDATA[Heuristic algorithms]]></term>
##       <term><![CDATA[Interference]]></term>
##       <term><![CDATA[Multi-cell scheduling]]></term>
##       <term><![CDATA[Resource management]]></term>
##       <term><![CDATA[Scheduling]]></term>
##     </thesaurusterms>
##     <pubtitle><![CDATA[Access, IEEE]]></pubtitle>
##     <punumber><![CDATA[6287639]]></punumber>
##     <pubtype><![CDATA[Journals & Magazines]]></pubtype>
##     <publisher><![CDATA[IEEE]]></publisher>
##     <volume><![CDATA[2]]></volume>
##     <py><![CDATA[2014]]></py>
##     <spage><![CDATA[897]]></spage>
##     <epage><![CDATA[908]]></epage>
##     <abstract><![CDATA[This paper proposes a novel graph-based multicell scheduling framework to efficiently mitigate downlink intercell interference in OFDMA-based small cell networks. We define a graph-based optimization framework based on interference condition between any two users in the network assuming they are served on similar resources. Furthermore, we prove that the proposed framework obtains a tight lower bound for conventional weighted sum-rate maximization problem in practical scenarios. Thereafter, we decompose the optimization problem into dynamic graph-partitioning-based subproblems across different subchannels and provide an optimal solution using branch-and-cut approach. Subsequently, due to high complexity of the solution, we propose heuristic algorithms that display near optimal performance. At the final stage, we apply cluster-based resource allocation per subchannel to find candidate users with maximum total weighted sum-rate. A case study on networked small cells is also presented with simulation results showing a significant improvement over the state-of-the-art multicell scheduling benchmarks in terms of outage probability as well as average cell throughput.]]></abstract>
##     <issn><![CDATA[2169-3536]]></issn>
##     <htmlFlag><![CDATA[1]]></htmlFlag>
##     <arnumber><![CDATA[6881614]]></arnumber>
##     <doi><![CDATA[10.1109/ACCESS.2014.2350556]]></doi>
##     <publicationId><![CDATA[6881614]]></publicationId>
##     <mdurl><![CDATA[http://ieeexplore.ieee.org/xpl/articleDetails.jsp?tp=&arnumber=6881614&contentType=Journals+%26+Magazines]]></mdurl>
##     <pdf><![CDATA[http://ieeexplore.ieee.org/stamp/stamp.jsp?arnumber=6881614]]></pdf>
##   </document>
##   <document>
##     <rank>946</rank>
##     <title><![CDATA[A Widely Tunable Directly Modulated DBR Laser With High Linearity]]></title>
##     <authors><![CDATA[Liqiang Yu;  Huitao Wang;  Dan Lu;  Song Liang;  Can Zhang;  Biwei Pan;  Limeng Zhang;  Lingjuan Zhao]]></authors>
##     <affiliations><![CDATA[Key Lab. of Semicond. Mater. Sci., Inst. of Semicond., Beijing, China]]></affiliations>
##     <controlledterms>
##       <term><![CDATA[diffraction gratings]]></term>
##       <term><![CDATA[distributed Bragg reflector lasers]]></term>
##       <term><![CDATA[laser tuning]]></term>
##       <term><![CDATA[optical communication equipment]]></term>
##       <term><![CDATA[optical modulation]]></term>
##       <term><![CDATA[passive optical networks]]></term>
##       <term><![CDATA[semiconductor lasers]]></term>
##     </controlledterms>
##     <thesaurusterms>
##       <term><![CDATA[Bandwidth]]></term>
##       <term><![CDATA[Distributed Bragg reflectors]]></term>
##       <term><![CDATA[Gain]]></term>
##       <term><![CDATA[Laser tuning]]></term>
##       <term><![CDATA[Materials]]></term>
##       <term><![CDATA[Modulation]]></term>
##     </thesaurusterms>
##     <pubtitle><![CDATA[Photonics Journal, IEEE]]></pubtitle>
##     <punumber><![CDATA[4563994]]></punumber>
##     <pubtype><![CDATA[Journals & Magazines]]></pubtype>
##     <publisher><![CDATA[IEEE]]></publisher>
##     <volume><![CDATA[6]]></volume>
##     <issue><![CDATA[4]]></issue>
##     <py><![CDATA[2014]]></py>
##     <spage><![CDATA[1]]></spage>
##     <epage><![CDATA[8]]></epage>
##     <abstract><![CDATA[We demonstrate a widely tunable and high-modulation-bandwidth distributed Bragg reflector (DBR) laser. With the selected butt-joint material (&#x03BB;g = 1.44 &#x03BC;m) and butt-joint angle (45&#x00B0;) for the grating section, a widely tuning range of 13.88 nm with 19 consecutive channels is obtained for the two-section DBR laser. Furthermore, a small-signal modulation bandwidth of 16 GHz with in-band flatness of &#x00B1;3 dB is realized at 10 &#x00B0;C. A uniform modulation bandwidth of more than 13 GHz is obtained for ten consecutive channels at 25 &#x00B0;C. Meanwhile, the DBR laser shows a high linearity, with an input 1-dB compression point of 17 dBm and an input third-order intercept point of 30 dBm at 13 GHz.]]></abstract>
##     <issn><![CDATA[1943-0655]]></issn>
##     <htmlFlag><![CDATA[1]]></htmlFlag>
##     <arnumber><![CDATA[6838951]]></arnumber>
##     <doi><![CDATA[10.1109/JPHOT.2014.2331247]]></doi>
##     <publicationId><![CDATA[6838951]]></publicationId>
##     <mdurl><![CDATA[http://ieeexplore.ieee.org/xpl/articleDetails.jsp?tp=&arnumber=6838951&contentType=Journals+%26+Magazines]]></mdurl>
##     <pdf><![CDATA[http://ieeexplore.ieee.org/stamp/stamp.jsp?arnumber=6838951]]></pdf>
##   </document>
##   <document>
##     <rank>947</rank>
##     <title><![CDATA[Spatial Modulation for Generalized MIMO: Challenges, Opportunities, and Implementation]]></title>
##     <authors><![CDATA[Di Renzo, M.;  Haas, H.;  Ghrayeb, A.;  Sugiura, S.;  Hanzo, L.]]></authors>
##     <affiliations><![CDATA[Lab. of Signals & Syst. (L2S), Univ. of Paris-Sud XI, Gif-sur-Yvette, France]]></affiliations>
##     <controlledterms>
##       <term><![CDATA[MIMO communication]]></term>
##       <term><![CDATA[cellular radio]]></term>
##       <term><![CDATA[energy conservation]]></term>
##       <term><![CDATA[modulation]]></term>
##       <term><![CDATA[next generation networks]]></term>
##     </controlledterms>
##     <thesaurusterms>
##       <term><![CDATA[MIMO]]></term>
##       <term><![CDATA[Modulation]]></term>
##       <term><![CDATA[Spatial resolution]]></term>
##       <term><![CDATA[Tutorials]]></term>
##     </thesaurusterms>
##     <pubtitle><![CDATA[Proceedings of the IEEE]]></pubtitle>
##     <punumber><![CDATA[5]]></punumber>
##     <pubtype><![CDATA[Journals & Magazines]]></pubtype>
##     <publisher><![CDATA[IEEE]]></publisher>
##     <volume><![CDATA[102]]></volume>
##     <issue><![CDATA[1]]></issue>
##     <py><![CDATA[2014]]></py>
##     <spage><![CDATA[56]]></spage>
##     <epage><![CDATA[103]]></epage>
##     <abstract><![CDATA[A key challenge of future mobile communication research is to strike an attractive compromise between wireless network's area spectral efficiency and energy efficiency. This necessitates a clean-slate approach to wireless system design, embracing the rich body of existing knowledge, especially on multiple-input-multiple-ouput (MIMO) technologies. This motivates the proposal of an emerging wireless communications concept conceived for single-radio-frequency (RF) large-scale MIMO communications, which is termed as SM. The concept of SM has established itself as a beneficial transmission paradigm, subsuming numerous members of the MIMO system family. The research of SM has reached sufficient maturity to motivate its comparison to state-of-the-art MIMO communications, as well as to inspire its application to other emerging wireless systems such as relay-aided, cooperative, small-cell, optical wireless, and power-efficient communications. Furthermore, it has received sufficient research attention to be implemented in testbeds, and it holds the promise of stimulating further vigorous interdisciplinary research in the years to come. This tutorial paper is intended to offer a comprehensive state-of-the-art survey on SM-MIMO research, to provide a critical appraisal of its potential advantages, and to promote the discussion of its beneficial application areas and their research challenges leading to the analysis of the technological issues associated with the implementation of SM-MIMO. The paper is concluded with the description of the world's first experimental activities in this vibrant research field.]]></abstract>
##     <issn><![CDATA[0018-9219]]></issn>
##     <htmlFlag><![CDATA[1]]></htmlFlag>
##     <arnumber><![CDATA[6678765]]></arnumber>
##     <doi><![CDATA[10.1109/JPROC.2013.2287851]]></doi>
##     <publicationId><![CDATA[6678765]]></publicationId>
##     <mdurl><![CDATA[http://ieeexplore.ieee.org/xpl/articleDetails.jsp?tp=&arnumber=6678765&contentType=Journals+%26+Magazines]]></mdurl>
##     <pdf><![CDATA[http://ieeexplore.ieee.org/stamp/stamp.jsp?arnumber=6678765]]></pdf>
##   </document>
##   <document>
##     <rank>948</rank>
##     <title><![CDATA[High Temperature Measurement Up to 1100 <formula formulatype="inline"> <img src="/images/tex/18628.gif" alt="^{\circ} \hbox {C}"> </formula> Using a Polarization-Maintaining Photonic Crystal Fiber]]></title>
##     <authors><![CDATA[Qiangzhou Rong;  Xueguang Qiao;  Tuan Guo;  Hangzhou Yang;  Yanying Du;  Dan Su;  Ruohui Wang;  Hao Sun;  Dingyi Feng;  Manli Hu]]></authors>
##     <affiliations><![CDATA[Dept. of Phys., Northwest Univ., Xian, China]]></affiliations>
##     <controlledterms>
##       <term><![CDATA[annealing]]></term>
##       <term><![CDATA[fibre optic sensors]]></term>
##       <term><![CDATA[holey fibres]]></term>
##       <term><![CDATA[internal stresses]]></term>
##       <term><![CDATA[light interferometry]]></term>
##       <term><![CDATA[photonic crystals]]></term>
##       <term><![CDATA[temperature measurement]]></term>
##     </controlledterms>
##     <thesaurusterms>
##       <term><![CDATA[Annealing]]></term>
##       <term><![CDATA[Interference]]></term>
##       <term><![CDATA[Optical fiber polarization]]></term>
##       <term><![CDATA[Optical fiber sensors]]></term>
##       <term><![CDATA[Temperature measurement]]></term>
##       <term><![CDATA[Temperature sensors]]></term>
##     </thesaurusterms>
##     <pubtitle><![CDATA[Photonics Journal, IEEE]]></pubtitle>
##     <punumber><![CDATA[4563994]]></punumber>
##     <pubtype><![CDATA[Journals & Magazines]]></pubtype>
##     <publisher><![CDATA[IEEE]]></publisher>
##     <volume><![CDATA[6]]></volume>
##     <issue><![CDATA[1]]></issue>
##     <py><![CDATA[2014]]></py>
##     <spage><![CDATA[1]]></spage>
##     <epage><![CDATA[10]]></epage>
##     <abstract><![CDATA[A reflective fiber-optic interferometer for ultra-high temperature measurement is proposed and experimentally demonstrated. The device consists of a short piece of polarization-maintaining photonic crystal fiber (PM-PCF) spliced with a lead-in single mode fiber (SMF) without any offset. The hollow holes within the PM-PCF are partly collapsed due to the directional arc-heating splicing and excite two linearly polarized (LP) modes over the downstream PM-PCF. These two LP-modes are reflected at the end face of PM-PCF and backward recoupled into the lead-in SMF through the collapsed region. A superimposed interference pattern is obtained as the result of interference of the polarized modes. The proposed interferometer is demonstrated for ultra-high temperature measurement up to 1100 &#x00B0;C with a sensitivity of 12.3 pm/&#x00B0;C over repeated measurements. Meanwhile, isochronally thermal annealing has been studied to address the effect of residual stress in the sensing probe and the feasibility of further improving temperature measurement range of the device.]]></abstract>
##     <issn><![CDATA[1943-0655]]></issn>
##     <htmlFlag><![CDATA[1]]></htmlFlag>
##     <arnumber><![CDATA[6718196]]></arnumber>
##     <doi><![CDATA[10.1109/JPHOT.2014.2300474]]></doi>
##     <publicationId><![CDATA[6718196]]></publicationId>
##     <mdurl><![CDATA[http://ieeexplore.ieee.org/xpl/articleDetails.jsp?tp=&arnumber=6718196&contentType=Journals+%26+Magazines]]></mdurl>
##     <pdf><![CDATA[http://ieeexplore.ieee.org/stamp/stamp.jsp?arnumber=6718196]]></pdf>
##   </document>
##   <document>
##     <rank>949</rank>
##     <title><![CDATA[Unipolar Induction Revisited: New Experiments and the &#x201C;Edge Effect&#x201D; Theory]]></title>
##     <authors><![CDATA[Muller, F.J.]]></authors>
##     <affiliations><![CDATA[Phys. Dept., Florida Int. Univ., Miami, FL, USA]]></affiliations>
##     <controlledterms>
##       <term><![CDATA[electromagnetic induction]]></term>
##       <term><![CDATA[inductors]]></term>
##     </controlledterms>
##     <thesaurusterms>
##       <term><![CDATA[Magnetic circuits]]></term>
##       <term><![CDATA[Magnetic confinement]]></term>
##       <term><![CDATA[Magnetic field measurement]]></term>
##       <term><![CDATA[Magnetic flux]]></term>
##       <term><![CDATA[Magnetic noise]]></term>
##       <term><![CDATA[Magnetic separation]]></term>
##     </thesaurusterms>
##     <pubtitle><![CDATA[Magnetics, IEEE Transactions on]]></pubtitle>
##     <punumber><![CDATA[20]]></punumber>
##     <pubtype><![CDATA[Journals & Magazines]]></pubtype>
##     <publisher><![CDATA[IEEE]]></publisher>
##     <volume><![CDATA[50]]></volume>
##     <issue><![CDATA[1]]></issue>
##     <part><![CDATA[2]]></part>
##     <py><![CDATA[2014]]></py>
##     <spage><![CDATA[1]]></spage>
##     <epage><![CDATA[11]]></epage>
##     <abstract><![CDATA[A brief historical review is made of the 180-year-old debate on Faraday's unipolar inductor. By introducing two convenient modifications of Faraday's original experiment of 1832, pertinent answers are experimentally found to the most debated problems: 1) Can Faraday's law be used? Yes; 2) Do the magnetic field lines rotate when the magnet rotates? No. 3) Can the seat of induction be unambiguously determined? Yes. 4) Is there a fundamental difference between rotational and translational motional induction? Yes: the &#x201C;edge effect&#x201D;, whereby a negative v&#x00D7;B field appears whenever a magnetic edge moves perpendicularly to itself. An additional experiment is presented to verify the theory. Finally, 5) Can Relativity Theory be applied? The Special Theory, no; the General one, yes.]]></abstract>
##     <issn><![CDATA[0018-9464]]></issn>
##     <htmlFlag><![CDATA[1]]></htmlFlag>
##     <arnumber><![CDATA[6600969]]></arnumber>
##     <doi><![CDATA[10.1109/TMAG.2013.2282133]]></doi>
##     <publicationId><![CDATA[6600969]]></publicationId>
##     <mdurl><![CDATA[http://ieeexplore.ieee.org/xpl/articleDetails.jsp?tp=&arnumber=6600969&contentType=Journals+%26+Magazines]]></mdurl>
##     <pdf><![CDATA[http://ieeexplore.ieee.org/stamp/stamp.jsp?arnumber=6600969]]></pdf>
##   </document>
##   <document>
##     <rank>950</rank>
##     <title><![CDATA[All-Graphene Planar Double Barrier Resonant Tunneling Diodes]]></title>
##     <authors><![CDATA[Al-Dirini, F.;  Hossain, F.M.;  Nirmalathas, A.;  Skafidas, E.]]></authors>
##     <affiliations><![CDATA[Dept. of Electr. & Electron. Eng., Univ. of Melbourne, Melbourne, VIC, Australia]]></affiliations>
##     <controlledterms>
##       <term><![CDATA[EHT calculations]]></term>
##       <term><![CDATA[Green's function methods]]></term>
##       <term><![CDATA[graphene]]></term>
##       <term><![CDATA[resonant tunnelling diodes]]></term>
##     </controlledterms>
##     <thesaurusterms>
##       <term><![CDATA[Australia]]></term>
##       <term><![CDATA[Educational institutions]]></term>
##       <term><![CDATA[Fabrication]]></term>
##       <term><![CDATA[Graphene]]></term>
##       <term><![CDATA[Nanoscale devices]]></term>
##       <term><![CDATA[Resonant tunneling devices]]></term>
##     </thesaurusterms>
##     <pubtitle><![CDATA[Electron Devices Society, IEEE Journal of the]]></pubtitle>
##     <punumber><![CDATA[6245494]]></punumber>
##     <pubtype><![CDATA[Journals & Magazines]]></pubtype>
##     <publisher><![CDATA[IEEE]]></publisher>
##     <volume><![CDATA[2]]></volume>
##     <issue><![CDATA[5]]></issue>
##     <py><![CDATA[2014]]></py>
##     <spage><![CDATA[118]]></spage>
##     <epage><![CDATA[122]]></epage>
##     <abstract><![CDATA[In this work, we propose an atomically-thin all-graphene planar double barrier resonant tunneling diode that can be realized within a single graphene nanoribbon. The proposed device does not require any doping or external gating and can be fabricated using minimal process steps. The planar architecture of the device allows a simple in-plane connection of multiple devices in parallel without any extra processing steps during fabrication, enhancing the current driving capabilities of the device. Quantum mechanical simulation results, based on non-equilibrium Green's function formalism and the extended Huckel method, show promising device performance with a high reverse-to-forward current rectification ratio exceeding 50 000, and confirm the presence of negative differential resistance within the device's current-voltage characteristics.]]></abstract>
##     <issn><![CDATA[2168-6734]]></issn>
##     <htmlFlag><![CDATA[1]]></htmlFlag>
##     <arnumber><![CDATA[6823085]]></arnumber>
##     <doi><![CDATA[10.1109/JEDS.2014.2327375]]></doi>
##     <publicationId><![CDATA[6823085]]></publicationId>
##     <mdurl><![CDATA[http://ieeexplore.ieee.org/xpl/articleDetails.jsp?tp=&arnumber=6823085&contentType=Journals+%26+Magazines]]></mdurl>
##     <pdf><![CDATA[http://ieeexplore.ieee.org/stamp/stamp.jsp?arnumber=6823085]]></pdf>
##   </document>
##   <document>
##     <rank>951</rank>
##     <title><![CDATA[Optimal Scheduling for Real-Time Jobs in Energy Harvesting Computing Systems]]></title>
##     <authors><![CDATA[Chetto, M.]]></authors>
##     <affiliations><![CDATA[IRCCyN Res. Inst., Univ. of Nantes, Nantes, France]]></affiliations>
##     <controlledterms>
##       <term><![CDATA[energy harvesting]]></term>
##       <term><![CDATA[power aware computing]]></term>
##       <term><![CDATA[processor scheduling]]></term>
##       <term><![CDATA[real-time systems]]></term>
##       <term><![CDATA[renewable energy sources]]></term>
##       <term><![CDATA[secondary cells]]></term>
##     </controlledterms>
##     <thesaurusterms>
##       <term><![CDATA[Energy harvesting]]></term>
##       <term><![CDATA[Energy storage]]></term>
##       <term><![CDATA[Licenses]]></term>
##       <term><![CDATA[Real-time systems]]></term>
##       <term><![CDATA[Scheduling]]></term>
##       <term><![CDATA[Scheduling algorithms]]></term>
##     </thesaurusterms>
##     <pubtitle><![CDATA[Emerging Topics in Computing, IEEE Transactions on]]></pubtitle>
##     <punumber><![CDATA[6245516]]></punumber>
##     <pubtype><![CDATA[Journals & Magazines]]></pubtype>
##     <publisher><![CDATA[IEEE]]></publisher>
##     <volume><![CDATA[2]]></volume>
##     <issue><![CDATA[2]]></issue>
##     <py><![CDATA[2014]]></py>
##     <spage><![CDATA[122]]></spage>
##     <epage><![CDATA[133]]></epage>
##     <abstract><![CDATA[In this paper, we study a scheduling problem, in which every job is associated with a release time, deadline, required computation time, and required energy. We focus on an important special case where the jobs execute on a uniprocessor system that is supplied by a renewable energy source and use a rechargeable storage unit with limited capacity. Earliest deadline first (EDF) is a class one online algorithm in the classical real-time scheduling theory where energy constraints are not considered. We propose a semi-online EDF-based scheduling algorithm theoretically optimal (i.e., processing and energy costs neglected). This algorithm relies on the notions of energy demand and slack energy, which are different from the well known notions of processor demand and slack time. We provide an exact feasibility test. There are no restrictions on this new scheduler: each job can be one instance of a periodic, aperiodic, or sporadic task with deadline.]]></abstract>
##     <issn><![CDATA[2168-6750]]></issn>
##     <htmlFlag><![CDATA[1]]></htmlFlag>
##     <arnumber><![CDATA[6697819]]></arnumber>
##     <doi><![CDATA[10.1109/TETC.2013.2296537]]></doi>
##     <publicationId><![CDATA[6697819]]></publicationId>
##     <mdurl><![CDATA[http://ieeexplore.ieee.org/xpl/articleDetails.jsp?tp=&arnumber=6697819&contentType=Journals+%26+Magazines]]></mdurl>
##     <pdf><![CDATA[http://ieeexplore.ieee.org/stamp/stamp.jsp?arnumber=6697819]]></pdf>
##   </document>
##   <document>
##     <rank>952</rank>
##     <title><![CDATA[Novel OFDM Based on C-Transform for Improving Multipath Transmission]]></title>
##     <authors><![CDATA[Leftah, H.A.;  Boussakta, S.]]></authors>
##     <affiliations><![CDATA[Electr. Dept., Southern Tech. Univ., Basrah, Iraq]]></affiliations>
##     <controlledterms>
##       <term><![CDATA[OFDM modulation]]></term>
##       <term><![CDATA[discrete Fourier transforms]]></term>
##       <term><![CDATA[discrete cosine transforms]]></term>
##       <term><![CDATA[error statistics]]></term>
##       <term><![CDATA[least mean squares methods]]></term>
##       <term><![CDATA[multipath channels]]></term>
##     </controlledterms>
##     <thesaurusterms>
##       <term><![CDATA[Bit error rate]]></term>
##       <term><![CDATA[Discrete Fourier transforms]]></term>
##       <term><![CDATA[Discrete cosine transforms]]></term>
##       <term><![CDATA[Modulation]]></term>
##       <term><![CDATA[Multipath channels]]></term>
##       <term><![CDATA[Peak to average power ratio]]></term>
##     </thesaurusterms>
##     <pubtitle><![CDATA[Signal Processing, IEEE Transactions on]]></pubtitle>
##     <punumber><![CDATA[78]]></punumber>
##     <pubtype><![CDATA[Journals & Magazines]]></pubtype>
##     <publisher><![CDATA[IEEE]]></publisher>
##     <volume><![CDATA[62]]></volume>
##     <issue><![CDATA[23]]></issue>
##     <py><![CDATA[2014]]></py>
##     <spage><![CDATA[6158]]></spage>
##     <epage><![CDATA[6170]]></epage>
##     <abstract><![CDATA[In this paper, a new orthogonal-frequency-division multiplexing using the C-transform (C-OFDM) is introduced. An exact closed-form bit-error rate (BER) is derived for the proposed C-OFDM system over multipath channels and different kinds of modulation formats. The BER performance of the proposed C-OFDM is evaluated by mathematics and simulation for different channel models, modulation formats, under zero-padding (ZP) and minimum mean-square-error (MMSE) detection. The results are then compared with those of the discrete cosine transform (DCT) and discrete Fourier transform (DFT)-based OFDM, showing that over multipath channel, the new C-OFDM has better BER performance than both the DCT-OFDM and the DFT-OFDM. The proposed scheme is also found to achieve some reduction in the peak-to-average power ratio (PAPR) in comparison with the aforementioned OFDM schemes as the block-diagonal-structure (BDS) property of the C-transform minimizes the input signal superposition. The proposed system has the merits of being resilience to multipath channel dispersion and relatively has lower PAPR.]]></abstract>
##     <issn><![CDATA[1053-587X]]></issn>
##     <htmlFlag><![CDATA[1]]></htmlFlag>
##     <arnumber><![CDATA[6918543]]></arnumber>
##     <doi><![CDATA[10.1109/TSP.2014.2362097]]></doi>
##     <publicationId><![CDATA[6918543]]></publicationId>
##     <mdurl><![CDATA[http://ieeexplore.ieee.org/xpl/articleDetails.jsp?tp=&arnumber=6918543&contentType=Journals+%26+Magazines]]></mdurl>
##     <pdf><![CDATA[http://ieeexplore.ieee.org/stamp/stamp.jsp?arnumber=6918543]]></pdf>
##   </document>
##   <document>
##     <rank>953</rank>
##     <title><![CDATA[Assessing Intervention Timing in Computer-Based Education Using Machine Learning Algorithms]]></title>
##     <authors><![CDATA[Stimpson, A.J.;  Cummings, M.L.]]></authors>
##     <affiliations><![CDATA[Dept. of Aeronaut. & Astronaut., Massachusetts Inst. of Technol., Cambridge, MA, USA]]></affiliations>
##     <controlledterms>
##       <term><![CDATA[computer based training]]></term>
##       <term><![CDATA[data analysis]]></term>
##       <term><![CDATA[decision support systems]]></term>
##       <term><![CDATA[educational courses]]></term>
##       <term><![CDATA[learning (artificial intelligence)]]></term>
##       <term><![CDATA[pattern classification]]></term>
##       <term><![CDATA[regression analysis]]></term>
##     </controlledterms>
##     <thesaurusterms>
##       <term><![CDATA[Artificial neural networks]]></term>
##       <term><![CDATA[Linear regression]]></term>
##       <term><![CDATA[Machine learning]]></term>
##       <term><![CDATA[Machine learning algorithms]]></term>
##       <term><![CDATA[Measurement]]></term>
##       <term><![CDATA[Prediction algorithms]]></term>
##       <term><![CDATA[Predictive models]]></term>
##     </thesaurusterms>
##     <pubtitle><![CDATA[Access, IEEE]]></pubtitle>
##     <punumber><![CDATA[6287639]]></punumber>
##     <pubtype><![CDATA[Journals & Magazines]]></pubtype>
##     <publisher><![CDATA[IEEE]]></publisher>
##     <volume><![CDATA[2]]></volume>
##     <py><![CDATA[2014]]></py>
##     <spage><![CDATA[78]]></spage>
##     <epage><![CDATA[87]]></epage>
##     <abstract><![CDATA[The use of computer-based and online education systems has made new data available that can describe the temporal and process-level progression of learning. To date, machine learning research has not considered the impacts of these properties on the machine learning prediction task in educational settings. Machine learning algorithms may have applications in supporting targeted intervention approaches. The goals of this paper are to: 1) determine the impact of process-level information on machine learning prediction results and 2) establish the effect of type of machine learning algorithm used on prediction results. Data were collected from a university level course in human factors engineering (n=35), which included both traditional classroom assessment and computer-based assessment methods. A set of common regression and classification algorithms were applied to the data to predict final course score. The overall prediction accuracy as well as the chronological progression of prediction accuracy was analyzed for each algorithm. Simple machine learning algorithms (linear regression, logistic regression) had comparable performance with more complex methods (support vector machines, artificial neural networks). Process-level information was not useful in post-hoc predictions, but contributed significantly to allowing for accurate predictions to be made earlier in the course. Process level information provides useful prediction features for development of targeted intervention techniques, as it allows more accurate predictions to be made earlier in the course. For small course data sets, the prediction accuracy and simplicity of linear regression and logistic regression make these methods preferable to more complex algorithms.]]></abstract>
##     <issn><![CDATA[2169-3536]]></issn>
##     <htmlFlag><![CDATA[1]]></htmlFlag>
##     <arnumber><![CDATA[6730683]]></arnumber>
##     <doi><![CDATA[10.1109/ACCESS.2014.2303071]]></doi>
##     <publicationId><![CDATA[6730683]]></publicationId>
##     <mdurl><![CDATA[http://ieeexplore.ieee.org/xpl/articleDetails.jsp?tp=&arnumber=6730683&contentType=Journals+%26+Magazines]]></mdurl>
##     <pdf><![CDATA[http://ieeexplore.ieee.org/stamp/stamp.jsp?arnumber=6730683]]></pdf>
##   </document>
##   <document>
##     <rank>954</rank>
##     <title><![CDATA[A Constrained Backpropagation Approach for the Adaptive Solution of Partial Differential Equations]]></title>
##     <authors><![CDATA[Rudd, K.;  Di Muro, G.;  Ferrari, S.]]></authors>
##     <affiliations><![CDATA[Dept. of Mech. Eng. & Mater. Sci., Duke Univ., Durham, NC, USA]]></affiliations>
##     <controlledterms>
##       <term><![CDATA[elliptic equations]]></term>
##       <term><![CDATA[learning (artificial intelligence)]]></term>
##       <term><![CDATA[neural nets]]></term>
##       <term><![CDATA[optimisation]]></term>
##       <term><![CDATA[partial differential equations]]></term>
##     </controlledterms>
##     <thesaurusterms>
##       <term><![CDATA[Artificial neural networks]]></term>
##       <term><![CDATA[Equations]]></term>
##       <term><![CDATA[Jacobian matrices]]></term>
##       <term><![CDATA[Linear programming]]></term>
##       <term><![CDATA[Optimization]]></term>
##       <term><![CDATA[Training]]></term>
##     </thesaurusterms>
##     <pubtitle><![CDATA[Neural Networks and Learning Systems, IEEE Transactions on]]></pubtitle>
##     <punumber><![CDATA[5962385]]></punumber>
##     <pubtype><![CDATA[Journals & Magazines]]></pubtype>
##     <publisher><![CDATA[IEEE]]></publisher>
##     <volume><![CDATA[25]]></volume>
##     <issue><![CDATA[3]]></issue>
##     <py><![CDATA[2014]]></py>
##     <spage><![CDATA[571]]></spage>
##     <epage><![CDATA[584]]></epage>
##     <abstract><![CDATA[This paper presents a constrained backpropagation (CPROP) methodology for solving nonlinear elliptic and parabolic partial differential equations (PDEs) adaptively, subject to changes in the PDE parameters or external forcing. Unlike existing methods based on penalty functions or Lagrange multipliers, CPROP solves the constrained optimization problem associated with training a neural network to approximate the PDE solution by means of direct elimination. As a result, CPROP reduces the dimensionality of the optimization problem, while satisfying the equality constraints associated with the boundary and initial conditions exactly, at every iteration of the algorithm. The effectiveness of this method is demonstrated through several examples, including nonlinear elliptic and parabolic PDEs with changing parameters and nonhomogeneous terms.]]></abstract>
##     <issn><![CDATA[2162-237X]]></issn>
##     <htmlFlag><![CDATA[1]]></htmlFlag>
##     <arnumber><![CDATA[6658964]]></arnumber>
##     <doi><![CDATA[10.1109/TNNLS.2013.2277601]]></doi>
##     <publicationId><![CDATA[6658964]]></publicationId>
##     <mdurl><![CDATA[http://ieeexplore.ieee.org/xpl/articleDetails.jsp?tp=&arnumber=6658964&contentType=Journals+%26+Magazines]]></mdurl>
##     <pdf><![CDATA[http://ieeexplore.ieee.org/stamp/stamp.jsp?arnumber=6658964]]></pdf>
##   </document>
##   <document>
##     <rank>955</rank>
##     <title><![CDATA[Online Social Networks: Threats and Solutions]]></title>
##     <authors><![CDATA[Fire, M.;  Goldschmidt, R.;  Elovici, Y.]]></authors>
##     <affiliations><![CDATA[Dept. of Inf. Syst. Eng., Ben-Gurion Univ. of the Negev, Beer-Sheva, Israel]]></affiliations>
##     <controlledterms>
##       <term><![CDATA[data privacy]]></term>
##       <term><![CDATA[social networking (online)]]></term>
##       <term><![CDATA[social sciences computing]]></term>
##     </controlledterms>
##     <thesaurusterms>
##       <term><![CDATA[Computer security]]></term>
##       <term><![CDATA[Facebook]]></term>
##       <term><![CDATA[Online services]]></term>
##       <term><![CDATA[Privacy]]></term>
##       <term><![CDATA[Social network services]]></term>
##       <term><![CDATA[Twitter]]></term>
##     </thesaurusterms>
##     <pubtitle><![CDATA[Communications Surveys & Tutorials, IEEE]]></pubtitle>
##     <punumber><![CDATA[9739]]></punumber>
##     <pubtype><![CDATA[Journals & Magazines]]></pubtype>
##     <publisher><![CDATA[IEEE]]></publisher>
##     <volume><![CDATA[16]]></volume>
##     <issue><![CDATA[4]]></issue>
##     <py><![CDATA[2014]]></py>
##     <spage><![CDATA[2019]]></spage>
##     <epage><![CDATA[2036]]></epage>
##     <abstract><![CDATA[Many online social network (OSN) users are unaware of the numerous security risks that exist in these networks, including privacy violations, identity theft, and sexual harassment, just to name a few. According to recent studies, OSN users readily expose personal and private details about themselves, such as relationship status, date of birth, school name, email address, phone number, and even home address. This information, if put into the wrong hands, can be used to harm users both in the virtual world and in the real world. These risks become even more severe when the users are children. In this paper, we present a thorough review of the different security and privacy risks, which threaten the well-being of OSN users in general, and children in particular. In addition, we present an overview of existing solutions that can provide better protection, security, and privacy for OSN users. We also offer simple-to-implement recommendations for OSN users, which can improve their security and privacy when using these platforms. Furthermore, we suggest future research directions.]]></abstract>
##     <issn><![CDATA[1553-877X]]></issn>
##     <htmlFlag><![CDATA[1]]></htmlFlag>
##     <arnumber><![CDATA[6809839]]></arnumber>
##     <doi><![CDATA[10.1109/COMST.2014.2321628]]></doi>
##     <publicationId><![CDATA[6809839]]></publicationId>
##     <mdurl><![CDATA[http://ieeexplore.ieee.org/xpl/articleDetails.jsp?tp=&arnumber=6809839&contentType=Journals+%26+Magazines]]></mdurl>
##     <pdf><![CDATA[http://ieeexplore.ieee.org/stamp/stamp.jsp?arnumber=6809839]]></pdf>
##   </document>
##   <document>
##     <rank>956</rank>
##     <title><![CDATA[Variable-Bandwidth Superchannels Using Synchronized Colorless Transceivers]]></title>
##     <authors><![CDATA[Sharif, M.;  Kahn, J.M.]]></authors>
##     <affiliations><![CDATA[Dept. of Electr. Eng., Stanford Univ., Stanford, CA, USA]]></affiliations>
##     <controlledterms>
##       <term><![CDATA[light transmission]]></term>
##       <term><![CDATA[optical communication equipment]]></term>
##       <term><![CDATA[optical design techniques]]></term>
##       <term><![CDATA[optical fibre communication]]></term>
##       <term><![CDATA[optical fibre polarisation]]></term>
##       <term><![CDATA[optical modulation]]></term>
##       <term><![CDATA[optical transceivers]]></term>
##       <term><![CDATA[quadrature phase shift keying]]></term>
##       <term><![CDATA[subcarrier multiplexing]]></term>
##       <term><![CDATA[synchronisation]]></term>
##     </controlledterms>
##     <thesaurusterms>
##       <term><![CDATA[Bandwidth]]></term>
##       <term><![CDATA[Frequency modulation]]></term>
##       <term><![CDATA[OFDM]]></term>
##       <term><![CDATA[Optical transmitters]]></term>
##       <term><![CDATA[Receivers]]></term>
##       <term><![CDATA[Synchronization]]></term>
##       <term><![CDATA[Transceivers]]></term>
##     </thesaurusterms>
##     <pubtitle><![CDATA[Lightwave Technology, Journal of]]></pubtitle>
##     <punumber><![CDATA[50]]></punumber>
##     <pubtype><![CDATA[Journals & Magazines]]></pubtype>
##     <publisher><![CDATA[IEEE]]></publisher>
##     <volume><![CDATA[32]]></volume>
##     <issue><![CDATA[10]]></issue>
##     <py><![CDATA[2014]]></py>
##     <spage><![CDATA[1921]]></spage>
##     <epage><![CDATA[1929]]></epage>
##     <abstract><![CDATA[We propose a modular architecture for long-haul optical networks supporting flexible-bandwidth superchannels. Colorless transceivers can be designed to modulate/detect up to M = 4 subcarriers, each at a symbol rate of 12.5 Gbaud, achieving a maximum bit rate of 200 Gbit/s, assuming polarization-multiplexed quadrature phase-shift keying (PM-QPSK). A set of N synchronized transceivers can cooperate to modulate/detect a superchannel comprising N&#x00B7;M subcarriers using no-guard-interval orthogonal frequency-division multiplexing, enabling transmission at bit rates beyond 1 Tbit/s. We analyze and simulate the performance of the proposed architecture in the presence of linear fiber impairments and synchronization errors and establish design requirements for practical deployment of the architecture. Simulation results are shown for transmission of a superchannel comprising 24 subcarriers, which conveys approximately 1.1 Tbit/s with a spectral efficiency of 3.5 bits/s/Hz using PM-QPSK.]]></abstract>
##     <issn><![CDATA[0733-8724]]></issn>
##     <htmlFlag><![CDATA[1]]></htmlFlag>
##     <arnumber><![CDATA[6767060]]></arnumber>
##     <doi><![CDATA[10.1109/JLT.2014.2310692]]></doi>
##     <publicationId><![CDATA[6767060]]></publicationId>
##     <mdurl><![CDATA[http://ieeexplore.ieee.org/xpl/articleDetails.jsp?tp=&arnumber=6767060&contentType=Journals+%26+Magazines]]></mdurl>
##     <pdf><![CDATA[http://ieeexplore.ieee.org/stamp/stamp.jsp?arnumber=6767060]]></pdf>
##   </document>
##   <document>
##     <rank>957</rank>
##     <title><![CDATA[Influence of <inline-formula> <img src="/images/tex/21708.gif" alt="A"> </inline-formula>-Site Deficiency on Magnetocaloric Effect in Nonstoichiometric <inline-formula> <img src="/images/tex/21709.gif" alt="({\rm La}_{0.8}{\rm Ca}_{0.2})_{0.975}{\rm MnO}_{3.01}"> </inline-formula>]]></title>
##     <authors><![CDATA[Nanto, D.;  Seong-Cho Yu;  Suhk-Kun Oh;  Chebotaev, N.;  Telegin, A.]]></authors>
##     <affiliations><![CDATA[Phys. Dept., Chungbuk Nat. Univ., Cheongju, South Korea]]></affiliations>
##     <controlledterms>
##       <term><![CDATA[Curie temperature]]></term>
##       <term><![CDATA[calcium compounds]]></term>
##       <term><![CDATA[entropy]]></term>
##       <term><![CDATA[lanthanum compounds]]></term>
##       <term><![CDATA[magnetic cooling]]></term>
##       <term><![CDATA[nanofabrication]]></term>
##       <term><![CDATA[nanomagnetics]]></term>
##       <term><![CDATA[nanostructured materials]]></term>
##       <term><![CDATA[stoichiometry]]></term>
##     </controlledterms>
##     <thesaurusterms>
##       <term><![CDATA[Compounds]]></term>
##       <term><![CDATA[Entropy]]></term>
##       <term><![CDATA[Lattices]]></term>
##       <term><![CDATA[Magnetic properties]]></term>
##       <term><![CDATA[Magnetization]]></term>
##       <term><![CDATA[Metals]]></term>
##       <term><![CDATA[Temperature measurement]]></term>
##     </thesaurusterms>
##     <pubtitle><![CDATA[Magnetics, IEEE Transactions on]]></pubtitle>
##     <punumber><![CDATA[20]]></punumber>
##     <pubtype><![CDATA[Journals & Magazines]]></pubtype>
##     <publisher><![CDATA[IEEE]]></publisher>
##     <volume><![CDATA[50]]></volume>
##     <issue><![CDATA[4]]></issue>
##     <part><![CDATA[1]]></part>
##     <py><![CDATA[2014]]></py>
##     <spage><![CDATA[1]]></spage>
##     <epage><![CDATA[4]]></epage>
##     <abstract><![CDATA[Manganite perovskite with A-site deficiency has been synthesized by a solid-state reaction technique. Our work does not support a general view that A-site deficiency gives a decrease on Curie temperature. The Curie temperature of (La<sub>0.8</sub>Ca<sub>0.2</sub>)<sub>0.975</sub>MnO<sub>3.01</sub> is 188 K with maximum entropy change -&#x0394;SM <sub>max</sub> = 1.6 J&#x00B7;kg<sup>-1</sup>&#x00B7;K<sup>-1</sup> and refrigerant capacity (RC) of 27 J/kg under an applied field of 10 kOe. The nonstoichiometric (La<sub>0.8</sub>Ca<sub>0.2</sub>)<sub>0.975</sub>MnO<sub>3.01</sub> may offer a wider temperature span of ~240% compared with those nanocrystalline La<sub>0.8</sub>Ca<sub>0.2</sub>MnO<sub>3.01</sub> that have similar RC, which has been reported by others.]]></abstract>
##     <issn><![CDATA[0018-9464]]></issn>
##     <arnumber><![CDATA[6798025]]></arnumber>
##     <doi><![CDATA[10.1109/TMAG.2013.2292075]]></doi>
##     <publicationId><![CDATA[6798025]]></publicationId>
##     <mdurl><![CDATA[http://ieeexplore.ieee.org/xpl/articleDetails.jsp?tp=&arnumber=6798025&contentType=Journals+%26+Magazines]]></mdurl>
##     <pdf><![CDATA[http://ieeexplore.ieee.org/stamp/stamp.jsp?arnumber=6798025]]></pdf>
##   </document>
##   <document>
##     <rank>958</rank>
##     <title><![CDATA[Resource Efficiency: A New Paradigm on Energy Efficiency and Spectral Efficiency Tradeoff]]></title>
##     <authors><![CDATA[Jie Tang;  So, D.K.C.;  Alsusa, E.;  Hamdi, K.A.]]></authors>
##     <affiliations><![CDATA[Sch. of Electr. & Electron. Eng., Univ. of Manchester, Manchester, UK]]></affiliations>
##     <controlledterms>
##       <term><![CDATA[OFDM modulation]]></term>
##       <term><![CDATA[cellular radio]]></term>
##       <term><![CDATA[channel allocation]]></term>
##       <term><![CDATA[communication complexity]]></term>
##       <term><![CDATA[frequency division multiple access]]></term>
##       <term><![CDATA[gradient methods]]></term>
##       <term><![CDATA[optimisation]]></term>
##       <term><![CDATA[quality of service]]></term>
##       <term><![CDATA[radio networks]]></term>
##       <term><![CDATA[spectral analysis]]></term>
##     </controlledterms>
##     <thesaurusterms>
##       <term><![CDATA[Bandwidth]]></term>
##       <term><![CDATA[Downlink]]></term>
##       <term><![CDATA[Measurement]]></term>
##       <term><![CDATA[Optimization]]></term>
##       <term><![CDATA[Quality of service]]></term>
##       <term><![CDATA[Resource management]]></term>
##       <term><![CDATA[Wireless communication]]></term>
##     </thesaurusterms>
##     <pubtitle><![CDATA[Wireless Communications, IEEE Transactions on]]></pubtitle>
##     <punumber><![CDATA[7693]]></punumber>
##     <pubtype><![CDATA[Journals & Magazines]]></pubtype>
##     <publisher><![CDATA[IEEE]]></publisher>
##     <volume><![CDATA[13]]></volume>
##     <issue><![CDATA[8]]></issue>
##     <py><![CDATA[2014]]></py>
##     <spage><![CDATA[4656]]></spage>
##     <epage><![CDATA[4669]]></epage>
##     <abstract><![CDATA[Spectral efficiency (SE) and energy efficiency (EE) are the main metrics for designing wireless networks. Rather than focusing on either SE or EE separately, recent works have focused on the relationship between EE and SE and provided good insight into the joint EE-SE tradeoff. However, such works have assumed that the bandwidth was fully occupied regardless of the transmission requirements and therefore are only valid for this type of scenario. In this paper, we propose a new paradigm for EE-SE tradeoff, namely the resource efficiency (RE) for orthogonal frequency division multiple access (OFDMA) cellular network in which we take into consideration different transmission-bandwidth requirements. We analyse the properties of the proposed RE and prove that it is capable of exploiting the tradeoff between EE and SE by balancing consumption power and occupied bandwidth; hence simultaneously optimizing both EE and SE. We then formulate the generalized RE optimization problem with guaranteed quality of service (QoS) and provide a gradient based optimal power adaptation scheme to solve it. We also provide an upper bound near optimal method to jointly solve the optimization problem. Furthermore, a low-complexity suboptimal algorithm based on a uniform power allocation scheme is proposed to reduce the complexity. Numerical results confirm the analytical findings and demonstrate the effectiveness of the proposed resource allocation schemes for efficient resource usage.]]></abstract>
##     <issn><![CDATA[1536-1276]]></issn>
##     <htmlFlag><![CDATA[1]]></htmlFlag>
##     <arnumber><![CDATA[6787103]]></arnumber>
##     <doi><![CDATA[10.1109/TWC.2014.2316791]]></doi>
##     <publicationId><![CDATA[6787103]]></publicationId>
##     <mdurl><![CDATA[http://ieeexplore.ieee.org/xpl/articleDetails.jsp?tp=&arnumber=6787103&contentType=Journals+%26+Magazines]]></mdurl>
##     <pdf><![CDATA[http://ieeexplore.ieee.org/stamp/stamp.jsp?arnumber=6787103]]></pdf>
##   </document>
##   <document>
##     <rank>959</rank>
##     <title><![CDATA[An Online Performance Prediction Framework for Service-Oriented Systems]]></title>
##     <authors><![CDATA[Yilei Zhang;  Zibin Zheng;  Lyu, M.R.]]></authors>
##     <affiliations><![CDATA[Shenzhen Key Lab. of Rich Media Big Data Analytics & Applic., Chinese Univ. of Hong Kong, Shenzhen, China]]></affiliations>
##     <controlledterms>
##       <term><![CDATA[Web services]]></term>
##       <term><![CDATA[service-oriented architecture]]></term>
##       <term><![CDATA[software performance evaluation]]></term>
##       <term><![CDATA[time series]]></term>
##     </controlledterms>
##     <thesaurusterms>
##       <term><![CDATA[Market research]]></term>
##       <term><![CDATA[Prediction algorithms]]></term>
##       <term><![CDATA[Predictive models]]></term>
##       <term><![CDATA[Runtime]]></term>
##       <term><![CDATA[Time factors]]></term>
##       <term><![CDATA[Vectors]]></term>
##       <term><![CDATA[Web services]]></term>
##     </thesaurusterms>
##     <pubtitle><![CDATA[Systems, Man, and Cybernetics: Systems, IEEE Transactions on]]></pubtitle>
##     <punumber><![CDATA[6221021]]></punumber>
##     <pubtype><![CDATA[Journals & Magazines]]></pubtype>
##     <publisher><![CDATA[IEEE]]></publisher>
##     <volume><![CDATA[44]]></volume>
##     <issue><![CDATA[9]]></issue>
##     <py><![CDATA[2014]]></py>
##     <spage><![CDATA[1169]]></spage>
##     <epage><![CDATA[1181]]></epage>
##     <abstract><![CDATA[The exponential growth of Web service makes building high-quality service-oriented systems an urgent and crucial research problem. Performance of the service-oriented systems highly depends on the remote Web services as well as the unpredictability of the Internet. Performance prediction of service-oriented systems is critical for automatically selecting the optimal Web service composition. Since the performance of Web services is highly related to the service status and network environments which are variable over time, it is an important task to predict the performance of service-oriented systems at run-time. To address this critical challenge, this paper proposes an online performance prediction framework, called OPred, to provide personalized service-oriented system performance prediction efficiently. Based on the past usage experience from different users, OPred builds feature models and employs time series analysis techniques on feature trends to make performance prediction. The results of large-scale real-world experiments show the effectiveness and efficiency of OPred.]]></abstract>
##     <issn><![CDATA[2168-2216]]></issn>
##     <htmlFlag><![CDATA[1]]></htmlFlag>
##     <arnumber><![CDATA[6720144]]></arnumber>
##     <doi><![CDATA[10.1109/TSMC.2013.2297401]]></doi>
##     <publicationId><![CDATA[6720144]]></publicationId>
##     <mdurl><![CDATA[http://ieeexplore.ieee.org/xpl/articleDetails.jsp?tp=&arnumber=6720144&contentType=Journals+%26+Magazines]]></mdurl>
##     <pdf><![CDATA[http://ieeexplore.ieee.org/stamp/stamp.jsp?arnumber=6720144]]></pdf>
##   </document>
##   <document>
##     <rank>960</rank>
##     <title><![CDATA[Breakthroughs in Photonics 2013: Flat Optics: Wavefronts Control With Huygens' Interfaces]]></title>
##     <authors><![CDATA[Genevet, P.;  Capasso, F.]]></authors>
##     <affiliations><![CDATA[Sch. of Eng. & Appl. Sci., Harvard Univ., Cambridge, MA, USA]]></affiliations>
##     <controlledterms>
##       <term><![CDATA[light propagation]]></term>
##       <term><![CDATA[nanophotonics]]></term>
##       <term><![CDATA[optical metamaterials]]></term>
##       <term><![CDATA[optical resonators]]></term>
##     </controlledterms>
##     <thesaurusterms>
##       <term><![CDATA[Optical reflection]]></term>
##       <term><![CDATA[Optical refraction]]></term>
##       <term><![CDATA[Optical resonators]]></term>
##       <term><![CDATA[Optical sensors]]></term>
##       <term><![CDATA[Photonics]]></term>
##       <term><![CDATA[Plasmons]]></term>
##     </thesaurusterms>
##     <pubtitle><![CDATA[Photonics Journal, IEEE]]></pubtitle>
##     <punumber><![CDATA[4563994]]></punumber>
##     <pubtype><![CDATA[Journals & Magazines]]></pubtype>
##     <publisher><![CDATA[IEEE]]></publisher>
##     <volume><![CDATA[6]]></volume>
##     <issue><![CDATA[2]]></issue>
##     <py><![CDATA[2014]]></py>
##     <spage><![CDATA[1]]></spage>
##     <epage><![CDATA[4]]></epage>
##     <abstract><![CDATA[Recent progress in the fields of nanophotonics and metamaterials has enabled the development of ultrathin and flat optical components, providing physicists and optical engineers a new method to control light. According to the Huygens-Fresnel principle, light gradually propagates step by step by exciting secondary waves that then reradiate to form the next wavefront; the phase and amplitude of these secondary waves are intimately related to the incoming optical wavefront. By using the response of nanoengineered subwavelength optical resonators at interfaces, it is now possible to engineer Huygens' interfaces to achieve an unprecedented control of the wavefront over large bandwidths and subwavelength propagation distances.]]></abstract>
##     <issn><![CDATA[1943-0655]]></issn>
##     <htmlFlag><![CDATA[1]]></htmlFlag>
##     <arnumber><![CDATA[6748002]]></arnumber>
##     <doi><![CDATA[10.1109/JPHOT.2014.2308194]]></doi>
##     <publicationId><![CDATA[6748002]]></publicationId>
##     <mdurl><![CDATA[http://ieeexplore.ieee.org/xpl/articleDetails.jsp?tp=&arnumber=6748002&contentType=Journals+%26+Magazines]]></mdurl>
##     <pdf><![CDATA[http://ieeexplore.ieee.org/stamp/stamp.jsp?arnumber=6748002]]></pdf>
##   </document>
##   <document>
##     <rank>961</rank>
##     <title><![CDATA[Mutual-Relationship-Based Community Partitioning for Social Networks]]></title>
##     <authors><![CDATA[Yuqing Zhu;  Deying Li;  Wen Xu;  Weili Wu;  Lidan Fan;  Willson, J.]]></authors>
##     <affiliations><![CDATA[Dept. of Comput. Sci., California State Univ., Los Angeles, CA, USA]]></affiliations>
##     <controlledterms>
##       <term><![CDATA[computational complexity]]></term>
##       <term><![CDATA[network theory (graphs)]]></term>
##       <term><![CDATA[social networking (online)]]></term>
##     </controlledterms>
##     <thesaurusterms>
##       <term><![CDATA[Communities]]></term>
##       <term><![CDATA[Linear programming]]></term>
##       <term><![CDATA[Partitioning algorithms]]></term>
##       <term><![CDATA[Social network services]]></term>
##     </thesaurusterms>
##     <pubtitle><![CDATA[Emerging Topics in Computing, IEEE Transactions on]]></pubtitle>
##     <punumber><![CDATA[6245516]]></punumber>
##     <pubtype><![CDATA[Journals & Magazines]]></pubtype>
##     <publisher><![CDATA[IEEE]]></publisher>
##     <volume><![CDATA[2]]></volume>
##     <issue><![CDATA[4]]></issue>
##     <py><![CDATA[2014]]></py>
##     <spage><![CDATA[436]]></spage>
##     <epage><![CDATA[447]]></epage>
##     <abstract><![CDATA[Social networks have shown increasing popularity in real-world applications. Community detection is one of the fundamental problems. In this paper, we study how to partition the social networks into communities from a novel perspective. We define the mutual closeness and strangeness between each vertex pairs, and formulate our problem as a semidefinite program considering both the tightness of the same community and the looseness across different communities. Two NP-hard issues are addressed. One is to partition the social networks into communities through maximizing the tightness within the same community and the looseness between different communities. In the other issue, we take community volume into consideration such that the obtained communities have similar sizes. We give the mathematical models and the objective functions, and then analyze the performance bounds of the proposed algorithms. At last, we validate our method's effectiveness by comparing them with a highly effective existing partitioning method on real-world and artificial data sets.]]></abstract>
##     <issn><![CDATA[2168-6750]]></issn>
##     <htmlFlag><![CDATA[1]]></htmlFlag>
##     <arnumber><![CDATA[6985677]]></arnumber>
##     <doi><![CDATA[10.1109/TETC.2014.2380391]]></doi>
##     <publicationId><![CDATA[6985677]]></publicationId>
##     <mdurl><![CDATA[http://ieeexplore.ieee.org/xpl/articleDetails.jsp?tp=&arnumber=6985677&contentType=Journals+%26+Magazines]]></mdurl>
##     <pdf><![CDATA[http://ieeexplore.ieee.org/stamp/stamp.jsp?arnumber=6985677]]></pdf>
##   </document>
##   <document>
##     <rank>962</rank>
##     <title><![CDATA[A Novel Lensless Miniature Contact Imaging System for Monitoring Calcium Changes in Live Neurons]]></title>
##     <authors><![CDATA[Mudraboyina, A.K.;  Blockstein, L.;  Luk, C.C.;  Syed, N.I.;  Yadid-Pecht, O.]]></authors>
##     <affiliations><![CDATA[Dept. of Electr. & Comput. Eng., Univ. of Calgary, Calgary, AB, Canada]]></affiliations>
##     <controlledterms>
##       <term><![CDATA[CMOS image sensors]]></term>
##       <term><![CDATA[biomedical optical imaging]]></term>
##       <term><![CDATA[dyes]]></term>
##       <term><![CDATA[fluorescence]]></term>
##       <term><![CDATA[neurophysiology]]></term>
##       <term><![CDATA[optical microscopy]]></term>
##     </controlledterms>
##     <thesaurusterms>
##       <term><![CDATA[Arrays]]></term>
##       <term><![CDATA[Band-pass filters]]></term>
##       <term><![CDATA[Calcium]]></term>
##       <term><![CDATA[Fluorescence]]></term>
##       <term><![CDATA[Glass]]></term>
##       <term><![CDATA[Imaging]]></term>
##       <term><![CDATA[Optical filters]]></term>
##     </thesaurusterms>
##     <pubtitle><![CDATA[Photonics Journal, IEEE]]></pubtitle>
##     <punumber><![CDATA[4563994]]></punumber>
##     <pubtype><![CDATA[Journals & Magazines]]></pubtype>
##     <publisher><![CDATA[IEEE]]></publisher>
##     <volume><![CDATA[6]]></volume>
##     <issue><![CDATA[1]]></issue>
##     <py><![CDATA[2014]]></py>
##     <spage><![CDATA[1]]></spage>
##     <epage><![CDATA[15]]></epage>
##     <abstract><![CDATA[Here, we report on the design, fabrication, and verification of a novel CMOS-imager-based contact imaging system. We acquired fluorescent images from live neurons by monitoring calcium changes with Fura-2 dye. Our current device consists of a removable absorption filter interfaced with a CMOS imaging sensor and an external DG-4 lamp for excitation. Fura-2 loaded Lymnaea stagnalis neurons were stimulated with dual excitation wavelengths of 340 and 380 nm; our image sensor detected 510-nm emission. We show that our system is capable of detecting intracellular calcium changes in Fura-2 loaded neurons. Further, this sensor also enabled viewing of multiple neurons over a large surface area simultaneously, an option that is not readily available in conventional light microscopy.]]></abstract>
##     <issn><![CDATA[1943-0655]]></issn>
##     <htmlFlag><![CDATA[1]]></htmlFlag>
##     <arnumber><![CDATA[6731561]]></arnumber>
##     <doi><![CDATA[10.1109/JPHOT.2014.2304554]]></doi>
##     <publicationId><![CDATA[6731561]]></publicationId>
##     <mdurl><![CDATA[http://ieeexplore.ieee.org/xpl/articleDetails.jsp?tp=&arnumber=6731561&contentType=Journals+%26+Magazines]]></mdurl>
##     <pdf><![CDATA[http://ieeexplore.ieee.org/stamp/stamp.jsp?arnumber=6731561]]></pdf>
##   </document>
##   <document>
##     <rank>963</rank>
##     <title><![CDATA[Breakthroughs in Photonics 2013: Electrically Pumped Semiconductor Entangled Sources]]></title>
##     <authors><![CDATA[Helmy, A.S.]]></authors>
##     <affiliations><![CDATA[Edward S. Rogers Dept. of Electr. & Comput. Eng., Univ. of Toronto, Toronto, ON, Canada]]></affiliations>
##     <controlledterms>
##       <term><![CDATA[integrated optics]]></term>
##       <term><![CDATA[integrated optoelectronics]]></term>
##       <term><![CDATA[quantum optics]]></term>
##       <term><![CDATA[semiconductor materials]]></term>
##       <term><![CDATA[silicon-on-insulator]]></term>
##     </controlledterms>
##     <thesaurusterms>
##       <term><![CDATA[Nonlinear optics]]></term>
##       <term><![CDATA[Optical polarization]]></term>
##       <term><![CDATA[Optical pumping]]></term>
##       <term><![CDATA[Optical waveguides]]></term>
##       <term><![CDATA[Photonics]]></term>
##       <term><![CDATA[Quantum entanglement]]></term>
##     </thesaurusterms>
##     <pubtitle><![CDATA[Photonics Journal, IEEE]]></pubtitle>
##     <punumber><![CDATA[4563994]]></punumber>
##     <pubtype><![CDATA[Journals & Magazines]]></pubtype>
##     <publisher><![CDATA[IEEE]]></publisher>
##     <volume><![CDATA[6]]></volume>
##     <issue><![CDATA[2]]></issue>
##     <py><![CDATA[2014]]></py>
##     <spage><![CDATA[1]]></spage>
##     <epage><![CDATA[6]]></epage>
##     <abstract><![CDATA[Recent advances enabling the realization of room-temperature electrically injected chip sources for entangled photons will be highlighted and discussed. Emphasis will be placed on monolithic and hybrid integration settings that enable room-temperature and self-contained operation through electrical current injection. Compound semiconductors, as well as hybrid integration platforms such as those utilizing Si-on-insulator technology, play a central role in defining the approaches sought to realize these sources.]]></abstract>
##     <issn><![CDATA[1943-0655]]></issn>
##     <htmlFlag><![CDATA[1]]></htmlFlag>
##     <arnumber><![CDATA[6776402]]></arnumber>
##     <doi><![CDATA[10.1109/JPHOT.2014.2312944]]></doi>
##     <publicationId><![CDATA[6776402]]></publicationId>
##     <mdurl><![CDATA[http://ieeexplore.ieee.org/xpl/articleDetails.jsp?tp=&arnumber=6776402&contentType=Journals+%26+Magazines]]></mdurl>
##     <pdf><![CDATA[http://ieeexplore.ieee.org/stamp/stamp.jsp?arnumber=6776402]]></pdf>
##   </document>
##   <document>
##     <rank>964</rank>
##     <title><![CDATA[Development and Specification of a Reference Architecture for Agent-Based Systems]]></title>
##     <authors><![CDATA[Regli, W.C.;  Mayk, I.;  Cannon, C.T.;  Kopena, J.B.;  Lass, R.N.;  Mongan, W.M.;  Nguyen, D.N.;  Salvage, J.K.;  Sultanik, E.A.;  Usbeck, K.]]></authors>
##     <affiliations><![CDATA[Drexel Univ., Philadelphia, PA, USA]]></affiliations>
##     <controlledterms>
##       <term><![CDATA[digital forensics]]></term>
##       <term><![CDATA[open systems]]></term>
##       <term><![CDATA[software agents]]></term>
##       <term><![CDATA[software architecture]]></term>
##     </controlledterms>
##     <thesaurusterms>
##       <term><![CDATA[Abstracts]]></term>
##       <term><![CDATA[Computer architecture]]></term>
##       <term><![CDATA[Concrete]]></term>
##       <term><![CDATA[Object oriented modeling]]></term>
##       <term><![CDATA[Software systems]]></term>
##       <term><![CDATA[Unified modeling language]]></term>
##     </thesaurusterms>
##     <pubtitle><![CDATA[Systems, Man, and Cybernetics: Systems, IEEE Transactions on]]></pubtitle>
##     <punumber><![CDATA[6221021]]></punumber>
##     <pubtype><![CDATA[Journals & Magazines]]></pubtype>
##     <publisher><![CDATA[IEEE]]></publisher>
##     <volume><![CDATA[44]]></volume>
##     <issue><![CDATA[2]]></issue>
##     <py><![CDATA[2014]]></py>
##     <spage><![CDATA[146]]></spage>
##     <epage><![CDATA[161]]></epage>
##     <abstract><![CDATA[The recent growth of agent-based software systems was achieved without the development of a reference architecture. From a software engineering standpoint, a reference architecture is necessary to compare, evaluate, and integrate past, current, and future agent-based software systems. The agent systems reference architecture (ASRA) advances the agent-based system development process by providing a set of key interaction patterns for functional areas that exist between the layers and protocols of agent-based systems. Furthermore, the ASRA identifies the points for interoperability between agent-based systems and increases the level of discussion when referring to agent-based systems. This paper presents methodology, grounded in software forensics, to develop the ASRA and provides an overview of the resulting architectural representation. The methodology uses an approach based on software engineering techniques adapted to study agent frameworks-the libraries and tools for building agent systems. The resulting ASRA can serve as an abstract representation of the components necessary for facilitating comparison, integration, and interoperation of software systems composed of agents.]]></abstract>
##     <issn><![CDATA[2168-2216]]></issn>
##     <htmlFlag><![CDATA[1]]></htmlFlag>
##     <arnumber><![CDATA[6663695]]></arnumber>
##     <doi><![CDATA[10.1109/TSMCC.2013.2263132]]></doi>
##     <publicationId><![CDATA[6663695]]></publicationId>
##     <mdurl><![CDATA[http://ieeexplore.ieee.org/xpl/articleDetails.jsp?tp=&arnumber=6663695&contentType=Journals+%26+Magazines]]></mdurl>
##     <pdf><![CDATA[http://ieeexplore.ieee.org/stamp/stamp.jsp?arnumber=6663695]]></pdf>
##   </document>
##   <document>
##     <rank>965</rank>
##     <title><![CDATA[Fault-Tolerant Two-Stage Open Queuing Systems With Server Failures at Both Stages]]></title>
##     <authors><![CDATA[Ever, E.]]></authors>
##     <affiliations><![CDATA[Comput. Eng. Program, Middle East Tech. Univ., Kalkanh, Turkey]]></affiliations>
##     <controlledterms>
##       <term><![CDATA[computer network reliability]]></term>
##       <term><![CDATA[fault tolerant computing]]></term>
##       <term><![CDATA[internetworking]]></term>
##       <term><![CDATA[network servers]]></term>
##       <term><![CDATA[queueing theory]]></term>
##       <term><![CDATA[radiocommunication]]></term>
##     </controlledterms>
##     <thesaurusterms>
##       <term><![CDATA[Analytical models]]></term>
##       <term><![CDATA[Computational modeling]]></term>
##       <term><![CDATA[Fault tolerance]]></term>
##       <term><![CDATA[Fault tolerant systems]]></term>
##       <term><![CDATA[Markov processes]]></term>
##       <term><![CDATA[Servers]]></term>
##       <term><![CDATA[Wireless communication]]></term>
##     </thesaurusterms>
##     <pubtitle><![CDATA[Communications Letters, IEEE]]></pubtitle>
##     <punumber><![CDATA[4234]]></punumber>
##     <pubtype><![CDATA[Journals & Magazines]]></pubtype>
##     <publisher><![CDATA[IEEE]]></publisher>
##     <volume><![CDATA[18]]></volume>
##     <issue><![CDATA[9]]></issue>
##     <py><![CDATA[2014]]></py>
##     <spage><![CDATA[1523]]></spage>
##     <epage><![CDATA[1526]]></epage>
##     <abstract><![CDATA[Two-stage open queuing systems are used to model and evaluate interaction between two systems in computer and communication networks. Application areas include two-stage internetworking mechanisms, memory servers, and interaction of wireless communication systems. Realistic features, such as finite capacities, feedback from one stage to another, and failures of servers usually complicate analytic solutions. This study presents a new analytical model and solution approach for two-stage open queuing systems with feedback, blocking, and multiple servers, as well as failures at both stages. Unlike the existing studies, systems considered in both stages can be fault tolerant. Numerical results presented comparatively with simulation show that the new approach performs well, in terms of accuracy and computation time.]]></abstract>
##     <issn><![CDATA[1089-7798]]></issn>
##     <htmlFlag><![CDATA[1]]></htmlFlag>
##     <arnumber><![CDATA[6850000]]></arnumber>
##     <doi><![CDATA[10.1109/LCOMM.2014.2336841]]></doi>
##     <publicationId><![CDATA[6850000]]></publicationId>
##     <mdurl><![CDATA[http://ieeexplore.ieee.org/xpl/articleDetails.jsp?tp=&arnumber=6850000&contentType=Journals+%26+Magazines]]></mdurl>
##     <pdf><![CDATA[http://ieeexplore.ieee.org/stamp/stamp.jsp?arnumber=6850000]]></pdf>
##   </document>
##   <document>
##     <rank>966</rank>
##     <title><![CDATA[Signals of Interest Recovery With Multiple Receivers Using Reference-Based Successive Interference Cancellation for Signal Collection Applications]]></title>
##     <authors><![CDATA[Romero, R.A.;  Rios, A.;  Ha, T.T.]]></authors>
##     <affiliations><![CDATA[Naval Postgrad. Sch., Monterey, CA, USA]]></affiliations>
##     <controlledterms>
##       <term><![CDATA[amplitude estimation]]></term>
##       <term><![CDATA[cellular radio]]></term>
##       <term><![CDATA[cochannel interference]]></term>
##       <term><![CDATA[cooperative communication]]></term>
##       <term><![CDATA[error statistics]]></term>
##       <term><![CDATA[interference suppression]]></term>
##       <term><![CDATA[least squares approximations]]></term>
##       <term><![CDATA[phase shift keying]]></term>
##       <term><![CDATA[radio receivers]]></term>
##       <term><![CDATA[signal processing]]></term>
##     </controlledterms>
##     <thesaurusterms>
##       <term><![CDATA[Interchannel interference]]></term>
##       <term><![CDATA[Interference cancellation]]></term>
##       <term><![CDATA[Least squares approximation]]></term>
##       <term><![CDATA[Phase shift keying]]></term>
##       <term><![CDATA[Receivers]]></term>
##       <term><![CDATA[Signal processing]]></term>
##     </thesaurusterms>
##     <pubtitle><![CDATA[Access, IEEE]]></pubtitle>
##     <punumber><![CDATA[6287639]]></punumber>
##     <pubtype><![CDATA[Journals & Magazines]]></pubtype>
##     <publisher><![CDATA[IEEE]]></publisher>
##     <volume><![CDATA[2]]></volume>
##     <py><![CDATA[2014]]></py>
##     <spage><![CDATA[725]]></spage>
##     <epage><![CDATA[756]]></epage>
##     <abstract><![CDATA[In this paper, we introduce a novel but intuitive scheme to recover multiple signals of interest (SoI) from multiple emitters in signal collection applications such as signal intelligence, electronic intelligence, and communications intelligence. We consider a case where the SoIs form a heavy interference environment. The scheme, which is referred to as reference-based successive interference cancellation (RSIC), involves a combination of strategic receiver placement and signal processing techniques. The scheme works by placing a network of cooperative receivers where each receiver catches its own SoI (despite multiple interferences). The first receiver demodulates the initial SoI (called a reference signal) and forwards it to the second receiver. The second receiver collects a received signal containing the second SoI but is interfered with by the initial SoI, which is a problem called co-channel interference in cellular communications. Unfortunately, the amplitude scaling of the interference is unknown in the second receiver and therefore has to be estimated via least squares error. It turns out that the estimation requires a priori knowledge of the second SoI, which is the very signal it tries to demodulate, thereby yielding a Catch-22 problem. We propose using an initial guess on the second SoI to form an amplitude estimate such that the interference is subtracted (cancelled) from the collected measurement at the second receiver. The procedure is applied to a third receiver (or multiple receivers) until the last of the desired SoI is separated from all of the co-channel interferences. The RSIC scheme performs well. Using quaternary phase shift keying as example modulation, we present major symbol error rate (SER) performance improvements with the use of RSIC over the highly degraded SER of receivers that are heavily interfered and do not employ any cancellation technique.]]></abstract>
##     <issn><![CDATA[2169-3536]]></issn>
##     <htmlFlag><![CDATA[1]]></htmlFlag>
##     <arnumber><![CDATA[6860223]]></arnumber>
##     <doi><![CDATA[10.1109/ACCESS.2014.2340874]]></doi>
##     <publicationId><![CDATA[6860223]]></publicationId>
##     <mdurl><![CDATA[http://ieeexplore.ieee.org/xpl/articleDetails.jsp?tp=&arnumber=6860223&contentType=Journals+%26+Magazines]]></mdurl>
##     <pdf><![CDATA[http://ieeexplore.ieee.org/stamp/stamp.jsp?arnumber=6860223]]></pdf>
##   </document>
##   <document>
##     <rank>967</rank>
##     <title><![CDATA[A Planar and Subwavelength Open Guided Wave Structure Based on Spoof Surface Plasmons]]></title>
##     <authors><![CDATA[Liang-Yu Ou Yang;  Cheng-Hao Tsai;  Shih-Yuan Chen]]></authors>
##     <affiliations><![CDATA[Dept. of Electr. Eng., Nat. Taiwan Univ., Taipei, Taiwan]]></affiliations>
##     <controlledterms>
##       <term><![CDATA[S-parameters]]></term>
##       <term><![CDATA[integrated optics]]></term>
##       <term><![CDATA[integrated optoelectronics]]></term>
##       <term><![CDATA[microwave photonics]]></term>
##       <term><![CDATA[optical losses]]></term>
##       <term><![CDATA[optical waveguides]]></term>
##       <term><![CDATA[permittivity]]></term>
##       <term><![CDATA[polaritons]]></term>
##       <term><![CDATA[printed circuits]]></term>
##       <term><![CDATA[surface plasmons]]></term>
##     </controlledterms>
##     <thesaurusterms>
##       <term><![CDATA[Dielectric constant]]></term>
##       <term><![CDATA[Dispersion]]></term>
##       <term><![CDATA[Plasmons]]></term>
##       <term><![CDATA[Slabs]]></term>
##       <term><![CDATA[Strips]]></term>
##       <term><![CDATA[Wires]]></term>
##     </thesaurusterms>
##     <pubtitle><![CDATA[Photonics Journal, IEEE]]></pubtitle>
##     <punumber><![CDATA[4563994]]></punumber>
##     <pubtype><![CDATA[Journals & Magazines]]></pubtype>
##     <publisher><![CDATA[IEEE]]></publisher>
##     <volume><![CDATA[6]]></volume>
##     <issue><![CDATA[6]]></issue>
##     <py><![CDATA[2014]]></py>
##     <spage><![CDATA[1]]></spage>
##     <epage><![CDATA[19]]></epage>
##     <abstract><![CDATA[A planar and compact open waveguiding structure based on spoof surface plasmon polaritons (SPPs) was demonstrated. For practicality, instead of the well-known wire medium, the uniaxial strip medium (USM) was proposed and used as the effective bulk material with a negative dielectric constant to support the spoof SPP modes. The relevant formulations, including the modal dispersion relations and the formulation for the waves in a multilayer anisotropic structure, are analytically presented in this paper. Interestingly, instead of taming and suppressing the spatial dispersion (SD), which had been done in most past studies, SD was exploited in the proposed structure to enhance the field confinement of the spoof SPP mode by approximately 41%. Moreover, the thickness of the USM slab could be reduced by 50%, using conductor backing and without perturbing the odd mode. This method and SD can help avoid electromagnetic interactions among various components of a multilayer printed circuit board structure and help miniaturize sensors or surface-wave waveguides in the microwave regime. In this study, the subwavelength thickness of the proposed structure was only 0.09 &#x03BB;<sub>0</sub> at 1.34 GHz. Additionally, the propagation loss for such slow-wave structures has seldom been discussed analytically and quantitatively. In this study, through calculations and simulations, low attenuation constants in the spoof SPP propagation direction of the proposed structures were investigated. Finally, an experiment was conducted, and an extraction method for obtaining the required reflection spectrum from the measured S-parameter was developed.]]></abstract>
##     <issn><![CDATA[1943-0655]]></issn>
##     <htmlFlag><![CDATA[1]]></htmlFlag>
##     <arnumber><![CDATA[6945241]]></arnumber>
##     <doi><![CDATA[10.1109/JPHOT.2014.2366172]]></doi>
##     <publicationId><![CDATA[6945241]]></publicationId>
##     <mdurl><![CDATA[http://ieeexplore.ieee.org/xpl/articleDetails.jsp?tp=&arnumber=6945241&contentType=Journals+%26+Magazines]]></mdurl>
##     <pdf><![CDATA[http://ieeexplore.ieee.org/stamp/stamp.jsp?arnumber=6945241]]></pdf>
##   </document>
##   <document>
##     <rank>968</rank>
##     <title><![CDATA[Joint Dimensioning of Server and Network Infrastructure for Resilient Optical Grids/Clouds]]></title>
##     <authors><![CDATA[Develder, C.;  Buysse, J.;  Dhoedt, B.;  Jaumard, B.]]></authors>
##     <affiliations><![CDATA[Dept. of Inf. Technol., Ghent Univ., Ghent, Belgium]]></affiliations>
##     <controlledterms>
##       <term><![CDATA[cloud computing]]></term>
##       <term><![CDATA[grid computing]]></term>
##       <term><![CDATA[integer programming]]></term>
##       <term><![CDATA[linear programming]]></term>
##       <term><![CDATA[network servers]]></term>
##       <term><![CDATA[optical fibre networks]]></term>
##       <term><![CDATA[optical links]]></term>
##       <term><![CDATA[telecommunication computing]]></term>
##       <term><![CDATA[telecommunication network reliability]]></term>
##       <term><![CDATA[telecommunication network routing]]></term>
##       <term><![CDATA[telecommunication network topology]]></term>
##       <term><![CDATA[virtual machines]]></term>
##     </controlledterms>
##     <thesaurusterms>
##       <term><![CDATA[Bandwidth]]></term>
##       <term><![CDATA[Data models]]></term>
##       <term><![CDATA[Network topology]]></term>
##       <term><![CDATA[Optical fiber networks]]></term>
##       <term><![CDATA[Routing]]></term>
##       <term><![CDATA[Servers]]></term>
##       <term><![CDATA[WDM networks]]></term>
##     </thesaurusterms>
##     <pubtitle><![CDATA[Networking, IEEE/ACM Transactions on]]></pubtitle>
##     <punumber><![CDATA[90]]></punumber>
##     <pubtype><![CDATA[Journals & Magazines]]></pubtype>
##     <publisher><![CDATA[IEEE]]></publisher>
##     <volume><![CDATA[22]]></volume>
##     <issue><![CDATA[5]]></issue>
##     <py><![CDATA[2014]]></py>
##     <spage><![CDATA[1591]]></spage>
##     <epage><![CDATA[1606]]></epage>
##     <abstract><![CDATA[We address the dimensioning of infrastructure, comprising both network and server resources, for large-scale decentralized distributed systems such as grids or clouds. We design the resulting grid/cloud to be resilient against network link or server failures. To this end, we exploit relocation: Under failure conditions, a grid job or cloud virtual machine may be served at an alternate destination (i.e., different from the one under failure-free conditions). We thus consider grid/cloud requests to have a known origin, but assume a degree of freedom as to where they end up being served, which is the case for grid applications of the bag-of-tasks (BoT) type or hosted virtual machines in the cloud case. We present a generic methodology based on integer linear programming (ILP) that: chooses a given number of sites in a given network topology where to install server infrastructure; and determines the amount of both network and server capacity to cater for both the failure-free scenario and failures of links or nodes. For the latter, we consider either failure-independent (FID) or failure-dependent (FD) recovery. Case studies on European-scale networks show that relocation allows considerable reduction of the total amount of network and server resources, especially in sparse topologies and for higher numbers of server sites. Adopting a failure-dependent backup routing strategy does lead to lower resource dimensions, but only when we adopt relocation (especially for a high number of server sites): Without exploiting relocation, potential savings of FD versus FID are not meaningful.]]></abstract>
##     <issn><![CDATA[1063-6692]]></issn>
##     <htmlFlag><![CDATA[1]]></htmlFlag>
##     <arnumber><![CDATA[6642129]]></arnumber>
##     <doi><![CDATA[10.1109/TNET.2013.2283924]]></doi>
##     <publicationId><![CDATA[6642129]]></publicationId>
##     <mdurl><![CDATA[http://ieeexplore.ieee.org/xpl/articleDetails.jsp?tp=&arnumber=6642129&contentType=Journals+%26+Magazines]]></mdurl>
##     <pdf><![CDATA[http://ieeexplore.ieee.org/stamp/stamp.jsp?arnumber=6642129]]></pdf>
##   </document>
##   <document>
##     <rank>969</rank>
##     <title><![CDATA[Priority- and Delay-Aware Medium Access for Wireless Sensor Networks in the Smart Grid]]></title>
##     <authors><![CDATA[Al-Anbagi, I.;  Erol-Kantarci, M.;  Mouftah, H.T.]]></authors>
##     <affiliations><![CDATA[Sch. of Electr. Eng. & Comput. Sci., Univ. of Ottawa, Ottawa, ON, Canada]]></affiliations>
##     <controlledterms>
##       <term><![CDATA[access protocols]]></term>
##       <term><![CDATA[data communication]]></term>
##       <term><![CDATA[delay estimation]]></term>
##       <term><![CDATA[power system measurement]]></term>
##       <term><![CDATA[smart power grids]]></term>
##       <term><![CDATA[wireless channels]]></term>
##       <term><![CDATA[wireless sensor networks]]></term>
##     </controlledterms>
##     <thesaurusterms>
##       <term><![CDATA[Delays]]></term>
##       <term><![CDATA[IEEE 802.15 Standards]]></term>
##       <term><![CDATA[Monitoring]]></term>
##       <term><![CDATA[Multiaccess communication]]></term>
##       <term><![CDATA[Reliability]]></term>
##       <term><![CDATA[Smart grids]]></term>
##       <term><![CDATA[Wireless sensor networks]]></term>
##     </thesaurusterms>
##     <pubtitle><![CDATA[Systems Journal, IEEE]]></pubtitle>
##     <punumber><![CDATA[4267003]]></punumber>
##     <pubtype><![CDATA[Journals & Magazines]]></pubtype>
##     <publisher><![CDATA[IEEE]]></publisher>
##     <volume><![CDATA[8]]></volume>
##     <issue><![CDATA[2]]></issue>
##     <py><![CDATA[2014]]></py>
##     <spage><![CDATA[608]]></spage>
##     <epage><![CDATA[618]]></epage>
##     <abstract><![CDATA[Monitoring smart-grid assets in a timely manner is highly desired for emerging smart-grid applications such as transformer monitoring, capacitor bank control, plug-in hybrid-electric-vehicle load management, and power quality assessment. Wireless sensor and actor networks (WSANs) are anticipated to be widely utilized in a wide range of smart-grid applications due to their numerous advantages along with their successful adoption in various critical areas including military and health. For resource-constrained WSANs, transmitting delay-critical data from smart-grid assets calls for data prioritization and delay responsiveness. In this paper, we introduce two medium-access approaches, namely, delay-responsive cross-layer (DRX) data transmission and fair and delay-aware cross-layer (FDRX) data transmission, which aim to address the delay and service requirements of smart grids. DRX is based on delay-estimation and data-prioritization steps that are performed by the application layer, in addition to the MAC layer parameters responding to the delay requirements of the smart-grid application and the network condition. On the other hand, FDRX incorporates fairness into DRX by preventing a few nodes from dominating the communication channel. We provide a comprehensive performance evaluation of those approaches. We show that DRX reduces the end-to-end delay while FDRX has lower collision rate compared with DRX. We outline the tradeoffs regarding these approaches and draw future research directions for robust communication protocols for smart-grid monitoring applications.]]></abstract>
##     <issn><![CDATA[1932-8184]]></issn>
##     <htmlFlag><![CDATA[1]]></htmlFlag>
##     <arnumber><![CDATA[6607219]]></arnumber>
##     <doi><![CDATA[10.1109/JSYST.2013.2260939]]></doi>
##     <publicationId><![CDATA[6607219]]></publicationId>
##     <mdurl><![CDATA[http://ieeexplore.ieee.org/xpl/articleDetails.jsp?tp=&arnumber=6607219&contentType=Journals+%26+Magazines]]></mdurl>
##     <pdf><![CDATA[http://ieeexplore.ieee.org/stamp/stamp.jsp?arnumber=6607219]]></pdf>
##   </document>
##   <document>
##     <rank>970</rank>
##     <title><![CDATA[Observation of Three Bound States From a Topological Insulator Mode-Locked Soliton Fiber Laser]]></title>
##     <authors><![CDATA[Ai-Ping Luo;  Hao Liu;  Nian Zhao;  Xu-Wu Zheng;  Meng Liu;  Rui Tang;  Zhi-Chao Luo;  Wen-Cheng Xu]]></authors>
##     <affiliations><![CDATA[Lab. of Nanophotonic Functional Mater. & Devices, South China Normal Univ., Guangzhou, China]]></affiliations>
##     <controlledterms>
##       <term><![CDATA[bound states]]></term>
##       <term><![CDATA[fibre lasers]]></term>
##       <term><![CDATA[laser mode locking]]></term>
##       <term><![CDATA[optical solitons]]></term>
##       <term><![CDATA[topological insulators]]></term>
##     </controlledterms>
##     <thesaurusterms>
##       <term><![CDATA[Correlation]]></term>
##       <term><![CDATA[Fiber lasers]]></term>
##       <term><![CDATA[Laser mode locking]]></term>
##       <term><![CDATA[Modulation]]></term>
##       <term><![CDATA[Optical fiber devices]]></term>
##       <term><![CDATA[Optical fiber polarization]]></term>
##       <term><![CDATA[Solitons]]></term>
##     </thesaurusterms>
##     <pubtitle><![CDATA[Photonics Journal, IEEE]]></pubtitle>
##     <punumber><![CDATA[4563994]]></punumber>
##     <pubtype><![CDATA[Journals & Magazines]]></pubtype>
##     <publisher><![CDATA[IEEE]]></publisher>
##     <volume><![CDATA[6]]></volume>
##     <issue><![CDATA[4]]></issue>
##     <py><![CDATA[2014]]></py>
##     <spage><![CDATA[1]]></spage>
##     <epage><![CDATA[8]]></epage>
##     <abstract><![CDATA[We report on the generation of three types of bound states in a topological insulator-based mode-locked fiber laser. Benefiting from the excellent performance of the polyvinyl alcohol-based topological insulator saturable absorber (SA), we have achieved three types of bound solitons. The bound states exhibit different features both in spectra and autocorrelation traces, which was found to be caused by the different peak intensities and pulsewidths of two closed pulses within the bound states. The experimental phenomena were further confirmed by the numerical simulations. The results reveal an alternative method to roughly estimate the constructed solitons within bound states and indicate that the topological insulator could serve as a highly nonlinear SA for observing soliton dynamics in fiber lasers.]]></abstract>
##     <issn><![CDATA[1943-0655]]></issn>
##     <htmlFlag><![CDATA[1]]></htmlFlag>
##     <arnumber><![CDATA[6872820]]></arnumber>
##     <doi><![CDATA[10.1109/JPHOT.2014.2345874]]></doi>
##     <publicationId><![CDATA[6872820]]></publicationId>
##     <mdurl><![CDATA[http://ieeexplore.ieee.org/xpl/articleDetails.jsp?tp=&arnumber=6872820&contentType=Journals+%26+Magazines]]></mdurl>
##     <pdf><![CDATA[http://ieeexplore.ieee.org/stamp/stamp.jsp?arnumber=6872820]]></pdf>
##   </document>
##   <document>
##     <rank>971</rank>
##     <title><![CDATA[Synergizing Wireless Communication Technologies to Improve Internet Downloading Experiences]]></title>
##     <authors><![CDATA[Ting-Yu Lin;  Tai-Yi Huang;  Chia-Fu Hsu]]></authors>
##     <affiliations><![CDATA[Dept. of Electr. & Comput. Eng., Nat. Chiao Tung Univ., Hsinchu, Taiwan]]></affiliations>
##     <controlledterms>
##       <term><![CDATA[Internet]]></term>
##       <term><![CDATA[cooperative communication]]></term>
##       <term><![CDATA[routing protocols]]></term>
##       <term><![CDATA[telecommunication traffic]]></term>
##       <term><![CDATA[wireless channels]]></term>
##     </controlledterms>
##     <thesaurusterms>
##       <term><![CDATA[Ad hoc networks]]></term>
##       <term><![CDATA[Downlink]]></term>
##       <term><![CDATA[Internet]]></term>
##       <term><![CDATA[Logic gates]]></term>
##       <term><![CDATA[Routing protocols]]></term>
##       <term><![CDATA[Spread spectrum communication]]></term>
##       <term><![CDATA[Wireless communication]]></term>
##     </thesaurusterms>
##     <pubtitle><![CDATA[Computers, IEEE Transactions on]]></pubtitle>
##     <punumber><![CDATA[12]]></punumber>
##     <pubtype><![CDATA[Journals & Magazines]]></pubtype>
##     <publisher><![CDATA[IEEE]]></publisher>
##     <volume><![CDATA[63]]></volume>
##     <issue><![CDATA[11]]></issue>
##     <py><![CDATA[2014]]></py>
##     <spage><![CDATA[2851]]></spage>
##     <epage><![CDATA[2865]]></epage>
##     <abstract><![CDATA[Considering downloading traffic from the internet, in this paper, we propose a synergized framework (SF), consisting of heterogeneous wireless communication technologies, multi-mode (multi-interface) mobile, and fixed wireless hosts capable of operating over multiple orthogonal (non-overlapping) radio channels, to realize better downloading experiences for users via cooperation between different wireless technologies. An SyNerGized (SNG) routing protocol is devised to enable the proposed framework. Given perceived network information, SNG performs computations based on linear formulations and obtains an optimized route for packet delivery. To adapt to network dynamics, a reactive version of SNG, entitled Reactive SyNerGized (RSNG) routing protocol, is proposed to alleviate the network from constantly keeping track of link capacities within a certain scope of neighborhood. Since the downloading throughput may be bounded by either the internet gateway capacity &#x03BB;<sub>gw</sub> or ad hoc throughput &#x03BB;<sub>ah</sub>, RSNG judiciously propagates Route REQuest (RREQ) until the downloading throughput is bounded by &#x03BB;<sub>ah</sub> over the ad hoc domain, effectively eliminating unnecessary RREQ flooding. Our main objective is to improve achieved user downloading throughput via the cooperative (synergized) communication model and its corresponding routing mechanisms. Simulation results demonstrate the benefits brought by the unified architecture and corroborate the efficacy of the proposed routing techniques.]]></abstract>
##     <issn><![CDATA[0018-9340]]></issn>
##     <htmlFlag><![CDATA[1]]></htmlFlag>
##     <arnumber><![CDATA[6565983]]></arnumber>
##     <doi><![CDATA[10.1109/TC.2013.147]]></doi>
##     <publicationId><![CDATA[6565983]]></publicationId>
##     <mdurl><![CDATA[http://ieeexplore.ieee.org/xpl/articleDetails.jsp?tp=&arnumber=6565983&contentType=Journals+%26+Magazines]]></mdurl>
##     <pdf><![CDATA[http://ieeexplore.ieee.org/stamp/stamp.jsp?arnumber=6565983]]></pdf>
##   </document>
##   <document>
##     <rank>972</rank>
##     <title><![CDATA[Automated Control of an Adaptive Bihormonal, Dual-Sensor Artificial Pancreas and Evaluation During Inpatient Studies]]></title>
##     <authors><![CDATA[Jacobs, P.G.;  El Youssef, J.;  Castle, J.;  Bakhtiani, P.;  Branigan, D.;  Breen, M.;  Bauer, D.;  Preiser, N.;  Leonard, G.;  Stonex, T.;  Ward, W.K.]]></authors>
##     <affiliations><![CDATA[Oregon Health & Sci. Univ., Portland, OR, USA]]></affiliations>
##     <controlledterms>
##       <term><![CDATA[PD control]]></term>
##       <term><![CDATA[artificial organs]]></term>
##       <term><![CDATA[biomedical measurement]]></term>
##       <term><![CDATA[blood]]></term>
##       <term><![CDATA[medical control systems]]></term>
##       <term><![CDATA[sensors]]></term>
##       <term><![CDATA[sugar]]></term>
##     </controlledterms>
##     <thesaurusterms>
##       <term><![CDATA[Adaptation models]]></term>
##       <term><![CDATA[Equations]]></term>
##       <term><![CDATA[Insulin]]></term>
##       <term><![CDATA[Mathematical model]]></term>
##       <term><![CDATA[Sensitivity]]></term>
##       <term><![CDATA[Sensors]]></term>
##       <term><![CDATA[Sugar]]></term>
##     </thesaurusterms>
##     <pubtitle><![CDATA[Biomedical Engineering, IEEE Transactions on]]></pubtitle>
##     <punumber><![CDATA[10]]></punumber>
##     <pubtype><![CDATA[Journals & Magazines]]></pubtype>
##     <publisher><![CDATA[IEEE]]></publisher>
##     <volume><![CDATA[61]]></volume>
##     <issue><![CDATA[10]]></issue>
##     <py><![CDATA[2014]]></py>
##     <spage><![CDATA[2569]]></spage>
##     <epage><![CDATA[2581]]></epage>
##     <abstract><![CDATA[Automated control of blood glucose in patients with type-1 diabetes has not yet been fully implemented. The aim of this study was to design and clinically evaluate a system that integrates a control algorithm with off-the-shelf subcutaneous sensors and pumps to automate the delivery of the hormones glucagon and insulin in response to continuous glucose sensor measurements. The automated component of the system runs an adaptive proportional derivative control algorithm which determines hormone delivery rates based on the sensed glucose measurements and the meal announcements by the patient. We provide details about the system design and the control algorithm, which incorporates both a fading memory proportional derivative controller (FMPD) and an adaptive system for estimating changing sensitivity to insulin based on a glucoregulatory model of insulin action. For an inpatient study carried out in eight subjects using Dexcom SEVEN PLUS sensors, prestudy HbA1c averaged 7.6, which translates to an estimated average glucose of 171 mg/dL. In contrast, during use of the automated system, after initial stabilization, glucose averaged 145 mg/dL and subjects were kept within the euglycemic range (between 70 and 180 mg/dL) for 73.1% of the time, indicating improved glycemic control. A further study on five additional subjects in which we used a newer and more reliable glucose sensor (Dexcom G4 PLATINUM) and made improvements to the insulin and glucagon pump communication system resulted in elimination of hypoglycemic events. For this G4 study, the system was able to maintain subjects' glucose levels within the near-euglycemic range for 71.6% of the study duration and the mean venous glucose level was 151 mg/dL.]]></abstract>
##     <issn><![CDATA[0018-9294]]></issn>
##     <htmlFlag><![CDATA[1]]></htmlFlag>
##     <arnumber><![CDATA[6814769]]></arnumber>
##     <doi><![CDATA[10.1109/TBME.2014.2323248]]></doi>
##     <publicationId><![CDATA[6814769]]></publicationId>
##     <mdurl><![CDATA[http://ieeexplore.ieee.org/xpl/articleDetails.jsp?tp=&arnumber=6814769&contentType=Journals+%26+Magazines]]></mdurl>
##     <pdf><![CDATA[http://ieeexplore.ieee.org/stamp/stamp.jsp?arnumber=6814769]]></pdf>
##   </document>
##   <document>
##     <rank>973</rank>
##     <title><![CDATA[Low-Sidelobe Antenna Beamforming Via Stochastic Optimization]]></title>
##     <authors><![CDATA[Krich, S.I.;  Weiner, I.]]></authors>
##     <affiliations><![CDATA[Massachusetts Inst. of Technol. Lincoln Lab., Lexington, MA, USA]]></affiliations>
##     <controlledterms>
##       <term><![CDATA[airborne radar]]></term>
##       <term><![CDATA[antenna phased arrays]]></term>
##       <term><![CDATA[iterative methods]]></term>
##       <term><![CDATA[optimisation]]></term>
##       <term><![CDATA[radar antennas]]></term>
##       <term><![CDATA[radar clutter]]></term>
##     </controlledterms>
##     <thesaurusterms>
##       <term><![CDATA[Clutter]]></term>
##       <term><![CDATA[Linear programming]]></term>
##       <term><![CDATA[Optimization]]></term>
##       <term><![CDATA[Radar]]></term>
##       <term><![CDATA[Radar antennas]]></term>
##       <term><![CDATA[Vectors]]></term>
##     </thesaurusterms>
##     <pubtitle><![CDATA[Antennas and Propagation, IEEE Transactions on]]></pubtitle>
##     <punumber><![CDATA[8]]></punumber>
##     <pubtype><![CDATA[Journals & Magazines]]></pubtype>
##     <publisher><![CDATA[IEEE]]></publisher>
##     <volume><![CDATA[62]]></volume>
##     <issue><![CDATA[12]]></issue>
##     <py><![CDATA[2014]]></py>
##     <spage><![CDATA[6482]]></spage>
##     <epage><![CDATA[6486]]></epage>
##     <abstract><![CDATA[We describe a novel iterative methodology for computing a set of low-sidelobe beamforming weights for an airborne, electronically-steered phased-array radar using an in-flight stochastic optimization routine performed over a number of coherent processing intervals (CPIs). The proposed approach is notable in that it does not rely upon a good antenna calibration and only requires digitization of the radar's sum beam. By observing the radar ground clutter, the algorithm iteratively adjusts the beamformer. Furthermore, it is computationally inexpensive and scales favorably to radars comprising very large numbers of antenna elements and requiring extremely low sidelobes.]]></abstract>
##     <issn><![CDATA[0018-926X]]></issn>
##     <htmlFlag><![CDATA[1]]></htmlFlag>
##     <arnumber><![CDATA[6905775]]></arnumber>
##     <doi><![CDATA[10.1109/TAP.2014.2359202]]></doi>
##     <publicationId><![CDATA[6905775]]></publicationId>
##     <mdurl><![CDATA[http://ieeexplore.ieee.org/xpl/articleDetails.jsp?tp=&arnumber=6905775&contentType=Journals+%26+Magazines]]></mdurl>
##     <pdf><![CDATA[http://ieeexplore.ieee.org/stamp/stamp.jsp?arnumber=6905775]]></pdf>
##   </document>
##   <document>
##     <rank>974</rank>
##     <title><![CDATA[Counterfeit Integrated Circuits: A Rising Threat in the Global Semiconductor Supply Chain]]></title>
##     <authors><![CDATA[Guin, U.;  Ke Huang;  DiMase, D.;  Carulli, J.M.;  Tehranipoor, M.;  Makris, Y.]]></authors>
##     <affiliations><![CDATA[Dept. of Electr. & Comput. Eng., Univ. of Connecticut, Storrs, CT, USA]]></affiliations>
##     <controlledterms>
##       <term><![CDATA[integrated circuits]]></term>
##       <term><![CDATA[reliability]]></term>
##       <term><![CDATA[risk management]]></term>
##       <term><![CDATA[supply chain management]]></term>
##     </controlledterms>
##     <thesaurusterms>
##       <term><![CDATA[Consumer electronics]]></term>
##       <term><![CDATA[Globalization]]></term>
##       <term><![CDATA[Hardware]]></term>
##       <term><![CDATA[Integrated circuit modeling]]></term>
##       <term><![CDATA[Integrated circuits]]></term>
##       <term><![CDATA[Semiconductor device measurement]]></term>
##       <term><![CDATA[Semiconductor devices]]></term>
##       <term><![CDATA[Supply chain management]]></term>
##       <term><![CDATA[Supply chains]]></term>
##     </thesaurusterms>
##     <pubtitle><![CDATA[Proceedings of the IEEE]]></pubtitle>
##     <punumber><![CDATA[5]]></punumber>
##     <pubtype><![CDATA[Journals & Magazines]]></pubtype>
##     <publisher><![CDATA[IEEE]]></publisher>
##     <volume><![CDATA[102]]></volume>
##     <issue><![CDATA[8]]></issue>
##     <py><![CDATA[2014]]></py>
##     <spage><![CDATA[1207]]></spage>
##     <epage><![CDATA[1228]]></epage>
##     <abstract><![CDATA[As the electronic component supply chain grows more complex due to globalization, with parts coming from a diverse set of suppliers, counterfeit electronics have become a major challenge that calls for immediate solutions. Currently, there are a few standards and programs available that address the testing for such counterfeit parts. However, not enough research has yet addressed the detection and avoidance of all counterfeit parts-recycled, remarked, overproduced, cloned, out-of-spec/defective, and forged documentation-currently infiltrating the electronic component supply chain. Even if they work initially, all these parts may have reduced lifetime and pose reliability risks. In this tutorial, we will provide a review of some of the existing counterfeit detection and avoidance methods. We will also discuss the challenges ahead for implementing these methods, as well as the development of new detection and avoidance mechanisms.]]></abstract>
##     <issn><![CDATA[0018-9219]]></issn>
##     <htmlFlag><![CDATA[1]]></htmlFlag>
##     <arnumber><![CDATA[6856206]]></arnumber>
##     <doi><![CDATA[10.1109/JPROC.2014.2332291]]></doi>
##     <publicationId><![CDATA[6856206]]></publicationId>
##     <mdurl><![CDATA[http://ieeexplore.ieee.org/xpl/articleDetails.jsp?tp=&arnumber=6856206&contentType=Journals+%26+Magazines]]></mdurl>
##     <pdf><![CDATA[http://ieeexplore.ieee.org/stamp/stamp.jsp?arnumber=6856206]]></pdf>
##   </document>
##   <document>
##     <rank>975</rank>
##     <title><![CDATA[Review on the History, Research, and Applications of Electrohydrodynamics]]></title>
##     <authors><![CDATA[Fylladitakis, E.D.;  Theodoridis, M.P.;  Moronis, A.X.]]></authors>
##     <affiliations><![CDATA[Dept. of Electron. & Comput. Eng., Brunel Univ., Uxbridge, UK]]></affiliations>
##     <controlledterms>
##       <term><![CDATA[cooling]]></term>
##       <term><![CDATA[corona]]></term>
##       <term><![CDATA[direct energy conversion]]></term>
##       <term><![CDATA[drying]]></term>
##       <term><![CDATA[electrohydrodynamics]]></term>
##       <term><![CDATA[ionisation]]></term>
##       <term><![CDATA[plasma boundary layers]]></term>
##       <term><![CDATA[plasma devices]]></term>
##       <term><![CDATA[reviews]]></term>
##     </controlledterms>
##     <thesaurusterms>
##       <term><![CDATA[Computational modeling]]></term>
##       <term><![CDATA[Corona]]></term>
##       <term><![CDATA[Discharges (electric)]]></term>
##       <term><![CDATA[Electrodes]]></term>
##       <term><![CDATA[Finite element analysis]]></term>
##       <term><![CDATA[Geometry]]></term>
##       <term><![CDATA[Heat transfer]]></term>
##     </thesaurusterms>
##     <pubtitle><![CDATA[Plasma Science, IEEE Transactions on]]></pubtitle>
##     <punumber><![CDATA[27]]></punumber>
##     <pubtype><![CDATA[Journals & Magazines]]></pubtype>
##     <publisher><![CDATA[IEEE]]></publisher>
##     <volume><![CDATA[42]]></volume>
##     <issue><![CDATA[2]]></issue>
##     <py><![CDATA[2014]]></py>
##     <spage><![CDATA[358]]></spage>
##     <epage><![CDATA[375]]></epage>
##     <abstract><![CDATA[Corona discharge refers to the phenomenon when the electric field near a conductor is strong enough to ionize the dielectric surrounding it but not strong enough to cause an electrical breakdown or arcing between conductors or other components. This phenomenon is unwanted and dangerous in high-voltage systems; however, a controlled corona discharge may be used to ionize a fluid and induce motion by directly converting the electrical energy into kinetic energy. Phenomena that involve the direct conversion of electrical energy into kinetic energy are known as electrohydrodynamic (EHD) and have a variety of possible applications today. This paper contains a literature review of the research regarding the EHD effects associated with corona discharges, from the first observation of the phenomenon to the most recent advancements on its mathematical modeling, as well as the advancements on specific applications, such as thrust, heat transfer improvement, boundary layer enhancement, drying, fluid pumping, and cooling.]]></abstract>
##     <issn><![CDATA[0093-3813]]></issn>
##     <htmlFlag><![CDATA[1]]></htmlFlag>
##     <arnumber><![CDATA[6714477]]></arnumber>
##     <doi><![CDATA[10.1109/TPS.2013.2297173]]></doi>
##     <publicationId><![CDATA[6714477]]></publicationId>
##     <mdurl><![CDATA[http://ieeexplore.ieee.org/xpl/articleDetails.jsp?tp=&arnumber=6714477&contentType=Journals+%26+Magazines]]></mdurl>
##     <pdf><![CDATA[http://ieeexplore.ieee.org/stamp/stamp.jsp?arnumber=6714477]]></pdf>
##   </document>
##   <document>
##     <rank>976</rank>
##     <title><![CDATA[Estimating the Statistical Characteristics of Remote Sensing Big Data in the Wavelet Transform Domain]]></title>
##     <authors><![CDATA[Lizhe Wang;  Hui Zhong;  Ranjan, R.;  Zomaya, A.;  Peng Liu]]></authors>
##     <affiliations><![CDATA[Inst. of Remote Sensing & Digital Earth, Beijing, China]]></affiliations>
##     <controlledterms>
##       <term><![CDATA[Big Data]]></term>
##       <term><![CDATA[Gaussian processes]]></term>
##       <term><![CDATA[geophysics computing]]></term>
##       <term><![CDATA[image resolution]]></term>
##       <term><![CDATA[mixture models]]></term>
##       <term><![CDATA[remote sensing]]></term>
##       <term><![CDATA[statistical analysis]]></term>
##       <term><![CDATA[wavelet transforms]]></term>
##     </controlledterms>
##     <thesaurusterms>
##       <term><![CDATA[Big data]]></term>
##       <term><![CDATA[Earth]]></term>
##       <term><![CDATA[Hidden Markov models]]></term>
##       <term><![CDATA[Multiresolution analysis]]></term>
##       <term><![CDATA[Remote sensing]]></term>
##       <term><![CDATA[Satellites]]></term>
##       <term><![CDATA[Wavelet transforms]]></term>
##     </thesaurusterms>
##     <pubtitle><![CDATA[Emerging Topics in Computing, IEEE Transactions on]]></pubtitle>
##     <punumber><![CDATA[6245516]]></punumber>
##     <pubtype><![CDATA[Journals & Magazines]]></pubtype>
##     <publisher><![CDATA[IEEE]]></publisher>
##     <volume><![CDATA[2]]></volume>
##     <issue><![CDATA[3]]></issue>
##     <py><![CDATA[2014]]></py>
##     <spage><![CDATA[324]]></spage>
##     <epage><![CDATA[337]]></epage>
##     <abstract><![CDATA[Since it is difficult to deal with big data using traditional models and algorithms, predicting and estimating the characteristics of big data is very important. Remote sensing big data consist of many large-scale images that are extremely complex in terms of their structural, spectral, and textual features. Based on multiresolution analysis theory, most of the natural images are sparse and have obvious clustering and persistence characters when they are transformed into another domain by a group of basic special functions. In this paper, we use a wavelet transform to represent remote sensing big data that are large scale in the space domain, correlated in the spectral domain, and continuous in the time domain. We decompose the big data set into approximate multiscale detail coefficients based on a wavelet transform. In order to determine whether the density function of wavelet coefficients in a big data set are peaky at zero and have a heavy tailed shape, a two-component Gaussian mixture model (GMM) is employed. For the first time, we use the expectation-maximization likelihood method to estimate the model parameters for the remote sensing big data set in the wavelet domain. The variance of the GMM with changing of bands, time, and scale are comprehensively analyzed. The statistical characteristics of different textures are also compared. We find that the cluster characteristics of the wavelet coefficients are still obvious in the remote sensing big data set for different bands and different scales. However, it is not always precise when we model the long-term sequence data set using the GMM. We also found that the scale features of different textures for the big data set are obviously reflected in the probability density function and GMM parameters of the wavelet coefficients.]]></abstract>
##     <issn><![CDATA[2168-6750]]></issn>
##     <htmlFlag><![CDATA[1]]></htmlFlag>
##     <arnumber><![CDATA[6897952]]></arnumber>
##     <doi><![CDATA[10.1109/TETC.2014.2356499]]></doi>
##     <publicationId><![CDATA[6897952]]></publicationId>
##     <mdurl><![CDATA[http://ieeexplore.ieee.org/xpl/articleDetails.jsp?tp=&arnumber=6897952&contentType=Journals+%26+Magazines]]></mdurl>
##     <pdf><![CDATA[http://ieeexplore.ieee.org/stamp/stamp.jsp?arnumber=6897952]]></pdf>
##   </document>
##   <document>
##     <rank>977</rank>
##     <title><![CDATA[On-the-Fly Massively Multitemporal Change Detection Using Statistical Quality Control Charts and Landsat Data]]></title>
##     <authors><![CDATA[Brooks, E.B.;  Wynne, R.H.;  Thomas, V.A.;  Blinn, C.E.;  Coulston, J.W.]]></authors>
##     <affiliations><![CDATA[Dept. of Forest Resources & Environ. Conservation, Virginia Polytech. Inst. & State Univ., Blacksburg, VA, USA]]></affiliations>
##     <controlledterms>
##       <term><![CDATA[clouds]]></term>
##       <term><![CDATA[control charts]]></term>
##       <term><![CDATA[forestry]]></term>
##       <term><![CDATA[land use]]></term>
##       <term><![CDATA[quality control]]></term>
##       <term><![CDATA[regression analysis]]></term>
##       <term><![CDATA[time series]]></term>
##       <term><![CDATA[vegetation mapping]]></term>
##     </controlledterms>
##     <thesaurusterms>
##       <term><![CDATA[Control charts]]></term>
##       <term><![CDATA[Earth]]></term>
##       <term><![CDATA[Harmonic analysis]]></term>
##       <term><![CDATA[Indexes]]></term>
##       <term><![CDATA[Remote sensing]]></term>
##       <term><![CDATA[Satellites]]></term>
##       <term><![CDATA[Time series analysis]]></term>
##     </thesaurusterms>
##     <pubtitle><![CDATA[Geoscience and Remote Sensing, IEEE Transactions on]]></pubtitle>
##     <punumber><![CDATA[36]]></punumber>
##     <pubtype><![CDATA[Journals & Magazines]]></pubtype>
##     <publisher><![CDATA[IEEE]]></publisher>
##     <volume><![CDATA[52]]></volume>
##     <issue><![CDATA[6]]></issue>
##     <py><![CDATA[2014]]></py>
##     <spage><![CDATA[3316]]></spage>
##     <epage><![CDATA[3332]]></epage>
##     <abstract><![CDATA[One challenge to implementing spectral change detection algorithms using multitemporal Landsat data is that key dates and periods are often missing from the record due to weather disturbances and lapses in continuous coverage. This paper presents a method that utilizes residuals from harmonic regression over years of Landsat data, in conjunction with statistical quality control charts, to signal subtle disturbances in vegetative cover. These charts are able to detect changes from both deforestation and subtler forest degradation and thinning. First, harmonic regression residuals are computed after fitting models to interannual training data. These residual time series are then subjected to Shewhart X-bar control charts and exponentially weighted moving average charts. The Shewhart X-bar charts are also utilized in the algorithm to generate a data-driven cloud filter, effectively removing clouds and cloud shadows on a location-specific basis. Disturbed pixels are indicated when the charts signal a deviation from data-driven control limits. The methods are applied to a collection of loblolly pine ( Pinus taeda) stands in Alabama, USA. The results are compared with stands for which known thinning has occurred at known times. The method yielded an overall accuracy of 85%, with the particular result that it provided afforestation/deforestation maps on a per-image basis, producing new maps with each successive incorporated image. These maps matched very well with observed changes in aerial photography over the test period. Accordingly, the method is highly recommended for on-the-fly change detection, for changes in both land use and land management within a given land use.]]></abstract>
##     <issn><![CDATA[0196-2892]]></issn>
##     <htmlFlag><![CDATA[1]]></htmlFlag>
##     <arnumber><![CDATA[6573358]]></arnumber>
##     <doi><![CDATA[10.1109/TGRS.2013.2272545]]></doi>
##     <publicationId><![CDATA[6573358]]></publicationId>
##     <mdurl><![CDATA[http://ieeexplore.ieee.org/xpl/articleDetails.jsp?tp=&arnumber=6573358&contentType=Journals+%26+Magazines]]></mdurl>
##     <pdf><![CDATA[http://ieeexplore.ieee.org/stamp/stamp.jsp?arnumber=6573358]]></pdf>
##   </document>
##   <document>
##     <rank>978</rank>
##     <title><![CDATA[Predicting Protein Relationships to Human Pathways through a Relational Learning Approach Based on Simple Sequence Features]]></title>
##     <authors><![CDATA[Garcia-Jimenez, B.;  Pons, T.;  Sanchis, A.;  Valencia, A.]]></authors>
##     <affiliations><![CDATA[Comput. Sci. Dept., Univ. Carlos III de Madrid, Legane&#x0301;s, Spain]]></affiliations>
##     <controlledterms>
##       <term><![CDATA[DNA]]></term>
##       <term><![CDATA[biochemistry]]></term>
##       <term><![CDATA[bioinformatics]]></term>
##       <term><![CDATA[cellular biophysics]]></term>
##       <term><![CDATA[feature extraction]]></term>
##       <term><![CDATA[genomics]]></term>
##       <term><![CDATA[learning (artificial intelligence)]]></term>
##       <term><![CDATA[molecular biophysics]]></term>
##       <term><![CDATA[molecular configurations]]></term>
##       <term><![CDATA[pattern classification]]></term>
##       <term><![CDATA[proteins]]></term>
##     </controlledterms>
##     <thesaurusterms>
##       <term><![CDATA[Bioinformatics]]></term>
##       <term><![CDATA[Computational biology]]></term>
##       <term><![CDATA[Databases]]></term>
##       <term><![CDATA[Decision trees]]></term>
##       <term><![CDATA[Prediction algorithms]]></term>
##       <term><![CDATA[Proteins]]></term>
##     </thesaurusterms>
##     <pubtitle><![CDATA[Computational Biology and Bioinformatics, IEEE/ACM Transactions on]]></pubtitle>
##     <punumber><![CDATA[8857]]></punumber>
##     <pubtype><![CDATA[Journals & Magazines]]></pubtype>
##     <publisher><![CDATA[IEEE]]></publisher>
##     <volume><![CDATA[11]]></volume>
##     <issue><![CDATA[4]]></issue>
##     <py><![CDATA[2014]]></py>
##     <spage><![CDATA[753]]></spage>
##     <epage><![CDATA[765]]></epage>
##     <abstract><![CDATA[Biological pathways are important elements of systems biology and in the past decade, an increasing number of pathway databases have been set up to document the growing understanding of complex cellular processes. Although more genome-sequence data are becoming available, a large fraction of it remains functionally uncharacterized. Thus, it is important to be able to predict the mapping of poorly annotated proteins to original pathway models. Results: We have developed a Relational Learning-based Extension (RLE) system to investigate pathway membership through a function prediction approach that mainly relies on combinations of simple properties attributed to each protein. RLE searches for proteins with molecular similarities to specific pathway components. Using RLE, we associated 383 uncharacterized proteins to 28 pre-defined human Reactome pathways, demonstrating relative confidence after proper evaluation. Indeed, in specific cases manual inspection of the database annotations and the related literature supported the proposed classifications. Examples of possible additional components of the Electron transport system, Telomere maintenance and Integrin cell surface interactions pathways are discussed in detail. Availability: All the human predicted proteins in the 2009 and 2012 releases 30 and 40 of Reactome are available at http://rle.bioinfo.cnio.es.]]></abstract>
##     <issn><![CDATA[1545-5963]]></issn>
##     <htmlFlag><![CDATA[1]]></htmlFlag>
##     <arnumber><![CDATA[6802366]]></arnumber>
##     <doi><![CDATA[10.1109/TCBB.2014.2318730]]></doi>
##     <publicationId><![CDATA[6802366]]></publicationId>
##     <mdurl><![CDATA[http://ieeexplore.ieee.org/xpl/articleDetails.jsp?tp=&arnumber=6802366&contentType=Journals+%26+Magazines]]></mdurl>
##     <pdf><![CDATA[http://ieeexplore.ieee.org/stamp/stamp.jsp?arnumber=6802366]]></pdf>
##   </document>
##   <document>
##     <rank>979</rank>
##     <title><![CDATA[Novel UWB and Spread Spectrum System Using Time Compression and Overlap-Add Techniques]]></title>
##     <authors><![CDATA[Harrison, S.;  Driessen, P.F.]]></authors>
##     <affiliations><![CDATA[Department of Electrical and Computer Engineering, University of Victoria, Victoria, BC, Canada]]></affiliations>
##     <thesaurusterms>
##       <term><![CDATA[Communication channels]]></term>
##       <term><![CDATA[Physical layer]]></term>
##       <term><![CDATA[Software radio]]></term>
##       <term><![CDATA[Spread spectrum communication]]></term>
##       <term><![CDATA[Ultra wideband communication]]></term>
##       <term><![CDATA[Wireless communication]]></term>
##     </thesaurusterms>
##     <pubtitle><![CDATA[Access, IEEE]]></pubtitle>
##     <punumber><![CDATA[6287639]]></punumber>
##     <pubtype><![CDATA[Journals & Magazines]]></pubtype>
##     <publisher><![CDATA[IEEE]]></publisher>
##     <volume><![CDATA[2]]></volume>
##     <py><![CDATA[2014]]></py>
##     <spage><![CDATA[1092]]></spage>
##     <epage><![CDATA[1105]]></epage>
##     <abstract><![CDATA[We present a UWB and spread spectrum communications method based on the idea of time compression where a sampled message signal is transmitted at a higher sampling rate. Robustness is achieved by dividing the signal into overlapping segments, transmitting each segment fast enough so that the segments no longer overlap, receiving these segments and reconstructing the message by overlap-adding the segments. A key feature of this scheme is that an exact sample rate match is not required to recover the signal. This method is implemented in a custom wideband software defined radio, with good results in the presence of interference and multipath. This method, referred to as time compression overlap-add (TC-OLA), represents a new concept and design approach and an advance in fundamental technology of the air interface physical layer that may be relevant to 5G wireless technologies.]]></abstract>
##     <issn><![CDATA[2169-3536]]></issn>
##     <htmlFlag><![CDATA[1]]></htmlFlag>
##     <arnumber><![CDATA[6897912]]></arnumber>
##     <doi><![CDATA[10.1109/ACCESS.2014.2357422]]></doi>
##     <publicationId><![CDATA[6897912]]></publicationId>
##     <mdurl><![CDATA[http://ieeexplore.ieee.org/xpl/articleDetails.jsp?tp=&arnumber=6897912&contentType=Journals+%26+Magazines]]></mdurl>
##     <pdf><![CDATA[http://ieeexplore.ieee.org/stamp/stamp.jsp?arnumber=6897912]]></pdf>
##   </document>
##   <document>
##     <rank>980</rank>
##     <title><![CDATA[Development of a Technique for Characterizing Bias Temperature Instability-Induced Device-to-Device Variation at SRAM-Relevant Conditions]]></title>
##     <authors><![CDATA[Meng Duan;  Jian Fu Zhang;  Zhigang Ji;  Wei Dong Zhang;  Kaczer, B.;  Schram, T.;  Ritzenthaler, R.;  Groeseneken, G.;  Asenov, A.]]></authors>
##     <affiliations><![CDATA[Sch. of Eng., John Moores Univ., Liverpool, UK]]></affiliations>
##     <controlledterms>
##       <term><![CDATA[SRAM chips]]></term>
##       <term><![CDATA[ageing]]></term>
##       <term><![CDATA[integrated circuit noise]]></term>
##       <term><![CDATA[negative bias temperature instability]]></term>
##       <term><![CDATA[random noise]]></term>
##     </controlledterms>
##     <thesaurusterms>
##       <term><![CDATA[Degradation]]></term>
##       <term><![CDATA[MOSFET]]></term>
##       <term><![CDATA[Random access memory]]></term>
##       <term><![CDATA[Stress]]></term>
##       <term><![CDATA[Stress measurement]]></term>
##       <term><![CDATA[Time measurement]]></term>
##       <term><![CDATA[Video recording]]></term>
##     </thesaurusterms>
##     <pubtitle><![CDATA[Electron Devices, IEEE Transactions on]]></pubtitle>
##     <punumber><![CDATA[16]]></punumber>
##     <pubtype><![CDATA[Journals & Magazines]]></pubtype>
##     <publisher><![CDATA[IEEE]]></publisher>
##     <volume><![CDATA[61]]></volume>
##     <issue><![CDATA[9]]></issue>
##     <py><![CDATA[2014]]></py>
##     <spage><![CDATA[3081]]></spage>
##     <epage><![CDATA[3089]]></epage>
##     <abstract><![CDATA[SRAM is vulnerable to device-to-device variation (DDV), since it uses minimum-sized devices and requires device matching. In addition to the as-fabricated DDV at time-zero, aging induces a time-dependent DDV (TDDV). Bias temperature instability (BTI) is a dominant aging process. A number of techniques have been developed to characterize the BTI, including the conventional pulse-(I) -(V) , random telegraph noises, time-dependent defect spectroscopy, and TDDV accounting for the within-device fluctuation. These techniques, however, cannot be directly applied to SRAM, because their test conditions do not comply with typical SRAM operation. The central objective of this paper is to develop a technique suitable for characterizing both the negative BTI (NBTI) and positive BTI (PBTI) in SRAM. The key issues addressed include the SRAM relevant sensing Vg, measurement delay, capturing the upper envelope of degradation, sampling rate, and measurement time window. The differences between NBTI and PBTI are highlighted. The impact of NBTI and PBTI on the cell-level performance is assessed by simulation, based on experimental results obtained from individual devices. The simulation results show that, for a given static noise margin, test conditions have a significant effect on the minimum operation bias.]]></abstract>
##     <issn><![CDATA[0018-9383]]></issn>
##     <htmlFlag><![CDATA[1]]></htmlFlag>
##     <arnumber><![CDATA[6866905]]></arnumber>
##     <doi><![CDATA[10.1109/TED.2014.2335053]]></doi>
##     <publicationId><![CDATA[6866905]]></publicationId>
##     <mdurl><![CDATA[http://ieeexplore.ieee.org/xpl/articleDetails.jsp?tp=&arnumber=6866905&contentType=Journals+%26+Magazines]]></mdurl>
##     <pdf><![CDATA[http://ieeexplore.ieee.org/stamp/stamp.jsp?arnumber=6866905]]></pdf>
##   </document>
##   <document>
##     <rank>981</rank>
##     <title><![CDATA[Multisource X-Ray and CT: Lessons Learned and Future Outlook]]></title>
##     <authors><![CDATA[Neculaes, V.B.;  Edic, P.M.;  Frontera, M.;  Caiafa, A.;  Ge Wang;  De Man, B.]]></authors>
##     <affiliations><![CDATA[GE Global Res., Niskayuna, NY, USA]]></affiliations>
##     <controlledterms>
##       <term><![CDATA[X-ray microscopy]]></term>
##       <term><![CDATA[computerised tomography]]></term>
##     </controlledterms>
##     <thesaurusterms>
##       <term><![CDATA[Computed tomography]]></term>
##       <term><![CDATA[Detectors]]></term>
##       <term><![CDATA[Distributed processing]]></term>
##       <term><![CDATA[Electron tubes]]></term>
##       <term><![CDATA[Image processing]]></term>
##       <term><![CDATA[X-ray imaging]]></term>
##     </thesaurusterms>
##     <pubtitle><![CDATA[Access, IEEE]]></pubtitle>
##     <punumber><![CDATA[6287639]]></punumber>
##     <pubtype><![CDATA[Journals & Magazines]]></pubtype>
##     <publisher><![CDATA[IEEE]]></publisher>
##     <volume><![CDATA[2]]></volume>
##     <py><![CDATA[2014]]></py>
##     <spage><![CDATA[1568]]></spage>
##     <epage><![CDATA[1585]]></epage>
##     <abstract><![CDATA[Distributed X-ray sources open the way to innovative system concepts in X-ray and computed tomography. They offer promising opportunities in terms of system performance, but they pose unique challenges in terms of source and system technologies. Several academic and industrial teams have proposed a variety of concepts and developed some unique prototypes. We present a broad review of multisource systems. We also discuss X-ray source components and challenges. We close with our perspective on the future prospects of multisource imaging.]]></abstract>
##     <issn><![CDATA[2169-3536]]></issn>
##     <htmlFlag><![CDATA[1]]></htmlFlag>
##     <arnumber><![CDATA[6928407]]></arnumber>
##     <doi><![CDATA[10.1109/ACCESS.2014.2363949]]></doi>
##     <publicationId><![CDATA[6928407]]></publicationId>
##     <mdurl><![CDATA[http://ieeexplore.ieee.org/xpl/articleDetails.jsp?tp=&arnumber=6928407&contentType=Journals+%26+Magazines]]></mdurl>
##     <pdf><![CDATA[http://ieeexplore.ieee.org/stamp/stamp.jsp?arnumber=6928407]]></pdf>
##   </document>
##   <document>
##     <rank>982</rank>
##     <title><![CDATA[X- and C-Band SAR Surface Displacement for the 2013 <italic>Lunigiana</italic> Earthquake (Northern Italy): A Breached Relay Ramp?]]></title>
##     <authors><![CDATA[Stramondo, S.;  Vannoli, P.;  Cannelli, V.;  Polcari, M.;  Melini, D.;  Samsonov, S.;  Moro, M.;  Bignami, C.;  Saroli, M.]]></authors>
##     <affiliations><![CDATA[Ist. Naz. di Geofisica e Vulcanologia, Rome, Italy]]></affiliations>
##     <controlledterms>
##       <term><![CDATA[earthquakes]]></term>
##       <term><![CDATA[faulting]]></term>
##       <term><![CDATA[geomorphology]]></term>
##       <term><![CDATA[geophysical techniques]]></term>
##       <term><![CDATA[radar interferometry]]></term>
##       <term><![CDATA[synthetic aperture radar]]></term>
##     </controlledterms>
##     <thesaurusterms>
##       <term><![CDATA[Deformable models]]></term>
##       <term><![CDATA[Earthquakes]]></term>
##       <term><![CDATA[Interferometry]]></term>
##       <term><![CDATA[Satellites]]></term>
##       <term><![CDATA[Surface topography]]></term>
##       <term><![CDATA[Synthetic aperture radar]]></term>
##     </thesaurusterms>
##     <pubtitle><![CDATA[Selected Topics in Applied Earth Observations and Remote Sensing, IEEE Journal of]]></pubtitle>
##     <punumber><![CDATA[4609443]]></punumber>
##     <pubtype><![CDATA[Journals & Magazines]]></pubtype>
##     <publisher><![CDATA[IEEE]]></publisher>
##     <volume><![CDATA[7]]></volume>
##     <issue><![CDATA[7]]></issue>
##     <py><![CDATA[2014]]></py>
##     <spage><![CDATA[2746]]></spage>
##     <epage><![CDATA[2753]]></epage>
##     <abstract><![CDATA[In this paper, we applied the differential interferometric synthetic aperture radar (DInSAR) technique to investigate and measure surface displacements due to the M<sub>w</sub> 5.3 (M<sub>1</sub> 5.2), June 21, 2013 earthquake, occurred north of the Apuan Alps (NW Italy), in the discontinuity zone between the Lunigiana and Garfagnana area. Two differential interferograms showing the coseismic displacement have been generated using X-band and C-band data, taken from COSMO-SkyMed and RADARSAT-2 satellites, respectively. Both interferograms highlighted a clear pattern of subsidence of few cm located between the Lunigiana and Garfagnana basins. We then modeled the observed SAR deformation fields using the Okada analytical formulation and found them to be consistent with an extensional fault plane dipping toward NW at about 50 . The integrated analysis of DInSAR, geological data, modeling, and historical seismicity suggest that the fault responsible for the June 2013 earthquake corresponds to a breached relay ramp connecting the Lunigiana and Garfagnana seismogenic sources.]]></abstract>
##     <issn><![CDATA[1939-1404]]></issn>
##     <htmlFlag><![CDATA[1]]></htmlFlag>
##     <arnumber><![CDATA[6819006]]></arnumber>
##     <doi><![CDATA[10.1109/JSTARS.2014.2313640]]></doi>
##     <publicationId><![CDATA[6819006]]></publicationId>
##     <mdurl><![CDATA[http://ieeexplore.ieee.org/xpl/articleDetails.jsp?tp=&arnumber=6819006&contentType=Journals+%26+Magazines]]></mdurl>
##     <pdf><![CDATA[http://ieeexplore.ieee.org/stamp/stamp.jsp?arnumber=6819006]]></pdf>
##   </document>
##   <document>
##     <rank>983</rank>
##     <title><![CDATA[Vector Brillouin Optical Time-Domain Analysis With Heterodyne Detection and IQ Demodulation Algorithm]]></title>
##     <authors><![CDATA[Xiaobo Tu;  Qiao Sun;  Wei Chen;  Mo Chen;  Zhou Meng]]></authors>
##     <affiliations><![CDATA[Coll. of Optoelectron. Sci. & Eng., Nat. Univ. of Defense Technol., Changsha, China]]></affiliations>
##     <controlledterms>
##       <term><![CDATA[acousto-optical devices]]></term>
##       <term><![CDATA[demodulation]]></term>
##       <term><![CDATA[fibre optic sensors]]></term>
##       <term><![CDATA[heterodyne detection]]></term>
##       <term><![CDATA[optical phase shifters]]></term>
##       <term><![CDATA[signal processing]]></term>
##       <term><![CDATA[time-domain analysis]]></term>
##     </controlledterms>
##     <thesaurusterms>
##       <term><![CDATA[Demodulation]]></term>
##       <term><![CDATA[Optical mixing]]></term>
##       <term><![CDATA[Optical sensors]]></term>
##       <term><![CDATA[Optical signal processing]]></term>
##       <term><![CDATA[Probes]]></term>
##       <term><![CDATA[Scattering]]></term>
##       <term><![CDATA[Temperature measurement]]></term>
##     </thesaurusterms>
##     <pubtitle><![CDATA[Photonics Journal, IEEE]]></pubtitle>
##     <punumber><![CDATA[4563994]]></punumber>
##     <pubtype><![CDATA[Journals & Magazines]]></pubtype>
##     <publisher><![CDATA[IEEE]]></publisher>
##     <volume><![CDATA[6]]></volume>
##     <issue><![CDATA[2]]></issue>
##     <py><![CDATA[2014]]></py>
##     <spage><![CDATA[1]]></spage>
##     <epage><![CDATA[8]]></epage>
##     <abstract><![CDATA[This paper reports a vector Brillouin optical time-domain analyzer based on heterodyne detection and IQ demodulation algorithm. This approach employs an acousto-optic frequency shifter to generate a single-frequency reference, which simplifies the probe spectrum and directly features the Brillouin gain and phase shift compared with the phase modulation. In addition, the proposed IQ demodulation algorithm makes signal processing easy to implement and allows fast and real-time demodulation. A proof-of-concept experiment is carried out in an ~1.6-km standard single-mode fiber consisting of two different sections of fibers. Both the Brillouin gain and phase-shift spectrograms are successfully measured with an ~2-m spatial resolution. The temperature dependence values of the Brillouin gain and phase-shift spectra are also measured, and the temperature dependence values of the Brillouin frequency shift show excellent linearity larger than 0.999 and almost identical slopes of 1.166 and 1.159 MHz/&#x00B0;C, respectively.]]></abstract>
##     <issn><![CDATA[1943-0655]]></issn>
##     <htmlFlag><![CDATA[1]]></htmlFlag>
##     <arnumber><![CDATA[6750016]]></arnumber>
##     <doi><![CDATA[10.1109/JPHOT.2014.2306835]]></doi>
##     <publicationId><![CDATA[6750016]]></publicationId>
##     <mdurl><![CDATA[http://ieeexplore.ieee.org/xpl/articleDetails.jsp?tp=&arnumber=6750016&contentType=Journals+%26+Magazines]]></mdurl>
##     <pdf><![CDATA[http://ieeexplore.ieee.org/stamp/stamp.jsp?arnumber=6750016]]></pdf>
##   </document>
##   <document>
##     <rank>984</rank>
##     <title><![CDATA[Two-Photon-Excited Emission in Polymer Optical Fibers Doped With a Conjugated Polymer]]></title>
##     <authors><![CDATA[Ayesta, I.;  Illarramendi, M.A.;  Arrue, J.;  Jimenez, F.;  Zubia, J.]]></authors>
##     <affiliations><![CDATA[Dept. of Commun. Eng., Univ. of the Basque Country UPV-EHU, Bilbao, Spain]]></affiliations>
##     <controlledterms>
##       <term><![CDATA[fluorescence]]></term>
##       <term><![CDATA[optical fibres]]></term>
##       <term><![CDATA[optical polymers]]></term>
##       <term><![CDATA[optical pumping]]></term>
##       <term><![CDATA[two-photon spectra]]></term>
##     </controlledterms>
##     <thesaurusterms>
##       <term><![CDATA[Absorption]]></term>
##       <term><![CDATA[Fiber lasers]]></term>
##       <term><![CDATA[Fluorescence]]></term>
##       <term><![CDATA[Measurement by laser beam]]></term>
##       <term><![CDATA[Optical fiber couplers]]></term>
##       <term><![CDATA[Polymers]]></term>
##     </thesaurusterms>
##     <pubtitle><![CDATA[Photonics Journal, IEEE]]></pubtitle>
##     <punumber><![CDATA[4563994]]></punumber>
##     <pubtype><![CDATA[Journals & Magazines]]></pubtype>
##     <publisher><![CDATA[IEEE]]></publisher>
##     <volume><![CDATA[6]]></volume>
##     <issue><![CDATA[4]]></issue>
##     <py><![CDATA[2014]]></py>
##     <spage><![CDATA[1]]></spage>
##     <epage><![CDATA[9]]></epage>
##     <abstract><![CDATA[In this paper, the two-photon-excited emission spectra of polymer optical fibers doped with the conjugated polymer Poly(9,9-dioctylfluorene-alt-benzothiadiazole) (F8BT) have been measured pumping the fibers transversely to their symmetry axis. Measurements include evolutions of the emission spectra with excitation wavelength and with propagation distance, together with an analysis of emission photostability. Comparisons with results for one-photon-excited emission are also presented and discussed.]]></abstract>
##     <issn><![CDATA[1943-0655]]></issn>
##     <htmlFlag><![CDATA[1]]></htmlFlag>
##     <arnumber><![CDATA[6844840]]></arnumber>
##     <doi><![CDATA[10.1109/JPHOT.2014.2331238]]></doi>
##     <publicationId><![CDATA[6844840]]></publicationId>
##     <mdurl><![CDATA[http://ieeexplore.ieee.org/xpl/articleDetails.jsp?tp=&arnumber=6844840&contentType=Journals+%26+Magazines]]></mdurl>
##     <pdf><![CDATA[http://ieeexplore.ieee.org/stamp/stamp.jsp?arnumber=6844840]]></pdf>
##   </document>
##   <document>
##     <rank>985</rank>
##     <title><![CDATA[Correction of Atmospheric Phase Screen in Time Series InSAR Using WRF Model for Monitoring Volcanic Activities]]></title>
##     <authors><![CDATA[Jungkyo Jung;  Duk-jin Kim;  Sang-Eun Park]]></authors>
##     <affiliations><![CDATA[Sch. of Earth & Environ. Sci., Seoul Nat. Univ., Seoul, South Korea]]></affiliations>
##     <controlledterms>
##       <term><![CDATA[Global Positioning System]]></term>
##       <term><![CDATA[atmospheric optics]]></term>
##       <term><![CDATA[geophysical signal processing]]></term>
##       <term><![CDATA[radar interferometry]]></term>
##       <term><![CDATA[remote sensing by radar]]></term>
##       <term><![CDATA[synthetic aperture radar]]></term>
##       <term><![CDATA[time series]]></term>
##       <term><![CDATA[topography (Earth)]]></term>
##       <term><![CDATA[volcanology]]></term>
##     </controlledterms>
##     <pubtitle><![CDATA[Geoscience and Remote Sensing, IEEE Transactions on]]></pubtitle>
##     <punumber><![CDATA[36]]></punumber>
##     <pubtype><![CDATA[Journals & Magazines]]></pubtype>
##     <publisher><![CDATA[IEEE]]></publisher>
##     <volume><![CDATA[52]]></volume>
##     <issue><![CDATA[5]]></issue>
##     <py><![CDATA[2014]]></py>
##     <spage><![CDATA[2678]]></spage>
##     <epage><![CDATA[2689]]></epage>
##     <abstract><![CDATA[One of the main limitations in measuring ground deformation using synthetic aperture radar interferometry (InSAR) is atmospheric phase delay effects. In volcanic regions, the atmospheric phase delay effects can cause serious problems in detecting volcanic unrest because atmospheric thickness is inversely related with the elevation of a volcanic mountain. It is commonly known that the atmospheric phase screen (APS) can be decomposed spatially into stratified and turbulent components. In this paper, the stratified and turbulent atmospheric conditions of a volcanic area were simulated using weather research and forecasting (WRF) model, and the simulated atmospheric conditions were compared with in situ radiosonde data. The comparison results proved that the stratified APS from the WRF model could reflect the reasonable patterns of seasonal changes and vertical profiles with dependable quality. We also found that the stratified APS was significantly correlated with time and sometimes severely contaminated the quality of volcanic deformation estimation. These results indicate that the temporal high-pass (HP) filtering, which has been usually applied in time series InSAR analysis for extracting and removing APS, cannot work properly in volcanic area. Thus, we propose a new method that employs the stratified APS obtained from the WRF model by correlating with topography and the (residual) turbulent APS that can be effectively eliminated by temporal HP filtering in persistent scatterer interferometry (PSI). We applied the proposed method (atmosphere-corrected PSI) to Advanced Land Observing Satellite Phased Array type L-band SAR data that cover Shinmoedake volcano, Japan, and found that the estimated surface deformation and APS agreed well with those measured from GPS and Moderate-Resolution Imaging Spectroradiometer data, respectively.]]></abstract>
##     <issn><![CDATA[0196-2892]]></issn>
##     <htmlFlag><![CDATA[1]]></htmlFlag>
##     <arnumber><![CDATA[6547675]]></arnumber>
##     <doi><![CDATA[10.1109/TGRS.2013.2264532]]></doi>
##     <publicationId><![CDATA[6547675]]></publicationId>
##     <mdurl><![CDATA[http://ieeexplore.ieee.org/xpl/articleDetails.jsp?tp=&arnumber=6547675&contentType=Journals+%26+Magazines]]></mdurl>
##     <pdf><![CDATA[http://ieeexplore.ieee.org/stamp/stamp.jsp?arnumber=6547675]]></pdf>
##   </document>
##   <document>
##     <rank>986</rank>
##     <title><![CDATA[Information theory, kelly betting, risk, reward, commission, and omission: An example problem in breast cancer]]></title>
##     <authors><![CDATA[Dalton, L.W.]]></authors>
##     <affiliations><![CDATA[Private Practice, Austin, TX, USA]]></affiliations>
##     <controlledterms>
##       <term><![CDATA[cancer]]></term>
##       <term><![CDATA[information theory]]></term>
##       <term><![CDATA[medical diagnostic computing]]></term>
##       <term><![CDATA[molecular biophysics]]></term>
##       <term><![CDATA[patient diagnosis]]></term>
##     </controlledterms>
##     <thesaurusterms>
##       <term><![CDATA[Breast cancer]]></term>
##       <term><![CDATA[Cancer]]></term>
##       <term><![CDATA[Classification]]></term>
##       <term><![CDATA[Data visualization]]></term>
##       <term><![CDATA[Information theory]]></term>
##       <term><![CDATA[Safety]]></term>
##       <term><![CDATA[Sensitivity and specificity]]></term>
##       <term><![CDATA[Shannon theory]]></term>
##     </thesaurusterms>
##     <pubtitle><![CDATA[Access, IEEE]]></pubtitle>
##     <punumber><![CDATA[6287639]]></punumber>
##     <pubtype><![CDATA[Journals & Magazines]]></pubtype>
##     <publisher><![CDATA[IEEE]]></publisher>
##     <volume><![CDATA[2]]></volume>
##     <py><![CDATA[2014]]></py>
##     <spage><![CDATA[1272]]></spage>
##     <epage><![CDATA[1280]]></epage>
##     <abstract><![CDATA[In binary classification, two-way confusion matrices, with corresponding measures, such as sensitivity and specificity, have become so ubiquitous that those who review results may not realize there are other and more realistic ways to visualize data. This is, particularly, true when risk and reward considerations are important. The approach suggested here proposes that classification need not offer a conclusion on every instance within a data set. If an algorithm finds instances (e.g., patient cases in a medical data set) in which attributes pertaining to a patient's disease offer zero to nil information, there should be no classification offered. From the physician's perspective, disclosure of nil information should be welcome because it might prevent potentially harmful treatment. It follows from this that the developer of a classifier can provide summary results amendable for helping the consumer decide whether or not it is prudent to pass or act (commission versus omission). It is not always about balancing sensitivity and specificity in all cases, but optimizing action on some cases. The explanation is centered on John Kelly's link of gambling with Shannon information theory. In addition, Graham's margin of safety, Bernoulli's utiles, and Hippocratic Oath are important. An example problem is provided using a Netherlands Cancer Institute breast cancer data set. Recurrence score, a popular molecular-based assay for breast cancer prognosis, was found to have an uninformative zone. The uninformative subset had been grouped with positive results to garner higher sensitivity. Yet, because of a positive result, patients might be advised to undergo potentially harmful treatment in the absence of useful information.]]></abstract>
##     <issn><![CDATA[2169-3536]]></issn>
##     <htmlFlag><![CDATA[1]]></htmlFlag>
##     <arnumber><![CDATA[6936324]]></arnumber>
##     <doi><![CDATA[10.1109/ACCESS.2014.2363134]]></doi>
##     <publicationId><![CDATA[6936324]]></publicationId>
##     <mdurl><![CDATA[http://ieeexplore.ieee.org/xpl/articleDetails.jsp?tp=&arnumber=6936324&contentType=Journals+%26+Magazines]]></mdurl>
##     <pdf><![CDATA[http://ieeexplore.ieee.org/stamp/stamp.jsp?arnumber=6936324]]></pdf>
##   </document>
##   <document>
##     <rank>987</rank>
##     <title><![CDATA[Wind Turbine Power Curve Modeling Using Advanced Parametric and Nonparametric Methods]]></title>
##     <authors><![CDATA[Shokrzadeh, S.;  Jozani, M.J.;  Bibeau, E.]]></authors>
##     <affiliations><![CDATA[Dept. of Mech. Eng., Univ. of Manitoba, Winnipeg, MB, Canada]]></affiliations>
##     <controlledterms>
##       <term><![CDATA[curve fitting]]></term>
##       <term><![CDATA[load forecasting]]></term>
##       <term><![CDATA[regression analysis]]></term>
##       <term><![CDATA[splines (mathematics)]]></term>
##       <term><![CDATA[wind turbines]]></term>
##     </controlledterms>
##     <thesaurusterms>
##       <term><![CDATA[Data models]]></term>
##       <term><![CDATA[Polynomials]]></term>
##       <term><![CDATA[Regression analysis]]></term>
##       <term><![CDATA[Splines (mathematics)]]></term>
##       <term><![CDATA[Wind energy]]></term>
##       <term><![CDATA[Wind power generation]]></term>
##       <term><![CDATA[Wind turbines]]></term>
##     </thesaurusterms>
##     <pubtitle><![CDATA[Sustainable Energy, IEEE Transactions on]]></pubtitle>
##     <punumber><![CDATA[5165391]]></punumber>
##     <pubtype><![CDATA[Journals & Magazines]]></pubtype>
##     <publisher><![CDATA[IEEE]]></publisher>
##     <volume><![CDATA[5]]></volume>
##     <issue><![CDATA[4]]></issue>
##     <py><![CDATA[2014]]></py>
##     <spage><![CDATA[1262]]></spage>
##     <epage><![CDATA[1269]]></epage>
##     <abstract><![CDATA[Wind turbine power curve modeling is an important tool in turbine performance monitoring and power forecasting. There are several statistical techniques to fit the empirical power curve of a wind turbine, which can be classified into parametric and nonparametric methods. In this paper, we study four of these methods to estimate the wind turbine power curve. Polynomial regression is studied as the benchmark parametric model, and issues associated with this technique are discussed. We then introduce the locally weighted polynomial regression method, and show its advantages over the polynomial regression. Also, the spline regression method is examined to achieve more flexibility for fitting the power curve. Finally, we develop a penalized spline regression model to address the issues of choosing the number and location of knots in the spline regression. The performance of the presented methods is evaluated using two simulated data sets as well as an actual operational power data of a wind farm in North America.]]></abstract>
##     <issn><![CDATA[1949-3029]]></issn>
##     <htmlFlag><![CDATA[1]]></htmlFlag>
##     <arnumber><![CDATA[6894235]]></arnumber>
##     <doi><![CDATA[10.1109/TSTE.2014.2345059]]></doi>
##     <publicationId><![CDATA[6894235]]></publicationId>
##     <mdurl><![CDATA[http://ieeexplore.ieee.org/xpl/articleDetails.jsp?tp=&arnumber=6894235&contentType=Journals+%26+Magazines]]></mdurl>
##     <pdf><![CDATA[http://ieeexplore.ieee.org/stamp/stamp.jsp?arnumber=6894235]]></pdf>
##   </document>
##   <document>
##     <rank>988</rank>
##     <title><![CDATA[Efficiency and Scalability of Dielectric Resonator Antennas at Optical Frequencies]]></title>
##     <authors><![CDATA[Longfang Zou;  Withayachumnankul, W.;  Shah, C.M.;  Mitchell, A.;  Klemm, M.;  Bhaskaran, M.;  Sriram, S.;  Fumeaux, C.]]></authors>
##     <affiliations><![CDATA[Sch. of Electr. & Electron. Eng., Univ. of Adelaide, Adelaide, SA, Australia]]></affiliations>
##     <controlledterms>
##       <term><![CDATA[dielectric resonator antennas]]></term>
##       <term><![CDATA[nanostructured materials]]></term>
##       <term><![CDATA[reflectivity]]></term>
##     </controlledterms>
##     <thesaurusterms>
##       <term><![CDATA[Dielectric resonator antennas]]></term>
##       <term><![CDATA[Dielectrics]]></term>
##       <term><![CDATA[Numerical models]]></term>
##       <term><![CDATA[Optical resonators]]></term>
##       <term><![CDATA[Resonant frequency]]></term>
##       <term><![CDATA[Silver]]></term>
##     </thesaurusterms>
##     <pubtitle><![CDATA[Photonics Journal, IEEE]]></pubtitle>
##     <punumber><![CDATA[4563994]]></punumber>
##     <pubtype><![CDATA[Journals & Magazines]]></pubtype>
##     <publisher><![CDATA[IEEE]]></publisher>
##     <volume><![CDATA[6]]></volume>
##     <issue><![CDATA[4]]></issue>
##     <py><![CDATA[2014]]></py>
##     <spage><![CDATA[1]]></spage>
##     <epage><![CDATA[10]]></epage>
##     <abstract><![CDATA[Dielectric resonators have been foreseen as a pathway for the realization of highly efficient nanoantennas and metamaterials at optical frequencies. In this paper, we study the resonant behavior of dielectric nanocylinders located on a metal plane, which in combination create dielectric resonator antennas operating in reflection mode. By implementing appropriate resonator models, the field distributions, the scaling behavior, and the efficiency of dielectric resonator antennas are studied across the spectrum from the microwave toward visible frequency bands. Numerical results confirm that a radiation efficiency above 80% can be retained up to the near-infrared with metal-backed dielectric resonators. This paper establishes fundamental knowledge toward development of high efficiency dielectric resonator antennas and reflection metasurfaces at optical frequencies. These dielectric resonators can be incorporated as basic elements in emerging applications, e.g., flat optical components, quantum dot emitters, and subwavelength sensors.]]></abstract>
##     <issn><![CDATA[1943-0655]]></issn>
##     <htmlFlag><![CDATA[1]]></htmlFlag>
##     <arnumber><![CDATA[6851849]]></arnumber>
##     <doi><![CDATA[10.1109/JPHOT.2014.2337891]]></doi>
##     <publicationId><![CDATA[6851849]]></publicationId>
##     <mdurl><![CDATA[http://ieeexplore.ieee.org/xpl/articleDetails.jsp?tp=&arnumber=6851849&contentType=Journals+%26+Magazines]]></mdurl>
##     <pdf><![CDATA[http://ieeexplore.ieee.org/stamp/stamp.jsp?arnumber=6851849]]></pdf>
##   </document>
##   <document>
##     <rank>989</rank>
##     <title><![CDATA[Long Period Fiber Gratings Inscribed With an Improved Two-Dimensional Scanning Technique]]></title>
##     <authors><![CDATA[Xiaoyong Zhong;  Yiping Wang;  Changrui Liao;  Guolu Yin;  Jiangtao Zhou;  Guanjun Wang;  Bing Sun;  Jian Tang]]></authors>
##     <affiliations><![CDATA[Key Lab. of Optoelectron. Devices & Syst. of the Minist. of Educ. & Guangdong Province, Shenzhen Univ., Shenzhen, China]]></affiliations>
##     <controlledterms>
##       <term><![CDATA[carbon compounds]]></term>
##       <term><![CDATA[diffraction gratings]]></term>
##       <term><![CDATA[gas lasers]]></term>
##       <term><![CDATA[laser materials processing]]></term>
##       <term><![CDATA[laser stability]]></term>
##       <term><![CDATA[optical fibre cladding]]></term>
##       <term><![CDATA[optical scanners]]></term>
##     </controlledterms>
##     <thesaurusterms>
##       <term><![CDATA[Fiber gratings]]></term>
##       <term><![CDATA[Fiber lasers]]></term>
##       <term><![CDATA[Gratings]]></term>
##       <term><![CDATA[Laser beams]]></term>
##       <term><![CDATA[Laser modes]]></term>
##       <term><![CDATA[Laser stability]]></term>
##       <term><![CDATA[Optical fibers]]></term>
##     </thesaurusterms>
##     <pubtitle><![CDATA[Photonics Journal, IEEE]]></pubtitle>
##     <punumber><![CDATA[4563994]]></punumber>
##     <pubtype><![CDATA[Journals & Magazines]]></pubtype>
##     <publisher><![CDATA[IEEE]]></publisher>
##     <volume><![CDATA[6]]></volume>
##     <issue><![CDATA[4]]></issue>
##     <py><![CDATA[2014]]></py>
##     <spage><![CDATA[1]]></spage>
##     <epage><![CDATA[8]]></epage>
##     <abstract><![CDATA[We demonstrated a promising CO<sub>2</sub> laser irradiation system based on an improved 2-D scanning technique. Such a system could be used to inscribe high-quality long period fiber gratings (LPFGs) with good reproducibility of grating inscription, which attributes to the fact that our system includes a CO<sub>2</sub> laser with an excellent power stability of less than &#x00B1;2% and a 3-D ultraprecision motorized translation stages with an excellent bidirectional repeatability value of 80 nm. Moreover, a control program with an easy-to-use operation interface was developed in our system so that a high-quality LPFG could be achieved as soon as grating parameters, such as grating pitch and number of grating periods, are entered, which has a widespread commercial value and prospects for development. Additionally, near mode fields of the CO<sub>2</sub>-laser-induced LPFG were observed and simulated to investigate mode coupling in the gratings.]]></abstract>
##     <issn><![CDATA[1943-0655]]></issn>
##     <htmlFlag><![CDATA[1]]></htmlFlag>
##     <arnumber><![CDATA[6851869]]></arnumber>
##     <doi><![CDATA[10.1109/JPHOT.2014.2337875]]></doi>
##     <publicationId><![CDATA[6851869]]></publicationId>
##     <mdurl><![CDATA[http://ieeexplore.ieee.org/xpl/articleDetails.jsp?tp=&arnumber=6851869&contentType=Journals+%26+Magazines]]></mdurl>
##     <pdf><![CDATA[http://ieeexplore.ieee.org/stamp/stamp.jsp?arnumber=6851869]]></pdf>
##   </document>
##   <document>
##     <rank>990</rank>
##     <title><![CDATA[Nonreciprocal Channels of Light Through the Coupling of Two Nonsymmetric Tamm Magnetoplasmon Polaritons]]></title>
##     <authors><![CDATA[Yun-tuan Fang;  Lin-kun Chen;  Jing Zheng;  Lian-ying Zhou;  Jun Zhou]]></authors>
##     <affiliations><![CDATA[Sch. of Comput. Sci. & Telecommun. Eng., Jiangsu Univ., Zhenjiang, China]]></affiliations>
##     <controlledterms>
##       <term><![CDATA[magneto-optical effects]]></term>
##       <term><![CDATA[optical couplers]]></term>
##       <term><![CDATA[photonic crystals]]></term>
##       <term><![CDATA[plasmons]]></term>
##       <term><![CDATA[polaritons]]></term>
##     </controlledterms>
##     <thesaurusterms>
##       <term><![CDATA[Couplings]]></term>
##       <term><![CDATA[Magnetic confinement]]></term>
##       <term><![CDATA[Magnetic fields]]></term>
##       <term><![CDATA[Magnetic materials]]></term>
##       <term><![CDATA[Magnetic resonance]]></term>
##       <term><![CDATA[Metals]]></term>
##       <term><![CDATA[Photonic crystals]]></term>
##     </thesaurusterms>
##     <pubtitle><![CDATA[Photonics Journal, IEEE]]></pubtitle>
##     <punumber><![CDATA[4563994]]></punumber>
##     <pubtype><![CDATA[Journals & Magazines]]></pubtype>
##     <publisher><![CDATA[IEEE]]></publisher>
##     <volume><![CDATA[6]]></volume>
##     <issue><![CDATA[4]]></issue>
##     <py><![CDATA[2014]]></py>
##     <spage><![CDATA[1]]></spage>
##     <epage><![CDATA[11]]></epage>
##     <abstract><![CDATA[We propose a coupled system that can produce nonreciprocal channels of light. The system consists of a metal layer sandwiched between two magnetophotonic crystals (MPCs). A resonance mode called Tamm magnetoplasmon polariton (TMP) around the interfaces between the metal and the MPCs can be achieved in this structure. It is found that nonreciprocal light-tunneling channels can be obtained through the coupling of two nonsymmetric TMPs. The nonreciprocal wavelength channels can be adjusted, depending on the incidence direction of the light, whereas the nonreciprocal direction channels can be adjusted, depending on the wavelength of the incident light. Results are demonstrated through electromagnetic-field simulations based on a finite-element solver.]]></abstract>
##     <issn><![CDATA[1943-0655]]></issn>
##     <htmlFlag><![CDATA[1]]></htmlFlag>
##     <arnumber><![CDATA[6872518]]></arnumber>
##     <doi><![CDATA[10.1109/JPHOT.2014.2345872]]></doi>
##     <publicationId><![CDATA[6872518]]></publicationId>
##     <mdurl><![CDATA[http://ieeexplore.ieee.org/xpl/articleDetails.jsp?tp=&arnumber=6872518&contentType=Journals+%26+Magazines]]></mdurl>
##     <pdf><![CDATA[http://ieeexplore.ieee.org/stamp/stamp.jsp?arnumber=6872518]]></pdf>
##   </document>
##   <document>
##     <rank>991</rank>
##     <title><![CDATA[Ultra-Sensitive Strain and Temperature Sensing Based on Modal Interference in Perfluorinated Polymer Optical Fibers]]></title>
##     <authors><![CDATA[Numata, G.;  Hayashi, N.;  Tabaru, M.;  Mizuno, Y.;  Nakamura, K.]]></authors>
##     <affiliations><![CDATA[Precision & Intell. Lab., Tokyo Inst. of Technol., Yokohama, Japan]]></affiliations>
##     <controlledterms>
##       <term><![CDATA[fibre optic sensors]]></term>
##       <term><![CDATA[gradient index optics]]></term>
##       <term><![CDATA[light interference]]></term>
##       <term><![CDATA[optical polymers]]></term>
##       <term><![CDATA[strain sensors]]></term>
##       <term><![CDATA[temperature sensors]]></term>
##     </controlledterms>
##     <thesaurusterms>
##       <term><![CDATA[Optical fibers]]></term>
##       <term><![CDATA[Sensitivity]]></term>
##       <term><![CDATA[Silicon compounds]]></term>
##       <term><![CDATA[Strain]]></term>
##       <term><![CDATA[Temperature dependence]]></term>
##       <term><![CDATA[Temperature measurement]]></term>
##       <term><![CDATA[Temperature sensors]]></term>
##     </thesaurusterms>
##     <pubtitle><![CDATA[Photonics Journal, IEEE]]></pubtitle>
##     <punumber><![CDATA[4563994]]></punumber>
##     <pubtype><![CDATA[Journals & Magazines]]></pubtype>
##     <publisher><![CDATA[IEEE]]></publisher>
##     <volume><![CDATA[6]]></volume>
##     <issue><![CDATA[5]]></issue>
##     <py><![CDATA[2014]]></py>
##     <spage><![CDATA[1]]></spage>
##     <epage><![CDATA[7]]></epage>
##     <abstract><![CDATA[We implement the strain and temperature sensors based on multimode interference in perfluorinated (PF) graded-index (GI) plastic optical fibers, and investigate their sensing performance at 1300 nm. We obtain strain and temperature sensitivities of -112 pm/&#x03BC;&#x03B5;/m and +49.8 nm/&#x00B0;C/m, the absolute value of which are 12.9 and over 1800 times as large as those in silica GI multimode fibers, respectively. These ultra-high strain and temperature sensitivities probably originate from the unique core material, i.e., PF polymer.]]></abstract>
##     <issn><![CDATA[1943-0655]]></issn>
##     <htmlFlag><![CDATA[1]]></htmlFlag>
##     <arnumber><![CDATA[6887320]]></arnumber>
##     <doi><![CDATA[10.1109/JPHOT.2014.2352637]]></doi>
##     <publicationId><![CDATA[6887320]]></publicationId>
##     <mdurl><![CDATA[http://ieeexplore.ieee.org/xpl/articleDetails.jsp?tp=&arnumber=6887320&contentType=Journals+%26+Magazines]]></mdurl>
##     <pdf><![CDATA[http://ieeexplore.ieee.org/stamp/stamp.jsp?arnumber=6887320]]></pdf>
##   </document>
##   <document>
##     <rank>992</rank>
##     <title><![CDATA[Electrooptical Characterization of New Classes of Silicon Carbide UV Photodetectors]]></title>
##     <authors><![CDATA[Adamo, G.;  Tomasino, A.;  Parisi, A.;  Agro, D.;  Stivala, S.;  Curcio, L.;  Ando, A.;  Pernice, R.;  Giaconia, C.;  Busacca, A.C.;  Mazzillo, M.C.;  Sanfilippo, D.;  Fallica, G.]]></authors>
##     <affiliations><![CDATA[Dipt. di Energia, Ing. Dell'Inf. e Modelli Matematici, Univ. of Palermo, Palermo, Italy]]></affiliations>
##     <controlledterms>
##       <term><![CDATA[Schottky diodes]]></term>
##       <term><![CDATA[electrical contacts]]></term>
##       <term><![CDATA[electro-optical effects]]></term>
##       <term><![CDATA[nickel compounds]]></term>
##       <term><![CDATA[photodetectors]]></term>
##       <term><![CDATA[silicon compounds]]></term>
##       <term><![CDATA[ultraviolet detectors]]></term>
##       <term><![CDATA[wide band gap semiconductors]]></term>
##     </controlledterms>
##     <thesaurusterms>
##       <term><![CDATA[Optical device fabrication]]></term>
##       <term><![CDATA[Photodetectors]]></term>
##       <term><![CDATA[Schottky diodes]]></term>
##       <term><![CDATA[Silicon carbide]]></term>
##       <term><![CDATA[Silicon compounds]]></term>
##       <term><![CDATA[Ultraviolet sources]]></term>
##     </thesaurusterms>
##     <pubtitle><![CDATA[Photonics Journal, IEEE]]></pubtitle>
##     <punumber><![CDATA[4563994]]></punumber>
##     <pubtype><![CDATA[Journals & Magazines]]></pubtype>
##     <publisher><![CDATA[IEEE]]></publisher>
##     <volume><![CDATA[6]]></volume>
##     <issue><![CDATA[6]]></issue>
##     <py><![CDATA[2014]]></py>
##     <spage><![CDATA[1]]></spage>
##     <epage><![CDATA[7]]></epage>
##     <abstract><![CDATA[In this paper, we present the fabrication process steps and the characterization of 4H-SiC vertical Schottky UV detectors, where interdigitated strips, acting as top metal contacts, have been realized in Ni<sub>2</sub>Si. These devices exploit the pinch-off surface effect. I-V and C-V characteristics, as functions of temperature, were measured in dark conditions. In addition, we have carried out responsivity measurements, for wavelengths ranging from 200 to 400 nm, at varying package temperature and applied reverse bias. A comparison among devices having different strip pitch sizes has been performed, thus finding out that the 10-&#x03BC;m pitch class demonstrates the top performances as regards the tradeoff between exposed surface area and complete merge of adjacent depleted regions under top contacts.]]></abstract>
##     <issn><![CDATA[1943-0655]]></issn>
##     <htmlFlag><![CDATA[1]]></htmlFlag>
##     <arnumber><![CDATA[6887321]]></arnumber>
##     <doi><![CDATA[10.1109/JPHOT.2014.2352611]]></doi>
##     <publicationId><![CDATA[6887321]]></publicationId>
##     <mdurl><![CDATA[http://ieeexplore.ieee.org/xpl/articleDetails.jsp?tp=&arnumber=6887321&contentType=Journals+%26+Magazines]]></mdurl>
##     <pdf><![CDATA[http://ieeexplore.ieee.org/stamp/stamp.jsp?arnumber=6887321]]></pdf>
##   </document>
##   <document>
##     <rank>993</rank>
##     <title><![CDATA[High-Transmission-Efficiency 120 <inline-formula> <img src="/images/tex/540.gif" alt="^{\circ}"> </inline-formula> Photonic Crystal Waveguide Bend by Using Flexible Structural Defects]]></title>
##     <authors><![CDATA[Tee, D.C.;  Tamchek, N.;  Abu Bakar, M.H.;  Mahamd Adikan, F.R.]]></authors>
##     <affiliations><![CDATA[Dept. of Electr. Eng., Univ. of Malaya, Kuala Lumpur, Malaysia]]></affiliations>
##     <controlledterms>
##       <term><![CDATA[finite difference time-domain analysis]]></term>
##       <term><![CDATA[optical design techniques]]></term>
##       <term><![CDATA[optical fabrication]]></term>
##       <term><![CDATA[optical losses]]></term>
##       <term><![CDATA[optical waveguides]]></term>
##       <term><![CDATA[photonic crystals]]></term>
##     </controlledterms>
##     <thesaurusterms>
##       <term><![CDATA[DH-HEMTs]]></term>
##       <term><![CDATA[Optical waveguides]]></term>
##       <term><![CDATA[Photonic crystals]]></term>
##       <term><![CDATA[Reflection]]></term>
##       <term><![CDATA[Slabs]]></term>
##       <term><![CDATA[Vectors]]></term>
##     </thesaurusterms>
##     <pubtitle><![CDATA[Photonics Journal, IEEE]]></pubtitle>
##     <punumber><![CDATA[4563994]]></punumber>
##     <pubtype><![CDATA[Journals & Magazines]]></pubtype>
##     <publisher><![CDATA[IEEE]]></publisher>
##     <volume><![CDATA[6]]></volume>
##     <issue><![CDATA[4]]></issue>
##     <py><![CDATA[2014]]></py>
##     <spage><![CDATA[1]]></spage>
##     <epage><![CDATA[10]]></epage>
##     <abstract><![CDATA[We numerically studied a high-output-transmission-efficiency low-reflection-loss 120&#x00B0; photonic crystal (PhC) waveguide bend based on a PhC slab with triangular-lattice air holes. The desired high output transmission efficiency was achieved by introducing flexible structural defects into the bend region of the waveguide. Simulation results obtained using a 3-D finite-difference time-domain method indicated that normalized output transmission as high as 94.3% and negligible normalized reflection loss of 0.1% were obtained at the 1550-nm optical wavelength. Furthermore, the normalized output transmission was more than 90% within the entire optical C-band. In addition, sensitivity of the design parameters of the structural defect was studied to understand the tolerance in the fabrication error, while maintaining high output transmission efficiency.]]></abstract>
##     <issn><![CDATA[1943-0655]]></issn>
##     <htmlFlag><![CDATA[1]]></htmlFlag>
##     <arnumber><![CDATA[6872784]]></arnumber>
##     <doi><![CDATA[10.1109/JPHOT.2014.2344000]]></doi>
##     <publicationId><![CDATA[6872784]]></publicationId>
##     <mdurl><![CDATA[http://ieeexplore.ieee.org/xpl/articleDetails.jsp?tp=&arnumber=6872784&contentType=Journals+%26+Magazines]]></mdurl>
##     <pdf><![CDATA[http://ieeexplore.ieee.org/stamp/stamp.jsp?arnumber=6872784]]></pdf>
##   </document>
##   <document>
##     <rank>994</rank>
##     <title><![CDATA[Potentials and Challenges of C-RAN Supporting Multi-RATs Toward 5G Mobile Networks]]></title>
##     <authors><![CDATA[Rui Wang;  Honglin Hu;  Xiumei Yang]]></authors>
##     <affiliations><![CDATA[Shanghai Res. Center for Wireless Commun., Shanghai Inst. of Microsyst. & Inf. Technol., Shanghai, China]]></affiliations>
##     <controlledterms>
##       <term><![CDATA[mobility management (mobile radio)]]></term>
##       <term><![CDATA[telecommunication traffic]]></term>
##     </controlledterms>
##     <thesaurusterms>
##       <term><![CDATA[Cellular networks]]></term>
##       <term><![CDATA[Cloud computing]]></term>
##       <term><![CDATA[Mobile communication]]></term>
##       <term><![CDATA[Mobile radio mobility management]]></term>
##       <term><![CDATA[Next generation networking]]></term>
##       <term><![CDATA[Radio access networks]]></term>
##       <term><![CDATA[Radio communication]]></term>
##       <term><![CDATA[Resource management]]></term>
##     </thesaurusterms>
##     <pubtitle><![CDATA[Access, IEEE]]></pubtitle>
##     <punumber><![CDATA[6287639]]></punumber>
##     <pubtype><![CDATA[Journals & Magazines]]></pubtype>
##     <publisher><![CDATA[IEEE]]></publisher>
##     <volume><![CDATA[2]]></volume>
##     <py><![CDATA[2014]]></py>
##     <spage><![CDATA[1187]]></spage>
##     <epage><![CDATA[1195]]></epage>
##     <abstract><![CDATA[This paper presents an overview of the cloud radio access network (C-RAN), which is a key enabler for future mobile networks in order to meet the explosive capacity demand of mobile traffic, and reduce the capital and operating expenditure burden faced by operators. We start by reviewing the requirements of future mobile networks, called 5G, followed by a discussion on emerging network concepts for 5G network architecture. Then, an overview of C-RAN and related works are presented. As a significant scenario of a 5G system, the ultra dense network deployment based on C-RAN is discussed with focuses on flexible backhauling, automated network organization, and advanced mobility management. Another import feature of a 5G system is the long-term coexistence of multiple radio access technologies (multi-RATs). Therefore, we present some directions and preliminary thoughts for future C-RAN-supporting Multi-RATs, including joint resource allocation, mobility management, as well as traffic steering and service mapping.]]></abstract>
##     <issn><![CDATA[2169-3536]]></issn>
##     <htmlFlag><![CDATA[1]]></htmlFlag>
##     <arnumber><![CDATA[6914530]]></arnumber>
##     <doi><![CDATA[10.1109/ACCESS.2014.2360555]]></doi>
##     <publicationId><![CDATA[6914530]]></publicationId>
##     <mdurl><![CDATA[http://ieeexplore.ieee.org/xpl/articleDetails.jsp?tp=&arnumber=6914530&contentType=Journals+%26+Magazines]]></mdurl>
##     <pdf><![CDATA[http://ieeexplore.ieee.org/stamp/stamp.jsp?arnumber=6914530]]></pdf>
##   </document>
##   <document>
##     <rank>995</rank>
##     <title><![CDATA[16-Wavelength DFB Laser Array With High Channel-Spacing Uniformity Based on Equivalent Phase-Shift Technique]]></title>
##     <authors><![CDATA[Yuechun Shi;  Lianyan Li;  Jilin Zheng;  Yunshan Zhang;  Bocang Qiu;  Xiangfei Chen]]></authors>
##     <affiliations><![CDATA[Microwave Photonics Technol. Lab., Nanjing Univ., Nanjing, China]]></affiliations>
##     <controlledterms>
##       <term><![CDATA[channel spacing]]></term>
##       <term><![CDATA[chirp modulation]]></term>
##       <term><![CDATA[compensation]]></term>
##       <term><![CDATA[diffraction gratings]]></term>
##       <term><![CDATA[distributed feedback lasers]]></term>
##       <term><![CDATA[optical design techniques]]></term>
##       <term><![CDATA[optical fibre dispersion]]></term>
##       <term><![CDATA[optical transmitters]]></term>
##       <term><![CDATA[semiconductor laser arrays]]></term>
##     </controlledterms>
##     <thesaurusterms>
##       <term><![CDATA[Arrayed waveguide gratings]]></term>
##       <term><![CDATA[Channel spacing]]></term>
##       <term><![CDATA[Dispersion]]></term>
##       <term><![CDATA[Distributed feedback devices]]></term>
##       <term><![CDATA[Photonic integrated circuits]]></term>
##       <term><![CDATA[Semiconductor laser arrays]]></term>
##     </thesaurusterms>
##     <pubtitle><![CDATA[Photonics Journal, IEEE]]></pubtitle>
##     <punumber><![CDATA[4563994]]></punumber>
##     <pubtype><![CDATA[Journals & Magazines]]></pubtype>
##     <publisher><![CDATA[IEEE]]></publisher>
##     <volume><![CDATA[6]]></volume>
##     <issue><![CDATA[6]]></issue>
##     <py><![CDATA[2014]]></py>
##     <spage><![CDATA[1]]></spage>
##     <epage><![CDATA[9]]></epage>
##     <abstract><![CDATA[The high accuracy of lasing wavelength spacing is one of the key requirements of a distributed feedback (DFB) semiconductor laser array. However, the nonuniformity of the wavelength spacing is increasingly deteriorating with the increase in the channel number in the laser array. In this paper, theoretical study was made to investigate the effects of sampling pattern deviation and seed grating, as well as waveguide dispersion on the wavelength-spacing uniformity for multiwavelength DFB semiconductor laser arrays (MLAs) fabricated using the reconstruction equivalent chirp (REC) technique. A simple measurement method of dispersion for DFB semiconductor lasers based on the REC technique is also proposed. With the dispersion compensation being included in the sampling period design and small deviation in the seed grating period being guaranteed, a high-channel-count (16-channel) DFB laser array with precise channel spacing of 0.7944 nm/channel (design value of 0.80 nm/channel) was achieved in our experiment. It shows excellent channel-spacing uniformity, and most wavelength residuals are within &#x00B1;0.10 nm.]]></abstract>
##     <issn><![CDATA[1943-0655]]></issn>
##     <htmlFlag><![CDATA[1]]></htmlFlag>
##     <arnumber><![CDATA[6979198]]></arnumber>
##     <doi><![CDATA[10.1109/JPHOT.2014.2374610]]></doi>
##     <publicationId><![CDATA[6979198]]></publicationId>
##     <mdurl><![CDATA[http://ieeexplore.ieee.org/xpl/articleDetails.jsp?tp=&arnumber=6979198&contentType=Journals+%26+Magazines]]></mdurl>
##     <pdf><![CDATA[http://ieeexplore.ieee.org/stamp/stamp.jsp?arnumber=6979198]]></pdf>
##   </document>
##   <document>
##     <rank>996</rank>
##     <title><![CDATA[Decentralized Throughput Maximization in Cognitive Radio Wireless Mesh Networks]]></title>
##     <authors><![CDATA[El-Sherif, A.A.;  Mohamed, A.]]></authors>
##     <affiliations><![CDATA[Dept. of Electr. Eng., Alexandria Univ., Alexandria, Egypt]]></affiliations>
##     <controlledterms>
##       <term><![CDATA[access protocols]]></term>
##       <term><![CDATA[cognitive radio]]></term>
##       <term><![CDATA[optimisation]]></term>
##       <term><![CDATA[radio spectrum management]]></term>
##       <term><![CDATA[resource allocation]]></term>
##       <term><![CDATA[scheduling]]></term>
##       <term><![CDATA[signal detection]]></term>
##       <term><![CDATA[wireless mesh networks]]></term>
##     </controlledterms>
##     <thesaurusterms>
##       <term><![CDATA[Availability]]></term>
##       <term><![CDATA[Cognitive radio]]></term>
##       <term><![CDATA[Interference]]></term>
##       <term><![CDATA[Mesh networks]]></term>
##       <term><![CDATA[Resource management]]></term>
##       <term><![CDATA[Sensors]]></term>
##       <term><![CDATA[Throughput]]></term>
##     </thesaurusterms>
##     <pubtitle><![CDATA[Mobile Computing, IEEE Transactions on]]></pubtitle>
##     <punumber><![CDATA[7755]]></punumber>
##     <pubtype><![CDATA[Journals & Magazines]]></pubtype>
##     <publisher><![CDATA[IEEE]]></publisher>
##     <volume><![CDATA[13]]></volume>
##     <issue><![CDATA[9]]></issue>
##     <py><![CDATA[2014]]></py>
##     <spage><![CDATA[1967]]></spage>
##     <epage><![CDATA[1980]]></epage>
##     <abstract><![CDATA[Scheduling and spectrum allocation are tasks affecting the performance of cognitive radio wireless networks, where heterogeneity in channel availability limits the performance and poses a great challenge on protocol design. In this paper, we present a distributed algorithm for scheduling and spectrum allocation with the objective of maximizing the network's throughout subject to a delay constraint. During each time slot, the scheduling and spectrum allocation problems involve selecting a subset of links to be activated, and based on spectrum sensing outcomes, allocate the available resources to these links. This problem is addressed as an aggregate utility maximization problem. Since the throughput of any data flow is limited by the throughput of the weakest link along its end-to-end path, the utility of each flow is chosen as a function of this weakest link's throughput. The throughput and delay performance of the network are characterized using a queueing theoretic analysis, and throughput is maximized via the application of Lagrangian duality theory. The dual decomposition framework decouples the problem into a set of subproblems that can be solved locally, hence, it allows us to develop a scalable distributed algorithm. Numerical results demonstrate the fast convergence rates of the proposed algorithm, as well as significant performance gains compared to conventional design methods.]]></abstract>
##     <issn><![CDATA[1536-1233]]></issn>
##     <htmlFlag><![CDATA[1]]></htmlFlag>
##     <arnumber><![CDATA[6746172]]></arnumber>
##     <doi><![CDATA[10.1109/TMC.2013.82]]></doi>
##     <publicationId><![CDATA[6746172]]></publicationId>
##     <mdurl><![CDATA[http://ieeexplore.ieee.org/xpl/articleDetails.jsp?tp=&arnumber=6746172&contentType=Journals+%26+Magazines]]></mdurl>
##     <pdf><![CDATA[http://ieeexplore.ieee.org/stamp/stamp.jsp?arnumber=6746172]]></pdf>
##   </document>
##   <document>
##     <rank>997</rank>
##     <title><![CDATA[User Grouping for Massive MIMO in FDD Systems: New Design Methods and Analysis]]></title>
##     <authors><![CDATA[Yi Xu;  Guosen Yue;  Shiwen Mao]]></authors>
##     <affiliations><![CDATA[Dept. of Electr. & Comput. Eng., Auburn Univ., Auburn, AL, USA]]></affiliations>
##     <controlledterms>
##       <term><![CDATA[MIMO communication]]></term>
##       <term><![CDATA[channel estimation]]></term>
##       <term><![CDATA[dynamic scheduling]]></term>
##       <term><![CDATA[frequency division multiplexing]]></term>
##       <term><![CDATA[group theory]]></term>
##       <term><![CDATA[pattern clustering]]></term>
##       <term><![CDATA[precoding]]></term>
##       <term><![CDATA[resource allocation]]></term>
##     </controlledterms>
##     <thesaurusterms>
##       <term><![CDATA[Channel estimation]]></term>
##       <term><![CDATA[Clustering methods]]></term>
##       <term><![CDATA[Costs]]></term>
##       <term><![CDATA[Design methodology]]></term>
##       <term><![CDATA[Dynamic scheduling]]></term>
##       <term><![CDATA[Finite difference methods]]></term>
##       <term><![CDATA[Frequency conversion]]></term>
##       <term><![CDATA[MIMO]]></term>
##       <term><![CDATA[Throughtput]]></term>
##       <term><![CDATA[Time division multiplexing]]></term>
##       <term><![CDATA[Weight measurement]]></term>
##     </thesaurusterms>
##     <pubtitle><![CDATA[Access, IEEE]]></pubtitle>
##     <punumber><![CDATA[6287639]]></punumber>
##     <pubtype><![CDATA[Journals & Magazines]]></pubtype>
##     <publisher><![CDATA[IEEE]]></publisher>
##     <volume><![CDATA[2]]></volume>
##     <py><![CDATA[2014]]></py>
##     <spage><![CDATA[947]]></spage>
##     <epage><![CDATA[959]]></epage>
##     <abstract><![CDATA[The massive multiple-input multiple-output (MIMO) system has drawn increasing attention recently as it is expected to boost the system throughput and result in lower costs. Previous studies mainly focus on time division duplexing (TDD) systems, which are more amenable to practical implementations due to channel reciprocity. However, there are many frequency division duplexing (FDD) systems deployed worldwide. Consequently, it is of great importance to investigate the design and performance of FDD massive MIMO systems. To reduce the overhead of channel estimation in FDD systems, a two-stage precoding scheme was recently proposed to decompose the precoding procedure into intergroup precoding and intragroup precoding. The problem of user grouping and scheduling thus arises. In this paper, we first propose three novel similarity measures for user grouping based on weighted likelihood, subspace projection, and Fubini-Study, respectively, as well as two novel clustering methods, including hierarchical and K-medoids clustering. We then propose a dynamic user scheduling scheme to further enhance the system throughput once the user groups are formed. The load balancing problem is considered when few users are active and solved with an effective algorithm. The efficacy of the proposed schemes are validated with theoretical analysis and simulations.]]></abstract>
##     <issn><![CDATA[2169-3536]]></issn>
##     <htmlFlag><![CDATA[1]]></htmlFlag>
##     <arnumber><![CDATA[6888467]]></arnumber>
##     <doi><![CDATA[10.1109/ACCESS.2014.2353297]]></doi>
##     <publicationId><![CDATA[6888467]]></publicationId>
##     <mdurl><![CDATA[http://ieeexplore.ieee.org/xpl/articleDetails.jsp?tp=&arnumber=6888467&contentType=Journals+%26+Magazines]]></mdurl>
##     <pdf><![CDATA[http://ieeexplore.ieee.org/stamp/stamp.jsp?arnumber=6888467]]></pdf>
##   </document>
##   <document>
##     <rank>998</rank>
##     <title><![CDATA[<inline-formula> <img src="/images/tex/21820.gif" alt="L_{n}"> </inline-formula> Slot Photonic Crystal Microcavity for Refractive Index Gas Sensing]]></title>
##     <authors><![CDATA[Li, K.;  Li, J.;  Song, Y.;  Fang, G.;  Li, C.;  Feng, Z.;  Su, R.;  Zeng, B.;  Wang, X.;  Jin, C.]]></authors>
##     <thesaurusterms>
##       <term><![CDATA[Gas detectors]]></term>
##       <term><![CDATA[Microcavities]]></term>
##       <term><![CDATA[Photonic crystals]]></term>
##       <term><![CDATA[Refractive index]]></term>
##       <term><![CDATA[Sensitivity]]></term>
##       <term><![CDATA[Sensors]]></term>
##     </thesaurusterms>
##     <pubtitle><![CDATA[Photonics Journal, IEEE]]></pubtitle>
##     <punumber><![CDATA[4563994]]></punumber>
##     <pubtype><![CDATA[Journals & Magazines]]></pubtype>
##     <publisher><![CDATA[IEEE]]></publisher>
##     <volume><![CDATA[6]]></volume>
##     <issue><![CDATA[5]]></issue>
##     <py><![CDATA[2014]]></py>
##     <spage><![CDATA[1]]></spage>
##     <epage><![CDATA[9]]></epage>
##     <abstract><![CDATA[We propose and experimentally demonstrate a series of <inline-formula> <img src="/images/tex/21820.gif" alt="L_{n}"> </inline-formula> slot photonic crystal (PhC) microcavities, which operate as refractive index (RI) gas sensors. The cavities are simply composed of a silicon slab triangular photonic crystal with <inline-formula> <img src="/images/tex/388.gif" alt="n"> </inline-formula> holes replaced by a slot, which do not require sophisticated design or high fabrication resolution. With the increase in <inline-formula> <img src="/images/tex/388.gif" alt="n"> </inline-formula>, the quality factor of the cavity exponentially increases, which is explained by the envelope of electric field approaching a Gaussian profile. An <inline-formula> <img src="/images/tex/24066.gif" alt="L_{9}"> </inline-formula> slot PhC microcavity with a quality factor exceeding 30&#x00A0;000, sensitivity of 421 nm per RI unit (RIU), and detection limit down to <inline-formula> <img src="/images/tex/24067.gif" alt="1\times 10^{-5}"> </inline-formula> RIU was experimentally demonstrated. The performance of the device is comparable with other fine-tuned PhC microcavity structures. Due to its simple structure and high fabrication tolerance, it could have wide applications in optical sensors.]]></abstract>
##     <issn><![CDATA[1943-0655]]></issn>
##     <htmlFlag><![CDATA[1]]></htmlFlag>
##     <arnumber><![CDATA[6910213]]></arnumber>
##     <doi><![CDATA[10.1109/JPHOT.2014.2360286]]></doi>
##     <publicationId><![CDATA[6910213]]></publicationId>
##     <mdurl><![CDATA[http://ieeexplore.ieee.org/xpl/articleDetails.jsp?tp=&arnumber=6910213&contentType=Journals+%26+Magazines]]></mdurl>
##     <pdf><![CDATA[http://ieeexplore.ieee.org/stamp/stamp.jsp?arnumber=6910213]]></pdf>
##   </document>
##   <document>
##     <rank>999</rank>
##     <title><![CDATA[CMOS Flat-Panel X-ray Detector With Dual-Gain Active Pixel Sensors and Column-Parallel Readout Circuits]]></title>
##     <authors><![CDATA[Yun-Rae Jo;  Seong-Kwan Hong;  Oh-Kyong Kwon]]></authors>
##     <affiliations><![CDATA[Dept. of Electron. Eng., Hanyang Univ., Seoul, South Korea]]></affiliations>
##     <controlledterms>
##       <term><![CDATA[CMOS integrated circuits]]></term>
##       <term><![CDATA[X-ray apparatus]]></term>
##       <term><![CDATA[X-ray detection]]></term>
##       <term><![CDATA[analogue-digital conversion]]></term>
##       <term><![CDATA[readout electronics]]></term>
##       <term><![CDATA[semiconductor counters]]></term>
##     </controlledterms>
##     <thesaurusterms>
##       <term><![CDATA[CMOS integrated circuits]]></term>
##       <term><![CDATA[Capacitance]]></term>
##       <term><![CDATA[Generators]]></term>
##       <term><![CDATA[Noise]]></term>
##       <term><![CDATA[Radiation detectors]]></term>
##       <term><![CDATA[Sensitivity]]></term>
##       <term><![CDATA[Timing]]></term>
##     </thesaurusterms>
##     <pubtitle><![CDATA[Nuclear Science, IEEE Transactions on]]></pubtitle>
##     <punumber><![CDATA[23]]></punumber>
##     <pubtype><![CDATA[Journals & Magazines]]></pubtype>
##     <publisher><![CDATA[IEEE]]></publisher>
##     <volume><![CDATA[61]]></volume>
##     <issue><![CDATA[5]]></issue>
##     <part><![CDATA[1]]></part>
##     <py><![CDATA[2014]]></py>
##     <spage><![CDATA[2472]]></spage>
##     <epage><![CDATA[2479]]></epage>
##     <abstract><![CDATA[This paper proposes a CMOS flat-panel X-ray detector (FPXD) with dual-gain active pixel sensors (APSs) and column-parallel readout circuits to reduce the random noise. The proposed dual-gain APS employs the conversion gain control in a pixel sensor array and supports both high and low sensitivity modes for FPXD. The in-pixel conversion gain control suppresses the amplification of the pixel noise, so it improves signal-to-noise characteristics. The column-parallel readout circuits include single-slope analog-to-digital converters (SS-ADCs) and charge-summing circuits for pixel binning and analog double delta sampling (DDS). SS-ADCs support 12-bit resolution and use the driving method of gray-code counters with different initial values to reduce the peak current and the power fluctuation. They also employ a high resolution continuous-type ramp generator to reduce the area. The proposed CMOS FPXD with a pixel size of 100 &#x03BC;m &#x00D7; 100 &#x03BC;m was fabricated using a 0.18-&#x03BC;m CMOS process. The conversion gains in high and low sensitivity modes are designed with 0.43 &#x03BC;V/e<sup>-</sup> and 3.00 &#x03BC;V/e<sup>-</sup>, respectively. The measured random noises in high and low sensitivity modes are 366 &#x03BC;V and 400 &#x03BC;V, respectively, at the resolution of 12 bits and the frame rate of 30 fps. The area of ramp generator and the peak current of the gray-code counter are reduced by 92% and 43%, respectively, compared with the conventional structures.]]></abstract>
##     <issn><![CDATA[0018-9499]]></issn>
##     <htmlFlag><![CDATA[1]]></htmlFlag>
##     <arnumber><![CDATA[6880835]]></arnumber>
##     <doi><![CDATA[10.1109/TNS.2014.2343459]]></doi>
##     <publicationId><![CDATA[6880835]]></publicationId>
##     <mdurl><![CDATA[http://ieeexplore.ieee.org/xpl/articleDetails.jsp?tp=&arnumber=6880835&contentType=Journals+%26+Magazines]]></mdurl>
##     <pdf><![CDATA[http://ieeexplore.ieee.org/stamp/stamp.jsp?arnumber=6880835]]></pdf>
##   </document>
##   <document>
##     <rank>1000</rank>
##     <title><![CDATA[In Vivo Brain Magnetic Resonance Spectroscopy: A Measurement of Biomarker Sensitivity to Post-Processing Algorithms]]></title>
##     <authors><![CDATA[Cocuzzo, D.;  Lin, A.;  Stanwell, P.;  Mountford, C.;  Keshava, N.]]></authors>
##     <affiliations><![CDATA[Dept. of Comput. Sci., Stanford Univ., Palo Alto, CA, USA]]></affiliations>
##     <controlledterms>
##       <term><![CDATA[biomedical MRI]]></term>
##       <term><![CDATA[brain]]></term>
##       <term><![CDATA[diseases]]></term>
##       <term><![CDATA[image classification]]></term>
##       <term><![CDATA[magnetic resonance spectroscopy]]></term>
##       <term><![CDATA[medical image processing]]></term>
##       <term><![CDATA[statistical analysis]]></term>
##     </controlledterms>
##     <thesaurusterms>
##       <term><![CDATA[Biomarkers]]></term>
##       <term><![CDATA[Classification algorithms]]></term>
##       <term><![CDATA[Clinical diagnosis]]></term>
##       <term><![CDATA[In vivo]]></term>
##       <term><![CDATA[Magnetic resonance]]></term>
##       <term><![CDATA[Pain management]]></term>
##       <term><![CDATA[Spectroscopy]]></term>
##     </thesaurusterms>
##     <pubtitle><![CDATA[Translational Engineering in Health and Medicine, IEEE Journal of]]></pubtitle>
##     <punumber><![CDATA[6221039]]></punumber>
##     <pubtype><![CDATA[Journals & Magazines]]></pubtype>
##     <publisher><![CDATA[IEEE]]></publisher>
##     <volume><![CDATA[2]]></volume>
##     <py><![CDATA[2014]]></py>
##     <spage><![CDATA[1]]></spage>
##     <epage><![CDATA[17]]></epage>
##     <abstract><![CDATA[Clinical translation of reported biomarkers requires reliable and consistent algorithms to derive biomarkers. However, the literature reports statistically significant differences between 1-D MRS measurements from control groups and subjects with disease states but frequently provides little information on the algorithms and parameters used to process the data. The sensitivity of in vivo brain magnetic resonance spectroscopy biomarkers is investigated with respect to parameter values for two key stages of post-acquisitional processing. Our effort is specifically motivated by the lack of consensus on approaches and parameter values for the two critical operations, water resonance removal, and baseline correction. The different stages of data processing also introduce varying levels of uncertainty and arbitrary selection of parameter values can significantly underutilize the intrinsic differences between two classes of signals. The sensitivity of biomarkers points to the need for a better understanding of how all stages of post-acquisitional processing affect biomarker discovery and ultimately, clinical translation. Our results also highlight the possibility of optimizing biomarker discovery by the careful selection of parameters that best reveal class differences. Using previously reported data and biomarkers, our results demonstrate that small changes in parameter values affect the statistical significance and corresponding effect size of biomarkers. Consequently, it is possible to increase the strength of biomarkers by selecting optimal parameter values in different spectral intervals. Our analyses with a previously reported data set demonstrate an increase in effect sizes for wavelet-based biomarkers of up to 36%, with increases in classification performance of up to 12%.]]></abstract>
##     <issn><![CDATA[2168-2372]]></issn>
##     <htmlFlag><![CDATA[1]]></htmlFlag>
##     <arnumber><![CDATA[6754141]]></arnumber>
##     <doi><![CDATA[10.1109/JTEHM.2014.2309333]]></doi>
##     <publicationId><![CDATA[6754141]]></publicationId>
##     <mdurl><![CDATA[http://ieeexplore.ieee.org/xpl/articleDetails.jsp?tp=&arnumber=6754141&contentType=Journals+%26+Magazines]]></mdurl>
##     <pdf><![CDATA[http://ieeexplore.ieee.org/stamp/stamp.jsp?arnumber=6754141]]></pdf>
##   </document>
##   <document>
##     <rank>1001</rank>
##     <title><![CDATA[Study on Bearing Impedance Properties at Several Hundred Kilohertz for Different Electric Machine Operating Parameters]]></title>
##     <authors><![CDATA[Niskanen, V.;  Muetze, A.;  Ahola, J.]]></authors>
##     <affiliations><![CDATA[Lappeenranta Univ., Lappeenranta, Finland]]></affiliations>
##     <controlledterms>
##       <term><![CDATA[electric impedance]]></term>
##       <term><![CDATA[invertors]]></term>
##       <term><![CDATA[motor drives]]></term>
##       <term><![CDATA[rolling bearings]]></term>
##       <term><![CDATA[variable speed drives]]></term>
##     </controlledterms>
##     <thesaurusterms>
##       <term><![CDATA[Current measurement]]></term>
##       <term><![CDATA[Discharges (electric)]]></term>
##       <term><![CDATA[Hafnium]]></term>
##       <term><![CDATA[Impedance]]></term>
##       <term><![CDATA[Inverters]]></term>
##       <term><![CDATA[Temperature measurement]]></term>
##       <term><![CDATA[Voltage measurement]]></term>
##     </thesaurusterms>
##     <pubtitle><![CDATA[Industry Applications, IEEE Transactions on]]></pubtitle>
##     <punumber><![CDATA[28]]></punumber>
##     <pubtype><![CDATA[Journals & Magazines]]></pubtype>
##     <publisher><![CDATA[IEEE]]></publisher>
##     <volume><![CDATA[50]]></volume>
##     <issue><![CDATA[5]]></issue>
##     <py><![CDATA[2014]]></py>
##     <spage><![CDATA[3438]]></spage>
##     <epage><![CDATA[3447]]></epage>
##     <abstract><![CDATA[The possibility of bearing damage caused by inverter-induced bearing currents in modern variable-speed drive systems has been well recognized. However, the breakdown and current conduction mechanisms have still not been well understood today. We present results on the HF impedance properties of rolling element bearings at 300-kHz and 1.5-MHz signals-frequencies taken from the typical spectra of HF circulating bearing currents. Different motor speeds, bearing temperatures, voltages across the bearing, and two different motor sizes are investigated. Notably, we analyze the frequency of transitions between ohmic and capacitive behavior (or vice versa)-the &#x201C;mode transitions&#x201D;-as a function of these operating parameters: For some operating conditions, the bearing impedance alternates between capacitive and ohmic behavior without any modification of the macroscopic conditions it is subjected to.]]></abstract>
##     <issn><![CDATA[0093-9994]]></issn>
##     <htmlFlag><![CDATA[1]]></htmlFlag>
##     <arnumber><![CDATA[6748940]]></arnumber>
##     <doi><![CDATA[10.1109/TIA.2014.2308392]]></doi>
##     <publicationId><![CDATA[6748940]]></publicationId>
##     <mdurl><![CDATA[http://ieeexplore.ieee.org/xpl/articleDetails.jsp?tp=&arnumber=6748940&contentType=Journals+%26+Magazines]]></mdurl>
##     <pdf><![CDATA[http://ieeexplore.ieee.org/stamp/stamp.jsp?arnumber=6748940]]></pdf>
##   </document>
##   <document>
##     <rank>1002</rank>
##     <title><![CDATA[High-Efficiency Low-Crosstalk 1310-nm Polarization Splitter and Rotator]]></title>
##     <authors><![CDATA[Hang Guan;  Novack, A.;  Streshinsky, M.;  Ruizhi Shi;  Yang Liu;  Qing Fang;  Lim, A.E.-J.;  Guo-Qiang Lo;  Baehr-Jones, T.;  Hochberg, M.]]></authors>
##     <affiliations><![CDATA[Dept. of Electr. & Comput. Eng., Nat. Univ. of Singapore, Singapore, Singapore]]></affiliations>
##     <controlledterms>
##       <term><![CDATA[integrated optoelectronics]]></term>
##       <term><![CDATA[light polarisation]]></term>
##       <term><![CDATA[optical beam splitters]]></term>
##       <term><![CDATA[optical crosstalk]]></term>
##       <term><![CDATA[optical directional couplers]]></term>
##       <term><![CDATA[optical losses]]></term>
##       <term><![CDATA[optical rotation]]></term>
##       <term><![CDATA[silicon-on-insulator]]></term>
##       <term><![CDATA[ultraviolet lithography]]></term>
##     </controlledterms>
##     <thesaurusterms>
##       <term><![CDATA[Directional couplers]]></term>
##       <term><![CDATA[Gratings]]></term>
##       <term><![CDATA[Loss measurement]]></term>
##       <term><![CDATA[Optimized production technology]]></term>
##       <term><![CDATA[Photonics]]></term>
##       <term><![CDATA[Silicon]]></term>
##     </thesaurusterms>
##     <pubtitle><![CDATA[Photonics Technology Letters, IEEE]]></pubtitle>
##     <punumber><![CDATA[68]]></punumber>
##     <pubtype><![CDATA[Journals & Magazines]]></pubtype>
##     <publisher><![CDATA[IEEE]]></publisher>
##     <volume><![CDATA[26]]></volume>
##     <issue><![CDATA[9]]></issue>
##     <py><![CDATA[2014]]></py>
##     <spage><![CDATA[925]]></spage>
##     <epage><![CDATA[928]]></epage>
##     <abstract><![CDATA[We demonstrate the first example of a polarization splitter and rotator (PSR) at 1310 nm built on a silicon-on-insulator platform using 248-nm deep-ultraviolet lithography. The PSR is constructed with a directional coupler, a bilevel taper-based TM0-to-TE1 mode converter, and an asymmetric Mach-Zehnder-based TE1-to-TE0 mode converter. A worst-case TM0-to-TE0 mode-conversion loss of 2 dB, with polarization crosstalk lower than -20 dB over a wide bandwidth of 40 nm is experimentally demonstrated. The worst-case polarization-dependent loss is 0.76 dB.]]></abstract>
##     <issn><![CDATA[1041-1135]]></issn>
##     <arnumber><![CDATA[6762848]]></arnumber>
##     <doi><![CDATA[10.1109/LPT.2014.2310635]]></doi>
##     <publicationId><![CDATA[6762848]]></publicationId>
##     <mdurl><![CDATA[http://ieeexplore.ieee.org/xpl/articleDetails.jsp?tp=&arnumber=6762848&contentType=Journals+%26+Magazines]]></mdurl>
##     <pdf><![CDATA[http://ieeexplore.ieee.org/stamp/stamp.jsp?arnumber=6762848]]></pdf>
##   </document>
##   <document>
##     <rank>1003</rank>
##     <title><![CDATA[Enhancement of Bilateral Cortical Somatosensory Evoked Potentials to Intact Forelimb Stimulation Following Thoracic Contusion Spinal Cord Injury in Rats]]></title>
##     <authors><![CDATA[Bazley, F.A.;  Maybhate, A.;  Chuen Seng Tan;  Thakor, N.V.;  Kerr, C.;  All, A.H.]]></authors>
##     <affiliations><![CDATA[Dept. of Biomed. Eng., Johns Hopkins Univ., Baltimore, MD, USA]]></affiliations>
##     <controlledterms>
##       <term><![CDATA[bioelectric potentials]]></term>
##       <term><![CDATA[biomedical MRI]]></term>
##       <term><![CDATA[image enhancement]]></term>
##       <term><![CDATA[injuries]]></term>
##       <term><![CDATA[medical image processing]]></term>
##       <term><![CDATA[neurophysiology]]></term>
##       <term><![CDATA[patient monitoring]]></term>
##       <term><![CDATA[somatosensory phenomena]]></term>
##     </controlledterms>
##     <thesaurusterms>
##       <term><![CDATA[Educational institutions]]></term>
##       <term><![CDATA[Electrodes]]></term>
##       <term><![CDATA[Fasteners]]></term>
##       <term><![CDATA[Injuries]]></term>
##       <term><![CDATA[Rats]]></term>
##       <term><![CDATA[Skin]]></term>
##       <term><![CDATA[Spinal cord]]></term>
##     </thesaurusterms>
##     <pubtitle><![CDATA[Neural Systems and Rehabilitation Engineering, IEEE Transactions on]]></pubtitle>
##     <punumber><![CDATA[7333]]></punumber>
##     <pubtype><![CDATA[Journals & Magazines]]></pubtype>
##     <publisher><![CDATA[IEEE]]></publisher>
##     <volume><![CDATA[22]]></volume>
##     <issue><![CDATA[5]]></issue>
##     <py><![CDATA[2014]]></py>
##     <spage><![CDATA[953]]></spage>
##     <epage><![CDATA[964]]></epage>
##     <abstract><![CDATA[The adult central nervous system is capable of significant reorganization and adaptation following neurotrauma. After a thoracic contusive spinal cord injury (SCI) neuropathways that innervate the cord below the epicenter of injury are damaged, with minimal prospects for functional recovery. In contrast, pathways above the site of injury remain intact and may undergo adaptive changes in response to injury. We used cortical somatosensory evoked potentials (SSEPs) to evaluate changes in intact forelimb pathways. Rats received a midline contusion SCI, unilateral contusion SCI, or laminectomy with no contusion at the T8 level and were monitored for 28 days post-injury. In the midline injury group, SSEPs recorded from the contralateral forelimb region of the primary somatosensory cortex were 59.7% (CI 34.7%, 84.8%; c<sup>2</sup> = 21.9; dof = 1; p = 2.9 &#x00D7;10<sup>-6</sup>) greater than the laminectomy group; SSEPs from the ipsilateral somatosensory cortex were 47.6% (CI 18.3%, 77%; c<sup>2</sup> = 10.1; dof = 1; p = 0.001) greater. Activation of the ipsilateral somatosensory cortex was further supported by BOLD-fMRI, which showed increased oxygenation at the ipsilateral hemisphere at day seven post-injury. In the unilateral injury group, ipsilesional side was compared to the contralesional side. SSEPs on day 14 (148%; CI 111%, 185%) and day 21 (137%; CI 110%, 163%) for ipsilesional forelimb stimulation were significantly increased over baseline (100%). SSEPs recorded from the hindlimb sensory cortex upon ipsilesional stimulation were 33.9% (CI 14.3%, 53.4%; c<sup>2</sup> = 11.6; dof = 1; p = 0.0007) greater than contralesional stimulation. Therefore, these results demonstrate the ability of SSEPs to detect significant enhancements in the activation of forelimb sensory pathways following both midline and unilateral contusive SCI at T8. Reorganization of forelimb pathways may occur after thoracic SCI, which SSEPs can monitor to aid the development of future therapie- .]]></abstract>
##     <issn><![CDATA[1534-4320]]></issn>
##     <htmlFlag><![CDATA[1]]></htmlFlag>
##     <arnumber><![CDATA[6807805]]></arnumber>
##     <doi><![CDATA[10.1109/TNSRE.2014.2319313]]></doi>
##     <publicationId><![CDATA[6807805]]></publicationId>
##     <mdurl><![CDATA[http://ieeexplore.ieee.org/xpl/articleDetails.jsp?tp=&arnumber=6807805&contentType=Journals+%26+Magazines]]></mdurl>
##     <pdf><![CDATA[http://ieeexplore.ieee.org/stamp/stamp.jsp?arnumber=6807805]]></pdf>
##   </document>
##   <document>
##     <rank>1004</rank>
##     <title><![CDATA[Think Small: Nanopores for Sensing and Synthesis]]></title>
##     <authors><![CDATA[Timp, W.;  Nice, A.M.;  Nelson, E.M.;  Kurz, V.;  McKelvey, K.;  Timp, G.]]></authors>
##     <affiliations><![CDATA[Dept. of Biomed. Eng., Johns Hopkins Univ., Baltimore, MD, USA]]></affiliations>
##     <controlledterms>
##       <term><![CDATA[DNA]]></term>
##       <term><![CDATA[atomic force microscopy]]></term>
##       <term><![CDATA[cellular biophysics]]></term>
##       <term><![CDATA[molecular biophysics]]></term>
##       <term><![CDATA[nanobiotechnology]]></term>
##       <term><![CDATA[nanosensors]]></term>
##       <term><![CDATA[proteins]]></term>
##     </controlledterms>
##     <thesaurusterms>
##       <term><![CDATA[Cells (biology)]]></term>
##       <term><![CDATA[DNA]]></term>
##       <term><![CDATA[Genomics]]></term>
##       <term><![CDATA[Nanobioscience]]></term>
##       <term><![CDATA[Nanoparticles]]></term>
##       <term><![CDATA[Proteins]]></term>
##     </thesaurusterms>
##     <pubtitle><![CDATA[Access, IEEE]]></pubtitle>
##     <punumber><![CDATA[6287639]]></punumber>
##     <pubtype><![CDATA[Journals & Magazines]]></pubtype>
##     <publisher><![CDATA[IEEE]]></publisher>
##     <volume><![CDATA[2]]></volume>
##     <py><![CDATA[2014]]></py>
##     <spage><![CDATA[1396]]></spage>
##     <epage><![CDATA[1408]]></epage>
##     <abstract><![CDATA[It is now possible to manipulate individual molecules using a nanopore to read DNA and proteins, or write DNA by inserting mini-genes into cells. Furthermore, development of these methodologies will kick open the door to new biology and chemistry that has been logistically intractable previously. Nanopore technology will place molecular and sub-molecular analysis within the reach of the typical bench-top scientist or clinical lab-no longer limited to genomics or mass spectrometry specialists. Moreover, the prospects for synthetic biology-using nanopores to program or reprogram cells-are promising as well, but have been examined only at the level of a single cell, so far.]]></abstract>
##     <issn><![CDATA[2169-3536]]></issn>
##     <htmlFlag><![CDATA[1]]></htmlFlag>
##     <arnumber><![CDATA[6962896]]></arnumber>
##     <doi><![CDATA[10.1109/ACCESS.2014.2369506]]></doi>
##     <publicationId><![CDATA[6962896]]></publicationId>
##     <mdurl><![CDATA[http://ieeexplore.ieee.org/xpl/articleDetails.jsp?tp=&arnumber=6962896&contentType=Journals+%26+Magazines]]></mdurl>
##     <pdf><![CDATA[http://ieeexplore.ieee.org/stamp/stamp.jsp?arnumber=6962896]]></pdf>
##   </document>
##   <document>
##     <rank>1005</rank>
##     <title><![CDATA[Quality Assessment of Despeckled SAR Images]]></title>
##     <authors><![CDATA[Dellepiane, S.G.;  Angiati, E.]]></authors>
##     <affiliations><![CDATA[Dept. of Electr., Electron., Telecommun. Eng. & Naval Archit. (DITEN), Univ. of Genoa, Genoa, Italy]]></affiliations>
##     <controlledterms>
##       <term><![CDATA[diffusion]]></term>
##       <term><![CDATA[image resolution]]></term>
##       <term><![CDATA[image texture]]></term>
##       <term><![CDATA[nonlinear filters]]></term>
##       <term><![CDATA[radar imaging]]></term>
##       <term><![CDATA[statistical analysis]]></term>
##       <term><![CDATA[synthetic aperture radar]]></term>
##     </controlledterms>
##     <thesaurusterms>
##       <term><![CDATA[Image edge detection]]></term>
##       <term><![CDATA[Indexes]]></term>
##       <term><![CDATA[Noise]]></term>
##       <term><![CDATA[Noise reduction]]></term>
##       <term><![CDATA[Quality assessment]]></term>
##       <term><![CDATA[Speckle]]></term>
##       <term><![CDATA[Synthetic aperture radar]]></term>
##     </thesaurusterms>
##     <pubtitle><![CDATA[Selected Topics in Applied Earth Observations and Remote Sensing, IEEE Journal of]]></pubtitle>
##     <punumber><![CDATA[4609443]]></punumber>
##     <pubtype><![CDATA[Journals & Magazines]]></pubtype>
##     <publisher><![CDATA[IEEE]]></publisher>
##     <volume><![CDATA[7]]></volume>
##     <issue><![CDATA[2]]></issue>
##     <py><![CDATA[2014]]></py>
##     <spage><![CDATA[691]]></spage>
##     <epage><![CDATA[707]]></epage>
##     <abstract><![CDATA[In this paper, a novel method for the quality assessment of despeckled SAR images is proposed. This method is based on the observation that the perceived quality of despeckled SAR images is not always appropriately described by classical statistical and deterministic parameters that are proposed in the literature. Various evaluations are performed here. A preliminary visual qualitative evaluation is taken as a reference for the subsequent quantitative assessment. Then, a revised statistical analysis that can solve some of the drawbacks of previous methods is proposed; however, the statistical approach still has certain drawbacks. To address this problem, a new frequency analysis approach is first proposed, together with a definition of the appropriate indexes. In this way, it is possible to select the best filter in terms of noise reduction, edge and texture preservation, while limiting the effect of introduced distortions. While statistical analysis is widely used in the literature, frequency analysis has never been presented for this aim, especially for non-linear filters. We prove that frequency analysis can robustly identify the best filter, taking perceptual considerations into account, even when statistical analysis fails. Despeckling methods based on anisotropic diffusion algorithms are used for a comparison, but the proposed analysis can be applied to any filtering method. Experiments are presented with SAR images from the Italian Cosmo/Skymed constellation. Both Stripmap and Spotlight acquisitions have been evaluated, and to prove the validity of the proposed method with respect to different spatial resolutions and different classes of interest, various classes are considered.]]></abstract>
##     <issn><![CDATA[1939-1404]]></issn>
##     <htmlFlag><![CDATA[1]]></htmlFlag>
##     <arnumber><![CDATA[6595646]]></arnumber>
##     <doi><![CDATA[10.1109/JSTARS.2013.2279501]]></doi>
##     <publicationId><![CDATA[6595646]]></publicationId>
##     <mdurl><![CDATA[http://ieeexplore.ieee.org/xpl/articleDetails.jsp?tp=&arnumber=6595646&contentType=Journals+%26+Magazines]]></mdurl>
##     <pdf><![CDATA[http://ieeexplore.ieee.org/stamp/stamp.jsp?arnumber=6595646]]></pdf>
##   </document>
##   <document>
##     <rank>1006</rank>
##     <title><![CDATA[Sub-10-nm Asymmetric Junctionless Tunnel Field-Effect Transistors]]></title>
##     <authors><![CDATA[Chun-Hsing Shih;  Nguyen Van Kien]]></authors>
##     <affiliations><![CDATA[Dept. of Electr. Eng., Nat. Chi Nan Univ., Nantou, Taiwan]]></affiliations>
##     <controlledterms>
##       <term><![CDATA[CMOS integrated circuits]]></term>
##       <term><![CDATA[Ge-Si alloys]]></term>
##       <term><![CDATA[field effect transistors]]></term>
##       <term><![CDATA[insulators]]></term>
##       <term><![CDATA[tunnel transistors]]></term>
##     </controlledterms>
##     <thesaurusterms>
##       <term><![CDATA[CMOS integrated circuits]]></term>
##       <term><![CDATA[CMOS technology]]></term>
##       <term><![CDATA[IEEE Electron Devices Society]]></term>
##       <term><![CDATA[Logic gates]]></term>
##       <term><![CDATA[Silicon]]></term>
##       <term><![CDATA[Switches]]></term>
##       <term><![CDATA[Transistors]]></term>
##     </thesaurusterms>
##     <pubtitle><![CDATA[Electron Devices Society, IEEE Journal of the]]></pubtitle>
##     <punumber><![CDATA[6245494]]></punumber>
##     <pubtype><![CDATA[Journals & Magazines]]></pubtype>
##     <publisher><![CDATA[IEEE]]></publisher>
##     <volume><![CDATA[2]]></volume>
##     <issue><![CDATA[5]]></issue>
##     <py><![CDATA[2014]]></py>
##     <spage><![CDATA[128]]></spage>
##     <epage><![CDATA[132]]></epage>
##     <abstract><![CDATA[This study presents a new asymmetric junctionless tunnel field-effect transistor (AJ-TFET) to scale TFETs into sub-10-nm regimes. The asymmetric junctionless p+ source/body and junctional n/p+ drain/body separately optimize the lateral source and drain coupling to efficiently switch the TFETs, producing an abrupt on-off switching. Because of n-drain/p+body junction, the off-state tunnel barrier can be extended into the drain, ensuring an excellent short-channel effect without the limitation of channel lengths. Si/Ge heterojunctions and high-k gate insulators are combined with the AJ-TFETs for additional on-current boosting. Using compact structures and feasible parameters from practical Si-based CMOS technologies, the advancement in the on-off switching and short-channel effect make the AJ-TFET highly promising as an ideal approach into the sub-10-nm regimes.]]></abstract>
##     <issn><![CDATA[2168-6734]]></issn>
##     <htmlFlag><![CDATA[1]]></htmlFlag>
##     <arnumber><![CDATA[6832427]]></arnumber>
##     <doi><![CDATA[10.1109/JEDS.2014.2330501]]></doi>
##     <publicationId><![CDATA[6832427]]></publicationId>
##     <mdurl><![CDATA[http://ieeexplore.ieee.org/xpl/articleDetails.jsp?tp=&arnumber=6832427&contentType=Journals+%26+Magazines]]></mdurl>
##     <pdf><![CDATA[http://ieeexplore.ieee.org/stamp/stamp.jsp?arnumber=6832427]]></pdf>
##   </document>
##   <document>
##     <rank>1007</rank>
##     <title><![CDATA[Utilizing Two-Dimensional Photonic Crystals in Different Arrangement to Investigate the Correlation Between the Air Duty Cycle and the Light Extraction Enhancement of InGaN-Based Light-Emitting Diodes]]></title>
##     <authors><![CDATA[Ming-Lun Lee;  Yao-Hong You;  Ray-Ming Lin;  Cheng-Ju Hsieh;  Vin-Cent Su;  Po-Hsun Chen;  Chieh-Hsiung Kuan]]></authors>
##     <affiliations><![CDATA[Dept. of Electr. Eng., Nat. Taiwan Univ., Taipei, Taiwan]]></affiliations>
##     <controlledterms>
##       <term><![CDATA[III-V semiconductors]]></term>
##       <term><![CDATA[gallium compounds]]></term>
##       <term><![CDATA[indium compounds]]></term>
##       <term><![CDATA[light emitting diodes]]></term>
##       <term><![CDATA[optical materials]]></term>
##       <term><![CDATA[photonic crystals]]></term>
##       <term><![CDATA[wide band gap semiconductors]]></term>
##     </controlledterms>
##     <thesaurusterms>
##       <term><![CDATA[Correlation]]></term>
##       <term><![CDATA[Gallium nitride]]></term>
##       <term><![CDATA[Lattices]]></term>
##       <term><![CDATA[Light emitting diodes]]></term>
##       <term><![CDATA[Measurement by laser beam]]></term>
##       <term><![CDATA[Optical variables measurement]]></term>
##       <term><![CDATA[Semiconductor device measurement]]></term>
##     </thesaurusterms>
##     <pubtitle><![CDATA[Photonics Journal, IEEE]]></pubtitle>
##     <punumber><![CDATA[4563994]]></punumber>
##     <pubtype><![CDATA[Journals & Magazines]]></pubtype>
##     <publisher><![CDATA[IEEE]]></publisher>
##     <volume><![CDATA[6]]></volume>
##     <issue><![CDATA[3]]></issue>
##     <py><![CDATA[2014]]></py>
##     <spage><![CDATA[1]]></spage>
##     <epage><![CDATA[8]]></epage>
##     <abstract><![CDATA[This paper demonstrates that the optimal light extraction enhancement of the 2-D photonic crystals (PhCs) light-emitting diodes (LEDs) among different air duty cycles (ADADCsCs) is independent of the geometry and the shape of the 2-D PhCs. Moreover, it also discusses that the side-directional emission property of the 2-D PhCs LEDs in hexagonal lattice arrangement (HLA) is better than that in square lattice arrangement (SLA). Finally, the light output power of the 2-D PhCs LEDs in SLA and in HLA with the ADC of 51% are significantly improved by 60.4% and 81.9%, respectively, as compared with the reference LED at an injection current of 350 mA.]]></abstract>
##     <issn><![CDATA[1943-0655]]></issn>
##     <htmlFlag><![CDATA[1]]></htmlFlag>
##     <arnumber><![CDATA[6814773]]></arnumber>
##     <doi><![CDATA[10.1109/JPHOT.2014.2323304]]></doi>
##     <publicationId><![CDATA[6814773]]></publicationId>
##     <mdurl><![CDATA[http://ieeexplore.ieee.org/xpl/articleDetails.jsp?tp=&arnumber=6814773&contentType=Journals+%26+Magazines]]></mdurl>
##     <pdf><![CDATA[http://ieeexplore.ieee.org/stamp/stamp.jsp?arnumber=6814773]]></pdf>
##   </document>
##   <document>
##     <rank>1008</rank>
##     <title><![CDATA[Passively Q-Switched Nd:YAG Laser With Graphene Oxide in Heavy Water]]></title>
##     <authors><![CDATA[Qiao Wen;  Xiaojun Zhang;  Yonggang Wang;  Yishan Wang;  Hanben Niu]]></authors>
##     <affiliations><![CDATA[Key Lab. of Optoelectron. Devices & Syst. of Minist. of Educ. & Guangdong Province, Shenzhen Univ., Shenzhen, China]]></affiliations>
##     <controlledterms>
##       <term><![CDATA[Q-switching]]></term>
##       <term><![CDATA[graphene]]></term>
##       <term><![CDATA[heavy water]]></term>
##       <term><![CDATA[laser beams]]></term>
##       <term><![CDATA[neodymium]]></term>
##       <term><![CDATA[optical saturable absorption]]></term>
##       <term><![CDATA[solid lasers]]></term>
##       <term><![CDATA[transparency]]></term>
##     </controlledterms>
##     <thesaurusterms>
##       <term><![CDATA[Graphene]]></term>
##       <term><![CDATA[Laser excitation]]></term>
##       <term><![CDATA[Laser mode locking]]></term>
##       <term><![CDATA[Liquids]]></term>
##       <term><![CDATA[Optical fiber dispersion]]></term>
##       <term><![CDATA[Pump lasers]]></term>
##       <term><![CDATA[Semiconductor lasers]]></term>
##     </thesaurusterms>
##     <pubtitle><![CDATA[Photonics Journal, IEEE]]></pubtitle>
##     <punumber><![CDATA[4563994]]></punumber>
##     <pubtype><![CDATA[Journals & Magazines]]></pubtype>
##     <publisher><![CDATA[IEEE]]></publisher>
##     <volume><![CDATA[6]]></volume>
##     <issue><![CDATA[2]]></issue>
##     <py><![CDATA[2014]]></py>
##     <spage><![CDATA[1]]></spage>
##     <epage><![CDATA[6]]></epage>
##     <abstract><![CDATA[Liquid saturable absorbers have some advantages in some aspects over solid saturable absorbers. In this paper, graphene oxide in heavy water (deuteroxide, D2O) dispersion was fabricated and used as absorber in Q-switched Nd:YAG laser for the first time, to the best of our knowledge. The shortest pulse width of the Q-switched laser is 523 ns, corresponding to a repetition rate of 92.9 kHz. The maximum peak power of Q-switched pulse is 6.6 W. Such a liquid absorber has the virtues of good optical transparency, high heat dissipation, and no contact damage.]]></abstract>
##     <issn><![CDATA[1943-0655]]></issn>
##     <htmlFlag><![CDATA[1]]></htmlFlag>
##     <arnumber><![CDATA[6755494]]></arnumber>
##     <doi><![CDATA[10.1109/JPHOT.2014.2309631]]></doi>
##     <publicationId><![CDATA[6755494]]></publicationId>
##     <mdurl><![CDATA[http://ieeexplore.ieee.org/xpl/articleDetails.jsp?tp=&arnumber=6755494&contentType=Journals+%26+Magazines]]></mdurl>
##     <pdf><![CDATA[http://ieeexplore.ieee.org/stamp/stamp.jsp?arnumber=6755494]]></pdf>
##   </document>
##   <document>
##     <rank>1009</rank>
##     <title><![CDATA[Terahertz Sensor for Non-Contact Thickness and Quality Measurement of Automobile Paints of Varying Complexity]]></title>
##     <authors><![CDATA[Ke Su;  Yao-chun Shen;  Zeitler, J.A.]]></authors>
##     <affiliations><![CDATA[Dept. of Chem. Eng. & Biotechnol., Univ. of Cambridge, Cambridge, UK]]></affiliations>
##     <controlledterms>
##       <term><![CDATA[automobile manufacture]]></term>
##       <term><![CDATA[coatings]]></term>
##       <term><![CDATA[computerised tomography]]></term>
##       <term><![CDATA[eddy current testing]]></term>
##       <term><![CDATA[paints]]></term>
##       <term><![CDATA[quality control]]></term>
##       <term><![CDATA[terahertz wave detectors]]></term>
##       <term><![CDATA[ultrasonic materials testing]]></term>
##     </controlledterms>
##     <thesaurusterms>
##       <term><![CDATA[Current measurement]]></term>
##       <term><![CDATA[Optical refraction]]></term>
##       <term><![CDATA[Optical variables control]]></term>
##       <term><![CDATA[Paints]]></term>
##       <term><![CDATA[Substrates]]></term>
##       <term><![CDATA[Thickness measurement]]></term>
##       <term><![CDATA[Ultrasonic variables measurement]]></term>
##     </thesaurusterms>
##     <pubtitle><![CDATA[Terahertz Science and Technology, IEEE Transactions on]]></pubtitle>
##     <punumber><![CDATA[5503871]]></punumber>
##     <pubtype><![CDATA[Journals & Magazines]]></pubtype>
##     <publisher><![CDATA[IEEE]]></publisher>
##     <volume><![CDATA[4]]></volume>
##     <issue><![CDATA[4]]></issue>
##     <py><![CDATA[2014]]></py>
##     <spage><![CDATA[432]]></spage>
##     <epage><![CDATA[439]]></epage>
##     <abstract><![CDATA[In this paper, we propose to use terahertz pulsed imaging (TPI) as a novel tool to measure the thickness and quality of up to four layers of car paint on both metallic and non-metallic substrates. Using a rigorous one-dimensional electromagnetic model for terahertz propagation in a multi-layered medium combined with a numerical fitting method, the refractive index, extinction coefficient, and thickness of individual paint layers were determined. This proposed method was shown to be able to resolve coating layers down to a thickness of 18 &#x03BC;m and was validated for both single- and multi-layer automobile paint samples. Results of the terahertz measurements were benchmarked against other techniques that are currently used for non-destructive testing during car manufacture: ultrasound and eddy current measurements, as well as two reference techniques, X-ray microcomputed tomography and surface profilometry. Good consistency was found between the techniques. Compared to conventional techniques, TPI has the advantage that it is a non-contact method and that it is able to spatially resolve the thickness uniformity distribution information by two-dimensional mapping.]]></abstract>
##     <issn><![CDATA[2156-342X]]></issn>
##     <htmlFlag><![CDATA[1]]></htmlFlag>
##     <arnumber><![CDATA[6827225]]></arnumber>
##     <doi><![CDATA[10.1109/TTHZ.2014.2325393]]></doi>
##     <publicationId><![CDATA[6827225]]></publicationId>
##     <mdurl><![CDATA[http://ieeexplore.ieee.org/xpl/articleDetails.jsp?tp=&arnumber=6827225&contentType=Journals+%26+Magazines]]></mdurl>
##     <pdf><![CDATA[http://ieeexplore.ieee.org/stamp/stamp.jsp?arnumber=6827225]]></pdf>
##   </document>
##   <document>
##     <rank>1010</rank>
##     <title><![CDATA[Shrinkage-Thresholding Enhanced Born Iterative Method for Solving 2D Inverse Electromagnetic Scattering Problem]]></title>
##     <authors><![CDATA[Desmal, A.;  Bagci, H.]]></authors>
##     <affiliations><![CDATA[Div. of Comput., Electr., & Math. Sci. & Eng., King Abdullah Univ. of Sci. & Technol., Thuwal, Saudi Arabia]]></affiliations>
##     <controlledterms>
##       <term><![CDATA[electromagnetic wave scattering]]></term>
##       <term><![CDATA[iterative methods]]></term>
##       <term><![CDATA[numerical analysis]]></term>
##     </controlledterms>
##     <thesaurusterms>
##       <term><![CDATA[Cost function]]></term>
##       <term><![CDATA[Equations]]></term>
##       <term><![CDATA[Iterative methods]]></term>
##       <term><![CDATA[Minimization]]></term>
##       <term><![CDATA[Noise]]></term>
##       <term><![CDATA[Permittivity]]></term>
##       <term><![CDATA[Scattering]]></term>
##     </thesaurusterms>
##     <pubtitle><![CDATA[Antennas and Propagation, IEEE Transactions on]]></pubtitle>
##     <punumber><![CDATA[8]]></punumber>
##     <pubtype><![CDATA[Journals & Magazines]]></pubtype>
##     <publisher><![CDATA[IEEE]]></publisher>
##     <volume><![CDATA[62]]></volume>
##     <issue><![CDATA[7]]></issue>
##     <py><![CDATA[2014]]></py>
##     <spage><![CDATA[3878]]></spage>
##     <epage><![CDATA[3884]]></epage>
##     <abstract><![CDATA[A numerical framework that incorporates recently developed iterative shrinkage thresholding (IST) algorithms within the Born iterative method (BIM) is proposed for solving the two-dimensional inverse electromagnetic scattering problem. IST algorithms minimize a cost function weighted between measurement-data misfit and a zeroth/first-norm penalty term and therefore promote &#x201C;sharpness&#x201D; in the solution. Consequently, when applied to domains with sharp variations, discontinuities, or sparse content, the proposed framework is more efficient and accurate than the &#x201C;classical&#x201D; BIM that minimizes a cost function with a second-norm penalty term. Indeed, numerical results demonstrate the superiority of the IST-BIM over the classical BIM when they are applied to sparse domains: Permittivity and conductivity profiles recovered using the IST-BIM are sharper and more accurate and converge faster.]]></abstract>
##     <issn><![CDATA[0018-926X]]></issn>
##     <htmlFlag><![CDATA[1]]></htmlFlag>
##     <arnumber><![CDATA[6808475]]></arnumber>
##     <doi><![CDATA[10.1109/TAP.2014.2321144]]></doi>
##     <publicationId><![CDATA[6808475]]></publicationId>
##     <mdurl><![CDATA[http://ieeexplore.ieee.org/xpl/articleDetails.jsp?tp=&arnumber=6808475&contentType=Journals+%26+Magazines]]></mdurl>
##     <pdf><![CDATA[http://ieeexplore.ieee.org/stamp/stamp.jsp?arnumber=6808475]]></pdf>
##   </document>
##   <document>
##     <rank>1011</rank>
##     <title><![CDATA[Component-Oriented Modeling of Thermoelectric Devices for Energy System Design]]></title>
##     <authors><![CDATA[Felgner, F.;  Exel, L.;  Nesarajah, M.;  Frey, G.]]></authors>
##     <affiliations><![CDATA[Dept. of Autom., Saarland Univ., Saarbrucken, Germany]]></affiliations>
##     <controlledterms>
##       <term><![CDATA[Seebeck effect]]></term>
##       <term><![CDATA[power systems]]></term>
##       <term><![CDATA[thermoelectric conversion]]></term>
##       <term><![CDATA[thermoelectric devices]]></term>
##       <term><![CDATA[waste heat]]></term>
##     </controlledterms>
##     <pubtitle><![CDATA[Industrial Electronics, IEEE Transactions on]]></pubtitle>
##     <punumber><![CDATA[41]]></punumber>
##     <pubtype><![CDATA[Journals & Magazines]]></pubtype>
##     <publisher><![CDATA[IEEE]]></publisher>
##     <volume><![CDATA[61]]></volume>
##     <issue><![CDATA[3]]></issue>
##     <py><![CDATA[2014]]></py>
##     <spage><![CDATA[1301]]></spage>
##     <epage><![CDATA[1310]]></epage>
##     <abstract><![CDATA[Thermoelectric (TE) devices are used in the form of Peltier coolers and as TE generators, with the latter producing electrical energy from waste heat, based on the Seebeck effect. In both cases, modeling of the TE device is a prerequisite for the design and control verification of the resulting overall energy system. To this end, the model has to be integrated seamlessly in an overall system model containing other electrical, thermodynamic, or even mechanical components. Following this premise, this paper presents a component-based model for TE devices described in the Modelica language. The model incorporates the temperature dependences of decisive material properties (Seebeck coefficient, thermal conductivity, and electrical resistivity) in 1-D spatial resolution. With the help of few additional geometrical parameters, e.g., the thickness of TE legs, the model is capable of describing the dynamic behavior of the TE device in accordance with the experimental results.]]></abstract>
##     <issn><![CDATA[0278-0046]]></issn>
##     <htmlFlag><![CDATA[1]]></htmlFlag>
##     <arnumber><![CDATA[6512017]]></arnumber>
##     <doi><![CDATA[10.1109/TIE.2013.2261037]]></doi>
##     <publicationId><![CDATA[6512017]]></publicationId>
##     <mdurl><![CDATA[http://ieeexplore.ieee.org/xpl/articleDetails.jsp?tp=&arnumber=6512017&contentType=Journals+%26+Magazines]]></mdurl>
##     <pdf><![CDATA[http://ieeexplore.ieee.org/stamp/stamp.jsp?arnumber=6512017]]></pdf>
##   </document>
##   <document>
##     <rank>1012</rank>
##     <title><![CDATA[Hyperspectral and LiDAR Data Fusion: Outcome of the 2013 GRSS Data Fusion Contest]]></title>
##     <authors><![CDATA[Debes, C.;  Merentitis, A.;  Heremans, R.;  Hahn, J.;  Frangiadakis, N.;  van Kasteren, T.;  Wenzhi Liao;  Bellens, R.;  Pizurica, A.;  Gautama, S.;  Philips, W.;  Prasad, S.;  Qian Du;  Pacifici, F.]]></authors>
##     <affiliations><![CDATA[AGT Int., Darmstadt, Germany]]></affiliations>
##     <controlledterms>
##       <term><![CDATA[geophysical image processing]]></term>
##       <term><![CDATA[graph theory]]></term>
##       <term><![CDATA[hyperspectral imaging]]></term>
##       <term><![CDATA[image classification]]></term>
##       <term><![CDATA[learning (artificial intelligence)]]></term>
##       <term><![CDATA[remote sensing by radar]]></term>
##     </controlledterms>
##     <thesaurusterms>
##       <term><![CDATA[Data integration]]></term>
##       <term><![CDATA[Feature extraction]]></term>
##       <term><![CDATA[Hyperspectral imaging]]></term>
##       <term><![CDATA[Laser radar]]></term>
##       <term><![CDATA[Vegetation mapping]]></term>
##     </thesaurusterms>
##     <pubtitle><![CDATA[Selected Topics in Applied Earth Observations and Remote Sensing, IEEE Journal of]]></pubtitle>
##     <punumber><![CDATA[4609443]]></punumber>
##     <pubtype><![CDATA[Journals & Magazines]]></pubtype>
##     <publisher><![CDATA[IEEE]]></publisher>
##     <volume><![CDATA[7]]></volume>
##     <issue><![CDATA[6]]></issue>
##     <py><![CDATA[2014]]></py>
##     <spage><![CDATA[2405]]></spage>
##     <epage><![CDATA[2418]]></epage>
##     <abstract><![CDATA[The 2013 Data Fusion Contest organized by the Data Fusion Technical Committee (DFTC) of the IEEE Geoscience and Remote Sensing Society aimed at investigating the synergistic use of hyperspectral and Light Detection And Ranging (LiDAR) data. The data sets distributed to the participants during the Contest, a hyperspectral imagery and the corresponding LiDAR-derived digital surface model (DSM), were acquired by the NSF-funded Center for Airborne Laser Mapping over the University of Houston campus and its neighboring area in the summer of 2012. This paper highlights the two awarded research contributions, which investigated different approaches for the fusion of hyperspectral and LiDAR data, including a combined unsupervised and supervised classification scheme, and a graph-based method for the fusion of spectral, spatial, and elevation information.]]></abstract>
##     <issn><![CDATA[1939-1404]]></issn>
##     <arnumber><![CDATA[6776408]]></arnumber>
##     <doi><![CDATA[10.1109/JSTARS.2014.2305441]]></doi>
##     <publicationId><![CDATA[6776408]]></publicationId>
##     <mdurl><![CDATA[http://ieeexplore.ieee.org/xpl/articleDetails.jsp?tp=&arnumber=6776408&contentType=Journals+%26+Magazines]]></mdurl>
##     <pdf><![CDATA[http://ieeexplore.ieee.org/stamp/stamp.jsp?arnumber=6776408]]></pdf>
##   </document>
##   <document>
##     <rank>1013</rank>
##     <title><![CDATA[Joint Channel and Doppler Offset Estimation in Dynamic Cooperative Relay Networks]]></title>
##     <authors><![CDATA[Nevat, I.;  Peters, G.W.;  Doucet, A.;  Jinhong Yuan]]></authors>
##     <affiliations><![CDATA[Inst. for Infocomm Res. (I2R), A*STAR, Singapore, Singapore]]></affiliations>
##     <controlledterms>
##       <term><![CDATA[Bayes methods]]></term>
##       <term><![CDATA[Markov processes]]></term>
##       <term><![CDATA[Monte Carlo methods]]></term>
##       <term><![CDATA[channel estimation]]></term>
##       <term><![CDATA[computational complexity]]></term>
##       <term><![CDATA[cooperative communication]]></term>
##       <term><![CDATA[mean square error methods]]></term>
##       <term><![CDATA[particle filtering (numerical methods)]]></term>
##       <term><![CDATA[relay networks (telecommunication)]]></term>
##       <term><![CDATA[state-space methods]]></term>
##       <term><![CDATA[time-varying channels]]></term>
##     </controlledterms>
##     <thesaurusterms>
##       <term><![CDATA[Channel estimation]]></term>
##       <term><![CDATA[Cooperative communication]]></term>
##       <term><![CDATA[Doppler effect]]></term>
##       <term><![CDATA[Markov processes]]></term>
##       <term><![CDATA[Relays]]></term>
##       <term><![CDATA[Wireless communication]]></term>
##     </thesaurusterms>
##     <pubtitle><![CDATA[Wireless Communications, IEEE Transactions on]]></pubtitle>
##     <punumber><![CDATA[7693]]></punumber>
##     <pubtype><![CDATA[Journals & Magazines]]></pubtype>
##     <publisher><![CDATA[IEEE]]></publisher>
##     <volume><![CDATA[13]]></volume>
##     <issue><![CDATA[12]]></issue>
##     <py><![CDATA[2014]]></py>
##     <spage><![CDATA[6570]]></spage>
##     <epage><![CDATA[6579]]></epage>
##     <abstract><![CDATA[We develop a new and efficient algorithm to solve the problem of joint channel and Doppler offset estimation in time-varying cooperative wireless relay networks. We first formulate the problem as a Bayesian dynamic nonlinear state space model, then develop an algorithm, which is based on particle adaptive marginal Markov chain Monte Carlo, method to jointly estimate the time-varying channels and static Doppler offsets. We perform detailed complexity analysis of the proposed algorithm and show that it is very efficient and requires moderate computational complexity. In addition, we develop a new version of the recursive marginal Crame&#x0301;r-Rao lower bound and derive expressions for the achievable mean-square error. Simulation results demonstrate that the proposed algorithm outperforms the state-of-the-art algorithms and performs close to the Crame&#x0301;r-Rao lower bound.]]></abstract>
##     <issn><![CDATA[1536-1276]]></issn>
##     <htmlFlag><![CDATA[1]]></htmlFlag>
##     <arnumber><![CDATA[6918531]]></arnumber>
##     <doi><![CDATA[10.1109/TWC.2014.2361760]]></doi>
##     <publicationId><![CDATA[6918531]]></publicationId>
##     <mdurl><![CDATA[http://ieeexplore.ieee.org/xpl/articleDetails.jsp?tp=&arnumber=6918531&contentType=Journals+%26+Magazines]]></mdurl>
##     <pdf><![CDATA[http://ieeexplore.ieee.org/stamp/stamp.jsp?arnumber=6918531]]></pdf>
##   </document>
##   <document>
##     <rank>1014</rank>
##     <title><![CDATA[Optimal Design of Multichannel Equalizers for the Structural Similarity Index]]></title>
##     <authors><![CDATA[Li Chai;  Yuxia Sheng]]></authors>
##     <affiliations><![CDATA[Sch. of Inf. Sci. & Eng., Wuhan Univ. of Sci. & Technol., Wuhan, China]]></affiliations>
##     <controlledterms>
##       <term><![CDATA[image restoration]]></term>
##       <term><![CDATA[mean square error methods]]></term>
##     </controlledterms>
##     <thesaurusterms>
##       <term><![CDATA[Algorithm design and analysis]]></term>
##       <term><![CDATA[Equalizers]]></term>
##       <term><![CDATA[Image reconstruction]]></term>
##       <term><![CDATA[Image restoration]]></term>
##       <term><![CDATA[Indexes]]></term>
##       <term><![CDATA[Noise]]></term>
##       <term><![CDATA[Optimization]]></term>
##     </thesaurusterms>
##     <pubtitle><![CDATA[Image Processing, IEEE Transactions on]]></pubtitle>
##     <punumber><![CDATA[83]]></punumber>
##     <pubtype><![CDATA[Journals & Magazines]]></pubtype>
##     <publisher><![CDATA[IEEE]]></publisher>
##     <volume><![CDATA[23]]></volume>
##     <issue><![CDATA[12]]></issue>
##     <py><![CDATA[2014]]></py>
##     <spage><![CDATA[5626]]></spage>
##     <epage><![CDATA[5637]]></epage>
##     <abstract><![CDATA[The optimization of multichannel equalizers is studied for the structural similarity (SSIM) criteria. The closed-form formula is provided for the optimal equalizer when the mean of the source is zero. The formula shows that the equalizer with maximal SSIM index is equal to the one with minimal mean square error (MSE) multiplied by a positive real number, which is shown to be equal to the inverse of the achieved SSIM index. The relation of the maximal SSIM index to the minimal MSE is also established for given blurring filters and fixed length equalizers. An algorithm is also presented to compute the suboptimal equalizer for the general sources. Various numerical examples are given to demonstrate the effectiveness of the results.]]></abstract>
##     <issn><![CDATA[1057-7149]]></issn>
##     <htmlFlag><![CDATA[1]]></htmlFlag>
##     <arnumber><![CDATA[6948352]]></arnumber>
##     <doi><![CDATA[10.1109/TIP.2014.2367320]]></doi>
##     <publicationId><![CDATA[6948352]]></publicationId>
##     <mdurl><![CDATA[http://ieeexplore.ieee.org/xpl/articleDetails.jsp?tp=&arnumber=6948352&contentType=Journals+%26+Magazines]]></mdurl>
##     <pdf><![CDATA[http://ieeexplore.ieee.org/stamp/stamp.jsp?arnumber=6948352]]></pdf>
##   </document>
##   <document>
##     <rank>1015</rank>
##     <title><![CDATA[Eigenvalue Signal Processing for Weather Radar Polarimetry: Removing the Bias Induced by Antenna Coherent Cross-Channel Coupling]]></title>
##     <authors><![CDATA[Galletti, M.;  Zrnic, D.S.;  Gekat, F.;  Goelz, P.]]></authors>
##     <affiliations><![CDATA[Environ. & Climate Sci. Dept., Brookhaven Nat. Lab., Upton, NY, USA]]></affiliations>
##     <controlledterms>
##       <term><![CDATA[antenna radiation patterns]]></term>
##       <term><![CDATA[eigenvalues and eigenfunctions]]></term>
##       <term><![CDATA[geophysical signal processing]]></term>
##       <term><![CDATA[meteorological radar]]></term>
##       <term><![CDATA[radar polarimetry]]></term>
##       <term><![CDATA[reflectivity]]></term>
##       <term><![CDATA[reflector antennas]]></term>
##     </controlledterms>
##     <thesaurusterms>
##       <term><![CDATA[Antenna measurements]]></term>
##       <term><![CDATA[Correlation]]></term>
##       <term><![CDATA[Covariance matrices]]></term>
##       <term><![CDATA[Eigenvalues and eigenfunctions]]></term>
##       <term><![CDATA[Meteorology]]></term>
##       <term><![CDATA[Radar polarimetry]]></term>
##     </thesaurusterms>
##     <pubtitle><![CDATA[Geoscience and Remote Sensing, IEEE Transactions on]]></pubtitle>
##     <punumber><![CDATA[36]]></punumber>
##     <pubtype><![CDATA[Journals & Magazines]]></pubtype>
##     <publisher><![CDATA[IEEE]]></publisher>
##     <volume><![CDATA[52]]></volume>
##     <issue><![CDATA[12]]></issue>
##     <py><![CDATA[2014]]></py>
##     <spage><![CDATA[7695]]></spage>
##     <epage><![CDATA[7707]]></epage>
##     <abstract><![CDATA[We present a novel digital signal processing procedure, named eigenvalue signal processing (henceforth ESP), patented by the first author with Brookhaven Science Associates in 2013. The method enables the removal of the bias due to antenna coherent cross-channel coupling and is applicable in the LDR mode, the ATSR mode and the STSR orthogonal mode of weather radar measurements. In this paper, we focus on the LDR mode and consider copolar reflectivity at horizontal transmit (Z<sub>HH</sub>), cross-polar reflectivity at horizontal transmit (Z<sub>VH</sub>), linear depolarization ratio at horizontal transmit (LDRH) and degree of polarization at horizontal transmit (DOP<sub>H</sub>). The ESP (ESP) method is substantiated by an experiment carried out in November 2012 using C-band weather radar with a parabolic reflector located at the Selex ES-Gematronik facilities in Neuss, Germany. The experiment involved comparison of weather radar measurements taken 1.5 minutes apart in two hardware configurations, namely with cross-coupling on (cc-on) and cross-coupling off (cc-off). It is experimentally demonstrated that eigenvaluederived variables are invariant with respect to antenna coherent cross-channel coupling. This property had to be expected, since the eigenvalues of the Coherency matrix are SU(2) invariant.]]></abstract>
##     <issn><![CDATA[0196-2892]]></issn>
##     <htmlFlag><![CDATA[1]]></htmlFlag>
##     <arnumber><![CDATA[6810149]]></arnumber>
##     <doi><![CDATA[10.1109/TGRS.2014.2316821]]></doi>
##     <publicationId><![CDATA[6810149]]></publicationId>
##     <mdurl><![CDATA[http://ieeexplore.ieee.org/xpl/articleDetails.jsp?tp=&arnumber=6810149&contentType=Journals+%26+Magazines]]></mdurl>
##     <pdf><![CDATA[http://ieeexplore.ieee.org/stamp/stamp.jsp?arnumber=6810149]]></pdf>
##   </document>
##   <document>
##     <rank>1016</rank>
##     <title><![CDATA[A Review of the Pinned Photodiode for CCD and CMOS Image Sensors]]></title>
##     <authors><![CDATA[Fossum, E.R.;  Hondongwa, D.B.]]></authors>
##     <affiliations><![CDATA[Thayer Sch. of Eng., Dartmouth Coll., Hanover, NH, USA]]></affiliations>
##     <controlledterms>
##       <term><![CDATA[CCD image sensors]]></term>
##       <term><![CDATA[CMOS image sensors]]></term>
##       <term><![CDATA[p-i-n photodiodes]]></term>
##       <term><![CDATA[photodetectors]]></term>
##     </controlledterms>
##     <thesaurusterms>
##       <term><![CDATA[CMOS image sensors]]></term>
##       <term><![CDATA[CMOS integrated circuits]]></term>
##       <term><![CDATA[Charge coupled devices]]></term>
##       <term><![CDATA[Charge transfer]]></term>
##       <term><![CDATA[Electric potential]]></term>
##       <term><![CDATA[Photodiodes]]></term>
##     </thesaurusterms>
##     <pubtitle><![CDATA[Electron Devices Society, IEEE Journal of the]]></pubtitle>
##     <punumber><![CDATA[6245494]]></punumber>
##     <pubtype><![CDATA[Journals & Magazines]]></pubtype>
##     <publisher><![CDATA[IEEE]]></publisher>
##     <volume><![CDATA[2]]></volume>
##     <issue><![CDATA[3]]></issue>
##     <py><![CDATA[2014]]></py>
##     <spage><![CDATA[33]]></spage>
##     <epage><![CDATA[43]]></epage>
##     <abstract><![CDATA[The pinned photodiode is the primary photodetector structure used in most CCD and CMOS image sensors. This paper reviews the development, physics, and technology of the pinned photodiode.]]></abstract>
##     <issn><![CDATA[2168-6734]]></issn>
##     <arnumber><![CDATA[6742594]]></arnumber>
##     <doi><![CDATA[10.1109/JEDS.2014.2306412]]></doi>
##     <publicationId><![CDATA[6742594]]></publicationId>
##     <mdurl><![CDATA[http://ieeexplore.ieee.org/xpl/articleDetails.jsp?tp=&arnumber=6742594&contentType=Journals+%26+Magazines]]></mdurl>
##     <pdf><![CDATA[http://ieeexplore.ieee.org/stamp/stamp.jsp?arnumber=6742594]]></pdf>
##   </document>
##   <document>
##     <rank>1017</rank>
##     <title><![CDATA[Influence of Heterogeneous Soils and Clutter on the Performance of Ground-Penetrating Radar for Landmine Detection]]></title>
##     <authors><![CDATA[Takahashi, K.;  Igel, J.;  Preetz, H.;  Sato, M.]]></authors>
##     <affiliations><![CDATA[Center for Northeast Asian Studies, Tohoku Univ., Sendai, Japan]]></affiliations>
##     <controlledterms>
##       <term><![CDATA[geophysical techniques]]></term>
##       <term><![CDATA[ground penetrating radar]]></term>
##       <term><![CDATA[landmine detection]]></term>
##       <term><![CDATA[radar clutter]]></term>
##       <term><![CDATA[sensor fusion]]></term>
##       <term><![CDATA[soil]]></term>
##     </controlledterms>
##     <thesaurusterms>
##       <term><![CDATA[Clutter]]></term>
##       <term><![CDATA[Detectors]]></term>
##       <term><![CDATA[Dielectrics]]></term>
##       <term><![CDATA[Ground penetrating radar]]></term>
##       <term><![CDATA[Metals]]></term>
##       <term><![CDATA[Permittivity]]></term>
##       <term><![CDATA[Soil]]></term>
##     </thesaurusterms>
##     <pubtitle><![CDATA[Geoscience and Remote Sensing, IEEE Transactions on]]></pubtitle>
##     <punumber><![CDATA[36]]></punumber>
##     <pubtype><![CDATA[Journals & Magazines]]></pubtype>
##     <publisher><![CDATA[IEEE]]></publisher>
##     <volume><![CDATA[52]]></volume>
##     <issue><![CDATA[6]]></issue>
##     <py><![CDATA[2014]]></py>
##     <spage><![CDATA[3464]]></spage>
##     <epage><![CDATA[3472]]></epage>
##     <abstract><![CDATA[Ground-penetrating radar (GPR) has been studied for landmine detection and identification. Since this application employs higher frequencies as compared to conventional largescale GPR measurements, the GPR performance is greatly influenced by soil properties and their spatial heterogeneity. In order to study the influence of soil heterogeneity on GPR performance, three types of soil were investigated. From the soil heterogeneity, GPR clutter was modeled with the aim of assessing the difficulty encountered in successful GPR performance. A handheld dual-sensor system that combines a metal detector and GPR was tested in these three test soils, and its performance for identifying buried objects was evaluated. The GPR performance obtained from the test showed a clear correlation with the modeled GPR clutter. Hence, the present study illustrates that clutter plays a major role in the detection of small objects in heterogeneous soil by GPR.]]></abstract>
##     <issn><![CDATA[0196-2892]]></issn>
##     <htmlFlag><![CDATA[1]]></htmlFlag>
##     <arnumber><![CDATA[6572856]]></arnumber>
##     <doi><![CDATA[10.1109/TGRS.2013.2273082]]></doi>
##     <publicationId><![CDATA[6572856]]></publicationId>
##     <mdurl><![CDATA[http://ieeexplore.ieee.org/xpl/articleDetails.jsp?tp=&arnumber=6572856&contentType=Journals+%26+Magazines]]></mdurl>
##     <pdf><![CDATA[http://ieeexplore.ieee.org/stamp/stamp.jsp?arnumber=6572856]]></pdf>
##   </document>
##   <document>
##     <rank>1018</rank>
##     <title><![CDATA[Design for a Single-Polarization Photonic Crystal Fiber Wavelength Splitter Based on Hybrid-Surface Plasmon Resonance]]></title>
##     <authors><![CDATA[Lei Chen;  Weigang Zhang;  Zhao Zhang;  Yongji Liu;  Sieg, J.;  Liyu Zhang;  Quan Zhou;  Li Wang;  Biao Wang;  Tieyi Yan]]></authors>
##     <affiliations><![CDATA[Lab. of Opto-Electron. Inf. Sci. & Technol., Minist. of Educ., Nankai Univ., Tianjin, China]]></affiliations>
##     <controlledterms>
##       <term><![CDATA[finite element analysis]]></term>
##       <term><![CDATA[holey fibres]]></term>
##       <term><![CDATA[optical beam splitters]]></term>
##       <term><![CDATA[optical design techniques]]></term>
##       <term><![CDATA[optical fibre losses]]></term>
##       <term><![CDATA[optical fibre polarisation]]></term>
##       <term><![CDATA[photonic crystals]]></term>
##       <term><![CDATA[surface plasmon resonance]]></term>
##     </controlledterms>
##     <thesaurusterms>
##       <term><![CDATA[Gold]]></term>
##       <term><![CDATA[Optical fiber communication]]></term>
##       <term><![CDATA[Optical fiber devices]]></term>
##       <term><![CDATA[Optical fiber dispersion]]></term>
##       <term><![CDATA[Optical fiber polarization]]></term>
##       <term><![CDATA[Wires]]></term>
##     </thesaurusterms>
##     <pubtitle><![CDATA[Photonics Journal, IEEE]]></pubtitle>
##     <punumber><![CDATA[4563994]]></punumber>
##     <pubtype><![CDATA[Journals & Magazines]]></pubtype>
##     <publisher><![CDATA[IEEE]]></publisher>
##     <volume><![CDATA[6]]></volume>
##     <issue><![CDATA[4]]></issue>
##     <py><![CDATA[2014]]></py>
##     <spage><![CDATA[1]]></spage>
##     <epage><![CDATA[9]]></epage>
##     <abstract><![CDATA[A novel single-polarization photonic crystal fiber wavelength splitter based on hybrid-surface plasmon resonance is proposed. A full-vector finite-element method is applied to analyze the guiding properties. Numerical simulations show that the proposed splitter, which is only several hundred microns in length, gives single polarization in the 1.31-&#x03BC;m and 1.55-&#x03BC;m bands. The loss of the unwanted polarized mode is 102.6 and 245.0 dB/cm in the two aforementioned communication windows, respectively, and the corresponding insertion loss is as low as 3.5 and 1.7 dB/cm, respectively. Moreover, the dependence of the bandwidth on the fiber length is given, and according to that function, the bandwidth can reach 40 nm (1.31-&#x03BC;m band) and 100 nm (1.55-&#x03BC;m band) when the fiber length is up to 1 mm. Additionally, the tolerances for a realistic fabrication are analyzed. In the last part, we discuss other methods to deal with an anticrossing phenomenon in detail.]]></abstract>
##     <issn><![CDATA[1943-0655]]></issn>
##     <htmlFlag><![CDATA[1]]></htmlFlag>
##     <arnumber><![CDATA[6837416]]></arnumber>
##     <doi><![CDATA[10.1109/JPHOT.2014.2331237]]></doi>
##     <publicationId><![CDATA[6837416]]></publicationId>
##     <mdurl><![CDATA[http://ieeexplore.ieee.org/xpl/articleDetails.jsp?tp=&arnumber=6837416&contentType=Journals+%26+Magazines]]></mdurl>
##     <pdf><![CDATA[http://ieeexplore.ieee.org/stamp/stamp.jsp?arnumber=6837416]]></pdf>
##   </document>
##   <document>
##     <rank>1019</rank>
##     <title><![CDATA[Guided-Mode Resonances in GaN Membrane Grating]]></title>
##     <authors><![CDATA[Zheng Shi;  Xumin Gao;  Xin Li;  Fangren Hu;  Lopez-Garcia, M.;  Cryan, M.J.;  Hongbo Zhu;  Yongjin Wang]]></authors>
##     <affiliations><![CDATA[Gruenberg Res. Centre, Nanjing Univ. of Posts & Telecommun., Nanjing, China]]></affiliations>
##     <controlledterms>
##       <term><![CDATA[III-V semiconductors]]></term>
##       <term><![CDATA[diffraction gratings]]></term>
##       <term><![CDATA[elemental semiconductors]]></term>
##       <term><![CDATA[etching]]></term>
##       <term><![CDATA[gallium compounds]]></term>
##       <term><![CDATA[integrated optics]]></term>
##       <term><![CDATA[nanophotonics]]></term>
##       <term><![CDATA[optical tuning]]></term>
##       <term><![CDATA[reflectivity]]></term>
##       <term><![CDATA[resonance]]></term>
##       <term><![CDATA[silicon]]></term>
##       <term><![CDATA[wide band gap semiconductors]]></term>
##     </controlledterms>
##     <thesaurusterms>
##       <term><![CDATA[Filling]]></term>
##       <term><![CDATA[Gallium nitride]]></term>
##       <term><![CDATA[Gratings]]></term>
##       <term><![CDATA[Optical device fabrication]]></term>
##       <term><![CDATA[Optical polarization]]></term>
##       <term><![CDATA[Optical reflection]]></term>
##       <term><![CDATA[Reflectivity]]></term>
##     </thesaurusterms>
##     <pubtitle><![CDATA[Photonics Journal, IEEE]]></pubtitle>
##     <punumber><![CDATA[4563994]]></punumber>
##     <pubtype><![CDATA[Journals & Magazines]]></pubtype>
##     <publisher><![CDATA[IEEE]]></publisher>
##     <volume><![CDATA[6]]></volume>
##     <issue><![CDATA[3]]></issue>
##     <py><![CDATA[2014]]></py>
##     <spage><![CDATA[1]]></spage>
##     <epage><![CDATA[7]]></epage>
##     <abstract><![CDATA[This paper proposes guided-mode resonant GaN grating that are implemented on a GaN-on-silicon wafer structure. Numerical simulations based on rigorous coupled wave analysis (RCWA) are performed to model multiple resonances in thick GaN membranes. Back wafer etching of freestanding GaN membranes is used as a tuning mechanism to manipulate the optical performance in the visible wavelength range. Angular resolved micro reflectance measurements are conducted to characterize the fabricated GaN gratings and show the dependence of guided-mode resonances on the filling factor of the gratings, the membrane thickness, and the polarization of incident beam. The experimental results agree well with numerical simulations. This paper opens the way to fabricate guided-mode resonant GaN grating for many diverse applications.]]></abstract>
##     <issn><![CDATA[1943-0655]]></issn>
##     <htmlFlag><![CDATA[1]]></htmlFlag>
##     <arnumber><![CDATA[6820731]]></arnumber>
##     <doi><![CDATA[10.1109/JPHOT.2014.2326680]]></doi>
##     <publicationId><![CDATA[6820731]]></publicationId>
##     <mdurl><![CDATA[http://ieeexplore.ieee.org/xpl/articleDetails.jsp?tp=&arnumber=6820731&contentType=Journals+%26+Magazines]]></mdurl>
##     <pdf><![CDATA[http://ieeexplore.ieee.org/stamp/stamp.jsp?arnumber=6820731]]></pdf>
##   </document>
##   <document>
##     <rank>1020</rank>
##     <title><![CDATA[Backpacking: Energy-Efficient Deployment of Heterogeneous Radios in Multi-Radio High-Data-Rate Wireless Sensor Networks]]></title>
##     <authors><![CDATA[Alim Al Islam, A.B.M.;  Hossain, M.S.;  Raghunathan, V.;  Hu, Y.C.]]></authors>
##     <affiliations><![CDATA[Dept. of Comput. Sci. & Eng., Bangladesh Univ. of Eng. & Technol., Dhaka, Bangladesh]]></affiliations>
##     <controlledterms>
##       <term><![CDATA[Zigbee]]></term>
##       <term><![CDATA[energy conservation]]></term>
##       <term><![CDATA[sensor placement]]></term>
##       <term><![CDATA[telecommunication power management]]></term>
##       <term><![CDATA[wireless LAN]]></term>
##       <term><![CDATA[wireless sensor networks]]></term>
##     </controlledterms>
##     <thesaurusterms>
##       <term><![CDATA[Bandwidth allocation]]></term>
##       <term><![CDATA[Energy consumption]]></term>
##       <term><![CDATA[Energy efficiency]]></term>
##       <term><![CDATA[IEEE 802.11 Standards]]></term>
##       <term><![CDATA[IEEE 802.15 Standards]]></term>
##       <term><![CDATA[Mathematical model]]></term>
##       <term><![CDATA[Power distribution]]></term>
##       <term><![CDATA[Throughput]]></term>
##       <term><![CDATA[Wireless sensor networks]]></term>
##     </thesaurusterms>
##     <pubtitle><![CDATA[Access, IEEE]]></pubtitle>
##     <punumber><![CDATA[6287639]]></punumber>
##     <pubtype><![CDATA[Journals & Magazines]]></pubtype>
##     <publisher><![CDATA[IEEE]]></publisher>
##     <volume><![CDATA[2]]></volume>
##     <py><![CDATA[2014]]></py>
##     <spage><![CDATA[1281]]></spage>
##     <epage><![CDATA[1306]]></epage>
##     <abstract><![CDATA[The early success of wireless sensor networks has led to a new generation of increasingly sophisticated sensor network applications, such as HP's CeNSE. These applications demand high network throughput that easily exceeds the capability of the low-power 802.15.4 radios most commonly used in today's sensor nodes. To address this issue, this paper investigates an energy-efficient approach to supplementing an 802.15.4-based wireless sensor network with high bandwidth, high power, longer range radios, such as 802.11. Exploiting a key observation that the high-bandwidth radio achieves low energy consumption per bit of transmitted data due to its inherent transmission efficiency, we propose a hybrid network architecture that utilizes an optimal density of dual-radio (802.15.4 and 802.11) nodes to augment a sensor network having only 802.15.4 radios. We present a cross-layer mathematical model to calculate this optimal density, which strikes a delicate balance between the low energy consumption per transmitted bit of the high-bandwidth radio and low sleep power of the 802.15.4. Experimental results obtained using a wireless sensor network testbed reveal that our architecture improves the average energy per bit, time elapsed before the first node drains its battery, time elapsed before half of the nodes drain their batteries, and end-to-end delay by significant margins compared with a network having only 802.15.4.]]></abstract>
##     <issn><![CDATA[2169-3536]]></issn>
##     <htmlFlag><![CDATA[1]]></htmlFlag>
##     <arnumber><![CDATA[6933867]]></arnumber>
##     <doi><![CDATA[10.1109/ACCESS.2014.2364234]]></doi>
##     <publicationId><![CDATA[6933867]]></publicationId>
##     <mdurl><![CDATA[http://ieeexplore.ieee.org/xpl/articleDetails.jsp?tp=&arnumber=6933867&contentType=Journals+%26+Magazines]]></mdurl>
##     <pdf><![CDATA[http://ieeexplore.ieee.org/stamp/stamp.jsp?arnumber=6933867]]></pdf>
##   </document>
##   <document>
##     <rank>1021</rank>
##     <title><![CDATA[A Novel Birefrigent Photonic Crystal Fiber Surface Plasmon Resonance Biosensor]]></title>
##     <authors><![CDATA[Otupiri, R.;  Akowuah, E.K.;  Haxha, S.;  Ademgil, H.;  AbdelMalek, F.;  Aggoun, A.]]></authors>
##     <affiliations><![CDATA[Dept. of Electr. & Electron. Eng., Kwame Nkrumah Univ. of Sci. & Technol., Kumansi, Ghana]]></affiliations>
##     <controlledterms>
##       <term><![CDATA[biosensors]]></term>
##       <term><![CDATA[birefringence]]></term>
##       <term><![CDATA[holey fibres]]></term>
##       <term><![CDATA[photonic crystals]]></term>
##       <term><![CDATA[surface plasmon resonance]]></term>
##     </controlledterms>
##     <thesaurusterms>
##       <term><![CDATA[Biosensors]]></term>
##       <term><![CDATA[Optical fiber sensors]]></term>
##       <term><![CDATA[Optical fibers]]></term>
##       <term><![CDATA[Optical surface waves]]></term>
##       <term><![CDATA[Plasmons]]></term>
##       <term><![CDATA[Refractive index]]></term>
##       <term><![CDATA[Surface waves]]></term>
##     </thesaurusterms>
##     <pubtitle><![CDATA[Photonics Journal, IEEE]]></pubtitle>
##     <punumber><![CDATA[4563994]]></punumber>
##     <pubtype><![CDATA[Journals & Magazines]]></pubtype>
##     <publisher><![CDATA[IEEE]]></publisher>
##     <volume><![CDATA[6]]></volume>
##     <issue><![CDATA[4]]></issue>
##     <py><![CDATA[2014]]></py>
##     <spage><![CDATA[1]]></spage>
##     <epage><![CDATA[11]]></epage>
##     <abstract><![CDATA[A numerical analysis of a novel birefringent photonic crystal fiber (PCF) biosensor constructed on the surface plasmon resonance (SPR) model is presented in this paper. This biosensor configuration utilizes circular air holes to introduce birefringence into the structure. This PCF biosensor model shows promise in the area of multiple detection using HE<sup>x</sup><sub>11</sub> and HE<sup>y</sup><sub>11</sub> modes to sense more than one analyte. A numerical study of the biosensor is performed in two interrogation modes: amplitude and wavelength. Sensor resolution values with spectral interrogation yielded 5 &#x00D7; 10<sup>-5</sup> RIU (refractive index units) for HE<sup>x</sup><sub>11</sub> modes and 6 &#x00D7; 10<sup>-5</sup> RIU for HE<sup>y</sup><sub>11</sub> modes, whereas 3 &#x00D7; 10<sup>-5</sup> RIU for HE<sup>x</sup><sub>11</sub> modes and 4 &#x00D7; 10<sup>-5</sup> RIU for HE<sup>y</sup><sub>11</sub> modes are demonstrated for the amplitude interrogation.]]></abstract>
##     <issn><![CDATA[1943-0655]]></issn>
##     <htmlFlag><![CDATA[1]]></htmlFlag>
##     <arnumber><![CDATA[6849423]]></arnumber>
##     <doi><![CDATA[10.1109/JPHOT.2014.2335716]]></doi>
##     <publicationId><![CDATA[6849423]]></publicationId>
##     <mdurl><![CDATA[http://ieeexplore.ieee.org/xpl/articleDetails.jsp?tp=&arnumber=6849423&contentType=Journals+%26+Magazines]]></mdurl>
##     <pdf><![CDATA[http://ieeexplore.ieee.org/stamp/stamp.jsp?arnumber=6849423]]></pdf>
##   </document>
##   <document>
##     <rank>1022</rank>
##     <title><![CDATA[On the Use of Gaussian Approximation in Analyzing the Performance of Optical Receivers]]></title>
##     <authors><![CDATA[El-Howayek, G.;  Zhang, C.;  Li, Y.;  Ng, J.S.;  David, J.P.R.;  Hayat, M.M.]]></authors>
##     <affiliations><![CDATA[Dept. of Electr. & Comput. Eng., Univ. of New Mexico, Albuquerque, NM, USA]]></affiliations>
##     <controlledterms>
##       <term><![CDATA[Gaussian distribution]]></term>
##       <term><![CDATA[avalanche photodiodes]]></term>
##       <term><![CDATA[error statistics]]></term>
##       <term><![CDATA[intersymbol interference]]></term>
##       <term><![CDATA[optical receivers]]></term>
##     </controlledterms>
##     <thesaurusterms>
##       <term><![CDATA[Approximation methods]]></term>
##       <term><![CDATA[Bit error rate]]></term>
##       <term><![CDATA[Educational institutions]]></term>
##       <term><![CDATA[Noise]]></term>
##       <term><![CDATA[Optical receivers]]></term>
##       <term><![CDATA[Photonics]]></term>
##     </thesaurusterms>
##     <pubtitle><![CDATA[Photonics Journal, IEEE]]></pubtitle>
##     <punumber><![CDATA[4563994]]></punumber>
##     <pubtype><![CDATA[Journals & Magazines]]></pubtype>
##     <publisher><![CDATA[IEEE]]></publisher>
##     <volume><![CDATA[6]]></volume>
##     <issue><![CDATA[1]]></issue>
##     <py><![CDATA[2014]]></py>
##     <spage><![CDATA[1]]></spage>
##     <epage><![CDATA[8]]></epage>
##     <abstract><![CDATA[The analytical calculation of the bit error rate (BER) of digital optical receivers that employ avalanche photodiodes (APDs) is challenging due to 1) the stochastic nature of the avalanche photodiode's impulse-response function and 2) the presence of intersymbol interference (ISI). At ultrafast transmission rates, ISI becomes a dominant component of the BER, and its effect on the BER should be carefully addressed. One solution to this problem, termed the bit-pattern-dependent (PD) approach, is to first calculate the conditional BER given a specific bit pattern and then average over all possible bit patterns. Alternatively, a simplifying method, termed the bit-pattern-independent (PI) approach, has been commonly used whereby the average bit stream is used to calculate the distribution of the receiver output, which, in turn, is used to calculate the BER. However, when ISI is dominant, the PI approximation is inaccurate. Here, the two approaches are analytically compared by analyzing their asymptotic behavior and their bounds. Conditions are found to determine when the PI method overestimates the BER. The BER found using the PD method exponentially decays with the received optical power, whereas for the PI approach, the BER converges to a constant, which is unrealistic. For an InP-based APD receiver with a 100-nm multiplication layer, the PI method is found to be inaccurate for transmission rates beyond 20 Gb/s.]]></abstract>
##     <issn><![CDATA[1943-0655]]></issn>
##     <htmlFlag><![CDATA[1]]></htmlFlag>
##     <arnumber><![CDATA[6725617]]></arnumber>
##     <doi><![CDATA[10.1109/JPHOT.2014.2302792]]></doi>
##     <publicationId><![CDATA[6725617]]></publicationId>
##     <mdurl><![CDATA[http://ieeexplore.ieee.org/xpl/articleDetails.jsp?tp=&arnumber=6725617&contentType=Journals+%26+Magazines]]></mdurl>
##     <pdf><![CDATA[http://ieeexplore.ieee.org/stamp/stamp.jsp?arnumber=6725617]]></pdf>
##   </document>
##   <document>
##     <rank>1023</rank>
##     <title><![CDATA[Confidentiality-Preserving Image Search: A Comparative Study Between Homomorphic Encryption and Distance-Preserving Randomization]]></title>
##     <authors><![CDATA[Wenjun Lu;  Varna, A.L.;  Min Wu]]></authors>
##     <affiliations><![CDATA[Google, Mountain View, CA, USA]]></affiliations>
##     <controlledterms>
##       <term><![CDATA[communication complexity]]></term>
##       <term><![CDATA[content-based retrieval]]></term>
##       <term><![CDATA[cryptography]]></term>
##       <term><![CDATA[image retrieval]]></term>
##     </controlledterms>
##     <thesaurusterms>
##       <term><![CDATA[Complexity theory]]></term>
##       <term><![CDATA[Encryption]]></term>
##       <term><![CDATA[Image processing]]></term>
##       <term><![CDATA[Image storage]]></term>
##       <term><![CDATA[Multimedia communication]]></term>
##       <term><![CDATA[Online services]]></term>
##       <term><![CDATA[Search methods]]></term>
##     </thesaurusterms>
##     <pubtitle><![CDATA[Access, IEEE]]></pubtitle>
##     <punumber><![CDATA[6287639]]></punumber>
##     <pubtype><![CDATA[Journals & Magazines]]></pubtype>
##     <publisher><![CDATA[IEEE]]></publisher>
##     <volume><![CDATA[2]]></volume>
##     <py><![CDATA[2014]]></py>
##     <spage><![CDATA[125]]></spage>
##     <epage><![CDATA[141]]></epage>
##     <abstract><![CDATA[Recent years have seen increasing popularity of storing and managing personal multimedia data using online services. Preserving confidentiality of online personal data while offering efficient functionalities thus becomes an important and pressing research issue. In this paper, we study the problem of content-based search of image data archived online while preserving content confidentiality. The problem has different settings from those typically considered in the secure computation literature, as it deals with data in rank-ordered search, and has a different security-efficiency requirement. Secure computation techniques, such as homomorphic encryption, can potentially be used in this application, at a cost of high computational and communication complexity. Alternatively, efficient techniques based on randomizing visual feature and search indexes have been proposed recently to enable similarity comparison between encrypted images. This paper focuses on comparing these two major paradigms of techniques, namely, homomorphic encryption-based techniques and feature/index randomization-based techniques, for confidentiality-preserving image search. We develop novel and systematic metrics to quantitatively evaluate security strength in this unique type of data and applications. We compare these two paradigms of techniques in terms of their search performance, security strength, and computational efficiency. The insights obtained through this paper and comparison will help design practical algorithms appropriate for privacy-aware cloud multimedia systems.]]></abstract>
##     <issn><![CDATA[2169-3536]]></issn>
##     <htmlFlag><![CDATA[1]]></htmlFlag>
##     <arnumber><![CDATA[6746001]]></arnumber>
##     <doi><![CDATA[10.1109/ACCESS.2014.2307057]]></doi>
##     <publicationId><![CDATA[6746001]]></publicationId>
##     <mdurl><![CDATA[http://ieeexplore.ieee.org/xpl/articleDetails.jsp?tp=&arnumber=6746001&contentType=Journals+%26+Magazines]]></mdurl>
##     <pdf><![CDATA[http://ieeexplore.ieee.org/stamp/stamp.jsp?arnumber=6746001]]></pdf>
##   </document>
##   <document>
##     <rank>1024</rank>
##     <title><![CDATA[Analysis of Cavity&#x2013;Waveguide Coupling in Two-Dimensional Photonic Crystals]]></title>
##     <authors><![CDATA[Tanaka, A.;  Kondow, M.;  Morifuji, M.]]></authors>
##     <affiliations><![CDATA[Dept. of Quantum Electron., Osaka Univ., Suita, Japan]]></affiliations>
##     <controlledterms>
##       <term><![CDATA[finite difference time-domain analysis]]></term>
##       <term><![CDATA[optical resonators]]></term>
##       <term><![CDATA[optical waveguides]]></term>
##       <term><![CDATA[photonic crystals]]></term>
##     </controlledterms>
##     <thesaurusterms>
##       <term><![CDATA[Cavity resonators]]></term>
##       <term><![CDATA[Couplings]]></term>
##       <term><![CDATA[Finite difference methods]]></term>
##       <term><![CDATA[Optical waveguides]]></term>
##       <term><![CDATA[Time-domain analysis]]></term>
##       <term><![CDATA[Whispering gallery modes]]></term>
##     </thesaurusterms>
##     <pubtitle><![CDATA[Photonics Journal, IEEE]]></pubtitle>
##     <punumber><![CDATA[4563994]]></punumber>
##     <pubtype><![CDATA[Journals & Magazines]]></pubtype>
##     <publisher><![CDATA[IEEE]]></publisher>
##     <volume><![CDATA[6]]></volume>
##     <issue><![CDATA[3]]></issue>
##     <py><![CDATA[2014]]></py>
##     <spage><![CDATA[1]]></spage>
##     <epage><![CDATA[9]]></epage>
##     <abstract><![CDATA[We investigate theoretically cavity-waveguide coupling in 2-D photonic crystals. In using photonic crystal cavities in a device such as an optical circuit, the cavities must be coupled to a waveguide. For optimum performance, careful design of the device structure is necessary. In order to find out the rule for the design, we evaluate coupling strength by carrying out finite-difference time-domain (FDTD) calculations with systematic variation of structural parameters. Results are interpreted in terms of the density of the states of waveguide bands and symmetry of the cavity and waveguide modes. We found that the selection rule based on parity of the modes breaks down because of linewidth broadening, which arises from transition in finite time.]]></abstract>
##     <issn><![CDATA[1943-0655]]></issn>
##     <htmlFlag><![CDATA[1]]></htmlFlag>
##     <arnumber><![CDATA[6804650]]></arnumber>
##     <doi><![CDATA[10.1109/JPHOT.2014.2317673]]></doi>
##     <publicationId><![CDATA[6804650]]></publicationId>
##     <mdurl><![CDATA[http://ieeexplore.ieee.org/xpl/articleDetails.jsp?tp=&arnumber=6804650&contentType=Journals+%26+Magazines]]></mdurl>
##     <pdf><![CDATA[http://ieeexplore.ieee.org/stamp/stamp.jsp?arnumber=6804650]]></pdf>
##   </document>
##   <document>
##     <rank>1025</rank>
##     <title><![CDATA[Improved Gait Algorithm and Mobility Performance of RT-Mover Type Personal Mobility Vehicle]]></title>
##     <authors><![CDATA[Nakajima, S.]]></authors>
##     <affiliations><![CDATA[Dept. of Adv. Robot., Chiba Inst. of Technol., Narashino, Japan]]></affiliations>
##     <controlledterms>
##       <term><![CDATA[mobile robots]]></term>
##       <term><![CDATA[motion control]]></term>
##       <term><![CDATA[position control]]></term>
##       <term><![CDATA[road vehicles]]></term>
##       <term><![CDATA[wheelchairs]]></term>
##     </controlledterms>
##     <thesaurusterms>
##       <term><![CDATA[Algorithm design and analysis]]></term>
##       <term><![CDATA[Collision avoidance]]></term>
##       <term><![CDATA[Gait recognition]]></term>
##       <term><![CDATA[Intelligent vehicles]]></term>
##       <term><![CDATA[Legged locomotion]]></term>
##       <term><![CDATA[Mobile robots]]></term>
##       <term><![CDATA[Motion control]]></term>
##       <term><![CDATA[Performance evaluation]]></term>
##       <term><![CDATA[Terrain mapping]]></term>
##       <term><![CDATA[Wheelchairs]]></term>
##     </thesaurusterms>
##     <pubtitle><![CDATA[Access, IEEE]]></pubtitle>
##     <punumber><![CDATA[6287639]]></punumber>
##     <pubtype><![CDATA[Journals & Magazines]]></pubtype>
##     <publisher><![CDATA[IEEE]]></publisher>
##     <volume><![CDATA[2]]></volume>
##     <py><![CDATA[2014]]></py>
##     <spage><![CDATA[26]]></spage>
##     <epage><![CDATA[39]]></epage>
##     <abstract><![CDATA[We have developed a personal mobility vehicle (PMV) with four driven wheels that is capable of negotiating obstacles with a leg motion mechanism. When obstacles are encountered, wheels are lifted, moved ahead in a stepping-like motion, and lowered back down, thereby allowing the PMV to advance further. In our previous paper, we discussed the principle of the gait algorithm used by our PMV, in which wheels are utilized as legs to negotiate obstacles. In the original algorithm, when the wheels encountered terrain that might require leg motion to traverse, the system determined whether such motion was applicable and, if it was, orchestrated a series of leg motions. However, there were terrains that could not be negotiated using the original algorithm. In this paper, we propose an improved gait algorithm, in which when the vehicle encounters terrain intractable by leg motion with its current posture, the vehicle changes its posture until it can traverse that terrain. We verified the effectiveness of the improved gait algorithm through a variety of mobility tests with a passenger. In addition, we present numerical data on the range of terrain topologies that could be negotiated by the proposed algorithm.]]></abstract>
##     <issn><![CDATA[2169-3536]]></issn>
##     <htmlFlag><![CDATA[1]]></htmlFlag>
##     <arnumber><![CDATA[6697815]]></arnumber>
##     <doi><![CDATA[10.1109/ACCESS.2013.2296557]]></doi>
##     <publicationId><![CDATA[6697815]]></publicationId>
##     <mdurl><![CDATA[http://ieeexplore.ieee.org/xpl/articleDetails.jsp?tp=&arnumber=6697815&contentType=Journals+%26+Magazines]]></mdurl>
##     <pdf><![CDATA[http://ieeexplore.ieee.org/stamp/stamp.jsp?arnumber=6697815]]></pdf>
##   </document>
##   <document>
##     <rank>1026</rank>
##     <title><![CDATA[On the Multiplicative Regularization of Graph Laplacians on Closed and Open Structures With Applications to Spectral Partitioning]]></title>
##     <authors><![CDATA[Mitharwal, R.;  Andriulli, F.P.]]></authors>
##     <affiliations><![CDATA[Microwave Dept., Telecom Bretagne/Inst. Mines-Telecom, Brest, France]]></affiliations>
##     <controlledterms>
##       <term><![CDATA[Laplace equations]]></term>
##       <term><![CDATA[computational electromagnetics]]></term>
##       <term><![CDATA[graph theory]]></term>
##       <term><![CDATA[mathematical operators]]></term>
##       <term><![CDATA[mesh generation]]></term>
##     </controlledterms>
##     <thesaurusterms>
##       <term><![CDATA[Geometry]]></term>
##       <term><![CDATA[Integral equations]]></term>
##       <term><![CDATA[Laplace equations]]></term>
##       <term><![CDATA[Manifolds]]></term>
##       <term><![CDATA[Octrees]]></term>
##       <term><![CDATA[Partitioning algorithms]]></term>
##       <term><![CDATA[Standards]]></term>
##     </thesaurusterms>
##     <pubtitle><![CDATA[Access, IEEE]]></pubtitle>
##     <punumber><![CDATA[6287639]]></punumber>
##     <pubtype><![CDATA[Journals & Magazines]]></pubtype>
##     <publisher><![CDATA[IEEE]]></publisher>
##     <volume><![CDATA[2]]></volume>
##     <py><![CDATA[2014]]></py>
##     <spage><![CDATA[788]]></spage>
##     <epage><![CDATA[796]]></epage>
##     <abstract><![CDATA[A new regularization technique for graph Laplacians arising from triangular meshes of closed and open structures is presented. The new technique is based on the analysis of graph Laplacian spectrally equivalent operators in terms of Sobolev norms and on the appropriate selection of operators of opposite differential strength to achieve a multiplicative regularization. In addition, a new 3-D/2-D nested regularization strategy is presented to deal with open geometries. Numerical results show the advantages of the proposed regularization as well as its effectiveness when used in spectral partitioning applications.]]></abstract>
##     <issn><![CDATA[2169-3536]]></issn>
##     <htmlFlag><![CDATA[1]]></htmlFlag>
##     <arnumber><![CDATA[6872516]]></arnumber>
##     <doi><![CDATA[10.1109/ACCESS.2014.2345657]]></doi>
##     <publicationId><![CDATA[6872516]]></publicationId>
##     <mdurl><![CDATA[http://ieeexplore.ieee.org/xpl/articleDetails.jsp?tp=&arnumber=6872516&contentType=Journals+%26+Magazines]]></mdurl>
##     <pdf><![CDATA[http://ieeexplore.ieee.org/stamp/stamp.jsp?arnumber=6872516]]></pdf>
##   </document>
##   <document>
##     <rank>1027</rank>
##     <title><![CDATA[Operation Characteristics of 12-Cavity Relativistic Magnetron With Single-Stepped Cavities]]></title>
##     <authors><![CDATA[Meiqin Liu;  Chunliang Liu;  Fuks, M.I.;  Schamiloglu, E.]]></authors>
##     <affiliations><![CDATA[Key Lab. of Phys. Electron. & Devices of the Minist. of Educ., Xi'an Jiaotong Univ., Xi'an, China]]></affiliations>
##     <controlledterms>
##       <term><![CDATA[magnetic fields]]></term>
##       <term><![CDATA[magnetrons]]></term>
##     </controlledterms>
##     <thesaurusterms>
##       <term><![CDATA[Cathodes]]></term>
##       <term><![CDATA[Cavity resonators]]></term>
##       <term><![CDATA[Educational institutions]]></term>
##       <term><![CDATA[Magnetic liquids]]></term>
##       <term><![CDATA[Magnetic switching]]></term>
##       <term><![CDATA[Power generation]]></term>
##       <term><![CDATA[Solids]]></term>
##     </thesaurusterms>
##     <pubtitle><![CDATA[Plasma Science, IEEE Transactions on]]></pubtitle>
##     <punumber><![CDATA[27]]></punumber>
##     <pubtype><![CDATA[Journals & Magazines]]></pubtype>
##     <publisher><![CDATA[IEEE]]></publisher>
##     <volume><![CDATA[42]]></volume>
##     <issue><![CDATA[10]]></issue>
##     <part><![CDATA[2]]></part>
##     <py><![CDATA[2014]]></py>
##     <spage><![CDATA[3283]]></spage>
##     <epage><![CDATA[3287]]></epage>
##     <abstract><![CDATA[The possibility of using single-stepped cavities to replace the common tapered cavities was studied using particle-in-cell simulations in an A6 magnetron with diffraction output (MDO). The replacing of the tapered cavities by the single-stepped cavities in a 12-cavity MDO increases the interaction space where the charged particles interact with the induced RF waves. The electronic efficiency of the 12-cavity MDO with single-stepped cavities driven by the transparent cathode [2] of GW output power level can be as high as 73% for &#x03B1; = 18.2&#x00B0;, 74% for &#x03B1; = 17.5&#x00B0;, and 72% for &#x03B1; = 12.5&#x00B0; at &#x03B2; = 32&#x00B0;, where &#x03B1; is the angle between the outer wall and z-axis, and &#x03B2; is the angle between the inner wall and z-axis. The depth of single-stepped cavities is changed when &#x03B1; is changed, which results in different frequency range of magnetron operating modes. When a 400-kV voltage pulse of 10-ns duration is applied to a transparent cathode or a solid cathode, the output power can be as high as 1 GW. Without loss of generality, for &#x03B1; = 12.5&#x00B0; at &#x03B2; = 32&#x00B0;, the peak efficiency around 70% of 12-cavity MDO with single-stepped cavities design occurs at the voltage (V ~ 400 &#x00B1; 50 kV). The results presented in this paper provide references for relativistic magnetron mode selection or mode switching experiments when choosing the input parameters (magnetic field and accelerating voltage) allowing the magnetron to operate in the desired operation mode.]]></abstract>
##     <issn><![CDATA[0093-3813]]></issn>
##     <htmlFlag><![CDATA[1]]></htmlFlag>
##     <arnumber><![CDATA[6778040]]></arnumber>
##     <doi><![CDATA[10.1109/TPS.2014.2311458]]></doi>
##     <publicationId><![CDATA[6778040]]></publicationId>
##     <mdurl><![CDATA[http://ieeexplore.ieee.org/xpl/articleDetails.jsp?tp=&arnumber=6778040&contentType=Journals+%26+Magazines]]></mdurl>
##     <pdf><![CDATA[http://ieeexplore.ieee.org/stamp/stamp.jsp?arnumber=6778040]]></pdf>
##   </document>
##   <document>
##     <rank>1028</rank>
##     <title><![CDATA[Measurement of Globally Visible DNS Injection]]></title>
##     <authors><![CDATA[Wander, M.;  Boelmann, C.;  Schwittmann, L.;  Weis, T.]]></authors>
##     <affiliations><![CDATA[Distrib. Syst. Group, Univ. of Duisburg-Essen, Duisburg, Germany]]></affiliations>
##     <controlledterms>
##       <term><![CDATA[Internet]]></term>
##       <term><![CDATA[security of data]]></term>
##     </controlledterms>
##     <thesaurusterms>
##       <term><![CDATA[Domain Name System]]></term>
##       <term><![CDATA[Filtering algorithms]]></term>
##       <term><![CDATA[IP networks]]></term>
##       <term><![CDATA[Inspection]]></term>
##       <term><![CDATA[Legal factors]]></term>
##       <term><![CDATA[Servers]]></term>
##     </thesaurusterms>
##     <pubtitle><![CDATA[Access, IEEE]]></pubtitle>
##     <punumber><![CDATA[6287639]]></punumber>
##     <pubtype><![CDATA[Journals & Magazines]]></pubtype>
##     <publisher><![CDATA[IEEE]]></publisher>
##     <volume><![CDATA[2]]></volume>
##     <py><![CDATA[2014]]></py>
##     <spage><![CDATA[526]]></spage>
##     <epage><![CDATA[536]]></epage>
##     <abstract><![CDATA[Domain Name System (DNS) injection is a censorship method for blocking access to blacklisted domain names. The method uses deep packet inspection on all DNS queries passing through the network and injects spoofed responses. Compared with other blocking mechanisms, DNS injection impacts uninvolved third-parties if their traffic is routed through a censored network. In this paper, we look for large deployments of DNS injection, measured from vantage points outside of the censored networks. DNS injection is known to be used in China since it leaked unintentionally into foreign networks. We find that DNS injection is also used in Iran and can be observed by sending DNS queries to Iranian networks. In mid 2013, the Iranian DNS filter was temporarily suspended for some names, which correlated with media coverage of political debates in Iran about blocking social media. Spoofed responses from China and Iran can be detected passively by the IP address returned. We propose an algorithm to obtain these addresses remotely. After testing 255002 open resolvers outside of China, we determined that 6% are potentially affected by Chinese DNS injection when querying top-level domains outside of China. This is essentially the result of one top-level domain name server for which an anycast instance is hosted in China.]]></abstract>
##     <issn><![CDATA[2169-3536]]></issn>
##     <htmlFlag><![CDATA[1]]></htmlFlag>
##     <arnumber><![CDATA[6814824]]></arnumber>
##     <doi><![CDATA[10.1109/ACCESS.2014.2323299]]></doi>
##     <publicationId><![CDATA[6814824]]></publicationId>
##     <mdurl><![CDATA[http://ieeexplore.ieee.org/xpl/articleDetails.jsp?tp=&arnumber=6814824&contentType=Journals+%26+Magazines]]></mdurl>
##     <pdf><![CDATA[http://ieeexplore.ieee.org/stamp/stamp.jsp?arnumber=6814824]]></pdf>
##   </document>
##   <document>
##     <rank>1029</rank>
##     <title><![CDATA[Outage Performance Analysis of Imperfect-CSI-Based Selection Cooperation in Random Networks]]></title>
##     <authors><![CDATA[Tukmanov, A.;  Boussakta, S.;  Zhiguo Ding;  Jamalipour, A.]]></authors>
##     <affiliations><![CDATA[BT Res. & Innovation, Ipswich, UK]]></affiliations>
##     <controlledterms>
##       <term><![CDATA[cooperative communication]]></term>
##       <term><![CDATA[decode and forward communication]]></term>
##       <term><![CDATA[diversity reception]]></term>
##       <term><![CDATA[random processes]]></term>
##       <term><![CDATA[relay networks (telecommunication)]]></term>
##       <term><![CDATA[statistical distributions]]></term>
##       <term><![CDATA[stochastic processes]]></term>
##       <term><![CDATA[wireless channels]]></term>
##     </controlledterms>
##     <thesaurusterms>
##       <term><![CDATA[Channel estimation]]></term>
##       <term><![CDATA[Decoding]]></term>
##       <term><![CDATA[Educational institutions]]></term>
##       <term><![CDATA[Loss measurement]]></term>
##       <term><![CDATA[Q measurement]]></term>
##       <term><![CDATA[Relays]]></term>
##       <term><![CDATA[Reliability]]></term>
##     </thesaurusterms>
##     <pubtitle><![CDATA[Communications, IEEE Transactions on]]></pubtitle>
##     <punumber><![CDATA[26]]></punumber>
##     <pubtype><![CDATA[Journals & Magazines]]></pubtype>
##     <publisher><![CDATA[IEEE]]></publisher>
##     <volume><![CDATA[62]]></volume>
##     <issue><![CDATA[8]]></issue>
##     <py><![CDATA[2014]]></py>
##     <spage><![CDATA[2747]]></spage>
##     <epage><![CDATA[2757]]></epage>
##     <abstract><![CDATA[Selection of relays is central to efficient utilization of cooperative diversity gains when multiple relays are available in the network. Such selection is generally based on some form of channel state information (CSI), which is always imperfect in practice. The effects of using imperfect CSI in a relay selection process have been generally considered in existing literature without the account for spatial distribution of relays, while current works on relay selection in random networks mainly assume perfect CSI. In this paper, we analyze the outage performance of a single source-destination pair communicating through a decode-and-forward relay, chosen from a Poisson point process (PPP) of candidate relays using perfect and imperfect CSI. We derive exact outage probability expressions for the selection cooperation strategy. Closed-form expressions are provided for special cases, and asymptotic analysis is conducted to highlight the highSNR system behavior.]]></abstract>
##     <issn><![CDATA[0090-6778]]></issn>
##     <htmlFlag><![CDATA[1]]></htmlFlag>
##     <arnumber><![CDATA[6810771]]></arnumber>
##     <doi><![CDATA[10.1109/TCOMM.2014.2322093]]></doi>
##     <publicationId><![CDATA[6810771]]></publicationId>
##     <mdurl><![CDATA[http://ieeexplore.ieee.org/xpl/articleDetails.jsp?tp=&arnumber=6810771&contentType=Journals+%26+Magazines]]></mdurl>
##     <pdf><![CDATA[http://ieeexplore.ieee.org/stamp/stamp.jsp?arnumber=6810771]]></pdf>
##   </document>
##   <document>
##     <rank>1030</rank>
##     <title><![CDATA[A Unified Formulation of Gaussian Versus Sparse Stochastic Processes&#x2014;Part I: Continuous-Domain Theory]]></title>
##     <authors><![CDATA[Unser, M.;  Tafti, P.D.;  Qiyu Sun]]></authors>
##     <affiliations><![CDATA[Biomed. Imaging Group, Ecole Polytech. Fed. de Lausanne, Lausanne, Switzerland]]></affiliations>
##     <controlledterms>
##       <term><![CDATA[Gaussian processes]]></term>
##       <term><![CDATA[signal processing]]></term>
##       <term><![CDATA[stochastic processes]]></term>
##     </controlledterms>
##     <thesaurusterms>
##       <term><![CDATA[Discrete cosine transforms]]></term>
##       <term><![CDATA[Mathematical model]]></term>
##       <term><![CDATA[Stochastic processes]]></term>
##       <term><![CDATA[Technological innovation]]></term>
##       <term><![CDATA[Wavelet transforms]]></term>
##       <term><![CDATA[White noise]]></term>
##     </thesaurusterms>
##     <pubtitle><![CDATA[Information Theory, IEEE Transactions on]]></pubtitle>
##     <punumber><![CDATA[18]]></punumber>
##     <pubtype><![CDATA[Journals & Magazines]]></pubtype>
##     <publisher><![CDATA[IEEE]]></publisher>
##     <volume><![CDATA[60]]></volume>
##     <issue><![CDATA[3]]></issue>
##     <py><![CDATA[2014]]></py>
##     <spage><![CDATA[1945]]></spage>
##     <epage><![CDATA[1962]]></epage>
##     <abstract><![CDATA[We introduce a general distributional framework that results in a unifying description and characterization of a rich variety of continuous-time stochastic processes. The cornerstone of our approach is an innovation model that is driven by some generalized white noise process, which may be Gaussian or not (e.g., Laplace, impulsive Poisson, or alpha stable). This allows for a conceptual decoupling between the correlation properties of the process, which are imposed by the whitening operator L, and its sparsity pattern, which is determined by the type of noise excitation. The latter is fully specified by a Le&#x0301;vy measure. We show that the range of admissible innovation behavior varies between the purely Gaussian and super-sparse extremes. We prove that the corresponding generalized stochastic processes are well-defined mathematically provided that the (adjoint) inverse of the whitening operator satisfies some Lp bound for p &#x2265; 1. We present a novel operator-based method that yields an explicit characterization of all Le&#x0301;vy-driven processes that are solutions of constant-coefficient stochastic differential equations. When the underlying system is stable, we recover the family of stationary continuous-time autoregressive moving average processes (CARMA), including the Gaussian ones. The approach remains valid when the system is unstable and leads to the identification of potentially useful generalizations of the Le&#x0301;vy processes, which are sparse and non-stationary. Finally, we show that these processes admit a sparse representation in some matched wavelet domain and provide a full characterization of their transform-domain statistics.]]></abstract>
##     <issn><![CDATA[0018-9448]]></issn>
##     <htmlFlag><![CDATA[1]]></htmlFlag>
##     <arnumber><![CDATA[6704775]]></arnumber>
##     <doi><![CDATA[10.1109/TIT.2014.2298453]]></doi>
##     <publicationId><![CDATA[6704775]]></publicationId>
##     <mdurl><![CDATA[http://ieeexplore.ieee.org/xpl/articleDetails.jsp?tp=&arnumber=6704775&contentType=Journals+%26+Magazines]]></mdurl>
##     <pdf><![CDATA[http://ieeexplore.ieee.org/stamp/stamp.jsp?arnumber=6704775]]></pdf>
##   </document>
##   <document>
##     <rank>1031</rank>
##     <title><![CDATA[Energy-Minimizing Error-Correcting Codes]]></title>
##     <authors><![CDATA[Cohn, H.;  Yufei Zhao]]></authors>
##     <affiliations><![CDATA[Microsoft Res. New England, Cambridge, MA, USA]]></affiliations>
##     <controlledterms>
##       <term><![CDATA[Golay codes]]></term>
##       <term><![CDATA[Hamming codes]]></term>
##       <term><![CDATA[Reed-Solomon codes]]></term>
##       <term><![CDATA[error correction codes]]></term>
##       <term><![CDATA[linear programming]]></term>
##       <term><![CDATA[telecommunication power management]]></term>
##     </controlledterms>
##     <thesaurusterms>
##       <term><![CDATA[Linear programming]]></term>
##       <term><![CDATA[Maximum likelihood decoding]]></term>
##       <term><![CDATA[Minimization]]></term>
##       <term><![CDATA[Physics]]></term>
##       <term><![CDATA[Polynomials]]></term>
##       <term><![CDATA[Potential energy]]></term>
##       <term><![CDATA[Vectors]]></term>
##     </thesaurusterms>
##     <pubtitle><![CDATA[Information Theory, IEEE Transactions on]]></pubtitle>
##     <punumber><![CDATA[18]]></punumber>
##     <pubtype><![CDATA[Journals & Magazines]]></pubtype>
##     <publisher><![CDATA[IEEE]]></publisher>
##     <volume><![CDATA[60]]></volume>
##     <issue><![CDATA[12]]></issue>
##     <py><![CDATA[2014]]></py>
##     <spage><![CDATA[7442]]></spage>
##     <epage><![CDATA[7450]]></epage>
##     <abstract><![CDATA[We study a discrete model of repelling particles, and we show using linear programming bounds that many familiar families of error-correcting codes minimize a broad class of potential energies when compared with all other codes of the same size and block length. Examples of these universally optimal codes include Hamming, Golay, and Reed-Solomon codes, among many others, and this helps to explain their robustness as the channel model varies. Universal optimality of these codes is equivalent to minimality of their binomial moments, which has been proved in many cases by Ashikhmin and Barg. We highlight connections with mathematical physics and the analogy between these results and previous work by Cohn and Kumar in the continuous setting, and we develop a framework for optimizing the linear programming bounds. Furthermore, we show that if these bounds prove a code is universally optimal, then the code remains universally optimal even if one codeword is removed.]]></abstract>
##     <issn><![CDATA[0018-9448]]></issn>
##     <htmlFlag><![CDATA[1]]></htmlFlag>
##     <arnumber><![CDATA[6914586]]></arnumber>
##     <doi><![CDATA[10.1109/TIT.2014.2359201]]></doi>
##     <publicationId><![CDATA[6914586]]></publicationId>
##     <mdurl><![CDATA[http://ieeexplore.ieee.org/xpl/articleDetails.jsp?tp=&arnumber=6914586&contentType=Journals+%26+Magazines]]></mdurl>
##     <pdf><![CDATA[http://ieeexplore.ieee.org/stamp/stamp.jsp?arnumber=6914586]]></pdf>
##   </document>
##   <document>
##     <rank>1032</rank>
##     <title><![CDATA[Novel Raman Fiber Lasers Emitting in the U-Band With Combined Volume Bragg Gratings]]></title>
##     <authors><![CDATA[Weichao Yao;  Jun Liu;  Nanyi Chen;  Jianing Zhang;  Yongguang Zhao;  Deyuan Shen]]></authors>
##     <affiliations><![CDATA[Key Lab. of Adv. Laser Mater. & Devices, Jiangsu Normal Univ., Xuzhou, China]]></affiliations>
##     <controlledterms>
##       <term><![CDATA[Bragg gratings]]></term>
##       <term><![CDATA[Raman lasers]]></term>
##       <term><![CDATA[fibre lasers]]></term>
##       <term><![CDATA[laser tuning]]></term>
##       <term><![CDATA[spectral line narrowing]]></term>
##     </controlledterms>
##     <thesaurusterms>
##       <term><![CDATA[Bragg gratings]]></term>
##       <term><![CDATA[Fiber lasers]]></term>
##       <term><![CDATA[Optical fiber amplifiers]]></term>
##       <term><![CDATA[Optical fiber communication]]></term>
##       <term><![CDATA[Optical fiber sensors]]></term>
##       <term><![CDATA[Pump lasers]]></term>
##     </thesaurusterms>
##     <pubtitle><![CDATA[Photonics Journal, IEEE]]></pubtitle>
##     <punumber><![CDATA[4563994]]></punumber>
##     <pubtype><![CDATA[Journals & Magazines]]></pubtype>
##     <publisher><![CDATA[IEEE]]></publisher>
##     <volume><![CDATA[6]]></volume>
##     <issue><![CDATA[6]]></issue>
##     <py><![CDATA[2014]]></py>
##     <spage><![CDATA[1]]></spage>
##     <epage><![CDATA[8]]></epage>
##     <abstract><![CDATA[Narrow linewidth and widely tunable high-power Raman fiber lasers emitting in the U-band wavelength were demonstrated based on combinations of volume Bragg gratings with different center wavelengths. The narrowed linewidth was 65 pm at a center wavelength of 1657.86 nm with a maximum output power of 8.3 W, corresponding to a slope efficiency of 55.6%. For the tunable Raman fiber laser, continuously tunable dual-wavelength emission was obtained in a wide band (1621.3-1683.5 nm) with a wavelength difference over 60 nm (0.7-62.2 nm). The maximum output power was 11 W, corresponding to a slope efficiency of 76.6%.]]></abstract>
##     <issn><![CDATA[1943-0655]]></issn>
##     <htmlFlag><![CDATA[1]]></htmlFlag>
##     <arnumber><![CDATA[6967742]]></arnumber>
##     <doi><![CDATA[10.1109/JPHOT.2014.2374609]]></doi>
##     <publicationId><![CDATA[6967742]]></publicationId>
##     <mdurl><![CDATA[http://ieeexplore.ieee.org/xpl/articleDetails.jsp?tp=&arnumber=6967742&contentType=Journals+%26+Magazines]]></mdurl>
##     <pdf><![CDATA[http://ieeexplore.ieee.org/stamp/stamp.jsp?arnumber=6967742]]></pdf>
##   </document>
##   <document>
##     <rank>1033</rank>
##     <title><![CDATA[InGaN-LD-Pumped <inline-formula> <img src="/images/tex/19371.gif" alt="{\rm \Pr}^{3+}"> </inline-formula>: <inline-formula> <img src="/images/tex/19382.gif" alt="{\rm LiYF}_{4}"> </inline-formula> Continuous-Wave Laser at 915 nm]]></title>
##     <authors><![CDATA[Biao Qu;  Bin Xu;  Yongjie Cheng;  Saiyu Luo;  Huiying Xu;  Yikun Bu;  Camy, P.;  Doualan, J.-L.;  Moncorge, R.;  Zhiping Cai]]></authors>
##     <affiliations><![CDATA[Dept. of Electron. Eng., Xiamen Univ., Xiamen, China]]></affiliations>
##     <controlledterms>
##       <term><![CDATA[III-V semiconductors]]></term>
##       <term><![CDATA[gallium compounds]]></term>
##       <term><![CDATA[indium compounds]]></term>
##       <term><![CDATA[lithium compounds]]></term>
##       <term><![CDATA[optical losses]]></term>
##       <term><![CDATA[optical pumping]]></term>
##       <term><![CDATA[semiconductor lasers]]></term>
##       <term><![CDATA[solid lasers]]></term>
##       <term><![CDATA[wide band gap semiconductors]]></term>
##       <term><![CDATA[yttrium compounds]]></term>
##     </controlledterms>
##     <thesaurusterms>
##       <term><![CDATA[Diode lasers]]></term>
##       <term><![CDATA[Laser excitation]]></term>
##       <term><![CDATA[Laser transitions]]></term>
##       <term><![CDATA[Pump lasers]]></term>
##       <term><![CDATA[Semiconductor lasers]]></term>
##       <term><![CDATA[Solid lasers]]></term>
##     </thesaurusterms>
##     <pubtitle><![CDATA[Photonics Journal, IEEE]]></pubtitle>
##     <punumber><![CDATA[4563994]]></punumber>
##     <pubtype><![CDATA[Journals & Magazines]]></pubtype>
##     <publisher><![CDATA[IEEE]]></publisher>
##     <volume><![CDATA[6]]></volume>
##     <issue><![CDATA[6]]></issue>
##     <py><![CDATA[2014]]></py>
##     <spage><![CDATA[1]]></spage>
##     <epage><![CDATA[11]]></epage>
##     <abstract><![CDATA[We demonstrate the first InGaN-LD-pumped room temperature and continuous-wave laser operation of a Pr<sup>3+</sup>: LiYF<sub>4</sub> crystal at 915 nm. A maximum output power up to 78 mW with a laser slope efficiency of about 17% is obtained. The round-trip optical losses are estimated to be about 0.45%, and the M<sup>2</sup> beam quality factors measured in x and y dimensions are about 1.07 and 1.04, respectively.]]></abstract>
##     <issn><![CDATA[1943-0655]]></issn>
##     <htmlFlag><![CDATA[1]]></htmlFlag>
##     <arnumber><![CDATA[6977881]]></arnumber>
##     <doi><![CDATA[10.1109/JPHOT.2014.2374618]]></doi>
##     <publicationId><![CDATA[6977881]]></publicationId>
##     <mdurl><![CDATA[http://ieeexplore.ieee.org/xpl/articleDetails.jsp?tp=&arnumber=6977881&contentType=Journals+%26+Magazines]]></mdurl>
##     <pdf><![CDATA[http://ieeexplore.ieee.org/stamp/stamp.jsp?arnumber=6977881]]></pdf>
##   </document>
##   <document>
##     <rank>1034</rank>
##     <title><![CDATA[WDM&#x2013;TDM NG-PON Power Budget Extension by Utilizing SOA in the Remote Node]]></title>
##     <authors><![CDATA[Emsia, A.;  Le, Q.T.;  Malekizandi, M.;  Briggmann, D.;  Djordjevic, I.B.;  Kuppers, F.]]></authors>
##     <affiliations><![CDATA[Inst. of Microwave Eng. & Photonics, Tech. Univ. Darmstadt, Darmstadt, Germany]]></affiliations>
##     <controlledterms>
##       <term><![CDATA[chirp modulation]]></term>
##       <term><![CDATA[differential phase shift keying]]></term>
##       <term><![CDATA[next generation networks]]></term>
##       <term><![CDATA[optical transmitters]]></term>
##       <term><![CDATA[passive optical networks]]></term>
##       <term><![CDATA[quadrature phase shift keying]]></term>
##       <term><![CDATA[semiconductor optical amplifiers]]></term>
##       <term><![CDATA[subscriber loops]]></term>
##       <term><![CDATA[time division multiplexing]]></term>
##       <term><![CDATA[wavelength division multiplexing]]></term>
##     </controlledterms>
##     <thesaurusterms>
##       <term><![CDATA[Optical network units]]></term>
##       <term><![CDATA[Optical receivers]]></term>
##       <term><![CDATA[Optical transmitters]]></term>
##       <term><![CDATA[Passive optical networks]]></term>
##       <term><![CDATA[Semiconductor optical amplifiers]]></term>
##       <term><![CDATA[Time division multiplexing]]></term>
##       <term><![CDATA[Wavelength division multiplexing]]></term>
##     </thesaurusterms>
##     <pubtitle><![CDATA[Photonics Journal, IEEE]]></pubtitle>
##     <punumber><![CDATA[4563994]]></punumber>
##     <pubtype><![CDATA[Journals & Magazines]]></pubtype>
##     <publisher><![CDATA[IEEE]]></publisher>
##     <volume><![CDATA[6]]></volume>
##     <issue><![CDATA[2]]></issue>
##     <py><![CDATA[2014]]></py>
##     <spage><![CDATA[1]]></spage>
##     <epage><![CDATA[10]]></epage>
##     <abstract><![CDATA[Today, even in access networks, data traffic is enormously increasing, a trend that causes existing passive optical network (PON) infrastructures to become bottlenecks in a tele- and data-communication infrastructure, which is aimed to be both broadband and seamless. Thus, two major objectives are considered for next-generation PONs (NG-PONs), i.e., first, bandwidth increase, and second, reach extension to reduce deployment costs. Here, we describe a new reach extension scheme that at the same time allows increasing the number of subscribers in the network. The amplification technique is based on a bidirectional semiconductor optical amplifier (SOA). It is shown that the extender configuration not only meets the bandwidth and budget requirements for NG-PONs but also remarkably improves them. Differential (quadrature) phase-shift keying (D(Q)PSK) signals are investigated in this paper. Hybrid wavelength-division multiplexing/time-division multiplexing (WDM/TDM) transmission up to 120-Gb/s downstream and 40-Gb/s upstream are experimentally demonstrated. The access budget of 33.4 dB is achieved at 10 Gb/s on every WDM channel in case of DPSK enabling a splitting ratio of 1:512 per wavelength. Furthermore, optical power budget of 40 dB is obtained when DQPSK is used, where a splitting ratio of 1024 per wavelength can be supported. Additionally, a cost-efficient chirped managed directly modulated laser scheme is proposed for DPSK signal generation in U.S. scenario enabling a high-power budget performance low-cost transmitter configuration, which appears suitable for NG-PON application. The proposed technique also alleviates nonlinear impairments [specifically, cross-phase modulation (XPM)], which appear in dense WDM transmission.]]></abstract>
##     <issn><![CDATA[1943-0655]]></issn>
##     <htmlFlag><![CDATA[1]]></htmlFlag>
##     <arnumber><![CDATA[6779654]]></arnumber>
##     <doi><![CDATA[10.1109/JPHOT.2014.2314108]]></doi>
##     <publicationId><![CDATA[6779654]]></publicationId>
##     <mdurl><![CDATA[http://ieeexplore.ieee.org/xpl/articleDetails.jsp?tp=&arnumber=6779654&contentType=Journals+%26+Magazines]]></mdurl>
##     <pdf><![CDATA[http://ieeexplore.ieee.org/stamp/stamp.jsp?arnumber=6779654]]></pdf>
##   </document>
##   <document>
##     <rank>1035</rank>
##     <title><![CDATA[The Effect of Light Conditions on Photoplethysmographic Image Acquisition Using a Commercial Camera]]></title>
##     <authors><![CDATA[He Liu;  Yadong Wang;  Lei Wang]]></authors>
##     <affiliations><![CDATA[Shenzhen Inst. of Adv. Technol., Shenzhen, China]]></affiliations>
##     <controlledterms>
##       <term><![CDATA[biomedical optical imaging]]></term>
##       <term><![CDATA[cameras]]></term>
##       <term><![CDATA[light sources]]></term>
##       <term><![CDATA[photoplethysmography]]></term>
##     </controlledterms>
##     <thesaurusterms>
##       <term><![CDATA[Cameras]]></term>
##       <term><![CDATA[Field programmable gate arrays]]></term>
##       <term><![CDATA[Lenses]]></term>
##       <term><![CDATA[Light emitting diodes]]></term>
##       <term><![CDATA[Light sources]]></term>
##       <term><![CDATA[Standards]]></term>
##       <term><![CDATA[Videos]]></term>
##     </thesaurusterms>
##     <pubtitle><![CDATA[Translational Engineering in Health and Medicine, IEEE Journal of]]></pubtitle>
##     <punumber><![CDATA[6221039]]></punumber>
##     <pubtype><![CDATA[Journals & Magazines]]></pubtype>
##     <publisher><![CDATA[IEEE]]></publisher>
##     <volume><![CDATA[2]]></volume>
##     <py><![CDATA[2014]]></py>
##     <spage><![CDATA[1]]></spage>
##     <epage><![CDATA[11]]></epage>
##     <abstract><![CDATA[Cameras embedded in consumer devices have previously been used as physiological information sensors. The waveform of the photoplethysmographic image (PPGi) signals may be significantly affected by the light spectra and intensity. The purpose of this paper is to evaluate the performance of PPGi waveform acquisition in the red, green, and blue channels using a commercial camera in different light conditions. The system, developed for this paper, comprises of a commercial camera and light sources with varied spectra and intensities. Signals were acquired from the fingertips of 12 healthy subjects. Extensive experiments, using different wavelength lights and white light with variation light intensities, respectively, reported in this paper, showed that almost all light spectra can acquire acceptable pulse rates, but only 470-, 490-, 505-, 590-, 600-, 610-, 625-, and 660-nm wavelength lights showed better performance in PPGi waveform compared with gold standard. With lower light intensity, the light spectra &gt;600 nm still showed better performance. The change in pulse amplitude (ac) and dc amplitude was also investigated with the different light intensity and light spectra. With increasing light intensity, the dc amplitude increased, whereas ac component showed an initial increase followed by a decrease. Most of the subjects achieved their maximum averaging ac output when averaging dc output was in the range from 180 to 220 pixel values (8 b, 255 maximum pixel value). The results suggested that an adaptive solution could be developed to optimize the design of PPGi-based physiological signal acquisition devices in different light conditions.]]></abstract>
##     <issn><![CDATA[2168-2372]]></issn>
##     <htmlFlag><![CDATA[1]]></htmlFlag>
##     <arnumber><![CDATA[6917212]]></arnumber>
##     <doi><![CDATA[10.1109/JTEHM.2014.2360200]]></doi>
##     <publicationId><![CDATA[6917212]]></publicationId>
##     <mdurl><![CDATA[http://ieeexplore.ieee.org/xpl/articleDetails.jsp?tp=&arnumber=6917212&contentType=Journals+%26+Magazines]]></mdurl>
##     <pdf><![CDATA[http://ieeexplore.ieee.org/stamp/stamp.jsp?arnumber=6917212]]></pdf>
##   </document>
##   <document>
##     <rank>1036</rank>
##     <title><![CDATA[A Reinforcement Learning-Based ToD Provisioning Dynamic Power Management for Sustainable Operation of Energy Harvesting Wireless Sensor Node]]></title>
##     <authors><![CDATA[Hsu, R.C.;  Cheng-Ting Liu;  Hao-Li Wang]]></authors>
##     <affiliations><![CDATA[Dept. of Electr. Eng., Nat. Chiayi Univ., Chiayi, Taiwan]]></affiliations>
##     <controlledterms>
##       <term><![CDATA[energy harvesting]]></term>
##       <term><![CDATA[learning (artificial intelligence)]]></term>
##       <term><![CDATA[solar cells]]></term>
##       <term><![CDATA[wireless sensor networks]]></term>
##     </controlledterms>
##     <thesaurusterms>
##       <term><![CDATA[Embedded systems]]></term>
##       <term><![CDATA[Energy harvesting]]></term>
##       <term><![CDATA[Energy storage]]></term>
##       <term><![CDATA[Power system management]]></term>
##       <term><![CDATA[Throughput]]></term>
##       <term><![CDATA[Wireless communication]]></term>
##       <term><![CDATA[Wireless sensor networks]]></term>
##     </thesaurusterms>
##     <pubtitle><![CDATA[Emerging Topics in Computing, IEEE Transactions on]]></pubtitle>
##     <punumber><![CDATA[6245516]]></punumber>
##     <pubtype><![CDATA[Journals & Magazines]]></pubtype>
##     <publisher><![CDATA[IEEE]]></publisher>
##     <volume><![CDATA[2]]></volume>
##     <issue><![CDATA[2]]></issue>
##     <py><![CDATA[2014]]></py>
##     <spage><![CDATA[181]]></spage>
##     <epage><![CDATA[191]]></epage>
##     <abstract><![CDATA[In this paper, a reinforcement learning-based throughput on demand (ToD) provisioning dynamic power management method (RLTDPM) is proposed for sustaining perpetual operation and satisfying the ToD requirements for today's energy harvesting wireless sensor node (EHWSN). The RLTDPM monitors the environmental state of the EHWS and adjusts their operational duty cycle under criteria of energy neutrality to meet the demanded throughput. Outcomes of these observation-adjustment interactions are then evaluated by feedback/reward that represents how well the ToD requests are met; subsequently, the observation-adjustment-evaluation process, so-called reinforcement learning, continues. After the learning process, the RLTDPM is able to autonomously adjust the duty cycle for satisfying the ToD requirement, and in doing so, sustain the perpetual operation of the EHWSN. Simulations of the proposed RLTDPM on a wireless sensor node powered by a battery and solar cell for image sensing tasks were performed. Experimental results demonstrate that the achieved demanded throughput is improved 10.7% for the most stringent ToD requirement, while the residual battery energy of the RLTDPM is improved 7.4% compared with an existing DPM algorithm for EHWSN with image sensing purpose.]]></abstract>
##     <issn><![CDATA[2168-6750]]></issn>
##     <htmlFlag><![CDATA[1]]></htmlFlag>
##     <arnumber><![CDATA[6797906]]></arnumber>
##     <doi><![CDATA[10.1109/TETC.2014.2316518]]></doi>
##     <publicationId><![CDATA[6797906]]></publicationId>
##     <mdurl><![CDATA[http://ieeexplore.ieee.org/xpl/articleDetails.jsp?tp=&arnumber=6797906&contentType=Journals+%26+Magazines]]></mdurl>
##     <pdf><![CDATA[http://ieeexplore.ieee.org/stamp/stamp.jsp?arnumber=6797906]]></pdf>
##   </document>
##   <document>
##     <rank>1037</rank>
##     <title><![CDATA[Radiometric Correction of Terrestrial LiDAR Point Cloud Data for Individual Maize Plant Detection]]></title>
##     <authors><![CDATA[Hofle, B.]]></authors>
##     <affiliations><![CDATA[Dept. of GIScience, Univ. of Heidelberg, Heidelberg, Germany]]></affiliations>
##     <controlledterms>
##       <term><![CDATA[agriculture]]></term>
##       <term><![CDATA[geophysical image processing]]></term>
##       <term><![CDATA[object detection]]></term>
##       <term><![CDATA[optical radar]]></term>
##       <term><![CDATA[radiometry]]></term>
##       <term><![CDATA[remote sensing by laser beam]]></term>
##       <term><![CDATA[vegetation mapping]]></term>
##     </controlledterms>
##     <thesaurusterms>
##       <term><![CDATA[Agriculture]]></term>
##       <term><![CDATA[Laser radar]]></term>
##       <term><![CDATA[Lasers]]></term>
##       <term><![CDATA[Radiometry]]></term>
##       <term><![CDATA[Remote sensing]]></term>
##       <term><![CDATA[Sensors]]></term>
##       <term><![CDATA[Soil]]></term>
##     </thesaurusterms>
##     <pubtitle><![CDATA[Geoscience and Remote Sensing Letters, IEEE]]></pubtitle>
##     <punumber><![CDATA[8859]]></punumber>
##     <pubtype><![CDATA[Journals & Magazines]]></pubtype>
##     <publisher><![CDATA[IEEE]]></publisher>
##     <volume><![CDATA[11]]></volume>
##     <issue><![CDATA[1]]></issue>
##     <py><![CDATA[2014]]></py>
##     <spage><![CDATA[94]]></spage>
##     <epage><![CDATA[98]]></epage>
##     <abstract><![CDATA[Detailed geoinformation on in-field variations of plant properties (e.g., density, height) is required in precision agriculture and serves as a valuable input for plant growth models and crop management strategies. This letter presents a novel workflow for object-based point cloud analysis for individual maize plant mapping, using radiometric and geometric features of terrestrial laser scanning. The performed radiometric correction achieves a reduction of amplitude variation of homogeneous areas to 1/3 of the original variation and offers a distinct separability of the target class maize plant from soil. The developed procedure, including 3-D point cloud filtering and segmentation, is able to reliably detect single plants with a completeness and correctness . Experiments on reduced point densities show stability of detection rates above 100 points per 0.01 m<sup>2</sup>. The results indicate that the developed workflow will lead to even higher detection accuracy with LiDAR point clouds captured by mobile platforms, with less occlusion effects and more homogeneous point density.]]></abstract>
##     <issn><![CDATA[1545-598X]]></issn>
##     <htmlFlag><![CDATA[1]]></htmlFlag>
##     <arnumber><![CDATA[6482592]]></arnumber>
##     <doi><![CDATA[10.1109/LGRS.2013.2247022]]></doi>
##     <publicationId><![CDATA[6482592]]></publicationId>
##     <mdurl><![CDATA[http://ieeexplore.ieee.org/xpl/articleDetails.jsp?tp=&arnumber=6482592&contentType=Journals+%26+Magazines]]></mdurl>
##     <pdf><![CDATA[http://ieeexplore.ieee.org/stamp/stamp.jsp?arnumber=6482592]]></pdf>
##   </document>
##   <document>
##     <rank>1038</rank>
##     <title><![CDATA[Double-Frequency Filter Based on Coupling of Cavity Modes and Surface Plasmon Polaritons]]></title>
##     <authors><![CDATA[Yun-tuan Fang;  Jian-xia Hu;  Ji-jun Wang]]></authors>
##     <affiliations><![CDATA[Sch. of Comput. Sci. & Telecommun. Eng., Jiangsu Univ., Zhenjiang, China]]></affiliations>
##     <controlledterms>
##       <term><![CDATA[optical filters]]></term>
##       <term><![CDATA[optical resonators]]></term>
##       <term><![CDATA[polaritons]]></term>
##       <term><![CDATA[reflectivity]]></term>
##       <term><![CDATA[resonant tunnelling]]></term>
##       <term><![CDATA[surface plasmons]]></term>
##       <term><![CDATA[transfer function matrices]]></term>
##     </controlledterms>
##     <thesaurusterms>
##       <term><![CDATA[Cavity resonators]]></term>
##       <term><![CDATA[Couplings]]></term>
##       <term><![CDATA[Plasmons]]></term>
##       <term><![CDATA[Resonant frequency]]></term>
##       <term><![CDATA[Silver]]></term>
##       <term><![CDATA[Tunneling]]></term>
##     </thesaurusterms>
##     <pubtitle><![CDATA[Photonics Journal, IEEE]]></pubtitle>
##     <punumber><![CDATA[4563994]]></punumber>
##     <pubtype><![CDATA[Journals & Magazines]]></pubtype>
##     <publisher><![CDATA[IEEE]]></publisher>
##     <volume><![CDATA[6]]></volume>
##     <issue><![CDATA[2]]></issue>
##     <py><![CDATA[2014]]></py>
##     <spage><![CDATA[1]]></spage>
##     <epage><![CDATA[7]]></epage>
##     <abstract><![CDATA[In order to achieve a double-frequency filter, we design a sandwiched structure with a dual-prism total reflection configuration. The sandwiched structure consists of one metal layer and two same media layers. The transmission properties are studied through the standard transfer matrix method. Such system can form two hybrid cavity-surface plasmon-polariton modes. The coupling of the two hybrid modes results in resonance tunneling effect and mode splitting. Mode splitting leads to the coupled double peaks. The difference of double-peak frequencies can be adjusted through changing the metal thickness, and the position of double peaks can be also tunable through changing incidence direction. Such system can be used in designing double-frequency filters and terahertz emission devices.]]></abstract>
##     <issn><![CDATA[1943-0655]]></issn>
##     <htmlFlag><![CDATA[1]]></htmlFlag>
##     <arnumber><![CDATA[6754200]]></arnumber>
##     <doi><![CDATA[10.1109/JPHOT.2014.2306824]]></doi>
##     <publicationId><![CDATA[6754200]]></publicationId>
##     <mdurl><![CDATA[http://ieeexplore.ieee.org/xpl/articleDetails.jsp?tp=&arnumber=6754200&contentType=Journals+%26+Magazines]]></mdurl>
##     <pdf><![CDATA[http://ieeexplore.ieee.org/stamp/stamp.jsp?arnumber=6754200]]></pdf>
##   </document>
##   <document>
##     <rank>1039</rank>
##     <title><![CDATA[Image-Based Mechanical Analysis of Stent Deformation: Concept and Exemplary Implementation for Aortic Valve Stents]]></title>
##     <authors><![CDATA[Gessat, M.;  Hopf, R.;  Pollok, T.;  Russ, C.;  Frauenfelder, T.;  Sundermann, S.H.;  Hirsch, S.;  Mazza, E.;  Szekely, G.;  Falk, V.]]></authors>
##     <affiliations><![CDATA[Hybrid Lab. for Cardiovascular Technol., Univ. of Zurich, Zurich, Switzerland]]></affiliations>
##     <controlledterms>
##       <term><![CDATA[biomechanics]]></term>
##       <term><![CDATA[computerised tomography]]></term>
##       <term><![CDATA[deformation]]></term>
##       <term><![CDATA[finite element analysis]]></term>
##       <term><![CDATA[medical image processing]]></term>
##       <term><![CDATA[stents]]></term>
##     </controlledterms>
##     <thesaurusterms>
##       <term><![CDATA[Calcium]]></term>
##       <term><![CDATA[Computational modeling]]></term>
##       <term><![CDATA[Force]]></term>
##       <term><![CDATA[Image reconstruction]]></term>
##       <term><![CDATA[Numerical models]]></term>
##       <term><![CDATA[Shape]]></term>
##       <term><![CDATA[Valves]]></term>
##     </thesaurusterms>
##     <pubtitle><![CDATA[Biomedical Engineering, IEEE Transactions on]]></pubtitle>
##     <punumber><![CDATA[10]]></punumber>
##     <pubtype><![CDATA[Journals & Magazines]]></pubtype>
##     <publisher><![CDATA[IEEE]]></publisher>
##     <volume><![CDATA[61]]></volume>
##     <issue><![CDATA[1]]></issue>
##     <py><![CDATA[2014]]></py>
##     <spage><![CDATA[4]]></spage>
##     <epage><![CDATA[15]]></epage>
##     <abstract><![CDATA[An approach for extracting the radial force load on an implanted stent from medical images is proposed. To exemplify the approach, a system is presented which computes a radial force estimation from computer tomography images acquired from patients who underwent transcatheter aortic valve implantation (TAVI). The deformed shape of the implanted valve prosthesis' Nitinol frame is extracted from the images. A set of displacement vectors is computed that parameterizes the observed deformation. An iterative relaxation algorithm is employed to adapt the information extracted from the images to a finite-element model of the stent, and the radial components of the interaction forces between the stent and the tissue are extracted. For the evaluation of the method, tests were run using the clinical data from 21 patients. Stent modeling and extraction of the radial forces were successful in 18 cases. Synthetic test cases were generated, in addition, for assessing the sensitivity to the measurement errors. In a sensitivity analysis, the geometric error of the stent reconstruction was below 0.3 mm, which is below the image resolution. The distribution of the radial forces was qualitatively and quantitatively reasonable. An uncertainty remains in the quantitative evaluation of the radial forces due to the uncertainty in defining a radial direction on the deformed stent. With our approach, the mechanical situation of TAVI stents after the implantation can be studied in vivo, which may help to understand the mechanisms that lead to the complications and improve stent design.]]></abstract>
##     <issn><![CDATA[0018-9294]]></issn>
##     <htmlFlag><![CDATA[1]]></htmlFlag>
##     <arnumber><![CDATA[6560362]]></arnumber>
##     <doi><![CDATA[10.1109/TBME.2013.2273496]]></doi>
##     <publicationId><![CDATA[6560362]]></publicationId>
##     <mdurl><![CDATA[http://ieeexplore.ieee.org/xpl/articleDetails.jsp?tp=&arnumber=6560362&contentType=Journals+%26+Magazines]]></mdurl>
##     <pdf><![CDATA[http://ieeexplore.ieee.org/stamp/stamp.jsp?arnumber=6560362]]></pdf>
##   </document>
##   <document>
##     <rank>1040</rank>
##     <title><![CDATA[Development of a Laser-Range-Finder-Based Human Tracking and Control Algorithm for a Marathoner Service Robot]]></title>
##     <authors><![CDATA[Eui-Jung Jung;  Jae Hoon Lee;  Byung-Ju Yi;  Jooyoung Park;  Yuta, S.;  Si-Tae Noh]]></authors>
##     <affiliations><![CDATA[Dept. of Electron. Electr., Control & Instrum. Eng., Hanyang Univ., Ansan, South Korea]]></affiliations>
##     <controlledterms>
##       <term><![CDATA[collision avoidance]]></term>
##       <term><![CDATA[laser ranging]]></term>
##       <term><![CDATA[mobile robots]]></term>
##       <term><![CDATA[service robots]]></term>
##       <term><![CDATA[support vector machines]]></term>
##       <term><![CDATA[target tracking]]></term>
##     </controlledterms>
##     <thesaurusterms>
##       <term><![CDATA[Feature extraction]]></term>
##       <term><![CDATA[Heuristic algorithms]]></term>
##       <term><![CDATA[Lasers]]></term>
##       <term><![CDATA[Robot sensing systems]]></term>
##       <term><![CDATA[Shape]]></term>
##       <term><![CDATA[Training]]></term>
##     </thesaurusterms>
##     <pubtitle><![CDATA[Mechatronics, IEEE/ASME Transactions on]]></pubtitle>
##     <punumber><![CDATA[3516]]></punumber>
##     <pubtype><![CDATA[Journals & Magazines]]></pubtype>
##     <publisher><![CDATA[IEEE]]></publisher>
##     <volume><![CDATA[19]]></volume>
##     <issue><![CDATA[6]]></issue>
##     <py><![CDATA[2014]]></py>
##     <spage><![CDATA[1963]]></spage>
##     <epage><![CDATA[1976]]></epage>
##     <abstract><![CDATA[This paper presents a human detection algorithm and an obstacle avoidance algorithm for a marathoner service robot (MSR) that provides a service to a marathoner while training. To be used as a MSR, the mobile robot should have the abilities to follow a running human and avoid dynamically moving obstacles in an unstructured outdoor environment. To detect a human by a laser range finder (LRF), we defined features of the human body in LRF data and employed a support vector data description method. In order to avoid moving obstacles while tracking a running person, we defined a weighted radius for each obstacle using the relative velocity between the robot and an obstacle. For smoothly bypassing obstacles without collision, a dynamic obstacle avoidance algorithm for the MSR is implemented, which directly employed a real-time position vector between the robot and the shortest path around the obstacle. We verified the feasibility of these proposed algorithms through experimentation in different outdoor environments.]]></abstract>
##     <issn><![CDATA[1083-4435]]></issn>
##     <arnumber><![CDATA[6690173]]></arnumber>
##     <doi><![CDATA[10.1109/TMECH.2013.2294180]]></doi>
##     <publicationId><![CDATA[6690173]]></publicationId>
##     <mdurl><![CDATA[http://ieeexplore.ieee.org/xpl/articleDetails.jsp?tp=&arnumber=6690173&contentType=Journals+%26+Magazines]]></mdurl>
##     <pdf><![CDATA[http://ieeexplore.ieee.org/stamp/stamp.jsp?arnumber=6690173]]></pdf>
##   </document>
##   <document>
##     <rank>1041</rank>
##     <title><![CDATA[Source Separation for Wideband Energy Emissions Using Complex Independent Component Analysis]]></title>
##     <authors><![CDATA[Arnaut, L.R.;  Obiekezie, C.S.]]></authors>
##     <affiliations><![CDATA[George Green Inst. of Electromagn. Res., Univ. of Nottingham, Nottingham, UK]]></affiliations>
##     <controlledterms>
##       <term><![CDATA[independent component analysis]]></term>
##       <term><![CDATA[iterative methods]]></term>
##       <term><![CDATA[maximum likelihood estimation]]></term>
##       <term><![CDATA[probability]]></term>
##       <term><![CDATA[source separation]]></term>
##       <term><![CDATA[ultra wideband communication]]></term>
##     </controlledterms>
##     <thesaurusterms>
##       <term><![CDATA[Correlation]]></term>
##       <term><![CDATA[Independent component analysis]]></term>
##       <term><![CDATA[Maximum likelihood estimation]]></term>
##       <term><![CDATA[Noise measurement]]></term>
##       <term><![CDATA[Principal component analysis]]></term>
##       <term><![CDATA[Source separation]]></term>
##       <term><![CDATA[Vectors]]></term>
##     </thesaurusterms>
##     <pubtitle><![CDATA[Electromagnetic Compatibility, IEEE Transactions on]]></pubtitle>
##     <punumber><![CDATA[15]]></punumber>
##     <pubtype><![CDATA[Journals & Magazines]]></pubtype>
##     <publisher><![CDATA[IEEE]]></publisher>
##     <volume><![CDATA[56]]></volume>
##     <issue><![CDATA[3]]></issue>
##     <py><![CDATA[2014]]></py>
##     <spage><![CDATA[559]]></spage>
##     <epage><![CDATA[570]]></epage>
##     <abstract><![CDATA[Complex independent component analysis is formulated for ultrawideband (UWB) incoherent fields, based on maximum-likelihood estimation with iterative natural gradient searching. It can be applied to measurements of complex power separated in contributions by propagating and reactive near-fields. A set of novel nonlinear asymmetric score functions is derived that are asymptotically matched to skewed probability density functions of incoherent energy sources departing from ideal chi-square ensemble distributions. The source decomposition is applied to two analog and digital integrated circuits. Based on the measured intensity of the emitted magnetic field, the technique is shown to enable nonlinear extraction of a set of independent spatially separated UWB sources, without knowledge of neither the detailed circuit geometry nor its on-board signals. Overcompleteness of the data set is found to have a relatively small effect on the accuracy of the estimated magnitude and location of the emission hot spots, but has a major influence on the global spatial maps and nature of the extracted sources.]]></abstract>
##     <issn><![CDATA[0018-9375]]></issn>
##     <htmlFlag><![CDATA[1]]></htmlFlag>
##     <arnumber><![CDATA[6701168]]></arnumber>
##     <doi><![CDATA[10.1109/TEMC.2013.2289384]]></doi>
##     <publicationId><![CDATA[6701168]]></publicationId>
##     <mdurl><![CDATA[http://ieeexplore.ieee.org/xpl/articleDetails.jsp?tp=&arnumber=6701168&contentType=Journals+%26+Magazines]]></mdurl>
##     <pdf><![CDATA[http://ieeexplore.ieee.org/stamp/stamp.jsp?arnumber=6701168]]></pdf>
##   </document>
##   <document>
##     <rank>1042</rank>
##     <title><![CDATA[Slot Spiral Silicon Photonic Crystal Fiber With Property of Both High Birefringence and High Nonlinearity]]></title>
##     <authors><![CDATA[Tianye Huang;  Jianfei Liao;  Songnian Fu;  Tang, M.;  Shum, P.;  Deming Liu]]></authors>
##     <affiliations><![CDATA[Wuhan Nat. Lab. for Optoelectron., Huazhong Univ. of Sci. & Technol., Wuhan, China]]></affiliations>
##     <controlledterms>
##       <term><![CDATA[birefringence]]></term>
##       <term><![CDATA[elemental semiconductors]]></term>
##       <term><![CDATA[holey fibres]]></term>
##       <term><![CDATA[nonlinear optics]]></term>
##       <term><![CDATA[optical fibre cladding]]></term>
##       <term><![CDATA[optical fibre polarisation]]></term>
##       <term><![CDATA[photonic crystals]]></term>
##       <term><![CDATA[silicon]]></term>
##     </controlledterms>
##     <thesaurusterms>
##       <term><![CDATA[Indexes]]></term>
##       <term><![CDATA[Optical fiber polarization]]></term>
##       <term><![CDATA[Optical fiber sensors]]></term>
##       <term><![CDATA[Photonic crystal fibers]]></term>
##       <term><![CDATA[Silicon]]></term>
##     </thesaurusterms>
##     <pubtitle><![CDATA[Photonics Journal, IEEE]]></pubtitle>
##     <punumber><![CDATA[4563994]]></punumber>
##     <pubtype><![CDATA[Journals & Magazines]]></pubtype>
##     <publisher><![CDATA[IEEE]]></publisher>
##     <volume><![CDATA[6]]></volume>
##     <issue><![CDATA[3]]></issue>
##     <py><![CDATA[2014]]></py>
##     <spage><![CDATA[1]]></spage>
##     <epage><![CDATA[7]]></epage>
##     <abstract><![CDATA[A photonic crystal fiber (PCF) made of silicon is proposed and numerically optimized with both ultrahigh birefringence and nonlinearity. By introducing an elliptical low-index slot in the core region, one of the orthogonally polarized fundamental modes can be confined in the low-index region due to the discontinuity of its electric field at the interface of the slot and the surrounding material, while the other fundamental mode is dominated by the modified total internal reflection (MTIR) induced by air holes of the cladding, leading to a birefringence as high as 10<sup>-1</sup> order. On the other hand, benefiting from the tight field confinement and highly nonlinear material, nonlinear coefficients up to 1068 W<sup>-1</sup> &#x00B7; m<sup>-1</sup> and 162 W<sup>-1</sup> &#x00B7; m<sup>-1</sup> can be achieved for the slot mode and the MTIR mode, respectively.]]></abstract>
##     <issn><![CDATA[1943-0655]]></issn>
##     <htmlFlag><![CDATA[1]]></htmlFlag>
##     <arnumber><![CDATA[6814776]]></arnumber>
##     <doi><![CDATA[10.1109/JPHOT.2014.2323312]]></doi>
##     <publicationId><![CDATA[6814776]]></publicationId>
##     <mdurl><![CDATA[http://ieeexplore.ieee.org/xpl/articleDetails.jsp?tp=&arnumber=6814776&contentType=Journals+%26+Magazines]]></mdurl>
##     <pdf><![CDATA[http://ieeexplore.ieee.org/stamp/stamp.jsp?arnumber=6814776]]></pdf>
##   </document>
##   <document>
##     <rank>1043</rank>
##     <title><![CDATA[Recent Advances in Radio Resource Management for Heterogeneous LTE/LTE-A Networks]]></title>
##     <authors><![CDATA[Ying Loong Lee;  Teong Chee Chuah;  Loo, J.;  Vinel, A.]]></authors>
##     <affiliations><![CDATA[Fac. of Eng., Multimedia Univ., Cyberjaya, Malaysia]]></affiliations>
##     <controlledterms>
##       <term><![CDATA[Long Term Evolution]]></term>
##       <term><![CDATA[femtocellular radio]]></term>
##       <term><![CDATA[interference suppression]]></term>
##       <term><![CDATA[quality of service]]></term>
##       <term><![CDATA[radio spectrum management]]></term>
##       <term><![CDATA[radiofrequency interference]]></term>
##       <term><![CDATA[relay networks (telecommunication)]]></term>
##     </controlledterms>
##     <thesaurusterms>
##       <term><![CDATA[Femtocells]]></term>
##       <term><![CDATA[Interference]]></term>
##       <term><![CDATA[Long Term Evolution]]></term>
##       <term><![CDATA[Macrocell networks]]></term>
##       <term><![CDATA[Quality of service]]></term>
##       <term><![CDATA[Relays]]></term>
##       <term><![CDATA[Resource management]]></term>
##     </thesaurusterms>
##     <pubtitle><![CDATA[Communications Surveys & Tutorials, IEEE]]></pubtitle>
##     <punumber><![CDATA[9739]]></punumber>
##     <pubtype><![CDATA[Journals & Magazines]]></pubtype>
##     <publisher><![CDATA[IEEE]]></publisher>
##     <volume><![CDATA[16]]></volume>
##     <issue><![CDATA[4]]></issue>
##     <py><![CDATA[2014]]></py>
##     <spage><![CDATA[2142]]></spage>
##     <epage><![CDATA[2180]]></epage>
##     <abstract><![CDATA[As heterogeneous networks (HetNets) emerge as one of the most promising developments toward realizing the target specifications of Long Term Evolution (LTE) and LTE-Advanced (LTE-A) networks, radio resource management (RRM) research for such networks has, in recent times, been intensively pursued. Clearly, recent research mainly concentrates on the aspect of interference mitigation. Other RRM aspects, such as radio resource utilization, fairness, complexity, and QoS, have not been given much attention. In this paper, we aim to provide an overview of the key challenges arising from HetNets and highlight their importance. Subsequently, we present a comprehensive survey of the RRM schemes that have been studied in recent years for LTE/LTE-A HetNets, with a particular focus on those for femtocells and relay nodes. Furthermore, we classify these RRM schemes according to their underlying approaches. In addition, these RRM schemes are qualitatively analyzed and compared to each other. We also identify a number of potential research directions for future RRM development. Finally, we discuss the lack of current RRM research and the importance of multi-objective RRM studies.]]></abstract>
##     <issn><![CDATA[1553-877X]]></issn>
##     <htmlFlag><![CDATA[1]]></htmlFlag>
##     <arnumber><![CDATA[6824744]]></arnumber>
##     <doi><![CDATA[10.1109/COMST.2014.2326303]]></doi>
##     <publicationId><![CDATA[6824744]]></publicationId>
##     <mdurl><![CDATA[http://ieeexplore.ieee.org/xpl/articleDetails.jsp?tp=&arnumber=6824744&contentType=Journals+%26+Magazines]]></mdurl>
##     <pdf><![CDATA[http://ieeexplore.ieee.org/stamp/stamp.jsp?arnumber=6824744]]></pdf>
##   </document>
##   <document>
##     <rank>1044</rank>
##     <title><![CDATA[Radio interface evolution towards 5G and enhanced local area communications]]></title>
##     <authors><![CDATA[Levanen, T.A.;  Pirskanen, J.;  Koskela, T.;  Talvitie, J.;  Valkama, M.]]></authors>
##     <affiliations><![CDATA[Dept. of Electron. & Commun. Eng., Tampere Univ. of Technol., Tampere, Finland]]></affiliations>
##     <controlledterms>
##       <term><![CDATA[4G mobile communication]]></term>
##       <term><![CDATA[Long Term Evolution]]></term>
##       <term><![CDATA[femtocellular radio]]></term>
##       <term><![CDATA[picocellular radio]]></term>
##       <term><![CDATA[radiofrequency interference]]></term>
##       <term><![CDATA[telecommunication standards]]></term>
##       <term><![CDATA[telecommunication traffic]]></term>
##       <term><![CDATA[time division multiplexing]]></term>
##       <term><![CDATA[wireless LAN]]></term>
##     </controlledterms>
##     <thesaurusterms>
##       <term><![CDATA[Long Term Evolution]]></term>
##       <term><![CDATA[Mobile communication]]></term>
##       <term><![CDATA[OFDM]]></term>
##       <term><![CDATA[Physical layer]]></term>
##       <term><![CDATA[Throughput]]></term>
##       <term><![CDATA[Uplink]]></term>
##       <term><![CDATA[Wireless LAN]]></term>
##     </thesaurusterms>
##     <pubtitle><![CDATA[Access, IEEE]]></pubtitle>
##     <punumber><![CDATA[6287639]]></punumber>
##     <pubtype><![CDATA[Journals & Magazines]]></pubtype>
##     <publisher><![CDATA[IEEE]]></publisher>
##     <volume><![CDATA[2]]></volume>
##     <py><![CDATA[2014]]></py>
##     <spage><![CDATA[1005]]></spage>
##     <epage><![CDATA[1029]]></epage>
##     <abstract><![CDATA[The exponential growth of mobile data in macronetworks has driven the evolution of communications systems toward spectrally efficient, energy efficient, and fast local area communications. It is a well-known fact that the best way to increase capacity in a unit area is to introduce smaller cells. Local area communications are currently mainly driven by the IEEE 802.11 WLAN family being cheap and energy efficient with a low number of users per access point. For the future high user density scenarios, following the 802.11 HEW study group, the 802.11ax project has been initiated to improve the WLAN system performance. The 3GPP LTE-advanced (LTE-A) also includes new methods for pico and femto cell's interference management functionalities for small cell communications. The main problem with LTE-A is, however, that the physical layer numerology is still optimized for macrocells and not for local area communications. Furthermore, the overall complexity and the overheads of the control plane and reference symbols are too large for spectrally and energy efficient local area communications. In this paper, we provide first an overview of WLAN 802.11ac and LTE/LTE-A, discuss the pros and cons of both technology areas, and then derive a new flexible TDD-based radio interface parametrization for 5G local area communications combining the best practices of both WiFi and LTE-A technologies. We justify the system design based on local area propagation characteristics and expected traffic distributions and derive targets for future local area concepts. We concentrate on initial physical layer design and discuss how it maps to higher layer improvements. This paper shows that the new design can significantly reduce the latency of the system, and offer increased sleeping opportunities on both base station and user equipment sides leading to enhanced power savings. In addition, through careful design of the control overhead, we are able to improve the channel utilization when compared- with LTE-A.]]></abstract>
##     <issn><![CDATA[2169-3536]]></issn>
##     <htmlFlag><![CDATA[1]]></htmlFlag>
##     <arnumber><![CDATA[6891105]]></arnumber>
##     <doi><![CDATA[10.1109/ACCESS.2014.2355415]]></doi>
##     <publicationId><![CDATA[6891105]]></publicationId>
##     <mdurl><![CDATA[http://ieeexplore.ieee.org/xpl/articleDetails.jsp?tp=&arnumber=6891105&contentType=Journals+%26+Magazines]]></mdurl>
##     <pdf><![CDATA[http://ieeexplore.ieee.org/stamp/stamp.jsp?arnumber=6891105]]></pdf>
##   </document>
##   <document>
##     <rank>1045</rank>
##     <title><![CDATA[Computation of Losses in HTS Under the Action of Varying Magnetic Fields and Currents]]></title>
##     <authors><![CDATA[Grilli, F.;  Pardo, E.;  Stenvall, A.;  Nguyen, D.N.;  Weijia Yuan;  Gomory, F.]]></authors>
##     <affiliations><![CDATA[Inst. for Tech. Phys., Karlsruhe Inst. of Technol., Karlsruhe, Germany]]></affiliations>
##     <controlledterms>
##       <term><![CDATA[dielectric losses]]></term>
##       <term><![CDATA[eddy current losses]]></term>
##       <term><![CDATA[high-temperature superconductors]]></term>
##       <term><![CDATA[superconducting tapes]]></term>
##     </controlledterms>
##     <thesaurusterms>
##       <term><![CDATA[Computational modeling]]></term>
##       <term><![CDATA[Current density]]></term>
##       <term><![CDATA[Electromagnetics]]></term>
##       <term><![CDATA[High-temperature superconductors]]></term>
##       <term><![CDATA[Magnetic hysteresis]]></term>
##       <term><![CDATA[Superconducting magnets]]></term>
##     </thesaurusterms>
##     <pubtitle><![CDATA[Applied Superconductivity, IEEE Transactions on]]></pubtitle>
##     <punumber><![CDATA[77]]></punumber>
##     <pubtype><![CDATA[Journals & Magazines]]></pubtype>
##     <publisher><![CDATA[IEEE]]></publisher>
##     <volume><![CDATA[24]]></volume>
##     <issue><![CDATA[1]]></issue>
##     <py><![CDATA[2014]]></py>
##     <spage><![CDATA[78]]></spage>
##     <epage><![CDATA[110]]></epage>
##     <abstract><![CDATA[Numerical modeling of superconductors is widely recognized as a powerful tool for interpreting experimental results, understanding physical mechanisms, and predicting the performance of high-temperature-superconductor (HTS) tapes, wires, and devices. This is particularly true for ac loss calculation since a sufficiently low ac loss value is imperative to make these materials attractive for commercialization. In recent years, a large variety of numerical models, which are based on different techniques and implementations, has been proposed by researchers around the world, with the purpose of being able to estimate ac losses in HTSs quickly and accurately. This paper presents a literature review of the methods for computing ac losses in HTS tapes, wires, and devices. Technical superconductors have a relatively complex geometry (filaments, which might be twisted or transposed, or layers) and consist of different materials. As a result, different loss contributions exist. In this paper, we describe the ways of computing such loss contributions, which include hysteresis losses, eddy-current losses, coupling losses, and losses in ferromagnetic materials. We also provide an estimation of the losses occurring in a variety of power applications.]]></abstract>
##     <issn><![CDATA[1051-8223]]></issn>
##     <htmlFlag><![CDATA[1]]></htmlFlag>
##     <arnumber><![CDATA[6648727]]></arnumber>
##     <doi><![CDATA[10.1109/TASC.2013.2259827]]></doi>
##     <publicationId><![CDATA[6648727]]></publicationId>
##     <mdurl><![CDATA[http://ieeexplore.ieee.org/xpl/articleDetails.jsp?tp=&arnumber=6648727&contentType=Journals+%26+Magazines]]></mdurl>
##     <pdf><![CDATA[http://ieeexplore.ieee.org/stamp/stamp.jsp?arnumber=6648727]]></pdf>
##   </document>
##   <document>
##     <rank>1046</rank>
##     <title><![CDATA[Structured Methodology for the Investigation of Contact Voltages]]></title>
##     <authors><![CDATA[Hanebuth, S.C.;  Kalokitis, D.;  Cedrone, A.]]></authors>
##     <affiliations><![CDATA[, Power Survey Company, Kearny, NJ, USA]]></affiliations>
##     <thesaurusterms>
##       <term><![CDATA[Conductors]]></term>
##       <term><![CDATA[Electrical safety]]></term>
##       <term><![CDATA[Energy measurement]]></term>
##       <term><![CDATA[Harmonic analysis]]></term>
##       <term><![CDATA[Impedance]]></term>
##       <term><![CDATA[Power system reliability]]></term>
##       <term><![CDATA[Resistors]]></term>
##       <term><![CDATA[Voltage measurement]]></term>
##       <term><![CDATA[Voltmeters]]></term>
##     </thesaurusterms>
##     <pubtitle><![CDATA[Power and Energy Technology Systems Journal, IEEE]]></pubtitle>
##     <punumber><![CDATA[6687318]]></punumber>
##     <pubtype><![CDATA[Journals & Magazines]]></pubtype>
##     <publisher><![CDATA[IEEE]]></publisher>
##     <volume><![CDATA[1]]></volume>
##     <py><![CDATA[2014]]></py>
##     <spage><![CDATA[1]]></spage>
##     <epage><![CDATA[11]]></epage>
##     <abstract><![CDATA[Utilities are often asked to investigate and determine the cause of elevated voltage in the urban environment. By understanding proper measurement techniques, how to interpret readings, and the possible sources of the voltage, investigators can reach speedy and accurate conclusions and eliminate unwanted voltage.]]></abstract>
##     <htmlFlag><![CDATA[1]]></htmlFlag>
##     <arnumber><![CDATA[6963586]]></arnumber>
##     <doi><![CDATA[10.1109/JPETS.2014.2363403]]></doi>
##     <publicationId><![CDATA[6963586]]></publicationId>
##     <mdurl><![CDATA[http://ieeexplore.ieee.org/xpl/articleDetails.jsp?tp=&arnumber=6963586&contentType=Journals+%26+Magazines]]></mdurl>
##     <pdf><![CDATA[http://ieeexplore.ieee.org/stamp/stamp.jsp?arnumber=6963586]]></pdf>
##   </document>
##   <document>
##     <rank>1047</rank>
##     <title><![CDATA[Wireless Power Transmission: R&amp;D Activities Within Europe]]></title>
##     <authors><![CDATA[Borges Carvalho, N.;  Georgiadis, A.;  Costanzo, A.;  Rogier, H.;  Collado, A.;  Garci&#x0301; a, J.A.;  Lucyszyn, S.;  Mezzanotte, P.;  Kracek, J.;  Masotti, D.;  Boaventura, A.J.S.;  de las Nieves Rui&#x0301; z Lavin, M.;  Pinuela, M.;  Yates, D.C.;  Mitcheson, P.D.;  Mazanek, M.;  Pankrac, V.]]></authors>
##     <affiliations><![CDATA[Dept. de Electron., Univ. de Aveiro, Aveiro, Portugal]]></affiliations>
##     <controlledterms>
##       <term><![CDATA[government policies]]></term>
##       <term><![CDATA[inductive power transmission]]></term>
##       <term><![CDATA[power electronics]]></term>
##       <term><![CDATA[research and development]]></term>
##     </controlledterms>
##     <thesaurusterms>
##       <term><![CDATA[Coils]]></term>
##       <term><![CDATA[Couplings]]></term>
##       <term><![CDATA[Educational institutions]]></term>
##       <term><![CDATA[Radio frequency]]></term>
##       <term><![CDATA[Substrates]]></term>
##       <term><![CDATA[Wireless communication]]></term>
##       <term><![CDATA[Wireless sensor networks]]></term>
##     </thesaurusterms>
##     <pubtitle><![CDATA[Microwave Theory and Techniques, IEEE Transactions on]]></pubtitle>
##     <punumber><![CDATA[22]]></punumber>
##     <pubtype><![CDATA[Journals & Magazines]]></pubtype>
##     <publisher><![CDATA[IEEE]]></publisher>
##     <volume><![CDATA[62]]></volume>
##     <issue><![CDATA[4]]></issue>
##     <part><![CDATA[2]]></part>
##     <py><![CDATA[2014]]></py>
##     <spage><![CDATA[1031]]></spage>
##     <epage><![CDATA[1045]]></epage>
##     <abstract><![CDATA[Wireless power transmission (WPT) is an emerging technology that is gaining increased visibility in recent years. Efficient WPT circuits, systems and strategies can address a large group of applications spanning from batteryless systems, battery-free sensors, passive RF identification, near-field communications, and many others. WPT is a fundamental enabling technology of the Internet of Things concept, as well as machine-to-machine communications, since it minimizes the use of batteries and eliminates wired power connections. WPT technology brings together RF and dc circuit and system designers with different backgrounds on circuit design, novel materials and applications, and regulatory issues, forming a cross disciplinary team in order to achieve an efficient transmission of power over the air interface. This paper aims to present WPT technology in an integrated way, addressing state-of-the-art and challenges, and to discuss future R&amp;D perspectives summarizing recent activities in Europe.]]></abstract>
##     <issn><![CDATA[0018-9480]]></issn>
##     <htmlFlag><![CDATA[1]]></htmlFlag>
##     <arnumber><![CDATA[6734736]]></arnumber>
##     <doi><![CDATA[10.1109/TMTT.2014.2303420]]></doi>
##     <publicationId><![CDATA[6734736]]></publicationId>
##     <mdurl><![CDATA[http://ieeexplore.ieee.org/xpl/articleDetails.jsp?tp=&arnumber=6734736&contentType=Journals+%26+Magazines]]></mdurl>
##     <pdf><![CDATA[http://ieeexplore.ieee.org/stamp/stamp.jsp?arnumber=6734736]]></pdf>
##   </document>
##   <document>
##     <rank>1048</rank>
##     <title><![CDATA[Device Concepts for Graphene-Based Terahertz Photonics]]></title>
##     <authors><![CDATA[Tredicucci, A.;  Vitiello, M.S.]]></authors>
##     <affiliations><![CDATA[NEST, Ist. Nanoscienze, Pisa, Italy]]></affiliations>
##     <controlledterms>
##       <term><![CDATA[bolometers]]></term>
##       <term><![CDATA[graphene]]></term>
##       <term><![CDATA[high-speed optical techniques]]></term>
##       <term><![CDATA[microwave photonics]]></term>
##       <term><![CDATA[optical materials]]></term>
##       <term><![CDATA[optical modulation]]></term>
##       <term><![CDATA[plasma applications]]></term>
##     </controlledterms>
##     <thesaurusterms>
##       <term><![CDATA[Detectors]]></term>
##       <term><![CDATA[Frequency modulation]]></term>
##       <term><![CDATA[Graphene]]></term>
##       <term><![CDATA[Logic gates]]></term>
##       <term><![CDATA[Photonics]]></term>
##       <term><![CDATA[Plasmons]]></term>
##     </thesaurusterms>
##     <pubtitle><![CDATA[Selected Topics in Quantum Electronics, IEEE Journal of]]></pubtitle>
##     <punumber><![CDATA[2944]]></punumber>
##     <pubtype><![CDATA[Journals & Magazines]]></pubtype>
##     <publisher><![CDATA[IEEE]]></publisher>
##     <volume><![CDATA[20]]></volume>
##     <issue><![CDATA[1]]></issue>
##     <py><![CDATA[2014]]></py>
##     <spage><![CDATA[130]]></spage>
##     <epage><![CDATA[138]]></epage>
##     <abstract><![CDATA[Graphene is establishing itself as a new photonic material with huge potential in a variety of applications ranging from transparent electrodes in displays and photovoltaic modules to saturable absorber in mode-locked lasers. Its peculiar bandstructure and electron transport characteristics naturally suggest graphene could also form the basis for a new generation of high-performance devices operating in the terahertz (THz) range of the electromagnetic spectrum. The region between 300 GHz and 10 THz is in fact still characterized by a lack of efficient, compact, solid state photonic components capable of operating well at 300 K. Recent works have already shown very promising results in the development of high-speed modulators as well as of bolometer and plasma-wave detectors. Furthermore, several concepts have been proposed aiming at the realization of lasers and oscillators. This paper will review the latest achievements in graphene-based THz photonics and discuss future perspectives of this rapidly developing research field.]]></abstract>
##     <issn><![CDATA[1077-260X]]></issn>
##     <htmlFlag><![CDATA[1]]></htmlFlag>
##     <arnumber><![CDATA[6556965]]></arnumber>
##     <doi><![CDATA[10.1109/JSTQE.2013.2271692]]></doi>
##     <publicationId><![CDATA[6556965]]></publicationId>
##     <mdurl><![CDATA[http://ieeexplore.ieee.org/xpl/articleDetails.jsp?tp=&arnumber=6556965&contentType=Journals+%26+Magazines]]></mdurl>
##     <pdf><![CDATA[http://ieeexplore.ieee.org/stamp/stamp.jsp?arnumber=6556965]]></pdf>
##   </document>
##   <document>
##     <rank>1049</rank>
##     <title><![CDATA[Mechanism of Spectrum Moving, Narrowing, Broadening, and Wavelength Switching of Dissipative Solitons in All-Normal-Dispersion Yb-Fiber Lasers]]></title>
##     <authors><![CDATA[Luo, J.L.;  Ge, Y.Q.;  Tang, D.Y.;  Zhang, S.M.;  Shen, D.Y.;  Zhao, L.M.]]></authors>
##     <affiliations><![CDATA[Jiangsu Key Lab. of Adv. Laser Mater. & Devices, Jiangsu Normal Univ., Xuzhou, China]]></affiliations>
##     <controlledterms>
##       <term><![CDATA[fibre lasers]]></term>
##       <term><![CDATA[optical solitons]]></term>
##       <term><![CDATA[ytterbium]]></term>
##     </controlledterms>
##     <thesaurusterms>
##       <term><![CDATA[Band-pass filters]]></term>
##       <term><![CDATA[Bandwidth]]></term>
##       <term><![CDATA[Cavity resonators]]></term>
##       <term><![CDATA[Laser mode locking]]></term>
##       <term><![CDATA[Optical fiber dispersion]]></term>
##       <term><![CDATA[Optical fiber polarization]]></term>
##       <term><![CDATA[Tuning]]></term>
##     </thesaurusterms>
##     <pubtitle><![CDATA[Photonics Journal, IEEE]]></pubtitle>
##     <punumber><![CDATA[4563994]]></punumber>
##     <pubtype><![CDATA[Journals & Magazines]]></pubtype>
##     <publisher><![CDATA[IEEE]]></publisher>
##     <volume><![CDATA[6]]></volume>
##     <issue><![CDATA[1]]></issue>
##     <py><![CDATA[2014]]></py>
##     <spage><![CDATA[1]]></spage>
##     <epage><![CDATA[8]]></epage>
##     <abstract><![CDATA[We report experimental observations and numerical simulation results on the spectrum moving, narrowing, broadening, and wavelength switching of dissipative solitons (DSs) in an all-normal-dispersion Yb-fiber laser that is passively mode-locked by using the nonlinear polarization rotation (NPR) technique. We found numerically that the DS spectrum moving, together with spectrum narrowing/broadening, is caused by the effective gain profile change resulted from the moving of the artificial spectral filter. Furthermore, we show that the wavelength switching observed in the laser is a natural consequence of the effective gain switching. The moving of the artificial spectral filter could be originated from either the cavity birefringence change or the polarizer rotation. Due to the broad gain and the artificial birefringent filter introduced by the NPR technique, apart from the central wavelength shifting and bandwidth changing, wavelength switching of DSs could be obtained by simply rotating the polarizer. Numerical simulations well reproduced the experimental observations. Our results suggest that extra effort should be made for wavelength tuning if there is any polarization-dependent component in the cavity as the wavelength switching will interrupt the continuously wavelength shift.]]></abstract>
##     <issn><![CDATA[1943-0655]]></issn>
##     <htmlFlag><![CDATA[1]]></htmlFlag>
##     <arnumber><![CDATA[6704764]]></arnumber>
##     <doi><![CDATA[10.1109/JPHOT.2014.2298405]]></doi>
##     <publicationId><![CDATA[6704764]]></publicationId>
##     <mdurl><![CDATA[http://ieeexplore.ieee.org/xpl/articleDetails.jsp?tp=&arnumber=6704764&contentType=Journals+%26+Magazines]]></mdurl>
##     <pdf><![CDATA[http://ieeexplore.ieee.org/stamp/stamp.jsp?arnumber=6704764]]></pdf>
##   </document>
##   <document>
##     <rank>1050</rank>
##     <title><![CDATA[Loss-Tolerant Allpass-Based Filter Bank Design Suitable for Integrated Optics]]></title>
##     <authors><![CDATA[Wang, Y.;  Aguinaldo, R.;  Nguyen, T.]]></authors>
##     <thesaurusterms>
##       <term><![CDATA[Algorithm design and analysis]]></term>
##       <term><![CDATA[Band-pass filters]]></term>
##       <term><![CDATA[Filter banks]]></term>
##       <term><![CDATA[Optical device fabrication]]></term>
##       <term><![CDATA[Optical losses]]></term>
##       <term><![CDATA[Optical ring resonators]]></term>
##       <term><![CDATA[Photonic integrated circuits]]></term>
##     </thesaurusterms>
##     <pubtitle><![CDATA[Photonics Journal, IEEE]]></pubtitle>
##     <punumber><![CDATA[4563994]]></punumber>
##     <pubtype><![CDATA[Journals & Magazines]]></pubtype>
##     <publisher><![CDATA[IEEE]]></publisher>
##     <volume><![CDATA[6]]></volume>
##     <issue><![CDATA[5]]></issue>
##     <py><![CDATA[2014]]></py>
##     <spage><![CDATA[1]]></spage>
##     <epage><![CDATA[12]]></epage>
##     <abstract><![CDATA[This paper presents a novel design algorithm for a filter bank structure that is suitable for photonic integrated circuits. The proposed design is entirely based on the principle of loss-exhibiting allpass filtering, which is a behavior naturally observed in a variety of microscale optical components. The setup and mathematical optimization of the system is explored in detail from a signal processing perspective, and an algorithm based on convex and nonlinear optimization techniques is subsequently presented. The algorithm methodologically utilizes the otherwise-undesirable effect of waveguide loss to derive an allpass-based filter bank structure that is amenable to photonic implementation. The proposed structure is ideal for applications in time-stretched analog-to-digital converters, WDM network channelizers, and general optical signal multiplexers.]]></abstract>
##     <issn><![CDATA[1943-0655]]></issn>
##     <htmlFlag><![CDATA[1]]></htmlFlag>
##     <arnumber><![CDATA[6910234]]></arnumber>
##     <doi><![CDATA[10.1109/JPHOT.2014.2360301]]></doi>
##     <publicationId><![CDATA[6910234]]></publicationId>
##     <mdurl><![CDATA[http://ieeexplore.ieee.org/xpl/articleDetails.jsp?tp=&arnumber=6910234&contentType=Journals+%26+Magazines]]></mdurl>
##     <pdf><![CDATA[http://ieeexplore.ieee.org/stamp/stamp.jsp?arnumber=6910234]]></pdf>
##   </document>
##   <document>
##     <rank>1051</rank>
##     <title><![CDATA[On Radio-Frequency-Based Detection of High-Frequency Circulating Bearing Current Flow]]></title>
##     <authors><![CDATA[Muetze, A.;  Niskanen, V.;  Ahola, J.]]></authors>
##     <affiliations><![CDATA[Graz Univ. of Technol., Graz, Austria]]></affiliations>
##     <controlledterms>
##       <term><![CDATA[discharges (electric)]]></term>
##       <term><![CDATA[electric current measurement]]></term>
##       <term><![CDATA[electric fields]]></term>
##       <term><![CDATA[invertors]]></term>
##       <term><![CDATA[machine bearings]]></term>
##       <term><![CDATA[signal detection]]></term>
##       <term><![CDATA[variable speed drives]]></term>
##     </controlledterms>
##     <thesaurusterms>
##       <term><![CDATA[Current measurement]]></term>
##       <term><![CDATA[Discharges (electric)]]></term>
##       <term><![CDATA[Hafnium]]></term>
##       <term><![CDATA[Inverters]]></term>
##       <term><![CDATA[Radio frequency]]></term>
##       <term><![CDATA[Temperature measurement]]></term>
##       <term><![CDATA[Voltage measurement]]></term>
##     </thesaurusterms>
##     <pubtitle><![CDATA[Industry Applications, IEEE Transactions on]]></pubtitle>
##     <punumber><![CDATA[28]]></punumber>
##     <pubtype><![CDATA[Journals & Magazines]]></pubtype>
##     <publisher><![CDATA[IEEE]]></publisher>
##     <volume><![CDATA[50]]></volume>
##     <issue><![CDATA[4]]></issue>
##     <py><![CDATA[2014]]></py>
##     <spage><![CDATA[2592]]></spage>
##     <epage><![CDATA[2601]]></epage>
##     <abstract><![CDATA[The possibility of bearing damage caused by inverter-induced bearing currents in modern variable-speed drive systems has been well recognized today. Further research is needed to develop appropriate nonintrusive methods for detection and monitoring of such currents. A radio-frequency-based nondestructive method has been applied to detect discharge bearing currents. The method is understood to work on the energy that is radiated in the electric field during the bearing discharge event. We show that the method is also applicable to high-frequency circulating bearing currents that have so far been associated with ohmic bearing characteristics and no discharges occurring. The analysis and understanding of the applicability of the method to detect such currents also contributes to further understanding of the electric characteristics of the bearing, notably the moment the current conduction begins.]]></abstract>
##     <issn><![CDATA[0093-9994]]></issn>
##     <htmlFlag><![CDATA[1]]></htmlFlag>
##     <arnumber><![CDATA[6698379]]></arnumber>
##     <doi><![CDATA[10.1109/TIA.2013.2296626]]></doi>
##     <publicationId><![CDATA[6698379]]></publicationId>
##     <mdurl><![CDATA[http://ieeexplore.ieee.org/xpl/articleDetails.jsp?tp=&arnumber=6698379&contentType=Journals+%26+Magazines]]></mdurl>
##     <pdf><![CDATA[http://ieeexplore.ieee.org/stamp/stamp.jsp?arnumber=6698379]]></pdf>
##   </document>
##   <document>
##     <rank>1052</rank>
##     <title><![CDATA[Designing Energy Routing Protocol With Power Consumption Optimization in MANET]]></title>
##     <authors><![CDATA[Shivashankar;  Suresh, H.N.;  Varaprasad, G.;  Jayanthi, G.]]></authors>
##     <affiliations><![CDATA[Dr. Ambedkar Inst. of Technol., Bangalore, India]]></affiliations>
##     <controlledterms>
##       <term><![CDATA[minimax techniques]]></term>
##       <term><![CDATA[mobile ad hoc networks]]></term>
##       <term><![CDATA[routing protocols]]></term>
##       <term><![CDATA[telecommunication network topology]]></term>
##       <term><![CDATA[telecommunication power management]]></term>
##     </controlledterms>
##     <thesaurusterms>
##       <term><![CDATA[Batteries]]></term>
##       <term><![CDATA[Mobile ad hoc networks]]></term>
##       <term><![CDATA[Mobile nodes]]></term>
##       <term><![CDATA[Power consumption]]></term>
##       <term><![CDATA[Routing protocols]]></term>
##     </thesaurusterms>
##     <pubtitle><![CDATA[Emerging Topics in Computing, IEEE Transactions on]]></pubtitle>
##     <punumber><![CDATA[6245516]]></punumber>
##     <pubtype><![CDATA[Journals & Magazines]]></pubtype>
##     <publisher><![CDATA[IEEE]]></publisher>
##     <volume><![CDATA[2]]></volume>
##     <issue><![CDATA[2]]></issue>
##     <py><![CDATA[2014]]></py>
##     <spage><![CDATA[192]]></spage>
##     <epage><![CDATA[197]]></epage>
##     <abstract><![CDATA[As technology rapidly increases, diverse sensing and mobility capabilities have become readily available to devices and, consequently, mobile ad hoc networks (MANETs) are being deployed to perform a number of important tasks. In MANET, power aware is important challenge issue to improve the communication energy efficiency at individual nodes. We propose efficient power aware routing (EPAR), a new power aware routing protocol that increases the network lifetime of MANET. In contrast to conventional power aware algorithms, EPAR identifies the capacity of a node not just by its residual battery power, but also by the expected energy spent in reliably forwarding data packets over a specific link. Using a mini-max formulation, EPAR selects the path that has the largest packet capacity at the smallest residual packet transmission capacity. This protocol must be able to handle high mobility of the nodes that often cause changes in the network topology. This paper evaluates three ad hoc network routing protocols (EPAR, MTPR, and DSR) in different network scales, taking into consideration the power consumption. Indeed, our proposed scheme reduces for more than 20% the total energy consumption and decreases the mean delay, especially for high load networks, while achieving a good packet delivery ratio.]]></abstract>
##     <issn><![CDATA[2168-6750]]></issn>
##     <htmlFlag><![CDATA[1]]></htmlFlag>
##     <arnumber><![CDATA[6648690]]></arnumber>
##     <doi><![CDATA[10.1109/TETC.2013.2287177]]></doi>
##     <publicationId><![CDATA[6648690]]></publicationId>
##     <mdurl><![CDATA[http://ieeexplore.ieee.org/xpl/articleDetails.jsp?tp=&arnumber=6648690&contentType=Journals+%26+Magazines]]></mdurl>
##     <pdf><![CDATA[http://ieeexplore.ieee.org/stamp/stamp.jsp?arnumber=6648690]]></pdf>
##   </document>
##   <document>
##     <rank>1053</rank>
##     <title><![CDATA[Spectral Sensitivity of Simulated Photovoltaic Module Soiling for a Variety of Synthesized Soil Types]]></title>
##     <authors><![CDATA[Burton, P.D.;  King, B.H.]]></authors>
##     <affiliations><![CDATA[Sandia Nat. Labs., Albuquerque, NM, USA]]></affiliations>
##     <controlledterms>
##       <term><![CDATA[infrared spectra]]></term>
##       <term><![CDATA[iron compounds]]></term>
##       <term><![CDATA[minerals]]></term>
##       <term><![CDATA[soil]]></term>
##       <term><![CDATA[solar cells]]></term>
##       <term><![CDATA[ultraviolet spectra]]></term>
##       <term><![CDATA[visible spectra]]></term>
##     </controlledterms>
##     <thesaurusterms>
##       <term><![CDATA[Glass]]></term>
##       <term><![CDATA[Photovoltaic systems]]></term>
##       <term><![CDATA[Pigments]]></term>
##       <term><![CDATA[Soil]]></term>
##       <term><![CDATA[Soil measurements]]></term>
##     </thesaurusterms>
##     <pubtitle><![CDATA[Photovoltaics, IEEE Journal of]]></pubtitle>
##     <punumber><![CDATA[5503869]]></punumber>
##     <pubtype><![CDATA[Journals & Magazines]]></pubtype>
##     <publisher><![CDATA[IEEE]]></publisher>
##     <volume><![CDATA[4]]></volume>
##     <issue><![CDATA[3]]></issue>
##     <py><![CDATA[2014]]></py>
##     <spage><![CDATA[890]]></spage>
##     <epage><![CDATA[898]]></epage>
##     <abstract><![CDATA[The accumulation of soil on photovoltaic (PV) modules may introduce a spectral loss due to the color profile of the accumulated material. In order to compare the spectral and total losses experienced by a cell, soil analogs were formulated to contain common mineral pigments (Fe<sub>2</sub>O<sub>3</sub> and go&#x0308;thite) with previously developed &#x201C;standard grime&#x201D; mixtures. These mixtures simulated a wide range of desert soil colors and were applied to glass test coupons. The light transmission through the deposited film was evaluated by UV/vis/NIR spectroscopy and by placing the coupon over a test cell in a 1-sun simulator and quantum efficiency test stand. Distinct peaks in the 300-600-nm range were observed by UV/vis/NIR spectroscopy corresponding to the Fe<sub>2</sub>O<sub>3</sub> and go&#x0308;thite. Approximately analogous features were noted in the QE measurement. Overall comparisons were made by integrating the response of a soiled coupon relative to a clean reference. Soils rich in red pigments (Fe<sub>2</sub>O<sub>3</sub>) caused a greater integrated response than soils rich in yellow pigment (go&#x0308;thite). The yellow soils caused a greater attenuation in a specific region of the spectrum (300-450 nm), which may have significant implications to specific devices, such as multijunction and CdTe technologies.]]></abstract>
##     <issn><![CDATA[2156-3381]]></issn>
##     <htmlFlag><![CDATA[1]]></htmlFlag>
##     <arnumber><![CDATA[6732891]]></arnumber>
##     <doi><![CDATA[10.1109/JPHOTOV.2014.2301895]]></doi>
##     <publicationId><![CDATA[6732891]]></publicationId>
##     <mdurl><![CDATA[http://ieeexplore.ieee.org/xpl/articleDetails.jsp?tp=&arnumber=6732891&contentType=Journals+%26+Magazines]]></mdurl>
##     <pdf><![CDATA[http://ieeexplore.ieee.org/stamp/stamp.jsp?arnumber=6732891]]></pdf>
##   </document>
##   <document>
##     <rank>1054</rank>
##     <title><![CDATA[Outage Analysis for Relay-Aided Free-Space Optical Communications Over Turbulence Channels With Nonzero Boresight Pointing Errors]]></title>
##     <authors><![CDATA[Jin-Yuan Wang;  Jun-Bo Wang;  Ming Chen;  Yang Tang;  Ying Zhang]]></authors>
##     <affiliations><![CDATA[Nat. Mobile Commun. Res. Lab., Southeast Univ., Nanjing, China]]></affiliations>
##     <controlledterms>
##       <term><![CDATA[atmospheric turbulence]]></term>
##       <term><![CDATA[fading channels]]></term>
##       <term><![CDATA[optical links]]></term>
##       <term><![CDATA[relays]]></term>
##       <term><![CDATA[statistical analysis]]></term>
##     </controlledterms>
##     <thesaurusterms>
##       <term><![CDATA[Atmospheric modeling]]></term>
##       <term><![CDATA[Channel models]]></term>
##       <term><![CDATA[Communication systems]]></term>
##       <term><![CDATA[Fading]]></term>
##       <term><![CDATA[Probability]]></term>
##       <term><![CDATA[Relays]]></term>
##       <term><![CDATA[System performance]]></term>
##     </thesaurusterms>
##     <pubtitle><![CDATA[Photonics Journal, IEEE]]></pubtitle>
##     <punumber><![CDATA[4563994]]></punumber>
##     <pubtype><![CDATA[Journals & Magazines]]></pubtype>
##     <publisher><![CDATA[IEEE]]></publisher>
##     <volume><![CDATA[6]]></volume>
##     <issue><![CDATA[4]]></issue>
##     <py><![CDATA[2014]]></py>
##     <spage><![CDATA[1]]></spage>
##     <epage><![CDATA[15]]></epage>
##     <abstract><![CDATA[Free-space optical (FSO) communications has recently received significant attention and commercial interest for a variety of applications. In this paper, the outage probabilities for both the parallel and serial relay-aided FSO communications are analyzed. Initially, a new aggregated channel fading model is established, which considers atmospheric attenuation, M-distributed atmospheric turbulence, and nonzero boresight pointing errors. After that, the statistical characteristic of the aggregated channel is analyzed. Under the aggregated channel model, closed-form expressions of the outage probabilities for both the parallel and serial relay-aided FSO communications are derived, respectively. Numerical results show that the derived theoretical expressions are quite accurate and can provide sufficient precision for evaluating the outage performance, which is helpful to the design of future FSO communication systems.]]></abstract>
##     <issn><![CDATA[1943-0655]]></issn>
##     <htmlFlag><![CDATA[1]]></htmlFlag>
##     <arnumber><![CDATA[6842662]]></arnumber>
##     <doi><![CDATA[10.1109/JPHOT.2014.2332554]]></doi>
##     <publicationId><![CDATA[6842662]]></publicationId>
##     <mdurl><![CDATA[http://ieeexplore.ieee.org/xpl/articleDetails.jsp?tp=&arnumber=6842662&contentType=Journals+%26+Magazines]]></mdurl>
##     <pdf><![CDATA[http://ieeexplore.ieee.org/stamp/stamp.jsp?arnumber=6842662]]></pdf>
##   </document>
##   <document>
##     <rank>1055</rank>
##     <title><![CDATA[A Modified Empirical Mode Decomposition Algorithm in TDLAS for Gas Detection]]></title>
##     <authors><![CDATA[Yunxia Meng;  Tiegen Liu;  Kun Liu;  Junfeng Jiang;  Ranran Wang;  Tao Wang;  Haofeng Hu]]></authors>
##     <affiliations><![CDATA[Coll. of Precision Instrum. & Opto-Electron. Eng., Tianjin Univ., Tianjin, China]]></affiliations>
##     <controlledterms>
##       <term><![CDATA[carbon compounds]]></term>
##       <term><![CDATA[filtering theory]]></term>
##       <term><![CDATA[gas sensors]]></term>
##       <term><![CDATA[measurement by laser beam]]></term>
##       <term><![CDATA[optical correlation]]></term>
##       <term><![CDATA[optical harmonic generation]]></term>
##       <term><![CDATA[optical information processing]]></term>
##       <term><![CDATA[semiconductor lasers]]></term>
##       <term><![CDATA[signal reconstruction]]></term>
##       <term><![CDATA[spectrochemical analysis]]></term>
##     </controlledterms>
##     <thesaurusterms>
##       <term><![CDATA[Filtering algorithms]]></term>
##       <term><![CDATA[Filtering theory]]></term>
##       <term><![CDATA[Harmonic analysis]]></term>
##       <term><![CDATA[Power harmonic filters]]></term>
##       <term><![CDATA[Signal processing algorithms]]></term>
##       <term><![CDATA[Signal to noise ratio]]></term>
##       <term><![CDATA[Wiener filters]]></term>
##     </thesaurusterms>
##     <pubtitle><![CDATA[Photonics Journal, IEEE]]></pubtitle>
##     <punumber><![CDATA[4563994]]></punumber>
##     <pubtype><![CDATA[Journals & Magazines]]></pubtype>
##     <publisher><![CDATA[IEEE]]></publisher>
##     <volume><![CDATA[6]]></volume>
##     <issue><![CDATA[6]]></issue>
##     <py><![CDATA[2014]]></py>
##     <spage><![CDATA[1]]></spage>
##     <epage><![CDATA[7]]></epage>
##     <abstract><![CDATA[Based on the research of the traditional empirical mode decomposition (EMD) method, we proposed a modified EMD algorithm for the detected signal processing in tunable diode laser absorption spectroscopy. The modified EMD algorithm introduced Savitzky-Golay filtering and cross-correlation operation into the traditional EMD algorithm and reconstructed the signal by using the cross-correlation coefficients effectively. Based on the modified EMD algorithm in theory, the second harmonic component analysis was simulated by comparing with some other filtering algorithms. The experiments system was performed for carbon monoxide (CO) concentration detection. Comparing the sensing performances without and with using EMD-FCR and other filtered methods, the experimental results show that the signal-to-noise ratio of the system was significantly improved from 7.32 to 14.31 dB by EMD-FCR corresponding to one absorption line of CO at 1567.32 nm, leading to the minimum detection limit of 2 ppm. The accuracy and stability of the system are both improved by proposing the modified EMD algorithm.]]></abstract>
##     <issn><![CDATA[1943-0655]]></issn>
##     <htmlFlag><![CDATA[1]]></htmlFlag>
##     <arnumber><![CDATA[6951343]]></arnumber>
##     <doi><![CDATA[10.1109/JPHOT.2014.2368785]]></doi>
##     <publicationId><![CDATA[6951343]]></publicationId>
##     <mdurl><![CDATA[http://ieeexplore.ieee.org/xpl/articleDetails.jsp?tp=&arnumber=6951343&contentType=Journals+%26+Magazines]]></mdurl>
##     <pdf><![CDATA[http://ieeexplore.ieee.org/stamp/stamp.jsp?arnumber=6951343]]></pdf>
##   </document>
##   <document>
##     <rank>1056</rank>
##     <title><![CDATA[Photonic Generation of Radio-Frequency Waveforms Based on Dual-Parallel Mach&#x2013;Zehnder Modulator]]></title>
##     <authors><![CDATA[Wei Li;  Wen Ting Wang;  Ning Hua Zhu]]></authors>
##     <affiliations><![CDATA[State Key Lab. on Integrated Optoelectron., Inst. of Semicond., Beijing, China]]></affiliations>
##     <controlledterms>
##       <term><![CDATA[Mach-Zehnder interferometers]]></term>
##       <term><![CDATA[microwave photonics]]></term>
##       <term><![CDATA[optical harmonic generation]]></term>
##       <term><![CDATA[optical modulation]]></term>
##       <term><![CDATA[photodetectors]]></term>
##     </controlledterms>
##     <thesaurusterms>
##       <term><![CDATA[Amplitude modulation]]></term>
##       <term><![CDATA[Harmonic analysis]]></term>
##       <term><![CDATA[Optical fibers]]></term>
##       <term><![CDATA[Optical filters]]></term>
##       <term><![CDATA[Power harmonic filters]]></term>
##       <term><![CDATA[Radio frequency]]></term>
##     </thesaurusterms>
##     <pubtitle><![CDATA[Photonics Journal, IEEE]]></pubtitle>
##     <punumber><![CDATA[4563994]]></punumber>
##     <pubtype><![CDATA[Journals & Magazines]]></pubtype>
##     <publisher><![CDATA[IEEE]]></publisher>
##     <volume><![CDATA[6]]></volume>
##     <issue><![CDATA[3]]></issue>
##     <py><![CDATA[2014]]></py>
##     <spage><![CDATA[1]]></spage>
##     <epage><![CDATA[8]]></epage>
##     <abstract><![CDATA[We report a photonic approach to generating radio-frequency (RF) waveforms based on a dual-parallel Mach-Zehnder modulator (DPMZM) driven by a sinusoidal RF signal. One of the sub-MZMs (MZM1) of the DPMZM is driven by the sinusoidal RF signal, whereas the other sub-MZM (MZM2) has no driven signal. In this way, the powers of even- and odd-order sidebands can be separately controlled by adjusting the bias of the MZM1. The power ratio between the odd- (or even-) order sidebands is tunable by adjusting the RF power. Moreover, the power and the phase of the optical carrier can be independently tunable by adjusting the biases of the MZM2 and the parent MZM of the DPMZM, respectively. After detecting by a photodetector, an RF signal with controllable harmonics is generated. Thus, a desired RF waveform can be generated by separately manipulating its harmonics. The proposed method is theoretically analyzed. In addition, proof-of-concept experiments are carried out to generate triangular and rectangular waveforms with repetition rates of 5 and 10 GHz.]]></abstract>
##     <issn><![CDATA[1943-0655]]></issn>
##     <htmlFlag><![CDATA[1]]></htmlFlag>
##     <arnumber><![CDATA[6819001]]></arnumber>
##     <doi><![CDATA[10.1109/JPHOT.2014.2325863]]></doi>
##     <publicationId><![CDATA[6819001]]></publicationId>
##     <mdurl><![CDATA[http://ieeexplore.ieee.org/xpl/articleDetails.jsp?tp=&arnumber=6819001&contentType=Journals+%26+Magazines]]></mdurl>
##     <pdf><![CDATA[http://ieeexplore.ieee.org/stamp/stamp.jsp?arnumber=6819001]]></pdf>
##   </document>
##   <document>
##     <rank>1057</rank>
##     <title><![CDATA[A Locally Adaptive System for the Fusion of Objective Quality Measures]]></title>
##     <authors><![CDATA[Barri, A.;  Dooms, A.;  Jansen, B.;  Schelkens, P.]]></authors>
##     <affiliations><![CDATA[Dept. of Electron. & Inf., Vrije Univ. Brussel, Brussels, Belgium]]></affiliations>
##     <controlledterms>
##       <term><![CDATA[image processing]]></term>
##       <term><![CDATA[learning (artificial intelligence)]]></term>
##       <term><![CDATA[visual perception]]></term>
##     </controlledterms>
##     <thesaurusterms>
##       <term><![CDATA[Accuracy]]></term>
##       <term><![CDATA[Databases]]></term>
##       <term><![CDATA[Distortion]]></term>
##       <term><![CDATA[Distortion measurement]]></term>
##       <term><![CDATA[Quality assessment]]></term>
##       <term><![CDATA[Standards]]></term>
##       <term><![CDATA[Transform coding]]></term>
##     </thesaurusterms>
##     <pubtitle><![CDATA[Image Processing, IEEE Transactions on]]></pubtitle>
##     <punumber><![CDATA[83]]></punumber>
##     <pubtype><![CDATA[Journals & Magazines]]></pubtype>
##     <publisher><![CDATA[IEEE]]></publisher>
##     <volume><![CDATA[23]]></volume>
##     <issue><![CDATA[6]]></issue>
##     <py><![CDATA[2014]]></py>
##     <spage><![CDATA[2446]]></spage>
##     <epage><![CDATA[2458]]></epage>
##     <abstract><![CDATA[Objective measures to automatically predict the perceptual quality of images or videos can reduce the time and cost requirements of end-to-end quality monitoring. For reliable quality predictions, these objective quality measures need to respond consistently with the behavior of the human visual system (HVS). In practice, many important HVS mechanisms are too complex to be modeled directly. Instead, they can be mimicked by machine learning systems, trained on subjective quality assessment databases, and applied on predefined objective quality measures for specific content or distortion classes. On the downside, machine learning systems are often difficult to interpret and may even contradict the input objective quality measures, leading to unreliable quality predictions. To address this problem, we developed an interpretable machine learning system for objective quality assessment, namely the locally adaptive fusion (LAF). This paper describes the LAF system and compares its performance with traditional machine learning. As it turns out, the LAF system is more consistent with the input measures and can better handle heteroscedastic training data.]]></abstract>
##     <issn><![CDATA[1057-7149]]></issn>
##     <htmlFlag><![CDATA[1]]></htmlFlag>
##     <arnumber><![CDATA[6786302]]></arnumber>
##     <doi><![CDATA[10.1109/TIP.2014.2316379]]></doi>
##     <publicationId><![CDATA[6786302]]></publicationId>
##     <mdurl><![CDATA[http://ieeexplore.ieee.org/xpl/articleDetails.jsp?tp=&arnumber=6786302&contentType=Journals+%26+Magazines]]></mdurl>
##     <pdf><![CDATA[http://ieeexplore.ieee.org/stamp/stamp.jsp?arnumber=6786302]]></pdf>
##   </document>
##   <document>
##     <rank>1058</rank>
##     <title><![CDATA[Enhancement of Hole Confinement by Monolayer Insertion in Asymmetric Quantum-Barrier UVB Light Emitting Diodes]]></title>
##     <authors><![CDATA[Janjua, B.;  Ng, T.K.;  Alyamani, A.Y.;  El-Desouki, M.M.;  Ooi, B.S.]]></authors>
##     <affiliations><![CDATA[Comput., Electr. & Math. Sci. & Eng. (CEMSE) Div., King Abdullah Univ. of Sci. & Technol. (KAUST), Thuwal, Saudi Arabia]]></affiliations>
##     <controlledterms>
##       <term><![CDATA[Auger effect]]></term>
##       <term><![CDATA[III-V semiconductors]]></term>
##       <term><![CDATA[aluminium compounds]]></term>
##       <term><![CDATA[electron-hole recombination]]></term>
##       <term><![CDATA[gallium compounds]]></term>
##       <term><![CDATA[k.p calculations]]></term>
##       <term><![CDATA[light emitting diodes]]></term>
##       <term><![CDATA[monolayers]]></term>
##       <term><![CDATA[semiconductor quantum wells]]></term>
##       <term><![CDATA[wave functions]]></term>
##       <term><![CDATA[wide band gap semiconductors]]></term>
##     </controlledterms>
##     <thesaurusterms>
##       <term><![CDATA[Aluminum gallium nitride]]></term>
##       <term><![CDATA[Charge carrier density]]></term>
##       <term><![CDATA[Charge carrier processes]]></term>
##       <term><![CDATA[III-V semiconductor materials]]></term>
##       <term><![CDATA[Light emitting diodes]]></term>
##       <term><![CDATA[Radiative recombination]]></term>
##     </thesaurusterms>
##     <pubtitle><![CDATA[Photonics Journal, IEEE]]></pubtitle>
##     <punumber><![CDATA[4563994]]></punumber>
##     <pubtype><![CDATA[Journals & Magazines]]></pubtype>
##     <publisher><![CDATA[IEEE]]></publisher>
##     <volume><![CDATA[6]]></volume>
##     <issue><![CDATA[2]]></issue>
##     <py><![CDATA[2014]]></py>
##     <spage><![CDATA[1]]></spage>
##     <epage><![CDATA[9]]></epage>
##     <abstract><![CDATA[We study the enhanced hole confinement by having a large bandgap AlGaN monolayer insertion (MLI) between the quantum well (QW) and the quantum barrier (QB). The numerical analysis examines the energy band alignment diagrams, using a self-consistent 6 &#x00D7; 6 k &#x00B7;p method and, considering carrier distribution, recombination rates (Shockley-Reed-Hall, Auger, and radiative recombination rates), under equilibrium and forward bias conditions. The active region is based on Al<sub>a</sub>Ga<sub>1-a</sub>N (barrier)/Al<sub>b</sub>Ga<sub>1-b</sub>N (MLI)/Al<sub>c</sub>Ga<sub>1-c</sub>N (well)/Al<sub>d</sub>Ga<sub>1-d</sub>N (barrier), where b d a c. A large bandgap AlbGa1 - bN mono layer, inserted between the QW and QB, was found to be effective in providing stronger hole confinement. With the proposed band engineering scheme, an increase of more than 30% in spatial overlap of carrier wavefunction was obtained, with a considerable increase in carrier density and direct radiative recombination rates. The single-QW-based UV-LED was designed to emit at 280 nm, which is an effective wavelength for water disinfection.]]></abstract>
##     <issn><![CDATA[1943-0655]]></issn>
##     <htmlFlag><![CDATA[1]]></htmlFlag>
##     <arnumber><![CDATA[6758387]]></arnumber>
##     <doi><![CDATA[10.1109/JPHOT.2014.2310199]]></doi>
##     <publicationId><![CDATA[6758387]]></publicationId>
##     <mdurl><![CDATA[http://ieeexplore.ieee.org/xpl/articleDetails.jsp?tp=&arnumber=6758387&contentType=Journals+%26+Magazines]]></mdurl>
##     <pdf><![CDATA[http://ieeexplore.ieee.org/stamp/stamp.jsp?arnumber=6758387]]></pdf>
##   </document>
##   <document>
##     <rank>1059</rank>
##     <title><![CDATA[Gold Nanoparticles-Modified Tapered Fiber Nanoprobe for Remote SERS Detection]]></title>
##     <authors><![CDATA[Zhenyi Chen;  Zhangmin Dai;  Na Chen;  Shupeng Liu;  Fufei Pang;  Bo Lu;  Tingyun Wang]]></authors>
##     <affiliations><![CDATA[Key Lab. of Specialty Fiber Opt. & Opt. Access Networks, Shanghai Univ., Shanghai, China]]></affiliations>
##     <controlledterms>
##       <term><![CDATA[chemical sensors]]></term>
##       <term><![CDATA[dyes]]></term>
##       <term><![CDATA[etching]]></term>
##       <term><![CDATA[fibre optic sensors]]></term>
##       <term><![CDATA[gold]]></term>
##       <term><![CDATA[microwave materials processing]]></term>
##       <term><![CDATA[nanofabrication]]></term>
##       <term><![CDATA[nanoparticles]]></term>
##       <term><![CDATA[nanophotonics]]></term>
##       <term><![CDATA[nanosensors]]></term>
##       <term><![CDATA[self-assembly]]></term>
##       <term><![CDATA[surface enhanced Raman scattering]]></term>
##     </controlledterms>
##     <thesaurusterms>
##       <term><![CDATA[Gold]]></term>
##       <term><![CDATA[Nanoparticles]]></term>
##       <term><![CDATA[Optical fiber sensors]]></term>
##       <term><![CDATA[Optical fibers]]></term>
##       <term><![CDATA[Raman scattering]]></term>
##       <term><![CDATA[Surface treatment]]></term>
##     </thesaurusterms>
##     <pubtitle><![CDATA[Photonics Technology Letters, IEEE]]></pubtitle>
##     <punumber><![CDATA[68]]></punumber>
##     <pubtype><![CDATA[Journals & Magazines]]></pubtype>
##     <publisher><![CDATA[IEEE]]></publisher>
##     <volume><![CDATA[26]]></volume>
##     <issue><![CDATA[8]]></issue>
##     <py><![CDATA[2014]]></py>
##     <spage><![CDATA[777]]></spage>
##     <epage><![CDATA[780]]></epage>
##     <abstract><![CDATA[This letter presents a surface-enhanced Raman scattering (SERS) nanoprobe based on gold nanoparticles-modified tapered optical fiber and demonstrates its ability to perform remote Raman detection. The nanoscale tapered fiber with the tip size of 40.7 nm was made by heated pulling and chemical etching methods. The gold nanoparticles, prepared beforehand by the Frens method with a microwave heating process, were deposited on the tapered surface of the nanoprobe with the electrostatic self-assembly technology. Raman spectra of Rhodamine 6G (R6G) molecules were measured, using this SERS nanoprobe in an optrode remote detection mode. Considerably high signal-to-noise ratio and high sensitivity were achieved. The detection limit for R6G aqueous solution reaches 10<sup>-8</sup> mol/L.]]></abstract>
##     <issn><![CDATA[1041-1135]]></issn>
##     <htmlFlag><![CDATA[1]]></htmlFlag>
##     <arnumber><![CDATA[6739998]]></arnumber>
##     <doi><![CDATA[10.1109/LPT.2014.2306134]]></doi>
##     <publicationId><![CDATA[6739998]]></publicationId>
##     <mdurl><![CDATA[http://ieeexplore.ieee.org/xpl/articleDetails.jsp?tp=&arnumber=6739998&contentType=Journals+%26+Magazines]]></mdurl>
##     <pdf><![CDATA[http://ieeexplore.ieee.org/stamp/stamp.jsp?arnumber=6739998]]></pdf>
##   </document>
##   <document>
##     <rank>1060</rank>
##     <title><![CDATA[Continuous Electrowetting of Non-toxic Liquid Metal for RF Applications]]></title>
##     <authors><![CDATA[Gough, R.C.;  Morishita, A.M.;  Dang, J.H.;  Wenqi Hu;  Shiroma, W.A.;  Ohta, A.T.]]></authors>
##     <affiliations><![CDATA[Dept. of Electr. Eng., Univ. of Hawai'i at Manoa, Honolulu, HI, USA]]></affiliations>
##     <controlledterms>
##       <term><![CDATA[fluidics]]></term>
##       <term><![CDATA[gallium]]></term>
##       <term><![CDATA[liquid metals]]></term>
##       <term><![CDATA[slot antennas]]></term>
##       <term><![CDATA[surface energy]]></term>
##       <term><![CDATA[surface tension]]></term>
##       <term><![CDATA[wetting]]></term>
##     </controlledterms>
##     <thesaurusterms>
##       <term><![CDATA[Actuators]]></term>
##       <term><![CDATA[Electrowetting]]></term>
##       <term><![CDATA[Fluids]]></term>
##       <term><![CDATA[Liquid metal]]></term>
##       <term><![CDATA[Radio frequency]]></term>
##       <term><![CDATA[Surface tension]]></term>
##     </thesaurusterms>
##     <pubtitle><![CDATA[Access, IEEE]]></pubtitle>
##     <punumber><![CDATA[6287639]]></punumber>
##     <pubtype><![CDATA[Journals & Magazines]]></pubtype>
##     <publisher><![CDATA[IEEE]]></publisher>
##     <volume><![CDATA[2]]></volume>
##     <py><![CDATA[2014]]></py>
##     <spage><![CDATA[874]]></spage>
##     <epage><![CDATA[882]]></epage>
##     <abstract><![CDATA[Continuous electrowetting (CEW) is demonstrated to be an effective actuation mechanism for reconfigurable radio frequency (RF) devices that use non-toxic liquid-metal tuning elements. Previous research has shown CEW is an efficient means of electrically inducing motion in a liquid-metal slug, but precise control of the slug's position within fluidic channels has not been demonstrated. Here, the precise positioning of liquid-metal slugs is achieved using CEW actuation in conjunction with channels designed to minimize the liquid-metal surface energy at discrete locations. This approach leverages the high surface tension of liquid metal to control its resting position with submillimeter accuracy. The CEW actuation and fluidic channel design were optimized to create reconfigurable RF devices. In addition, solutions for the reliable actuation of a gallium-based, non-toxic liquid-metal alloy (Galinstan) are presented that mitigate the tendency of the alloy to form a surface oxide layer capable of wetting to the channel walls, inhibiting motion. A reconfigurable slot antenna utilizing these techniques to achieve a 15.2% tunable frequency bandwidth is demonstrated.]]></abstract>
##     <issn><![CDATA[2169-3536]]></issn>
##     <htmlFlag><![CDATA[1]]></htmlFlag>
##     <arnumber><![CDATA[6881663]]></arnumber>
##     <doi><![CDATA[10.1109/ACCESS.2014.2350531]]></doi>
##     <publicationId><![CDATA[6881663]]></publicationId>
##     <mdurl><![CDATA[http://ieeexplore.ieee.org/xpl/articleDetails.jsp?tp=&arnumber=6881663&contentType=Journals+%26+Magazines]]></mdurl>
##     <pdf><![CDATA[http://ieeexplore.ieee.org/stamp/stamp.jsp?arnumber=6881663]]></pdf>
##   </document>
##   <document>
##     <rank>1061</rank>
##     <title><![CDATA[Alternative Implementation of Simplified Brillouin Optical Correlation-Domain Reflectometry]]></title>
##     <authors><![CDATA[Hayashi, N.;  Mizuno, Y.;  Nakamura, K.]]></authors>
##     <affiliations><![CDATA[Precision & Intell. Lab., Tokyo Inst. of Technol., Yokohama, Japan]]></affiliations>
##     <controlledterms>
##       <term><![CDATA[Brillouin spectra]]></term>
##       <term><![CDATA[light reflection]]></term>
##       <term><![CDATA[optical circulators]]></term>
##       <term><![CDATA[optical correlation]]></term>
##       <term><![CDATA[optical fibres]]></term>
##       <term><![CDATA[reflectometry]]></term>
##     </controlledterms>
##     <thesaurusterms>
##       <term><![CDATA[Correlation]]></term>
##       <term><![CDATA[Optical fiber sensors]]></term>
##       <term><![CDATA[Optical fibers]]></term>
##       <term><![CDATA[Optical reflection]]></term>
##       <term><![CDATA[Scattering]]></term>
##       <term><![CDATA[Spatial resolution]]></term>
##     </thesaurusterms>
##     <pubtitle><![CDATA[Photonics Journal, IEEE]]></pubtitle>
##     <punumber><![CDATA[4563994]]></punumber>
##     <pubtype><![CDATA[Journals & Magazines]]></pubtype>
##     <publisher><![CDATA[IEEE]]></publisher>
##     <volume><![CDATA[6]]></volume>
##     <issue><![CDATA[6]]></issue>
##     <py><![CDATA[2014]]></py>
##     <spage><![CDATA[1]]></spage>
##     <epage><![CDATA[8]]></epage>
##     <abstract><![CDATA[We developed an alternative configuration of simplified Brillouin optical correlation-domain reflectometry, which can overcome the drawbacks of the original configuration. This system uses, as reference light, the light that is Fresnel reflected at a partial reflection point artificially produced near an optical circulator. We show that the influence of the zeroth correlation peak fixed at the partial reflection point can be suppressed by replacing the nearby fibers with other fibers having different Brillouin frequency shift values (here, multimode fibers are used). Finally, we demonstrate a distributed measurement for detecting a 1.46-m-long strained section with a high signal-to-noise (SN) ratio.]]></abstract>
##     <issn><![CDATA[1943-0655]]></issn>
##     <htmlFlag><![CDATA[1]]></htmlFlag>
##     <arnumber><![CDATA[6955891]]></arnumber>
##     <doi><![CDATA[10.1109/JPHOT.2014.2366107]]></doi>
##     <publicationId><![CDATA[6955891]]></publicationId>
##     <mdurl><![CDATA[http://ieeexplore.ieee.org/xpl/articleDetails.jsp?tp=&arnumber=6955891&contentType=Journals+%26+Magazines]]></mdurl>
##     <pdf><![CDATA[http://ieeexplore.ieee.org/stamp/stamp.jsp?arnumber=6955891]]></pdf>
##   </document>
##   <document>
##     <rank>1062</rank>
##     <title><![CDATA[Toward Environmentally Sustainable Mobile Computing Through an Economic Framework]]></title>
##     <authors><![CDATA[Joseph, S.;  Namboodiri, V.;  Dev, V.C.]]></authors>
##     <affiliations><![CDATA[Dept. of Arts, Sci., & Bus., Kansas State Univ., Salina, KS, USA]]></affiliations>
##     <controlledterms>
##       <term><![CDATA[DP industry]]></term>
##       <term><![CDATA[energy conservation]]></term>
##       <term><![CDATA[environmental economics]]></term>
##       <term><![CDATA[mobile computing]]></term>
##       <term><![CDATA[product life cycle management]]></term>
##       <term><![CDATA[sustainable development]]></term>
##     </controlledterms>
##     <thesaurusterms>
##       <term><![CDATA[Economics]]></term>
##       <term><![CDATA[Energy consumption]]></term>
##       <term><![CDATA[Manufacturing]]></term>
##       <term><![CDATA[Mobile communication]]></term>
##       <term><![CDATA[Mobile computing]]></term>
##       <term><![CDATA[Mobile handsets]]></term>
##       <term><![CDATA[Sustainable development]]></term>
##     </thesaurusterms>
##     <pubtitle><![CDATA[Emerging Topics in Computing, IEEE Transactions on]]></pubtitle>
##     <punumber><![CDATA[6245516]]></punumber>
##     <pubtype><![CDATA[Journals & Magazines]]></pubtype>
##     <publisher><![CDATA[IEEE]]></publisher>
##     <volume><![CDATA[2]]></volume>
##     <issue><![CDATA[2]]></issue>
##     <py><![CDATA[2014]]></py>
##     <spage><![CDATA[212]]></spage>
##     <epage><![CDATA[224]]></epage>
##     <abstract><![CDATA[Amid the plethora of initiatives and research endeavors targeting the minimization of power and energy consumption of information and communication technologies, what has been largely missing is an effort to reduce the energy consumption and electronic waste generated by the rapidly growing segment of mobile computing and communication devices. Prior work with energy efficiency in mobile devices has primarily focused on the goal of maximizing battery life of these devices and not on the broader concept of environmentally sustainable mobile computing. This paper provides an overview of the concept of environmentally sustainable mobile computing and identifies reduction in manufacturing energy costs and electronic waste generated as two important outcomes that can be achieved by increasing device lifespan. Increased device lifespans, however, are possible only if the underlying market forces support such a paradigm shift. This paper develops an analytical economic framework as it applies to mobile phones by analyzing a market scenario of two firms competing under a differentiated Bertrand duopoly model. The framework and its analysis helps verify intuitions about the reasons that affect a firm's decision to offer an environmentally sustainable choice for consumers and considers the feasibility, possible benefits, and challenges in increasing device lifespan, including technical challenges. The results of this paper also provide guidance on the relative impacts of various factors involved on device lifespan such as user-experience, subsidies, and differences in underlying costs to providers.]]></abstract>
##     <issn><![CDATA[2168-6750]]></issn>
##     <htmlFlag><![CDATA[1]]></htmlFlag>
##     <arnumber><![CDATA[6698386]]></arnumber>
##     <doi><![CDATA[10.1109/TETC.2013.2296521]]></doi>
##     <publicationId><![CDATA[6698386]]></publicationId>
##     <mdurl><![CDATA[http://ieeexplore.ieee.org/xpl/articleDetails.jsp?tp=&arnumber=6698386&contentType=Journals+%26+Magazines]]></mdurl>
##     <pdf><![CDATA[http://ieeexplore.ieee.org/stamp/stamp.jsp?arnumber=6698386]]></pdf>
##   </document>
##   <document>
##     <rank>1063</rank>
##     <title><![CDATA[Enhanced Natural Visual Perception for Augmented Reality-Workstations by Simulation of Perspective]]></title>
##     <authors><![CDATA[Radkowski, R.;  Oliver, J.]]></authors>
##     <affiliations><![CDATA[Virtual Reality Applic. Center, Iowa State Univ., Ames, IA, USA]]></affiliations>
##     <controlledterms>
##       <term><![CDATA[augmented reality]]></term>
##       <term><![CDATA[human computer interaction]]></term>
##       <term><![CDATA[workstations]]></term>
##     </controlledterms>
##     <thesaurusterms>
##       <term><![CDATA[Cameras]]></term>
##       <term><![CDATA[Face]]></term>
##       <term><![CDATA[Monitoring]]></term>
##       <term><![CDATA[Rendering (computer graphics)]]></term>
##       <term><![CDATA[Solid modeling]]></term>
##       <term><![CDATA[Three-dimensional displays]]></term>
##     </thesaurusterms>
##     <pubtitle><![CDATA[Display Technology, Journal of]]></pubtitle>
##     <punumber><![CDATA[9425]]></punumber>
##     <pubtype><![CDATA[Journals & Magazines]]></pubtype>
##     <publisher><![CDATA[IEEE]]></publisher>
##     <volume><![CDATA[10]]></volume>
##     <issue><![CDATA[5]]></issue>
##     <py><![CDATA[2014]]></py>
##     <spage><![CDATA[333]]></spage>
##     <epage><![CDATA[344]]></epage>
##     <abstract><![CDATA[A novel method is presented for enhancing the natural visual perception afforded by augmented reality (AR) workstations. The approach incorporates a method for simulating perspective viewing using a monitor-based AR workstation that acts as a window to the physical workspace in front of it. Although similar AR workstations are often used in industry, they do not provide natural visual perception. By incorporating user head tracking and a spherical mirror, the method proposed enhances visual perception by simulating the depth cue of perspective viewing. The method interactively adjusts the viewing parameters to provide accurate perspective viewing of both the video stream (representing the physical environment) as well as the virtual objects superimposed within it. Hence, the user gains the impression of 3D viewing of the entire AR scene when moving ones head in front of the monitor. This paper describes the hardware setup and the method for perspective simulation within an AR software application. Finally, an experimental evaluation of the accuracy of the system is described to demonstrate the feasibility of the approach.]]></abstract>
##     <issn><![CDATA[1551-319X]]></issn>
##     <arnumber><![CDATA[6720163]]></arnumber>
##     <doi><![CDATA[10.1109/JDT.2014.2299978]]></doi>
##     <publicationId><![CDATA[6720163]]></publicationId>
##     <mdurl><![CDATA[http://ieeexplore.ieee.org/xpl/articleDetails.jsp?tp=&arnumber=6720163&contentType=Journals+%26+Magazines]]></mdurl>
##     <pdf><![CDATA[http://ieeexplore.ieee.org/stamp/stamp.jsp?arnumber=6720163]]></pdf>
##   </document>
##   <document>
##     <rank>1064</rank>
##     <title><![CDATA[Near Real-Time Computer Assisted Surgery for Brain Shift Correction Using Biomechanical Models]]></title>
##     <authors><![CDATA[Kay Sun;  Pheiffer, T.S.;  Simpson, A.L.;  Weis, J.A.;  Thompson, R.C.;  Miga, M.I.]]></authors>
##     <affiliations><![CDATA[Dept. of Biomed. Eng., Vanderbilt Univ., Nashville, TN, USA]]></affiliations>
##     <controlledterms>
##       <term><![CDATA[biomechanics]]></term>
##       <term><![CDATA[biomedical MRI]]></term>
##       <term><![CDATA[brain models]]></term>
##       <term><![CDATA[data acquisition]]></term>
##       <term><![CDATA[medical image processing]]></term>
##       <term><![CDATA[neurophysiology]]></term>
##       <term><![CDATA[real-time systems]]></term>
##       <term><![CDATA[surgery]]></term>
##     </controlledterms>
##     <thesaurusterms>
##       <term><![CDATA[Biological system modeling]]></term>
##       <term><![CDATA[Boundary conditions]]></term>
##       <term><![CDATA[Brain models]]></term>
##       <term><![CDATA[Surgery]]></term>
##       <term><![CDATA[Tumors]]></term>
##     </thesaurusterms>
##     <pubtitle><![CDATA[Translational Engineering in Health and Medicine, IEEE Journal of]]></pubtitle>
##     <punumber><![CDATA[6221039]]></punumber>
##     <pubtype><![CDATA[Journals & Magazines]]></pubtype>
##     <publisher><![CDATA[IEEE]]></publisher>
##     <volume><![CDATA[2]]></volume>
##     <py><![CDATA[2014]]></py>
##     <spage><![CDATA[1]]></spage>
##     <epage><![CDATA[13]]></epage>
##     <abstract><![CDATA[Conventional image-guided neurosurgery relies on preoperative images to provide surgical navigational information and visualization. However, these images are no longer accurate once the skull has been opened and brain shift occurs. To account for changes in the shape of the brain caused by mechanical (e.g., gravity-induced deformations) and physiological effects (e.g., hyperosmotic drug-induced shrinking, or edema-induced swelling), updated images of the brain must be provided to the neuronavigation system in a timely manner for practical use in the operating room. In this paper, a novel preoperative and intraoperative computational processing pipeline for near real-time brain shift correction in the operating room was developed to automate and simplify the processing steps. Preoperatively, a computer model of the patient's brain with a subsequent atlas of potential deformations due to surgery is generated from diagnostic image volumes. In the case of interim gross changes between diagnosis, and surgery when reimaging is necessary, our preoperative pipeline can be generated within one day of surgery. Intraoperatively, sparse data measuring the cortical brain surface is collected using an optically tracked portable laser range scanner. These data are then used to guide an inverse modeling framework whereby full volumetric brain deformations are reconstructed from precomputed atlas solutions to rapidly match intraoperative cortical surface shift measurements. Once complete, the volumetric displacement field is used to update, i.e., deform, preoperative brain images to their intraoperative shifted state. In this paper, five surgical cases were analyzed with respect to the computational pipeline and workflow timing. With respect to postcortical surface data acquisition, the approximate execution time was 4.5 min. The total update process which included positioning the scanner, data acquisition, inverse model processing, and image deforming was ~ 11-13 min. In additio- , easily implemented hardware, software, and workflow processes were identified for improved performance in the near future.]]></abstract>
##     <issn><![CDATA[2168-2372]]></issn>
##     <htmlFlag><![CDATA[1]]></htmlFlag>
##     <arnumber><![CDATA[6823628]]></arnumber>
##     <doi><![CDATA[10.1109/JTEHM.2014.2327628]]></doi>
##     <publicationId><![CDATA[6823628]]></publicationId>
##     <mdurl><![CDATA[http://ieeexplore.ieee.org/xpl/articleDetails.jsp?tp=&arnumber=6823628&contentType=Journals+%26+Magazines]]></mdurl>
##     <pdf><![CDATA[http://ieeexplore.ieee.org/stamp/stamp.jsp?arnumber=6823628]]></pdf>
##   </document>
##   <document>
##     <rank>1065</rank>
##     <title><![CDATA[Nondestructive Characterization of PEC-Backed Materials Using the Combined Measurements of a Rectangular Waveguide and Coaxial Probe]]></title>
##     <authors><![CDATA[Hyde, M.W.;  Bogle, A.E.;  Havrilla, M.J.]]></authors>
##     <affiliations><![CDATA[Dept. of Electr. & Comput. Eng., Air Force Inst. of Technol., Wright-Patterson AFB, OH, USA]]></affiliations>
##     <controlledterms>
##       <term><![CDATA[coaxial waveguides]]></term>
##       <term><![CDATA[magnetic permeability]]></term>
##       <term><![CDATA[method of moments]]></term>
##       <term><![CDATA[nondestructive testing]]></term>
##       <term><![CDATA[permeability]]></term>
##       <term><![CDATA[permittivity]]></term>
##       <term><![CDATA[rectangular waveguides]]></term>
##     </controlledterms>
##     <thesaurusterms>
##       <term><![CDATA[Integral equations]]></term>
##       <term><![CDATA[Measurement uncertainty]]></term>
##       <term><![CDATA[Method of moments]]></term>
##       <term><![CDATA[Microwave measurement]]></term>
##       <term><![CDATA[Permeability measurement]]></term>
##       <term><![CDATA[Permittivity measurement]]></term>
##       <term><![CDATA[Rectangular waveguides]]></term>
##       <term><![CDATA[Uncertainty]]></term>
##     </thesaurusterms>
##     <pubtitle><![CDATA[Microwave and Wireless Components Letters, IEEE]]></pubtitle>
##     <punumber><![CDATA[7260]]></punumber>
##     <pubtype><![CDATA[Journals & Magazines]]></pubtype>
##     <publisher><![CDATA[IEEE]]></publisher>
##     <volume><![CDATA[24]]></volume>
##     <issue><![CDATA[11]]></issue>
##     <py><![CDATA[2014]]></py>
##     <spage><![CDATA[808]]></spage>
##     <epage><![CDATA[810]]></epage>
##     <abstract><![CDATA[A novel one-port probe technique, which combines the measurements of a rectangular waveguide and coaxial probe to nondestructively yield the permittivity and permeability of a PEC-backed material, is presented. A brief description of the derivation of the theoretical probe reflection coefficients, necessary for permittivity and permeability extraction via numerical inversion, is provided. Experimental characterization results of a PEC-backed magnetic material are presented to validate the proposed approach. Error analysis is also undertaken to quantify the new technique's sensitivity to common experimental errors.]]></abstract>
##     <issn><![CDATA[1531-1309]]></issn>
##     <htmlFlag><![CDATA[1]]></htmlFlag>
##     <arnumber><![CDATA[6880416]]></arnumber>
##     <doi><![CDATA[10.1109/LMWC.2014.2348496]]></doi>
##     <publicationId><![CDATA[6880416]]></publicationId>
##     <mdurl><![CDATA[http://ieeexplore.ieee.org/xpl/articleDetails.jsp?tp=&arnumber=6880416&contentType=Journals+%26+Magazines]]></mdurl>
##     <pdf><![CDATA[http://ieeexplore.ieee.org/stamp/stamp.jsp?arnumber=6880416]]></pdf>
##   </document>
##   <document>
##     <rank>1066</rank>
##     <title><![CDATA[Tunable Group Delay of the Optical Pulse Reflection From Fabry&#x2013;Perot Cavity With the Insertion of Graphene Sheets]]></title>
##     <authors><![CDATA[Leyong Jiang;  Xiaoyu Dai;  Yuanjiang Xiang;  Shuangchun Wen]]></authors>
##     <affiliations><![CDATA[Key Lab. of Optoelectron. Devices & Syst. of the Minist. of Educ. & Guangdong Province, Shenzhen Univ., Shenzhen, China]]></affiliations>
##     <controlledterms>
##       <term><![CDATA[Fermi level]]></term>
##       <term><![CDATA[carrier density]]></term>
##       <term><![CDATA[delays]]></term>
##       <term><![CDATA[graphene]]></term>
##       <term><![CDATA[laser beams]]></term>
##       <term><![CDATA[laser tuning]]></term>
##       <term><![CDATA[mirrors]]></term>
##       <term><![CDATA[nanophotonics]]></term>
##     </controlledterms>
##     <thesaurusterms>
##       <term><![CDATA[Cavity resonators]]></term>
##       <term><![CDATA[Conductivity]]></term>
##       <term><![CDATA[Delays]]></term>
##       <term><![CDATA[Graphene]]></term>
##       <term><![CDATA[Optical imaging]]></term>
##       <term><![CDATA[Optical pulses]]></term>
##       <term><![CDATA[Optical reflection]]></term>
##     </thesaurusterms>
##     <pubtitle><![CDATA[Photonics Journal, IEEE]]></pubtitle>
##     <punumber><![CDATA[4563994]]></punumber>
##     <pubtype><![CDATA[Journals & Magazines]]></pubtype>
##     <publisher><![CDATA[IEEE]]></publisher>
##     <volume><![CDATA[6]]></volume>
##     <issue><![CDATA[6]]></issue>
##     <py><![CDATA[2014]]></py>
##     <spage><![CDATA[1]]></spage>
##     <epage><![CDATA[9]]></epage>
##     <abstract><![CDATA[We have theoretically investigated the group delay of the TE-polarized beam reflected from a Fabry-Perot cavity with the insertion of the graphene sheets in the near-infrared band. It is shown that even a single-layer graphene allows for notable variation of group delay. Group delay can be enlarged negatively and can be switched from positive to negative, or vice versa. Importantly, the group delay depends on the Fermi energy of the graphene sheets, and thus, it can be actively controlled through electrical or chemical modification of the charge carrier density of the graphene. Furthermore, the influences of the position of graphene in the Fabry-Perot cavity, the mirror transmittance, and the number of graphene layers on group delay are clarified.]]></abstract>
##     <issn><![CDATA[1943-0655]]></issn>
##     <htmlFlag><![CDATA[1]]></htmlFlag>
##     <arnumber><![CDATA[6965593]]></arnumber>
##     <doi><![CDATA[10.1109/JPHOT.2014.2368783]]></doi>
##     <publicationId><![CDATA[6965593]]></publicationId>
##     <mdurl><![CDATA[http://ieeexplore.ieee.org/xpl/articleDetails.jsp?tp=&arnumber=6965593&contentType=Journals+%26+Magazines]]></mdurl>
##     <pdf><![CDATA[http://ieeexplore.ieee.org/stamp/stamp.jsp?arnumber=6965593]]></pdf>
##   </document>
##   <document>
##     <rank>1067</rank>
##     <title><![CDATA[Determination of Quantum Efficiency in In<inline-formula> <img src="/images/tex/21810.gif" alt="_{\bf {0.53}}"> </inline-formula>Ga<inline-formula> <img src="/images/tex/21811.gif" alt="_{\bf{0.47}}"> </inline-formula>As-InP-Based APDs]]></title>
##     <authors><![CDATA[Clark, W.R.;  Vaccaro, K.;  Waters, W.D.;  Gribbon, C.L.;  Krejca, B.D.]]></authors>
##     <affiliations><![CDATA[OptoGration, Inc., Wilmington, MA, USA]]></affiliations>
##     <controlledterms>
##       <term><![CDATA[III-V semiconductors]]></term>
##       <term><![CDATA[avalanche photodiodes]]></term>
##       <term><![CDATA[impact ionisation]]></term>
##       <term><![CDATA[indium compounds]]></term>
##       <term><![CDATA[optical design techniques]]></term>
##       <term><![CDATA[optical materials]]></term>
##       <term><![CDATA[optical noise]]></term>
##       <term><![CDATA[photoconductivity]]></term>
##       <term><![CDATA[photoemission]]></term>
##     </controlledterms>
##     <thesaurusterms>
##       <term><![CDATA[Avalanche photodiodes]]></term>
##       <term><![CDATA[Current measurement]]></term>
##       <term><![CDATA[Equations]]></term>
##       <term><![CDATA[Indium compounds]]></term>
##       <term><![CDATA[Mathematical model]]></term>
##       <term><![CDATA[Noise]]></term>
##       <term><![CDATA[Noise measurement]]></term>
##     </thesaurusterms>
##     <pubtitle><![CDATA[Lightwave Technology, Journal of]]></pubtitle>
##     <punumber><![CDATA[50]]></punumber>
##     <pubtype><![CDATA[Journals & Magazines]]></pubtype>
##     <publisher><![CDATA[IEEE]]></publisher>
##     <volume><![CDATA[32]]></volume>
##     <issue><![CDATA[24]]></issue>
##     <py><![CDATA[2014]]></py>
##     <spage><![CDATA[4780]]></spage>
##     <epage><![CDATA[4784]]></epage>
##     <abstract><![CDATA[A method to precisely determine the quantum efficiency and primary photocurrent in avalanche photodiodes (APDs) is presented based on a linear relationship between excess noise factor F and gain, M. The new method is used to accurately compare performance of modern APD designs when nonlocal impact ionization effects govern the relationship between noise and gain.]]></abstract>
##     <issn><![CDATA[0733-8724]]></issn>
##     <htmlFlag><![CDATA[1]]></htmlFlag>
##     <arnumber><![CDATA[6936306]]></arnumber>
##     <doi><![CDATA[10.1109/JLT.2014.2364744]]></doi>
##     <publicationId><![CDATA[6936306]]></publicationId>
##     <mdurl><![CDATA[http://ieeexplore.ieee.org/xpl/articleDetails.jsp?tp=&arnumber=6936306&contentType=Journals+%26+Magazines]]></mdurl>
##     <pdf><![CDATA[http://ieeexplore.ieee.org/stamp/stamp.jsp?arnumber=6936306]]></pdf>
##   </document>
##   <document>
##     <rank>1068</rank>
##     <title><![CDATA[A Generalized Sum-Rate Optimizer for Cooperative Multiuser Massive MIMO Link Topologies]]></title>
##     <authors><![CDATA[Anderson, A.L.;  Jensen, M.A.]]></authors>
##     <affiliations><![CDATA[Dept. of Electr. & Comput. Eng., Tennessee Technol. Univ., Cookeville, TN, USA]]></affiliations>
##     <controlledterms>
##       <term><![CDATA[MIMO communication]]></term>
##       <term><![CDATA[antenna arrays]]></term>
##       <term><![CDATA[array signal processing]]></term>
##       <term><![CDATA[cooperative communication]]></term>
##       <term><![CDATA[multiuser detection]]></term>
##       <term><![CDATA[optimisation]]></term>
##       <term><![CDATA[precoding]]></term>
##       <term><![CDATA[telecommunication network topology]]></term>
##     </controlledterms>
##     <thesaurusterms>
##       <term><![CDATA[Antennas]]></term>
##       <term><![CDATA[Large-scale systems]]></term>
##       <term><![CDATA[MIMO]]></term>
##       <term><![CDATA[Multiuser channels]]></term>
##       <term><![CDATA[Network topology]]></term>
##       <term><![CDATA[Optimization]]></term>
##       <term><![CDATA[Signal processing algorithms]]></term>
##     </thesaurusterms>
##     <pubtitle><![CDATA[Access, IEEE]]></pubtitle>
##     <punumber><![CDATA[6287639]]></punumber>
##     <pubtype><![CDATA[Journals & Magazines]]></pubtype>
##     <publisher><![CDATA[IEEE]]></publisher>
##     <volume><![CDATA[2]]></volume>
##     <py><![CDATA[2014]]></py>
##     <spage><![CDATA[1040]]></spage>
##     <epage><![CDATA[1050]]></epage>
##     <abstract><![CDATA[Large-scale, or massive, multiple-input multiple-output (MIMO) systems are typified by the number of antennas contributing to a communication link. This type of link can consist of single nodes with a large number of antennas or a large number of cooperating nodes-each contributing a small number of antennas. Such massive systems naturally lead to link topologies that are not often considered in studies of smaller scale cooperative MIMO scenarios. For this system to be economically practical, each node participating in the massive link likely has limited transmit power capability, and therefore properly limiting the per-node transmit power must be incorporated into the signal processing algorithm. This paper develops a generalized multiuser massive MIMO (G4M) optimization algorithm for colocated or cooperative signaling, subject to any sum-, per-antenna, or per-node power constraint, and that can also accommodate nonlinear precoding and detection and any number of antennas. Using the G4M algorithm, a number of topologies unique to cooperative massive MIMO are described, demonstrating the facility this algorithm provides in optimizing the performance of multiuser massive links with atypical topologies.]]></abstract>
##     <issn><![CDATA[2169-3536]]></issn>
##     <htmlFlag><![CDATA[1]]></htmlFlag>
##     <arnumber><![CDATA[6877656]]></arnumber>
##     <doi><![CDATA[10.1109/ACCESS.2014.2347241]]></doi>
##     <publicationId><![CDATA[6877656]]></publicationId>
##     <mdurl><![CDATA[http://ieeexplore.ieee.org/xpl/articleDetails.jsp?tp=&arnumber=6877656&contentType=Journals+%26+Magazines]]></mdurl>
##     <pdf><![CDATA[http://ieeexplore.ieee.org/stamp/stamp.jsp?arnumber=6877656]]></pdf>
##   </document>
##   <document>
##     <rank>1069</rank>
##     <title><![CDATA[Sliding HDCA: Single-Trial EEG Classification to Overcome and Quantify Temporal Variability]]></title>
##     <authors><![CDATA[Marathe, A.R.;  Ries, A.J.;  McDowell, K.]]></authors>
##     <affiliations><![CDATA[Human Res. & Eng. Directorate, U.S. Army Res. Lab., Aberdeen Proving Grounds, MD, USA]]></affiliations>
##     <controlledterms>
##       <term><![CDATA[brain-computer interfaces]]></term>
##       <term><![CDATA[electroencephalography]]></term>
##       <term><![CDATA[learning (artificial intelligence)]]></term>
##       <term><![CDATA[medical signal processing]]></term>
##       <term><![CDATA[neurophysiology]]></term>
##       <term><![CDATA[pattern recognition]]></term>
##       <term><![CDATA[signal classification]]></term>
##     </controlledterms>
##     <thesaurusterms>
##       <term><![CDATA[Accuracy]]></term>
##       <term><![CDATA[Classification algorithms]]></term>
##       <term><![CDATA[Electroencephalography]]></term>
##       <term><![CDATA[Feature extraction]]></term>
##       <term><![CDATA[Signal to noise ratio]]></term>
##       <term><![CDATA[Standards]]></term>
##       <term><![CDATA[Time measurement]]></term>
##     </thesaurusterms>
##     <pubtitle><![CDATA[Neural Systems and Rehabilitation Engineering, IEEE Transactions on]]></pubtitle>
##     <punumber><![CDATA[7333]]></punumber>
##     <pubtype><![CDATA[Journals & Magazines]]></pubtype>
##     <publisher><![CDATA[IEEE]]></publisher>
##     <volume><![CDATA[22]]></volume>
##     <issue><![CDATA[2]]></issue>
##     <py><![CDATA[2014]]></py>
##     <spage><![CDATA[201]]></spage>
##     <epage><![CDATA[211]]></epage>
##     <abstract><![CDATA[Patterns of neural data obtained from electroencephalography (EEG) can be classified by machine learning techniques to increase human-system performance. In controlled laboratory settings this classification approach works well; however, transitioning these approaches into more dynamic, unconstrained environments will present several significant challenges. One such challenge is an increase in temporal variability in measured behavioral and neural responses, which often results in suboptimal classification performance. Previously, we reported a novel classification method designed to account for temporal variability in the neural response in order to improve classification performance by using sliding windows in hierarchical discriminant component analysis (HDCA), and demonstrated a decrease in classification error by over 50% when compared to the standard HDCA method (Marathe et al., 2013). Here, we expand upon this approach and show that embedded within this new method is a novel signal transformation that, when applied to EEG signals, significantly improves the signal-to-noise ratio and thereby enables more accurate single-trial analysis. The results presented here have significant implications for both brain-computer interaction technologies and basic science research into neural processes.]]></abstract>
##     <issn><![CDATA[1534-4320]]></issn>
##     <htmlFlag><![CDATA[1]]></htmlFlag>
##     <arnumber><![CDATA[6739167]]></arnumber>
##     <doi><![CDATA[10.1109/TNSRE.2014.2304884]]></doi>
##     <publicationId><![CDATA[6739167]]></publicationId>
##     <mdurl><![CDATA[http://ieeexplore.ieee.org/xpl/articleDetails.jsp?tp=&arnumber=6739167&contentType=Journals+%26+Magazines]]></mdurl>
##     <pdf><![CDATA[http://ieeexplore.ieee.org/stamp/stamp.jsp?arnumber=6739167]]></pdf>
##   </document>
##   <document>
##     <rank>1070</rank>
##     <title><![CDATA[Optimal Power Flow in Direct Current Networks]]></title>
##     <authors><![CDATA[Lingwen Gan;  Low, S.H.]]></authors>
##     <affiliations><![CDATA[EE, Caltech, Pasadena, CA, USA]]></affiliations>
##     <controlledterms>
##       <term><![CDATA[load flow]]></term>
##       <term><![CDATA[optimisation]]></term>
##     </controlledterms>
##     <thesaurusterms>
##       <term><![CDATA[Approximation methods]]></term>
##       <term><![CDATA[Load flow]]></term>
##       <term><![CDATA[Numerical stability]]></term>
##       <term><![CDATA[Relaxation methods]]></term>
##       <term><![CDATA[Voltage control]]></term>
##     </thesaurusterms>
##     <pubtitle><![CDATA[Power Systems, IEEE Transactions on]]></pubtitle>
##     <punumber><![CDATA[59]]></punumber>
##     <pubtype><![CDATA[Journals & Magazines]]></pubtype>
##     <publisher><![CDATA[IEEE]]></publisher>
##     <volume><![CDATA[29]]></volume>
##     <issue><![CDATA[6]]></issue>
##     <py><![CDATA[2014]]></py>
##     <spage><![CDATA[2892]]></spage>
##     <epage><![CDATA[2904]]></epage>
##     <abstract><![CDATA[The optimal power flow (OPF) problem determines power generations/demands that minimize a certain objective such as generation cost or power loss. It is non-convex and NP-hard in general. In this paper, we study the OPF problem in direct current (DC) networks. A second-order cone programming (SOCP) relaxation is considered for solving the OPF problem. We prove that the SOCP relaxation is exact if either 1) voltage upper bounds do not bind; or 2) voltage upper bounds are uniform and power injection lower bounds are negative. Based on 1), a modified OPF problem is proposed, whose corresponding SOCP is guaranteed to be exact. We also prove that SOCP has at most one optimal solution if it is exact. Finally, we discuss how to improve numerical stability and how to include line constraints.]]></abstract>
##     <issn><![CDATA[0885-8950]]></issn>
##     <htmlFlag><![CDATA[1]]></htmlFlag>
##     <arnumber><![CDATA[6803964]]></arnumber>
##     <doi><![CDATA[10.1109/TPWRS.2014.2313514]]></doi>
##     <publicationId><![CDATA[6803964]]></publicationId>
##     <mdurl><![CDATA[http://ieeexplore.ieee.org/xpl/articleDetails.jsp?tp=&arnumber=6803964&contentType=Journals+%26+Magazines]]></mdurl>
##     <pdf><![CDATA[http://ieeexplore.ieee.org/stamp/stamp.jsp?arnumber=6803964]]></pdf>
##   </document>
##   <document>
##     <rank>1071</rank>
##     <title><![CDATA[Snapshot Performance of the Dominant Mode Rejection Beamformer]]></title>
##     <authors><![CDATA[Wage, K.E.;  Buck, J.R.]]></authors>
##     <affiliations><![CDATA[Dept. of Electr. & Comput. Eng., George Mason Univ., Fairfax, VA, USA]]></affiliations>
##     <controlledterms>
##       <term><![CDATA[adaptive signal detection]]></term>
##       <term><![CDATA[adaptive signal processing]]></term>
##       <term><![CDATA[array signal processing]]></term>
##       <term><![CDATA[covariance matrices]]></term>
##       <term><![CDATA[eigenvalues and eigenfunctions]]></term>
##       <term><![CDATA[sonar signal processing]]></term>
##       <term><![CDATA[white noise]]></term>
##     </controlledterms>
##     <pubtitle><![CDATA[Oceanic Engineering, IEEE Journal of]]></pubtitle>
##     <punumber><![CDATA[48]]></punumber>
##     <pubtype><![CDATA[Journals & Magazines]]></pubtype>
##     <publisher><![CDATA[IEEE]]></publisher>
##     <volume><![CDATA[39]]></volume>
##     <issue><![CDATA[2]]></issue>
##     <py><![CDATA[2014]]></py>
##     <spage><![CDATA[212]]></spage>
##     <epage><![CDATA[225]]></epage>
##     <abstract><![CDATA[The dominant mode rejection (DMR) beamformer constructs its weight vector using a structured covariance estimate derived from the eigendecomposition of the sample covariance matrix (SCM). Like all adaptive beamformers (ABFs), the DMR ABF places notches in the direction of loud interferers to facilitate the detection of quiet targets. This paper investigates how DMR performs as a function of the number of snapshots used to estimate the SCM. The analysis focuses on the fundamental case of a single interferer in white noise. Theoretical calculations for the ensemble case reveal the relationships among notch depth, white noise gain, and SINR. The centerpiece of the paper is a detailed empirical study of the single-interferer case, which includes snapshot-deficient scenarios often ignored in previous work. Empirical data demonstrate that the sample eigenvectors determine the mean performance of the DMR ABF. On a log-log plot the mean notch depth is a piecewise linear function of the number of snapshots and the interference-to-noise ratio. The paper interprets the behavior of the DMR ABF using recent results on sample eigenvectors derived from random matrix theory.]]></abstract>
##     <issn><![CDATA[0364-9059]]></issn>
##     <htmlFlag><![CDATA[1]]></htmlFlag>
##     <arnumber><![CDATA[6516096]]></arnumber>
##     <doi><![CDATA[10.1109/JOE.2013.2251538]]></doi>
##     <publicationId><![CDATA[6516096]]></publicationId>
##     <mdurl><![CDATA[http://ieeexplore.ieee.org/xpl/articleDetails.jsp?tp=&arnumber=6516096&contentType=Journals+%26+Magazines]]></mdurl>
##     <pdf><![CDATA[http://ieeexplore.ieee.org/stamp/stamp.jsp?arnumber=6516096]]></pdf>
##   </document>
##   <document>
##     <rank>1072</rank>
##     <title><![CDATA[Affective State Level Recognition in Naturalistic Facial and Vocal Expressions]]></title>
##     <authors><![CDATA[Hongying Meng;  Bianchi-Berthouze, N.]]></authors>
##     <affiliations><![CDATA[Sch. of Eng. & Design, Brunel Univ., Uxbridge, UK]]></affiliations>
##     <controlledterms>
##       <term><![CDATA[audio signal processing]]></term>
##       <term><![CDATA[emotion recognition]]></term>
##       <term><![CDATA[face recognition]]></term>
##       <term><![CDATA[feature extraction]]></term>
##       <term><![CDATA[hidden Markov models]]></term>
##       <term><![CDATA[image classification]]></term>
##       <term><![CDATA[learning (artificial intelligence)]]></term>
##       <term><![CDATA[matrix algebra]]></term>
##       <term><![CDATA[speech recognition]]></term>
##     </controlledterms>
##     <pubtitle><![CDATA[Cybernetics, IEEE Transactions on]]></pubtitle>
##     <punumber><![CDATA[6221036]]></punumber>
##     <pubtype><![CDATA[Journals & Magazines]]></pubtype>
##     <publisher><![CDATA[IEEE]]></publisher>
##     <volume><![CDATA[44]]></volume>
##     <issue><![CDATA[3]]></issue>
##     <py><![CDATA[2014]]></py>
##     <spage><![CDATA[315]]></spage>
##     <epage><![CDATA[328]]></epage>
##     <abstract><![CDATA[Naturalistic affective expressions change at a rate much slower than the typical rate at which video or audio is recorded. This increases the probability that consecutive recorded instants of expressions represent the same affective content. In this paper, we exploit such a relationship to improve the recognition performance of continuous naturalistic affective expressions. Using datasets of naturalistic affective expressions (AVEC 2011 audio and video dataset, PAINFUL video dataset) continuously labeled over time and over different dimensions, we analyze the transitions between levels of those dimensions (e.g., transitions in pain intensity level). We use an information theory approach to show that the transitions occur very slowly and hence suggest modeling them as first-order Markov models. The dimension levels are considered to be the hidden states in the Hidden Markov Model (HMM) framework. Their discrete transition and emission matrices are trained by using the labels provided with the training set. The recognition problem is converted into a best path-finding problem to obtain the best hidden states sequence in HMMs. This is a key difference from previous use of HMMs as classifiers. Modeling of the transitions between dimension levels is integrated in a multistage approach, where the first level performs a mapping between the affective expression features and a soft decision value (e.g., an affective dimension level), and further classification stages are modeled as HMMs that refine that mapping by taking into account the temporal relationships between the output decision labels. The experimental results for each of the unimodal datasets show overall performance to be significantly above that of a standard classification system that does not take into account temporal relationships. In particular, the results on the AVEC 2011 audio dataset outperform all other systems presented at the international competition.]]></abstract>
##     <issn><![CDATA[2168-2267]]></issn>
##     <htmlFlag><![CDATA[1]]></htmlFlag>
##     <arnumber><![CDATA[6507321]]></arnumber>
##     <doi><![CDATA[10.1109/TCYB.2013.2253768]]></doi>
##     <publicationId><![CDATA[6507321]]></publicationId>
##     <mdurl><![CDATA[http://ieeexplore.ieee.org/xpl/articleDetails.jsp?tp=&arnumber=6507321&contentType=Journals+%26+Magazines]]></mdurl>
##     <pdf><![CDATA[http://ieeexplore.ieee.org/stamp/stamp.jsp?arnumber=6507321]]></pdf>
##   </document>
##   <document>
##     <rank>1073</rank>
##     <title><![CDATA[A Noncontact Capacitive Sensing System for Recognizing Locomotion Modes of Transtibial Amputees]]></title>
##     <authors><![CDATA[Enhao Zheng;  Long Wang;  Kunlin Wei;  Qining Wang]]></authors>
##     <affiliations><![CDATA[Intell. Control Lab., Peking Univ., Beijing, China]]></affiliations>
##     <controlledterms>
##       <term><![CDATA[biomechanics]]></term>
##       <term><![CDATA[biomedical telemetry]]></term>
##       <term><![CDATA[body sensor networks]]></term>
##       <term><![CDATA[capacitive sensors]]></term>
##       <term><![CDATA[electromyography]]></term>
##       <term><![CDATA[feature extraction]]></term>
##       <term><![CDATA[feature selection]]></term>
##       <term><![CDATA[medical control systems]]></term>
##       <term><![CDATA[medical signal detection]]></term>
##       <term><![CDATA[medical signal processing]]></term>
##       <term><![CDATA[prosthetics]]></term>
##       <term><![CDATA[quadratic programming]]></term>
##       <term><![CDATA[signal classification]]></term>
##       <term><![CDATA[skin]]></term>
##       <term><![CDATA[telemedicine]]></term>
##     </controlledterms>
##     <thesaurusterms>
##       <term><![CDATA[Capacitive sensors]]></term>
##       <term><![CDATA[Electromyography]]></term>
##       <term><![CDATA[Motion detection]]></term>
##       <term><![CDATA[Prosthetics]]></term>
##       <term><![CDATA[Wearable sensors]]></term>
##     </thesaurusterms>
##     <pubtitle><![CDATA[Biomedical Engineering, IEEE Transactions on]]></pubtitle>
##     <punumber><![CDATA[10]]></punumber>
##     <pubtype><![CDATA[Journals & Magazines]]></pubtype>
##     <publisher><![CDATA[IEEE]]></publisher>
##     <volume><![CDATA[61]]></volume>
##     <issue><![CDATA[12]]></issue>
##     <py><![CDATA[2014]]></py>
##     <spage><![CDATA[2911]]></spage>
##     <epage><![CDATA[2920]]></epage>
##     <abstract><![CDATA[This paper presents a noncontact capacitive sensing system (C-Sens) for locomotion mode recognition of transtibial amputees. C-Sens detects changes in physical distance between the residual limb and the prosthesis. The sensing front ends are built into the prosthetic socket without contacting the skin. This novel signal source improves the usability of locomotion mode recognition systems based on electromyography (EMG) signals and systems based on capacitance signals obtained from skin contact. To evaluate the performance of C-Sens, we carried out experiments among six transtibial amputees with varying levels of amputation when they engaged in six common locomotive activities. The capacitance signals were consistent and stereotypical for different locomotion modes. Importantly, we were able to obtain sufficiently informative signals even for amputees with severe muscle atrophy (i.e., amputees lacking of quality EMG from shank muscles for mode classification). With phase-dependent quadratic classifier and selected feature set, the proposed system was capable of making continuous judgments about locomotion modes with an average accuracy of 96.3% and 94.8% for swing phase and stance phase, respectively (Experiment 1). Furthermore, the system was able to achieve satisfactory recognition performance after the subjects redonned the socket (Experiment 2). We also validated that C-Sens was robust to load bearing changes when amputees carried 5-kg weights during activities (Experiment 3). These results suggest that noncontact capacitive sensing is capable of circumventing practical problems of EMG systems without sacrificing performance and it is, thus, promising for automatic recognition of human motion intent for controlling powered prostheses.]]></abstract>
##     <issn><![CDATA[0018-9294]]></issn>
##     <htmlFlag><![CDATA[1]]></htmlFlag>
##     <arnumber><![CDATA[6847122]]></arnumber>
##     <doi><![CDATA[10.1109/TBME.2014.2334316]]></doi>
##     <publicationId><![CDATA[6847122]]></publicationId>
##     <mdurl><![CDATA[http://ieeexplore.ieee.org/xpl/articleDetails.jsp?tp=&arnumber=6847122&contentType=Journals+%26+Magazines]]></mdurl>
##     <pdf><![CDATA[http://ieeexplore.ieee.org/stamp/stamp.jsp?arnumber=6847122]]></pdf>
##   </document>
##   <document>
##     <rank>1074</rank>
##     <title><![CDATA[X-Ray Fluorescence Computed Tomography With Polycapillary Focusing]]></title>
##     <authors><![CDATA[Wenxiang Cong;  Yan Xi;  Ge Wang]]></authors>
##     <affiliations><![CDATA[Dept. of Biomed. Eng., Rensselaer Polytech. Inst., Troy, NY, USA]]></affiliations>
##     <controlledterms>
##       <term><![CDATA[X-ray fluorescence analysis]]></term>
##       <term><![CDATA[biomedical optical imaging]]></term>
##       <term><![CDATA[computerised tomography]]></term>
##       <term><![CDATA[diagnostic radiography]]></term>
##       <term><![CDATA[diseases]]></term>
##       <term><![CDATA[fluorescence]]></term>
##       <term><![CDATA[image reconstruction]]></term>
##       <term><![CDATA[iodine]]></term>
##       <term><![CDATA[nanomedicine]]></term>
##       <term><![CDATA[nanoparticles]]></term>
##       <term><![CDATA[numerical analysis]]></term>
##       <term><![CDATA[patient diagnosis]]></term>
##     </controlledterms>
##     <thesaurusterms>
##       <term><![CDATA[Biomedical image processing]]></term>
##       <term><![CDATA[Computed tomography]]></term>
##       <term><![CDATA[Fluorescence]]></term>
##       <term><![CDATA[Image reconstruction]]></term>
##       <term><![CDATA[Nanoparticles]]></term>
##       <term><![CDATA[Probes]]></term>
##       <term><![CDATA[X-ray imaging]]></term>
##     </thesaurusterms>
##     <pubtitle><![CDATA[Access, IEEE]]></pubtitle>
##     <punumber><![CDATA[6287639]]></punumber>
##     <pubtype><![CDATA[Journals & Magazines]]></pubtype>
##     <publisher><![CDATA[IEEE]]></publisher>
##     <volume><![CDATA[2]]></volume>
##     <py><![CDATA[2014]]></py>
##     <spage><![CDATA[1138]]></spage>
##     <epage><![CDATA[1142]]></epage>
##     <abstract><![CDATA[Liposomal iodine nanoparticles (LINPs) have a long half-life and provide an excellent intravascular contrast. The nanoparticles can be functionalized as molecular probes for biological targets to facilitate numerous preclinical studies for translation toward diagnosis and therapy of various human diseases. Iodine has a K-edge at 33 keV due to the photoelectric absorption of photons, which emit X-ray fluorescence at 28 keV with a fluorescence yield of 0.88. Detections of the characteristic X-rays can be used for the imaging of iodine concentration distribution in an object. In this paper, we propose an X-ray fluorescence computed tomography method for reconstruction of a LINPs distribution over a region of interest (ROI) in a small animal. X-rays are focused onto a submillimeter focal spot utilizing a polycapillary lens, generating a pair of X-ray cones in the animal. This focused beam irradiates LINPs, the most strongly at the focal spot. Then, the focal spot can be scanned over an ROI in the object to produce X-ray fluorescence signals. From measured fluorescence data, a reliable image reconstruction can be achieved with a high spatial resolution. Numerical simulation studies are performed to demonstrate the superior imaging performance of this methodology.]]></abstract>
##     <issn><![CDATA[2169-3536]]></issn>
##     <htmlFlag><![CDATA[1]]></htmlFlag>
##     <arnumber><![CDATA[6919257]]></arnumber>
##     <doi><![CDATA[10.1109/ACCESS.2014.2359831]]></doi>
##     <publicationId><![CDATA[6919257]]></publicationId>
##     <mdurl><![CDATA[http://ieeexplore.ieee.org/xpl/articleDetails.jsp?tp=&arnumber=6919257&contentType=Journals+%26+Magazines]]></mdurl>
##     <pdf><![CDATA[http://ieeexplore.ieee.org/stamp/stamp.jsp?arnumber=6919257]]></pdf>
##   </document>
##   <document>
##     <rank>1075</rank>
##     <title><![CDATA[Toward Scalable Systems for Big Data Analytics: A Technology Tutorial]]></title>
##     <authors><![CDATA[Han Hu;  Yonggang Wen;  Tat-Seng Chua;  Xuelong Li]]></authors>
##     <affiliations><![CDATA[Sch. of Comput., Nat. Univ. of Singapore, Singapore, Singapore]]></affiliations>
##     <controlledterms>
##       <term><![CDATA[Big Data]]></term>
##       <term><![CDATA[data acquisition]]></term>
##       <term><![CDATA[data analysis]]></term>
##       <term><![CDATA[data communication]]></term>
##       <term><![CDATA[public domain software]]></term>
##       <term><![CDATA[storage management]]></term>
##     </controlledterms>
##     <thesaurusterms>
##       <term><![CDATA[Big data]]></term>
##       <term><![CDATA[Data acquisition]]></term>
##       <term><![CDATA[Information analysis]]></term>
##       <term><![CDATA[Medical services]]></term>
##       <term><![CDATA[Real-time systems]]></term>
##       <term><![CDATA[Scalability]]></term>
##       <term><![CDATA[Sensor phenomena and characterization]]></term>
##       <term><![CDATA[Sensor systems]]></term>
##       <term><![CDATA[Supply chain management]]></term>
##     </thesaurusterms>
##     <pubtitle><![CDATA[Access, IEEE]]></pubtitle>
##     <punumber><![CDATA[6287639]]></punumber>
##     <pubtype><![CDATA[Journals & Magazines]]></pubtype>
##     <publisher><![CDATA[IEEE]]></publisher>
##     <volume><![CDATA[2]]></volume>
##     <py><![CDATA[2014]]></py>
##     <spage><![CDATA[652]]></spage>
##     <epage><![CDATA[687]]></epage>
##     <abstract><![CDATA[Recent technological advancements have led to a deluge of data from distinctive domains (e.g., health care and scientific sensors, user-generated data, Internet and financial companies, and supply chain systems) over the past two decades. The term big data was coined to capture the meaning of this emerging trend. In addition to its sheer volume, big data also exhibits other unique characteristics as compared with traditional data. For instance, big data is commonly unstructured and require more real-time analysis. This development calls for new system architectures for data acquisition, transmission, storage, and large-scale data processing mechanisms. In this paper, we present a literature survey and system tutorial for big data analytics platforms, aiming to provide an overall picture for nonexpert readers and instill a do-it-yourself spirit for advanced audiences to customize their own big-data solutions. First, we present the definition of big data and discuss big data challenges. Next, we present a systematic framework to decompose big data systems into four sequential modules, namely data generation, data acquisition, data storage, and data analytics. These four modules form a big data value chain. Following that, we present a detailed survey of numerous approaches and mechanisms from research and industry communities. In addition, we present the prevalent Hadoop framework for addressing big data challenges. Finally, we outline several evaluation benchmarks and potential research directions for big data systems.]]></abstract>
##     <issn><![CDATA[2169-3536]]></issn>
##     <htmlFlag><![CDATA[1]]></htmlFlag>
##     <arnumber><![CDATA[6842585]]></arnumber>
##     <doi><![CDATA[10.1109/ACCESS.2014.2332453]]></doi>
##     <publicationId><![CDATA[6842585]]></publicationId>
##     <mdurl><![CDATA[http://ieeexplore.ieee.org/xpl/articleDetails.jsp?tp=&arnumber=6842585&contentType=Journals+%26+Magazines]]></mdurl>
##     <pdf><![CDATA[http://ieeexplore.ieee.org/stamp/stamp.jsp?arnumber=6842585]]></pdf>
##   </document>
##   <document>
##     <rank>1076</rank>
##     <title><![CDATA[Magnetic FEM Design and Experimental Validation of an Innovative Fail-Safe Magnetorheological Clutch Excited by Permanent Magnets]]></title>
##     <authors><![CDATA[Rizzo, R.;  Musolino, A.;  Bucchi, F.;  Forte, P.;  Frendo, F.]]></authors>
##     <affiliations><![CDATA[Dept. of Energy & Syst. Eng., Univ. of Pisa, Pisa, Italy]]></affiliations>
##     <controlledterms>
##       <term><![CDATA[clutches]]></term>
##       <term><![CDATA[finite element analysis]]></term>
##       <term><![CDATA[magnetic flux]]></term>
##       <term><![CDATA[magnetorheology]]></term>
##       <term><![CDATA[permanent magnets]]></term>
##       <term><![CDATA[torque]]></term>
##     </controlledterms>
##     <thesaurusterms>
##       <term><![CDATA[Iron]]></term>
##       <term><![CDATA[Magnetic analysis]]></term>
##       <term><![CDATA[Magnetomechanical effects]]></term>
##       <term><![CDATA[Materials]]></term>
##       <term><![CDATA[Shafts]]></term>
##       <term><![CDATA[Stress]]></term>
##       <term><![CDATA[Torque]]></term>
##     </thesaurusterms>
##     <pubtitle><![CDATA[Energy Conversion, IEEE Transactions on]]></pubtitle>
##     <punumber><![CDATA[60]]></punumber>
##     <pubtype><![CDATA[Journals & Magazines]]></pubtype>
##     <publisher><![CDATA[IEEE]]></publisher>
##     <volume><![CDATA[29]]></volume>
##     <issue><![CDATA[3]]></issue>
##     <py><![CDATA[2014]]></py>
##     <spage><![CDATA[628]]></spage>
##     <epage><![CDATA[640]]></epage>
##     <abstract><![CDATA[This paper describes the magnetic design of an innovative fail-safe clutch based on magnetorheological fluid (MRF). A cylindrical arrangement of permanent magnets (PMs) is used to excite the fluid. The suitable distribution of magnetic field inside the MRF and the transmissible torque is obtained by moving the PMs along the axial direction. The device is designed using a magneto/mechanical FEM model, developed on purpose and based on a three-dimensional (3-D) finite-element code, which takes into account the B-H and &#x03C4;-H functions of the nonlinear materials (e.g., MRF, PM, and ferromagnetic materials). The flux density maps and the shear stress maps inside the fluid are carefully analyzed. Furthermore, in order to validate the FEM model, some preliminary experimental measurements are performed on a prototype. Finally, the magnetic axial force acting on the PM system is investigated.]]></abstract>
##     <issn><![CDATA[0885-8969]]></issn>
##     <htmlFlag><![CDATA[1]]></htmlFlag>
##     <arnumber><![CDATA[6832543]]></arnumber>
##     <doi><![CDATA[10.1109/TEC.2014.2325964]]></doi>
##     <publicationId><![CDATA[6832543]]></publicationId>
##     <mdurl><![CDATA[http://ieeexplore.ieee.org/xpl/articleDetails.jsp?tp=&arnumber=6832543&contentType=Journals+%26+Magazines]]></mdurl>
##     <pdf><![CDATA[http://ieeexplore.ieee.org/stamp/stamp.jsp?arnumber=6832543]]></pdf>
##   </document>
##   <document>
##     <rank>1077</rank>
##     <title><![CDATA[Cognitive Packet Network for Bilateral Asymmetric Connections]]></title>
##     <authors><![CDATA[Gelenbe, E.;  Kazhmaganbetova, Z.]]></authors>
##     <affiliations><![CDATA[Dept. of Electr. & Electron. Eng., Imperial Coll. London, London, UK]]></affiliations>
##     <controlledterms>
##       <term><![CDATA[cognitive radio]]></term>
##       <term><![CDATA[minimisation]]></term>
##       <term><![CDATA[quality of service]]></term>
##       <term><![CDATA[telecommunication traffic]]></term>
##     </controlledterms>
##     <thesaurusterms>
##       <term><![CDATA[Delays]]></term>
##       <term><![CDATA[Downlink]]></term>
##       <term><![CDATA[Neurons]]></term>
##       <term><![CDATA[Packet loss]]></term>
##       <term><![CDATA[Quality of service]]></term>
##       <term><![CDATA[Uplink]]></term>
##     </thesaurusterms>
##     <pubtitle><![CDATA[Industrial Informatics, IEEE Transactions on]]></pubtitle>
##     <punumber><![CDATA[9424]]></punumber>
##     <pubtype><![CDATA[Journals & Magazines]]></pubtype>
##     <publisher><![CDATA[IEEE]]></publisher>
##     <volume><![CDATA[10]]></volume>
##     <issue><![CDATA[3]]></issue>
##     <py><![CDATA[2014]]></py>
##     <spage><![CDATA[1717]]></spage>
##     <epage><![CDATA[1725]]></epage>
##     <abstract><![CDATA[Network testbed experimentation is useful to evaluate new protocols, since it offers realism and repeatability under controllable conditions. Thus, in this paper, we make use of a software network platform, the cognitive packet network (CPN), to offer best-effort quality of service (QoS) to end users and to develop a new bilateral QoS differentiation between pairs of communicating nodes. In the proposed approach, each CPN edge or user node is a source and a destination at the same time, managing uplink user-originated traffic and downlink traffic sent back in response to the uplink. The bilateral communication is implemented with four distinct QoS objectives that can be met between sender nodes (original source or destination). Traffic volume asymmetry between the received and the sent data is used to trigger changes in QoS. The lower traffic rate requires short-delay QoS, whereas the higher traffic rate requires loss minimization. The effectiveness of the approach is evaluated by several measurements.]]></abstract>
##     <issn><![CDATA[1551-3203]]></issn>
##     <htmlFlag><![CDATA[1]]></htmlFlag>
##     <arnumber><![CDATA[6810145]]></arnumber>
##     <doi><![CDATA[10.1109/TII.2014.2321740]]></doi>
##     <publicationId><![CDATA[6810145]]></publicationId>
##     <mdurl><![CDATA[http://ieeexplore.ieee.org/xpl/articleDetails.jsp?tp=&arnumber=6810145&contentType=Journals+%26+Magazines]]></mdurl>
##     <pdf><![CDATA[http://ieeexplore.ieee.org/stamp/stamp.jsp?arnumber=6810145]]></pdf>
##   </document>
##   <document>
##     <rank>1078</rank>
##     <title><![CDATA[A 15-Gb/s Wireless ON-OFF Keying Link]]></title>
##     <authors><![CDATA[Ohlsson, L.;  Wernersson, L.-E.]]></authors>
##     <affiliations><![CDATA[Dept. of Electr. & Inf. Technol., Lund Univ., Lund, Sweden]]></affiliations>
##     <controlledterms>
##       <term><![CDATA[amplitude shift keying]]></term>
##       <term><![CDATA[radio links]]></term>
##       <term><![CDATA[random sequences]]></term>
##     </controlledterms>
##     <thesaurusterms>
##       <term><![CDATA[Bit error rate]]></term>
##       <term><![CDATA[Data transfer]]></term>
##       <term><![CDATA[Generators]]></term>
##       <term><![CDATA[IEEE standards]]></term>
##       <term><![CDATA[Modulation]]></term>
##       <term><![CDATA[Receivers]]></term>
##       <term><![CDATA[Transmitters]]></term>
##       <term><![CDATA[Wireless communication]]></term>
##     </thesaurusterms>
##     <pubtitle><![CDATA[Access, IEEE]]></pubtitle>
##     <punumber><![CDATA[6287639]]></punumber>
##     <pubtype><![CDATA[Journals & Magazines]]></pubtype>
##     <publisher><![CDATA[IEEE]]></publisher>
##     <volume><![CDATA[2]]></volume>
##     <py><![CDATA[2014]]></py>
##     <spage><![CDATA[1307]]></spage>
##     <epage><![CDATA[1313]]></epage>
##     <abstract><![CDATA[Bit-error rate measurements for ON-OFF keying modulation at multigigabit per second rates over a V-band wireless link are presented. Serial data-rates from 2.5 to 20 Gb/s were studied for a 2<sup>31</sup> -1 bit random sequence. Error-free data transfer over a 0.3-m link was achieved at up to 10 Gb/s. Acceptable bit-error rates, &lt;; 10<sup>-5</sup> and 10<sup>-3</sup>, were measured at up to 1.5 m for 10- and 15-Gb/s data-rate, respectively. The performance was achieved using a transmitter that consists of an integrated wavelet generator, whereas the receiver was built from off-the-shelf waveguide components. The results demonstrate that very high data-rates may be achieved using binary modulation and short symbols generated in an efficient V-band transmitter. The system is benchmarked against state-of-the-art transceiver systems with multigigabit per second data-rates.]]></abstract>
##     <issn><![CDATA[2169-3536]]></issn>
##     <htmlFlag><![CDATA[1]]></htmlFlag>
##     <arnumber><![CDATA[6934981]]></arnumber>
##     <doi><![CDATA[10.1109/ACCESS.2014.2364638]]></doi>
##     <publicationId><![CDATA[6934981]]></publicationId>
##     <mdurl><![CDATA[http://ieeexplore.ieee.org/xpl/articleDetails.jsp?tp=&arnumber=6934981&contentType=Journals+%26+Magazines]]></mdurl>
##     <pdf><![CDATA[http://ieeexplore.ieee.org/stamp/stamp.jsp?arnumber=6934981]]></pdf>
##   </document>
##   <document>
##     <rank>1079</rank>
##     <title><![CDATA[Experimental Demonstration of 429.96-Gb/s OFDM/OQAM&#x2013;64QAM Over 400-km SSMF Transmission Within a 50-GHz Grid]]></title>
##     <authors><![CDATA[Chao Li;  Xuebing Zhang;  Haibo Li;  Cai Li;  Min Lou;  Zhaohui Li;  Jing Xu;  Qi Yang;  Shaohua Yu]]></authors>
##     <affiliations><![CDATA[Sch. of Opt. & Electron. Inf., Huazhong Univ. of Sci. & Technol., Wuhan, China]]></affiliations>
##     <controlledterms>
##       <term><![CDATA[OFDM modulation]]></term>
##       <term><![CDATA[optical fibre communication]]></term>
##       <term><![CDATA[quadrature amplitude modulation]]></term>
##       <term><![CDATA[telecommunication channels]]></term>
##     </controlledterms>
##     <thesaurusterms>
##       <term><![CDATA[Adaptive optics]]></term>
##       <term><![CDATA[OFDM]]></term>
##       <term><![CDATA[Optical filters]]></term>
##       <term><![CDATA[Optical modulation]]></term>
##       <term><![CDATA[Optical polarization]]></term>
##       <term><![CDATA[Optical transmitters]]></term>
##     </thesaurusterms>
##     <pubtitle><![CDATA[Photonics Journal, IEEE]]></pubtitle>
##     <punumber><![CDATA[4563994]]></punumber>
##     <pubtype><![CDATA[Journals & Magazines]]></pubtype>
##     <publisher><![CDATA[IEEE]]></publisher>
##     <volume><![CDATA[6]]></volume>
##     <issue><![CDATA[4]]></issue>
##     <py><![CDATA[2014]]></py>
##     <spage><![CDATA[1]]></spage>
##     <epage><![CDATA[8]]></epage>
##     <abstract><![CDATA[We experimentally demonstrate a 429.96-Gb/s signal transmission over a 400-km standard single-mode fiber within the 50-GHz grid and successfully achieve spectral efficiency as high as 8.63 bit/s/Hz. Orthogonal frequency-division multiplexing/offset quadrature amplitude modulation with 64-quadrature amplitude modulation is selected as the modulation format to provide a &#x201C;perfect&#x201D; rectangular spectrum that efficiently reduces the channel crosstalk. No timing or frequency alignment is required for the subbands to form the superchannel.]]></abstract>
##     <issn><![CDATA[1943-0655]]></issn>
##     <htmlFlag><![CDATA[1]]></htmlFlag>
##     <arnumber><![CDATA[6844866]]></arnumber>
##     <doi><![CDATA[10.1109/JPHOT.2014.2331258]]></doi>
##     <publicationId><![CDATA[6844866]]></publicationId>
##     <mdurl><![CDATA[http://ieeexplore.ieee.org/xpl/articleDetails.jsp?tp=&arnumber=6844866&contentType=Journals+%26+Magazines]]></mdurl>
##     <pdf><![CDATA[http://ieeexplore.ieee.org/stamp/stamp.jsp?arnumber=6844866]]></pdf>
##   </document>
##   <document>
##     <rank>1080</rank>
##     <title><![CDATA[Segmentation-Driven Image Registration-Application to 4D DCE-MRI Recordings of the Moving Kidneys]]></title>
##     <authors><![CDATA[Hodneland, E.;  Hanson, E.A.;  Lundervold, A.;  Modersitzki, J.;  Eikefjord, E.;  Munthe-Kaas, A.Z.]]></authors>
##     <affiliations><![CDATA[Dept. of Biomed., Univ. of Bergen, Bergen, Norway]]></affiliations>
##     <controlledterms>
##       <term><![CDATA[biomedical MRI]]></term>
##       <term><![CDATA[image motion analysis]]></term>
##       <term><![CDATA[image registration]]></term>
##       <term><![CDATA[image segmentation]]></term>
##       <term><![CDATA[kidney]]></term>
##       <term><![CDATA[numerical analysis]]></term>
##       <term><![CDATA[time series]]></term>
##     </controlledterms>
##     <thesaurusterms>
##       <term><![CDATA[Educational institutions]]></term>
##       <term><![CDATA[Image segmentation]]></term>
##       <term><![CDATA[Kidney]]></term>
##       <term><![CDATA[Licenses]]></term>
##       <term><![CDATA[Motion segmentation]]></term>
##       <term><![CDATA[Time series analysis]]></term>
##       <term><![CDATA[Training]]></term>
##     </thesaurusterms>
##     <pubtitle><![CDATA[Image Processing, IEEE Transactions on]]></pubtitle>
##     <punumber><![CDATA[83]]></punumber>
##     <pubtype><![CDATA[Journals & Magazines]]></pubtype>
##     <publisher><![CDATA[IEEE]]></publisher>
##     <volume><![CDATA[23]]></volume>
##     <issue><![CDATA[5]]></issue>
##     <py><![CDATA[2014]]></py>
##     <spage><![CDATA[2392]]></spage>
##     <epage><![CDATA[2404]]></epage>
##     <abstract><![CDATA[Dynamic contrast enhanced magnetic resonance imaging (DCE-MRI) of the kidneys requires proper motion correction and segmentation to enable an estimation of glomerular filtration rate through pharmacokinetic modeling. Traditionally, co-registration, segmentation, and pharmacokinetic modeling have been applied sequentially as separate processing steps. In this paper, a combined 4D model for simultaneous registration and segmentation of the whole kidney is presented. To demonstrate the model in numerical experiments, we used normalized gradients as data term in the registration and a Mahalanobis distance from the time courses of the segmented regions to a training set for supervised segmentation. By applying this framework to an input consisting of 4D image time series, we conduct simultaneous motion correction and two-region segmentation into kidney and background. The potential of the new approach is demonstrated on real DCE-MRI data from ten healthy volunteers.]]></abstract>
##     <issn><![CDATA[1057-7149]]></issn>
##     <htmlFlag><![CDATA[1]]></htmlFlag>
##     <arnumber><![CDATA[6781630]]></arnumber>
##     <doi><![CDATA[10.1109/TIP.2014.2315155]]></doi>
##     <publicationId><![CDATA[6781630]]></publicationId>
##     <mdurl><![CDATA[http://ieeexplore.ieee.org/xpl/articleDetails.jsp?tp=&arnumber=6781630&contentType=Journals+%26+Magazines]]></mdurl>
##     <pdf><![CDATA[http://ieeexplore.ieee.org/stamp/stamp.jsp?arnumber=6781630]]></pdf>
##   </document>
##   <document>
##     <rank>1081</rank>
##     <title><![CDATA[Real-Time and Offline Performance of Pattern Recognition Myoelectric Control Using a Generic Electrode Grid With Targeted Muscle Reinnervation Patients]]></title>
##     <authors><![CDATA[Tkach, D.C.;  Young, A.J.;  Smith, L.H.;  Rouse, E.J.;  Hargrove, L.J.]]></authors>
##     <affiliations><![CDATA[Center for Bionic Med., Rehabilitation Inst. of Chicago, Chicago, IL, USA]]></affiliations>
##     <controlledterms>
##       <term><![CDATA[biomedical electrodes]]></term>
##       <term><![CDATA[electromyography]]></term>
##       <term><![CDATA[medical control systems]]></term>
##       <term><![CDATA[medical signal processing]]></term>
##       <term><![CDATA[muscle]]></term>
##       <term><![CDATA[pattern recognition]]></term>
##       <term><![CDATA[prosthetics]]></term>
##       <term><![CDATA[real-time systems]]></term>
##       <term><![CDATA[surgery]]></term>
##     </controlledterms>
##     <thesaurusterms>
##       <term><![CDATA[Elbow]]></term>
##       <term><![CDATA[Electrodes]]></term>
##       <term><![CDATA[Electromyography]]></term>
##       <term><![CDATA[Muscles]]></term>
##       <term><![CDATA[Pattern recognition]]></term>
##       <term><![CDATA[Prosthetics]]></term>
##       <term><![CDATA[Real-time systems]]></term>
##     </thesaurusterms>
##     <pubtitle><![CDATA[Neural Systems and Rehabilitation Engineering, IEEE Transactions on]]></pubtitle>
##     <punumber><![CDATA[7333]]></punumber>
##     <pubtype><![CDATA[Journals & Magazines]]></pubtype>
##     <publisher><![CDATA[IEEE]]></publisher>
##     <volume><![CDATA[22]]></volume>
##     <issue><![CDATA[4]]></issue>
##     <py><![CDATA[2014]]></py>
##     <spage><![CDATA[727]]></spage>
##     <epage><![CDATA[734]]></epage>
##     <abstract><![CDATA[Targeted muscle reinnervation (TMR) is a surgical technique that creates myoelectric prosthesis control sites for high-level amputees. The electromyographic (EMG) signal patterns provided by the reinnervated muscles are well-suited for pattern recognition control. Pattern recognition allows for control of a greater number of degrees of freedom (DOF) than the conventional, EMG amplitude-based approach. Previous pattern recognition studies have shown benefit in placing electrodes directly over the reinnervated muscles. Localizing the optimal TMR locations is inconvenient and time consuming. In this contribution, we demonstrate that a clinically practical grid arrangement of electrodes yields real-time control performance that is equivalent to, or better than, the site-specific electrode placement for simultaneous control of multiple DOFs using pattern recognition. Additional findings indicate that grid-like electrode arrangement yields significantly lower classification errors for classifiers with a large number of movement classes (&gt;9). These findings suggest that a grid electrode arrangement can be effectively used to control a multi-DOF upper limb prosthesis while reducing the time and effort associated with fitting the prosthesis due to clinical localization of control sites on amputee patients.]]></abstract>
##     <issn><![CDATA[1534-4320]]></issn>
##     <htmlFlag><![CDATA[1]]></htmlFlag>
##     <arnumber><![CDATA[6737321]]></arnumber>
##     <doi><![CDATA[10.1109/TNSRE.2014.2302799]]></doi>
##     <publicationId><![CDATA[6737321]]></publicationId>
##     <mdurl><![CDATA[http://ieeexplore.ieee.org/xpl/articleDetails.jsp?tp=&arnumber=6737321&contentType=Journals+%26+Magazines]]></mdurl>
##     <pdf><![CDATA[http://ieeexplore.ieee.org/stamp/stamp.jsp?arnumber=6737321]]></pdf>
##   </document>
##   <document>
##     <rank>1082</rank>
##     <title><![CDATA[All-Raman-Amplified, 73 nm Seamless Band Transmission of 9 Tb/s Over 6000 km of Fiber]]></title>
##     <authors><![CDATA[Nelson, L.E.;  Xiang Zhou;  Benyuan Zhu;  Yan, M.F.;  Wisk, P.W.;  Magill, P.D.]]></authors>
##     <affiliations><![CDATA[AT&T Labs., Middletown, NJ, USA]]></affiliations>
##     <controlledterms>
##       <term><![CDATA[Raman spectra]]></term>
##       <term><![CDATA[distributed amplifiers]]></term>
##       <term><![CDATA[optical communication equipment]]></term>
##       <term><![CDATA[optical fibre amplifiers]]></term>
##       <term><![CDATA[optical fibre communication]]></term>
##       <term><![CDATA[optical pumping]]></term>
##       <term><![CDATA[optical switches]]></term>
##       <term><![CDATA[quadrature phase shift keying]]></term>
##       <term><![CDATA[wavelength division multiplexing]]></term>
##     </controlledterms>
##     <thesaurusterms>
##       <term><![CDATA[Optical fiber amplifiers]]></term>
##       <term><![CDATA[Optical fiber dispersion]]></term>
##       <term><![CDATA[Optical fiber polarization]]></term>
##       <term><![CDATA[Optical noise]]></term>
##       <term><![CDATA[Signal to noise ratio]]></term>
##       <term><![CDATA[Stimulated emission]]></term>
##     </thesaurusterms>
##     <pubtitle><![CDATA[Photonics Technology Letters, IEEE]]></pubtitle>
##     <punumber><![CDATA[68]]></punumber>
##     <pubtype><![CDATA[Journals & Magazines]]></pubtype>
##     <publisher><![CDATA[IEEE]]></publisher>
##     <volume><![CDATA[26]]></volume>
##     <issue><![CDATA[3]]></issue>
##     <py><![CDATA[2014]]></py>
##     <spage><![CDATA[242]]></spage>
##     <epage><![CDATA[245]]></epage>
##     <abstract><![CDATA[We report wavelength-division-multiplexed transmission of ninety 128-Gb/s polarization-multiplexed quadrature-phase-shift-keyed channels in a seamless 73 nm band over 60 &#x00D7; 100 km of ultra-large-area fiber. Co- and counter-pumped all-Raman amplification in the fiber spans, single-stage discrete Raman amplifiers, and broadband wavelength selective switch and channel equalizer were utilized to achieve optical signal-to-noise-ratio margin of more than 3 dB after 6000 km transmission of the 9 Tb/s capacity.]]></abstract>
##     <issn><![CDATA[1041-1135]]></issn>
##     <htmlFlag><![CDATA[1]]></htmlFlag>
##     <arnumber><![CDATA[6670699]]></arnumber>
##     <doi><![CDATA[10.1109/LPT.2013.2291399]]></doi>
##     <publicationId><![CDATA[6670699]]></publicationId>
##     <mdurl><![CDATA[http://ieeexplore.ieee.org/xpl/articleDetails.jsp?tp=&arnumber=6670699&contentType=Journals+%26+Magazines]]></mdurl>
##     <pdf><![CDATA[http://ieeexplore.ieee.org/stamp/stamp.jsp?arnumber=6670699]]></pdf>
##   </document>
##   <document>
##     <rank>1083</rank>
##     <title><![CDATA[A Vacation-Based Performance Analysis of an Energy-Efficient Motorway Vehicular Communication System]]></title>
##     <authors><![CDATA[Kumar, W.;  Bhattacharya, S.;  Qazi, B.R.;  Elmirghani, J.M.H.]]></authors>
##     <affiliations><![CDATA[Dept. of Electron. Eng., Mehran Univ. of Eng. & Technol., Jamshoro, Pakistan]]></affiliations>
##     <controlledterms>
##       <term><![CDATA[mobile radio]]></term>
##       <term><![CDATA[packet reservation multiple access]]></term>
##       <term><![CDATA[power consumption]]></term>
##       <term><![CDATA[quality of service]]></term>
##       <term><![CDATA[teleconferencing]]></term>
##       <term><![CDATA[video communication]]></term>
##     </controlledterms>
##     <thesaurusterms>
##       <term><![CDATA[Analytical models]]></term>
##       <term><![CDATA[Delays]]></term>
##       <term><![CDATA[Protocols]]></term>
##       <term><![CDATA[Quality of service]]></term>
##       <term><![CDATA[Servers]]></term>
##       <term><![CDATA[Vehicles]]></term>
##       <term><![CDATA[Wireless communication]]></term>
##     </thesaurusterms>
##     <pubtitle><![CDATA[Vehicular Technology, IEEE Transactions on]]></pubtitle>
##     <punumber><![CDATA[25]]></punumber>
##     <pubtype><![CDATA[Journals & Magazines]]></pubtype>
##     <publisher><![CDATA[IEEE]]></publisher>
##     <volume><![CDATA[63]]></volume>
##     <issue><![CDATA[4]]></issue>
##     <py><![CDATA[2014]]></py>
##     <spage><![CDATA[1827]]></spage>
##     <epage><![CDATA[1842]]></epage>
##     <abstract><![CDATA[Due to the unprecedented growth in bandwidth requirement, the increasing number of access points (APs) deployed within a macrocell for services such as video conferencing, video gaming, and data off-loading leads to significantly higher energy consumption. This advancement in mobile networks has forced researchers to explore various methods of energy saving, although with little emphasis on motorway vehicular networks where mobility is also an important aspect. Energy saving in these networks is extremely challenging due to the dynamic nature of the environment in which they operate. To analyze such a network, we first develop a performance model for a medium access control (MAC) protocol, namely, the modified version of packet reservation multiple access (M-PRMA) with wireless channel impairments in a motorway vehicular environment. The M-PRMA protocol provides communication links (time slots) between an AP and the vehicles in range. The time slots of the M-PRMA protocol are modeled as servers where each outage of the channel is represented as a server on queue-length-independent vacation. Then, each AP, in a hierarchical micro-macro topology, is modeled as a single-server queue where the AP takes queue-length-dependent vacations (switches to sleep mode) to save energy during its inactivity period, although at the expense of degraded quality of service (QoS). To address this, a number of sleep strategies for the AP are studied. Finally, both of these proposed models (M-PRMA with channel impairments and AP with sleep cycles) are analyzed and verified through simulations. The performance results reveal that the introduction of sleep strategies at an AP can save up to 80% transmission energy during off-peak hours and 66% on average during the day in a motorway vehicular environment while supporting end-to-end QoS for video and audio conferencing applications.]]></abstract>
##     <issn><![CDATA[0018-9545]]></issn>
##     <arnumber><![CDATA[6656956]]></arnumber>
##     <doi><![CDATA[10.1109/TVT.2013.2289889]]></doi>
##     <publicationId><![CDATA[6656956]]></publicationId>
##     <mdurl><![CDATA[http://ieeexplore.ieee.org/xpl/articleDetails.jsp?tp=&arnumber=6656956&contentType=Journals+%26+Magazines]]></mdurl>
##     <pdf><![CDATA[http://ieeexplore.ieee.org/stamp/stamp.jsp?arnumber=6656956]]></pdf>
##   </document>
##   <document>
##     <rank>1084</rank>
##     <title><![CDATA[Equivalence of Virtual Synchronous Machines and Frequency-Droops for Converter-Based MicroGrids]]></title>
##     <authors><![CDATA[D'Arco, S.;  Suul, J.A.]]></authors>
##     <affiliations><![CDATA[SINTEF Energy Res., Trondheim, Norway]]></affiliations>
##     <controlledterms>
##       <term><![CDATA[distributed power generation]]></term>
##       <term><![CDATA[frequency control]]></term>
##       <term><![CDATA[power convertors]]></term>
##       <term><![CDATA[power generation control]]></term>
##       <term><![CDATA[synchronous machines]]></term>
##     </controlledterms>
##     <thesaurusterms>
##       <term><![CDATA[Converters]]></term>
##       <term><![CDATA[Equations]]></term>
##       <term><![CDATA[Frequency control]]></term>
##       <term><![CDATA[Mathematical model]]></term>
##       <term><![CDATA[Microgrids]]></term>
##       <term><![CDATA[Synchronous machines]]></term>
##       <term><![CDATA[Voltage control]]></term>
##     </thesaurusterms>
##     <pubtitle><![CDATA[Smart Grid, IEEE Transactions on]]></pubtitle>
##     <punumber><![CDATA[5165411]]></punumber>
##     <pubtype><![CDATA[Journals & Magazines]]></pubtype>
##     <publisher><![CDATA[IEEE]]></publisher>
##     <volume><![CDATA[5]]></volume>
##     <issue><![CDATA[1]]></issue>
##     <py><![CDATA[2014]]></py>
##     <spage><![CDATA[394]]></spage>
##     <epage><![CDATA[395]]></epage>
##     <abstract><![CDATA[Over the last decade, frequency-droop-based control schemes have become the preferred solution in microgrids dominated by power electronic converters. More recently, the concept of virtual synchronous machines (VSMs) has emerged as an effective method for adding virtual inertia to the power system through the control of power electronic converters. These two approaches have been developed in two separate contexts, but present strong similarities. In fact, they are equivalent under certain conditions, as demonstrated in this letter. Analysis of this equivalence provides additional physics-based insight into the tuning and operation of both types of controllers.]]></abstract>
##     <issn><![CDATA[1949-3053]]></issn>
##     <htmlFlag><![CDATA[1]]></htmlFlag>
##     <arnumber><![CDATA[6683080]]></arnumber>
##     <doi><![CDATA[10.1109/TSG.2013.2288000]]></doi>
##     <publicationId><![CDATA[6683080]]></publicationId>
##     <mdurl><![CDATA[http://ieeexplore.ieee.org/xpl/articleDetails.jsp?tp=&arnumber=6683080&contentType=Journals+%26+Magazines]]></mdurl>
##     <pdf><![CDATA[http://ieeexplore.ieee.org/stamp/stamp.jsp?arnumber=6683080]]></pdf>
##   </document>
##   <document>
##     <rank>1085</rank>
##     <title><![CDATA[Optical Vehicle-to-Vehicle Communication System Using LED Transmitter and Camera Receiver]]></title>
##     <authors><![CDATA[Takai, I.;  Harada, T.;  Andoh, M.;  Yasutomi, K.;  Kagawa, K.;  Kawahito, S.]]></authors>
##     <affiliations><![CDATA[Syst. & Electron. Eng. Dept. 2, Toyota Central R&D Labs., Inc., Nagakute, Japan]]></affiliations>
##     <controlledterms>
##       <term><![CDATA[CMOS image sensors]]></term>
##       <term><![CDATA[cameras]]></term>
##       <term><![CDATA[light emitting diodes]]></term>
##       <term><![CDATA[optical receivers]]></term>
##       <term><![CDATA[optical transmitters]]></term>
##       <term><![CDATA[vehicles]]></term>
##     </controlledterms>
##     <thesaurusterms>
##       <term><![CDATA[Cameras]]></term>
##       <term><![CDATA[Light emitting diodes]]></term>
##       <term><![CDATA[Optical imaging]]></term>
##       <term><![CDATA[Optical receivers]]></term>
##       <term><![CDATA[Optical sensors]]></term>
##       <term><![CDATA[Optical transmitters]]></term>
##     </thesaurusterms>
##     <pubtitle><![CDATA[Photonics Journal, IEEE]]></pubtitle>
##     <punumber><![CDATA[4563994]]></punumber>
##     <pubtype><![CDATA[Journals & Magazines]]></pubtype>
##     <publisher><![CDATA[IEEE]]></publisher>
##     <volume><![CDATA[6]]></volume>
##     <issue><![CDATA[5]]></issue>
##     <py><![CDATA[2014]]></py>
##     <spage><![CDATA[1]]></spage>
##     <epage><![CDATA[14]]></epage>
##     <abstract><![CDATA[This paper introduces an optical vehicle-to-vehicle (V2V) communication system based on an optical wireless communication technology using an LED transmitter and a camera receiver, which employs a special CMOS image sensor, i.e, an optical communication image sensor (OCI). The OCI has a &#x201C;communication pixel (CPx)&#x201D; that can promptly respond to light intensity variations and an output circuit of a &#x201C;flag image&#x201D; in which only high-intensity light sources, such as LEDs, have emerged. The OCI that employs these two technologies provides capabilities for a 10-Mb/s optical signal reception and real-time LED detection to the camera receiver. The optical V2V communication system consisting of the LED transmitters mounted on a leading vehicle and the camera receiver mounted on a following vehicle is constructed, and various experiments are conducted under real driving and outdoor lighting conditions. Due to the LED detection method using the flag image, the camera receiver correctly detects LEDs, in real time, in challenging outdoor conditions. Furthermore, between two vehicles, various vehicle internal data (such as speed) and image data (320 &#x00D7; 240, color) are transmitted successfully, and the 13.0-fps image data reception is achieved while driving outside.]]></abstract>
##     <issn><![CDATA[1943-0655]]></issn>
##     <htmlFlag><![CDATA[1]]></htmlFlag>
##     <arnumber><![CDATA[6887317]]></arnumber>
##     <doi><![CDATA[10.1109/JPHOT.2014.2352620]]></doi>
##     <publicationId><![CDATA[6887317]]></publicationId>
##     <mdurl><![CDATA[http://ieeexplore.ieee.org/xpl/articleDetails.jsp?tp=&arnumber=6887317&contentType=Journals+%26+Magazines]]></mdurl>
##     <pdf><![CDATA[http://ieeexplore.ieee.org/stamp/stamp.jsp?arnumber=6887317]]></pdf>
##   </document>
##   <document>
##     <rank>1086</rank>
##     <title><![CDATA[Selected diagnostic methods of electrical machines operating in industrial conditions]]></title>
##     <authors><![CDATA[Baranski, M.;  Decner, A.;  Polak, A.]]></authors>
##     <affiliations><![CDATA[Inst. of Electr. Drives & Machines KOMEL, Katowice, Poland]]></affiliations>
##     <controlledterms>
##       <term><![CDATA[condition monitoring]]></term>
##       <term><![CDATA[electric machines]]></term>
##       <term><![CDATA[insulation testing]]></term>
##       <term><![CDATA[machine insulation]]></term>
##       <term><![CDATA[production equipment]]></term>
##       <term><![CDATA[vibration measurement]]></term>
##     </controlledterms>
##     <thesaurusterms>
##       <term><![CDATA[Coils]]></term>
##       <term><![CDATA[Electrical resistance measurement]]></term>
##       <term><![CDATA[Insulation]]></term>
##       <term><![CDATA[Resistance]]></term>
##       <term><![CDATA[Vibrations]]></term>
##       <term><![CDATA[Voltage measurement]]></term>
##       <term><![CDATA[Windings]]></term>
##     </thesaurusterms>
##     <pubtitle><![CDATA[Dielectrics and Electrical Insulation, IEEE Transactions on]]></pubtitle>
##     <punumber><![CDATA[94]]></punumber>
##     <pubtype><![CDATA[Journals & Magazines]]></pubtype>
##     <publisher><![CDATA[IEEE]]></publisher>
##     <volume><![CDATA[21]]></volume>
##     <issue><![CDATA[5]]></issue>
##     <py><![CDATA[2014]]></py>
##     <spage><![CDATA[2047]]></spage>
##     <epage><![CDATA[2054]]></epage>
##     <abstract><![CDATA[The article proposes a series of tests on the groundwall insulation, utilizing the dc voltage and consisting of three related procedures. After running the tests, the obtained data are analyzed and insulation is graded in accordance with proposed criteria. These criteria are presented in the paper. The method has been illustrated by real-life example of electrical machine tests' results. The article describes the turn-to-turn insulation substitute circuit and results of simulations and measurements. There is shown a new diagnostic method to determinate condition of this insulation. The suggested method can be used in any kind of machine and it is easy to use. The most commonly damaged machine parts, except the insulation, are bearings, so the authors decided to mention in the article, about vibration measurements.]]></abstract>
##     <issn><![CDATA[1070-9878]]></issn>
##     <htmlFlag><![CDATA[1]]></htmlFlag>
##     <arnumber><![CDATA[6927332]]></arnumber>
##     <doi><![CDATA[10.1109/TDEI.2014.004602]]></doi>
##     <publicationId><![CDATA[6927332]]></publicationId>
##     <mdurl><![CDATA[http://ieeexplore.ieee.org/xpl/articleDetails.jsp?tp=&arnumber=6927332&contentType=Journals+%26+Magazines]]></mdurl>
##     <pdf><![CDATA[http://ieeexplore.ieee.org/stamp/stamp.jsp?arnumber=6927332]]></pdf>
##   </document>
##   <document>
##     <rank>1087</rank>
##     <title><![CDATA[X-Ray Luminescence and X-Ray Fluorescence Computed Tomography: New Molecular Imaging Modalities]]></title>
##     <authors><![CDATA[Ahmad, M.;  Pratx, G.;  Bazalova, M.;  Lei Xing]]></authors>
##     <affiliations><![CDATA[Dept. of Radiat. Oncology, Stanford Univ., Stanford, CA, USA]]></affiliations>
##     <controlledterms>
##       <term><![CDATA[X-ray fluorescence analysis]]></term>
##       <term><![CDATA[computerised tomography]]></term>
##       <term><![CDATA[image reconstruction]]></term>
##       <term><![CDATA[medical image processing]]></term>
##       <term><![CDATA[reviews]]></term>
##     </controlledterms>
##     <thesaurusterms>
##       <term><![CDATA[Computed tomography]]></term>
##       <term><![CDATA[Emissions]]></term>
##       <term><![CDATA[Luminescence]]></term>
##       <term><![CDATA[Optical imaging]]></term>
##       <term><![CDATA[Optical sensors]]></term>
##       <term><![CDATA[Stimulated emission]]></term>
##       <term><![CDATA[X-ray imaging]]></term>
##     </thesaurusterms>
##     <pubtitle><![CDATA[Access, IEEE]]></pubtitle>
##     <punumber><![CDATA[6287639]]></punumber>
##     <pubtype><![CDATA[Journals & Magazines]]></pubtype>
##     <publisher><![CDATA[IEEE]]></publisher>
##     <volume><![CDATA[2]]></volume>
##     <py><![CDATA[2014]]></py>
##     <spage><![CDATA[1051]]></spage>
##     <epage><![CDATA[1061]]></epage>
##     <abstract><![CDATA[X-ray luminescence and X-ray fluorescence computed tomography (CT) are two emerging technologies in X-ray imaging that provide functional and molecular imaging capability. Both emission-type tomographic imaging modalities use external X-rays to stimulate secondary emissions, either light or secondary X-rays, which are then acquired for tomographic reconstruction. These modalities surpass the limits of sensitivity in current X-ray imaging and have the potential of enabling X-ray imaging to extract molecular imaging information. These new modalities also promise to break through the spatial resolution limits of other in vivo molecular imaging modalities. This paper reviews the development of X-ray luminescence and X-ray fluorescence CT and their relative merits. The discussion includes current problems and future research directions and the role of these modalities in future molecular imaging applications.]]></abstract>
##     <issn><![CDATA[2169-3536]]></issn>
##     <htmlFlag><![CDATA[1]]></htmlFlag>
##     <arnumber><![CDATA[6891106]]></arnumber>
##     <doi><![CDATA[10.1109/ACCESS.2014.2353041]]></doi>
##     <publicationId><![CDATA[6891106]]></publicationId>
##     <mdurl><![CDATA[http://ieeexplore.ieee.org/xpl/articleDetails.jsp?tp=&arnumber=6891106&contentType=Journals+%26+Magazines]]></mdurl>
##     <pdf><![CDATA[http://ieeexplore.ieee.org/stamp/stamp.jsp?arnumber=6891106]]></pdf>
##   </document>
##   <document>
##     <rank>1088</rank>
##     <title><![CDATA[Model-Based Nonlinear Embedding for Power-Amplifier Design]]></title>
##     <authors><![CDATA[Haedong Jang;  Roblin, P.;  Zhijian Xie]]></authors>
##     <affiliations><![CDATA[Dept. of Electr. & Comput. Eng., Ohio State Univ., Columbus, OH, USA]]></affiliations>
##     <controlledterms>
##       <term><![CDATA[III-V semiconductors]]></term>
##       <term><![CDATA[gallium compounds]]></term>
##       <term><![CDATA[high electron mobility transistors]]></term>
##       <term><![CDATA[nonlinear network analysis]]></term>
##       <term><![CDATA[radiofrequency power amplifiers]]></term>
##     </controlledterms>
##     <thesaurusterms>
##       <term><![CDATA[Harmonic analysis]]></term>
##       <term><![CDATA[Impedance]]></term>
##       <term><![CDATA[Integrated circuit modeling]]></term>
##       <term><![CDATA[Load modeling]]></term>
##       <term><![CDATA[Logic gates]]></term>
##       <term><![CDATA[Mathematical model]]></term>
##       <term><![CDATA[Power generation]]></term>
##     </thesaurusterms>
##     <pubtitle><![CDATA[Microwave Theory and Techniques, IEEE Transactions on]]></pubtitle>
##     <punumber><![CDATA[22]]></punumber>
##     <pubtype><![CDATA[Journals & Magazines]]></pubtype>
##     <publisher><![CDATA[IEEE]]></publisher>
##     <volume><![CDATA[62]]></volume>
##     <issue><![CDATA[9]]></issue>
##     <py><![CDATA[2014]]></py>
##     <spage><![CDATA[1986]]></spage>
##     <epage><![CDATA[2002]]></epage>
##     <abstract><![CDATA[A fully model-based nonlinear embedding device model including low- and high-frequency dispersion effects is implemented for the Angelov device model and successfully demonstrated for load modulation power-amplifier (PA) applications. Using this nonlinear embedding device model, any desired PA mode of operation at the current source plane can be projected to the external reference planes to synthesize the required multi-harmonic source and load terminations. A 2-D identification of the intrinsic PA operation modes is performed first at the current source reference planes. For intrinsic modes defined without lossy parasitics, most of the required source impedance terminations will exhibit a substantial negative resistance after projection to the external reference planes. These terminations can then be implemented by active harmonic injection at the input. It is verified experimentally for a 15-W GaN HEMT class-AB mode that, using the second harmonic injection synthesized by the embedding device model at the input, yields an improved drain efficiency of up to 5% in agreement with the simulation. A figure-of-merit is also introduced to evaluate the efficacy of the nonlinear embedding PA design methodology in achieving the targeted intrinsic mode operation given the model accuracy.]]></abstract>
##     <issn><![CDATA[0018-9480]]></issn>
##     <htmlFlag><![CDATA[1]]></htmlFlag>
##     <arnumber><![CDATA[6851951]]></arnumber>
##     <doi><![CDATA[10.1109/TMTT.2014.2333498]]></doi>
##     <publicationId><![CDATA[6851951]]></publicationId>
##     <mdurl><![CDATA[http://ieeexplore.ieee.org/xpl/articleDetails.jsp?tp=&arnumber=6851951&contentType=Journals+%26+Magazines]]></mdurl>
##     <pdf><![CDATA[http://ieeexplore.ieee.org/stamp/stamp.jsp?arnumber=6851951]]></pdf>
##   </document>
##   <document>
##     <rank>1089</rank>
##     <title><![CDATA[Bipolar Piezoelectric Buckling Actuators]]></title>
##     <authors><![CDATA[Neal, D.M.;  Asada, H.H.]]></authors>
##     <affiliations><![CDATA[Dept. of Mech. Eng., Massachusetts Inst. of Technol., Cambridge, MA, USA]]></affiliations>
##     <controlledterms>
##       <term><![CDATA[amplifiers]]></term>
##       <term><![CDATA[bending]]></term>
##       <term><![CDATA[buckling]]></term>
##       <term><![CDATA[displacement measurement]]></term>
##       <term><![CDATA[force measurement]]></term>
##       <term><![CDATA[piezoelectric actuators]]></term>
##       <term><![CDATA[piezoelectric oscillations]]></term>
##     </controlledterms>
##     <thesaurusterms>
##       <term><![CDATA[Actuators]]></term>
##       <term><![CDATA[Force]]></term>
##       <term><![CDATA[Joints]]></term>
##       <term><![CDATA[Kinematics]]></term>
##       <term><![CDATA[Springs]]></term>
##       <term><![CDATA[Strain]]></term>
##     </thesaurusterms>
##     <pubtitle><![CDATA[Mechatronics, IEEE/ASME Transactions on]]></pubtitle>
##     <punumber><![CDATA[3516]]></punumber>
##     <pubtype><![CDATA[Journals & Magazines]]></pubtype>
##     <publisher><![CDATA[IEEE]]></publisher>
##     <volume><![CDATA[19]]></volume>
##     <issue><![CDATA[1]]></issue>
##     <py><![CDATA[2014]]></py>
##     <spage><![CDATA[9]]></spage>
##     <epage><![CDATA[19]]></epage>
##     <abstract><![CDATA[A nonlinear piezoelectric amplification mechanism utilizing structural buckling is presented, and its static and dynamic properties are measured and analyzed. Buckling is a pronounced nonlinear effect that occurs at a structurally singular point. A small piezoelectric displacement on the order of 10 &#x03BC;m results in a large buckling displacement on the order of millimeter. Furthermore, the usable stroke is doubled if both sides of the singular point can be reached resulting in bipolar motion. Despite the large gain, buckling is an erratic, singular phenomenon; the side on which deflection will occur is unpredictable. In this paper, multiple design concepts are presented for regulating the buckling direction as well as for extending its usable stroke to bipolar motion. Nonlinear force-displacement relationships are modeled and measured. Nonlinear dynamic analysis using phase planes reveals that the buckling actuator can generate bipolar motion above a specific amplitude. Below this amplitude, it generates only monopolar oscillation. The proposed design concepts are implemented on monolithic flexure mechanisms, and prototype buckling actuators are tested to verify the concepts. Experiments show promising results: 20 N of peak-to-peak output force, and 6.2 mm of bipolar displacement generated by piezoelectric actuators with free displacement of 42 &#x03BC;m.]]></abstract>
##     <issn><![CDATA[1083-4435]]></issn>
##     <htmlFlag><![CDATA[1]]></htmlFlag>
##     <arnumber><![CDATA[6311477]]></arnumber>
##     <doi><![CDATA[10.1109/TMECH.2012.2216888]]></doi>
##     <publicationId><![CDATA[6311477]]></publicationId>
##     <mdurl><![CDATA[http://ieeexplore.ieee.org/xpl/articleDetails.jsp?tp=&arnumber=6311477&contentType=Journals+%26+Magazines]]></mdurl>
##     <pdf><![CDATA[http://ieeexplore.ieee.org/stamp/stamp.jsp?arnumber=6311477]]></pdf>
##   </document>
##   <document>
##     <rank>1090</rank>
##     <title><![CDATA[Reliability of Capacitors for DC-Link Applications in Power Electronic Converters&#x2014;An Overview]]></title>
##     <authors><![CDATA[Huai Wang;  Blaabjerg, F.]]></authors>
##     <affiliations><![CDATA[Dept. of Energy Technol., Aalborg Univ., Aalborg, Denmark]]></affiliations>
##     <controlledterms>
##       <term><![CDATA[ceramic capacitors]]></term>
##       <term><![CDATA[condition monitoring]]></term>
##       <term><![CDATA[failure analysis]]></term>
##       <term><![CDATA[power convertors]]></term>
##       <term><![CDATA[reliability]]></term>
##     </controlledterms>
##     <thesaurusterms>
##       <term><![CDATA[Capacitance]]></term>
##       <term><![CDATA[Capacitors]]></term>
##       <term><![CDATA[Films]]></term>
##       <term><![CDATA[Integrated circuits]]></term>
##       <term><![CDATA[Reliability]]></term>
##       <term><![CDATA[Stress]]></term>
##     </thesaurusterms>
##     <pubtitle><![CDATA[Industry Applications, IEEE Transactions on]]></pubtitle>
##     <punumber><![CDATA[28]]></punumber>
##     <pubtype><![CDATA[Journals & Magazines]]></pubtype>
##     <publisher><![CDATA[IEEE]]></publisher>
##     <volume><![CDATA[50]]></volume>
##     <issue><![CDATA[5]]></issue>
##     <py><![CDATA[2014]]></py>
##     <spage><![CDATA[3569]]></spage>
##     <epage><![CDATA[3578]]></epage>
##     <abstract><![CDATA[DC-link capacitors are an important part in the majority of power electronic converters which contribute to cost, size and failure rate on a considerable scale. From capacitor users' viewpoint, this paper presents a review on the improvement of reliability of dc link in power electronic converters from two aspects: 1) reliability-oriented dc-link design solutions; 2) conditioning monitoring of dc-link capacitors during operation. Failure mechanisms, failure modes and lifetime models of capacitors suitable for the applications are also discussed as a basis to understand the physics-of-failure. This review serves to provide a clear picture of the state-of-the-art research in this area and to identify the corresponding challenges and future research directions for capacitors and their dc-link applications.]]></abstract>
##     <issn><![CDATA[0093-9994]]></issn>
##     <htmlFlag><![CDATA[1]]></htmlFlag>
##     <arnumber><![CDATA[6748007]]></arnumber>
##     <doi><![CDATA[10.1109/TIA.2014.2308357]]></doi>
##     <publicationId><![CDATA[6748007]]></publicationId>
##     <mdurl><![CDATA[http://ieeexplore.ieee.org/xpl/articleDetails.jsp?tp=&arnumber=6748007&contentType=Journals+%26+Magazines]]></mdurl>
##     <pdf><![CDATA[http://ieeexplore.ieee.org/stamp/stamp.jsp?arnumber=6748007]]></pdf>
##   </document>
##   <document>
##     <rank>1091</rank>
##     <title><![CDATA[All-Depolarized Interferometric Fiber-Optic Gyroscope Based on Optical Compensation]]></title>
##     <authors><![CDATA[Zinan Wang;  Yi Yang;  Ping Lu;  Yongxiao Li;  Dayu Zhao;  Chao Peng;  Zhenrong Zhang;  Zhengbin Li]]></authors>
##     <affiliations><![CDATA[Dept. of Electron., Peking Univ., Beijing, China]]></affiliations>
##     <controlledterms>
##       <term><![CDATA[fibre optic sensors]]></term>
##       <term><![CDATA[light interferometry]]></term>
##     </controlledterms>
##     <thesaurusterms>
##       <term><![CDATA[Adaptive optics]]></term>
##       <term><![CDATA[Coils]]></term>
##       <term><![CDATA[Licenses]]></term>
##       <term><![CDATA[Optical fiber sensors]]></term>
##       <term><![CDATA[Optical interferometry]]></term>
##       <term><![CDATA[Optical polarization]]></term>
##     </thesaurusterms>
##     <pubtitle><![CDATA[Photonics Journal, IEEE]]></pubtitle>
##     <punumber><![CDATA[4563994]]></punumber>
##     <pubtype><![CDATA[Journals & Magazines]]></pubtype>
##     <publisher><![CDATA[IEEE]]></publisher>
##     <volume><![CDATA[6]]></volume>
##     <issue><![CDATA[1]]></issue>
##     <py><![CDATA[2014]]></py>
##     <spage><![CDATA[1]]></spage>
##     <epage><![CDATA[8]]></epage>
##     <abstract><![CDATA[We propose and demonstrate a novel configuration for depolarized interferometric fiber-optic gyroscopes (IFOGs). This configuration utilizes optical compensation between two orthogonal polarizations for suppressing errors induced by polarization nonreciprocity. Theoretical analysis shows that it is a new approach different from conventional IFOGs where a polarizer is mandatory. An experimental demonstration of the proposed IFOG (2097-m coil, open-loop configuration) achieves a low bias drift of 0.016 &#x00B0;/h in detecting the Earth's rotation rate. Furthermore, this configuration requires no polarizer or any other polarization-maintaining device.]]></abstract>
##     <issn><![CDATA[1943-0655]]></issn>
##     <htmlFlag><![CDATA[1]]></htmlFlag>
##     <arnumber><![CDATA[6708419]]></arnumber>
##     <doi><![CDATA[10.1109/JPHOT.2014.2299212]]></doi>
##     <publicationId><![CDATA[6708419]]></publicationId>
##     <mdurl><![CDATA[http://ieeexplore.ieee.org/xpl/articleDetails.jsp?tp=&arnumber=6708419&contentType=Journals+%26+Magazines]]></mdurl>
##     <pdf><![CDATA[http://ieeexplore.ieee.org/stamp/stamp.jsp?arnumber=6708419]]></pdf>
##   </document>
##   <document>
##     <rank>1092</rank>
##     <title><![CDATA[A Novel Solar Simulator Based on a Supercontinuum Laser for Solar Cell Device and Materials Characterization]]></title>
##     <authors><![CDATA[Dennis, T.;  Schlager, J.B.;  Bertness, K.A.]]></authors>
##     <affiliations><![CDATA[Nat. Inst. of Stand. & Technol., Boulder, CO, USA]]></affiliations>
##     <controlledterms>
##       <term><![CDATA[III-V semiconductors]]></term>
##       <term><![CDATA[amorphous semiconductors]]></term>
##       <term><![CDATA[copper compounds]]></term>
##       <term><![CDATA[elemental semiconductors]]></term>
##       <term><![CDATA[fibre lasers]]></term>
##       <term><![CDATA[gallium arsenide]]></term>
##       <term><![CDATA[gallium compounds]]></term>
##       <term><![CDATA[indium compounds]]></term>
##       <term><![CDATA[semiconductor junctions]]></term>
##       <term><![CDATA[semiconductor thin films]]></term>
##       <term><![CDATA[silicon]]></term>
##       <term><![CDATA[solar cells]]></term>
##       <term><![CDATA[ternary semiconductors]]></term>
##     </controlledterms>
##     <thesaurusterms>
##       <term><![CDATA[Current measurement]]></term>
##       <term><![CDATA[Gallium arsenide]]></term>
##       <term><![CDATA[Laser beams]]></term>
##       <term><![CDATA[Measurement by laser beam]]></term>
##       <term><![CDATA[NIST]]></term>
##       <term><![CDATA[Photovoltaic cells]]></term>
##     </thesaurusterms>
##     <pubtitle><![CDATA[Photovoltaics, IEEE Journal of]]></pubtitle>
##     <punumber><![CDATA[5503869]]></punumber>
##     <pubtype><![CDATA[Journals & Magazines]]></pubtype>
##     <publisher><![CDATA[IEEE]]></publisher>
##     <volume><![CDATA[4]]></volume>
##     <issue><![CDATA[4]]></issue>
##     <py><![CDATA[2014]]></py>
##     <spage><![CDATA[1119]]></spage>
##     <epage><![CDATA[1127]]></epage>
##     <abstract><![CDATA[The design, operation, and application of a novel solar simulator based on a high-power supercontinuum fiber laser are described. The simulator features a multisun irradiance with continuous spectral coverage from the visible to the infrared. By use of a prism-based spectral shaper, the simulator can be matched to any desired spectral profile, including the ASTM G-173-03 air-mass 1.5 reference spectrum. The simulator was used to measure the efficiency of gallium arsenide (GaAs), crystalline silicon (Si), amorphous Si, and copper-indium-gallium-selenide (CIGS) thin-film solar cells, showing agreement with independent measurements. The pulsed temporal characteristic of the simulator was studied and would appear to have a negligible influence on measured cell efficiency. The simulator light was focused to a spot of approximately 8 &#x03BC;m in diameter and used to create micrometer-scale spatial maps of full spectrum optical-beam-induced current. Microscopic details such as grid lines, damage spots, and material variations were selectively excited and resolved on GaAs and CIGS cells. The spectral shaping capabilities were used to create output spectra appropriate for selectively light-biasing multijunction cell layers. The simulator was used to create variable blue-rich and red-rich spectra that were applied to a GaInP/GaAs tandem solar cell to illustrate the current-limiting behavior.]]></abstract>
##     <issn><![CDATA[2156-3381]]></issn>
##     <htmlFlag><![CDATA[1]]></htmlFlag>
##     <arnumber><![CDATA[6821268]]></arnumber>
##     <doi><![CDATA[10.1109/JPHOTOV.2014.2321659]]></doi>
##     <publicationId><![CDATA[6821268]]></publicationId>
##     <mdurl><![CDATA[http://ieeexplore.ieee.org/xpl/articleDetails.jsp?tp=&arnumber=6821268&contentType=Journals+%26+Magazines]]></mdurl>
##     <pdf><![CDATA[http://ieeexplore.ieee.org/stamp/stamp.jsp?arnumber=6821268]]></pdf>
##   </document>
##   <document>
##     <rank>1093</rank>
##     <title><![CDATA[The Limited Relevance of SWE Dangling Bonds to Degradation in High-Quality a-Si:H Solar Cells]]></title>
##     <authors><![CDATA[Wronski, C.R.;  Xinwei Niu]]></authors>
##     <affiliations><![CDATA[Center for Thin Film Devices, Pennsylvania State Univ., University Park, PA, USA]]></affiliations>
##     <controlledterms>
##       <term><![CDATA[Hall effect]]></term>
##       <term><![CDATA[amorphous semiconductors]]></term>
##       <term><![CDATA[carrier lifetime]]></term>
##       <term><![CDATA[dangling bonds]]></term>
##       <term><![CDATA[defect states]]></term>
##       <term><![CDATA[electron-hole recombination]]></term>
##       <term><![CDATA[elemental semiconductors]]></term>
##       <term><![CDATA[energy gap]]></term>
##       <term><![CDATA[hydrogen]]></term>
##       <term><![CDATA[nanofabrication]]></term>
##       <term><![CDATA[nanostructured materials]]></term>
##       <term><![CDATA[passivation]]></term>
##       <term><![CDATA[photoconductivity]]></term>
##       <term><![CDATA[plasma CVD]]></term>
##       <term><![CDATA[semiconductor growth]]></term>
##       <term><![CDATA[semiconductor thin films]]></term>
##       <term><![CDATA[silicon]]></term>
##       <term><![CDATA[solar cells]]></term>
##       <term><![CDATA[vacancies (crystal)]]></term>
##     </controlledterms>
##     <thesaurusterms>
##       <term><![CDATA[Annealing]]></term>
##       <term><![CDATA[Degradation]]></term>
##       <term><![CDATA[Lighting]]></term>
##       <term><![CDATA[Photoconductivity]]></term>
##       <term><![CDATA[Photovoltaic cells]]></term>
##       <term><![CDATA[Spontaneous emission]]></term>
##       <term><![CDATA[Sun]]></term>
##     </thesaurusterms>
##     <pubtitle><![CDATA[Photovoltaics, IEEE Journal of]]></pubtitle>
##     <punumber><![CDATA[5503869]]></punumber>
##     <pubtype><![CDATA[Journals & Magazines]]></pubtype>
##     <publisher><![CDATA[IEEE]]></publisher>
##     <volume><![CDATA[4]]></volume>
##     <issue><![CDATA[3]]></issue>
##     <py><![CDATA[2014]]></py>
##     <spage><![CDATA[778]]></spage>
##     <epage><![CDATA[784]]></epage>
##     <abstract><![CDATA[Contributions of different light-induced defect states to degradation of solar cells have been established for high-quality p-i-n solar cells with i layers of protocrystalline a-Si:H deposited at very low rates, whose nanostructure is dominated by hydrogen-passivated divacancies. Nature of the different light-induced gap states and their respective roles as electron and hole recombination centers were characterized in the thin films from their photocurrents, and in corresponding solar cells from their Shockley-Reed-Hall carrier recombination currents. The results were directly related to three light-induced states, with &#x201C;A&#x201D; and &#x201C;B&#x201D; within 0.2 eV and &#x201C;C&#x201D; 0.4 eV below midgap, identified from subgap absorption. The A and B states are efficient electron, while the C states are very efficient hole recombination centers. Under 1-sun illumination, the former dominate the electron lifetimes, while the latter are key to solar cell operation as is confirmed by the direct correlation of their creation with the degradation of VOC and 1-sun fill factor (FF). It is also shown that the apparent correlation found earlier between the cell FF and electron lifetimes is due to the same long-term degradation kinetics of the light-induced changes in the B t and C states.]]></abstract>
##     <issn><![CDATA[2156-3381]]></issn>
##     <htmlFlag><![CDATA[1]]></htmlFlag>
##     <arnumber><![CDATA[6783732]]></arnumber>
##     <doi><![CDATA[10.1109/JPHOTOV.2014.2311498]]></doi>
##     <publicationId><![CDATA[6783732]]></publicationId>
##     <mdurl><![CDATA[http://ieeexplore.ieee.org/xpl/articleDetails.jsp?tp=&arnumber=6783732&contentType=Journals+%26+Magazines]]></mdurl>
##     <pdf><![CDATA[http://ieeexplore.ieee.org/stamp/stamp.jsp?arnumber=6783732]]></pdf>
##   </document>
##   <document>
##     <rank>1094</rank>
##     <title><![CDATA[Transmit-Power Efficient Linear Precoding Utilizing Known Interference for the Multiantenna Downlink]]></title>
##     <authors><![CDATA[Razavi, S.M.;  Ratnarajah, T.;  Masouros, C.]]></authors>
##     <affiliations><![CDATA[Inst. for Digital Commun., Univ. of Edinburgh, Edinburgh, UK]]></affiliations>
##     <controlledterms>
##       <term><![CDATA[antenna arrays]]></term>
##       <term><![CDATA[energy conservation]]></term>
##       <term><![CDATA[error statistics]]></term>
##       <term><![CDATA[precoding]]></term>
##       <term><![CDATA[radiofrequency interference]]></term>
##       <term><![CDATA[telecommunication power management]]></term>
##       <term><![CDATA[wireless channels]]></term>
##     </controlledterms>
##     <thesaurusterms>
##       <term><![CDATA[Covariance matrices]]></term>
##       <term><![CDATA[Downlink]]></term>
##       <term><![CDATA[Interference]]></term>
##       <term><![CDATA[Receiving antennas]]></term>
##       <term><![CDATA[Signal to noise ratio]]></term>
##       <term><![CDATA[Transmitting antennas]]></term>
##       <term><![CDATA[Vectors]]></term>
##     </thesaurusterms>
##     <pubtitle><![CDATA[Vehicular Technology, IEEE Transactions on]]></pubtitle>
##     <punumber><![CDATA[25]]></punumber>
##     <pubtype><![CDATA[Journals & Magazines]]></pubtype>
##     <publisher><![CDATA[IEEE]]></publisher>
##     <volume><![CDATA[63]]></volume>
##     <issue><![CDATA[9]]></issue>
##     <py><![CDATA[2014]]></py>
##     <spage><![CDATA[4383]]></spage>
##     <epage><![CDATA[4394]]></epage>
##     <abstract><![CDATA[It has been shown that the knowledge of both channel and data information at the base station prior to downlink transmission can help increase the received signal-to-noise ratio (SNR) of each user without the need to increase the transmitted power. Achievability is based on the idea of phase alignment (PA) precoding, where instead of nulling out the destructive interference, it judiciously rotates the phases of the transmitted symbols. In this way, they add up coherently at the intended user and yield higher received SNRs. In addition, it is well known that regularized channel inversion (RCI) precoding improves the performance of channel inversion (CI) in multiantenna downlink communications. In line with this and similar to the RCI precoding, in this paper, we propose the idea of regularized PA (RPA), which is shown to improve the performance of original PA precoding. To do this, we first rectify the original PA precoding, deriving a closed-form expression to evaluate the amount of transmit-power reduction achieved for the same average output SNR compared with CI precoding. We then use this new analysis to select the appropriate regularization factor for our proposed RPA scheme. It is shown by means of theoretical analysis and simulations that the proposed RPA precoding outperforms CI, RCI, and PA precoders from both symbol error rate (SER) and throughput perspectives and provides a more power-efficient alternative. This is particularly pronounced as the number of transmit antennas becomes larger, where up to a 50-times reduction in the transmit power is achieved by RPA (PA) compared with RCI (CI) precoding for a given performance.]]></abstract>
##     <issn><![CDATA[0018-9545]]></issn>
##     <htmlFlag><![CDATA[1]]></htmlFlag>
##     <arnumber><![CDATA[6799269]]></arnumber>
##     <doi><![CDATA[10.1109/TVT.2014.2317716]]></doi>
##     <publicationId><![CDATA[6799269]]></publicationId>
##     <mdurl><![CDATA[http://ieeexplore.ieee.org/xpl/articleDetails.jsp?tp=&arnumber=6799269&contentType=Journals+%26+Magazines]]></mdurl>
##     <pdf><![CDATA[http://ieeexplore.ieee.org/stamp/stamp.jsp?arnumber=6799269]]></pdf>
##   </document>
##   <document>
##     <rank>1095</rank>
##     <title><![CDATA[Power Budget Improved Symmetric 40-Gb/s Long Reach Stacked WDM-OFDM-PON System Based on Single Tunable Optical Filter]]></title>
##     <authors><![CDATA[Meihua Bi;  Shilin Xiao;  Hao He;  Jun Li;  Ling Liu;  Weisheng Hu]]></authors>
##     <affiliations><![CDATA[Dept. of Electron. Eng., Shanghai Jiao Tong Univ., Shanghai, China]]></affiliations>
##     <controlledterms>
##       <term><![CDATA[OFDM modulation]]></term>
##       <term><![CDATA[amplitude shift keying]]></term>
##       <term><![CDATA[optical communication equipment]]></term>
##       <term><![CDATA[optical filters]]></term>
##       <term><![CDATA[passive optical networks]]></term>
##       <term><![CDATA[wavelength division multiplexing]]></term>
##     </controlledterms>
##     <thesaurusterms>
##       <term><![CDATA[OFDM]]></term>
##       <term><![CDATA[Optical filters]]></term>
##       <term><![CDATA[Optical network units]]></term>
##       <term><![CDATA[Optical transmitters]]></term>
##       <term><![CDATA[Passive optical networks]]></term>
##       <term><![CDATA[Wavelength division multiplexing]]></term>
##     </thesaurusterms>
##     <pubtitle><![CDATA[Photonics Journal, IEEE]]></pubtitle>
##     <punumber><![CDATA[4563994]]></punumber>
##     <pubtype><![CDATA[Journals & Magazines]]></pubtype>
##     <publisher><![CDATA[IEEE]]></publisher>
##     <volume><![CDATA[6]]></volume>
##     <issue><![CDATA[2]]></issue>
##     <py><![CDATA[2014]]></py>
##     <spage><![CDATA[1]]></spage>
##     <epage><![CDATA[8]]></epage>
##     <abstract><![CDATA[We propose a stacked wavelength-division-multiplexed orthogonal-frequency-division-multiplexed passive optical network (WDM-OFDM-PON) system that is capable of transmitting 10-Gb/s per wavelength downstream OFDM signals and directly modulated upstream on-off keying (OOK) signals up to 100 km without any repeater. In this scheme, we use our specially designed electronic controlled liquid tunable optical filter to simultaneously select downstream signals and process upstream signal chirp. By experiment, we demonstrate this bidirectional transmission system and achieve error-free transmission performance over different fiber distances. A power budget of 35.6 dB for 25-km fiber transmission is obtained to support 1 : 512 splitting ratios. Through 100-km fiber transmission, a 34.4-dB optical power budget with a nonrepeater is also achieved. Moreover, the effects of various system parameters such as filter profile and downstream signals' launched power, are also investigated in detail by simulation and experiment for improving system performance.]]></abstract>
##     <issn><![CDATA[1943-0655]]></issn>
##     <htmlFlag><![CDATA[1]]></htmlFlag>
##     <arnumber><![CDATA[6758347]]></arnumber>
##     <doi><![CDATA[10.1109/JPHOT.2014.2308635]]></doi>
##     <publicationId><![CDATA[6758347]]></publicationId>
##     <mdurl><![CDATA[http://ieeexplore.ieee.org/xpl/articleDetails.jsp?tp=&arnumber=6758347&contentType=Journals+%26+Magazines]]></mdurl>
##     <pdf><![CDATA[http://ieeexplore.ieee.org/stamp/stamp.jsp?arnumber=6758347]]></pdf>
##   </document>
##   <document>
##     <rank>1096</rank>
##     <title><![CDATA[Flexible Tracking Speed Data-Aided Coherent Optical System Using Golay Sequences]]></title>
##     <authors><![CDATA[Do, C.C.;  Zhu, C.;  Tran, A.V.]]></authors>
##     <affiliations><![CDATA[Dept. of Electr. & Electron. Eng., Univ. of Melbourne, Parkville, VIC, Australia]]></affiliations>
##     <controlledterms>
##       <term><![CDATA[Golay codes]]></term>
##       <term><![CDATA[light coherence]]></term>
##       <term><![CDATA[optical fibre communication]]></term>
##       <term><![CDATA[training]]></term>
##     </controlledterms>
##     <thesaurusterms>
##       <term><![CDATA[Channel estimation]]></term>
##       <term><![CDATA[Digital signal processing]]></term>
##       <term><![CDATA[Noise]]></term>
##       <term><![CDATA[Optical filters]]></term>
##       <term><![CDATA[Optical noise]]></term>
##       <term><![CDATA[Optical receivers]]></term>
##       <term><![CDATA[Training]]></term>
##     </thesaurusterms>
##     <pubtitle><![CDATA[Photonics Journal, IEEE]]></pubtitle>
##     <punumber><![CDATA[4563994]]></punumber>
##     <pubtype><![CDATA[Journals & Magazines]]></pubtype>
##     <publisher><![CDATA[IEEE]]></publisher>
##     <volume><![CDATA[6]]></volume>
##     <issue><![CDATA[3]]></issue>
##     <py><![CDATA[2014]]></py>
##     <spage><![CDATA[1]]></spage>
##     <epage><![CDATA[9]]></epage>
##     <abstract><![CDATA[We investigate the performance of data-aided coherent optical systems under different training sequence lengths and propose an adaptive length frame synchronization method that can adapt with required tracking speed and training sequence lengths. The technique is verified through simulations and experiments with good and stable performance after 1000 km transmission.]]></abstract>
##     <issn><![CDATA[1943-0655]]></issn>
##     <htmlFlag><![CDATA[1]]></htmlFlag>
##     <arnumber><![CDATA[6802340]]></arnumber>
##     <doi><![CDATA[10.1109/JPHOT.2014.2317679]]></doi>
##     <publicationId><![CDATA[6802340]]></publicationId>
##     <mdurl><![CDATA[http://ieeexplore.ieee.org/xpl/articleDetails.jsp?tp=&arnumber=6802340&contentType=Journals+%26+Magazines]]></mdurl>
##     <pdf><![CDATA[http://ieeexplore.ieee.org/stamp/stamp.jsp?arnumber=6802340]]></pdf>
##   </document>
##   <document>
##     <rank>1097</rank>
##     <title><![CDATA[Peak Force Measurements on a Cylindrical Buoy With Limited Elastic Mooring]]></title>
##     <authors><![CDATA[Svensson, O.;  Leijon, M.]]></authors>
##     <affiliations><![CDATA[Div. for Electr., Uppsala Univ., Uppsala, Sweden]]></affiliations>
##     <controlledterms>
##       <term><![CDATA[accelerometers]]></term>
##       <term><![CDATA[buoyancy]]></term>
##       <term><![CDATA[cellular radio]]></term>
##       <term><![CDATA[design engineering]]></term>
##       <term><![CDATA[elasticity]]></term>
##       <term><![CDATA[fatigue]]></term>
##       <term><![CDATA[force measurement]]></term>
##       <term><![CDATA[force sensors]]></term>
##       <term><![CDATA[marine engineering]]></term>
##       <term><![CDATA[ocean waves]]></term>
##       <term><![CDATA[offshore installations]]></term>
##       <term><![CDATA[wave power generation]]></term>
##     </controlledterms>
##     <pubtitle><![CDATA[Oceanic Engineering, IEEE Journal of]]></pubtitle>
##     <punumber><![CDATA[48]]></punumber>
##     <pubtype><![CDATA[Journals & Magazines]]></pubtype>
##     <publisher><![CDATA[IEEE]]></publisher>
##     <volume><![CDATA[39]]></volume>
##     <issue><![CDATA[2]]></issue>
##     <py><![CDATA[2014]]></py>
##     <spage><![CDATA[398]]></spage>
##     <epage><![CDATA[403]]></epage>
##     <abstract><![CDATA[This paper investigates the line force of a moored floating buoy. The experiment was the first experiment made at the Swedish wave energy research site located close to Lysekil on the Swedish west coast. The Lysekil project is run by the Swedish Centre for Renewable Electric Energy Conversion at Uppsala University (Uppsala, Sweden). The experimental setup consists of a cylindrical buoy, with a diameter of 3 m and a height of 0.8 m. The buoy is moored with a line connected to a set of springs in parallel with a rope. The rope in parallel with the springs represents the limited stroke length of a linear-generator-type wave energy converter. The measurement system consists of a force transducer between the buoy and the rope, a three-axis accelerometer inside the buoy, and a remote data logger operated through the Global System for Mobile Communications (GSM) network. The peak forces related to the significant wave height showed a trend of 33 kN/m. Trends were also calculated in 10-kN bins. The data could be used in fatigue simulations of similar devices. The result was very important to set the mechanical design parameters of the first wave energy converter that was deployed at the Lysekil Research Site in March 2006.]]></abstract>
##     <issn><![CDATA[0364-9059]]></issn>
##     <htmlFlag><![CDATA[1]]></htmlFlag>
##     <arnumber><![CDATA[6519954]]></arnumber>
##     <doi><![CDATA[10.1109/JOE.2013.2247825]]></doi>
##     <publicationId><![CDATA[6519954]]></publicationId>
##     <mdurl><![CDATA[http://ieeexplore.ieee.org/xpl/articleDetails.jsp?tp=&arnumber=6519954&contentType=Journals+%26+Magazines]]></mdurl>
##     <pdf><![CDATA[http://ieeexplore.ieee.org/stamp/stamp.jsp?arnumber=6519954]]></pdf>
##   </document>
##   <document>
##     <rank>1098</rank>
##     <title><![CDATA[Humblet's Decomposition of the Electromagnetic Angular Moment in Metallic Waveguides]]></title>
##     <authors><![CDATA[Berglind, E.;  Bjork, G.]]></authors>
##     <affiliations><![CDATA[Dept. of Appl. Phys., R. Inst. of Technol. (KTH), Stockholm, Sweden]]></affiliations>
##     <controlledterms>
##       <term><![CDATA[rectangular waveguides]]></term>
##     </controlledterms>
##     <thesaurusterms>
##       <term><![CDATA[Electromagnetics]]></term>
##       <term><![CDATA[Extraterrestrial measurements]]></term>
##       <term><![CDATA[Hollow waveguides]]></term>
##       <term><![CDATA[MIMO]]></term>
##       <term><![CDATA[Optical waveguides]]></term>
##       <term><![CDATA[Photonics]]></term>
##     </thesaurusterms>
##     <pubtitle><![CDATA[Microwave Theory and Techniques, IEEE Transactions on]]></pubtitle>
##     <punumber><![CDATA[22]]></punumber>
##     <pubtype><![CDATA[Journals & Magazines]]></pubtype>
##     <publisher><![CDATA[IEEE]]></publisher>
##     <volume><![CDATA[62]]></volume>
##     <issue><![CDATA[4]]></issue>
##     <part><![CDATA[1]]></part>
##     <py><![CDATA[2014]]></py>
##     <spage><![CDATA[779]]></spage>
##     <epage><![CDATA[788]]></epage>
##     <abstract><![CDATA[In a seminal paper, Humblet decomposed the angular momentum of a classical electromagnetic field as a sum of three terms: the orbital angular momentum (OAM), the spin, and the more unfamiliar surface angular momentum. In this paper, we present the result of such decomposition for various metallic waveguides. We investigate two hollow metal waveguides with circular and rectangular cross sections, respectively. The waveguides are excited with two TE eigenmodes driven in phase quadrature. As references, two better known modes are also analyzed: a plane, a circularly polarized wave (a TEM mode), and a TE-Bessel beam, both of infinite transverse extent and with no metallic boundaries. Our analysis shows that modes carrying OAM and spin can also propagate in the metallic waveguides, even when the cross section of the waveguide is distinctly non-circular. However, the mode density of orthogonal modes carrying OAM is at most equal to that of the waveguides' eigenmodes.]]></abstract>
##     <issn><![CDATA[0018-9480]]></issn>
##     <htmlFlag><![CDATA[1]]></htmlFlag>
##     <arnumber><![CDATA[6762817]]></arnumber>
##     <doi><![CDATA[10.1109/TMTT.2014.2308891]]></doi>
##     <publicationId><![CDATA[6762817]]></publicationId>
##     <mdurl><![CDATA[http://ieeexplore.ieee.org/xpl/articleDetails.jsp?tp=&arnumber=6762817&contentType=Journals+%26+Magazines]]></mdurl>
##     <pdf><![CDATA[http://ieeexplore.ieee.org/stamp/stamp.jsp?arnumber=6762817]]></pdf>
##   </document>
##   <document>
##     <rank>1099</rank>
##     <title><![CDATA[A Device for Local or Remote Monitoring of Hand Rehabilitation Sessions for Rheumatic Patients]]></title>
##     <authors><![CDATA[Pani, D.;  Barabino, G.;  Dessi, A.;  Tradori, I.;  Piga, M.;  Mathieu, A.;  Raffo, L.]]></authors>
##     <affiliations><![CDATA[Dept. of Electr. & Electron. Eng., Univ. of Cagliari, Cagliari, Italy]]></affiliations>
##     <controlledterms>
##       <term><![CDATA[adhesion]]></term>
##       <term><![CDATA[biomedical equipment]]></term>
##       <term><![CDATA[cellular radio]]></term>
##       <term><![CDATA[diseases]]></term>
##       <term><![CDATA[injuries]]></term>
##       <term><![CDATA[medical diagnostic computing]]></term>
##       <term><![CDATA[packet radio networks]]></term>
##       <term><![CDATA[patient monitoring]]></term>
##       <term><![CDATA[patient rehabilitation]]></term>
##       <term><![CDATA[patient treatment]]></term>
##       <term><![CDATA[statistical analysis]]></term>
##       <term><![CDATA[transport protocols]]></term>
##     </controlledterms>
##     <thesaurusterms>
##       <term><![CDATA[Biomedical monitoring]]></term>
##       <term><![CDATA[Clinical diagnosis]]></term>
##       <term><![CDATA[Diseases]]></term>
##       <term><![CDATA[Injuries]]></term>
##       <term><![CDATA[Performance evaluation]]></term>
##       <term><![CDATA[Remote monitoring]]></term>
##     </thesaurusterms>
##     <pubtitle><![CDATA[Translational Engineering in Health and Medicine, IEEE Journal of]]></pubtitle>
##     <punumber><![CDATA[6221039]]></punumber>
##     <pubtype><![CDATA[Journals & Magazines]]></pubtype>
##     <publisher><![CDATA[IEEE]]></publisher>
##     <volume><![CDATA[2]]></volume>
##     <py><![CDATA[2014]]></py>
##     <spage><![CDATA[1]]></spage>
##     <epage><![CDATA[11]]></epage>
##     <abstract><![CDATA[Current clinical practice suggests that recovering the hand functionality lost or reduced by injuries, interventions, and chronic diseases requires, beyond pharmacological treatments, a kinesiotherapic intervention. This form of rehabilitation consists of physical exercises adapted to the specific pathology. Its effectiveness is strongly dependent on the patient's adhesion to such a program. In this paper, we present a novel device with remote monitoring capabilities expressly conceived for the needs of rheumatic patients. It comprises several sensorized tools and can be used either in an outpatient clinic for hand functional evaluation, connected to a PC, or afforded to the patient for home kinesiotherapic sessions. In the latter case, the device guides the patient in the rehabilitation session, transmitting the relevant statistics about his performance to a TCP/IP server exploiting a GSM/GPRS connection for deferred analysis. An approved clinical trial has been set up in Italy, involving 10 patients with rheumatoid arthritis and 10 with systemic sclerosis, enrolled for 12 weeks in a home rehabilitation program with the proposed device. Their evaluation has been performed not only with traditional methods, but also with the proposed device. Subjective (hand algofunctional Dreiser's index) and objective (ROM, strength, and dexterity) parameters showed a sustained improvement throughout the follow-up. The obtained results proved that the device is an effective and safe tool for assessing hand disability and monitoring kinesiotherapy exercise, portending the potential exploitability of such a methodology in the clinical practice.]]></abstract>
##     <issn><![CDATA[2168-2372]]></issn>
##     <htmlFlag><![CDATA[1]]></htmlFlag>
##     <arnumber><![CDATA[6708427]]></arnumber>
##     <doi><![CDATA[10.1109/JTEHM.2014.2299274]]></doi>
##     <publicationId><![CDATA[6708427]]></publicationId>
##     <mdurl><![CDATA[http://ieeexplore.ieee.org/xpl/articleDetails.jsp?tp=&arnumber=6708427&contentType=Journals+%26+Magazines]]></mdurl>
##     <pdf><![CDATA[http://ieeexplore.ieee.org/stamp/stamp.jsp?arnumber=6708427]]></pdf>
##   </document>
##   <document>
##     <rank>1100</rank>
##     <title><![CDATA[Edge-Guided Dual-Modality Image Reconstruction]]></title>
##     <authors><![CDATA[Yang Lu;  Jun Zhao;  Ge Wang]]></authors>
##     <affiliations><![CDATA[Mol. Imaging Bus. Unit, Shanghai United Imaging Healthcare Co. Ltd., Shanghai, China]]></affiliations>
##     <controlledterms>
##       <term><![CDATA[biomedical MRI]]></term>
##       <term><![CDATA[computerised tomography]]></term>
##       <term><![CDATA[image fusion]]></term>
##       <term><![CDATA[image reconstruction]]></term>
##       <term><![CDATA[image segmentation]]></term>
##       <term><![CDATA[medical image processing]]></term>
##     </controlledterms>
##     <thesaurusterms>
##       <term><![CDATA[Biomedical image processing]]></term>
##       <term><![CDATA[Computed tomography]]></term>
##       <term><![CDATA[Image processing]]></term>
##       <term><![CDATA[Image reconstruction]]></term>
##       <term><![CDATA[Magnetic resonance imaging]]></term>
##     </thesaurusterms>
##     <pubtitle><![CDATA[Access, IEEE]]></pubtitle>
##     <punumber><![CDATA[6287639]]></punumber>
##     <pubtype><![CDATA[Journals & Magazines]]></pubtype>
##     <publisher><![CDATA[IEEE]]></publisher>
##     <volume><![CDATA[2]]></volume>
##     <py><![CDATA[2014]]></py>
##     <spage><![CDATA[1359]]></spage>
##     <epage><![CDATA[1363]]></epage>
##     <abstract><![CDATA[To utilize the synergy between computed tomography (CT) and magnetic resonance imaging (MRI) data sets from an object at the same time, an edge-guided dual-modality image reconstruction approach is proposed. The key is to establish a knowledge-based connection between these two data sets for the tight fusion of different imaging modalities. Our scheme consists of four inter-related elements: 1) segmentation; 2) initial guess generation; 3) CT image reconstruction; and 4) MRI image reconstruction. Our experiments show that, aided by the image obtained from one imaging modality, even with highly under-sampled data, we can better reconstruct the image of the other modality. This approach can be potentially useful for a simultaneous CT-MRI system.]]></abstract>
##     <issn><![CDATA[2169-3536]]></issn>
##     <htmlFlag><![CDATA[1]]></htmlFlag>
##     <arnumber><![CDATA[6962888]]></arnumber>
##     <doi><![CDATA[10.1109/ACCESS.2014.2371994]]></doi>
##     <publicationId><![CDATA[6962888]]></publicationId>
##     <mdurl><![CDATA[http://ieeexplore.ieee.org/xpl/articleDetails.jsp?tp=&arnumber=6962888&contentType=Journals+%26+Magazines]]></mdurl>
##     <pdf><![CDATA[http://ieeexplore.ieee.org/stamp/stamp.jsp?arnumber=6962888]]></pdf>
##   </document>
##   <document>
##     <rank>1101</rank>
##     <title><![CDATA[Use of Landsat 5 for Change Detection at 1998 Indian and Pakistani Nuclear Test Sites]]></title>
##     <authors><![CDATA[Zelinski, M.E.;  Henderson, J.;  Smith, M.]]></authors>
##     <affiliations><![CDATA[Passive Remote Sensing Group, Lawrence Livermore Nat. Lab., Livermore, CA, USA]]></affiliations>
##     <controlledterms>
##       <term><![CDATA[geophysical techniques]]></term>
##       <term><![CDATA[nuclear explosions]]></term>
##       <term><![CDATA[remote sensing]]></term>
##       <term><![CDATA[statistical analysis]]></term>
##     </controlledterms>
##     <thesaurusterms>
##       <term><![CDATA[Earth]]></term>
##       <term><![CDATA[Inspection]]></term>
##       <term><![CDATA[Remote sensing]]></term>
##       <term><![CDATA[Satellites]]></term>
##       <term><![CDATA[Shape]]></term>
##       <term><![CDATA[Standards]]></term>
##       <term><![CDATA[Vectors]]></term>
##     </thesaurusterms>
##     <pubtitle><![CDATA[Selected Topics in Applied Earth Observations and Remote Sensing, IEEE Journal of]]></pubtitle>
##     <punumber><![CDATA[4609443]]></punumber>
##     <pubtype><![CDATA[Journals & Magazines]]></pubtype>
##     <publisher><![CDATA[IEEE]]></publisher>
##     <volume><![CDATA[7]]></volume>
##     <issue><![CDATA[8]]></issue>
##     <py><![CDATA[2014]]></py>
##     <spage><![CDATA[3453]]></spage>
##     <epage><![CDATA[3460]]></epage>
##     <abstract><![CDATA[An underground nuclear explosion (UNE) can generate a shock wave that lofts surface material, resulting in surface changes that might be detectable. The Comprehensive Nuclear Test-Ban Treaty (CTBT) allows ground and airborne spectral and thermal imaging to help locate such events. Landsat 5 data on the 1998 Indian and Pakistani tests are used here to demonstrate that there are detectable changes in surface features which might be used to localize an underground nuclear test and to develop change detection techniques specific to the use of satellite data to support a CTBT on-site inspection. Landsat 5 has been active for over 20 years providing repeat coverage of the Earth's surface every 16 days. Most locations have Landsat data available for a variety of dates, allowing for statistical analysis of the data to understand temporal trends and data variability on a pixel-by-pixel basis. Given the right conditions, these usual patterns of change (such as seasonal changes or weathering) can be discerned from unusual patterns of change, such as features relating to a UNE. This paper extends known change detection techniques to a temporal series of data and shows that multispectral change detection can be used to help localize a UNE.]]></abstract>
##     <issn><![CDATA[1939-1404]]></issn>
##     <htmlFlag><![CDATA[1]]></htmlFlag>
##     <arnumber><![CDATA[6730730]]></arnumber>
##     <doi><![CDATA[10.1109/JSTARS.2013.2294322]]></doi>
##     <publicationId><![CDATA[6730730]]></publicationId>
##     <mdurl><![CDATA[http://ieeexplore.ieee.org/xpl/articleDetails.jsp?tp=&arnumber=6730730&contentType=Journals+%26+Magazines]]></mdurl>
##     <pdf><![CDATA[http://ieeexplore.ieee.org/stamp/stamp.jsp?arnumber=6730730]]></pdf>
##   </document>
##   <document>
##     <rank>1102</rank>
##     <title><![CDATA[Breakthroughs in Photonics 2013: X-Ray Optics]]></title>
##     <authors><![CDATA[Soufli, R.]]></authors>
##     <affiliations><![CDATA[Lawrence Livermore Nat. Lab., Livermore, CA, USA]]></affiliations>
##     <controlledterms>
##       <term><![CDATA[X-ray optics]]></term>
##       <term><![CDATA[antireflection coatings]]></term>
##       <term><![CDATA[optical fabrication]]></term>
##       <term><![CDATA[optical multilayers]]></term>
##     </controlledterms>
##     <thesaurusterms>
##       <term><![CDATA[Adaptive optics]]></term>
##       <term><![CDATA[Mirrors]]></term>
##       <term><![CDATA[Nonhomogeneous media]]></term>
##       <term><![CDATA[Optical device fabrication]]></term>
##       <term><![CDATA[Ultraviolet sources]]></term>
##     </thesaurusterms>
##     <pubtitle><![CDATA[Photonics Journal, IEEE]]></pubtitle>
##     <punumber><![CDATA[4563994]]></punumber>
##     <pubtype><![CDATA[Journals & Magazines]]></pubtype>
##     <publisher><![CDATA[IEEE]]></publisher>
##     <volume><![CDATA[6]]></volume>
##     <issue><![CDATA[2]]></issue>
##     <py><![CDATA[2014]]></py>
##     <spage><![CDATA[1]]></spage>
##     <epage><![CDATA[6]]></epage>
##     <abstract><![CDATA[This review discusses the latest advances in extreme ultraviolet/X-ray optics development, which are motivated by the availability and demands of new X-ray sources and scientific and industrial applications. Among the breakthroughs highlighted are the following: i) fabrication, metrology, and mounting technologies for large-area optical substrates with improved figure, roughness, and focusing properties; ii) multilayer coatings with especially optimized layer properties, achieving improved reflectance, stability, and out-of-band suppression; and iii) nanodiffractive optics with improved efficiency and resolution.]]></abstract>
##     <issn><![CDATA[1943-0655]]></issn>
##     <htmlFlag><![CDATA[1]]></htmlFlag>
##     <arnumber><![CDATA[6807862]]></arnumber>
##     <doi><![CDATA[10.1109/JPHOT.2014.2309640]]></doi>
##     <publicationId><![CDATA[6807862]]></publicationId>
##     <mdurl><![CDATA[http://ieeexplore.ieee.org/xpl/articleDetails.jsp?tp=&arnumber=6807862&contentType=Journals+%26+Magazines]]></mdurl>
##     <pdf><![CDATA[http://ieeexplore.ieee.org/stamp/stamp.jsp?arnumber=6807862]]></pdf>
##   </document>
##   <document>
##     <rank>1103</rank>
##     <title><![CDATA[Object-Based Image Analysis and Digital Terrain Analysis for Locating Landslides in the Urmia Lake Basin, Iran]]></title>
##     <authors><![CDATA[Blaschke, T.;  Feizizadeh, B.;  Holbling, D.]]></authors>
##     <controlledterms>
##       <term><![CDATA[digital elevation models]]></term>
##       <term><![CDATA[geographic information systems]]></term>
##       <term><![CDATA[geomorphology]]></term>
##       <term><![CDATA[geophysical image processing]]></term>
##       <term><![CDATA[image resolution]]></term>
##       <term><![CDATA[image segmentation]]></term>
##       <term><![CDATA[image sequences]]></term>
##       <term><![CDATA[image texture]]></term>
##       <term><![CDATA[object-oriented methods]]></term>
##       <term><![CDATA[remote sensing]]></term>
##       <term><![CDATA[terrain mapping]]></term>
##       <term><![CDATA[vegetation mapping]]></term>
##     </controlledterms>
##     <thesaurusterms>
##       <term><![CDATA[Image analysis]]></term>
##       <term><![CDATA[Image segmentation]]></term>
##       <term><![CDATA[Object detection]]></term>
##       <term><![CDATA[Satellites]]></term>
##       <term><![CDATA[Terrain factors]]></term>
##       <term><![CDATA[Terrain mapping]]></term>
##       <term><![CDATA[Vegetation mapping]]></term>
##     </thesaurusterms>
##     <pubtitle><![CDATA[Selected Topics in Applied Earth Observations and Remote Sensing, IEEE Journal of]]></pubtitle>
##     <punumber><![CDATA[4609443]]></punumber>
##     <pubtype><![CDATA[Journals & Magazines]]></pubtype>
##     <publisher><![CDATA[IEEE]]></publisher>
##     <volume><![CDATA[7]]></volume>
##     <issue><![CDATA[12]]></issue>
##     <py><![CDATA[2014]]></py>
##     <spage><![CDATA[4806]]></spage>
##     <epage><![CDATA[4817]]></epage>
##     <abstract><![CDATA[The main objective of this research was to establish a semiautomated object-based image analysis (OBIA) methodology for locating landslides. We have detected and delineated landslides within a study area in north-western Iran using normalized difference vegetation index (NDVI), brightness, and textural features derived from satellite imagery (IRS-ID and SPOT-5) in combination with slope and flow direction derivatives from a digital elevation model (DEM) and topographically oriented gray-level cooccurrence matrices (GLCMs). We utilized particular combinations of these information layers to generate objects by applying multiresolution segmentation in a sequence of feature selection and object classification steps. The results were validated by using a landslide inventory database including 109 landslide events. In this study, a combination of these parameters led to a high accuracy of landslide delineation yielding an overall accuracy of 93.07%. Our results confirm the potential of OBIA for accurate delineation of landslides from satellite imagery and, in particular, the ability of OBIA to incorporate heterogeneous parameters such as DEM derivatives and surface texture measures directly in a classification process. The study contributes to the establishment of geographic object-based image analysis (GEOBIA) as a paradigm in remote sensing and geographic information science.]]></abstract>
##     <issn><![CDATA[1939-1404]]></issn>
##     <htmlFlag><![CDATA[1]]></htmlFlag>
##     <arnumber><![CDATA[6905735]]></arnumber>
##     <doi><![CDATA[10.1109/JSTARS.2014.2350036]]></doi>
##     <publicationId><![CDATA[6905735]]></publicationId>
##     <mdurl><![CDATA[http://ieeexplore.ieee.org/xpl/articleDetails.jsp?tp=&arnumber=6905735&contentType=Journals+%26+Magazines]]></mdurl>
##     <pdf><![CDATA[http://ieeexplore.ieee.org/stamp/stamp.jsp?arnumber=6905735]]></pdf>
##   </document>
##   <document>
##     <rank>1104</rank>
##     <title><![CDATA[Dynamic Control of Mobile Multirobot Systems: The Cluster Space Formulation]]></title>
##     <authors><![CDATA[Mas, I.;  Kitts, C.A.]]></authors>
##     <affiliations><![CDATA[Inst. Tecnol. de Buenos Aires, Consejo Nac. de Investig. Cientificas y Tec., Buenos Aires, Argentina]]></affiliations>
##     <controlledterms>
##       <term><![CDATA[mobile robots]]></term>
##       <term><![CDATA[multi-robot systems]]></term>
##       <term><![CDATA[position control]]></term>
##       <term><![CDATA[remotely operated vehicles]]></term>
##     </controlledterms>
##     <thesaurusterms>
##       <term><![CDATA[Aerospace electronics]]></term>
##       <term><![CDATA[Computational modeling]]></term>
##       <term><![CDATA[Control systems]]></term>
##       <term><![CDATA[Mobile communication]]></term>
##       <term><![CDATA[Mobile robots]]></term>
##       <term><![CDATA[Monitoring]]></term>
##       <term><![CDATA[Multi-robot systems]]></term>
##       <term><![CDATA[Robot sensing systems]]></term>
##       <term><![CDATA[Robots]]></term>
##     </thesaurusterms>
##     <pubtitle><![CDATA[Access, IEEE]]></pubtitle>
##     <punumber><![CDATA[6287639]]></punumber>
##     <pubtype><![CDATA[Journals & Magazines]]></pubtype>
##     <publisher><![CDATA[IEEE]]></publisher>
##     <volume><![CDATA[2]]></volume>
##     <py><![CDATA[2014]]></py>
##     <spage><![CDATA[558]]></spage>
##     <epage><![CDATA[570]]></epage>
##     <abstract><![CDATA[The formation control technique called cluster space control promotes simplified specification and monitoring of the motion of mobile multirobot systems of limited size. Previous paper has established the conceptual foundation of this approach and has experimentally verified and validated its use for various systems implementing kinematic controllers. In this paper, we briefly review the definition of the cluster space framework and introduce a new cluster space dynamic model. This model represents the dynamics of the formation as a whole as a function of the dynamics of the member robots. Given this model, generalized cluster space forces can be applied to the formation, and a Jacobian transpose controller can be implemented to transform cluster space compensation forces into robot-level forces to be applied to the robots in the formation. Then, a nonlinear model-based partition controller is proposed. This controller cancels out the formation dynamics and effectively decouples the cluster space variables. Computer simulations and experimental results using three autonomous surface vessels and four land rovers show the effectiveness of the approach. Finally, sensitivity to errors in the estimation of cluster model parameters is analyzed.]]></abstract>
##     <issn><![CDATA[2169-3536]]></issn>
##     <htmlFlag><![CDATA[1]]></htmlFlag>
##     <arnumber><![CDATA[6818372]]></arnumber>
##     <doi><![CDATA[10.1109/ACCESS.2014.2325742]]></doi>
##     <publicationId><![CDATA[6818372]]></publicationId>
##     <mdurl><![CDATA[http://ieeexplore.ieee.org/xpl/articleDetails.jsp?tp=&arnumber=6818372&contentType=Journals+%26+Magazines]]></mdurl>
##     <pdf><![CDATA[http://ieeexplore.ieee.org/stamp/stamp.jsp?arnumber=6818372]]></pdf>
##   </document>
##   <document>
##     <rank>1105</rank>
##     <title><![CDATA[Enhancing Stereo Matching With Classification]]></title>
##     <authors><![CDATA[Baydoun, M.;  Al-Alaoui, M.A.]]></authors>
##     <affiliations><![CDATA[Dept. of Electr. & Comput. Eng., American Univ. of Beirut, Beirut, Lebanon]]></affiliations>
##     <controlledterms>
##       <term><![CDATA[image classification]]></term>
##       <term><![CDATA[image matching]]></term>
##       <term><![CDATA[learning (artificial intelligence)]]></term>
##       <term><![CDATA[neural nets]]></term>
##       <term><![CDATA[stereo image processing]]></term>
##     </controlledterms>
##     <thesaurusterms>
##       <term><![CDATA[Classification]]></term>
##       <term><![CDATA[IEEE standards]]></term>
##       <term><![CDATA[Real-time systems]]></term>
##       <term><![CDATA[Stereo matching]]></term>
##     </thesaurusterms>
##     <pubtitle><![CDATA[Access, IEEE]]></pubtitle>
##     <punumber><![CDATA[6287639]]></punumber>
##     <pubtype><![CDATA[Journals & Magazines]]></pubtype>
##     <publisher><![CDATA[IEEE]]></publisher>
##     <volume><![CDATA[2]]></volume>
##     <py><![CDATA[2014]]></py>
##     <spage><![CDATA[485]]></spage>
##     <epage><![CDATA[499]]></epage>
##     <abstract><![CDATA[This paper presents a novel approach that employs classification to enhance the accuracy of the stereo matching problem. First, the images are treated in order to improve their pixel to pixel correspondence and reduce illumination differences. After that, stereo matching is addressed using different methods with emphasis on local ones like the sum of absolute distances and normalized cross correlation. Other state-of-the-art approaches are also considered. Then, and for every pixel, different features are computed from the input stereo image and the initially found depth map. Afterward, boosting and neural networks, as classification methods, are used to handle occlusion and enhance stereo matching by finding the erroneous disparity values. These values are then corrected through a completion stage. The accuracy of the proposed implementation improves on the problem in an efficient manner. A timing analysis of the method is provided to validate the real time performance. This paper further clarifies some of the possible developments based on various discussions.]]></abstract>
##     <issn><![CDATA[2169-3536]]></issn>
##     <htmlFlag><![CDATA[1]]></htmlFlag>
##     <arnumber><![CDATA[6819773]]></arnumber>
##     <doi><![CDATA[10.1109/ACCESS.2014.2322101]]></doi>
##     <publicationId><![CDATA[6819773]]></publicationId>
##     <mdurl><![CDATA[http://ieeexplore.ieee.org/xpl/articleDetails.jsp?tp=&arnumber=6819773&contentType=Journals+%26+Magazines]]></mdurl>
##     <pdf><![CDATA[http://ieeexplore.ieee.org/stamp/stamp.jsp?arnumber=6819773]]></pdf>
##   </document>
##   <document>
##     <rank>1106</rank>
##     <title><![CDATA[Wavelength-Selective Optical Filters Based on Metal-Patch Cavities With Slot Waveguide Interfaces]]></title>
##     <authors><![CDATA[Youngho Jung;  Jong-Bum You;  Kyungmook Kwon;  Kyoungsik Yu]]></authors>
##     <affiliations><![CDATA[Dept. of Electr. Eng., Korea Adv. Inst. of Sci. & Technol. (KAIST), Daejeon, South Korea]]></affiliations>
##     <controlledterms>
##       <term><![CDATA[coupled mode analysis]]></term>
##       <term><![CDATA[finite difference time-domain analysis]]></term>
##       <term><![CDATA[nanophotonics]]></term>
##       <term><![CDATA[optical couplers]]></term>
##       <term><![CDATA[optical filters]]></term>
##       <term><![CDATA[optical waveguides]]></term>
##     </controlledterms>
##     <thesaurusterms>
##       <term><![CDATA[Cavity resonators]]></term>
##       <term><![CDATA[Couplings]]></term>
##       <term><![CDATA[Metals]]></term>
##       <term><![CDATA[Optical coupling]]></term>
##       <term><![CDATA[Optical filters]]></term>
##       <term><![CDATA[Optical waveguides]]></term>
##     </thesaurusterms>
##     <pubtitle><![CDATA[Photonics Journal, IEEE]]></pubtitle>
##     <punumber><![CDATA[4563994]]></punumber>
##     <pubtype><![CDATA[Journals & Magazines]]></pubtype>
##     <publisher><![CDATA[IEEE]]></publisher>
##     <volume><![CDATA[6]]></volume>
##     <issue><![CDATA[4]]></issue>
##     <py><![CDATA[2014]]></py>
##     <spage><![CDATA[1]]></spage>
##     <epage><![CDATA[10]]></epage>
##     <abstract><![CDATA[We theoretically propose and investigate the resonant wavelength-selective optical filters based on the subwavelength metallo-dielectric nanopatch cavities and the metal slot waveguides. Using the temporal coupled-mode theory and three-dimensional finite-difference time-domain simulations, we analyze the optical coupling efficiencies and filtering characteristics, which are optimized by adjusting the waveguide-cavity distance to modify the decay rates from the cavity. The proposed optical filters can be extended to multiport channel-dropping filters while maintaining subwavelength-scale dimensions including the coupling regions.]]></abstract>
##     <issn><![CDATA[1943-0655]]></issn>
##     <htmlFlag><![CDATA[1]]></htmlFlag>
##     <arnumber><![CDATA[6837448]]></arnumber>
##     <doi><![CDATA[10.1109/JPHOT.2014.2331253]]></doi>
##     <publicationId><![CDATA[6837448]]></publicationId>
##     <mdurl><![CDATA[http://ieeexplore.ieee.org/xpl/articleDetails.jsp?tp=&arnumber=6837448&contentType=Journals+%26+Magazines]]></mdurl>
##     <pdf><![CDATA[http://ieeexplore.ieee.org/stamp/stamp.jsp?arnumber=6837448]]></pdf>
##   </document>
##   <document>
##     <rank>1107</rank>
##     <title><![CDATA[Big Data and the SP Theory of Intelligence]]></title>
##     <authors><![CDATA[Wolff, J.G.]]></authors>
##     <affiliations><![CDATA[CognitionResearch.org, Menai Bridge, UK]]></affiliations>
##     <controlledterms>
##       <term><![CDATA[Big Data]]></term>
##       <term><![CDATA[data analysis]]></term>
##       <term><![CDATA[data compression]]></term>
##       <term><![CDATA[data mining]]></term>
##       <term><![CDATA[data structures]]></term>
##       <term><![CDATA[natural language processing]]></term>
##       <term><![CDATA[unsupervised learning]]></term>
##     </controlledterms>
##     <thesaurusterms>
##       <term><![CDATA[Artificial intelligence]]></term>
##       <term><![CDATA[Cognitive science]]></term>
##       <term><![CDATA[Computational efficiency]]></term>
##       <term><![CDATA[Data compression]]></term>
##       <term><![CDATA[Data storage systems]]></term>
##       <term><![CDATA[Pattern recognition]]></term>
##       <term><![CDATA[Unsupervised learning]]></term>
##     </thesaurusterms>
##     <pubtitle><![CDATA[Access, IEEE]]></pubtitle>
##     <punumber><![CDATA[6287639]]></punumber>
##     <pubtype><![CDATA[Journals & Magazines]]></pubtype>
##     <publisher><![CDATA[IEEE]]></publisher>
##     <volume><![CDATA[2]]></volume>
##     <py><![CDATA[2014]]></py>
##     <spage><![CDATA[301]]></spage>
##     <epage><![CDATA[315]]></epage>
##     <abstract><![CDATA[This paper is about how the SP theory of intelligence and its realization in the SP machine may, with advantage, be applied to the management and analysis of big data. The SP system-introduced in this paper and fully described elsewhere-may help to overcome the problem of variety in big data; it has potential as a universal framework for the representation and processing of diverse kinds of knowledge, helping to reduce the diversity of formalisms and formats for knowledge, and the different ways in which they are processed. It has strengths in the unsupervised learning or discovery of structure in data, in pattern recognition, in the parsing and production of natural language, in several kinds of reasoning, and more. It lends itself to the analysis of streaming data, helping to overcome the problem of velocity in big data. Central in the workings of the system is lossless compression of information: making big data smaller and reducing problems of storage and management. There is potential for substantial economies in the transmission of data, for big cuts in the use of energy in computing, for faster processing, and for smaller and lighter computers. The system provides a handle on the problem of veracity in big data, with potential to assist in the management of errors and uncertainties in data. It lends itself to the visualization of knowledge structures and inferential processes. A high-parallel, open-source version of the SP machine would provide a means for researchers everywhere to explore what can be done with the system and to create new versions of it.]]></abstract>
##     <issn><![CDATA[2169-3536]]></issn>
##     <htmlFlag><![CDATA[1]]></htmlFlag>
##     <arnumber><![CDATA[6782396]]></arnumber>
##     <doi><![CDATA[10.1109/ACCESS.2014.2315297]]></doi>
##     <publicationId><![CDATA[6782396]]></publicationId>
##     <mdurl><![CDATA[http://ieeexplore.ieee.org/xpl/articleDetails.jsp?tp=&arnumber=6782396&contentType=Journals+%26+Magazines]]></mdurl>
##     <pdf><![CDATA[http://ieeexplore.ieee.org/stamp/stamp.jsp?arnumber=6782396]]></pdf>
##   </document>
##   <document>
##     <rank>1108</rank>
##     <title><![CDATA[Energy-Efficient Dynamic Drive Control for Wind Power Conversion With PMSG: Modeling and Application of Transfer Function Analysis]]></title>
##     <authors><![CDATA[Kuschke, M.;  Strunz, K.]]></authors>
##     <affiliations><![CDATA[Tech. Univ. Berlin, Berlin, Germany]]></affiliations>
##     <controlledterms>
##       <term><![CDATA[air gaps]]></term>
##       <term><![CDATA[angular velocity control]]></term>
##       <term><![CDATA[electric current control]]></term>
##       <term><![CDATA[machine control]]></term>
##       <term><![CDATA[nonlinear control systems]]></term>
##       <term><![CDATA[permanent magnet generators]]></term>
##       <term><![CDATA[poles and zeros]]></term>
##       <term><![CDATA[power control]]></term>
##       <term><![CDATA[power conversion]]></term>
##       <term><![CDATA[power convertors]]></term>
##       <term><![CDATA[power generation control]]></term>
##       <term><![CDATA[power system stability]]></term>
##       <term><![CDATA[stators]]></term>
##       <term><![CDATA[stochastic processes]]></term>
##       <term><![CDATA[synchronous motor drives]]></term>
##       <term><![CDATA[time-domain analysis]]></term>
##       <term><![CDATA[torque control]]></term>
##       <term><![CDATA[transfer functions]]></term>
##       <term><![CDATA[wind power]]></term>
##       <term><![CDATA[wind power plants]]></term>
##     </controlledterms>
##     <thesaurusterms>
##       <term><![CDATA[Power electronics]]></term>
##       <term><![CDATA[Rotors]]></term>
##       <term><![CDATA[Stators]]></term>
##       <term><![CDATA[Synthesizers]]></term>
##       <term><![CDATA[Torque]]></term>
##       <term><![CDATA[Transfer functions]]></term>
##       <term><![CDATA[Vectors]]></term>
##     </thesaurusterms>
##     <pubtitle><![CDATA[Emerging and Selected Topics in Power Electronics, IEEE Journal of]]></pubtitle>
##     <punumber><![CDATA[6245517]]></punumber>
##     <pubtype><![CDATA[Journals & Magazines]]></pubtype>
##     <publisher><![CDATA[IEEE]]></publisher>
##     <volume><![CDATA[2]]></volume>
##     <issue><![CDATA[1]]></issue>
##     <py><![CDATA[2014]]></py>
##     <spage><![CDATA[35]]></spage>
##     <epage><![CDATA[46]]></epage>
##     <abstract><![CDATA[A method for transfer function based modeling offering an in-depth insight into the systemic behavior of wind energy conversion systems (WECS) is developed. The originally nonlinear behavior of the drive system covering turbine, permanent magnet synchronous generator, and power electronic converter is rearranged and linearized resulting in a compact transfer function description. The locations of transfer function poles and zeros and related stability are readily identified as a function of WECS parameters and the operating point. Drive control design rules making use of the transfer functions for setting the compensation parameters depending on the wind speed are established. The behavioral differences between speed and power control loops can readily be appreciated. The synthesis of a power control loop to closely follow maximum available wind power is performed based on the design rules. In this context, the voltage sourced converter is operated in current mode control to contribute to fast adjustment of air-gap torque while maintaining currents within limits. The direct and quadrature current references are calculated to attain the desired torque at minimal stator current magnitude and so enhance energy efficiency. The dynamic performance of the design is evidenced by time-domain simulation and stochastic analysis.]]></abstract>
##     <issn><![CDATA[2168-6777]]></issn>
##     <htmlFlag><![CDATA[1]]></htmlFlag>
##     <arnumber><![CDATA[6678202]]></arnumber>
##     <doi><![CDATA[10.1109/JESTPE.2013.2293632]]></doi>
##     <publicationId><![CDATA[6678202]]></publicationId>
##     <mdurl><![CDATA[http://ieeexplore.ieee.org/xpl/articleDetails.jsp?tp=&arnumber=6678202&contentType=Journals+%26+Magazines]]></mdurl>
##     <pdf><![CDATA[http://ieeexplore.ieee.org/stamp/stamp.jsp?arnumber=6678202]]></pdf>
##   </document>
##   <document>
##     <rank>1109</rank>
##     <title><![CDATA[Low Power Very High Frequency Switch-Mode Power Supply With 50 V Input and 5 V Output]]></title>
##     <authors><![CDATA[Madsen, M.;  Knott, A.;  Andersen, M.A.E.]]></authors>
##     <affiliations><![CDATA[Dept. of Electr. Eng., Tech. Univ. of Denmark, Lyngby, Denmark]]></affiliations>
##     <controlledterms>
##       <term><![CDATA[DC-DC power convertors]]></term>
##       <term><![CDATA[rectifiers]]></term>
##       <term><![CDATA[resonant power convertors]]></term>
##       <term><![CDATA[switched mode power supplies]]></term>
##     </controlledterms>
##     <thesaurusterms>
##       <term><![CDATA[Capacitance]]></term>
##       <term><![CDATA[Power generation]]></term>
##       <term><![CDATA[Resonant frequency]]></term>
##       <term><![CDATA[Resonant inverters]]></term>
##       <term><![CDATA[Switches]]></term>
##       <term><![CDATA[Switching frequency]]></term>
##     </thesaurusterms>
##     <pubtitle><![CDATA[Power Electronics, IEEE Transactions on]]></pubtitle>
##     <punumber><![CDATA[63]]></punumber>
##     <pubtype><![CDATA[Journals & Magazines]]></pubtype>
##     <publisher><![CDATA[IEEE]]></publisher>
##     <volume><![CDATA[29]]></volume>
##     <issue><![CDATA[12]]></issue>
##     <py><![CDATA[2014]]></py>
##     <spage><![CDATA[6569]]></spage>
##     <epage><![CDATA[6580]]></epage>
##     <abstract><![CDATA[This paper presents the design of a resonant converter with a switching frequency in the very high frequency range (30-300 MHz), a large step down ratio (ten times), and low output power (1 W). Several different inverters and rectifiers are analyzed and compared. The class E inverter and rectifier are selected based on complexity and efficiency estimates. Three different power stages are implemented; one with a large input inductor, one with a switch with small capacitances, and one with a switch with low on-resistance. The power stages are designed with the same specifications and efficiencies from 60.7-82.9% are achieved.]]></abstract>
##     <issn><![CDATA[0885-8993]]></issn>
##     <htmlFlag><![CDATA[1]]></htmlFlag>
##     <arnumber><![CDATA[6739180]]></arnumber>
##     <doi><![CDATA[10.1109/TPEL.2014.2305738]]></doi>
##     <publicationId><![CDATA[6739180]]></publicationId>
##     <mdurl><![CDATA[http://ieeexplore.ieee.org/xpl/articleDetails.jsp?tp=&arnumber=6739180&contentType=Journals+%26+Magazines]]></mdurl>
##     <pdf><![CDATA[http://ieeexplore.ieee.org/stamp/stamp.jsp?arnumber=6739180]]></pdf>
##   </document>
##   <document>
##     <rank>1110</rank>
##     <title><![CDATA[Implementation of a Thread-Parallel, GPU-Friendly Function Evaluation Library]]></title>
##     <authors><![CDATA[Andreassen, R.E.;  de Silva, W.M.;  Meadows, B.T.;  Sokoloff, M.D.;  Tomko, K.A.]]></authors>
##     <affiliations><![CDATA[Phys. Dept., Univ. of Cincinnati, Cincinnati, OH, USA]]></affiliations>
##     <controlledterms>
##       <term><![CDATA[application program interfaces]]></term>
##       <term><![CDATA[graphics processing units]]></term>
##       <term><![CDATA[maximum likelihood estimation]]></term>
##       <term><![CDATA[parallel architectures]]></term>
##       <term><![CDATA[parallel programming]]></term>
##       <term><![CDATA[software libraries]]></term>
##     </controlledterms>
##     <thesaurusterms>
##       <term><![CDATA[Graphics processing units]]></term>
##       <term><![CDATA[Libraries]]></term>
##       <term><![CDATA[Maximum likelihood estimation]]></term>
##       <term><![CDATA[Parallel processing]]></term>
##       <term><![CDATA[Parallel programming]]></term>
##       <term><![CDATA[Parameter estimation]]></term>
##     </thesaurusterms>
##     <pubtitle><![CDATA[Access, IEEE]]></pubtitle>
##     <punumber><![CDATA[6287639]]></punumber>
##     <pubtype><![CDATA[Journals & Magazines]]></pubtype>
##     <publisher><![CDATA[IEEE]]></publisher>
##     <volume><![CDATA[2]]></volume>
##     <py><![CDATA[2014]]></py>
##     <spage><![CDATA[160]]></spage>
##     <epage><![CDATA[176]]></epage>
##     <abstract><![CDATA[GooFit is a thread-parallel, GPU-friendly function evaluation library, nominally designed for use with the maximum likelihood fitting program MINUIT. In this use case, it provides highly parallel calculations of normalization intergrals and log (likelihood) sums. A key feature of the design is its use of the Thrust library to manage all parallel kernel launches. This allows GooFit to execute on any architecture for which Thrust has a backend, currently, including CUDA for nVidia GPUs and OpenMP for single- and multicore CPUs. Running on an nVidia C2050, GooFit executes 300 times more quickly for a complex high energy physics problem than does the prior (algorithmically equivalent) code running on a single CPU core. The design and implementation choices, discussed in detail, can help to guide developers of other highly parallel, compute-intensive libraries.]]></abstract>
##     <issn><![CDATA[2169-3536]]></issn>
##     <htmlFlag><![CDATA[1]]></htmlFlag>
##     <arnumber><![CDATA[6746000]]></arnumber>
##     <doi><![CDATA[10.1109/ACCESS.2014.2306895]]></doi>
##     <publicationId><![CDATA[6746000]]></publicationId>
##     <mdurl><![CDATA[http://ieeexplore.ieee.org/xpl/articleDetails.jsp?tp=&arnumber=6746000&contentType=Journals+%26+Magazines]]></mdurl>
##     <pdf><![CDATA[http://ieeexplore.ieee.org/stamp/stamp.jsp?arnumber=6746000]]></pdf>
##   </document>
##   <document>
##     <rank>1111</rank>
##     <title><![CDATA[Firelight LED Source: Toward a Balanced Approach to the Performance of Solid-State Lighting for Outdoor Environments]]></title>
##     <authors><![CDATA[Zukauskas, A.;  Vaicekauskas, R.;  Tuzikas, A.;  Petrulis, A.;  Stanikunas, R.;  Svegzda, A.;  Eidikas, P.;  Vitta, P.]]></authors>
##     <affiliations><![CDATA[Inst. of Appl. Res., Vilnius Univ., Vilnius, Lithuania]]></affiliations>
##     <controlledterms>
##       <term><![CDATA[colour]]></term>
##       <term><![CDATA[colour vision]]></term>
##       <term><![CDATA[light emitting diodes]]></term>
##       <term><![CDATA[lighting]]></term>
##     </controlledterms>
##     <thesaurusterms>
##       <term><![CDATA[Color]]></term>
##       <term><![CDATA[Image color analysis]]></term>
##       <term><![CDATA[Indexes]]></term>
##       <term><![CDATA[Light emitting diodes]]></term>
##       <term><![CDATA[Lighting]]></term>
##       <term><![CDATA[Phosphors]]></term>
##     </thesaurusterms>
##     <pubtitle><![CDATA[Photonics Journal, IEEE]]></pubtitle>
##     <punumber><![CDATA[4563994]]></punumber>
##     <pubtype><![CDATA[Journals & Magazines]]></pubtype>
##     <publisher><![CDATA[IEEE]]></publisher>
##     <volume><![CDATA[6]]></volume>
##     <issue><![CDATA[3]]></issue>
##     <py><![CDATA[2014]]></py>
##     <spage><![CDATA[1]]></spage>
##     <epage><![CDATA[16]]></epage>
##     <abstract><![CDATA[We report on a blue-amber (&#x201C;firelight&#x201D;) cluster of light-emitting diodes (LEDs) with extra-low correlated color temperature (~1860 K) optimized for outdoor lighting under mesopic conditions. When compared with common white LEDs, the firelight LED cluster shows considerably reduced indexes of melatonin suppression and skyglow, increased retinal illuminance for elderly people, but a reduced performance of perceiving colors, which, however, can be tolerated at mesopic luminance. In comparison with an almost metameric high-pressure sodium lamp, the cluster exhibits a potentially higher luminous efficacy, similar reaction time and detection threshold of luminance contrasts for achromatic targets, and noticeably improved color discrimination characteristics.]]></abstract>
##     <issn><![CDATA[1943-0655]]></issn>
##     <htmlFlag><![CDATA[1]]></htmlFlag>
##     <arnumber><![CDATA[6804694]]></arnumber>
##     <doi><![CDATA[10.1109/JPHOT.2014.2319102]]></doi>
##     <publicationId><![CDATA[6804694]]></publicationId>
##     <mdurl><![CDATA[http://ieeexplore.ieee.org/xpl/articleDetails.jsp?tp=&arnumber=6804694&contentType=Journals+%26+Magazines]]></mdurl>
##     <pdf><![CDATA[http://ieeexplore.ieee.org/stamp/stamp.jsp?arnumber=6804694]]></pdf>
##   </document>
##   <document>
##     <rank>1112</rank>
##     <title><![CDATA[On the Equivalence Between Canonical Forms of Recursive Systematic Convolutional Transducers Based on Single Shift Registers]]></title>
##     <authors><![CDATA[Balta, H.;  Douillard, C.;  Isar, A.]]></authors>
##     <affiliations><![CDATA[Univ. Politeh. Timisoara, Timis&#x0327;oara, Romania]]></affiliations>
##     <controlledterms>
##       <term><![CDATA[controllability]]></term>
##       <term><![CDATA[convolutional codes]]></term>
##       <term><![CDATA[feedback]]></term>
##       <term><![CDATA[linear systems]]></term>
##       <term><![CDATA[matrix algebra]]></term>
##       <term><![CDATA[observers]]></term>
##       <term><![CDATA[polynomials]]></term>
##       <term><![CDATA[shift registers]]></term>
##       <term><![CDATA[transducers]]></term>
##       <term><![CDATA[turbo codes]]></term>
##     </controlledterms>
##     <thesaurusterms>
##       <term><![CDATA[Complexity theory]]></term>
##       <term><![CDATA[Convolutional codes]]></term>
##       <term><![CDATA[Decoding]]></term>
##       <term><![CDATA[Encoding]]></term>
##       <term><![CDATA[Polynomials]]></term>
##       <term><![CDATA[Systematics]]></term>
##       <term><![CDATA[Transducers]]></term>
##     </thesaurusterms>
##     <pubtitle><![CDATA[Access, IEEE]]></pubtitle>
##     <punumber><![CDATA[6287639]]></punumber>
##     <pubtype><![CDATA[Journals & Magazines]]></pubtype>
##     <publisher><![CDATA[IEEE]]></publisher>
##     <volume><![CDATA[2]]></volume>
##     <py><![CDATA[2014]]></py>
##     <spage><![CDATA[381]]></spage>
##     <epage><![CDATA[394]]></epage>
##     <abstract><![CDATA[Standardized turbo codes (TCs) use recursive systematic convolutional transducers of rate b/(b + d), having a single feedback polynomial (b+dRSCT). In this paper, we investigate the realizability of the b+dRSCT set through two single shift register canonical forms (SSRCFs), called, in the theory of linear systems, constructibility, and controllability. The two investigated SSRCF are the adaptations, for the implementation of b+dRSCT, of the better-known canonical forms controller (constructibility) and observer (controllability). Constructibility is the implementation form actually used for convolutional transducers in TCs. This paper shows that any b+1RSCT can be implemented in a unique SSRCF observer. As a result, we build a function, &#x03BE;:H &#x2192; G, which has as definition domain the set of encoders in SSRCF constructibility, denoted by H, and as codomain a subset of encoders in SSRCF observer, denoted by G. By proving the noninjectivity and nonsurjectivity properties of the function &#x03BE;, we prove that H is redundant and incomplete in comparison with G, i.e., the SSRCF observer is more efficient than the SSRCF constructibility for the implementation of b+1RSCT. We show that the redundancy of the set H is dependent on the memory m and on the number of inputs b of the considered b+1RSCT. In addition, the difference between G and &#x03BE;(H) contains encoders with very good performance, when used in a TC structure. This difference is consistent for m &#x2248; b &gt; 1. The results on the realizability of the b+1RSCT allowed us some considerations on b+dRSCT, with b, d &gt; 1, as well, for which we proposed the SSRCF controllability. These results could be useful in the design of TC based on exhaustive search. So, the proposed implementation form permits the design of new TCs, which cannot be conceived based on the actual form. It is possible, even probable, among these new TCs to find better performance than in the current communication standards- such as LTE, DVB, or deep-space communications.]]></abstract>
##     <issn><![CDATA[2169-3536]]></issn>
##     <htmlFlag><![CDATA[1]]></htmlFlag>
##     <arnumber><![CDATA[6802341]]></arnumber>
##     <doi><![CDATA[10.1109/ACCESS.2014.2316413]]></doi>
##     <publicationId><![CDATA[6802341]]></publicationId>
##     <mdurl><![CDATA[http://ieeexplore.ieee.org/xpl/articleDetails.jsp?tp=&arnumber=6802341&contentType=Journals+%26+Magazines]]></mdurl>
##     <pdf><![CDATA[http://ieeexplore.ieee.org/stamp/stamp.jsp?arnumber=6802341]]></pdf>
##   </document>
##   <document>
##     <rank>1113</rank>
##     <title><![CDATA[Tunable Oscillations in Optically Injected Semiconductor Lasers With Reduced Sensitivity to Perturbations]]></title>
##     <authors><![CDATA[Simpson, T.B.;  Jia-Ming Liu;  Almulla, M.;  Usechak, N.G.;  Kovanis, V.]]></authors>
##     <affiliations><![CDATA[L-3 Appl. Technol., Inc., San Diego, CA, USA]]></affiliations>
##     <controlledterms>
##       <term><![CDATA[laser tuning]]></term>
##       <term><![CDATA[light interference]]></term>
##       <term><![CDATA[multiwave mixing]]></term>
##       <term><![CDATA[oscillations]]></term>
##       <term><![CDATA[semiconductor lasers]]></term>
##     </controlledterms>
##     <thesaurusterms>
##       <term><![CDATA[Optical attenuators]]></term>
##       <term><![CDATA[Optical mixing]]></term>
##       <term><![CDATA[Optical polarization]]></term>
##       <term><![CDATA[Oscillators]]></term>
##       <term><![CDATA[Power lasers]]></term>
##       <term><![CDATA[Resonant frequency]]></term>
##       <term><![CDATA[Semiconductor lasers]]></term>
##     </thesaurusterms>
##     <pubtitle><![CDATA[Lightwave Technology, Journal of]]></pubtitle>
##     <punumber><![CDATA[50]]></punumber>
##     <pubtype><![CDATA[Journals & Magazines]]></pubtype>
##     <publisher><![CDATA[IEEE]]></publisher>
##     <volume><![CDATA[32]]></volume>
##     <issue><![CDATA[20]]></issue>
##     <py><![CDATA[2014]]></py>
##     <spage><![CDATA[3749]]></spage>
##     <epage><![CDATA[3758]]></epage>
##     <abstract><![CDATA[Narrow linewidth optical injection into a semiconductor laser can induce periodic oscillations in the injected laser's output power with a frequency that is widely tunable by simply varying the steady-state bias current and operating temperature. Recently, it has been demonstrated that this oscillation frequency can be made nearly insensitive to small-signal fluctuations of these two parameters at certain operating points [1]. Here, we demonstrate that this insensitivity arises from multiwave mixing and interference that minimizes the response of both the gain medium and the circulating optical power at the oscillation frequency. Both experimental measurements and model calculations of optical spectra show that at the operating points of reduced oscillation frequency sensitivity, all strong components of the optical spectrum still exhibit a response to the perturbations. However, in the power spectra and the (calculated) carrier-density spectra, the response is strongly attenuated. Novel operating points that limit the sensitivity of the laser power oscillation frequency to perturbations offer the promise for improved operation of tunable photonic oscillators for radio- and microwave-frequency applications.]]></abstract>
##     <issn><![CDATA[0733-8724]]></issn>
##     <htmlFlag><![CDATA[1]]></htmlFlag>
##     <arnumber><![CDATA[6841598]]></arnumber>
##     <doi><![CDATA[10.1109/JLT.2014.2332415]]></doi>
##     <publicationId><![CDATA[6841598]]></publicationId>
##     <mdurl><![CDATA[http://ieeexplore.ieee.org/xpl/articleDetails.jsp?tp=&arnumber=6841598&contentType=Journals+%26+Magazines]]></mdurl>
##     <pdf><![CDATA[http://ieeexplore.ieee.org/stamp/stamp.jsp?arnumber=6841598]]></pdf>
##   </document>
##   <document>
##     <rank>1114</rank>
##     <title><![CDATA[NBTI in Nanoscale MOSFETs&#x2014;The Ultimate Modeling Benchmark]]></title>
##     <authors><![CDATA[Grasser, T.;  Rott, K.;  Reisinger, H.;  Waltl, M.;  Schanovsky, F.;  Kaczer, B.]]></authors>
##     <affiliations><![CDATA[Inst. for Microelectron., Tech. Univ. Wien, Vienna, Austria]]></affiliations>
##     <controlledterms>
##       <term><![CDATA[MOSFET]]></term>
##       <term><![CDATA[diffusion]]></term>
##       <term><![CDATA[nanoelectronics]]></term>
##       <term><![CDATA[negative bias temperature instability]]></term>
##       <term><![CDATA[semiconductor device models]]></term>
##       <term><![CDATA[semiconductor device reliability]]></term>
##     </controlledterms>
##     <thesaurusterms>
##       <term><![CDATA[Annealing]]></term>
##       <term><![CDATA[Data models]]></term>
##       <term><![CDATA[Degradation]]></term>
##       <term><![CDATA[Dispersion]]></term>
##       <term><![CDATA[Nanoscale devices]]></term>
##       <term><![CDATA[Predictive models]]></term>
##       <term><![CDATA[Stress]]></term>
##     </thesaurusterms>
##     <pubtitle><![CDATA[Electron Devices, IEEE Transactions on]]></pubtitle>
##     <punumber><![CDATA[16]]></punumber>
##     <pubtype><![CDATA[Journals & Magazines]]></pubtype>
##     <publisher><![CDATA[IEEE]]></publisher>
##     <volume><![CDATA[61]]></volume>
##     <issue><![CDATA[11]]></issue>
##     <py><![CDATA[2014]]></py>
##     <spage><![CDATA[3586]]></spage>
##     <epage><![CDATA[3593]]></epage>
##     <abstract><![CDATA[After nearly half a century of research into the bias temperature instability, two classes of models have emerged as the strongest contenders. One class of models, the reaction-diffusion models, is built around the idea that hydrogen is released from the interface and that it is the diffusion of some form of hydrogen that controls both degradation and recovery. Although various variants of the reaction-diffusion idea have been published over the years, the most commonly used recent models are based on nondispersive reaction rates and nondispersive diffusion. The other class of models is based on the idea that degradation is controlled by first-order reactions with widely distributed (dispersive) reaction rates. We demonstrate that these two classes give fundamentally different predictions for the stochastic degradation and recovery of nanoscale devices, therefore providing the ultimate modeling benchmark. Using detailed experimental time-dependent defect spectroscopy data obtained on such nanoscale devices, we investigate the compatibility of these models with experiment. Our results show that the diffusion of hydrogen (or any other species) is unlikely to be the limiting aspect that determines degradation. On the other hand, the data are fully consistent with reaction-limited models. We finally argue that only the correct understanding of the physical mechanisms leading to the significant device-to-device variation observed in the degradation in nanoscale devices will enable accurate reliability projections and device optimization.]]></abstract>
##     <issn><![CDATA[0018-9383]]></issn>
##     <htmlFlag><![CDATA[1]]></htmlFlag>
##     <arnumber><![CDATA[6902764]]></arnumber>
##     <doi><![CDATA[10.1109/TED.2014.2353578]]></doi>
##     <publicationId><![CDATA[6902764]]></publicationId>
##     <mdurl><![CDATA[http://ieeexplore.ieee.org/xpl/articleDetails.jsp?tp=&arnumber=6902764&contentType=Journals+%26+Magazines]]></mdurl>
##     <pdf><![CDATA[http://ieeexplore.ieee.org/stamp/stamp.jsp?arnumber=6902764]]></pdf>
##   </document>
##   <document>
##     <rank>1115</rank>
##     <title><![CDATA[Reducing Environmental Impact and Improving Safety and Performance of Power Transformers With Natural Ester Dielectric Insulating Fluids]]></title>
##     <authors><![CDATA[Asano, R.;  Page, S.A.]]></authors>
##     <affiliations><![CDATA[ABB, Guarulhos, Brazil]]></affiliations>
##     <controlledterms>
##       <term><![CDATA[flameproofing]]></term>
##       <term><![CDATA[power transformer insulation]]></term>
##       <term><![CDATA[transformer oil]]></term>
##     </controlledterms>
##     <thesaurusterms>
##       <term><![CDATA[Circuit faults]]></term>
##       <term><![CDATA[Dielectrics]]></term>
##       <term><![CDATA[Minerals]]></term>
##       <term><![CDATA[Oil insulation]]></term>
##       <term><![CDATA[Power transformer insulation]]></term>
##       <term><![CDATA[Vegetable oils]]></term>
##     </thesaurusterms>
##     <pubtitle><![CDATA[Industry Applications, IEEE Transactions on]]></pubtitle>
##     <punumber><![CDATA[28]]></punumber>
##     <pubtype><![CDATA[Journals & Magazines]]></pubtype>
##     <publisher><![CDATA[IEEE]]></publisher>
##     <volume><![CDATA[50]]></volume>
##     <issue><![CDATA[1]]></issue>
##     <py><![CDATA[2014]]></py>
##     <spage><![CDATA[134]]></spage>
##     <epage><![CDATA[141]]></epage>
##     <abstract><![CDATA[Natural ester fluids may be used in new transformers as well as to retrofill existing units in order to improve their performance and reliability. Designing or retrofilling power transformers with natural ester fluids requires, however, to account for a number of significant differences in properties, characteristics, and material parameters between natural ester fluids and mineral oil in order to obtain the desired performance (thermally and dielectrically). This paper gives a comprehensive review of natural-ester-fluid main properties and associated values when it comes to environmental impact, fire resistance, and overall performance of transformers filled with such fluids. The key fluid characteristics impacting the design of power transformers are also highlighted.]]></abstract>
##     <issn><![CDATA[0093-9994]]></issn>
##     <htmlFlag><![CDATA[1]]></htmlFlag>
##     <arnumber><![CDATA[6544258]]></arnumber>
##     <doi><![CDATA[10.1109/TIA.2013.2269532]]></doi>
##     <publicationId><![CDATA[6544258]]></publicationId>
##     <mdurl><![CDATA[http://ieeexplore.ieee.org/xpl/articleDetails.jsp?tp=&arnumber=6544258&contentType=Journals+%26+Magazines]]></mdurl>
##     <pdf><![CDATA[http://ieeexplore.ieee.org/stamp/stamp.jsp?arnumber=6544258]]></pdf>
##   </document>
##   <document>
##     <rank>1116</rank>
##     <title><![CDATA[Zigzag-Shaped Coil Array Structure for Wireless Chip-to-Chip Communication Applications]]></title>
##     <authors><![CDATA[Changhyun Lee;  Jonghoon Park;  Changkun Park]]></authors>
##     <affiliations><![CDATA[Sch. of Electron. Eng., Soongsil Univ., Seoul, South Korea]]></affiliations>
##     <controlledterms>
##       <term><![CDATA[coils]]></term>
##       <term><![CDATA[crosstalk]]></term>
##       <term><![CDATA[integrated circuit interconnections]]></term>
##       <term><![CDATA[integrated circuit layout]]></term>
##       <term><![CDATA[microprocessor chips]]></term>
##       <term><![CDATA[transceivers]]></term>
##     </controlledterms>
##     <thesaurusterms>
##       <term><![CDATA[Arrays]]></term>
##       <term><![CDATA[Coils]]></term>
##       <term><![CDATA[Couplings]]></term>
##       <term><![CDATA[Magnetic noise]]></term>
##       <term><![CDATA[Magnetic shielding]]></term>
##       <term><![CDATA[Transceivers]]></term>
##       <term><![CDATA[Wireless communication]]></term>
##     </thesaurusterms>
##     <pubtitle><![CDATA[Electron Devices, IEEE Transactions on]]></pubtitle>
##     <punumber><![CDATA[16]]></punumber>
##     <pubtype><![CDATA[Journals & Magazines]]></pubtype>
##     <publisher><![CDATA[IEEE]]></publisher>
##     <volume><![CDATA[61]]></volume>
##     <issue><![CDATA[9]]></issue>
##     <py><![CDATA[2014]]></py>
##     <spage><![CDATA[3245]]></spage>
##     <epage><![CDATA[3251]]></epage>
##     <abstract><![CDATA[In this paper, we propose a zigzag-shaped coil array structure for wireless chip-to-chip communication. The proposed structure is designed for high-speed memory, which requires a pad number on the order of tens. First, a general coil array is investigated with respect to undesired coupling between adjacent coils. We also investigated a coil array for which shielding patterns are inserted between adjacent coils to reduce cross-talk problems. To solve the cross-talk and chip area problems associated with coil arrays, a zigzag pattern is proposed. Additionally, a layout technique for the circuit block of the memory transceiver that is connected to the coil array is proposed to minimize the total chip size. With the experimental results, the feasibility of the proposed zigzag-shaped coil array is successfully demonstrated.]]></abstract>
##     <issn><![CDATA[0018-9383]]></issn>
##     <htmlFlag><![CDATA[1]]></htmlFlag>
##     <arnumber><![CDATA[6849986]]></arnumber>
##     <doi><![CDATA[10.1109/TED.2014.2333517]]></doi>
##     <publicationId><![CDATA[6849986]]></publicationId>
##     <mdurl><![CDATA[http://ieeexplore.ieee.org/xpl/articleDetails.jsp?tp=&arnumber=6849986&contentType=Journals+%26+Magazines]]></mdurl>
##     <pdf><![CDATA[http://ieeexplore.ieee.org/stamp/stamp.jsp?arnumber=6849986]]></pdf>
##   </document>
##   <document>
##     <rank>1117</rank>
##     <title><![CDATA[Time-Lapse Imaging of Human Heart Motion With Switched Array UWB Radar]]></title>
##     <authors><![CDATA[Brovoll, S.;  Berger, T.;  Paichard, Y.;  Aardal, O.;  Lande, T.S.;  Hamran, S.-E.]]></authors>
##     <affiliations><![CDATA[Norwegian Defence Res. Establ. (FFI), Kjeller, Norway]]></affiliations>
##     <controlledterms>
##       <term><![CDATA[antenna arrays]]></term>
##       <term><![CDATA[electrocardiography]]></term>
##       <term><![CDATA[image sequences]]></term>
##       <term><![CDATA[medical image processing]]></term>
##       <term><![CDATA[ultra wideband antennas]]></term>
##       <term><![CDATA[ultra wideband radar]]></term>
##     </controlledterms>
##     <thesaurusterms>
##       <term><![CDATA[Arrays]]></term>
##       <term><![CDATA[Heart]]></term>
##       <term><![CDATA[Radar antennas]]></term>
##       <term><![CDATA[Radar imaging]]></term>
##       <term><![CDATA[Switches]]></term>
##     </thesaurusterms>
##     <pubtitle><![CDATA[Biomedical Circuits and Systems, IEEE Transactions on]]></pubtitle>
##     <punumber><![CDATA[4156126]]></punumber>
##     <pubtype><![CDATA[Journals & Magazines]]></pubtype>
##     <publisher><![CDATA[IEEE]]></publisher>
##     <volume><![CDATA[8]]></volume>
##     <issue><![CDATA[5]]></issue>
##     <py><![CDATA[2014]]></py>
##     <spage><![CDATA[704]]></spage>
##     <epage><![CDATA[715]]></epage>
##     <abstract><![CDATA[Radar systems for detection of human heartbeats have mostly been single-channel systems with limited spatial resolution. In this paper, a radar system for ultra-wideband (UWB) imaging of the human heart is presented. To make the radar waves penetrate the human tissue the antenna is placed very close to the body. The antenna is an array with eight elements, and an antenna switch system connects the radar to the individual elements in sequence to form an image. Successive images are used to build up time-lapse movies of the beating heart. Measurements on a human test subject are presented and the heart motion is estimated at different locations inside the body. The movies show rhythmic motion consistent with the beating heart, and the location and shape of the reflections correspond well with the expected response form the heart wall. The spatial dependent heart motion is compared to ECG recordings, and it is confirmed that heartbeat modulations are seen in the radar data. This work shows that radar imaging of the human heart may provide valuable information on the mechanical movement of the heart.]]></abstract>
##     <issn><![CDATA[1932-4545]]></issn>
##     <htmlFlag><![CDATA[1]]></htmlFlag>
##     <arnumber><![CDATA[6936390]]></arnumber>
##     <doi><![CDATA[10.1109/TBCAS.2014.2359995]]></doi>
##     <publicationId><![CDATA[6936390]]></publicationId>
##     <mdurl><![CDATA[http://ieeexplore.ieee.org/xpl/articleDetails.jsp?tp=&arnumber=6936390&contentType=Journals+%26+Magazines]]></mdurl>
##     <pdf><![CDATA[http://ieeexplore.ieee.org/stamp/stamp.jsp?arnumber=6936390]]></pdf>
##   </document>
##   <document>
##     <rank>1118</rank>
##     <title><![CDATA[Giant Tunable Goos&#x2013;H&#x00E4;nchen Shifts Based on Prism/Graphene Structure in Terahertz Wave Region]]></title>
##     <authors><![CDATA[Li Jiu-Sheng;  Wu Jing-fang;  Zhang Le]]></authors>
##     <affiliations><![CDATA[Center for Terahertz Res., China Jiliang Univ., Hangzhou, China]]></affiliations>
##     <controlledterms>
##       <term><![CDATA[electric fields]]></term>
##       <term><![CDATA[graphene]]></term>
##       <term><![CDATA[optical prisms]]></term>
##       <term><![CDATA[terahertz wave spectra]]></term>
##     </controlledterms>
##     <thesaurusterms>
##       <term><![CDATA[Conductivity]]></term>
##       <term><![CDATA[Electric fields]]></term>
##       <term><![CDATA[Graphene]]></term>
##       <term><![CDATA[Permittivity]]></term>
##     </thesaurusterms>
##     <pubtitle><![CDATA[Photonics Journal, IEEE]]></pubtitle>
##     <punumber><![CDATA[4563994]]></punumber>
##     <pubtype><![CDATA[Journals & Magazines]]></pubtype>
##     <publisher><![CDATA[IEEE]]></publisher>
##     <volume><![CDATA[6]]></volume>
##     <issue><![CDATA[6]]></issue>
##     <py><![CDATA[2014]]></py>
##     <spage><![CDATA[1]]></spage>
##     <epage><![CDATA[7]]></epage>
##     <abstract><![CDATA[We investigate the giant tunable Goos-Ha&#x0308;nchen shifts in prism/graphene structure. It is found that large positive and negative lateral Goos-Ha&#x0308;nchen shifts can be easily controlled by adjusting the chemical potential of the graphene, which is modulated by the external applied electric field. By using the stationary-phase method, we theoretically analyze the effect of the external applied electric field on Goos-Ha&#x0308;nchen shifts of the proposed prism/graphene structure. Our theoretical study shows that the magnitude of the Goos-Ha&#x0308;nchen shift is more than 200 times that of the operating terahertz wavelength. Numerical calculation results further indicate that the present structure has the potential application for the terahertz-wave switch.]]></abstract>
##     <issn><![CDATA[1943-0655]]></issn>
##     <htmlFlag><![CDATA[1]]></htmlFlag>
##     <arnumber><![CDATA[6967762]]></arnumber>
##     <doi><![CDATA[10.1109/JPHOT.2014.2374591]]></doi>
##     <publicationId><![CDATA[6967762]]></publicationId>
##     <mdurl><![CDATA[http://ieeexplore.ieee.org/xpl/articleDetails.jsp?tp=&arnumber=6967762&contentType=Journals+%26+Magazines]]></mdurl>
##     <pdf><![CDATA[http://ieeexplore.ieee.org/stamp/stamp.jsp?arnumber=6967762]]></pdf>
##   </document>
##   <document>
##     <rank>1119</rank>
##     <title><![CDATA[ClubCF: A Clustering-Based Collaborative Filtering Approach for Big Data Application]]></title>
##     <authors><![CDATA[Rong Hu;  Wanchun Dou;  Jianxun Liu]]></authors>
##     <affiliations><![CDATA[Dept. of Comput. Sci. & Technol., Nanjing Univ., Nanjing, China]]></affiliations>
##     <controlledterms>
##       <term><![CDATA[Big Data]]></term>
##       <term><![CDATA[cloud computing]]></term>
##       <term><![CDATA[collaborative filtering]]></term>
##       <term><![CDATA[pattern clustering]]></term>
##       <term><![CDATA[service-oriented architecture]]></term>
##     </controlledterms>
##     <thesaurusterms>
##       <term><![CDATA[Cloud computing]]></term>
##       <term><![CDATA[Clustering algorithms]]></term>
##       <term><![CDATA[Data handling]]></term>
##       <term><![CDATA[Data storage systems]]></term>
##       <term><![CDATA[Filtering]]></term>
##       <term><![CDATA[Information management]]></term>
##       <term><![CDATA[Mashups]]></term>
##     </thesaurusterms>
##     <pubtitle><![CDATA[Emerging Topics in Computing, IEEE Transactions on]]></pubtitle>
##     <punumber><![CDATA[6245516]]></punumber>
##     <pubtype><![CDATA[Journals & Magazines]]></pubtype>
##     <publisher><![CDATA[IEEE]]></publisher>
##     <volume><![CDATA[2]]></volume>
##     <issue><![CDATA[3]]></issue>
##     <py><![CDATA[2014]]></py>
##     <spage><![CDATA[302]]></spage>
##     <epage><![CDATA[313]]></epage>
##     <abstract><![CDATA[Spurred by service computing and cloud computing, an increasing number of services are emerging on the Internet. As a result, service-relevant data become too big to be effectively processed by traditional approaches. In view of this challenge, a clustering-based collaborative filtering approach is proposed in this paper, which aims at recruiting similar services in the same clusters to recommend services collaboratively. Technically, this approach is enacted around two stages. In the first stage, the available services are divided into small-scale clusters, in logic, for further processing. At the second stage, a collaborative filtering algorithm is imposed on one of the clusters. Since the number of the services in a cluster is much less than the total number of the services available on the web, it is expected to reduce the online execution time of collaborative filtering. At last, several experiments are conducted to verify the availability of the approach, on a real data set of 6225 mashup services collected from ProgrammableWeb.]]></abstract>
##     <issn><![CDATA[2168-6750]]></issn>
##     <htmlFlag><![CDATA[1]]></htmlFlag>
##     <arnumber><![CDATA[6763038]]></arnumber>
##     <doi><![CDATA[10.1109/TETC.2014.2310485]]></doi>
##     <publicationId><![CDATA[6763038]]></publicationId>
##     <mdurl><![CDATA[http://ieeexplore.ieee.org/xpl/articleDetails.jsp?tp=&arnumber=6763038&contentType=Journals+%26+Magazines]]></mdurl>
##     <pdf><![CDATA[http://ieeexplore.ieee.org/stamp/stamp.jsp?arnumber=6763038]]></pdf>
##   </document>
##   <document>
##     <rank>1120</rank>
##     <title><![CDATA[A Multiobjective Evolutionary Algorithm Based on Similarity for Community Detection From Signed Social Networks]]></title>
##     <authors><![CDATA[Chenlong Liu;  Jing Liu;  Zhongzhou Jiang]]></authors>
##     <affiliations><![CDATA[Key Lab. of Intell. Perception & Image Understanding of Minist. of Educ., Xidian Univ., Xian, China]]></affiliations>
##     <controlledterms>
##       <term><![CDATA[complex networks]]></term>
##       <term><![CDATA[evolutionary computation]]></term>
##       <term><![CDATA[network theory (graphs)]]></term>
##     </controlledterms>
##     <thesaurusterms>
##       <term><![CDATA[Communities]]></term>
##       <term><![CDATA[Decoding]]></term>
##       <term><![CDATA[Evolutionary computation]]></term>
##       <term><![CDATA[Linear programming]]></term>
##       <term><![CDATA[Social network services]]></term>
##       <term><![CDATA[Time complexity]]></term>
##       <term><![CDATA[Vectors]]></term>
##     </thesaurusterms>
##     <pubtitle><![CDATA[Cybernetics, IEEE Transactions on]]></pubtitle>
##     <punumber><![CDATA[6221036]]></punumber>
##     <pubtype><![CDATA[Journals & Magazines]]></pubtype>
##     <publisher><![CDATA[IEEE]]></publisher>
##     <volume><![CDATA[44]]></volume>
##     <issue><![CDATA[12]]></issue>
##     <py><![CDATA[2014]]></py>
##     <spage><![CDATA[2274]]></spage>
##     <epage><![CDATA[2287]]></epage>
##     <abstract><![CDATA[Various types of social relationships, such as friends and foes, can be represented as signed social networks (SNs) that contain both positive and negative links. Although many community detection (CD) algorithms have been proposed, most of them were designed primarily for networks containing only positive links. Thus, it is important to design CD algorithms which can handle large-scale SNs. To this purpose, we first extend the original similarity to the signed similarity based on the social balance theory. Then, based on the signed similarity and the natural contradiction between positive and negative links, two objective functions are designed to model the problem of detecting communities in SNs as a multiobjective problem. Afterward, we propose a multiobjective evolutionary algorithm, called MEAsSN. In MEAs-SN, to overcome the defects of direct and indirect representations for communities, a direct and indirect combined representation is designed. Attributing to this representation, MEAs-SN can switch between different representations during the evolutionary process. As a result, MEAs-SN can benefit from both representations. Moreover, owing to this representation, MEAs-SN can also detect overlapping communities directly. In the experiments, both benchmark problems and large-scale synthetic networks generated by various parameter settings are used to validate the performance of MEAs-SN. The experimental results show the effectiveness and efficacy of MEAs-SN on networks with 1000, 5000, and 10000 nodes and also in various noisy situations. A thorough comparison is also made between MEAs-SN and three existing algorithms, and the results show that MEAs-SN outperforms other algorithms.]]></abstract>
##     <issn><![CDATA[2168-2267]]></issn>
##     <htmlFlag><![CDATA[1]]></htmlFlag>
##     <arnumber><![CDATA[6755451]]></arnumber>
##     <doi><![CDATA[10.1109/TCYB.2014.2305974]]></doi>
##     <publicationId><![CDATA[6755451]]></publicationId>
##     <mdurl><![CDATA[http://ieeexplore.ieee.org/xpl/articleDetails.jsp?tp=&arnumber=6755451&contentType=Journals+%26+Magazines]]></mdurl>
##     <pdf><![CDATA[http://ieeexplore.ieee.org/stamp/stamp.jsp?arnumber=6755451]]></pdf>
##   </document>
##   <document>
##     <rank>1121</rank>
##     <title><![CDATA[BER-Driven Resource Allocation for Scalable Bitstreams Over OFDMA Networks]]></title>
##     <authors><![CDATA[Mau-Luen Tham;  Chee-Onn Chow;  Iwahashi, M.;  Ishii, H.]]></authors>
##     <affiliations><![CDATA[Dept. of Electr. Eng., Univ. of Malaya, Kuala Lumpur, Malaysia]]></affiliations>
##     <controlledterms>
##       <term><![CDATA[OFDM modulation]]></term>
##       <term><![CDATA[error statistics]]></term>
##       <term><![CDATA[frequency division multiple access]]></term>
##       <term><![CDATA[image reconstruction]]></term>
##       <term><![CDATA[resource allocation]]></term>
##     </controlledterms>
##     <thesaurusterms>
##       <term><![CDATA[Decoding]]></term>
##       <term><![CDATA[Encoding]]></term>
##       <term><![CDATA[Licenses]]></term>
##       <term><![CDATA[Modulation]]></term>
##       <term><![CDATA[Multimedia communication]]></term>
##       <term><![CDATA[OFDM]]></term>
##       <term><![CDATA[Resource management]]></term>
##     </thesaurusterms>
##     <pubtitle><![CDATA[Vehicular Technology, IEEE Transactions on]]></pubtitle>
##     <punumber><![CDATA[25]]></punumber>
##     <pubtype><![CDATA[Journals & Magazines]]></pubtype>
##     <publisher><![CDATA[IEEE]]></publisher>
##     <volume><![CDATA[63]]></volume>
##     <issue><![CDATA[6]]></issue>
##     <py><![CDATA[2014]]></py>
##     <spage><![CDATA[2755]]></spage>
##     <epage><![CDATA[2768]]></epage>
##     <abstract><![CDATA[Conventional resource-allocation schemes in orthogonal frequency-division multiple-access (OFDMA) systems consider only one bit-error-rate (BER) requirement. When these schemes are applied to transmit bitstreams with multiple BER requirements, low resource utilization will occur since the target BER has to be the strictest one. We develop a BER-driven resource allocation for generic bitstreams (BRA-G) algorithm by exploiting multiple predetermined target BERs. Then, this scheme is extended to transmission of realistic scalable bitstreams (BRA-SB) where more important (MI) layers possess lower target BERs as compared with less important (LI) layers. The objective here is to maximize the expected quality of all JPEG 2000 (J2K) scalable bitstreams subject to a total power constraint. Instead of using static target BERs for different layers, BRA-SB water-fills over all layers to determine the number of transmit layers and the target BER of each layer simultaneously based on the importance of each layer in terms of reconstructed quality and average channel conditions. Bit-level results show that BRA-SB provides significant image quality improvement over a scheme that ignores the layer's importance in terms of target BER and a BRA-SB-like scheme with static target BERs. A suboptimal algorithm is also proposed to reduce the computational complexity and its performance is close to the optimal one.]]></abstract>
##     <issn><![CDATA[0018-9545]]></issn>
##     <htmlFlag><![CDATA[1]]></htmlFlag>
##     <arnumber><![CDATA[6690241]]></arnumber>
##     <doi><![CDATA[10.1109/TVT.2013.2295417]]></doi>
##     <publicationId><![CDATA[6690241]]></publicationId>
##     <mdurl><![CDATA[http://ieeexplore.ieee.org/xpl/articleDetails.jsp?tp=&arnumber=6690241&contentType=Journals+%26+Magazines]]></mdurl>
##     <pdf><![CDATA[http://ieeexplore.ieee.org/stamp/stamp.jsp?arnumber=6690241]]></pdf>
##   </document>
##   <document>
##     <rank>1122</rank>
##     <title><![CDATA[Automatic Model Creation to Support Network Monitoring]]></title>
##     <authors><![CDATA[Maatta, M.;  Raty, T.]]></authors>
##     <affiliations><![CDATA[VTT Tech. Res. Centre of Finland, Oulu, Finland]]></affiliations>
##     <controlledterms>
##       <term><![CDATA[IP networks]]></term>
##       <term><![CDATA[telecommunication traffic]]></term>
##       <term><![CDATA[transport protocols]]></term>
##     </controlledterms>
##     <thesaurusterms>
##       <term><![CDATA[IP networks]]></term>
##       <term><![CDATA[Mathematical model]]></term>
##       <term><![CDATA[Monitoring]]></term>
##       <term><![CDATA[Ports (Computers)]]></term>
##       <term><![CDATA[Protocols]]></term>
##       <term><![CDATA[Telecommunication traffic]]></term>
##       <term><![CDATA[XML]]></term>
##     </thesaurusterms>
##     <pubtitle><![CDATA[Access, IEEE]]></pubtitle>
##     <punumber><![CDATA[6287639]]></punumber>
##     <pubtype><![CDATA[Journals & Magazines]]></pubtype>
##     <publisher><![CDATA[IEEE]]></publisher>
##     <volume><![CDATA[2]]></volume>
##     <py><![CDATA[2014]]></py>
##     <spage><![CDATA[142]]></spage>
##     <epage><![CDATA[152]]></epage>
##     <abstract><![CDATA[The large variety of network traffic sets many challenges in modeling the essential aspects of network traffic flows. Analyzing and collecting features for the model creation process from the network traffic traces is a time-consuming and error-prone task. Automating these procedures are a challenge. The research problem discussed in this paper concentrates on the analysis and collection of features from the network traffic traces for the model development process, by automating the analysis and collection. The proposed system of this paper, called MGtoolV2, supports the model development process through the automation of collection and analysis in the actual model creation procedures. The model development process aims to enhance the development of a model by reducing the development cost and time. The proposed tool automatically creates large sets of models according to the network traffic traces and minimizes the errors of manual modeling. The experiments conducted with MGtoolV2 indicate that the tool is able to create the models from the traffic traces cost effectively. MGtoolV2 is able to unify similarities between packets, to create very detailed models describing specific information, and to raise the abstraction level of the created models. The research is based on the constructive method of the related publications and technologies, and the results are established from the testing, validation, and analysis of the implemented MGtoolV2.]]></abstract>
##     <issn><![CDATA[2169-3536]]></issn>
##     <arnumber><![CDATA[6748008]]></arnumber>
##     <doi><![CDATA[10.1109/ACCESS.2014.2308320]]></doi>
##     <publicationId><![CDATA[6748008]]></publicationId>
##     <mdurl><![CDATA[http://ieeexplore.ieee.org/xpl/articleDetails.jsp?tp=&arnumber=6748008&contentType=Journals+%26+Magazines]]></mdurl>
##     <pdf><![CDATA[http://ieeexplore.ieee.org/stamp/stamp.jsp?arnumber=6748008]]></pdf>
##   </document>
##   <document>
##     <rank>1123</rank>
##     <title><![CDATA[Simulation of the Expected Performance of a Seamless Scanner for Brain PET Based on Highly Pixelated CdTe Detectors]]></title>
##     <authors><![CDATA[Mikhaylova, E.;  De Lorenzo, G.;  Chmeissani, M.;  Kolstein, M.;  Canadas, M.;  Arce, P.;  Calderon, Y.;  Uzun, D.;  Arino, G.;  Macias-Montero, J.G.;  Martinez, R.;  Puigdengoles, C.;  Cabruja, E.]]></authors>
##     <affiliations><![CDATA[Inst. de Fis. d'Altes Energies (IFAE), Univ. Autonoma de Barcelona (UAB), Bellaterra, Spain]]></affiliations>
##     <controlledterms>
##       <term><![CDATA[II-VI semiconductors]]></term>
##       <term><![CDATA[biomedical equipment]]></term>
##       <term><![CDATA[brain]]></term>
##       <term><![CDATA[cadmium compounds]]></term>
##       <term><![CDATA[image reconstruction]]></term>
##       <term><![CDATA[image resolution]]></term>
##       <term><![CDATA[medical image processing]]></term>
##       <term><![CDATA[optimisation]]></term>
##       <term><![CDATA[positron emission tomography]]></term>
##       <term><![CDATA[scintillation]]></term>
##       <term><![CDATA[scintillation counters]]></term>
##       <term><![CDATA[semiconductor counters]]></term>
##       <term><![CDATA[sensitivity]]></term>
##       <term><![CDATA[wide band gap semiconductors]]></term>
##     </controlledterms>
##     <thesaurusterms>
##       <term><![CDATA[Detectors]]></term>
##       <term><![CDATA[Image quality]]></term>
##       <term><![CDATA[Materials]]></term>
##       <term><![CDATA[Phantoms]]></term>
##       <term><![CDATA[Photonics]]></term>
##       <term><![CDATA[Positron emission tomography]]></term>
##       <term><![CDATA[Sensitivity]]></term>
##     </thesaurusterms>
##     <pubtitle><![CDATA[Medical Imaging, IEEE Transactions on]]></pubtitle>
##     <punumber><![CDATA[42]]></punumber>
##     <pubtype><![CDATA[Journals & Magazines]]></pubtype>
##     <publisher><![CDATA[IEEE]]></publisher>
##     <volume><![CDATA[33]]></volume>
##     <issue><![CDATA[2]]></issue>
##     <py><![CDATA[2014]]></py>
##     <spage><![CDATA[332]]></spage>
##     <epage><![CDATA[339]]></epage>
##     <abstract><![CDATA[The aim of this work is the evaluation of the design for a nonconventional PET scanner, the voxel imaging PET (VIP), based on pixelated room-temperature CdTe detectors yielding a true 3-D impact point with a density of 450 channels/cm<sup>3</sup>, for a total 6 336 000 channels in a seamless ring shaped volume. The system is simulated and evaluated following the prescriptions of the NEMA NU 2-2001 and the NEMA NU 4-2008 standards. Results show that the excellent energy resolution of the CdTe detectors (1.6% for 511 keV photons), together with the small voxel pitch (1 &#x00D7; 1 &#x00D7; 2 mm<sup>3</sup>), and the crack-free ring geometry, give the design the potential to overcome the current limitations of PET scanners and to approach the intrinsic image resolution limits set by physics. The VIP is expected to reach a competitive sensitivity and a superior signal purity with respect to values commonly quoted for state-of-the-art scintillating crystal PETs. The system can provide 14 cps/kBq with a scatter fraction of 3.95% and 21 cps/kBq with a scatter fraction of 0.73% according to NEMA NU 2-2001 and NEMA NU 4-2008, respectively. The calculated NEC curve has a peak value of 122 kcps at 5.3 kBq/mL for NEMA NU 2-2001 and 908 kcps at 1.6 MBq/mL for NEMA NU 4-2008. The proposed scanner can achieve an image resolution of ~ 1 mm full-width at half-maximum in all directions. The virtually noise-free data sample leads to direct positive impact on the quality of the reconstructed images. As a consequence, high-quality high-resolution images can be obtained with significantly lower number of events compared to conventional scanners. Overall, simulation results suggest the VIP scanner can be operated either at normal dose for fast scanning and high patient throughput, or at low dose to decrease the patient radioactivity exposure. The design evaluation presented in this work is driving the development and the optimization of a fully operative prototype to prove the feasibili- y of the VIP concept.]]></abstract>
##     <issn><![CDATA[0278-0062]]></issn>
##     <htmlFlag><![CDATA[1]]></htmlFlag>
##     <arnumber><![CDATA[6620952]]></arnumber>
##     <doi><![CDATA[10.1109/TMI.2013.2284657]]></doi>
##     <publicationId><![CDATA[6620952]]></publicationId>
##     <mdurl><![CDATA[http://ieeexplore.ieee.org/xpl/articleDetails.jsp?tp=&arnumber=6620952&contentType=Journals+%26+Magazines]]></mdurl>
##     <pdf><![CDATA[http://ieeexplore.ieee.org/stamp/stamp.jsp?arnumber=6620952]]></pdf>
##   </document>
##   <document>
##     <rank>1124</rank>
##     <title><![CDATA[On the Development of Nonoverlapping and Stable Hybrid FETD-FDTD Formulations]]></title>
##     <authors><![CDATA[Akbarzadeh-Sharbaf, A.;  Giannacopoulos, D.D.]]></authors>
##     <affiliations><![CDATA[Dept. of Electr. & Comput. Eng., McGill Univ., Montreal, QC, Canada]]></affiliations>
##     <controlledterms>
##       <term><![CDATA[electromagnetic field theory]]></term>
##       <term><![CDATA[finite difference time-domain analysis]]></term>
##       <term><![CDATA[finite element analysis]]></term>
##       <term><![CDATA[wave equations]]></term>
##     </controlledterms>
##     <thesaurusterms>
##       <term><![CDATA[Finite difference methods]]></term>
##       <term><![CDATA[Finite element analysis]]></term>
##       <term><![CDATA[Numerical stability]]></term>
##       <term><![CDATA[Stability criteria]]></term>
##       <term><![CDATA[Time-domain analysis]]></term>
##     </thesaurusterms>
##     <pubtitle><![CDATA[Antennas and Propagation, IEEE Transactions on]]></pubtitle>
##     <punumber><![CDATA[8]]></punumber>
##     <pubtype><![CDATA[Journals & Magazines]]></pubtype>
##     <publisher><![CDATA[IEEE]]></publisher>
##     <volume><![CDATA[62]]></volume>
##     <issue><![CDATA[12]]></issue>
##     <py><![CDATA[2014]]></py>
##     <spage><![CDATA[6299]]></spage>
##     <epage><![CDATA[6306]]></epage>
##     <abstract><![CDATA[A general framework to combine the finite-difference time-domain (FDTD) and the finite-element time-domain (FETD) formulations, both based on the vector wave equation, is proposed. In contrast to the existing stable hybrid FETD-FDTD, there is no transition layer between two subdomains. In addition, the stability of the proposed approach is analytically proved. This framework allows combining different FDTD and FETD formulations together. Particularly, a fully unconditionally stable hybrid method is proposed, which is proved to be energy conservative too. The key ingredient is a finite-element tearing and interconnecting method for electromagnetic problems with a new interface condition that preserves the stability of the numerical method in each region. Several numerical examples are considered in order to validate the proposed methods. The numerical results match with the reference solutions very well in all cases.]]></abstract>
##     <issn><![CDATA[0018-926X]]></issn>
##     <htmlFlag><![CDATA[1]]></htmlFlag>
##     <arnumber><![CDATA[6905779]]></arnumber>
##     <doi><![CDATA[10.1109/TAP.2014.2359496]]></doi>
##     <publicationId><![CDATA[6905779]]></publicationId>
##     <mdurl><![CDATA[http://ieeexplore.ieee.org/xpl/articleDetails.jsp?tp=&arnumber=6905779&contentType=Journals+%26+Magazines]]></mdurl>
##     <pdf><![CDATA[http://ieeexplore.ieee.org/stamp/stamp.jsp?arnumber=6905779]]></pdf>
##   </document>
##   <document>
##     <rank>1125</rank>
##     <title><![CDATA[Achieving Source Location Privacy and Network Lifetime Maximization Through Tree-Based Diversionary Routing in Wireless Sensor Networks]]></title>
##     <authors><![CDATA[Jun Long;  Mianxiong Dong;  Ota, K.;  Anfeng Liu]]></authors>
##     <affiliations><![CDATA[Sch. of Inf. Sci. & Eng., Central South Univ., Changsha, China]]></affiliations>
##     <controlledterms>
##       <term><![CDATA[telecommunication network routing]]></term>
##       <term><![CDATA[telecommunication security]]></term>
##       <term><![CDATA[trees (mathematics)]]></term>
##       <term><![CDATA[wireless sensor networks]]></term>
##     </controlledterms>
##     <thesaurusterms>
##       <term><![CDATA[Energy consumption]]></term>
##       <term><![CDATA[Position measurement]]></term>
##       <term><![CDATA[Privacy]]></term>
##       <term><![CDATA[Routing protocols]]></term>
##       <term><![CDATA[Source location]]></term>
##       <term><![CDATA[Wireless sensor networks]]></term>
##     </thesaurusterms>
##     <pubtitle><![CDATA[Access, IEEE]]></pubtitle>
##     <punumber><![CDATA[6287639]]></punumber>
##     <pubtype><![CDATA[Journals & Magazines]]></pubtype>
##     <publisher><![CDATA[IEEE]]></publisher>
##     <volume><![CDATA[2]]></volume>
##     <py><![CDATA[2014]]></py>
##     <spage><![CDATA[633]]></spage>
##     <epage><![CDATA[651]]></epage>
##     <abstract><![CDATA[Wireless sensor networks (WSNs) have been proliferating due to their wide applications in both military and commercial use. However, one critical challenge to WSNs implementation is source location privacy. In this paper, we propose a novel tree-based diversionary routing scheme for preserving source location privacy using hide and seek strategy to create diversionary or decoy routes along the path to the sink from the real source, where the end of each diversionary route is a decoy (fake source node), which periodically emits fake events. Meanwhile, the proposed scheme is able to maximize the network lifetime of WSNs. The main idea is that the lifetime of WSNs depends on the nodes with high energy consumption or hotspot, and then the proposed scheme minimizes energy consumption in hotspot and creates redundancy diversionary routes in nonhotspot regions with abundant energy. Hence, it achieves not only privacy preservation, but also network lifetime maximization. Furthermore, we systematically analyze the energy consumption in WSNs, and provide guidance on the number of diversionary routes, which can be created in different regions away from the sink. In addition, we identify a novel attack against phantom routing, which is widely used for source location privacy preservation, namely, direction-oriented attack. We also perform a comprehensive analysis on how the direction-oriented attack can be defeated by the proposed scheme. Theoretical and experimental results show that our scheme is very effective to improve the privacy protection while maximizing the network lifetime.]]></abstract>
##     <issn><![CDATA[2169-3536]]></issn>
##     <htmlFlag><![CDATA[1]]></htmlFlag>
##     <arnumber><![CDATA[6842655]]></arnumber>
##     <doi><![CDATA[10.1109/ACCESS.2014.2332817]]></doi>
##     <publicationId><![CDATA[6842655]]></publicationId>
##     <mdurl><![CDATA[http://ieeexplore.ieee.org/xpl/articleDetails.jsp?tp=&arnumber=6842655&contentType=Journals+%26+Magazines]]></mdurl>
##     <pdf><![CDATA[http://ieeexplore.ieee.org/stamp/stamp.jsp?arnumber=6842655]]></pdf>
##   </document>
##   <document>
##     <rank>1126</rank>
##     <title><![CDATA[A Hybrid WDM Lightwave Transport System Based on Fiber-Wireless and Fiber-VLLC Convergences]]></title>
##     <authors><![CDATA[Cheng-Ling Ying;  Chung-Yi Li;  Hai-Han Lu;  Ching-Hung Chang;  Jian-Hua Chen;  Jun-Ren Zheng]]></authors>
##     <affiliations><![CDATA[Dept. of Electron. Eng., JinWen Univ. of Sci. & Technol., New Taipei, Taiwan]]></affiliations>
##     <controlledterms>
##       <term><![CDATA[error statistics]]></term>
##       <term><![CDATA[optical fibre communication]]></term>
##       <term><![CDATA[wavelength division multiplexing]]></term>
##     </controlledterms>
##     <thesaurusterms>
##       <term><![CDATA[Optical amplifiers]]></term>
##       <term><![CDATA[Optical attenuators]]></term>
##       <term><![CDATA[Optical fiber communication]]></term>
##       <term><![CDATA[Optical fibers]]></term>
##       <term><![CDATA[Optical filters]]></term>
##       <term><![CDATA[Wavelength division multiplexing]]></term>
##     </thesaurusterms>
##     <pubtitle><![CDATA[Photonics Journal, IEEE]]></pubtitle>
##     <punumber><![CDATA[4563994]]></punumber>
##     <pubtype><![CDATA[Journals & Magazines]]></pubtype>
##     <publisher><![CDATA[IEEE]]></publisher>
##     <volume><![CDATA[6]]></volume>
##     <issue><![CDATA[6]]></issue>
##     <py><![CDATA[2014]]></py>
##     <spage><![CDATA[1]]></spage>
##     <epage><![CDATA[9]]></epage>
##     <abstract><![CDATA[A hybrid wavelength-division-multiplexing (WDM) lightwave transport system for millimeter-wave (MMW)/microwave (MW)/baseband (BB) signals that are transmission based on fiber-wireless and fiber-visible laser light communication (VLLC) convergences is proposed and demonstrated. A broadband light source with an optical signal-to-noise ratio enhancement scheme in a hybrid lightwave transport system is employed. Light is optically promoted from a 5-Gbps/15-GHz RF data signal to 5-Gbps/60-GHz MMW and 5-Gbps/30-GHz MW data signals in fiber-wireless convergence. Light is also optically demoted from a 5-Gbps/15-GHz RF data signal to a 5-Gbps data stream in fiber-VLLC convergence. Over a 40-km single-mode fiber and a 4-m RF wireless/a 10-m free-space VLLC transmission, bit error rate performs impressively for 60-GHz MMW, 30-GHz MW, and 5-Gbps BB signal transmission. Such a hybrid WDM lightwave transport system could have practical applications for fiber-wireless and fiber-VLLC convergences to provide broadband integrated services.]]></abstract>
##     <issn><![CDATA[1943-0655]]></issn>
##     <htmlFlag><![CDATA[1]]></htmlFlag>
##     <arnumber><![CDATA[6957639]]></arnumber>
##     <doi><![CDATA[10.1109/JPHOT.2014.2366124]]></doi>
##     <publicationId><![CDATA[6957639]]></publicationId>
##     <mdurl><![CDATA[http://ieeexplore.ieee.org/xpl/articleDetails.jsp?tp=&arnumber=6957639&contentType=Journals+%26+Magazines]]></mdurl>
##     <pdf><![CDATA[http://ieeexplore.ieee.org/stamp/stamp.jsp?arnumber=6957639]]></pdf>
##   </document>
##   <document>
##     <rank>1127</rank>
##     <title><![CDATA[Efficiency Enhancement of III-V Triple-Junction Solar Cell Using Nanostructured Bifunctional Coverglass With Enhanced Transmittance and Self-Cleaning Property]]></title>
##     <authors><![CDATA[Chan Il Yeo;  Eun Kyu Kang;  Soo Kyung Lee;  Young Min Song;  Yong Tak Lee]]></authors>
##     <affiliations><![CDATA[Sch. of Inf. & Commun., Gwangju Inst. of Sci. & Technol., Gwangju, South Korea]]></affiliations>
##     <controlledterms>
##       <term><![CDATA[III-V semiconductors]]></term>
##       <term><![CDATA[cleaning]]></term>
##       <term><![CDATA[current density]]></term>
##       <term><![CDATA[elemental semiconductors]]></term>
##       <term><![CDATA[gallium arsenide]]></term>
##       <term><![CDATA[germanium]]></term>
##       <term><![CDATA[glass]]></term>
##       <term><![CDATA[indium compounds]]></term>
##       <term><![CDATA[light absorption]]></term>
##       <term><![CDATA[nanofabrication]]></term>
##       <term><![CDATA[nanostructured materials]]></term>
##       <term><![CDATA[photoconductivity]]></term>
##       <term><![CDATA[photoemission]]></term>
##       <term><![CDATA[short-circuit currents]]></term>
##       <term><![CDATA[solar cells]]></term>
##     </controlledterms>
##     <thesaurusterms>
##       <term><![CDATA[Absorption]]></term>
##       <term><![CDATA[Gallium arsenide]]></term>
##       <term><![CDATA[Glass]]></term>
##       <term><![CDATA[Ink]]></term>
##       <term><![CDATA[Nanostructures]]></term>
##       <term><![CDATA[Photoconductivity]]></term>
##       <term><![CDATA[Photovoltaic cells]]></term>
##     </thesaurusterms>
##     <pubtitle><![CDATA[Photonics Journal, IEEE]]></pubtitle>
##     <punumber><![CDATA[4563994]]></punumber>
##     <pubtype><![CDATA[Journals & Magazines]]></pubtype>
##     <publisher><![CDATA[IEEE]]></publisher>
##     <volume><![CDATA[6]]></volume>
##     <issue><![CDATA[3]]></issue>
##     <py><![CDATA[2014]]></py>
##     <spage><![CDATA[1]]></spage>
##     <epage><![CDATA[9]]></epage>
##     <abstract><![CDATA[We present bifunctional nanostructured (NS) coverglasses with enhanced transmittance and self-cleaning function for improving the efficiency of a photovoltaic (PV) module with an InGaP/GaAs/Ge triple-junction solar cell. Prior to the fabrication of NS coverglasses, theoretical investigations were carried out using a rigorous coupled-wave analysis method to determine the desirable glass nanostructures that can effectively enhance the light absorption of the PV module. The transmission properties of the fabricated NS coverglasses with different glass nanostructures were systematically analyzed by considering the absorption spectrum of three subcells of the solar cell in order to find the most effective NS coverglass for enhancing the photocurrent of the PV module and thereby improving the conversion efficiency. The PV module with the most effective NS coverglass showed 4.2% enhanced short-circuit current density and 4.3% enhanced efficiency compared with the PV module with a bare coverglass measured under one sun of the air mass 1.5 global, showing the necessity of integrating a finely designed NS coverglass to effectively improve the efficiency of various PV systems with multijunction solar cells.]]></abstract>
##     <issn><![CDATA[1943-0655]]></issn>
##     <htmlFlag><![CDATA[1]]></htmlFlag>
##     <arnumber><![CDATA[6803882]]></arnumber>
##     <doi><![CDATA[10.1109/JPHOT.2014.2319100]]></doi>
##     <publicationId><![CDATA[6803882]]></publicationId>
##     <mdurl><![CDATA[http://ieeexplore.ieee.org/xpl/articleDetails.jsp?tp=&arnumber=6803882&contentType=Journals+%26+Magazines]]></mdurl>
##     <pdf><![CDATA[http://ieeexplore.ieee.org/stamp/stamp.jsp?arnumber=6803882]]></pdf>
##   </document>
##   <document>
##     <rank>1128</rank>
##     <title><![CDATA[A PUF Based on a Transient Effect Ring Oscillator and Insensitive to Locking Phenomenon]]></title>
##     <authors><![CDATA[Bossuet, L.;  Xuan Thuy Ngo;  Cherif, Z.;  Fischer, V.]]></authors>
##     <controlledterms>
##       <term><![CDATA[elemental semiconductors]]></term>
##       <term><![CDATA[field programmable gate arrays]]></term>
##       <term><![CDATA[oscillators]]></term>
##       <term><![CDATA[silicon]]></term>
##     </controlledterms>
##     <thesaurusterms>
##       <term><![CDATA[Computer architecture]]></term>
##       <term><![CDATA[Field programmable gate arrays]]></term>
##       <term><![CDATA[Hardware]]></term>
##       <term><![CDATA[Oscillators]]></term>
##       <term><![CDATA[Radiation detectors]]></term>
##       <term><![CDATA[Ring oscillators]]></term>
##     </thesaurusterms>
##     <pubtitle><![CDATA[Emerging Topics in Computing, IEEE Transactions on]]></pubtitle>
##     <punumber><![CDATA[6245516]]></punumber>
##     <pubtype><![CDATA[Journals & Magazines]]></pubtype>
##     <publisher><![CDATA[IEEE]]></publisher>
##     <volume><![CDATA[2]]></volume>
##     <issue><![CDATA[1]]></issue>
##     <py><![CDATA[2014]]></py>
##     <spage><![CDATA[30]]></spage>
##     <epage><![CDATA[36]]></epage>
##     <abstract><![CDATA[This paper presents a new silicon physical unclonable function (PUF) based on a transient effect ring oscillator (TERO). The proposed PUF has state of the art PUF characteristics with a good ratio of PUF response variability to response length. Unlike RO-PUF, it is not sensitive to the locking phenomenon, which challenges the use of ring oscillators for the design of both PUF and TRNG. The novel architecture using differential structures guarantees high stability of the TERO-PUF. The area of the TERO-PUF is relatively high, but is still comparable with other PUF designs. However, since the same piece of hardware can be used for both PUF and random number generation, the proposed principle offers an interesting low area mixed solution.]]></abstract>
##     <issn><![CDATA[2168-6750]]></issn>
##     <htmlFlag><![CDATA[1]]></htmlFlag>
##     <arnumber><![CDATA[6648644]]></arnumber>
##     <doi><![CDATA[10.1109/TETC.2013.2287182]]></doi>
##     <publicationId><![CDATA[6648644]]></publicationId>
##     <mdurl><![CDATA[http://ieeexplore.ieee.org/xpl/articleDetails.jsp?tp=&arnumber=6648644&contentType=Journals+%26+Magazines]]></mdurl>
##     <pdf><![CDATA[http://ieeexplore.ieee.org/stamp/stamp.jsp?arnumber=6648644]]></pdf>
##   </document>
##   <document>
##     <rank>1129</rank>
##     <title><![CDATA[Utilization of LDPC Code and Optical Hard-Limiter in OCDMA Communication Systems]]></title>
##     <authors><![CDATA[Maw-Yang Liu;  Yi-Kai Hsu;  Joe-Air Jiang]]></authors>
##     <affiliations><![CDATA[Dept. of Electr. Eng., Nat. Ilan Univ., Ilan, Taiwan]]></affiliations>
##     <controlledterms>
##       <term><![CDATA[channel capacity]]></term>
##       <term><![CDATA[code division multiple access]]></term>
##       <term><![CDATA[code division multiplexing]]></term>
##       <term><![CDATA[error correction codes]]></term>
##       <term><![CDATA[interference suppression]]></term>
##       <term><![CDATA[optical fibre networks]]></term>
##       <term><![CDATA[optical limiters]]></term>
##       <term><![CDATA[parity check codes]]></term>
##     </controlledterms>
##     <thesaurusterms>
##       <term><![CDATA[Adaptive optics]]></term>
##       <term><![CDATA[High-speed optical techniques]]></term>
##       <term><![CDATA[Optical fiber communication]]></term>
##       <term><![CDATA[Optical fibers]]></term>
##       <term><![CDATA[Optical pulses]]></term>
##       <term><![CDATA[Parity check codes]]></term>
##     </thesaurusterms>
##     <pubtitle><![CDATA[Photonics Journal, IEEE]]></pubtitle>
##     <punumber><![CDATA[4563994]]></punumber>
##     <pubtype><![CDATA[Journals & Magazines]]></pubtype>
##     <publisher><![CDATA[IEEE]]></publisher>
##     <volume><![CDATA[6]]></volume>
##     <issue><![CDATA[5]]></issue>
##     <py><![CDATA[2014]]></py>
##     <spage><![CDATA[1]]></spage>
##     <epage><![CDATA[11]]></epage>
##     <abstract><![CDATA[In this paper, we investigate the utilization of low-density parity-check (LDPC) code and optical hard-limiter in OCDMA communication systems. Since the multiple access interference will impose significant penalty on system performance, using optical hard-limiter is an effective countermeasure to relieve this adverse impact. To further increase the aggregate capacity, the LDPC code is employed as an error correction mechanism that can allow more simultaneous users accommodated in the network. As the aggregate capacity and system complexity are the main considerations in practical system design, both of which are mutually trade off. Utilization of LDPC code and optical hard-limiter can effectively reduce the system complexity to reach the maximum aggregate capacity. Our proposed scheme can accommodate large number of users in the network simultaneously and support the traffic asynchronously with each user. Furthermore, the aggregate capacity can reach above 100 Gbps; therefore, it is the preferable way as the platform for fiber distribution network in some applications.]]></abstract>
##     <issn><![CDATA[1943-0655]]></issn>
##     <htmlFlag><![CDATA[1]]></htmlFlag>
##     <arnumber><![CDATA[6842627]]></arnumber>
##     <doi><![CDATA[10.1109/JPHOT.2014.2331256]]></doi>
##     <publicationId><![CDATA[6842627]]></publicationId>
##     <mdurl><![CDATA[http://ieeexplore.ieee.org/xpl/articleDetails.jsp?tp=&arnumber=6842627&contentType=Journals+%26+Magazines]]></mdurl>
##     <pdf><![CDATA[http://ieeexplore.ieee.org/stamp/stamp.jsp?arnumber=6842627]]></pdf>
##   </document>
##   <document>
##     <rank>1130</rank>
##     <title><![CDATA[Stochastic Analysis of Wideband Near-Field Emissions From Dipole Antennas and Integrated Circuits]]></title>
##     <authors><![CDATA[Arnaut, L.R.;  Obiekezie, C.S.]]></authors>
##     <affiliations><![CDATA[George Green Inst. of Electromagn. Res., Univ. of Nottingham, Nottingham, UK]]></affiliations>
##     <controlledterms>
##       <term><![CDATA[ULSI]]></term>
##       <term><![CDATA[dipole antenna arrays]]></term>
##       <term><![CDATA[distortion]]></term>
##       <term><![CDATA[electromagnetic field theory]]></term>
##       <term><![CDATA[planar antenna arrays]]></term>
##       <term><![CDATA[statistical analysis]]></term>
##       <term><![CDATA[stochastic processes]]></term>
##     </controlledterms>
##     <thesaurusterms>
##       <term><![CDATA[Correlation]]></term>
##       <term><![CDATA[Distortion measurement]]></term>
##       <term><![CDATA[Integrated circuit modeling]]></term>
##       <term><![CDATA[Interference]]></term>
##       <term><![CDATA[Noise measurement]]></term>
##       <term><![CDATA[Stochastic processes]]></term>
##     </thesaurusterms>
##     <pubtitle><![CDATA[Electromagnetic Compatibility, IEEE Transactions on]]></pubtitle>
##     <punumber><![CDATA[15]]></punumber>
##     <pubtype><![CDATA[Journals & Magazines]]></pubtype>
##     <publisher><![CDATA[IEEE]]></publisher>
##     <volume><![CDATA[56]]></volume>
##     <issue><![CDATA[1]]></issue>
##     <py><![CDATA[2014]]></py>
##     <spage><![CDATA[93]]></spage>
##     <epage><![CDATA[101]]></epage>
##     <abstract><![CDATA[A statistical method is developed for characterizing wideband emissions from planar antennas and circuits containing multiple radiating elements. It is shown how the space-time and space-frequency correlation functions for source currents can be deduced from near-field measurements. These correlations produce additional spectral distortion of the emitted near field, over and above the far-field spectra of individual elements and their combined spectrum for uncorrelated sources. For an arbitrary configuration and number of emitting dipoles, the contribution to this distortion by pairwise correlations is fully calculable from the second-order covariance theory. Simulation results for a 2 &#x00D7;2 dipole array and near-field measurements on an L-shaped microstrip antenna with wide-band excitation demonstrate and validate the feasibility of the method and the theoretically predicted spectral distortion based on pairwise correlations only.]]></abstract>
##     <issn><![CDATA[0018-9375]]></issn>
##     <htmlFlag><![CDATA[1]]></htmlFlag>
##     <arnumber><![CDATA[6575113]]></arnumber>
##     <doi><![CDATA[10.1109/TEMC.2013.2273737]]></doi>
##     <publicationId><![CDATA[6575113]]></publicationId>
##     <mdurl><![CDATA[http://ieeexplore.ieee.org/xpl/articleDetails.jsp?tp=&arnumber=6575113&contentType=Journals+%26+Magazines]]></mdurl>
##     <pdf><![CDATA[http://ieeexplore.ieee.org/stamp/stamp.jsp?arnumber=6575113]]></pdf>
##   </document>
##   <document>
##     <rank>1131</rank>
##     <title><![CDATA[Signal-Processing Strategy for Restoration of Cross-Channel Suppression in Hearing-Impaired Listeners]]></title>
##     <authors><![CDATA[Rasetshwane, D.M.;  Gorga, M.P.;  Neely, S.T.]]></authors>
##     <affiliations><![CDATA[Boys Town Nat. Res. Hosp., Omaha, NE, USA]]></affiliations>
##     <controlledterms>
##       <term><![CDATA[ear]]></term>
##       <term><![CDATA[hearing]]></term>
##       <term><![CDATA[medical disorders]]></term>
##       <term><![CDATA[medical signal processing]]></term>
##       <term><![CDATA[otoacoustic emissions]]></term>
##     </controlledterms>
##     <thesaurusterms>
##       <term><![CDATA[Auditory system]]></term>
##       <term><![CDATA[Delays]]></term>
##       <term><![CDATA[Filter banks]]></term>
##       <term><![CDATA[Frequency measurement]]></term>
##       <term><![CDATA[Gain measurement]]></term>
##       <term><![CDATA[IIR filters]]></term>
##       <term><![CDATA[Speech]]></term>
##     </thesaurusterms>
##     <pubtitle><![CDATA[Biomedical Engineering, IEEE Transactions on]]></pubtitle>
##     <punumber><![CDATA[10]]></punumber>
##     <pubtype><![CDATA[Journals & Magazines]]></pubtype>
##     <publisher><![CDATA[IEEE]]></publisher>
##     <volume><![CDATA[61]]></volume>
##     <issue><![CDATA[1]]></issue>
##     <py><![CDATA[2014]]></py>
##     <spage><![CDATA[64]]></spage>
##     <epage><![CDATA[75]]></epage>
##     <abstract><![CDATA[Because frequency components interact nonlinearly with each other inside the cochlea, the loudness growth of tones is relatively simple in comparison to the loudness growth of complex sounds. The term suppression refers to a reduction in the response growth of one tone in the presence of a second tone. Suppression is a salient feature of normal cochlear processing and contributes to psychophysical masking. Suppression is evident in many measurements of cochlear function in subjects with normal hearing, including distortion-product otoacoustic emissions (DPOAEs). Suppression is also evident, to a lesser extent, in subjects with mild-to-moderate hearing loss. This paper describes a hearing-aid signal-processing strategy that aims to restore both loudness growth and two-tone suppression in hearing-impaired listeners. The prescription of gain for this strategy is based on measurements of loudness by a method known as categorical loudness scaling. The proposed signal-processing strategy reproduces measured DPOAE suppression tuning curves and generalizes to any number of frequency components. The restoration of both normal suppression and normal loudness has the potential to improve hearing-aid performance and user satisfaction.]]></abstract>
##     <issn><![CDATA[0018-9294]]></issn>
##     <htmlFlag><![CDATA[1]]></htmlFlag>
##     <arnumber><![CDATA[6574223]]></arnumber>
##     <doi><![CDATA[10.1109/TBME.2013.2276351]]></doi>
##     <publicationId><![CDATA[6574223]]></publicationId>
##     <mdurl><![CDATA[http://ieeexplore.ieee.org/xpl/articleDetails.jsp?tp=&arnumber=6574223&contentType=Journals+%26+Magazines]]></mdurl>
##     <pdf><![CDATA[http://ieeexplore.ieee.org/stamp/stamp.jsp?arnumber=6574223]]></pdf>
##   </document>
##   <document>
##     <rank>1132</rank>
##     <title><![CDATA[Probability of Severe Adverse Events as a Function of Hospital Occupancy]]></title>
##     <authors><![CDATA[Boyle, J.;  Zeitz, K.;  Hoffman, R.;  Khanna, S.;  Beltrame, J.]]></authors>
##     <affiliations><![CDATA[ICT Centre, R. Brisbane & Women's Hosp., CSIRO, Herston, QLD, Australia]]></affiliations>
##     <controlledterms>
##       <term><![CDATA[hospitals]]></term>
##       <term><![CDATA[medical information systems]]></term>
##       <term><![CDATA[probability]]></term>
##       <term><![CDATA[regression analysis]]></term>
##     </controlledterms>
##     <pubtitle><![CDATA[Biomedical and Health Informatics, IEEE Journal of]]></pubtitle>
##     <punumber><![CDATA[6221020]]></punumber>
##     <pubtype><![CDATA[Journals & Magazines]]></pubtype>
##     <publisher><![CDATA[IEEE]]></publisher>
##     <volume><![CDATA[18]]></volume>
##     <issue><![CDATA[1]]></issue>
##     <py><![CDATA[2014]]></py>
##     <spage><![CDATA[15]]></spage>
##     <epage><![CDATA[20]]></epage>
##     <abstract><![CDATA[A unique application of regression modeling is described to compare hospital bed occupancy with reported severe adverse events amongst inpatients. The probabilities of the occurrence of adverse events as a function of hospital occupancy are calculated using logistic and multinomial regression models. All models indicate that higher occupancy rates lead to an increase in adverse events. The analysis identified that at an occupancy level of 100%, there is a 22% chance of one severe event occurring and a 28% chance of at least one severe event occurring. This modeling contributes evidence toward the management of hospital occupancy to benefit patient outcomes.]]></abstract>
##     <issn><![CDATA[2168-2194]]></issn>
##     <htmlFlag><![CDATA[1]]></htmlFlag>
##     <arnumber><![CDATA[6514573]]></arnumber>
##     <doi><![CDATA[10.1109/JBHI.2013.2262053]]></doi>
##     <publicationId><![CDATA[6514573]]></publicationId>
##     <mdurl><![CDATA[http://ieeexplore.ieee.org/xpl/articleDetails.jsp?tp=&arnumber=6514573&contentType=Journals+%26+Magazines]]></mdurl>
##     <pdf><![CDATA[http://ieeexplore.ieee.org/stamp/stamp.jsp?arnumber=6514573]]></pdf>
##   </document>
##   <document>
##     <rank>1133</rank>
##     <title><![CDATA[Medium Access with Adaptive Relay Selection in Cooperative Wireless Networks]]></title>
##     <authors><![CDATA[Adam, H.;  Yanmaz, E.;  Bettstetter, C.]]></authors>
##     <affiliations><![CDATA[Inst. of Networked & Embedded Syst., Univ. of Klagenfurt, Klagenfurt, Austria]]></affiliations>
##     <controlledterms>
##       <term><![CDATA[access protocols]]></term>
##       <term><![CDATA[cooperative communication]]></term>
##       <term><![CDATA[relay networks (telecommunication)]]></term>
##       <term><![CDATA[resource allocation]]></term>
##       <term><![CDATA[telecommunication network reliability]]></term>
##     </controlledterms>
##     <thesaurusterms>
##       <term><![CDATA[Data communication]]></term>
##       <term><![CDATA[Estimation]]></term>
##       <term><![CDATA[Media Access Protocol]]></term>
##       <term><![CDATA[Multiaccess communication]]></term>
##       <term><![CDATA[Relays]]></term>
##       <term><![CDATA[Throughput]]></term>
##     </thesaurusterms>
##     <pubtitle><![CDATA[Mobile Computing, IEEE Transactions on]]></pubtitle>
##     <punumber><![CDATA[7755]]></punumber>
##     <pubtype><![CDATA[Journals & Magazines]]></pubtype>
##     <publisher><![CDATA[IEEE]]></publisher>
##     <volume><![CDATA[13]]></volume>
##     <issue><![CDATA[9]]></issue>
##     <py><![CDATA[2014]]></py>
##     <spage><![CDATA[2042]]></spage>
##     <epage><![CDATA[2057]]></epage>
##     <abstract><![CDATA[We specify and evaluate a protocol for cooperative relay communications in wireless networks targeted for low-budget and energy-constrained off-the-shelf hardware. The protocol located at the Medium Access Control (MAC) layer integrates radio resource reservation, relay selection, and packet flow. Performance is evaluated with different parameters, such as node density, channel coherence time, and data packet size. Higher network-wide reliability and throughput compared to noncooperative protocols can be achieved in dense networks and unreliable channels. At the same time, throughput does not degrade in sparse networks or good channel conditions.]]></abstract>
##     <issn><![CDATA[1536-1233]]></issn>
##     <htmlFlag><![CDATA[1]]></htmlFlag>
##     <arnumber><![CDATA[6570480]]></arnumber>
##     <doi><![CDATA[10.1109/TMC.2013.97]]></doi>
##     <publicationId><![CDATA[6570480]]></publicationId>
##     <mdurl><![CDATA[http://ieeexplore.ieee.org/xpl/articleDetails.jsp?tp=&arnumber=6570480&contentType=Journals+%26+Magazines]]></mdurl>
##     <pdf><![CDATA[http://ieeexplore.ieee.org/stamp/stamp.jsp?arnumber=6570480]]></pdf>
##   </document>
##   <document>
##     <rank>1134</rank>
##     <title><![CDATA[Controlled Light&#x2013;Matter Interaction in Graphene Electrooptic Devices Using Nanophotonic Cavities and Waveguides]]></title>
##     <authors><![CDATA[Xuetao Gan;  Shiue, R.-J.;  Yuanda Gao;  Assefa, S.;  Hone, J.;  Englund, D.]]></authors>
##     <affiliations><![CDATA[Sch. of Sci., Northwestern Polytech. Univ., Xi'an, China]]></affiliations>
##     <controlledterms>
##       <term><![CDATA[III-V semiconductors]]></term>
##       <term><![CDATA[Raman spectra]]></term>
##       <term><![CDATA[boron compounds]]></term>
##       <term><![CDATA[capacitors]]></term>
##       <term><![CDATA[electro-optical modulation]]></term>
##       <term><![CDATA[graphene]]></term>
##       <term><![CDATA[infrared spectra]]></term>
##       <term><![CDATA[integrated optics]]></term>
##       <term><![CDATA[integrated optoelectronics]]></term>
##       <term><![CDATA[nanophotonics]]></term>
##       <term><![CDATA[optical couplers]]></term>
##       <term><![CDATA[optical fabrication]]></term>
##       <term><![CDATA[optical waveguides]]></term>
##       <term><![CDATA[photoluminescence]]></term>
##       <term><![CDATA[photonic crystals]]></term>
##       <term><![CDATA[visible spectra]]></term>
##     </controlledterms>
##     <thesaurusterms>
##       <term><![CDATA[Absorption]]></term>
##       <term><![CDATA[Cavity resonators]]></term>
##       <term><![CDATA[Couplings]]></term>
##       <term><![CDATA[Graphene]]></term>
##       <term><![CDATA[Modulation]]></term>
##       <term><![CDATA[Optical waveguides]]></term>
##       <term><![CDATA[Raman scattering]]></term>
##     </thesaurusterms>
##     <pubtitle><![CDATA[Selected Topics in Quantum Electronics, IEEE Journal of]]></pubtitle>
##     <punumber><![CDATA[2944]]></punumber>
##     <pubtype><![CDATA[Journals & Magazines]]></pubtype>
##     <publisher><![CDATA[IEEE]]></publisher>
##     <volume><![CDATA[20]]></volume>
##     <issue><![CDATA[1]]></issue>
##     <py><![CDATA[2014]]></py>
##     <spage><![CDATA[95]]></spage>
##     <epage><![CDATA[105]]></epage>
##     <abstract><![CDATA[Nanophotonic devices, such as waveguides and cavities, can strongly enhance the interaction of light with graphene. We describe techniques for enhancing the interaction of photons with graphene using chip-integrated nanophotonic devices. Transferring single-layer graphene onto planar photonic crystal nanocavities enables a spectrally selective, order-of-magnitude enhancement of optical coupling with graphene, as shown by spectroscopic studies of cavity modes in visible and infrared spectral ranges. We observed dramatically cavity-enhanced absorption, hot photoluminescence emission, and Raman scattering of the monolayer graphene. We also described a broad-spectrum enhancement of the light-matter interaction by coupling graphene with a bus waveguide on a silicon-on-insulator photonic integrated circuit, which enables a 6.2-dB transmission attenuation due to the graphene absorption over a waveguide length of 70 &#x03BC;m. By electrically gating the graphene monolayer coupled with a planar photonic crystal nanocavity, electrooptic modulation of the cavity reflection was possible with a contrast in excess of 10 dB. Moreover, a novel modulator device based on the cavity-coupled graphene-boron nitride-graphene capacitor was fabricated, showing a modulation speed up to 0.57 GHz. These results indicate the applications of graphene-cavity devices in high-speed and high-contrast modulators with low energy consumption. The integration of graphene with nanophotonic architectures promises a new generation of compact, energy-efficient, and ultrafast electrooptic graphene devices for on-chip optical communications.]]></abstract>
##     <issn><![CDATA[1077-260X]]></issn>
##     <htmlFlag><![CDATA[1]]></htmlFlag>
##     <arnumber><![CDATA[6576195]]></arnumber>
##     <doi><![CDATA[10.1109/JSTQE.2013.2273412]]></doi>
##     <publicationId><![CDATA[6576195]]></publicationId>
##     <mdurl><![CDATA[http://ieeexplore.ieee.org/xpl/articleDetails.jsp?tp=&arnumber=6576195&contentType=Journals+%26+Magazines]]></mdurl>
##     <pdf><![CDATA[http://ieeexplore.ieee.org/stamp/stamp.jsp?arnumber=6576195]]></pdf>
##   </document>
##   <document>
##     <rank>1135</rank>
##     <title><![CDATA[Algorithms for Smartphone and Tablet Image Analysis for Healthcare Applications]]></title>
##     <authors><![CDATA[White, P.J.F.;  Podaima, B.W.;  Friesen, M.R.]]></authors>
##     <affiliations><![CDATA[Dept. of Electr. & Comput. Eng., Univ. of Manitoba, Winnipeg, MB, Canada]]></affiliations>
##     <controlledterms>
##       <term><![CDATA[Android (operating system)]]></term>
##       <term><![CDATA[image sensors]]></term>
##       <term><![CDATA[medical image processing]]></term>
##       <term><![CDATA[mobile computing]]></term>
##       <term><![CDATA[notebook computers]]></term>
##       <term><![CDATA[object tracking]]></term>
##       <term><![CDATA[patient diagnosis]]></term>
##       <term><![CDATA[sensor fusion]]></term>
##       <term><![CDATA[smart phones]]></term>
##     </controlledterms>
##     <thesaurusterms>
##       <term><![CDATA[Algorithm design and analysis]]></term>
##       <term><![CDATA[Biomedical image processing]]></term>
##       <term><![CDATA[Electronic medical records]]></term>
##       <term><![CDATA[Image analysis]]></term>
##       <term><![CDATA[Image sensors]]></term>
##       <term><![CDATA[Mobile communication]]></term>
##       <term><![CDATA[Smart phones]]></term>
##     </thesaurusterms>
##     <pubtitle><![CDATA[Access, IEEE]]></pubtitle>
##     <punumber><![CDATA[6287639]]></punumber>
##     <pubtype><![CDATA[Journals & Magazines]]></pubtype>
##     <publisher><![CDATA[IEEE]]></publisher>
##     <volume><![CDATA[2]]></volume>
##     <py><![CDATA[2014]]></py>
##     <spage><![CDATA[831]]></spage>
##     <epage><![CDATA[840]]></epage>
##     <abstract><![CDATA[Smartphones and tablets are finding their way into healthcare delivery to the extent that mobile health (mHealth) has become an identifiable field within eHealth. In prior work, a mobile app to document chronic wounds and wound care, specifically pressure ulcers (bedsores) was developed for Android smartphones and tablets. One feature of the mobile app allowed users to take images of the wound using the smartphone or tablet's integrated camera. In a user trial with nurses at a personal care home, this feature emerged as a key benefit of the mobile app. This paper developed image analysis algorithms that facilitate noncontact measurements of irregularly shaped images (e.g., wounds), where the image is taken with a sole smartphone or tablet camera. The image analysis relies on the sensors integrated in the smartphone or tablet with no auxiliary or add-on instrumentation on the device. Three approaches to image analysis were developed and evaluated: 1) computing depth using autofocus data; 2) a custom sensor fusion of inertial sensors and feature tracking in a video stream; and 3) a custom pinch/zoom approach. The pinch/zoom approach demonstrated the strongest potential and thus developed into a fully functional prototype complete with a measurement mechanism. While image analysis is a very well developed field, this paper contributes to image analysis applications and implementation in mHealth, specifically for wound care.]]></abstract>
##     <issn><![CDATA[2169-3536]]></issn>
##     <htmlFlag><![CDATA[1]]></htmlFlag>
##     <arnumber><![CDATA[6879475]]></arnumber>
##     <doi><![CDATA[10.1109/ACCESS.2014.2348943]]></doi>
##     <publicationId><![CDATA[6879475]]></publicationId>
##     <mdurl><![CDATA[http://ieeexplore.ieee.org/xpl/articleDetails.jsp?tp=&arnumber=6879475&contentType=Journals+%26+Magazines]]></mdurl>
##     <pdf><![CDATA[http://ieeexplore.ieee.org/stamp/stamp.jsp?arnumber=6879475]]></pdf>
##   </document>
##   <document>
##     <rank>1136</rank>
##     <title><![CDATA[Information Security in Big Data: Privacy and Data Mining]]></title>
##     <authors><![CDATA[Lei Xu;  Chunxiao Jiang;  Jian Wang;  Jian Yuan;  Yong Ren]]></authors>
##     <affiliations><![CDATA[Dept. of Electron. Eng., Tsinghua Univ., Beijing, China]]></affiliations>
##     <controlledterms>
##       <term><![CDATA[Big Data]]></term>
##       <term><![CDATA[data acquisition]]></term>
##       <term><![CDATA[data mining]]></term>
##       <term><![CDATA[data protection]]></term>
##       <term><![CDATA[game theory]]></term>
##       <term><![CDATA[security of data]]></term>
##     </controlledterms>
##     <thesaurusterms>
##       <term><![CDATA[Algorithm design and analysis]]></term>
##       <term><![CDATA[Computer security]]></term>
##       <term><![CDATA[Data mining]]></term>
##       <term><![CDATA[Data privacy]]></term>
##       <term><![CDATA[Game theory]]></term>
##       <term><![CDATA[Privacy]]></term>
##       <term><![CDATA[Tracking]]></term>
##     </thesaurusterms>
##     <pubtitle><![CDATA[Access, IEEE]]></pubtitle>
##     <punumber><![CDATA[6287639]]></punumber>
##     <pubtype><![CDATA[Journals & Magazines]]></pubtype>
##     <publisher><![CDATA[IEEE]]></publisher>
##     <volume><![CDATA[2]]></volume>
##     <py><![CDATA[2014]]></py>
##     <spage><![CDATA[1149]]></spage>
##     <epage><![CDATA[1176]]></epage>
##     <abstract><![CDATA[The growing popularity and development of data mining technologies bring serious threat to the security of individual,'s sensitive information. An emerging research topic in data mining, known as privacy-preserving data mining (PPDM), has been extensively studied in recent years. The basic idea of PPDM is to modify the data in such a way so as to perform data mining algorithms effectively without compromising the security of sensitive information contained in the data. Current studies of PPDM mainly focus on how to reduce the privacy risk brought by data mining operations, while in fact, unwanted disclosure of sensitive information may also happen in the process of data collecting, data publishing, and information (i.e., the data mining results) delivering. In this paper, we view the privacy issues related to data mining from a wider perspective and investigate various approaches that can help to protect sensitive information. In particular, we identify four different types of users involved in data mining applications, namely, data provider, data collector, data miner, and decision maker. For each type of user, we discuss his privacy concerns and the methods that can be adopted to protect sensitive information. We briefly introduce the basics of related research topics, review state-of-the-art approaches, and present some preliminary thoughts on future research directions. Besides exploring the privacy-preserving approaches for each type of user, we also review the game theoretical approaches, which are proposed for analyzing the interactions among different users in a data mining scenario, each of whom has his own valuation on the sensitive information. By differentiating the responsibilities of different users with respect to security of sensitive information, we would like to provide some useful insights into the study of PPDM.]]></abstract>
##     <issn><![CDATA[2169-3536]]></issn>
##     <htmlFlag><![CDATA[1]]></htmlFlag>
##     <arnumber><![CDATA[6919256]]></arnumber>
##     <doi><![CDATA[10.1109/ACCESS.2014.2362522]]></doi>
##     <publicationId><![CDATA[6919256]]></publicationId>
##     <mdurl><![CDATA[http://ieeexplore.ieee.org/xpl/articleDetails.jsp?tp=&arnumber=6919256&contentType=Journals+%26+Magazines]]></mdurl>
##     <pdf><![CDATA[http://ieeexplore.ieee.org/stamp/stamp.jsp?arnumber=6919256]]></pdf>
##   </document>
##   <document>
##     <rank>1137</rank>
##     <title><![CDATA[EReLA: A Low-Power Reliable Coarse-Grained Reconfigurable Architecture Processor and Its Irradiation Tests]]></title>
##     <authors><![CDATA[Yao, J.;  Saito, M.;  Okada, S.;  Kobayashi, K.;  Nakashima, Y.]]></authors>
##     <affiliations><![CDATA[Grad. Sch. of Inf. Sci., Nara Inst. of Sci. & Technol., Ikoma, Japan]]></affiliations>
##     <controlledterms>
##       <term><![CDATA[CMOS integrated circuits]]></term>
##       <term><![CDATA[low-power electronics]]></term>
##       <term><![CDATA[microprocessor chips]]></term>
##       <term><![CDATA[radiation hardening (electronics)]]></term>
##       <term><![CDATA[reconfigurable architectures]]></term>
##       <term><![CDATA[reliability]]></term>
##     </controlledterms>
##     <thesaurusterms>
##       <term><![CDATA[Data flow computing]]></term>
##       <term><![CDATA[Fault tolerance]]></term>
##       <term><![CDATA[Radiation effects]]></term>
##       <term><![CDATA[Reconfigurable architectures]]></term>
##       <term><![CDATA[Redundancy]]></term>
##       <term><![CDATA[VLIW]]></term>
##     </thesaurusterms>
##     <pubtitle><![CDATA[Nuclear Science, IEEE Transactions on]]></pubtitle>
##     <punumber><![CDATA[23]]></punumber>
##     <pubtype><![CDATA[Journals & Magazines]]></pubtype>
##     <publisher><![CDATA[IEEE]]></publisher>
##     <volume><![CDATA[61]]></volume>
##     <issue><![CDATA[6]]></issue>
##     <part><![CDATA[1]]></part>
##     <py><![CDATA[2014]]></py>
##     <spage><![CDATA[3250]]></spage>
##     <epage><![CDATA[3257]]></epage>
##     <abstract><![CDATA[In this work, facing pressure from both the increasing vulnerability to single event effects (SEEs) and design constraints of the power consumption, we have proposed a Coarse-Grained Reconfigurable Architecture (CGRA) processor. Our goal is to translate a user programmable redundancy to a guide for balancing energy consumption on the one hand and the reliability requirements on the other. We designed software (SW) and hardware (HW) approaches, coordinating them closely to achieve this purpose. The framework provides several user-assignable patterns of redundancy and the hardware modules to interpret well these patterns. A first version prototype processor, with the name EReLA (Explicit Redundancy Linear Array) has been implemented and manufactured with a 0.18 &#x03BC;m CMOS technology. Stress tests based on alpha particle irradiation were conducted to verify the tradeoff between the robustness and the power efficiency of the proposed schemes.]]></abstract>
##     <issn><![CDATA[0018-9499]]></issn>
##     <htmlFlag><![CDATA[1]]></htmlFlag>
##     <arnumber><![CDATA[6966813]]></arnumber>
##     <doi><![CDATA[10.1109/TNS.2014.2367541]]></doi>
##     <publicationId><![CDATA[6966813]]></publicationId>
##     <mdurl><![CDATA[http://ieeexplore.ieee.org/xpl/articleDetails.jsp?tp=&arnumber=6966813&contentType=Journals+%26+Magazines]]></mdurl>
##     <pdf><![CDATA[http://ieeexplore.ieee.org/stamp/stamp.jsp?arnumber=6966813]]></pdf>
##   </document>
##   <document>
##     <rank>1138</rank>
##     <title><![CDATA[An Informative Interpretation of Decision Theory: Scalar Performance Measures for Binary Decisions]]></title>
##     <authors><![CDATA[Polcari, J.]]></authors>
##     <affiliations><![CDATA[Center for Eng. Sci. Adv. Res., Oak Ridge Nat. Lab., Oak Ridge, TN, USA]]></affiliations>
##     <controlledterms>
##       <term><![CDATA[decision theory]]></term>
##     </controlledterms>
##     <thesaurusterms>
##       <term><![CDATA[Bayes methods]]></term>
##       <term><![CDATA[Data models]]></term>
##       <term><![CDATA[Decision theory]]></term>
##       <term><![CDATA[Signal to noise ratio]]></term>
##       <term><![CDATA[Statistical analysis]]></term>
##     </thesaurusterms>
##     <pubtitle><![CDATA[Access, IEEE]]></pubtitle>
##     <punumber><![CDATA[6287639]]></punumber>
##     <pubtype><![CDATA[Journals & Magazines]]></pubtype>
##     <publisher><![CDATA[IEEE]]></publisher>
##     <volume><![CDATA[2]]></volume>
##     <py><![CDATA[2014]]></py>
##     <spage><![CDATA[1456]]></spage>
##     <epage><![CDATA[1480]]></epage>
##     <abstract><![CDATA[A previous formulation for the application of information accounting to binary decision theory is extended to permit the quality of the decision to be quantitatively measured by evaluation of the underlying informational support. Both a single exemplar measure of information, separability, and its ensemble average equivalent, separation, are shown to measure the information support for decision quality (i.e., how well-informed is the decision), rather than the information support for decision adjudication (i.e., which hypothesis is the better choice) provided by predecision information measures. When compared to the traditional receiver operating characteristic, these measures present several functional advantages. They are scalar in nature, and may be directly optimized over secondary parameters, as well as being rigorously well posed and universally comparable. They incorporate the effects of all relevant decision components (prior information, observational information, and decision rule) in a unified manner while still being easily related to the predecision information measures of log likelihood ratio and generalized signal-to-noise ratio. They can be applied equally well to individual trials or composite averages, and evaluation does not require knowledge of the underlying truth. Compared to false alarm-oriented methods for assessing decision performance, their construction reduces sensitivity to tail effects in the underlying distributions.]]></abstract>
##     <issn><![CDATA[2169-3536]]></issn>
##     <htmlFlag><![CDATA[1]]></htmlFlag>
##     <arnumber><![CDATA[6975031]]></arnumber>
##     <doi><![CDATA[10.1109/ACCESS.2014.2377593]]></doi>
##     <publicationId><![CDATA[6975031]]></publicationId>
##     <mdurl><![CDATA[http://ieeexplore.ieee.org/xpl/articleDetails.jsp?tp=&arnumber=6975031&contentType=Journals+%26+Magazines]]></mdurl>
##     <pdf><![CDATA[http://ieeexplore.ieee.org/stamp/stamp.jsp?arnumber=6975031]]></pdf>
##   </document>
##   <document>
##     <rank>1139</rank>
##     <title><![CDATA[Determining Semantically Related Significant Genes]]></title>
##     <authors><![CDATA[Taha, K.]]></authors>
##     <affiliations><![CDATA[Dept. of Electr. & Comput. Eng., Khalifa Univ., Abu Dhabi, United Arab Emirates]]></affiliations>
##     <controlledterms>
##       <term><![CDATA[bioinformatics]]></term>
##       <term><![CDATA[encoding]]></term>
##       <term><![CDATA[genetics]]></term>
##       <term><![CDATA[search engines]]></term>
##     </controlledterms>
##     <thesaurusterms>
##       <term><![CDATA[Bioinformatics]]></term>
##       <term><![CDATA[Biological information theory]]></term>
##       <term><![CDATA[Biological system modeling]]></term>
##       <term><![CDATA[Muscles]]></term>
##       <term><![CDATA[Semantics]]></term>
##       <term><![CDATA[Statistics]]></term>
##     </thesaurusterms>
##     <pubtitle><![CDATA[Computational Biology and Bioinformatics, IEEE/ACM Transactions on]]></pubtitle>
##     <punumber><![CDATA[8857]]></punumber>
##     <pubtype><![CDATA[Journals & Magazines]]></pubtype>
##     <publisher><![CDATA[IEEE]]></publisher>
##     <volume><![CDATA[11]]></volume>
##     <issue><![CDATA[6]]></issue>
##     <py><![CDATA[2014]]></py>
##     <spage><![CDATA[1119]]></spage>
##     <epage><![CDATA[1130]]></epage>
##     <abstract><![CDATA[GO relation embodies some aspects of existence dependency. If GO term xis existence-dependent on GO term y, the presence of y implies the presence of x. Therefore, the genes annotated with the function of the GO term y are usually functionally and semantically related to the genes annotated with the function of the GO term x. A large number of gene set enrichment analysis methods have been developed in recent years for analyzing gene sets enrichment. However, most of these methods overlook the structural dependencies between GO terms in GO graph by not considering the concept of existence dependency. We propose in this paper a biological search engine called RSGSearch that identifies enriched sets of genes annotated with different functions using the concept of existence dependency. We observe that GO term xcannot be existence-dependent on GO term y, if x- and y- have the same specificity (biological characteristics). After encoding into a numeric format the contributions of GO terms annotating target genes to the semantics of their lowest common ancestors (LCAs), RSGSearch uses microarray experiment to identify the most significant LCA that annotates the result genes. We evaluated RSGSearch experimentally and compared it with five gene set enrichment systems. Results showed marked improvement.]]></abstract>
##     <issn><![CDATA[1545-5963]]></issn>
##     <htmlFlag><![CDATA[1]]></htmlFlag>
##     <arnumber><![CDATA[6868276]]></arnumber>
##     <doi><![CDATA[10.1109/TCBB.2014.2344668]]></doi>
##     <publicationId><![CDATA[6868276]]></publicationId>
##     <mdurl><![CDATA[http://ieeexplore.ieee.org/xpl/articleDetails.jsp?tp=&arnumber=6868276&contentType=Journals+%26+Magazines]]></mdurl>
##     <pdf><![CDATA[http://ieeexplore.ieee.org/stamp/stamp.jsp?arnumber=6868276]]></pdf>
##   </document>
##   <document>
##     <rank>1140</rank>
##     <title><![CDATA[Ultralow Propagation Loss Slot-Waveguide in High Absorption Active Material]]></title>
##     <authors><![CDATA[Yongzhuo Li;  Kaiyu Cui;  Xue Feng;  Yidong Huang;  Fang Liu;  Wei Zhang]]></authors>
##     <affiliations><![CDATA[Dept. of Electron. Eng., Tsinghua Univ., Beijing, China]]></affiliations>
##     <controlledterms>
##       <term><![CDATA[absorption coefficients]]></term>
##       <term><![CDATA[integrated optics]]></term>
##       <term><![CDATA[optical losses]]></term>
##       <term><![CDATA[optical waveguides]]></term>
##     </controlledterms>
##     <thesaurusterms>
##       <term><![CDATA[Absorption]]></term>
##       <term><![CDATA[Indium phosphide]]></term>
##       <term><![CDATA[Optical device fabrication]]></term>
##       <term><![CDATA[Optical waveguides]]></term>
##       <term><![CDATA[Propagation losses]]></term>
##       <term><![CDATA[Silicon]]></term>
##     </thesaurusterms>
##     <pubtitle><![CDATA[Photonics Journal, IEEE]]></pubtitle>
##     <punumber><![CDATA[4563994]]></punumber>
##     <pubtype><![CDATA[Journals & Magazines]]></pubtype>
##     <publisher><![CDATA[IEEE]]></publisher>
##     <volume><![CDATA[6]]></volume>
##     <issue><![CDATA[3]]></issue>
##     <py><![CDATA[2014]]></py>
##     <spage><![CDATA[1]]></spage>
##     <epage><![CDATA[6]]></epage>
##     <abstract><![CDATA[A slot waveguide formed by high absorption active material is proposed to reduce the propagation loss for monolithic integration. The low propagation loss is attained by concentrating the optical field inside the low-index slot region without absorption. The simulation results show that the propagation loss at 1.55 &#x03BC;m for the slot waveguide can be as low as 1.5 dB/cm in active material with an absorption coefficient of 3000 cm<sup>-1</sup>, whereas the optical profile is set to be confined within 650 nm.]]></abstract>
##     <issn><![CDATA[1943-0655]]></issn>
##     <htmlFlag><![CDATA[1]]></htmlFlag>
##     <arnumber><![CDATA[6807516]]></arnumber>
##     <doi><![CDATA[10.1109/JPHOT.2014.2320746]]></doi>
##     <publicationId><![CDATA[6807516]]></publicationId>
##     <mdurl><![CDATA[http://ieeexplore.ieee.org/xpl/articleDetails.jsp?tp=&arnumber=6807516&contentType=Journals+%26+Magazines]]></mdurl>
##     <pdf><![CDATA[http://ieeexplore.ieee.org/stamp/stamp.jsp?arnumber=6807516]]></pdf>
##   </document>
##   <document>
##     <rank>1141</rank>
##     <title><![CDATA[Improving Argos Doppler Location Using Multiple-Model Kalman Filtering]]></title>
##     <authors><![CDATA[Lopez, R.;  Malarde, J.-P.;  Royer, F.;  Gaspar, P.]]></authors>
##     <affiliations><![CDATA[Collecte Localisation Satellites, Ramonville-Saint-Agne, France]]></affiliations>
##     <controlledterms>
##       <term><![CDATA[Doppler effect]]></term>
##       <term><![CDATA[Global Positioning System]]></term>
##       <term><![CDATA[Kalman filters]]></term>
##       <term><![CDATA[artificial satellites]]></term>
##       <term><![CDATA[data acquisition]]></term>
##       <term><![CDATA[least mean squares methods]]></term>
##       <term><![CDATA[measurement errors]]></term>
##       <term><![CDATA[nonlinear estimation]]></term>
##       <term><![CDATA[radio direction-finding]]></term>
##       <term><![CDATA[radio receivers]]></term>
##       <term><![CDATA[radio transmitters]]></term>
##       <term><![CDATA[satellite telemetry]]></term>
##     </controlledterms>
##     <thesaurusterms>
##       <term><![CDATA[Doppler effect]]></term>
##       <term><![CDATA[Frequency measurement]]></term>
##       <term><![CDATA[Kalman filters]]></term>
##       <term><![CDATA[Noise]]></term>
##       <term><![CDATA[Satellites]]></term>
##       <term><![CDATA[Standards]]></term>
##       <term><![CDATA[Vectors]]></term>
##     </thesaurusterms>
##     <pubtitle><![CDATA[Geoscience and Remote Sensing, IEEE Transactions on]]></pubtitle>
##     <punumber><![CDATA[36]]></punumber>
##     <pubtype><![CDATA[Journals & Magazines]]></pubtype>
##     <publisher><![CDATA[IEEE]]></publisher>
##     <volume><![CDATA[52]]></volume>
##     <issue><![CDATA[8]]></issue>
##     <py><![CDATA[2014]]></py>
##     <spage><![CDATA[4744]]></spage>
##     <epage><![CDATA[4755]]></epage>
##     <abstract><![CDATA[The Argos service was launched in 1978 to serve environmental applications, including oceanography, wildlife tracking, fishing vessel monitoring, and maritime safety. The system allows for worldwide near-real-time positioning and data collection of platform terminal transmitters (PTTs). The positioning of the PTTs is achieved by exploiting the Doppler shift in the carrier frequency of the transmitter as recorded by satelliteborne Argos receivers. Until March 15, 2011, a classical nonlinear least squares estimation technique was systematically used to estimate Argos positions. Since then, a second positioning algorithm using a multiple-model Kalman filter was implemented in the operational Argos positioning software. This paper presents this new algorithm and analyzes its performance using a large data set obtained from over 200 mobiles carrying both an Argos transmitter and a GPS receiver used as ground truth. The results show that the new algorithm significantly improves the positioning accuracy, particularly in difficult conditions (for class-A and class-B locations, in the Argos terminology). Moreover, the new algorithm enables the retrieval of a larger number of estimated positions and the systematic estimation of the location error.]]></abstract>
##     <issn><![CDATA[0196-2892]]></issn>
##     <htmlFlag><![CDATA[1]]></htmlFlag>
##     <arnumber><![CDATA[6648418]]></arnumber>
##     <doi><![CDATA[10.1109/TGRS.2013.2284293]]></doi>
##     <publicationId><![CDATA[6648418]]></publicationId>
##     <mdurl><![CDATA[http://ieeexplore.ieee.org/xpl/articleDetails.jsp?tp=&arnumber=6648418&contentType=Journals+%26+Magazines]]></mdurl>
##     <pdf><![CDATA[http://ieeexplore.ieee.org/stamp/stamp.jsp?arnumber=6648418]]></pdf>
##   </document>
##   <document>
##     <rank>1142</rank>
##     <title><![CDATA[Linear Physical-Layer Network Coding Over Hybrid Finite Ring for Rayleigh Fading Two-Way Relay Channels]]></title>
##     <authors><![CDATA[Dong Fang;  Burr, A.;  Jinhong Yuan]]></authors>
##     <affiliations><![CDATA[Dept. of Electron., Univ. of York, York, UK]]></affiliations>
##     <controlledterms>
##       <term><![CDATA[Rayleigh channels]]></term>
##       <term><![CDATA[access protocols]]></term>
##       <term><![CDATA[fading channels]]></term>
##       <term><![CDATA[network coding]]></term>
##       <term><![CDATA[quadrature amplitude modulation]]></term>
##       <term><![CDATA[relay networks (telecommunication)]]></term>
##       <term><![CDATA[source coding]]></term>
##     </controlledterms>
##     <thesaurusterms>
##       <term><![CDATA[Decoding]]></term>
##       <term><![CDATA[Network coding]]></term>
##       <term><![CDATA[Phase shift keying]]></term>
##       <term><![CDATA[Rayleigh channels]]></term>
##       <term><![CDATA[Relays]]></term>
##     </thesaurusterms>
##     <pubtitle><![CDATA[Communications, IEEE Transactions on]]></pubtitle>
##     <punumber><![CDATA[26]]></punumber>
##     <pubtype><![CDATA[Journals & Magazines]]></pubtype>
##     <publisher><![CDATA[IEEE]]></publisher>
##     <volume><![CDATA[62]]></volume>
##     <issue><![CDATA[9]]></issue>
##     <py><![CDATA[2014]]></py>
##     <spage><![CDATA[3249]]></spage>
##     <epage><![CDATA[3261]]></epage>
##     <abstract><![CDATA[In this paper, we propose a novel linear physical-layer network coding scheme over hybrid finite ring (HFR-LPNC) for Rayleigh fading two-way relay channels. The relay maps the superimposed signal of the two users to a linear network coded combination (LNCC) in hybrid finite ring, rather than using the simple bit-wise eXclusive-OR mapping. The optimal linear coefficients are selected to generate the LNCC, aiming to: 1) maximize the sum-rate in the MAC phase; and 2) ensure unambiguous decoding. To avoid the performance degradation caused by high-order irregular mappings, properly designed source coding is used for compressing the LNCC alphabet over the hybrid finite ring into the unifying 4-ary alphabet. We derive the constellation constrained sum-rates for HFR-LPNC in comparison with 5QAM denoise-and-forward (5QAM-DNF), which we use as a reference scheme. Furthermore, we explicitly characterize the rate difference between HFR-LPNC and 5QAM-DNF. Our analysis and simulation show that: 1) HFR-LPNC has a superior ability to mitigate the singular fading compared with 5QAM-DNF; and 2) HFR-LPNC is superior to 5QAM-DNF over a wide range of SNRs.]]></abstract>
##     <issn><![CDATA[0090-6778]]></issn>
##     <htmlFlag><![CDATA[1]]></htmlFlag>
##     <arnumber><![CDATA[6860266]]></arnumber>
##     <doi><![CDATA[10.1109/TCOMM.2014.2340855]]></doi>
##     <publicationId><![CDATA[6860266]]></publicationId>
##     <mdurl><![CDATA[http://ieeexplore.ieee.org/xpl/articleDetails.jsp?tp=&arnumber=6860266&contentType=Journals+%26+Magazines]]></mdurl>
##     <pdf><![CDATA[http://ieeexplore.ieee.org/stamp/stamp.jsp?arnumber=6860266]]></pdf>
##   </document>
##   <document>
##     <rank>1143</rank>
##     <title><![CDATA[Tunnel Field-Effect Transistors: State-of-the-Art]]></title>
##     <authors><![CDATA[Hao Lu;  Seabaugh, A.]]></authors>
##     <affiliations><![CDATA[Dept. of Electr. Eng., Univ. of Notre Dame, Notre Dame, IN, USA]]></affiliations>
##     <controlledterms>
##       <term><![CDATA[CMOS integrated circuits]]></term>
##       <term><![CDATA[field effect transistors]]></term>
##       <term><![CDATA[tunnel transistors]]></term>
##     </controlledterms>
##     <thesaurusterms>
##       <term><![CDATA[Field effect transistors]]></term>
##       <term><![CDATA[Indium gallium arsenide]]></term>
##       <term><![CDATA[Logic gates]]></term>
##       <term><![CDATA[Silicon]]></term>
##       <term><![CDATA[Tunneling]]></term>
##     </thesaurusterms>
##     <pubtitle><![CDATA[Electron Devices Society, IEEE Journal of the]]></pubtitle>
##     <punumber><![CDATA[6245494]]></punumber>
##     <pubtype><![CDATA[Journals & Magazines]]></pubtype>
##     <publisher><![CDATA[IEEE]]></publisher>
##     <volume><![CDATA[2]]></volume>
##     <issue><![CDATA[4]]></issue>
##     <py><![CDATA[2014]]></py>
##     <spage><![CDATA[44]]></spage>
##     <epage><![CDATA[49]]></epage>
##     <abstract><![CDATA[Progress in the development of tunnel field-effect transistors (TFETs) is reviewed by comparing experimental results and theoretical predictions against 16-nm FinFET CMOS technology. Experiments lag the projections, but sub-threshold swings less than 60 mV/decade are now reported in 14 TFETs. The lowest measured sub-threshold swings approaches 20 mV/decade, however, the measurements at these lowest values are not based on many points. The highest current at which sub-threshold swing below 60 mV/decade is observed is in the range 1-10 nA/&#x03BC;m. A common approach to TFET characterization is proposed to facilitate future comparisons.]]></abstract>
##     <issn><![CDATA[2168-6734]]></issn>
##     <htmlFlag><![CDATA[1]]></htmlFlag>
##     <arnumber><![CDATA[6820751]]></arnumber>
##     <doi><![CDATA[10.1109/JEDS.2014.2326622]]></doi>
##     <publicationId><![CDATA[6820751]]></publicationId>
##     <mdurl><![CDATA[http://ieeexplore.ieee.org/xpl/articleDetails.jsp?tp=&arnumber=6820751&contentType=Journals+%26+Magazines]]></mdurl>
##     <pdf><![CDATA[http://ieeexplore.ieee.org/stamp/stamp.jsp?arnumber=6820751]]></pdf>
##   </document>
##   <document>
##     <rank>1144</rank>
##     <title><![CDATA[A Statistical Model for the Headed and Tail Distributions of Random Telegraph Signal Magnitudes in Nanoscale MOSFETs]]></title>
##     <authors><![CDATA[Ming-Jer Chen;  Kong-Chiang Tu;  Huan-Hsiung Wang;  Chuan-Li Chen;  Shiou-Yi Lai;  You-Sheng Liu]]></authors>
##     <affiliations><![CDATA[Dept. of Electron. Eng., Nat. Chiao Tung Univ., Hsinchu, Taiwan]]></affiliations>
##     <controlledterms>
##       <term><![CDATA[MOSFET]]></term>
##       <term><![CDATA[dielectric materials]]></term>
##       <term><![CDATA[nanoelectronics]]></term>
##       <term><![CDATA[percolation]]></term>
##       <term><![CDATA[semiconductor device models]]></term>
##       <term><![CDATA[statistical distributions]]></term>
##       <term><![CDATA[telegraphy]]></term>
##     </controlledterms>
##     <thesaurusterms>
##       <term><![CDATA[Computational modeling]]></term>
##       <term><![CDATA[Electron traps]]></term>
##       <term><![CDATA[Logic gates]]></term>
##       <term><![CDATA[MOSFET]]></term>
##       <term><![CDATA[Semiconductor device modeling]]></term>
##       <term><![CDATA[Solid modeling]]></term>
##       <term><![CDATA[Threshold voltage]]></term>
##     </thesaurusterms>
##     <pubtitle><![CDATA[Electron Devices, IEEE Transactions on]]></pubtitle>
##     <punumber><![CDATA[16]]></punumber>
##     <pubtype><![CDATA[Journals & Magazines]]></pubtype>
##     <publisher><![CDATA[IEEE]]></publisher>
##     <volume><![CDATA[61]]></volume>
##     <issue><![CDATA[7]]></issue>
##     <py><![CDATA[2014]]></py>
##     <spage><![CDATA[2495]]></spage>
##     <epage><![CDATA[2502]]></epage>
##     <abstract><![CDATA[Trapping-detrapping of a single electron via an individual trap in metal-oxide-semiconductor field-effect transistor (MOSFET) gate dielectric constitutes two-level random telegraph signals. Recent 3-D technology computer-aided design (TCAD) simulations, on an individual MOSFET, revealed that with the position of the trap as a random variable, resulting random telegraph signals relative magnitude &#x0394;I<sub>d</sub>/I<sub>d</sub> in the subthreshold current at low drain voltage can have two distinct distributions: a headed one for a percolation-free channel and a tail one for a percolative channel. The latter may be effectively treated by a literature formula: (&#x0394;I<sub>d</sub>/I<sub>d</sub>) = (I<sub>loc</sub>/I<sub>d</sub>)<sup>2</sup>, where Iloc is the local current around the trap. In this paper, we show how to make this formula practically useful. First, we conduct 3-D TCAD simulations on a 35 &#x00D7; 35-nm<sup>2</sup> channel to provide &#x0394;I<sub>d</sub>/I<sub>d</sub> for a few positions of the trap. This leads to a new statistical model in closed form, which can reproduce headed distributions. Straightforwardly, key criteria are drawn from the model, which can act as guidelines for the adequate use of the I<sub>loc</sub>/I<sub>d</sub> formula. Extension to threshold voltage shift counterparts, from subthreshold through transition to inversion, is successfully achieved. Importantly, use of the model may overcome the drawbacks of the statistical experiment or simulation in the field.]]></abstract>
##     <issn><![CDATA[0018-9383]]></issn>
##     <htmlFlag><![CDATA[1]]></htmlFlag>
##     <arnumber><![CDATA[6823111]]></arnumber>
##     <doi><![CDATA[10.1109/TED.2014.2323259]]></doi>
##     <publicationId><![CDATA[6823111]]></publicationId>
##     <mdurl><![CDATA[http://ieeexplore.ieee.org/xpl/articleDetails.jsp?tp=&arnumber=6823111&contentType=Journals+%26+Magazines]]></mdurl>
##     <pdf><![CDATA[http://ieeexplore.ieee.org/stamp/stamp.jsp?arnumber=6823111]]></pdf>
##   </document>
##   <document>
##     <rank>1145</rank>
##     <title><![CDATA[Latent fingerprint matching: A survey]]></title>
##     <authors><![CDATA[Sankaran, A.;  Vatsa, M.;  Singh, R.]]></authors>
##     <affiliations><![CDATA[Indraprastha Inst. of Inf. Technol., New Delhi, India]]></affiliations>
##     <controlledterms>
##       <term><![CDATA[fingerprint identification]]></term>
##       <term><![CDATA[image matching]]></term>
##     </controlledterms>
##     <thesaurusterms>
##       <term><![CDATA[Fingerprint recognition]]></term>
##       <term><![CDATA[Law]]></term>
##       <term><![CDATA[Noise measurement]]></term>
##       <term><![CDATA[Pattern recognition]]></term>
##     </thesaurusterms>
##     <pubtitle><![CDATA[Access, IEEE]]></pubtitle>
##     <punumber><![CDATA[6287639]]></punumber>
##     <pubtype><![CDATA[Journals & Magazines]]></pubtype>
##     <publisher><![CDATA[IEEE]]></publisher>
##     <volume><![CDATA[2]]></volume>
##     <py><![CDATA[2014]]></py>
##     <spage><![CDATA[982]]></spage>
##     <epage><![CDATA[1004]]></epage>
##     <abstract><![CDATA[Latent fingerprint has been used as evidence in the court of law for over 100 years. However, even today, a completely automated latent fingerprint system has not been achieved. Researchers have identified several important challenges in latent fingerprint recognition: 1) low information content; 2) presence of background noise and nonlinear ridge distortion; 3) need for an established scientific procedure for matching latent fingerprints; and 4) lack of publicly available latent fingerprint databases. The process of automatic latent fingerprint matching is divided into five definite stages, and this paper discusses the existing algorithms, limitations, and future research directions in each of the stages.]]></abstract>
##     <issn><![CDATA[2169-3536]]></issn>
##     <htmlFlag><![CDATA[1]]></htmlFlag>
##     <arnumber><![CDATA[6880753]]></arnumber>
##     <doi><![CDATA[10.1109/ACCESS.2014.2349879]]></doi>
##     <publicationId><![CDATA[6880753]]></publicationId>
##     <mdurl><![CDATA[http://ieeexplore.ieee.org/xpl/articleDetails.jsp?tp=&arnumber=6880753&contentType=Journals+%26+Magazines]]></mdurl>
##     <pdf><![CDATA[http://ieeexplore.ieee.org/stamp/stamp.jsp?arnumber=6880753]]></pdf>
##   </document>
##   <document>
##     <rank>1146</rank>
##     <title><![CDATA[Estimation of Propagation Losses for Narrow Strip and Rib Waveguides]]></title>
##     <authors><![CDATA[Lindecrantz, S.M.;  Helleso, O.G.]]></authors>
##     <affiliations><![CDATA[Dept. of Phys. & Technol., Arctic Univ. of Norway, Troms&#x00F8;, Norway]]></affiliations>
##     <controlledterms>
##       <term><![CDATA[light polarisation]]></term>
##       <term><![CDATA[light propagation]]></term>
##       <term><![CDATA[optical losses]]></term>
##       <term><![CDATA[optical waveguides]]></term>
##       <term><![CDATA[refractive index]]></term>
##       <term><![CDATA[rib waveguides]]></term>
##       <term><![CDATA[tantalum compounds]]></term>
##     </controlledterms>
##     <thesaurusterms>
##       <term><![CDATA[Loss measurement]]></term>
##       <term><![CDATA[Optical sensors]]></term>
##       <term><![CDATA[Optical waveguides]]></term>
##       <term><![CDATA[Propagation losses]]></term>
##       <term><![CDATA[Refractive index]]></term>
##       <term><![CDATA[Scattering]]></term>
##       <term><![CDATA[Strips]]></term>
##     </thesaurusterms>
##     <pubtitle><![CDATA[Photonics Technology Letters, IEEE]]></pubtitle>
##     <punumber><![CDATA[68]]></punumber>
##     <pubtype><![CDATA[Journals & Magazines]]></pubtype>
##     <publisher><![CDATA[IEEE]]></publisher>
##     <volume><![CDATA[26]]></volume>
##     <issue><![CDATA[18]]></issue>
##     <py><![CDATA[2014]]></py>
##     <spage><![CDATA[1836]]></spage>
##     <epage><![CDATA[1839]]></epage>
##     <abstract><![CDATA[The dependence of propagation losses on waveguide width and polarization is measured for strip and rib waveguides made of tantalum pentoxide. For strip waveguides, the propagation losses are found to increase rapidly for widths &lt;;3 &#x03BC; m. The losses were significantly smaller for rib than for strip waveguides, as expected. A method is developed for estimating the dependence of propagation losses on waveguide width. The method is based on approximating sidewall imperfections with an area with complex refractive index in a 2D model and showed a good agreement with the measured dependence on waveguide width. The method is also used to predict that propagation losses will decrease rapidly for rib heights less than 20 nm for TM-polarization.]]></abstract>
##     <issn><![CDATA[1041-1135]]></issn>
##     <htmlFlag><![CDATA[1]]></htmlFlag>
##     <arnumber><![CDATA[6851126]]></arnumber>
##     <doi><![CDATA[10.1109/LPT.2014.2337055]]></doi>
##     <publicationId><![CDATA[6851126]]></publicationId>
##     <mdurl><![CDATA[http://ieeexplore.ieee.org/xpl/articleDetails.jsp?tp=&arnumber=6851126&contentType=Journals+%26+Magazines]]></mdurl>
##     <pdf><![CDATA[http://ieeexplore.ieee.org/stamp/stamp.jsp?arnumber=6851126]]></pdf>
##   </document>
##   <document>
##     <rank>1147</rank>
##     <title><![CDATA[Diamond Metal&#x2013;Semiconductor Field-Effect Transistor With Breakdown Voltage Over 1.5 kV]]></title>
##     <authors><![CDATA[Umezawa, H.;  Matsumoto, T.;  Shikata, S.-I.]]></authors>
##     <affiliations><![CDATA[Diamond Res. Group, Nat. Inst. of Adv. Ind. Sci. & Technol., Ikeda, Japan]]></affiliations>
##     <controlledterms>
##       <term><![CDATA[Schottky gate field effect transistors]]></term>
##       <term><![CDATA[diamond]]></term>
##       <term><![CDATA[semiconductor device breakdown]]></term>
##     </controlledterms>
##     <thesaurusterms>
##       <term><![CDATA[Diamonds]]></term>
##       <term><![CDATA[Logic gates]]></term>
##       <term><![CDATA[MESFETs]]></term>
##       <term><![CDATA[Schottky diodes]]></term>
##       <term><![CDATA[Temperature]]></term>
##       <term><![CDATA[Temperature measurement]]></term>
##     </thesaurusterms>
##     <pubtitle><![CDATA[Electron Device Letters, IEEE]]></pubtitle>
##     <punumber><![CDATA[55]]></punumber>
##     <pubtype><![CDATA[Journals & Magazines]]></pubtype>
##     <publisher><![CDATA[IEEE]]></publisher>
##     <volume><![CDATA[35]]></volume>
##     <issue><![CDATA[11]]></issue>
##     <py><![CDATA[2014]]></py>
##     <spage><![CDATA[1112]]></spage>
##     <epage><![CDATA[1114]]></epage>
##     <abstract><![CDATA[A diamond metal-semiconductor field-effect transistor (MESFET) with a Pt Schottky gate was fabricated. The MESFET exhibited clear saturation and pinchoff characteristics. The drain current of the MESFET operated at 300 &#x00B0;C was 20 times higher than that at room temperature due to the activation of acceptors. The breakdown voltage was highly dependent on the gate-drain length and reached 1.5 kV at a gate-drain length of 30 &#x03BC;m, which is the highest reported for a diamond FET.]]></abstract>
##     <issn><![CDATA[0741-3106]]></issn>
##     <htmlFlag><![CDATA[1]]></htmlFlag>
##     <arnumber><![CDATA[6910291]]></arnumber>
##     <doi><![CDATA[10.1109/LED.2014.2356191]]></doi>
##     <publicationId><![CDATA[6910291]]></publicationId>
##     <mdurl><![CDATA[http://ieeexplore.ieee.org/xpl/articleDetails.jsp?tp=&arnumber=6910291&contentType=Journals+%26+Magazines]]></mdurl>
##     <pdf><![CDATA[http://ieeexplore.ieee.org/stamp/stamp.jsp?arnumber=6910291]]></pdf>
##   </document>
##   <document>
##     <rank>1148</rank>
##     <title><![CDATA[Adaptive Postdistortion for Nonlinear LEDs in Visible Light Communications]]></title>
##     <authors><![CDATA[Qian, H.;  Yao, S.J.;  Cai, S.Z.;  Zhou, T.]]></authors>
##     <affiliations><![CDATA[Shanghai Inst. of Microsyst. & Inf. Technol., Shanghai, China]]></affiliations>
##     <controlledterms>
##       <term><![CDATA[light emitting diodes]]></term>
##       <term><![CDATA[nonlinear optics]]></term>
##       <term><![CDATA[optical receivers]]></term>
##       <term><![CDATA[telecommunication channels]]></term>
##     </controlledterms>
##     <thesaurusterms>
##       <term><![CDATA[Adaptation models]]></term>
##       <term><![CDATA[Adaptive systems]]></term>
##       <term><![CDATA[Light emitting diodes]]></term>
##       <term><![CDATA[OFDM]]></term>
##       <term><![CDATA[Polynomials]]></term>
##       <term><![CDATA[Predistortion]]></term>
##       <term><![CDATA[Training]]></term>
##     </thesaurusterms>
##     <pubtitle><![CDATA[Photonics Journal, IEEE]]></pubtitle>
##     <punumber><![CDATA[4563994]]></punumber>
##     <pubtype><![CDATA[Journals & Magazines]]></pubtype>
##     <publisher><![CDATA[IEEE]]></publisher>
##     <volume><![CDATA[6]]></volume>
##     <issue><![CDATA[4]]></issue>
##     <py><![CDATA[2014]]></py>
##     <spage><![CDATA[1]]></spage>
##     <epage><![CDATA[8]]></epage>
##     <abstract><![CDATA[In a visible light communication (VLC) system, the light-emitting diode (LED) is the major source of nonlinearity. The nonlinear effects in the VLC system are different from the conventional wireless communications system. The channel separation in the VLC system is significantly larger than the signal bandwidth; thus, the adjacent channel interference is not an issue. Predistortion technique may not be a cost-efficient approach since it needs additional feedback physical circuits at the transmitter. In this paper, we propose a postdistortion technique to estimate and compensate for the LED's nonlinearity at the receiver. The postdistortion technique only needs some additional computational resources. In addition, the proposed approach significantly improves the error-vector-magnitude and bit-error-rate performance of the VLC system. Simulation results validate the theoretical analysis.]]></abstract>
##     <issn><![CDATA[1943-0655]]></issn>
##     <htmlFlag><![CDATA[1]]></htmlFlag>
##     <arnumber><![CDATA[6837447]]></arnumber>
##     <doi><![CDATA[10.1109/JPHOT.2014.2331242]]></doi>
##     <publicationId><![CDATA[6837447]]></publicationId>
##     <mdurl><![CDATA[http://ieeexplore.ieee.org/xpl/articleDetails.jsp?tp=&arnumber=6837447&contentType=Journals+%26+Magazines]]></mdurl>
##     <pdf><![CDATA[http://ieeexplore.ieee.org/stamp/stamp.jsp?arnumber=6837447]]></pdf>
##   </document>
##   <document>
##     <rank>1149</rank>
##     <title><![CDATA[Soft-In Soft-Out Detection Using Partial Gaussian Approximation]]></title>
##     <authors><![CDATA[Qinghua Guo;  Huang, D.D.;  Nordholm, S.;  Jiangtao Xi;  Li Ping]]></authors>
##     <affiliations><![CDATA[Sch. of Electr., Comput. & Telecommun. Eng., Univ. of Wollongong, Wollongong, NSW, Australia]]></affiliations>
##     <controlledterms>
##       <term><![CDATA[Gaussian processes]]></term>
##       <term><![CDATA[MIMO communication]]></term>
##       <term><![CDATA[approximation theory]]></term>
##       <term><![CDATA[computational complexity]]></term>
##       <term><![CDATA[equalisers]]></term>
##       <term><![CDATA[intersymbol interference]]></term>
##       <term><![CDATA[maximum likelihood detection]]></term>
##       <term><![CDATA[probability]]></term>
##     </controlledterms>
##     <thesaurusterms>
##       <term><![CDATA[Approximation methods]]></term>
##       <term><![CDATA[Communication systems]]></term>
##       <term><![CDATA[Computational complexity]]></term>
##       <term><![CDATA[Detectors]]></term>
##       <term><![CDATA[Electronics packaging]]></term>
##       <term><![CDATA[Encoding]]></term>
##       <term><![CDATA[Gaussian approximation]]></term>
##     </thesaurusterms>
##     <pubtitle><![CDATA[Access, IEEE]]></pubtitle>
##     <punumber><![CDATA[6287639]]></punumber>
##     <pubtype><![CDATA[Journals & Magazines]]></pubtype>
##     <publisher><![CDATA[IEEE]]></publisher>
##     <volume><![CDATA[2]]></volume>
##     <py><![CDATA[2014]]></py>
##     <spage><![CDATA[427]]></spage>
##     <epage><![CDATA[436]]></epage>
##     <abstract><![CDATA[This paper concerns the soft-in soft-out detection in a coded communication system, where the transmitted symbols are discrete valued, and the exact a posteriori probability (APP) detection often involves prohibitive complexity. By using the properties of Gaussian functions, an approximate approach to the APP detection is devised with the idea that, in the computation of the APP of each symbol, the remaining symbols are distinguished based on their contributions to the APP of the concerned symbol, and the symbols with less contributions are approximated as (continuous) Gaussian variables [hence the name partial Gaussian approximation (PGA)] to reduce the computational complexity. The connection between the PGA detector and the reduced dimension maximum a posteriori detector (RDMAP) is investigated. It is shown that, PGA is equivalent to RDMAP, but it has a complexity much lower than that of RDMAP, i.e., PGA can be regarded as an efficient implementation of RDMAP. In addition, the application of PGA in intersymbol interference (ISI) channel equalization is also investigated. We show that PGA allows further significant complexity reduction by exploiting the circulant structure of the system transfer matrix, which makes PGA very attractive in handling severe ISI channels with large memory length.]]></abstract>
##     <issn><![CDATA[2169-3536]]></issn>
##     <htmlFlag><![CDATA[1]]></htmlFlag>
##     <arnumber><![CDATA[6810827]]></arnumber>
##     <doi><![CDATA[10.1109/ACCESS.2014.2322078]]></doi>
##     <publicationId><![CDATA[6810827]]></publicationId>
##     <mdurl><![CDATA[http://ieeexplore.ieee.org/xpl/articleDetails.jsp?tp=&arnumber=6810827&contentType=Journals+%26+Magazines]]></mdurl>
##     <pdf><![CDATA[http://ieeexplore.ieee.org/stamp/stamp.jsp?arnumber=6810827]]></pdf>
##   </document>
##   <document>
##     <rank>1150</rank>
##     <title><![CDATA[The Importance of Physical Quantities for the Analysis of Multitemporal and Multiangular Optical Very High Spatial Resolution Images]]></title>
##     <authors><![CDATA[Pacifici, F.;  Longbotham, N.;  Emery, W.J.]]></authors>
##     <affiliations><![CDATA[DigitalGlobe Inc., Longmont, CO, USA]]></affiliations>
##     <controlledterms>
##       <term><![CDATA[geophysical image processing]]></term>
##       <term><![CDATA[land cover]]></term>
##       <term><![CDATA[terrain mapping]]></term>
##     </controlledterms>
##     <thesaurusterms>
##       <term><![CDATA[Atmospheric measurements]]></term>
##       <term><![CDATA[Geometry]]></term>
##       <term><![CDATA[Satellites]]></term>
##       <term><![CDATA[Scattering]]></term>
##       <term><![CDATA[Sensors]]></term>
##       <term><![CDATA[Spatial resolution]]></term>
##       <term><![CDATA[Time series analysis]]></term>
##     </thesaurusterms>
##     <pubtitle><![CDATA[Geoscience and Remote Sensing, IEEE Transactions on]]></pubtitle>
##     <punumber><![CDATA[36]]></punumber>
##     <pubtype><![CDATA[Journals & Magazines]]></pubtype>
##     <publisher><![CDATA[IEEE]]></publisher>
##     <volume><![CDATA[52]]></volume>
##     <issue><![CDATA[10]]></issue>
##     <py><![CDATA[2014]]></py>
##     <spage><![CDATA[6241]]></spage>
##     <epage><![CDATA[6256]]></epage>
##     <abstract><![CDATA[The analysis of multitemporal very high spatial resolution imagery is too often limited to the sole use of pixel digital numbers which do not accurately describe the observed targets between the various collections due to the effects of changing illumination, viewing geometries, and atmospheric conditions. This paper demonstrates both qualitatively and quantitatively that not only physically based quantities are necessary to consistently and efficiently analyze these data sets but also the angular information of the acquisitions should not be neglected as it can provide unique features on the scenes being analyzed. The data set used is composed of 21 images acquired between 2002 and 2009 by QuickBird over the city of Denver, Colorado. The images were collected near the downtown area and include single family houses, skyscrapers, apartment complexes, industrial buildings, roads/highways, urban parks, and bodies of water. Experiments show that atmospheric and geometric properties of the acquisitions substantially affect the pixel values and, more specifically, that the raw counts are significantly correlated to the atmospheric visibility. Results of a 22-class urban land cover experiment show that an improvement of 0.374 in terms of Kappa coefficient can be achieved over the base case of raw pixels when surface reflectance values are combined to the angular decomposition of the time series.]]></abstract>
##     <issn><![CDATA[0196-2892]]></issn>
##     <htmlFlag><![CDATA[1]]></htmlFlag>
##     <arnumber><![CDATA[6737293]]></arnumber>
##     <doi><![CDATA[10.1109/TGRS.2013.2295819]]></doi>
##     <publicationId><![CDATA[6737293]]></publicationId>
##     <mdurl><![CDATA[http://ieeexplore.ieee.org/xpl/articleDetails.jsp?tp=&arnumber=6737293&contentType=Journals+%26+Magazines]]></mdurl>
##     <pdf><![CDATA[http://ieeexplore.ieee.org/stamp/stamp.jsp?arnumber=6737293]]></pdf>
##   </document>
##   <document>
##     <rank>1151</rank>
##     <title><![CDATA[Toward Energy Efficient Big Data Gathering in Densely Distributed Sensor Networks]]></title>
##     <authors><![CDATA[Takaishi, D.;  Nishiyama, H.;  Kato, N.;  Miura, R.]]></authors>
##     <affiliations><![CDATA[Grad. Sch. of Inf. Sci., Tohoku Univ., Sendai, Japan]]></affiliations>
##     <controlledterms>
##       <term><![CDATA[Big Data]]></term>
##       <term><![CDATA[energy conservation]]></term>
##       <term><![CDATA[expectation-maximisation algorithm]]></term>
##       <term><![CDATA[mobility management (mobile radio)]]></term>
##       <term><![CDATA[wireless sensor networks]]></term>
##     </controlledterms>
##     <thesaurusterms>
##       <term><![CDATA[Big data]]></term>
##       <term><![CDATA[Clustering algorithms]]></term>
##       <term><![CDATA[Data handling]]></term>
##       <term><![CDATA[Data storage systems]]></term>
##       <term><![CDATA[Energy consumption]]></term>
##       <term><![CDATA[Information management]]></term>
##       <term><![CDATA[Mobile communication]]></term>
##       <term><![CDATA[Wireless sensor networks]]></term>
##     </thesaurusterms>
##     <pubtitle><![CDATA[Emerging Topics in Computing, IEEE Transactions on]]></pubtitle>
##     <punumber><![CDATA[6245516]]></punumber>
##     <pubtype><![CDATA[Journals & Magazines]]></pubtype>
##     <publisher><![CDATA[IEEE]]></publisher>
##     <volume><![CDATA[2]]></volume>
##     <issue><![CDATA[3]]></issue>
##     <py><![CDATA[2014]]></py>
##     <spage><![CDATA[388]]></spage>
##     <epage><![CDATA[397]]></epage>
##     <abstract><![CDATA[Recently, the big data emerged as a hot topic because of the tremendous growth of the information and communication technology. One of the highly anticipated key contributors of the big data in the future networks is the distributed wireless sensor networks (WSNs). Although the data generated by an individual sensor may not appear to be significant, the overall data generated across numerous sensors in the densely distributed WSNs can produce a significant portion of the big data. Energy-efficient big data gathering in the densely distributed sensor networks is, therefore, a challenging research area. One of the most effective solutions to address this challenge is to utilize the sink node's mobility to facilitate the data gathering. While this technique can reduce energy consumption of the sensor nodes, the use of mobile sink presents additional challenges such as determining the sink node's trajectory and cluster formation prior to data collection. In this paper, we propose a new mobile sink routing and data gathering method through network clustering based on modified expectation-maximization technique. In addition, we derive an optimal number of clusters to minimize the energy consumption. The effectiveness of our proposal is verified through numerical results.]]></abstract>
##     <issn><![CDATA[2168-6750]]></issn>
##     <htmlFlag><![CDATA[1]]></htmlFlag>
##     <arnumber><![CDATA[6800057]]></arnumber>
##     <doi><![CDATA[10.1109/TETC.2014.2318177]]></doi>
##     <publicationId><![CDATA[6800057]]></publicationId>
##     <mdurl><![CDATA[http://ieeexplore.ieee.org/xpl/articleDetails.jsp?tp=&arnumber=6800057&contentType=Journals+%26+Magazines]]></mdurl>
##     <pdf><![CDATA[http://ieeexplore.ieee.org/stamp/stamp.jsp?arnumber=6800057]]></pdf>
##   </document>
##   <document>
##     <rank>1152</rank>
##     <title><![CDATA[Transceiver Optimization for Multi-Hop MIMO Relay Multicasting From Multiple Sources]]></title>
##     <authors><![CDATA[Khandaker, M.R.A.;  Yue Rong]]></authors>
##     <affiliations><![CDATA[Dept. of Electron. & Electr. Eng., Univ. Coll. London, London, UK]]></affiliations>
##     <controlledterms>
##       <term><![CDATA[MIMO communication]]></term>
##       <term><![CDATA[concave programming]]></term>
##       <term><![CDATA[convex programming]]></term>
##       <term><![CDATA[iterative methods]]></term>
##       <term><![CDATA[matrix algebra]]></term>
##       <term><![CDATA[mean square error methods]]></term>
##       <term><![CDATA[minimax techniques]]></term>
##       <term><![CDATA[multicast communication]]></term>
##       <term><![CDATA[precoding]]></term>
##       <term><![CDATA[radio transceivers]]></term>
##       <term><![CDATA[relay networks (telecommunication)]]></term>
##     </controlledterms>
##     <thesaurusterms>
##       <term><![CDATA[MIMO]]></term>
##       <term><![CDATA[Multicast communication]]></term>
##       <term><![CDATA[Receivers]]></term>
##       <term><![CDATA[Relays]]></term>
##       <term><![CDATA[Transmitters]]></term>
##       <term><![CDATA[Vectors]]></term>
##       <term><![CDATA[Wireless communication]]></term>
##     </thesaurusterms>
##     <pubtitle><![CDATA[Wireless Communications, IEEE Transactions on]]></pubtitle>
##     <punumber><![CDATA[7693]]></punumber>
##     <pubtype><![CDATA[Journals & Magazines]]></pubtype>
##     <publisher><![CDATA[IEEE]]></publisher>
##     <volume><![CDATA[13]]></volume>
##     <issue><![CDATA[9]]></issue>
##     <py><![CDATA[2014]]></py>
##     <spage><![CDATA[5162]]></spage>
##     <epage><![CDATA[5172]]></epage>
##     <abstract><![CDATA[In this paper, we consider a multicasting multiple-input multiple-output (MIMO) relay system where multiple transmitters multicast their own messages to a group of receivers over multiple hops, and all nodes are equipped with multiple antennas. The joint transmit and relay precoding design problem has been investigated for multicasting multiple data streams based on min-max mean-squared error (MSE) criterion. We aim at minimizing the maximal MSE of the signal waveform estimation among all receivers subjecting to power constraints at the transmitters and all the relay nodes. This problem is highly nonconvex with matrix variables and the exactly optimal solution is very hard to obtain. We develop an iterative algorithm to jointly optimize the transmitter, relay, and receiver matrices by solving convex subproblems. By exploiting the optimal structure of the relay precoding matrices, we then propose a low complexity solution for the problem under some mild approximation. In particular, we show that under (moderately) high signal-to-noise ratio assumption, the min-max optimization problem can be solved using the semidefinite programming technique. Numerical simulations demonstrate the effectiveness of the proposed algorithms.]]></abstract>
##     <issn><![CDATA[1536-1276]]></issn>
##     <htmlFlag><![CDATA[1]]></htmlFlag>
##     <arnumber><![CDATA[6811195]]></arnumber>
##     <doi><![CDATA[10.1109/TWC.2014.2322361]]></doi>
##     <publicationId><![CDATA[6811195]]></publicationId>
##     <mdurl><![CDATA[http://ieeexplore.ieee.org/xpl/articleDetails.jsp?tp=&arnumber=6811195&contentType=Journals+%26+Magazines]]></mdurl>
##     <pdf><![CDATA[http://ieeexplore.ieee.org/stamp/stamp.jsp?arnumber=6811195]]></pdf>
##   </document>
##   <document>
##     <rank>1153</rank>
##     <title><![CDATA[Data Detection and Code Channel Allocation for Frequency-Domain Spread ACO-OFDM Systems Over Indoor Diffuse Wireless Channels]]></title>
##     <authors><![CDATA[Jun-Bo Wang;  Peng Jiang;  Jiangzhou Wang;  Ming Chen;  Jin-Yuan Wang]]></authors>
##     <affiliations><![CDATA[Nat. Mobile Commun. Res. Lab., Southeast Univ., Nanjing, China]]></affiliations>
##     <controlledterms>
##       <term><![CDATA[OFDM modulation]]></term>
##       <term><![CDATA[channel allocation]]></term>
##       <term><![CDATA[channel coding]]></term>
##       <term><![CDATA[error statistics]]></term>
##       <term><![CDATA[indoor radio]]></term>
##       <term><![CDATA[numerical analysis]]></term>
##       <term><![CDATA[optical receivers]]></term>
##     </controlledterms>
##     <thesaurusterms>
##       <term><![CDATA[Channel allocation]]></term>
##       <term><![CDATA[Frequency-domain analysis]]></term>
##       <term><![CDATA[High-speed optical techniques]]></term>
##       <term><![CDATA[Nonlinear optics]]></term>
##       <term><![CDATA[OFDM]]></term>
##       <term><![CDATA[Optical receivers]]></term>
##       <term><![CDATA[Optical transmitters]]></term>
##     </thesaurusterms>
##     <pubtitle><![CDATA[Photonics Journal, IEEE]]></pubtitle>
##     <punumber><![CDATA[4563994]]></punumber>
##     <pubtype><![CDATA[Journals & Magazines]]></pubtype>
##     <publisher><![CDATA[IEEE]]></publisher>
##     <volume><![CDATA[6]]></volume>
##     <issue><![CDATA[1]]></issue>
##     <py><![CDATA[2014]]></py>
##     <spage><![CDATA[1]]></spage>
##     <epage><![CDATA[16]]></epage>
##     <abstract><![CDATA[Future optical wireless communication systems promise to provide high-speed data transmission in indoor diffuse environments. This paper considers frequency-domain spread asymmetrically clipped optical orthogonal frequency-division multiplexing (ACO-OFDM) systems in indoor diffuse channels and aims to develop efficient data detection and code channel allocation schemes. By exploiting the frequency-domain spread concept, a linear multi-code detection scheme is proposed to maximize the signal to interference plus noise ratio (SINR) at the receiver. The achieved SINR and bit error ratio (BER) performance are analyzed. A computationally efficient code channel allocation algorithm is proposed to improve the BER performance of the frequency-domain spread ACO-OFDM system. Numerical results show that the frequency-domain spread ACO-OFDM system outperforms conventional ACO-OFDM systems in indoor diffuse channels. Moreover, the proposed linear multi-code detection and code channel allocation algorithm can improve the performance of optical peak-to-average power ratio (PAPR).]]></abstract>
##     <issn><![CDATA[1943-0655]]></issn>
##     <htmlFlag><![CDATA[1]]></htmlFlag>
##     <arnumber><![CDATA[6739127]]></arnumber>
##     <doi><![CDATA[10.1109/JPHOT.2014.2305736]]></doi>
##     <publicationId><![CDATA[6739127]]></publicationId>
##     <mdurl><![CDATA[http://ieeexplore.ieee.org/xpl/articleDetails.jsp?tp=&arnumber=6739127&contentType=Journals+%26+Magazines]]></mdurl>
##     <pdf><![CDATA[http://ieeexplore.ieee.org/stamp/stamp.jsp?arnumber=6739127]]></pdf>
##   </document>
##   <document>
##     <rank>1154</rank>
##     <title><![CDATA[Partial Discharge Localization in Power Transformers Using Neuro-Fuzzy Technique]]></title>
##     <authors><![CDATA[Homaei, M.;  Moosavian, S.M.;  Illias, H.A.]]></authors>
##     <affiliations><![CDATA[Dept. of Electr. Eng., K.N Toosi Univ. of Technol., Tehran, Iran]]></affiliations>
##     <controlledterms>
##       <term><![CDATA[calibration]]></term>
##       <term><![CDATA[fuzzy set theory]]></term>
##       <term><![CDATA[maintenance engineering]]></term>
##       <term><![CDATA[partial discharge measurement]]></term>
##       <term><![CDATA[pattern recognition]]></term>
##       <term><![CDATA[power transformer insulation]]></term>
##       <term><![CDATA[transfer functions]]></term>
##       <term><![CDATA[transformer windings]]></term>
##     </controlledterms>
##     <thesaurusterms>
##       <term><![CDATA[Insulators]]></term>
##       <term><![CDATA[Partial discharges]]></term>
##       <term><![CDATA[Power transformer insulation]]></term>
##       <term><![CDATA[Transforms]]></term>
##       <term><![CDATA[Vectors]]></term>
##       <term><![CDATA[Windings]]></term>
##     </thesaurusterms>
##     <pubtitle><![CDATA[Power Delivery, IEEE Transactions on]]></pubtitle>
##     <punumber><![CDATA[61]]></punumber>
##     <pubtype><![CDATA[Journals & Magazines]]></pubtype>
##     <publisher><![CDATA[IEEE]]></publisher>
##     <volume><![CDATA[29]]></volume>
##     <issue><![CDATA[5]]></issue>
##     <py><![CDATA[2014]]></py>
##     <spage><![CDATA[2066]]></spage>
##     <epage><![CDATA[2076]]></epage>
##     <abstract><![CDATA[Partial discharge (PD) is the most common sources of insulation failure in power transformers. The most important tools for quality assessment of power transformers are PD detection, measurement, and classification. As for the maintenance and repair of transformers, the major importance is the techniques for locating a PD source. The transfer function-based (TF) method for power transformers' winding in the high-frequency range is commonly used in power engineering applications, such as transient analysis, insulation coordination, and in transformer design. Although it is possible to localize PD in transformer winding using the transfer function (TF) method, this method cannot be used for transformers with no design data. Previous attempts toward finding a feature that localizes PD in transformers in general that lineate with PD location were found to be less successful. Therefore, in this paper, a neuro-fuzzy technique that uses unsupervised pattern recognition was proposed to localize PD source in power transformers. The proposed method was tested on a medium-voltage transformer winding in the laboratory. The results showed a significant improvement in localizing PD for major types of PD compared to currently available techniques, such as orthogonal transforms and the calibration line method.]]></abstract>
##     <issn><![CDATA[0885-8977]]></issn>
##     <htmlFlag><![CDATA[1]]></htmlFlag>
##     <arnumber><![CDATA[6866921]]></arnumber>
##     <doi><![CDATA[10.1109/TPWRD.2014.2339274]]></doi>
##     <publicationId><![CDATA[6866921]]></publicationId>
##     <mdurl><![CDATA[http://ieeexplore.ieee.org/xpl/articleDetails.jsp?tp=&arnumber=6866921&contentType=Journals+%26+Magazines]]></mdurl>
##     <pdf><![CDATA[http://ieeexplore.ieee.org/stamp/stamp.jsp?arnumber=6866921]]></pdf>
##   </document>
##   <document>
##     <rank>1155</rank>
##     <title><![CDATA[A Multi-Channel Multi-Bit Programmable Photonic Beamformer Based on Cascaded DWDM]]></title>
##     <authors><![CDATA[Anliang Yu;  Weiwen Zou;  Shuguang Li;  Jianping Chen]]></authors>
##     <affiliations><![CDATA[Dept. of Electron. Eng., Shanghai Jiao Tong Univ., Shanghai, China]]></affiliations>
##     <controlledterms>
##       <term><![CDATA[antenna phased arrays]]></term>
##       <term><![CDATA[array signal processing]]></term>
##       <term><![CDATA[delays]]></term>
##       <term><![CDATA[optical beam splitters]]></term>
##       <term><![CDATA[optical switches]]></term>
##       <term><![CDATA[wavelength division multiplexing]]></term>
##     </controlledterms>
##     <thesaurusterms>
##       <term><![CDATA[Delays]]></term>
##       <term><![CDATA[Optical fibers]]></term>
##       <term><![CDATA[Optical switches]]></term>
##       <term><![CDATA[Photonics]]></term>
##       <term><![CDATA[Radio frequency]]></term>
##       <term><![CDATA[Wavelength division multiplexing]]></term>
##     </thesaurusterms>
##     <pubtitle><![CDATA[Photonics Journal, IEEE]]></pubtitle>
##     <punumber><![CDATA[4563994]]></punumber>
##     <pubtype><![CDATA[Journals & Magazines]]></pubtype>
##     <publisher><![CDATA[IEEE]]></publisher>
##     <volume><![CDATA[6]]></volume>
##     <issue><![CDATA[4]]></issue>
##     <py><![CDATA[2014]]></py>
##     <spage><![CDATA[1]]></spage>
##     <epage><![CDATA[10]]></epage>
##     <abstract><![CDATA[We propose and experimentally demonstrate a novel multi-channel multi-bit programmable photonic beamformer, which consists of multiple parallel photonic true-time delay (PTTD) subnetworks based on cascaded DWDMs. Each PTTD subnetwork can provide multi-bit true-time delays (TTDs) for a multi-channel modulated RF signal of a phased array antenna (PAA) and operate over a wide RF range from 10 MHz to 12 GHz. The RF signal is modulated to a spectrally separated laser array combined by a DWDM to form a multi-channel spatially separated PAA. The multi-bit TTDs for scanning PAA angles are rapidly programmed by means of a control circuit of optical switches in the PTTD subnetwork. By using an optical splitter, multiple parallel PTTD subnetworks can be topologically configured out for a PAA with a large number of elements. The obtained radiation patterns demonstrate that the proof-of-concept experimental result agrees well with the theoretical analysis, verifying the feasibility of the proposed photonic beamformer.]]></abstract>
##     <issn><![CDATA[1943-0655]]></issn>
##     <htmlFlag><![CDATA[1]]></htmlFlag>
##     <arnumber><![CDATA[6867359]]></arnumber>
##     <doi><![CDATA[10.1109/JPHOT.2014.2344007]]></doi>
##     <publicationId><![CDATA[6867359]]></publicationId>
##     <mdurl><![CDATA[http://ieeexplore.ieee.org/xpl/articleDetails.jsp?tp=&arnumber=6867359&contentType=Journals+%26+Magazines]]></mdurl>
##     <pdf><![CDATA[http://ieeexplore.ieee.org/stamp/stamp.jsp?arnumber=6867359]]></pdf>
##   </document>
##   <document>
##     <rank>1156</rank>
##     <title><![CDATA[Natural Language Generation as Incremental Planning Under Uncertainty: Adaptive Information Presentation for Statistical Dialogue Systems]]></title>
##     <authors><![CDATA[Rieser, V.;  Lemon, O.;  Keizer, S.]]></authors>
##     <affiliations><![CDATA[Sch. of Math. & Comput. Sci. (MACS), Heriot-Watt Univ., Edinburgh, UK]]></affiliations>
##     <controlledterms>
##       <term><![CDATA[learning (artificial intelligence)]]></term>
##       <term><![CDATA[natural language processing]]></term>
##       <term><![CDATA[planning (artificial intelligence)]]></term>
##     </controlledterms>
##     <thesaurusterms>
##       <term><![CDATA[Context]]></term>
##       <term><![CDATA[Databases]]></term>
##       <term><![CDATA[IP networks]]></term>
##       <term><![CDATA[Planning]]></term>
##       <term><![CDATA[Speech]]></term>
##       <term><![CDATA[Speech processing]]></term>
##       <term><![CDATA[Uncertainty]]></term>
##     </thesaurusterms>
##     <pubtitle><![CDATA[Audio, Speech, and Language Processing, IEEE/ACM Transactions on]]></pubtitle>
##     <punumber><![CDATA[6570655]]></punumber>
##     <pubtype><![CDATA[Journals & Magazines]]></pubtype>
##     <publisher><![CDATA[IEEE]]></publisher>
##     <volume><![CDATA[22]]></volume>
##     <issue><![CDATA[5]]></issue>
##     <py><![CDATA[2014]]></py>
##     <spage><![CDATA[979]]></spage>
##     <epage><![CDATA[994]]></epage>
##     <abstract><![CDATA[We present and evaluate a novel approach to natural language generation (NLG) in statistical spoken dialogue systems (SDS) using a data-driven statistical optimization framework for incremental information presentation (IP), where there is a trade-off to be solved between presenting &#x201C;enough" information to the user while keeping the utterances short and understandable. The trained IP model is adaptive to variation from the current generation context (e.g. a user and a non-deterministic sentence planner), and it incrementally adapts the IP policy at the turn level. Reinforcement learning is used to automatically optimize the IP policy with respect to a data-driven objective function. In a case study on presenting restaurant information, we show that an optimized IP strategy trained on Wizard-of-Oz data outperforms a baseline mimicking the wizard behavior in terms of total reward gained. The policy is then also tested with real users, and improves on a conventional hand-coded IP strategy used in a deployed SDS in terms of overall task success. The evaluation found that the trained IP strategy significantly improves dialogue task completion for real users, with up to a 8.2% increase in task success. This methodology also provides new insights into the nature of the IP problem, which has previously been treated as a module following dialogue management with no access to lower-level context features (e.g. from a surface realizer and/or speech synthesizer).]]></abstract>
##     <issn><![CDATA[2329-9290]]></issn>
##     <htmlFlag><![CDATA[1]]></htmlFlag>
##     <arnumber><![CDATA[6782700]]></arnumber>
##     <doi><![CDATA[10.1109/TASL.2014.2315271]]></doi>
##     <publicationId><![CDATA[6782700]]></publicationId>
##     <mdurl><![CDATA[http://ieeexplore.ieee.org/xpl/articleDetails.jsp?tp=&arnumber=6782700&contentType=Journals+%26+Magazines]]></mdurl>
##     <pdf><![CDATA[http://ieeexplore.ieee.org/stamp/stamp.jsp?arnumber=6782700]]></pdf>
##   </document>
##   <document>
##     <rank>1157</rank>
##     <title><![CDATA[Low Temperature Plasma Causes Double-Strand Break DNA Damage in Primary Epithelial Cells Cultured From a Human Prostate Tumor]]></title>
##     <authors><![CDATA[Hirst, A.M.;  Frame, F.M.;  Maitland, N.J.;  O'Connell, D.]]></authors>
##     <affiliations><![CDATA[Dept. of Phys. & Biol., Univ. of York, York, UK]]></affiliations>
##     <controlledterms>
##       <term><![CDATA[DNA]]></term>
##       <term><![CDATA[cancer]]></term>
##       <term><![CDATA[cellular biophysics]]></term>
##       <term><![CDATA[microorganisms]]></term>
##       <term><![CDATA[molecular biophysics]]></term>
##       <term><![CDATA[plasma applications]]></term>
##       <term><![CDATA[radiation therapy]]></term>
##       <term><![CDATA[tumours]]></term>
##       <term><![CDATA[wounds]]></term>
##     </controlledterms>
##     <thesaurusterms>
##       <term><![CDATA[Cancer]]></term>
##       <term><![CDATA[Cells (biology)]]></term>
##       <term><![CDATA[DNA]]></term>
##       <term><![CDATA[Plasma temperature]]></term>
##       <term><![CDATA[Principal component analysis]]></term>
##       <term><![CDATA[Tumors]]></term>
##     </thesaurusterms>
##     <pubtitle><![CDATA[Plasma Science, IEEE Transactions on]]></pubtitle>
##     <punumber><![CDATA[27]]></punumber>
##     <pubtype><![CDATA[Journals & Magazines]]></pubtype>
##     <publisher><![CDATA[IEEE]]></publisher>
##     <volume><![CDATA[42]]></volume>
##     <issue><![CDATA[10]]></issue>
##     <part><![CDATA[1]]></part>
##     <py><![CDATA[2014]]></py>
##     <spage><![CDATA[2740]]></spage>
##     <epage><![CDATA[2741]]></epage>
##     <abstract><![CDATA[Research in the new field of plasma medicine continues to demonstrate the efficacy of low temperature plasmas (LTPs) for numerous biomedical applications. Responses such as reduction in cell viability and cell death for cancer therapy, cell proliferation for wound healing, and bacterial inactivation have been observed as a result of plasma treatment. In this paper, we applied LTP to prostate cancer primary cells derived from patient tumour tissue to inflict irreparable DNA damage.]]></abstract>
##     <issn><![CDATA[0093-3813]]></issn>
##     <htmlFlag><![CDATA[1]]></htmlFlag>
##     <arnumber><![CDATA[6894576]]></arnumber>
##     <doi><![CDATA[10.1109/TPS.2014.2351453]]></doi>
##     <publicationId><![CDATA[6894576]]></publicationId>
##     <mdurl><![CDATA[http://ieeexplore.ieee.org/xpl/articleDetails.jsp?tp=&arnumber=6894576&contentType=Journals+%26+Magazines]]></mdurl>
##     <pdf><![CDATA[http://ieeexplore.ieee.org/stamp/stamp.jsp?arnumber=6894576]]></pdf>
##   </document>
##   <document>
##     <rank>1158</rank>
##     <title><![CDATA[Multi-State Logging Freeze Detection Passive RFID Tags]]></title>
##     <authors><![CDATA[Ziai, M.A.;  Batchelor, J.C.]]></authors>
##     <affiliations><![CDATA[Sch. of Eng., Univ. of Kent, Canterbury, UK]]></affiliations>
##     <controlledterms>
##       <term><![CDATA[environmental monitoring (geophysics)]]></term>
##       <term><![CDATA[geophysical techniques]]></term>
##       <term><![CDATA[radiofrequency identification]]></term>
##     </controlledterms>
##     <thesaurusterms>
##       <term><![CDATA[Impedance]]></term>
##       <term><![CDATA[Metals]]></term>
##       <term><![CDATA[Reservoirs]]></term>
##       <term><![CDATA[Slot antennas]]></term>
##       <term><![CDATA[Temperature sensors]]></term>
##     </thesaurusterms>
##     <pubtitle><![CDATA[Antennas and Propagation, IEEE Transactions on]]></pubtitle>
##     <punumber><![CDATA[8]]></punumber>
##     <pubtype><![CDATA[Journals & Magazines]]></pubtype>
##     <publisher><![CDATA[IEEE]]></publisher>
##     <volume><![CDATA[62]]></volume>
##     <issue><![CDATA[12]]></issue>
##     <py><![CDATA[2014]]></py>
##     <spage><![CDATA[6406]]></spage>
##     <epage><![CDATA[6411]]></epage>
##     <abstract><![CDATA[In this paper, the design and measurement of a passive UHF RFID smart tag suitable for monitoring and recording critical temperature violations in cold chain management are presented. The tag uses moving parts to detect and log different temperate states without the requirement for transceivers, memory, and batteries. A simple mechanical method is proposed whereby a moving metallic plate is trapped in one of four possible positions by specific environmental temperatures whereby inducing a permanent state-based change in the passive RFID tag response. The latched product critical temperature violations can be monitored via the read power required to turn on the tag transponder chip, which differs in each state.]]></abstract>
##     <issn><![CDATA[0018-926X]]></issn>
##     <htmlFlag><![CDATA[1]]></htmlFlag>
##     <arnumber><![CDATA[6905780]]></arnumber>
##     <doi><![CDATA[10.1109/TAP.2014.2359474]]></doi>
##     <publicationId><![CDATA[6905780]]></publicationId>
##     <mdurl><![CDATA[http://ieeexplore.ieee.org/xpl/articleDetails.jsp?tp=&arnumber=6905780&contentType=Journals+%26+Magazines]]></mdurl>
##     <pdf><![CDATA[http://ieeexplore.ieee.org/stamp/stamp.jsp?arnumber=6905780]]></pdf>
##   </document>
##   <document>
##     <rank>1159</rank>
##     <title><![CDATA[Adaptive Parameter Estimation of Power System Dynamic Model Using Modal Information]]></title>
##     <authors><![CDATA[Song Guo;  Norris, S.;  Bialek, J.]]></authors>
##     <affiliations><![CDATA[London Power Assoc. Ltd., Manchester, UK]]></affiliations>
##     <controlledterms>
##       <term><![CDATA[damping]]></term>
##       <term><![CDATA[frequency measurement]]></term>
##       <term><![CDATA[power system measurement]]></term>
##       <term><![CDATA[power system parameter estimation]]></term>
##       <term><![CDATA[regression analysis]]></term>
##       <term><![CDATA[synchronous generators]]></term>
##     </controlledterms>
##     <thesaurusterms>
##       <term><![CDATA[Damping]]></term>
##       <term><![CDATA[Parameter estimation]]></term>
##       <term><![CDATA[Power system dynamics]]></term>
##       <term><![CDATA[Power system modeling]]></term>
##       <term><![CDATA[Signal analysis]]></term>
##       <term><![CDATA[Synchronous generators]]></term>
##       <term><![CDATA[Wide area measurements]]></term>
##     </thesaurusterms>
##     <pubtitle><![CDATA[Power Systems, IEEE Transactions on]]></pubtitle>
##     <punumber><![CDATA[59]]></punumber>
##     <pubtype><![CDATA[Journals & Magazines]]></pubtype>
##     <publisher><![CDATA[IEEE]]></publisher>
##     <volume><![CDATA[29]]></volume>
##     <issue><![CDATA[6]]></issue>
##     <py><![CDATA[2014]]></py>
##     <spage><![CDATA[2854]]></spage>
##     <epage><![CDATA[2861]]></epage>
##     <abstract><![CDATA[A novel method for estimating parameters of a dynamic system model is presented using estimates of dynamic system modes (frequency and damping) obtained from wide area measurement systems (WAMS). The parameter estimation scheme is based on weighted least squares (WLS) method that utilizes sensitivities of the measured modal frequencies and damping to the parameters. The paper concentrates on estimating the values of generator inertias but the proposed methodology is general and can be used to identify other generator parameters such as damping coefficients. The methodology has been tested using a wide range of accuracy in the measured modes of oscillations. The results suggest that the methodology is capable of estimating accurately inertias and replicating the dynamic behavior of the power system. It has been shown that the damping measurements do not influence estimation of generator inertia. The method has overcome the problem of observability, when there were fewer measurements than the parameters to be estimated, by including the assumed values of parameters as pseudo-measurements.]]></abstract>
##     <issn><![CDATA[0885-8950]]></issn>
##     <htmlFlag><![CDATA[1]]></htmlFlag>
##     <arnumber><![CDATA[6805671]]></arnumber>
##     <doi><![CDATA[10.1109/TPWRS.2014.2316916]]></doi>
##     <publicationId><![CDATA[6805671]]></publicationId>
##     <mdurl><![CDATA[http://ieeexplore.ieee.org/xpl/articleDetails.jsp?tp=&arnumber=6805671&contentType=Journals+%26+Magazines]]></mdurl>
##     <pdf><![CDATA[http://ieeexplore.ieee.org/stamp/stamp.jsp?arnumber=6805671]]></pdf>
##   </document>
##   <document>
##     <rank>1160</rank>
##     <title><![CDATA[Virtual Reference Device-Based NLOS Localization in Multipath Environment]]></title>
##     <authors><![CDATA[Chen, S.W.;  Seow, C.K.;  Tan, S.Y.]]></authors>
##     <affiliations><![CDATA[Sch. of Electr. & Electron. Eng., Nanyang Technol. Univ., Singapore, Singapore]]></affiliations>
##     <controlledterms>
##       <term><![CDATA[mobile communication]]></term>
##       <term><![CDATA[multipath channels]]></term>
##     </controlledterms>
##     <thesaurusterms>
##       <term><![CDATA[Accuracy]]></term>
##       <term><![CDATA[Antenna measurements]]></term>
##       <term><![CDATA[Educational institutions]]></term>
##       <term><![CDATA[Position measurement]]></term>
##       <term><![CDATA[Ray tracing]]></term>
##       <term><![CDATA[Standards]]></term>
##       <term><![CDATA[Wireless communication]]></term>
##     </thesaurusterms>
##     <pubtitle><![CDATA[Antennas and Wireless Propagation Letters, IEEE]]></pubtitle>
##     <punumber><![CDATA[7727]]></punumber>
##     <pubtype><![CDATA[Journals & Magazines]]></pubtype>
##     <publisher><![CDATA[IEEE]]></publisher>
##     <volume><![CDATA[13]]></volume>
##     <py><![CDATA[2014]]></py>
##     <spage><![CDATA[1409]]></spage>
##     <epage><![CDATA[1412]]></epage>
##     <abstract><![CDATA[This letter presents a novel non-line-of-sight (NLOS) localization scheme based on the concept of virtual reference device. The position of virtual reference device for each NLOS path can be determined if the initial mobile device position can be estimated or when the mobile device transits from LOS to NLOS region. With the position of virtual reference devices, the subsequent localization of mobile device requires only one signal path. Simulation and experiment conducted in indoor multipath environment show that the performance of our proposed localization scheme surpasses the existing localization schemes by a significant margin at all simulated and measured locations.]]></abstract>
##     <issn><![CDATA[1536-1225]]></issn>
##     <htmlFlag><![CDATA[1]]></htmlFlag>
##     <arnumber><![CDATA[6858043]]></arnumber>
##     <doi><![CDATA[10.1109/LAWP.2014.2340472]]></doi>
##     <publicationId><![CDATA[6858043]]></publicationId>
##     <mdurl><![CDATA[http://ieeexplore.ieee.org/xpl/articleDetails.jsp?tp=&arnumber=6858043&contentType=Journals+%26+Magazines]]></mdurl>
##     <pdf><![CDATA[http://ieeexplore.ieee.org/stamp/stamp.jsp?arnumber=6858043]]></pdf>
##   </document>
##   <document>
##     <rank>1161</rank>
##     <title><![CDATA[Topological Insulator Simultaneously Q-Switched Dual-Wavelength <formula formulatype="inline"> <img src="/images/tex/21468.gif" alt=" \hbox {Nd}:\hbox {Lu}_{2}\hbox {O}_{3}"> </formula> Laser]]></title>
##     <authors><![CDATA[Baolin Wang;  Haohai Yu;  Han Zhang;  Chujun Zhao;  Shuangchun Wen;  Huaijin Zhang;  Jiyang Wang]]></authors>
##     <affiliations><![CDATA[State Key Lab. of Crystal Mater., Shandong Univ., Jinan, China]]></affiliations>
##     <controlledterms>
##       <term><![CDATA[Q-switching]]></term>
##       <term><![CDATA[bismuth compounds]]></term>
##       <term><![CDATA[lutetium compounds]]></term>
##       <term><![CDATA[neodymium]]></term>
##       <term><![CDATA[optical saturable absorption]]></term>
##       <term><![CDATA[solid lasers]]></term>
##       <term><![CDATA[topological insulators]]></term>
##     </controlledterms>
##     <thesaurusterms>
##       <term><![CDATA[Crystals]]></term>
##       <term><![CDATA[Laser excitation]]></term>
##       <term><![CDATA[Laser modes]]></term>
##       <term><![CDATA[Optimized production technology]]></term>
##       <term><![CDATA[Power generation]]></term>
##       <term><![CDATA[Pump lasers]]></term>
##       <term><![CDATA[Topological insulators]]></term>
##     </thesaurusterms>
##     <pubtitle><![CDATA[Photonics Journal, IEEE]]></pubtitle>
##     <punumber><![CDATA[4563994]]></punumber>
##     <pubtype><![CDATA[Journals & Magazines]]></pubtype>
##     <publisher><![CDATA[IEEE]]></publisher>
##     <volume><![CDATA[6]]></volume>
##     <issue><![CDATA[3]]></issue>
##     <py><![CDATA[2014]]></py>
##     <spage><![CDATA[1]]></spage>
##     <epage><![CDATA[7]]></epage>
##     <abstract><![CDATA[We demonstrated the dual-wavelength simultaneously Q-switched Nd: Lu<sub>2</sub>O<sub>3</sub> laser with a topological insulator Bi<sub>2</sub>Se<sub>3</sub> as the Q-switcher. The continuous-wave Nd:Lu<sub>2</sub>O<sub>3</sub> crystal laser with the slope efficiency of 31% was achieved, and the first Nd:Lu<sub>2</sub>O<sub>3</sub> pulsed laser operation was reported, as far as we know. By using the wavelength-insensitive saturable absorption of topological insulators, the dual wavelength of the Nd:Lu<sub>2</sub>O<sub>3</sub> crystal laser can be Q-switched simultaneously without the frequency selection of the saturable absorber, since the emission cross-sections of Nd:Lu<sub>2</sub>O<sub>3</sub> at the wavelengths of 1077 and 1081 nm are comparable. The results indicate that the topological insulator is wavelength-insensitive and should be suitable for the dual-wavelength simultaneously Q-switched laser as a passive Q-switcher. It can be proposed that this work would provide an efficient technology for the dual-wavelength simultaneously Q-switched laser, which has promising applications in many regimes.]]></abstract>
##     <issn><![CDATA[1943-0655]]></issn>
##     <htmlFlag><![CDATA[1]]></htmlFlag>
##     <arnumber><![CDATA[6807511]]></arnumber>
##     <doi><![CDATA[10.1109/JPHOT.2014.2320734]]></doi>
##     <publicationId><![CDATA[6807511]]></publicationId>
##     <mdurl><![CDATA[http://ieeexplore.ieee.org/xpl/articleDetails.jsp?tp=&arnumber=6807511&contentType=Journals+%26+Magazines]]></mdurl>
##     <pdf><![CDATA[http://ieeexplore.ieee.org/stamp/stamp.jsp?arnumber=6807511]]></pdf>
##   </document>
##   <document>
##     <rank>1162</rank>
##     <title><![CDATA[Quaternion Reproducing Kernel Hilbert Spaces: Existence and Uniqueness Conditions]]></title>
##     <authors><![CDATA[Tobar, F.A.;  Mandic, D.P.]]></authors>
##     <affiliations><![CDATA[Dept. of Electr. & Electron. Eng, Imperial Coll. London, London, UK]]></affiliations>
##     <controlledterms>
##       <term><![CDATA[Gaussian processes]]></term>
##       <term><![CDATA[Hilbert spaces]]></term>
##       <term><![CDATA[computational complexity]]></term>
##       <term><![CDATA[learning (artificial intelligence)]]></term>
##       <term><![CDATA[regression analysis]]></term>
##     </controlledterms>
##     <thesaurusterms>
##       <term><![CDATA[Estimation]]></term>
##       <term><![CDATA[Hafnium]]></term>
##       <term><![CDATA[Hilbert space]]></term>
##       <term><![CDATA[Kernel]]></term>
##       <term><![CDATA[Quaternions]]></term>
##       <term><![CDATA[Standards]]></term>
##       <term><![CDATA[Vectors]]></term>
##     </thesaurusterms>
##     <pubtitle><![CDATA[Information Theory, IEEE Transactions on]]></pubtitle>
##     <punumber><![CDATA[18]]></punumber>
##     <pubtype><![CDATA[Journals & Magazines]]></pubtype>
##     <publisher><![CDATA[IEEE]]></publisher>
##     <volume><![CDATA[60]]></volume>
##     <issue><![CDATA[9]]></issue>
##     <py><![CDATA[2014]]></py>
##     <spage><![CDATA[5736]]></spage>
##     <epage><![CDATA[5749]]></epage>
##     <abstract><![CDATA[The existence and uniqueness conditions of quaternion reproducing kernel Hilbert spaces (QRKHS) are established in order to provide a mathematical foundation for the development of quaternion-valued kernel learning algorithms. This is achieved through a rigorous account of left quaternion Hilbert spaces, which makes it possible to generalise standard RKHS to quaternion RKHS. Quaternion versions of the Riesz representation and Moore-Aronszajn theorems are next introduced, thus underpinning kernel estimation algorithms operating on quaternion-valued feature spaces. The difference between the proposed quaternion kernel concept and the existing real and vector approaches is also established in terms of both theoretical advantages and computational complexity. The enhanced estimation ability of the so-introduced quaternion-valued kernels over their real- and vector-valued counterparts is validated through kernel ridge regression applications. Simulations on real world 3D inertial body sensor data and nonlinear channel equalisation using novel quaternion cubic and Gaussian kernels support the approach.]]></abstract>
##     <issn><![CDATA[0018-9448]]></issn>
##     <htmlFlag><![CDATA[1]]></htmlFlag>
##     <arnumber><![CDATA[6846303]]></arnumber>
##     <doi><![CDATA[10.1109/TIT.2014.2333734]]></doi>
##     <publicationId><![CDATA[6846303]]></publicationId>
##     <mdurl><![CDATA[http://ieeexplore.ieee.org/xpl/articleDetails.jsp?tp=&arnumber=6846303&contentType=Journals+%26+Magazines]]></mdurl>
##     <pdf><![CDATA[http://ieeexplore.ieee.org/stamp/stamp.jsp?arnumber=6846303]]></pdf>
##   </document>
##   <document>
##     <rank>1163</rank>
##     <title><![CDATA[Junctionless Silicon Nanowire Resonator]]></title>
##     <authors><![CDATA[Bartsch, S.T.;  Arp, M.;  Ionescu, A.M.]]></authors>
##     <affiliations><![CDATA[Sch. of Eng., Swiss Fed. Inst. of Technol., Lausanne, Switzerland]]></affiliations>
##     <controlledterms>
##       <term><![CDATA[field effect transistors]]></term>
##       <term><![CDATA[nanoelectromechanical devices]]></term>
##       <term><![CDATA[nanowires]]></term>
##       <term><![CDATA[piezoresistive devices]]></term>
##       <term><![CDATA[resonators]]></term>
##     </controlledterms>
##     <thesaurusterms>
##       <term><![CDATA[Logic gates]]></term>
##       <term><![CDATA[Modulation]]></term>
##       <term><![CDATA[Nanoelectromechanical systems]]></term>
##       <term><![CDATA[Piezoresistance]]></term>
##       <term><![CDATA[Resonant frequency]]></term>
##       <term><![CDATA[Silicon]]></term>
##       <term><![CDATA[Transistors]]></term>
##     </thesaurusterms>
##     <pubtitle><![CDATA[Electron Devices Society, IEEE Journal of the]]></pubtitle>
##     <punumber><![CDATA[6245494]]></punumber>
##     <pubtype><![CDATA[Journals & Magazines]]></pubtype>
##     <publisher><![CDATA[IEEE]]></publisher>
##     <volume><![CDATA[2]]></volume>
##     <issue><![CDATA[2]]></issue>
##     <py><![CDATA[2014]]></py>
##     <spage><![CDATA[8]]></spage>
##     <epage><![CDATA[15]]></epage>
##     <abstract><![CDATA[The development of nanoelectromechanical systems (NEMS) is likely to open up a broad spectrum of applications in science and technology. In this paper, we demonstrate a novel double-transduction principle for silicon nanowire resonators, which exploits the depletion charge modulation in a junctionless field effect transistor body and the piezoresistive modulation. A mechanical resonance at the very high frequency of 100 MHz is detected in the drain current of the highly doped silicon wire with a cross-section down to ~ 30 nm. We show that the depletion charge modulation provides a ~ 35 dB increase in output signal-to-noise compared to the second-order piezoresistive detection, which can be separately investigated within the same device. The proposed junctionless resonator stands, therefore, as a unique and valuable tool for comparing the field effect and the piezoresistive modulation efficiency in the same structure, depending on size and doping. The experimental frequency stability of 10 ppm translates into an estimated mass detection noise floor of ~ 60 kDa at a few seconds integration time in high vacuum and at room temperature. Integrated with conventional semiconductor technology, this device offers new opportunities for NEMS-based sensor and signal processing systems hybridized with CMOS circuitry on a single chip.]]></abstract>
##     <issn><![CDATA[2168-6734]]></issn>
##     <htmlFlag><![CDATA[1]]></htmlFlag>
##     <arnumber><![CDATA[6687248]]></arnumber>
##     <doi><![CDATA[10.1109/JEDS.2013.2295246]]></doi>
##     <publicationId><![CDATA[6687248]]></publicationId>
##     <mdurl><![CDATA[http://ieeexplore.ieee.org/xpl/articleDetails.jsp?tp=&arnumber=6687248&contentType=Journals+%26+Magazines]]></mdurl>
##     <pdf><![CDATA[http://ieeexplore.ieee.org/stamp/stamp.jsp?arnumber=6687248]]></pdf>
##   </document>
##   <document>
##     <rank>1164</rank>
##     <title><![CDATA[A Unified Model for the Design and Analysis of Spatially-Correlated Load-Aware HetNets]]></title>
##     <authors><![CDATA[Shojaeifard, A.;  Hamdi, K.A.;  Alsusa, E.;  So, D.K.C.;  Jie Tang]]></authors>
##     <affiliations><![CDATA[Microwave & Commun. Syst. (MCS) Res. Group, Univ. of Manchester, Manchester, UK]]></affiliations>
##     <controlledterms>
##       <term><![CDATA[Monte Carlo methods]]></term>
##       <term><![CDATA[Nakagami channels]]></term>
##       <term><![CDATA[Rayleigh channels]]></term>
##       <term><![CDATA[cellular radio]]></term>
##       <term><![CDATA[stochastic processes]]></term>
##     </controlledterms>
##     <thesaurusterms>
##       <term><![CDATA[Aggregates]]></term>
##       <term><![CDATA[Analytical models]]></term>
##       <term><![CDATA[Fading]]></term>
##       <term><![CDATA[Interference]]></term>
##       <term><![CDATA[Load modeling]]></term>
##       <term><![CDATA[Shadow mapping]]></term>
##       <term><![CDATA[Stochastic processes]]></term>
##     </thesaurusterms>
##     <pubtitle><![CDATA[Communications, IEEE Transactions on]]></pubtitle>
##     <punumber><![CDATA[26]]></punumber>
##     <pubtype><![CDATA[Journals & Magazines]]></pubtype>
##     <publisher><![CDATA[IEEE]]></publisher>
##     <volume><![CDATA[62]]></volume>
##     <issue><![CDATA[11]]></issue>
##     <py><![CDATA[2014]]></py>
##     <spage><![CDATA[4110]]></spage>
##     <epage><![CDATA[4125]]></epage>
##     <abstract><![CDATA[We develop a unified framework for the performance analysis of arbitrary-loaded downlink heterogeneous networks (HetNets) in which interfering sources are inherently spatially-correlated. Considering a randomly-deployed multi-tier cellular network comprised of a diverse set of large-and small-cells, we incorporate the notion of load-awareness and spatial-correlations in characterizing the activities of base stations (BSs) using binary decision variables. A stochastic geometry-based approach is accordingly employed to systematically develop a bounded expression of ergodic rate with different cellular association and load-balancing strategies. Employing the proposed unified framework hence allows for relaxation of several major limitations in the existing state-of-the-art models, in particular the always-transmitting-BSs, uncorrelated interferers, and Rayleigh fading assumptions. We elaborate on the usefulness of adopting this methodology by providing detailed analysis of the aggregate network interference generated by interdependent load-proportional sources over Nakagami-m fading interfering channels. The analytical formulations are validated through Monte-Carlo (MC) simulations for various scenarios and system settings of interest. We observe that the heavily-adopted fully-loaded model as well as the more recent interference-thinning-based approximations are significantly limited in capturing the actual performance curve. The proposed bounded load-aware model and MC trials reveal several important trends and design guidelines for the practical deployment of HetNets.]]></abstract>
##     <issn><![CDATA[0090-6778]]></issn>
##     <htmlFlag><![CDATA[1]]></htmlFlag>
##     <arnumber><![CDATA[6918448]]></arnumber>
##     <doi><![CDATA[10.1109/TCOMM.2014.2361758]]></doi>
##     <publicationId><![CDATA[6918448]]></publicationId>
##     <mdurl><![CDATA[http://ieeexplore.ieee.org/xpl/articleDetails.jsp?tp=&arnumber=6918448&contentType=Journals+%26+Magazines]]></mdurl>
##     <pdf><![CDATA[http://ieeexplore.ieee.org/stamp/stamp.jsp?arnumber=6918448]]></pdf>
##   </document>
##   <document>
##     <rank>1165</rank>
##     <title><![CDATA[An Analogue Front-End Model for Developing Neural Spike Sorting Systems]]></title>
##     <authors><![CDATA[Barsakcioglu, D.Y.;  Yan Liu;  Bhunjun, P.;  Navajas, J.;  Eftekhar, A.;  Jackson, A.;  Quian Quiroga, R.;  Constandinou, T.G.]]></authors>
##     <affiliations><![CDATA[Dept. of Electr. & Electron. Eng., Imperial Coll. London, London, UK]]></affiliations>
##     <controlledterms>
##       <term><![CDATA[bioelectric potentials]]></term>
##       <term><![CDATA[filters]]></term>
##       <term><![CDATA[medical signal detection]]></term>
##       <term><![CDATA[medical signal processing]]></term>
##       <term><![CDATA[neural nets]]></term>
##       <term><![CDATA[neurophysiology]]></term>
##       <term><![CDATA[noise]]></term>
##       <term><![CDATA[sorting]]></term>
##     </controlledterms>
##     <thesaurusterms>
##       <term><![CDATA[Accuracy]]></term>
##       <term><![CDATA[Electrodes]]></term>
##       <term><![CDATA[Integrated circuit modeling]]></term>
##       <term><![CDATA[Mathematical model]]></term>
##       <term><![CDATA[Neurons]]></term>
##       <term><![CDATA[Noise]]></term>
##       <term><![CDATA[Sorting]]></term>
##     </thesaurusterms>
##     <pubtitle><![CDATA[Biomedical Circuits and Systems, IEEE Transactions on]]></pubtitle>
##     <punumber><![CDATA[4156126]]></punumber>
##     <pubtype><![CDATA[Journals & Magazines]]></pubtype>
##     <publisher><![CDATA[IEEE]]></publisher>
##     <volume><![CDATA[8]]></volume>
##     <issue><![CDATA[2]]></issue>
##     <py><![CDATA[2014]]></py>
##     <spage><![CDATA[216]]></spage>
##     <epage><![CDATA[227]]></epage>
##     <abstract><![CDATA[In spike sorting systems, front-end electronics is a crucial pre-processing step that not only has a direct impact on detection and sorting accuracy, but also on power and silicon area. In this work, a behavioural front-end model is proposed to assess the impact of the design parameters (including signal-to-noise ratio, filter type/order, bandwidth, converter resolution/rate) on subsequent spike processing. Initial validation of the model is provided by applying a test stimulus to a hardware platform and comparing the measured circuit response to the expected from the behavioural model. Our model is then used to demonstrate the effect of the Analogue Front-End (AFE) on subsequent spike processing by testing established spike detection and sorting methods on a selection of systems reported in the literature. It is revealed that although these designs have a wide variation in design parameters (and thus also circuit complexity), the ultimate impact on spike processing performance is relatively low (10-15%). This can be used to inform the design of future systems to have an efficient AFE whilst also maintaining good processing performance.]]></abstract>
##     <issn><![CDATA[1932-4545]]></issn>
##     <htmlFlag><![CDATA[1]]></htmlFlag>
##     <arnumber><![CDATA[6807520]]></arnumber>
##     <doi><![CDATA[10.1109/TBCAS.2014.2313087]]></doi>
##     <publicationId><![CDATA[6807520]]></publicationId>
##     <mdurl><![CDATA[http://ieeexplore.ieee.org/xpl/articleDetails.jsp?tp=&arnumber=6807520&contentType=Journals+%26+Magazines]]></mdurl>
##     <pdf><![CDATA[http://ieeexplore.ieee.org/stamp/stamp.jsp?arnumber=6807520]]></pdf>
##   </document>
##   <document>
##     <rank>1166</rank>
##     <title><![CDATA[Characterization of Integrated Optical Strain Sensors Based on Silicon Waveguides]]></title>
##     <authors><![CDATA[Westerveld, W.J.;  Leinders, S.M.;  Muilwijk, P.M.;  Pozo, J.;  van den Dool, T.C.;  Verweij, M.D.;  Yousefi, M.;  Urbach, H.P.]]></authors>
##     <affiliations><![CDATA[Opt. Res. Group, Delft Univ. of Technol., Delft, Netherlands]]></affiliations>
##     <controlledterms>
##       <term><![CDATA[elemental semiconductors]]></term>
##       <term><![CDATA[integrated optics]]></term>
##       <term><![CDATA[micro-optics]]></term>
##       <term><![CDATA[micromechanical resonators]]></term>
##       <term><![CDATA[microsensors]]></term>
##       <term><![CDATA[optical resonators]]></term>
##       <term><![CDATA[optical sensors]]></term>
##       <term><![CDATA[optical waveguides]]></term>
##       <term><![CDATA[silicon]]></term>
##       <term><![CDATA[strain sensors]]></term>
##     </controlledterms>
##     <thesaurusterms>
##       <term><![CDATA[Indexes]]></term>
##       <term><![CDATA[Optical ring resonators]]></term>
##       <term><![CDATA[Optical waveguides]]></term>
##       <term><![CDATA[Sensors]]></term>
##       <term><![CDATA[Silicon]]></term>
##       <term><![CDATA[Strain]]></term>
##     </thesaurusterms>
##     <pubtitle><![CDATA[Selected Topics in Quantum Electronics, IEEE Journal of]]></pubtitle>
##     <punumber><![CDATA[2944]]></punumber>
##     <pubtype><![CDATA[Journals & Magazines]]></pubtype>
##     <publisher><![CDATA[IEEE]]></publisher>
##     <volume><![CDATA[20]]></volume>
##     <issue><![CDATA[4]]></issue>
##     <py><![CDATA[2014]]></py>
##     <spage><![CDATA[101]]></spage>
##     <epage><![CDATA[110]]></epage>
##     <abstract><![CDATA[Microscale strain gauges are widely used in micro electro-mechanical systems (MEMS) to measure strains such as those induced by force, acceleration, pressure or sound. We propose all-optical strain sensors based on micro-ring resonators to be integrated with MEMS. We characterized the strain-induced shift of the resonances of such devices. Depending on the width of the waveguide and the orientation of the silicon crystal, the linear wavelength shift per applied strain varies between 0.5 and 0.75 pm/microstrain for infrared light around 1550 nm wavelength. The influence of the increasing ring circumference is about three times larger than the influence of the change in waveguide effective index, and the two effects oppose each other. The strong dispersion in 220 nm high silicon sub-wavelength waveguides accounts for a decrease in sensitivity of a factor 2.2 to 1.4 for waveguide widths of 310 nm to 860 nm. These figures and insights are necessary for the design of strain sensors based on silicon waveguides.]]></abstract>
##     <issn><![CDATA[1077-260X]]></issn>
##     <htmlFlag><![CDATA[1]]></htmlFlag>
##     <arnumber><![CDATA[6657704]]></arnumber>
##     <doi><![CDATA[10.1109/JSTQE.2013.2289992]]></doi>
##     <publicationId><![CDATA[6657704]]></publicationId>
##     <mdurl><![CDATA[http://ieeexplore.ieee.org/xpl/articleDetails.jsp?tp=&arnumber=6657704&contentType=Journals+%26+Magazines]]></mdurl>
##     <pdf><![CDATA[http://ieeexplore.ieee.org/stamp/stamp.jsp?arnumber=6657704]]></pdf>
##   </document>
##   <document>
##     <rank>1167</rank>
##     <title><![CDATA[Spatial Scale and Landscape Heterogeneity Effects on FAPAR in an Open-Canopy Black Spruce Forest in Interior Alaska]]></title>
##     <authors><![CDATA[Kobayashi, H.;  Suzuki, R.;  Nagai, S.;  Nakai, T.;  Yongwon Kim]]></authors>
##     <affiliations><![CDATA[Res. Inst. for Global Change (RIGC), Japan Agency for Marine-Earth Sci. & Technol. (JAMSTEC), Yokosuka, Japan]]></affiliations>
##     <controlledterms>
##       <term><![CDATA[ecology]]></term>
##       <term><![CDATA[forestry]]></term>
##       <term><![CDATA[geomorphology]]></term>
##       <term><![CDATA[photosynthesis]]></term>
##       <term><![CDATA[vegetation]]></term>
##       <term><![CDATA[vegetation mapping]]></term>
##     </controlledterms>
##     <thesaurusterms>
##       <term><![CDATA[Indexes]]></term>
##       <term><![CDATA[Needles]]></term>
##       <term><![CDATA[Remote sensing]]></term>
##       <term><![CDATA[Solid modeling]]></term>
##       <term><![CDATA[Spatial resolution]]></term>
##       <term><![CDATA[Three-dimensional displays]]></term>
##       <term><![CDATA[Vegetation]]></term>
##     </thesaurusterms>
##     <pubtitle><![CDATA[Geoscience and Remote Sensing Letters, IEEE]]></pubtitle>
##     <punumber><![CDATA[8859]]></punumber>
##     <pubtype><![CDATA[Journals & Magazines]]></pubtype>
##     <publisher><![CDATA[IEEE]]></publisher>
##     <volume><![CDATA[11]]></volume>
##     <issue><![CDATA[2]]></issue>
##     <py><![CDATA[2014]]></py>
##     <spage><![CDATA[564]]></spage>
##     <epage><![CDATA[568]]></epage>
##     <abstract><![CDATA[Black spruce forests dominate the land cover in interior Alaska. In this region, satellite remote sensing of ecosystem productivity is useful for evaluating black spruce forest status and recovery processes. The fraction of absorbed photosynthetically active radiation (FAPAR) by green leaves is a particularly important input parameter for ecosystem models. FAPAR<sub>1d</sub> is computed as the ratio of absorbed photosynthetically active radiation (APAR<sub>3d</sub>) to the incident photosynthetically active radiation at the horizontal plane above the canopy (PAR<sub>1d</sub>, FAPAR<sub>1d</sub> = APAR<sub>3d</sub>/PAR<sub>1d</sub>). The parameter FAPAR<sub>1d</sub> is scale dependent and can be larger than 1 as a result of laterally incident PAR. We investigated the dependence of FAPAR<sub>1d</sub> on spatial scale in an open-canopy black spruce forest in interior Alaska. We compared FAPAR<sub>1d</sub> with FAPAR<sub>3d</sub>( = APAR<sub>3d</sub>/PAR<sub>3d</sub>), the latter of which considers incident PAR as actinic flux (spheradiance) (PAR<sub>3d</sub>). Our results showed the following: 1) landscape scale FAPAR<sub>3d</sub>(30&#x00D7;30 m<sup>2</sup>) was always larger (0.39-0.43) than FAPAR<sub>1d</sub> (0.19-0.27) due to the landscape heterogeneity and incident PAR regime, and 2) at the individual tree scale, FAPAR<sub>1d</sub> was highly variable, with 34% (day of year [DOY] 180) to 52% (DOY 258) of , whereas FAPAR<sub>3d</sub> varied across a much narrower range (0.2-0.5). The spatial-scale dependence of the ratio of PAR<sub>3d</sub> to PAR<sub>1d</sub> converged at the pixel size larger than 5 m. Thus, a 5-m or coarser resolution was necessary to ignore the lateral PAR effect in the open-canopy black spruce forest.]]></abstract>
##     <issn><![CDATA[1545-598X]]></issn>
##     <htmlFlag><![CDATA[1]]></htmlFlag>
##     <arnumber><![CDATA[6589947]]></arnumber>
##     <doi><![CDATA[10.1109/LGRS.2013.2278426]]></doi>
##     <publicationId><![CDATA[6589947]]></publicationId>
##     <mdurl><![CDATA[http://ieeexplore.ieee.org/xpl/articleDetails.jsp?tp=&arnumber=6589947&contentType=Journals+%26+Magazines]]></mdurl>
##     <pdf><![CDATA[http://ieeexplore.ieee.org/stamp/stamp.jsp?arnumber=6589947]]></pdf>
##   </document>
##   <document>
##     <rank>1168</rank>
##     <title><![CDATA[Highly Efficient Phase-Matched Third Harmonic Generation From Mid-IR to Near-IR Regions Using an Asymmetric Plasmonic Slot Waveguide]]></title>
##     <authors><![CDATA[Wu, T.;  Sun, Y.;  Shao, X.;  Shum, P.P.;  Huang, T.]]></authors>
##     <thesaurusterms>
##       <term><![CDATA[Fiber optics]]></term>
##       <term><![CDATA[Optical fiber communication]]></term>
##       <term><![CDATA[Optical harmonic generation]]></term>
##       <term><![CDATA[Optical signal processing]]></term>
##       <term><![CDATA[Optical waveguides]]></term>
##       <term><![CDATA[Plasmons]]></term>
##     </thesaurusterms>
##     <pubtitle><![CDATA[Photonics Journal, IEEE]]></pubtitle>
##     <punumber><![CDATA[4563994]]></punumber>
##     <pubtype><![CDATA[Journals & Magazines]]></pubtype>
##     <publisher><![CDATA[IEEE]]></publisher>
##     <volume><![CDATA[6]]></volume>
##     <issue><![CDATA[5]]></issue>
##     <py><![CDATA[2014]]></py>
##     <spage><![CDATA[1]]></spage>
##     <epage><![CDATA[9]]></epage>
##     <abstract><![CDATA[We theoretically propose an asymmetric plasmonic slot waveguide (APSW) with a thin silicon nanocrystal (Si-nc) layer and a thinner silicon layer that fills the bottom and the top of the metallic slot region to increase third-harmonic generation (THG) efficiency. The needed modal phase-matching condition is satisfied with properly mode dispersion engineering by exploiting the waveguide geometrical parameters. Combining the asymmetric waveguide structure and the high third-order susceptibility of the nonlinear materials, efficient phase-matched THG from mid-infrared (IR) to near-IR regions is realized. Then, the THG performance was further improved by increasing the silicon slot width wherein the THG nonlinear coefficient as large as 21452 <inline-formula> <img src="/images/tex/23887.gif" alt="\hbox {m}^{-1} \cdot \hbox {W}^{-1}"> </inline-formula> is achieved. The corresponding THG efficiency comes up to 0.5% at a waveguide length of 9.3 <inline-formula> <img src="/images/tex/527.gif" alt="\mu\hbox {m}"> </inline-formula> with a pump power of 1 W.]]></abstract>
##     <issn><![CDATA[1943-0655]]></issn>
##     <htmlFlag><![CDATA[1]]></htmlFlag>
##     <arnumber><![CDATA[6895149]]></arnumber>
##     <doi><![CDATA[10.1109/JPHOT.2014.2356494]]></doi>
##     <publicationId><![CDATA[6895149]]></publicationId>
##     <mdurl><![CDATA[http://ieeexplore.ieee.org/xpl/articleDetails.jsp?tp=&arnumber=6895149&contentType=Journals+%26+Magazines]]></mdurl>
##     <pdf><![CDATA[http://ieeexplore.ieee.org/stamp/stamp.jsp?arnumber=6895149]]></pdf>
##   </document>
##   <document>
##     <rank>1169</rank>
##     <title><![CDATA[Experimental Indicators of Current Unbalance in Building-Integrated Photovoltaic Systems]]></title>
##     <authors><![CDATA[Chicco, G.;  Corona, F.;  Porumb, R.;  Spertino, F.]]></authors>
##     <affiliations><![CDATA[Dept. of Energy, Politec. di Torino, Turin, Italy]]></affiliations>
##     <controlledterms>
##       <term><![CDATA[building integrated photovoltaics]]></term>
##       <term><![CDATA[harmonic distortion]]></term>
##       <term><![CDATA[photovoltaic effects]]></term>
##       <term><![CDATA[solar cell arrays]]></term>
##       <term><![CDATA[transformers]]></term>
##     </controlledterms>
##     <thesaurusterms>
##       <term><![CDATA[Harmonic analysis]]></term>
##       <term><![CDATA[Harmonic distortion]]></term>
##       <term><![CDATA[Inverters]]></term>
##       <term><![CDATA[Phase distortion]]></term>
##       <term><![CDATA[Photovoltaic systems]]></term>
##       <term><![CDATA[Standards]]></term>
##     </thesaurusterms>
##     <pubtitle><![CDATA[Photovoltaics, IEEE Journal of]]></pubtitle>
##     <punumber><![CDATA[5503869]]></punumber>
##     <pubtype><![CDATA[Journals & Magazines]]></pubtype>
##     <publisher><![CDATA[IEEE]]></publisher>
##     <volume><![CDATA[4]]></volume>
##     <issue><![CDATA[3]]></issue>
##     <py><![CDATA[2014]]></py>
##     <spage><![CDATA[924]]></spage>
##     <epage><![CDATA[934]]></epage>
##     <abstract><![CDATA[Unbalance of the three-phase currents in photovoltaic (PV) systems may depend on structural aspects of the installation, the effect of partial shading, or both. In this paper, a number of unbalance indicators are calculated starting from data that are measured during experimental analyses on a real building-integrated PV system that represents different types of unbalance. Detailed information is obtained from indices that identify the balance and unbalance components that are also in the presence of waveform distortion. These indices extend the current definitions of unbalance given in the power quality standards. The results show that the unbalance cannot be considered negligible, even with no single-phase inverters and is more significant if nonlinear loads add a contribution to both harmonic distortion and unbalance seen from the distribution transformer.]]></abstract>
##     <issn><![CDATA[2156-3381]]></issn>
##     <htmlFlag><![CDATA[1]]></htmlFlag>
##     <arnumber><![CDATA[6766200]]></arnumber>
##     <doi><![CDATA[10.1109/JPHOTOV.2014.2307491]]></doi>
##     <publicationId><![CDATA[6766200]]></publicationId>
##     <mdurl><![CDATA[http://ieeexplore.ieee.org/xpl/articleDetails.jsp?tp=&arnumber=6766200&contentType=Journals+%26+Magazines]]></mdurl>
##     <pdf><![CDATA[http://ieeexplore.ieee.org/stamp/stamp.jsp?arnumber=6766200]]></pdf>
##   </document>
##   <document>
##     <rank>1170</rank>
##     <title><![CDATA[Reconfigurable Waveguide for Vector Network Analyzer Verification]]></title>
##     <authors><![CDATA[Papantonis, S.;  Ridler, N.M.;  Wilson, A.;  Lucyszyn, S.]]></authors>
##     <affiliations><![CDATA[Opt. & Semicond. Devices Group, Imperial Coll. London, London, UK]]></affiliations>
##     <controlledterms>
##       <term><![CDATA[calibration]]></term>
##       <term><![CDATA[millimetre wave measurement]]></term>
##       <term><![CDATA[network analysers]]></term>
##       <term><![CDATA[rectangular waveguides]]></term>
##     </controlledterms>
##     <thesaurusterms>
##       <term><![CDATA[Calibration]]></term>
##       <term><![CDATA[Coordinate measuring machines]]></term>
##       <term><![CDATA[Pins]]></term>
##       <term><![CDATA[Standards]]></term>
##       <term><![CDATA[Vectors]]></term>
##       <term><![CDATA[Waveguide components]]></term>
##     </thesaurusterms>
##     <pubtitle><![CDATA[Microwave Theory and Techniques, IEEE Transactions on]]></pubtitle>
##     <punumber><![CDATA[22]]></punumber>
##     <pubtype><![CDATA[Journals & Magazines]]></pubtype>
##     <publisher><![CDATA[IEEE]]></publisher>
##     <volume><![CDATA[62]]></volume>
##     <issue><![CDATA[10]]></issue>
##     <py><![CDATA[2014]]></py>
##     <spage><![CDATA[2415]]></spage>
##     <epage><![CDATA[2422]]></epage>
##     <abstract><![CDATA[A novel simple approach to the verification process for millimeter-wave vector network analyzer waveguide calibration is reported using a single reconfigurable component verification kit. Conventional techniques require multiple verification components and these only exist commercially for operation up to 110 GHz. At millimeter-wave frequencies, the use of multiple components can lead to significant errors due to imperfections in waveguide flange misalignments during the multiple component connections. The reconfigurable component is designed so that its electrical properties can be changed quickly to a broad range of predetermined values without introducing additional errors due to changes in flange alignment. Once connected, the component can be reconfigured to introduce relative changes in the reflected and transmitted signals. For millimeter-wave metrology, where mechanical precision is of paramount importance, this single-component verification approach represents an attractive solution. A proof-of-concept verification process is described, based on full-wave electromagnetic modeling, hardware implementation, and validation measurements using standard WR-15 waveguide (50-75 GHz).]]></abstract>
##     <issn><![CDATA[0018-9480]]></issn>
##     <htmlFlag><![CDATA[1]]></htmlFlag>
##     <arnumber><![CDATA[6876052]]></arnumber>
##     <doi><![CDATA[10.1109/TMTT.2014.2344625]]></doi>
##     <publicationId><![CDATA[6876052]]></publicationId>
##     <mdurl><![CDATA[http://ieeexplore.ieee.org/xpl/articleDetails.jsp?tp=&arnumber=6876052&contentType=Journals+%26+Magazines]]></mdurl>
##     <pdf><![CDATA[http://ieeexplore.ieee.org/stamp/stamp.jsp?arnumber=6876052]]></pdf>
##   </document>
##   <document>
##     <rank>1171</rank>
##     <title><![CDATA[The Generalized Loneliness Detector and Weak System Models for k-Set Agreement]]></title>
##     <authors><![CDATA[Biely, M.;  Robinson, P.;  Schmid, U.]]></authors>
##     <affiliations><![CDATA[IC IIF LSR, EPFL, Lausanne, Switzerland]]></affiliations>
##     <controlledterms>
##       <term><![CDATA[message passing]]></term>
##       <term><![CDATA[set theory]]></term>
##     </controlledterms>
##     <thesaurusterms>
##       <term><![CDATA[Biological system modeling]]></term>
##       <term><![CDATA[Computational modeling]]></term>
##       <term><![CDATA[Computer crashes]]></term>
##       <term><![CDATA[Delays]]></term>
##       <term><![CDATA[Detectors]]></term>
##       <term><![CDATA[Electronic mail]]></term>
##       <term><![CDATA[Message passing]]></term>
##     </thesaurusterms>
##     <pubtitle><![CDATA[Parallel and Distributed Systems, IEEE Transactions on]]></pubtitle>
##     <punumber><![CDATA[71]]></punumber>
##     <pubtype><![CDATA[Journals & Magazines]]></pubtype>
##     <publisher><![CDATA[IEEE]]></publisher>
##     <volume><![CDATA[25]]></volume>
##     <issue><![CDATA[4]]></issue>
##     <py><![CDATA[2014]]></py>
##     <spage><![CDATA[1078]]></spage>
##     <epage><![CDATA[1088]]></epage>
##     <abstract><![CDATA[This paper presents two weak partially synchronous system models M<sup>anti(n-k)</sup> and M<sup>sink(n-k)</sup>, which are just strong enough for solving k-set agreement: We introduce the generalized (n-k)-loneliness failure detector L(k), which we first prove to be sufficient for solving k-set agreement, and show that L(k) but not L(k-1) can be implemented in both models. M<sup>anti(n-k)</sup> and M<sup>sink(n-k)</sup> are hence the first message passing models that lie between models where &#x03A9; (and therefore consensus) can be implemented and the purely asynchronous model. We also address k-set agreement in anonymous systems, that is, in systems where (unique) process identifiers are not available. Since our novel k -set agreement algorithm using L(k) also works in anonymous systems, it turns out that the loneliness failure detector L=L(n-1) introduced by Delporte et al. is also the weakest failure detector for set agreement in anonymous systems. Finally, we analyze the relationship between L(k) and other failure detectors suitable for solving k-set agreement.]]></abstract>
##     <issn><![CDATA[1045-9219]]></issn>
##     <htmlFlag><![CDATA[1]]></htmlFlag>
##     <arnumber><![CDATA[6482555]]></arnumber>
##     <doi><![CDATA[10.1109/TPDS.2013.77]]></doi>
##     <publicationId><![CDATA[6482555]]></publicationId>
##     <mdurl><![CDATA[http://ieeexplore.ieee.org/xpl/articleDetails.jsp?tp=&arnumber=6482555&contentType=Journals+%26+Magazines]]></mdurl>
##     <pdf><![CDATA[http://ieeexplore.ieee.org/stamp/stamp.jsp?arnumber=6482555]]></pdf>
##   </document>
##   <document>
##     <rank>1172</rank>
##     <title><![CDATA[Extraction and Estimation of Pinned Photodiode Capacitance in CMOS Image Sensors]]></title>
##     <authors><![CDATA[Chao, C.Y.-P.;  Yi-Che Chen;  Kuo-Yu Chou;  Jhy-Jyi Sze;  Fu-Lung Hsueh;  Shou-Gwo Wuu]]></authors>
##     <affiliations><![CDATA[Taiwan Semicond. Manuf. Co., Hsinchu, Taiwan]]></affiliations>
##     <controlledterms>
##       <term><![CDATA[CMOS image sensors]]></term>
##       <term><![CDATA[photodetectors]]></term>
##       <term><![CDATA[photodiodes]]></term>
##     </controlledterms>
##     <thesaurusterms>
##       <term><![CDATA[CMOS image sensors]]></term>
##       <term><![CDATA[Capacitance]]></term>
##       <term><![CDATA[Educational institutions]]></term>
##       <term><![CDATA[Photodiodes]]></term>
##       <term><![CDATA[Photonics]]></term>
##       <term><![CDATA[Voltage measurement]]></term>
##     </thesaurusterms>
##     <pubtitle><![CDATA[Electron Devices Society, IEEE Journal of the]]></pubtitle>
##     <punumber><![CDATA[6245494]]></punumber>
##     <pubtype><![CDATA[Journals & Magazines]]></pubtype>
##     <publisher><![CDATA[IEEE]]></publisher>
##     <volume><![CDATA[2]]></volume>
##     <issue><![CDATA[4]]></issue>
##     <py><![CDATA[2014]]></py>
##     <spage><![CDATA[59]]></spage>
##     <epage><![CDATA[64]]></epage>
##     <abstract><![CDATA[The pinned photodiode capacitance extraction method proposed by Goiffon et al. is discussed, and two additional new methods are presented and analyzed; one based on the full well dependence on photon flux and the other based on the full well dependence on transfer-gate off-voltage.]]></abstract>
##     <issn><![CDATA[2168-6734]]></issn>
##     <htmlFlag><![CDATA[1]]></htmlFlag>
##     <arnumber><![CDATA[6799991]]></arnumber>
##     <doi><![CDATA[10.1109/JEDS.2014.2318060]]></doi>
##     <publicationId><![CDATA[6799991]]></publicationId>
##     <mdurl><![CDATA[http://ieeexplore.ieee.org/xpl/articleDetails.jsp?tp=&arnumber=6799991&contentType=Journals+%26+Magazines]]></mdurl>
##     <pdf><![CDATA[http://ieeexplore.ieee.org/stamp/stamp.jsp?arnumber=6799991]]></pdf>
##   </document>
##   <document>
##     <rank>1173</rank>
##     <title><![CDATA[GPU-Based Acceleration for Interior Tomography]]></title>
##     <authors><![CDATA[Rui Liu;  Yan Luo;  Hengyong Yu]]></authors>
##     <affiliations><![CDATA[Dept. of Biomed. Eng., Wake Forest Univ. Health Sci., Winston-Salem, NC, USA]]></affiliations>
##     <controlledterms>
##       <term><![CDATA[compressed sensing]]></term>
##       <term><![CDATA[computational geometry]]></term>
##       <term><![CDATA[computerised tomography]]></term>
##       <term><![CDATA[graphics processing units]]></term>
##       <term><![CDATA[image reconstruction]]></term>
##       <term><![CDATA[iterative methods]]></term>
##       <term><![CDATA[medical image processing]]></term>
##       <term><![CDATA[parallel processing]]></term>
##       <term><![CDATA[piecewise constant techniques]]></term>
##       <term><![CDATA[piecewise polynomial techniques]]></term>
##     </controlledterms>
##     <thesaurusterms>
##       <term><![CDATA[Compressive sensing]]></term>
##       <term><![CDATA[Computed tomography]]></term>
##       <term><![CDATA[Graphics]]></term>
##       <term><![CDATA[Graphics processing unit]]></term>
##       <term><![CDATA[IEEE standards]]></term>
##       <term><![CDATA[Parallel processing]]></term>
##       <term><![CDATA[Three dimensional displays]]></term>
##       <term><![CDATA[Tomography]]></term>
##       <term><![CDATA[X-ray tomography]]></term>
##     </thesaurusterms>
##     <pubtitle><![CDATA[Access, IEEE]]></pubtitle>
##     <punumber><![CDATA[6287639]]></punumber>
##     <pubtype><![CDATA[Journals & Magazines]]></pubtype>
##     <publisher><![CDATA[IEEE]]></publisher>
##     <volume><![CDATA[2]]></volume>
##     <py><![CDATA[2014]]></py>
##     <spage><![CDATA[757]]></spage>
##     <epage><![CDATA[770]]></epage>
##     <abstract><![CDATA[The compressive sensing (CS) theory shows that real signals can be exactly recovered from very few samplings. Inspired by the CS theory, the interior problem in computed tomography is proved uniquely solvable by minimizing the region-of-interest's total variation if the imaging object is piecewise constant or polynomial. This is called CS-based interior tomography. However, the CS-based algorithms require high computational cost due to their iterative nature. In this paper, a graphics processing unit (GPU)-based parallel computing technique is applied to accelerate the CS-based interior reconstruction for practical application in both fan-beam and cone-beam geometries. Our results show that the CS-based interior tomography is able to reconstruct excellent volumetric images with GPU acceleration in a few minutes.]]></abstract>
##     <issn><![CDATA[2169-3536]]></issn>
##     <htmlFlag><![CDATA[1]]></htmlFlag>
##     <arnumber><![CDATA[6857986]]></arnumber>
##     <doi><![CDATA[10.1109/ACCESS.2014.2340372]]></doi>
##     <publicationId><![CDATA[6857986]]></publicationId>
##     <mdurl><![CDATA[http://ieeexplore.ieee.org/xpl/articleDetails.jsp?tp=&arnumber=6857986&contentType=Journals+%26+Magazines]]></mdurl>
##     <pdf><![CDATA[http://ieeexplore.ieee.org/stamp/stamp.jsp?arnumber=6857986]]></pdf>
##   </document>
##   <document>
##     <rank>1174</rank>
##     <title><![CDATA[Predicting the Performance of Cooperative Wireless Networking Schemes With Random Network Coding]]></title>
##     <authors><![CDATA[Jin-Taek Seong;  Heung-No Lee]]></authors>
##     <affiliations><![CDATA[Sch. of Inf. & Commun., Gwangju Inst. of Sci. & Technol., Gwangju, South Korea]]></affiliations>
##     <controlledterms>
##       <term><![CDATA[cooperative communication]]></term>
##       <term><![CDATA[matrix algebra]]></term>
##       <term><![CDATA[network coding]]></term>
##       <term><![CDATA[probability]]></term>
##       <term><![CDATA[wireless channels]]></term>
##     </controlledterms>
##     <thesaurusterms>
##       <term><![CDATA[Base stations]]></term>
##       <term><![CDATA[Broadcasting]]></term>
##       <term><![CDATA[Decoding]]></term>
##       <term><![CDATA[Network coding]]></term>
##       <term><![CDATA[Relays]]></term>
##       <term><![CDATA[Wireless networks]]></term>
##     </thesaurusterms>
##     <pubtitle><![CDATA[Communications, IEEE Transactions on]]></pubtitle>
##     <punumber><![CDATA[26]]></punumber>
##     <pubtype><![CDATA[Journals & Magazines]]></pubtype>
##     <publisher><![CDATA[IEEE]]></publisher>
##     <volume><![CDATA[62]]></volume>
##     <issue><![CDATA[8]]></issue>
##     <py><![CDATA[2014]]></py>
##     <spage><![CDATA[2951]]></spage>
##     <epage><![CDATA[2964]]></epage>
##     <abstract><![CDATA[In this paper, we consider a cooperative wireless network in which there are multiple sources and multiple relays. Owing to unreliable wireless channels, the quality of network links between nodes can vary. This results in the failure of intermediate nodes that generate linear combinations of incoming messages in network coding schemes. We propose an analytical framework to evaluate the recovery performance of source messages at the base station. To this end, we consider a random transmission matrix in which each element of the transmission matrix is processed a random variable, where its distribution is a function of the outage probability. We derive an upper bound for the reconstruction performance, i.e., decoding failure probability and nullity. The proposed framework provides an evaluation tool that enables us to investigate the impact of a large number of sources and relays, as well as the field size of the network codes on system performance.]]></abstract>
##     <issn><![CDATA[0090-6778]]></issn>
##     <htmlFlag><![CDATA[1]]></htmlFlag>
##     <arnumber><![CDATA[6834790]]></arnumber>
##     <doi><![CDATA[10.1109/TCOMM.2014.2330825]]></doi>
##     <publicationId><![CDATA[6834790]]></publicationId>
##     <mdurl><![CDATA[http://ieeexplore.ieee.org/xpl/articleDetails.jsp?tp=&arnumber=6834790&contentType=Journals+%26+Magazines]]></mdurl>
##     <pdf><![CDATA[http://ieeexplore.ieee.org/stamp/stamp.jsp?arnumber=6834790]]></pdf>
##   </document>
##   <document>
##     <rank>1175</rank>
##     <title><![CDATA[Photonic Nanojet Analysis by Spectral Element Method]]></title>
##     <authors><![CDATA[Mahariq, I.;  Kuzuoglu, M.;  Tarman, I.H.;  Kurt, H.]]></authors>
##     <affiliations><![CDATA[Dept. of Electr. & Electron. Eng., TOBB Univ. of Econ. & Technol., Ankara, Turkey]]></affiliations>
##     <controlledterms>
##       <term><![CDATA[electromagnetic wave scattering]]></term>
##       <term><![CDATA[finite difference time-domain analysis]]></term>
##       <term><![CDATA[jets]]></term>
##       <term><![CDATA[nanophotonics]]></term>
##     </controlledterms>
##     <thesaurusterms>
##       <term><![CDATA[Accuracy]]></term>
##       <term><![CDATA[Dielectrics]]></term>
##       <term><![CDATA[Finite difference methods]]></term>
##       <term><![CDATA[Finite element analysis]]></term>
##       <term><![CDATA[Photonics]]></term>
##       <term><![CDATA[Polynomials]]></term>
##     </thesaurusterms>
##     <pubtitle><![CDATA[Photonics Journal, IEEE]]></pubtitle>
##     <punumber><![CDATA[4563994]]></punumber>
##     <pubtype><![CDATA[Journals & Magazines]]></pubtype>
##     <publisher><![CDATA[IEEE]]></publisher>
##     <volume><![CDATA[6]]></volume>
##     <issue><![CDATA[5]]></issue>
##     <py><![CDATA[2014]]></py>
##     <spage><![CDATA[1]]></spage>
##     <epage><![CDATA[14]]></epage>
##     <abstract><![CDATA[Although it is known that the spectral element method (SEM) has both high accuracy and a lower computational cost when compared with finite-element or finite-difference methods, the SEM is not widely utilized in the modeling of boundary value problems in electromagnetics. This paper provides a 2-D formulation of the well-known perfectly-matched-layer approach in the context of the SEM for the frequency-domain electromagnetic problems in which dielectric scatterers are involved. The formulation is then utilized to numerically study photonic nanojets after the demonstration of SEM accuracy in an electromagnetic scattering problem. Interesting cases where unusual results are obtained from scattering dielectric cylinders are reported and discussed in this paper. On the other hand, a finite-difference time-domain method that is widely deployed for investigating photonic nanojets is found to fail in successfully capturing such resonance cases. Sharp resonances are characteristic of high-Q cavities and numerical methods with high accuracy, e.g., the SEM can provide superior performance while exploring such resonators.]]></abstract>
##     <issn><![CDATA[1943-0655]]></issn>
##     <htmlFlag><![CDATA[1]]></htmlFlag>
##     <arnumber><![CDATA[6916995]]></arnumber>
##     <doi><![CDATA[10.1109/JPHOT.2014.2361615]]></doi>
##     <publicationId><![CDATA[6916995]]></publicationId>
##     <mdurl><![CDATA[http://ieeexplore.ieee.org/xpl/articleDetails.jsp?tp=&arnumber=6916995&contentType=Journals+%26+Magazines]]></mdurl>
##     <pdf><![CDATA[http://ieeexplore.ieee.org/stamp/stamp.jsp?arnumber=6916995]]></pdf>
##   </document>
##   <document>
##     <rank>1176</rank>
##     <title><![CDATA[A High-Reliability and Determinacy Architecture for Smart Substation Process-Level Network Based on Cobweb Topology]]></title>
##     <authors><![CDATA[Xiaosheng Liu;  Jiwei Pang;  Liang Zhang;  Dianguo Xu]]></authors>
##     <affiliations><![CDATA[Sch. of Electr. Eng. & Autom., Harbin Inst. of Technol., Harbin, China]]></affiliations>
##     <controlledterms>
##       <term><![CDATA[IEC standards]]></term>
##       <term><![CDATA[data acquisition]]></term>
##       <term><![CDATA[delays]]></term>
##       <term><![CDATA[fault trees]]></term>
##       <term><![CDATA[power system reliability]]></term>
##       <term><![CDATA[substation automation]]></term>
##       <term><![CDATA[substation protection]]></term>
##       <term><![CDATA[telecommunication network topology]]></term>
##     </controlledterms>
##     <thesaurusterms>
##       <term><![CDATA[Communication networks]]></term>
##       <term><![CDATA[Instruments]]></term>
##       <term><![CDATA[Network topology]]></term>
##       <term><![CDATA[Power system reliability]]></term>
##       <term><![CDATA[Reliability]]></term>
##       <term><![CDATA[Substations]]></term>
##       <term><![CDATA[Switches]]></term>
##     </thesaurusterms>
##     <pubtitle><![CDATA[Power Delivery, IEEE Transactions on]]></pubtitle>
##     <punumber><![CDATA[61]]></punumber>
##     <pubtype><![CDATA[Journals & Magazines]]></pubtype>
##     <publisher><![CDATA[IEEE]]></publisher>
##     <volume><![CDATA[29]]></volume>
##     <issue><![CDATA[2]]></issue>
##     <py><![CDATA[2014]]></py>
##     <spage><![CDATA[842]]></spage>
##     <epage><![CDATA[850]]></epage>
##     <abstract><![CDATA[A highly reliable and deterministic process-level communication network is required to guarantee the protection switch control and data acquisition of substation automation systems (SASs), as it involves the important primary equipment in smart substations. The cobweb architecture is an artificial communication network topology based on cobwebs as they occur in nature. This study designs novel single- and dual-network architectures for process-level network for D2-1 typical smart substation based on the architecture of natural cobweb, which has structural properties that have been studied by numerical simulation and reliability theory. To demonstrate the feasibility of the process-level network based on cobweb architecture, fault tree analysis (FTA) is used to assess the reliability of the novel cobweb architecture and other traditional architectures. OPNET Modeler is used to simulate the message communication in the cobweb architecture, where the end-to-end time delay needs to conform to IEC 61850. The results of the theoretical analysis and simulation indicate that a process-level network based on cobweb architecture exhibits excellent reliability and determinacy.]]></abstract>
##     <issn><![CDATA[0885-8977]]></issn>
##     <htmlFlag><![CDATA[1]]></htmlFlag>
##     <arnumber><![CDATA[6737316]]></arnumber>
##     <doi><![CDATA[10.1109/TPWRD.2013.2280763]]></doi>
##     <publicationId><![CDATA[6737316]]></publicationId>
##     <mdurl><![CDATA[http://ieeexplore.ieee.org/xpl/articleDetails.jsp?tp=&arnumber=6737316&contentType=Journals+%26+Magazines]]></mdurl>
##     <pdf><![CDATA[http://ieeexplore.ieee.org/stamp/stamp.jsp?arnumber=6737316]]></pdf>
##   </document>
##   <document>
##     <rank>1177</rank>
##     <title><![CDATA[Reducing Energy Waste for Computers by Human-in-the-Loop Control]]></title>
##     <authors><![CDATA[Munir, S.;  Stankovic, J.A.;  Liang, C.-J.M.;  Shan Lin]]></authors>
##     <affiliations><![CDATA[Dept. of Comput. Sci., Univ. of Virginia, Charlottesville, VA, USA]]></affiliations>
##     <controlledterms>
##       <term><![CDATA[energy conservation]]></term>
##       <term><![CDATA[power aware computing]]></term>
##     </controlledterms>
##     <thesaurusterms>
##       <term><![CDATA[Computer workstations]]></term>
##       <term><![CDATA[Energy management]]></term>
##       <term><![CDATA[Human factors]]></term>
##       <term><![CDATA[IP networks]]></term>
##       <term><![CDATA[Waste management]]></term>
##       <term><![CDATA[Workstations]]></term>
##     </thesaurusterms>
##     <pubtitle><![CDATA[Emerging Topics in Computing, IEEE Transactions on]]></pubtitle>
##     <punumber><![CDATA[6245516]]></punumber>
##     <pubtype><![CDATA[Journals & Magazines]]></pubtype>
##     <publisher><![CDATA[IEEE]]></publisher>
##     <volume><![CDATA[2]]></volume>
##     <issue><![CDATA[4]]></issue>
##     <py><![CDATA[2014]]></py>
##     <spage><![CDATA[448]]></spage>
##     <epage><![CDATA[460]]></epage>
##     <abstract><![CDATA[Although current cyber physical systems (CPSs) act as the bridge between humans and environment, their implementation mostly assumes humans as an external component to the control loops. We use a case study of energy waste on computer workstations to motivate the incorporation of humans into the control loops. The benefits include better response accuracy and timeliness of the CPS systems. However, incorporating humans into tight control loops remains a challenge as it requires understanding complex human behavior. In our case study, we collect empirical data to understand human behavior regarding distractions in computer usage and develop a human-in-the-loop control that can put workstations into sleep by early detection of distraction. Our control loop implements strategies such as an adaptive timeout interval, multilevel sensing, and addressing background processing. Evaluation on multiple subjects show an accuracy of 97.28% in detecting distractions, which cuts the energy waste of computers by 80.19%.]]></abstract>
##     <issn><![CDATA[2168-6750]]></issn>
##     <htmlFlag><![CDATA[1]]></htmlFlag>
##     <arnumber><![CDATA[6595133]]></arnumber>
##     <doi><![CDATA[10.1109/TETC.2013.2281204]]></doi>
##     <publicationId><![CDATA[6595133]]></publicationId>
##     <mdurl><![CDATA[http://ieeexplore.ieee.org/xpl/articleDetails.jsp?tp=&arnumber=6595133&contentType=Journals+%26+Magazines]]></mdurl>
##     <pdf><![CDATA[http://ieeexplore.ieee.org/stamp/stamp.jsp?arnumber=6595133]]></pdf>
##   </document>
##   <document>
##     <rank>1178</rank>
##     <title><![CDATA[A Sub-Critical Barrier Thickness Normally-Off AlGaN/GaN MOS-HEMT]]></title>
##     <authors><![CDATA[Brown, R.;  Macfarlane, D.;  Al-Khalidi, A.;  Xu Li;  Ternent, G.;  Haiping Zhou;  Thayne, I.;  Wasige, E.]]></authors>
##     <affiliations><![CDATA[High Freq. Electron. Group, Univ. of Glasgow, Glasgow, UK]]></affiliations>
##     <controlledterms>
##       <term><![CDATA[III-V semiconductors]]></term>
##       <term><![CDATA[MOSFET]]></term>
##       <term><![CDATA[aluminium compounds]]></term>
##       <term><![CDATA[gallium compounds]]></term>
##       <term><![CDATA[high electron mobility transistors]]></term>
##       <term><![CDATA[plasma CVD]]></term>
##       <term><![CDATA[semiconductor device breakdown]]></term>
##       <term><![CDATA[two-dimensional electron gas]]></term>
##       <term><![CDATA[wide band gap semiconductors]]></term>
##     </controlledterms>
##     <thesaurusterms>
##       <term><![CDATA[Aluminum gallium nitride]]></term>
##       <term><![CDATA[Dielectrics]]></term>
##       <term><![CDATA[Gallium nitride]]></term>
##       <term><![CDATA[HEMTs]]></term>
##       <term><![CDATA[Logic gates]]></term>
##       <term><![CDATA[MODFETs]]></term>
##       <term><![CDATA[Threshold voltage]]></term>
##     </thesaurusterms>
##     <pubtitle><![CDATA[Electron Device Letters, IEEE]]></pubtitle>
##     <punumber><![CDATA[55]]></punumber>
##     <pubtype><![CDATA[Journals & Magazines]]></pubtype>
##     <publisher><![CDATA[IEEE]]></publisher>
##     <volume><![CDATA[35]]></volume>
##     <issue><![CDATA[9]]></issue>
##     <py><![CDATA[2014]]></py>
##     <spage><![CDATA[906]]></spage>
##     <epage><![CDATA[908]]></epage>
##     <abstract><![CDATA[A new high-performance normally-off gallium nitride (GaN)-based metal-oxide-semiconductor high electron mobility transistor that employs an ultrathin subcritical 3 nm thick aluminium gallium nitride (Al<sub>0.25</sub>Ga<sub>0.75</sub>N) barrier layer and relies on an induced two-dimensional electron gas for operation is presented. Single finger devices were fabricated using 10 and 20 nm plasma-enhanced chemical vapor-deposited silicon dioxide (SiO<sub>2</sub>) as the gate dielectric. They demonstrated threshold voltages (V<sub>th</sub>) of 3 and 2 V, and very high maximum drain currents (IDSmax) of over 450 and 650 mA/mm, at a gate voltage (V<sub>GS</sub>) of 6 V, respectively. The proposed device is seen as a building block for future power electronic devices, specifically as the driven device in the cascode configuration that employs GaN-based enhancement-mode and depletion-mode devices.]]></abstract>
##     <issn><![CDATA[0741-3106]]></issn>
##     <htmlFlag><![CDATA[1]]></htmlFlag>
##     <arnumber><![CDATA[6856144]]></arnumber>
##     <doi><![CDATA[10.1109/LED.2014.2334394]]></doi>
##     <publicationId><![CDATA[6856144]]></publicationId>
##     <mdurl><![CDATA[http://ieeexplore.ieee.org/xpl/articleDetails.jsp?tp=&arnumber=6856144&contentType=Journals+%26+Magazines]]></mdurl>
##     <pdf><![CDATA[http://ieeexplore.ieee.org/stamp/stamp.jsp?arnumber=6856144]]></pdf>
##   </document>
##   <document>
##     <rank>1179</rank>
##     <title><![CDATA[Architectural Considerations in the Design of a Superconducting Quantum Annealing Processor]]></title>
##     <authors><![CDATA[Bunyk, P.I.;  Hoskinson, E.M.;  Johnson, M.W.;  Tolkacheva, E.;  Altomare, F.;  Berkley, A.J.;  Harris, R.;  Hilton, J.P.;  Lanting, T.;  Przybysz, A.J.;  Whittaker, J.]]></authors>
##     <affiliations><![CDATA[D-Wave Syst. Inc., Burnaby, BC, Canada]]></affiliations>
##     <controlledterms>
##       <term><![CDATA[SQUIDs]]></term>
##       <term><![CDATA[circuit analysis computing]]></term>
##       <term><![CDATA[digital-analogue conversion]]></term>
##       <term><![CDATA[integrated circuit layout]]></term>
##       <term><![CDATA[program processors]]></term>
##       <term><![CDATA[programmable circuits]]></term>
##       <term><![CDATA[quantum computing]]></term>
##       <term><![CDATA[superconducting processor circuits]]></term>
##     </controlledterms>
##     <thesaurusterms>
##       <term><![CDATA[Annealing]]></term>
##       <term><![CDATA[Couplers]]></term>
##       <term><![CDATA[Couplings]]></term>
##       <term><![CDATA[Hardware]]></term>
##       <term><![CDATA[Network architecture]]></term>
##       <term><![CDATA[Quantum processors]]></term>
##       <term><![CDATA[Topology]]></term>
##     </thesaurusterms>
##     <pubtitle><![CDATA[Applied Superconductivity, IEEE Transactions on]]></pubtitle>
##     <punumber><![CDATA[77]]></punumber>
##     <pubtype><![CDATA[Journals & Magazines]]></pubtype>
##     <publisher><![CDATA[IEEE]]></publisher>
##     <volume><![CDATA[24]]></volume>
##     <issue><![CDATA[4]]></issue>
##     <py><![CDATA[2014]]></py>
##     <spage><![CDATA[1]]></spage>
##     <epage><![CDATA[10]]></epage>
##     <abstract><![CDATA[We have developed a quantum annealing processor, based on an array of tunable coupled rf-SQUID flux qubits, fabricated in a superconducting integrated circuit process. Implementing this type of processor at a scale of 512 qubits and 1472 programmable interqubit couplers and operating at ~ 20 mK has required attention to a number of considerations that one may ignore at the smaller scale of a few dozen or so devices. Here, we discuss some of these considerations, and the delicate balance necessary for the construction of a practical processor that respects the demanding physical requirements imposed by a quantum algorithm. In particular, we will review some of the design tradeoffs at play in the floor planning of the physical layout, driven by the desire to have an algorithmically useful set of interqubit couplers, and the simultaneous need to embed programmable control circuitry into the processor fabric. In this context, we have developed a new ultralow-power embedded superconducting digital-to-analog flux converter (DAC) used to program the processor with zero static power dissipation, optimized to achieve maximum flux storage density per unit area. The 512 single-stage, 3520 two-stage, and 512 three-stage flux DACs are controlled with an XYZ addressing scheme requiring 56 wires. Our estimate of on-chip dissipated energy for worst-case reprogramming of the whole processor is ~ 65 fJ. Several chips based on this architecture have been fabricated and operated successfully at our facility, as well as two outside facilities (see, for example, the recent reporting by Jones).]]></abstract>
##     <issn><![CDATA[1051-8223]]></issn>
##     <htmlFlag><![CDATA[1]]></htmlFlag>
##     <arnumber><![CDATA[6802426]]></arnumber>
##     <doi><![CDATA[10.1109/TASC.2014.2318294]]></doi>
##     <publicationId><![CDATA[6802426]]></publicationId>
##     <mdurl><![CDATA[http://ieeexplore.ieee.org/xpl/articleDetails.jsp?tp=&arnumber=6802426&contentType=Journals+%26+Magazines]]></mdurl>
##     <pdf><![CDATA[http://ieeexplore.ieee.org/stamp/stamp.jsp?arnumber=6802426]]></pdf>
##   </document>
##   <document>
##     <rank>1180</rank>
##     <title><![CDATA[Cold-Start Recommendation Using Bi-Clustering and Fusion for Large-Scale Social Recommender Systems]]></title>
##     <authors><![CDATA[Daqiang Zhang;  Ching-Hsien Hsu;  Min Chen;  Quan Chen;  Naixue Xiong;  Lloret, J.]]></authors>
##     <affiliations><![CDATA[Sch. of Software Eng., Tongji Univ., Shanghai, China]]></affiliations>
##     <controlledterms>
##       <term><![CDATA[cloud computing]]></term>
##       <term><![CDATA[collaborative filtering]]></term>
##       <term><![CDATA[pattern clustering]]></term>
##       <term><![CDATA[recommender systems]]></term>
##       <term><![CDATA[sensor fusion]]></term>
##       <term><![CDATA[social networking (online)]]></term>
##     </controlledterms>
##     <thesaurusterms>
##       <term><![CDATA[Collaboration]]></term>
##       <term><![CDATA[Content management]]></term>
##       <term><![CDATA[Media]]></term>
##       <term><![CDATA[Recommender systems]]></term>
##       <term><![CDATA[Social network services]]></term>
##     </thesaurusterms>
##     <pubtitle><![CDATA[Emerging Topics in Computing, IEEE Transactions on]]></pubtitle>
##     <punumber><![CDATA[6245516]]></punumber>
##     <pubtype><![CDATA[Journals & Magazines]]></pubtype>
##     <publisher><![CDATA[IEEE]]></publisher>
##     <volume><![CDATA[2]]></volume>
##     <issue><![CDATA[2]]></issue>
##     <py><![CDATA[2014]]></py>
##     <spage><![CDATA[239]]></spage>
##     <epage><![CDATA[250]]></epage>
##     <abstract><![CDATA[Social recommender systems leverage collaborative filtering (CF) to serve users with content that is of potential interesting to active users. A wide spectrum of CF schemes has been proposed. However, most of them cannot deal with the cold-start problem that denotes a situation that social media sites fail to draw recommendation for new items, users or both. In addition, they regard that all ratings equally contribute to the social media recommendation. This supposition is against the fact that low-level ratings contribute little to suggesting items that are likely to be of interest of users. To this end, we propose bi-clustering and fusion (BiFu)-a newly-fashioned scheme for the cold-start problem based on the BiFu techniques under a cloud computing setting. To identify the rating sources for recommendation, it introduces the concepts of popular items and frequent raters. To reduce the dimensionality of the rating matrix, BiFu leverages the bi-clustering technique. To overcome the data sparsity and rating diversity, it employs the smoothing and fusion technique. Finally, BiFu recommends social media contents from both item and user clusters. Experimental results show that BiFu significantly alleviates the cold-start problem in terms of accuracy and scalability.]]></abstract>
##     <issn><![CDATA[2168-6750]]></issn>
##     <htmlFlag><![CDATA[1]]></htmlFlag>
##     <arnumber><![CDATA[6607178]]></arnumber>
##     <doi><![CDATA[10.1109/TETC.2013.2283233]]></doi>
##     <publicationId><![CDATA[6607178]]></publicationId>
##     <mdurl><![CDATA[http://ieeexplore.ieee.org/xpl/articleDetails.jsp?tp=&arnumber=6607178&contentType=Journals+%26+Magazines]]></mdurl>
##     <pdf><![CDATA[http://ieeexplore.ieee.org/stamp/stamp.jsp?arnumber=6607178]]></pdf>
##   </document>
##   <document>
##     <rank>1181</rank>
##     <title><![CDATA[Stored Energy, Transmission Group Delay and Mode Field Distortion in Optical Fibers]]></title>
##     <authors><![CDATA[Varallyay, Z.;  Szipocs, R.]]></authors>
##     <affiliations><![CDATA[Inst. for Solid State Phys. & Opt., Wigner Res. Centre for Phys., Budapest, Hungary]]></affiliations>
##     <controlledterms>
##       <term><![CDATA[Bragg gratings]]></term>
##       <term><![CDATA[delays]]></term>
##       <term><![CDATA[optical design techniques]]></term>
##       <term><![CDATA[optical distortion]]></term>
##       <term><![CDATA[optical fibre dispersion]]></term>
##       <term><![CDATA[optical fibre losses]]></term>
##     </controlledterms>
##     <thesaurusterms>
##       <term><![CDATA[Delays]]></term>
##       <term><![CDATA[Optical distortion]]></term>
##       <term><![CDATA[Optical fiber amplifiers]]></term>
##       <term><![CDATA[Optical fiber dispersion]]></term>
##       <term><![CDATA[Optical fiber losses]]></term>
##     </thesaurusterms>
##     <pubtitle><![CDATA[Selected Topics in Quantum Electronics, IEEE Journal of]]></pubtitle>
##     <punumber><![CDATA[2944]]></punumber>
##     <pubtype><![CDATA[Journals & Magazines]]></pubtype>
##     <publisher><![CDATA[IEEE]]></publisher>
##     <volume><![CDATA[20]]></volume>
##     <issue><![CDATA[5]]></issue>
##     <py><![CDATA[2014]]></py>
##     <spage><![CDATA[126]]></spage>
##     <epage><![CDATA[131]]></epage>
##     <abstract><![CDATA[The relationship between transmission group delay and stored energy in optical fibers is discussed. We show by numerical computations that the group delay of an optical pulse of finite bandwidth transmitted through a piece of a low loss optical fiber of unit length is proportional to the energy stored by the standing wave electromagnetic field. The stored energy-group delay ratio typically approaches unity as the confinement loss converges to zero. In case of a dispersion tailored Bragg fiber, we found that the stored energy-group delay ratio decreased while the confinement loss increased compared to those of the standard quarterwave Bragg fiber configuration. Furthermore, a rapid variation in the group delay versus wavelength function due to mode-crossing events (in hollow core photonic bandgap fibers for instance) or resonances originating from slightly coupled cavities, surface or leaking modes in index guiding, photonic bandgap, or photonic crystal fibers always results in a rapid change in the mode-field distribution, which seriously affects splicing losses and focusability of the transmitted laser beam. All of these factors must be taken into consideration during the design of dispersion tailored fibers for different applications.]]></abstract>
##     <issn><![CDATA[1077-260X]]></issn>
##     <htmlFlag><![CDATA[1]]></htmlFlag>
##     <arnumber><![CDATA[6818386]]></arnumber>
##     <doi><![CDATA[10.1109/JSTQE.2014.2325394]]></doi>
##     <publicationId><![CDATA[6818386]]></publicationId>
##     <mdurl><![CDATA[http://ieeexplore.ieee.org/xpl/articleDetails.jsp?tp=&arnumber=6818386&contentType=Journals+%26+Magazines]]></mdurl>
##     <pdf><![CDATA[http://ieeexplore.ieee.org/stamp/stamp.jsp?arnumber=6818386]]></pdf>
##   </document>
##   <document>
##     <rank>1182</rank>
##     <title><![CDATA[Design of an Optical Trapping Device Based on an Ultra-High Q/V Resonant Structure]]></title>
##     <authors><![CDATA[Ciminelli, C.;  Conteduca, D.;  Dell'Olio, F.;  Armenise, M.N.]]></authors>
##     <affiliations><![CDATA[Optoelectron. Lab., Politec. di Bari, Bari, Italy]]></affiliations>
##     <controlledterms>
##       <term><![CDATA[cancer]]></term>
##       <term><![CDATA[finite element analysis]]></term>
##       <term><![CDATA[gold]]></term>
##       <term><![CDATA[particle traps]]></term>
##       <term><![CDATA[photonic crystals]]></term>
##       <term><![CDATA[plasmonics]]></term>
##       <term><![CDATA[proteomics]]></term>
##       <term><![CDATA[radiation pressure]]></term>
##       <term><![CDATA[tumours]]></term>
##     </controlledterms>
##     <thesaurusterms>
##       <term><![CDATA[Cavity resonators]]></term>
##       <term><![CDATA[Dielectrics]]></term>
##       <term><![CDATA[Light trapping]]></term>
##       <term><![CDATA[Photonic crystals]]></term>
##       <term><![CDATA[Plasmons]]></term>
##       <term><![CDATA[Q-factor]]></term>
##     </thesaurusterms>
##     <pubtitle><![CDATA[Photonics Journal, IEEE]]></pubtitle>
##     <punumber><![CDATA[4563994]]></punumber>
##     <pubtype><![CDATA[Journals & Magazines]]></pubtype>
##     <publisher><![CDATA[IEEE]]></publisher>
##     <volume><![CDATA[6]]></volume>
##     <issue><![CDATA[6]]></issue>
##     <py><![CDATA[2014]]></py>
##     <spage><![CDATA[1]]></spage>
##     <epage><![CDATA[16]]></epage>
##     <abstract><![CDATA[A novel photonic/plasmonic cavity based on a 1-D photonic crystal cavity vertically coupled to a plasmonic gold structure is reported. The design has been optimized to achieve an ultra-high Q/V ratio, therefore improving the light-matter interaction and making the device suitable for optical trapping applications. Accurate 3-D finite element method (FEM) simulations have been carried out to evaluate the device behavior and performance. The device shows Q = 2:8 &#x00D7; 10<sup>3</sup> and V = 4 &#x00D7; 10<sup>-4</sup>(&#x03BB;=n)<sup>3</sup>, which correspond to a Q=V = 7 &#x00D7; 10<sup>6</sup>(&#x03BB;=n)<sup>-3</sup> with a resonance transmission around 50% at &#x03BB;<sub>R</sub> = 1589:62 nm. A strong gradient of the optical energy has been observed in the metal structure at the resonance, inducing a strong optical force and allowing a single particle trapping with a diameter less than 100 nm. The device turns out very useful for novel biomedical applications, such as proteomics and oncology.]]></abstract>
##     <issn><![CDATA[1943-0655]]></issn>
##     <htmlFlag><![CDATA[1]]></htmlFlag>
##     <arnumber><![CDATA[6895109]]></arnumber>
##     <doi><![CDATA[10.1109/JPHOT.2014.2356496]]></doi>
##     <publicationId><![CDATA[6895109]]></publicationId>
##     <mdurl><![CDATA[http://ieeexplore.ieee.org/xpl/articleDetails.jsp?tp=&arnumber=6895109&contentType=Journals+%26+Magazines]]></mdurl>
##     <pdf><![CDATA[http://ieeexplore.ieee.org/stamp/stamp.jsp?arnumber=6895109]]></pdf>
##   </document>
##   <document>
##     <rank>1183</rank>
##     <title><![CDATA[Magneto-Optical Faraday Effects in Dispersive Properties and Unusual Surface Plasmon Modes in the Three-Dimensional Magnetized Plasma Photonic Crystals]]></title>
##     <authors><![CDATA[Hai-Feng Zhang;  Shao-Bin Liu]]></authors>
##     <affiliations><![CDATA[Coll. of Electron. & Inf. Eng., Nanjing Univ. of Aeronaut. & Astronaut., Nanjing, China]]></affiliations>
##     <controlledterms>
##       <term><![CDATA[Faraday effect]]></term>
##       <term><![CDATA[optical lattices]]></term>
##       <term><![CDATA[photonic crystals]]></term>
##       <term><![CDATA[surface plasmons]]></term>
##       <term><![CDATA[tellurium]]></term>
##     </controlledterms>
##     <thesaurusterms>
##       <term><![CDATA[Dielectrics]]></term>
##       <term><![CDATA[Licenses]]></term>
##       <term><![CDATA[Magnetic cores]]></term>
##       <term><![CDATA[Magnetic fields]]></term>
##       <term><![CDATA[Plasmons]]></term>
##       <term><![CDATA[Three-dimensional displays]]></term>
##     </thesaurusterms>
##     <pubtitle><![CDATA[Photonics Journal, IEEE]]></pubtitle>
##     <punumber><![CDATA[4563994]]></punumber>
##     <pubtype><![CDATA[Journals & Magazines]]></pubtype>
##     <publisher><![CDATA[IEEE]]></publisher>
##     <volume><![CDATA[6]]></volume>
##     <issue><![CDATA[1]]></issue>
##     <py><![CDATA[2014]]></py>
##     <spage><![CDATA[1]]></spage>
##     <epage><![CDATA[12]]></epage>
##     <abstract><![CDATA[The dispersive properties and unusual surface plasmon modes in three-dimensional (3-D) magnetized plasma photonic crystals (MPPCs) with face-centered-cubic lattices that are composed of the core tellurium (Te) spheres surrounded by the magnetized plasma shells inserted in the air are theoretically studied in detail by the plane-wave expansion method, as the magneto-optical Faraday effects of magnetized plasma are considered. Our analysis shows that the proposed 3-D MPPCs can obtain the complete photonic band gaps, which can be manipulated by the radius of core Te sphere, the plasma density, and the external magnetic field, respectively. We also find that a flatband region can be achieved, which is determined by the existence of surface plasmon modes. If the thickness of the magnetized plasma shell is less than a threshold value, the band structures of such 3-D MPPCs will be similar to those obtained from the same structure containing the pure magnetized plasma spheres. In this case, the inserted core sphere also will not affect the band structures. It is also noticed that the upper edge of flatband region does not depend on the topology of lattice.]]></abstract>
##     <issn><![CDATA[1943-0655]]></issn>
##     <htmlFlag><![CDATA[1]]></htmlFlag>
##     <arnumber><![CDATA[6714386]]></arnumber>
##     <doi><![CDATA[10.1109/JPHOT.2014.2300503]]></doi>
##     <publicationId><![CDATA[6714386]]></publicationId>
##     <mdurl><![CDATA[http://ieeexplore.ieee.org/xpl/articleDetails.jsp?tp=&arnumber=6714386&contentType=Journals+%26+Magazines]]></mdurl>
##     <pdf><![CDATA[http://ieeexplore.ieee.org/stamp/stamp.jsp?arnumber=6714386]]></pdf>
##   </document>
##   <document>
##     <rank>1184</rank>
##     <title><![CDATA[SiGe HBT Technology Based on a 0.13-<inline-formula> <img src="/images/tex/21655.gif" alt="\mu{\rm m}"> </inline-formula> Process Featuring an <inline-formula> <img src="/images/tex/21691.gif" alt="{f}_{\rm MAX}"> </inline-formula> of 325 GHz]]></title>
##     <authors><![CDATA[Hashimoto, T.;  Tokunaga, K.;  Fukumoto, K.;  Yoshida, Y.;  Satoh, H.;  Kubo, M.;  Shima, A.;  Oda, K.]]></authors>
##     <affiliations><![CDATA[Micro Device Div., Hitachi, Ltd., Tokyo, Japan]]></affiliations>
##     <controlledterms>
##       <term><![CDATA[BiCMOS integrated circuits]]></term>
##       <term><![CDATA[Ge-Si alloys]]></term>
##       <term><![CDATA[bipolar MIMIC]]></term>
##       <term><![CDATA[epitaxial growth]]></term>
##       <term><![CDATA[heterojunction bipolar transistors]]></term>
##       <term><![CDATA[semiconductor growth]]></term>
##       <term><![CDATA[submillimetre wave transistors]]></term>
##     </controlledterms>
##     <thesaurusterms>
##       <term><![CDATA[Boron]]></term>
##       <term><![CDATA[Epitaxial growth]]></term>
##       <term><![CDATA[Germanium]]></term>
##       <term><![CDATA[Heterojunction bipolar transistors]]></term>
##       <term><![CDATA[Resistance]]></term>
##       <term><![CDATA[Silicon]]></term>
##       <term><![CDATA[Silicon germanium]]></term>
##     </thesaurusterms>
##     <pubtitle><![CDATA[Electron Devices Society, IEEE Journal of the]]></pubtitle>
##     <punumber><![CDATA[6245494]]></punumber>
##     <pubtype><![CDATA[Journals & Magazines]]></pubtype>
##     <publisher><![CDATA[IEEE]]></publisher>
##     <volume><![CDATA[2]]></volume>
##     <issue><![CDATA[4]]></issue>
##     <py><![CDATA[2014]]></py>
##     <spage><![CDATA[50]]></spage>
##     <epage><![CDATA[58]]></epage>
##     <abstract><![CDATA[A self-aligned SiGe HBT technology achieving a cutoff frequency (f<sub>T</sub>) of 253 GHz was developed using a selective SiGe epitaxial growth process. Germanium concentration in an i-SiGe layer just under a p<sup>+</sup> intrinsic base region was raised to 27.4% to improve f<sub>T</sub>, and boron concentration in the intrinsic base region reached 2.4 &#x00D7; 10<sup>20</sup> cm<sup>-3</sup> as a deposition to maintain a breakdown voltage of 1.5 V. A 0.13-&#x03BC;m SiGe BiCMOS technology geometrically advanced from an earlier 0.18-&#x03BC;m version shrinks the emitter width from 0.2 to 0.12 &#x03BC;m to reduce collector-base capacitance and base resistance. It achieves a maximum oscillation frequency (f<sub>MAX</sub>) of 325 GHz. This technology can be applied to optical and mm wave communication systems.]]></abstract>
##     <issn><![CDATA[2168-6734]]></issn>
##     <htmlFlag><![CDATA[1]]></htmlFlag>
##     <arnumber><![CDATA[6783783]]></arnumber>
##     <doi><![CDATA[10.1109/JEDS.2014.2315854]]></doi>
##     <publicationId><![CDATA[6783783]]></publicationId>
##     <mdurl><![CDATA[http://ieeexplore.ieee.org/xpl/articleDetails.jsp?tp=&arnumber=6783783&contentType=Journals+%26+Magazines]]></mdurl>
##     <pdf><![CDATA[http://ieeexplore.ieee.org/stamp/stamp.jsp?arnumber=6783783]]></pdf>
##   </document>
##   <document>
##     <rank>1185</rank>
##     <title><![CDATA[Optimal Algorithms for <formula formulatype="inline"> <img src="/images/tex/692.gif" alt="L_{1}"> </formula>-subspace Signal Processing]]></title>
##     <authors><![CDATA[Markopoulos, P.P.;  Karystinos, G.N.;  Pados, D.A.]]></authors>
##     <affiliations><![CDATA[Dept. of Electr. Eng., SUNY - Univ. at Buffalo, Buffalo, NY, USA]]></affiliations>
##     <controlledterms>
##       <term><![CDATA[computational complexity]]></term>
##       <term><![CDATA[direction-of-arrival estimation]]></term>
##       <term><![CDATA[image restoration]]></term>
##       <term><![CDATA[principal component analysis]]></term>
##       <term><![CDATA[signal processing]]></term>
##     </controlledterms>
##     <thesaurusterms>
##       <term><![CDATA[Approximation methods]]></term>
##       <term><![CDATA[Complexity theory]]></term>
##       <term><![CDATA[Licenses]]></term>
##       <term><![CDATA[Maximum likelihood estimation]]></term>
##       <term><![CDATA[Principal component analysis]]></term>
##       <term><![CDATA[Signal processing]]></term>
##       <term><![CDATA[Signal processing algorithms]]></term>
##     </thesaurusterms>
##     <pubtitle><![CDATA[Signal Processing, IEEE Transactions on]]></pubtitle>
##     <punumber><![CDATA[78]]></punumber>
##     <pubtype><![CDATA[Journals & Magazines]]></pubtype>
##     <publisher><![CDATA[IEEE]]></publisher>
##     <volume><![CDATA[62]]></volume>
##     <issue><![CDATA[19]]></issue>
##     <py><![CDATA[2014]]></py>
##     <spage><![CDATA[5046]]></spage>
##     <epage><![CDATA[5058]]></epage>
##     <abstract><![CDATA[We describe ways to define and calculate L<sub>1</sub>-norm signal subspaces that are less sensitive to outlying data than L<sub>2</sub>-calculated subspaces. We start with the computation of the L<sub>1</sub> maximum-projection principal component of a data matrix containing N signal samples of dimension D. We show that while the general problem is formally NP-hard in asymptotically large N, D, the case of engineering interest of fixed dimension D and asymptotically large sample size N is not. In particular, for the case where the sample size is less than the fixed dimension , we present in explicit form an optimal algorithm of computational cost 2<sup>N</sup>. For the case N &#x2265; D, we present an optimal algorithm of complexity O(N<sup>D</sup>). We generalize to multiple L<sub>1</sub>-max-projection components and present an explicit optimal L<sub>1</sub> subspace calculation algorithm of complexity O(N<sup>DK-K+1</sup>) where K is the desired number of L<sub>1</sub> principal components (subspace rank). We conclude with illustrations of L<sub>1</sub>-subspace signal processing in the fields of data dimensionality reduction, direction-of-arrival estimation, and image conditioning/restoration.]]></abstract>
##     <issn><![CDATA[1053-587X]]></issn>
##     <htmlFlag><![CDATA[1]]></htmlFlag>
##     <arnumber><![CDATA[6851920]]></arnumber>
##     <doi><![CDATA[10.1109/TSP.2014.2338077]]></doi>
##     <publicationId><![CDATA[6851920]]></publicationId>
##     <mdurl><![CDATA[http://ieeexplore.ieee.org/xpl/articleDetails.jsp?tp=&arnumber=6851920&contentType=Journals+%26+Magazines]]></mdurl>
##     <pdf><![CDATA[http://ieeexplore.ieee.org/stamp/stamp.jsp?arnumber=6851920]]></pdf>
##   </document>
##   <document>
##     <rank>1186</rank>
##     <title><![CDATA[Symmetric 40-Gb/s, 100-km Passive Reach TWDM-PON with 53-dB Loss Budget]]></title>
##     <authors><![CDATA[Zhengxuan Li;  Lilin Yi;  Wei Wei;  Meihua Bi;  Hao He;  Shilin Xiao;  Weisheng Hu]]></authors>
##     <affiliations><![CDATA[State Key Lab. of Adv. Opt. Commun. Syst. & Networks, Shanghai Jiao Tong Univ., Shanghai, China]]></affiliations>
##     <controlledterms>
##       <term><![CDATA[fibre lasers]]></term>
##       <term><![CDATA[light transmission]]></term>
##       <term><![CDATA[optical fibre dispersion]]></term>
##       <term><![CDATA[optical modulation]]></term>
##       <term><![CDATA[optical transmitters]]></term>
##       <term><![CDATA[passive optical networks]]></term>
##       <term><![CDATA[time division multiplexing]]></term>
##       <term><![CDATA[wavelength division multiplexing]]></term>
##     </controlledterms>
##     <thesaurusterms>
##       <term><![CDATA[Optical fiber dispersion]]></term>
##       <term><![CDATA[Optical fibers]]></term>
##       <term><![CDATA[Optical losses]]></term>
##       <term><![CDATA[Optical transmitters]]></term>
##       <term><![CDATA[Passive optical networks]]></term>
##       <term><![CDATA[Sensitivity]]></term>
##       <term><![CDATA[Vertical cavity surface emitting lasers]]></term>
##     </thesaurusterms>
##     <pubtitle><![CDATA[Lightwave Technology, Journal of]]></pubtitle>
##     <punumber><![CDATA[50]]></punumber>
##     <pubtype><![CDATA[Journals & Magazines]]></pubtype>
##     <publisher><![CDATA[IEEE]]></publisher>
##     <volume><![CDATA[32]]></volume>
##     <issue><![CDATA[21]]></issue>
##     <py><![CDATA[2014]]></py>
##     <spage><![CDATA[3991]]></spage>
##     <epage><![CDATA[3998]]></epage>
##     <abstract><![CDATA[A truly passive long-reach, symmetric 40-Gb/s time and wavelength division multiplexed passive optical network (TWDM-PON) with a high loss budget is demonstrated, using direct modulation and direct detection in both upstream and downstream directions. Thermally tuned directly modulated lasers (DMLs) are employed to serve as both upstream and downstream transmitters, not only owing to their low cost, but also, as a carrier-less modulation method, where the signal generated by direct modulation is demonstrated to be more robust to high launch power induced fiber nonlinearities compared with external intensity modulation formats with strong carrier power. Therefore, DML is suitable for applying in TWDM-PON to achieve a high loss budget. Moreover, the frequency chirp induced dispersion of directly-modulated signal is managed thanks to the combination of optical spectral reshaping and dispersion supported transmission effects, which makes it possible for the directly-modulated signal to reach a distance of 100 km and still with a good quality. As a result, a system loss budget of 53 dB is achieved, supporting more than 1000 users with 100-km purely passive reach, which is the first demonstration of high loss budget, long reach TWDM-PONs to our best knowledge.]]></abstract>
##     <issn><![CDATA[0733-8724]]></issn>
##     <htmlFlag><![CDATA[1]]></htmlFlag>
##     <arnumber><![CDATA[6857318]]></arnumber>
##     <doi><![CDATA[10.1109/JLT.2014.2338055]]></doi>
##     <publicationId><![CDATA[6857318]]></publicationId>
##     <mdurl><![CDATA[http://ieeexplore.ieee.org/xpl/articleDetails.jsp?tp=&arnumber=6857318&contentType=Journals+%26+Magazines]]></mdurl>
##     <pdf><![CDATA[http://ieeexplore.ieee.org/stamp/stamp.jsp?arnumber=6857318]]></pdf>
##   </document>
##   <document>
##     <rank>1187</rank>
##     <title><![CDATA[Graphene for Electron Devices: The Panorama of a Decade]]></title>
##     <authors><![CDATA[Dash, G.N.;  Pattanaik, S.R.;  Behera, S.]]></authors>
##     <affiliations><![CDATA[Electron Devices Group, Sambalpur Univ., Sambalpur, India]]></affiliations>
##     <controlledterms>
##       <term><![CDATA[graphene]]></term>
##     </controlledterms>
##     <thesaurusterms>
##       <term><![CDATA[Charge carrier density]]></term>
##       <term><![CDATA[Conductivity]]></term>
##       <term><![CDATA[Electron devices]]></term>
##       <term><![CDATA[Graphene]]></term>
##       <term><![CDATA[Logic gates]]></term>
##       <term><![CDATA[Materials]]></term>
##       <term><![CDATA[Scattering]]></term>
##     </thesaurusterms>
##     <pubtitle><![CDATA[Electron Devices Society, IEEE Journal of the]]></pubtitle>
##     <punumber><![CDATA[6245494]]></punumber>
##     <pubtype><![CDATA[Journals & Magazines]]></pubtype>
##     <publisher><![CDATA[IEEE]]></publisher>
##     <volume><![CDATA[2]]></volume>
##     <issue><![CDATA[5]]></issue>
##     <py><![CDATA[2014]]></py>
##     <spage><![CDATA[77]]></spage>
##     <epage><![CDATA[104]]></epage>
##     <abstract><![CDATA[Graphene emerged in 2004 as the first 2-D material with exotic properties. Since then the literature has been flooded with reports, with physicists, material scientists, and engineers grabbing their respective shares. Numerous reviews have also been published. While these reviews have done excellent works in their own ways, new reports are coming up faster than they could draw the attentions of researchers. The authors, therefore, feel that there is a demanding scope for a fresh review. Further, many aspects of graphene are not covered in the reviews so far. New concept devices are also entering into the arena of graphene day by day. The purpose of this paper is therefore to present a comprehensive review on the conventional as well as novel device applications of graphene. While we believe that graphene is the material which will transform the electron devices from the classical regime to the quantum world, it is difficult to believe that it will be a complete substitute to silicon in the near future.]]></abstract>
##     <issn><![CDATA[2168-6734]]></issn>
##     <htmlFlag><![CDATA[1]]></htmlFlag>
##     <arnumber><![CDATA[6824734]]></arnumber>
##     <doi><![CDATA[10.1109/JEDS.2014.2328032]]></doi>
##     <publicationId><![CDATA[6824734]]></publicationId>
##     <mdurl><![CDATA[http://ieeexplore.ieee.org/xpl/articleDetails.jsp?tp=&arnumber=6824734&contentType=Journals+%26+Magazines]]></mdurl>
##     <pdf><![CDATA[http://ieeexplore.ieee.org/stamp/stamp.jsp?arnumber=6824734]]></pdf>
##   </document>
##   <document>
##     <rank>1188</rank>
##     <title><![CDATA[Dense Dielectric Patch Array Antenna With Improved Radiation Characteristics Using EBG Ground Structure and Dielectric Superstrate for Future 5G Cellular Networks]]></title>
##     <authors><![CDATA[Haraz, O.M.;  Elboushi, A.;  Alshebeili, S.A.;  Sebak, A.-R.]]></authors>
##     <affiliations><![CDATA[King Abdulaziz City of Sci. & Technol.-Technol. Innovation Center in Radiofreq. & photonics for the e-Soc. (RFTONICS), King Saud Univ., Riyadh, Saudi Arabia]]></affiliations>
##     <controlledterms>
##       <term><![CDATA[antenna radiation patterns]]></term>
##       <term><![CDATA[cellular radio]]></term>
##       <term><![CDATA[microstrip antenna arrays]]></term>
##       <term><![CDATA[photonic band gap]]></term>
##       <term><![CDATA[power dividers]]></term>
##     </controlledterms>
##     <thesaurusterms>
##       <term><![CDATA[Antenna arrays]]></term>
##       <term><![CDATA[Antenna measurements]]></term>
##       <term><![CDATA[Cellular networks]]></term>
##       <term><![CDATA[Dielectrics]]></term>
##       <term><![CDATA[Electromagnetic band gap]]></term>
##       <term><![CDATA[Microwave antenna arrays]]></term>
##       <term><![CDATA[Mobile communication]]></term>
##       <term><![CDATA[Patch antennas]]></term>
##       <term><![CDATA[Prototypes]]></term>
##       <term><![CDATA[Radiators]]></term>
##     </thesaurusterms>
##     <pubtitle><![CDATA[Access, IEEE]]></pubtitle>
##     <punumber><![CDATA[6287639]]></punumber>
##     <pubtype><![CDATA[Journals & Magazines]]></pubtype>
##     <publisher><![CDATA[IEEE]]></publisher>
##     <volume><![CDATA[2]]></volume>
##     <py><![CDATA[2014]]></py>
##     <spage><![CDATA[909]]></spage>
##     <epage><![CDATA[913]]></epage>
##     <abstract><![CDATA[In this paper, a new dense dielectric (DD) patch array antenna prototype operating at 28 GHz for future fifth generation (5G) cellular networks is presented. This array antenna is proposed and designed with a standard printed circuit board process to be suitable for integration with radio frequency/microwave circuitry. The proposed structure employs four circular-shaped DD patch radiator antenna elements fed by a 1-to-4 Wilkinson power divider. To improve the array radiation characteristics, a ground structure based on a compact uniplanar electromagnetic bandgap unit cell has been used. The DD patch shows better radiation and total efficiencies compared with the metallic patch radiator. For further gain improvement, a dielectric layer of a superstrate is applied above the array antenna. The measured impedance bandwidth of the proposed array antenna ranges from 27 to beyond 32 GHz for a reflection coefficient (S11) of less than -10 dB. The proposed design exhibits stable radiation patterns over the whole frequency band of interest, with a total realized gain more than 16 dBi. Due to the remarkable performance of the proposed array, it can be considered as a good candidate for 5G communication applications.]]></abstract>
##     <issn><![CDATA[2169-3536]]></issn>
##     <htmlFlag><![CDATA[1]]></htmlFlag>
##     <arnumber><![CDATA[6887340]]></arnumber>
##     <doi><![CDATA[10.1109/ACCESS.2014.2352679]]></doi>
##     <publicationId><![CDATA[6887340]]></publicationId>
##     <mdurl><![CDATA[http://ieeexplore.ieee.org/xpl/articleDetails.jsp?tp=&arnumber=6887340&contentType=Journals+%26+Magazines]]></mdurl>
##     <pdf><![CDATA[http://ieeexplore.ieee.org/stamp/stamp.jsp?arnumber=6887340]]></pdf>
##   </document>
##   <document>
##     <rank>1189</rank>
##     <title><![CDATA[An Efficient Multiple Cell Upsets Tolerant Content-Addressable Memory]]></title>
##     <authors><![CDATA[Abbas, S.M.;  Soonyoung Lee;  Baeg, S.;  Sungju Park]]></authors>
##     <affiliations><![CDATA[Comput. Sci. & Eng. Dept., Hanyang Univ., Ansan, South Korea]]></affiliations>
##     <controlledterms>
##       <term><![CDATA[content-addressable storage]]></term>
##       <term><![CDATA[error correction codes]]></term>
##     </controlledterms>
##     <pubtitle><![CDATA[Computers, IEEE Transactions on]]></pubtitle>
##     <punumber><![CDATA[12]]></punumber>
##     <pubtype><![CDATA[Journals & Magazines]]></pubtype>
##     <publisher><![CDATA[IEEE]]></publisher>
##     <volume><![CDATA[63]]></volume>
##     <issue><![CDATA[8]]></issue>
##     <py><![CDATA[2014]]></py>
##     <spage><![CDATA[2094]]></spage>
##     <epage><![CDATA[2098]]></epage>
##     <abstract><![CDATA[Multiple cell upsets (MCUs) become more and more problematic as the size of technology reaches or goes below 65 nm. The percentage of MCUs is reported significantly larger than that of single cell upsets (SCUs) in 20 nm technology. In SRAM and DRAM, MCUs are tackled by incorporating single-error correcting double-error detecting (SEC-DED) code and interleaved data columns. However, in content-addressable memory (CAM), column interleaving is not practically possible. A novel error correction code (ECC) scheme is proposed in this paper that will cater for ever-increasing MCUs. This work demonstrated that m parity bits are sufficient to cater for up to m-bit MCUs, with an understanding of the physical grouping of MCUs. The results showed that the proposed scheme requires 85% fewer parity bits compared to traditional Hamming distance based schemes.]]></abstract>
##     <issn><![CDATA[0018-9340]]></issn>
##     <htmlFlag><![CDATA[1]]></htmlFlag>
##     <arnumber><![CDATA[6497044]]></arnumber>
##     <doi><![CDATA[10.1109/TC.2013.90]]></doi>
##     <publicationId><![CDATA[6497044]]></publicationId>
##     <mdurl><![CDATA[http://ieeexplore.ieee.org/xpl/articleDetails.jsp?tp=&arnumber=6497044&contentType=Journals+%26+Magazines]]></mdurl>
##     <pdf><![CDATA[http://ieeexplore.ieee.org/stamp/stamp.jsp?arnumber=6497044]]></pdf>
##   </document>
##   <document>
##     <rank>1190</rank>
##     <title><![CDATA[Influence of Magnetizing and Filtering Frequencies on Barkhausen Noise Response]]></title>
##     <authors><![CDATA[Stupakov, O.;  Melikhov, Y.]]></authors>
##     <affiliations><![CDATA[Inst. of Phys., Prague, Czech Republic]]></affiliations>
##     <controlledterms>
##       <term><![CDATA[Barkhausen effect]]></term>
##       <term><![CDATA[iron alloys]]></term>
##       <term><![CDATA[magnetic field measurement]]></term>
##       <term><![CDATA[magnetic hysteresis]]></term>
##       <term><![CDATA[magnetic noise]]></term>
##       <term><![CDATA[magnetic sensors]]></term>
##       <term><![CDATA[sensor arrays]]></term>
##       <term><![CDATA[silicon alloys]]></term>
##     </controlledterms>
##     <thesaurusterms>
##       <term><![CDATA[Coercive force]]></term>
##       <term><![CDATA[Frequency measurement]]></term>
##       <term><![CDATA[Magnetic hysteresis]]></term>
##       <term><![CDATA[Magnetic separation]]></term>
##       <term><![CDATA[Noise]]></term>
##       <term><![CDATA[Steel]]></term>
##     </thesaurusterms>
##     <pubtitle><![CDATA[Magnetics, IEEE Transactions on]]></pubtitle>
##     <punumber><![CDATA[20]]></punumber>
##     <pubtype><![CDATA[Journals & Magazines]]></pubtype>
##     <publisher><![CDATA[IEEE]]></publisher>
##     <volume><![CDATA[50]]></volume>
##     <issue><![CDATA[4]]></issue>
##     <part><![CDATA[1]]></part>
##     <py><![CDATA[2014]]></py>
##     <spage><![CDATA[1]]></spage>
##     <epage><![CDATA[4]]></epage>
##     <abstract><![CDATA[This paper is devoted to frequency issues of the Barkhausen noise (BN) measurements, namely, to investigation of influence of magnetizing and filtering frequencies on the BN response. The measurements were performed for typical industrial steels at controllable magnetizing conditions: fixed induction or field waveforms. A vertically mounted array of three Hall sensors was used for direct determination of the sample magnetic field. The BN signal was detected locally by a surface-mounted pancake coil. In ac magnetizing frequency range, the BN quantities were observed to follow a nearly square root dependence on the frequency, provoking the classical discussion about possible correlation between the magnetic hysteresis and the BN effects. Influence of the filtering frequencies on the BN parameters (reading depth adjustment) was studied on a decarburized spring steel with surface ferrite layers of different thicknesses.]]></abstract>
##     <issn><![CDATA[0018-9464]]></issn>
##     <htmlFlag><![CDATA[1]]></htmlFlag>
##     <arnumber><![CDATA[6786387]]></arnumber>
##     <doi><![CDATA[10.1109/TMAG.2013.2291933]]></doi>
##     <publicationId><![CDATA[6786387]]></publicationId>
##     <mdurl><![CDATA[http://ieeexplore.ieee.org/xpl/articleDetails.jsp?tp=&arnumber=6786387&contentType=Journals+%26+Magazines]]></mdurl>
##     <pdf><![CDATA[http://ieeexplore.ieee.org/stamp/stamp.jsp?arnumber=6786387]]></pdf>
##   </document>
##   <document>
##     <rank>1191</rank>
##     <title><![CDATA[Low-Overhead Network-on-Chip Support for Location-Oblivious Task Placement]]></title>
##     <authors><![CDATA[Gwangsun Kim;  Lee, M.M.-J.;  Kim, J.;  Lee, J.W.;  Abts, D.;  Marty, M.]]></authors>
##     <affiliations><![CDATA[Dept. of Comput. Sci., Korea Adv. Inst. of Sci. & Technol. (KAIST), Daejeon, South Korea]]></affiliations>
##     <controlledterms>
##       <term><![CDATA[asynchronous circuits]]></term>
##       <term><![CDATA[cache storage]]></term>
##       <term><![CDATA[circuit complexity]]></term>
##       <term><![CDATA[graphics processing units]]></term>
##       <term><![CDATA[multiprocessing systems]]></term>
##       <term><![CDATA[network-on-chip]]></term>
##       <term><![CDATA[performance evaluation]]></term>
##       <term><![CDATA[probability]]></term>
##     </controlledterms>
##     <thesaurusterms>
##       <term><![CDATA[Bandwidth]]></term>
##       <term><![CDATA[Measurement]]></term>
##       <term><![CDATA[Probabilistic logic]]></term>
##       <term><![CDATA[Program processors]]></term>
##       <term><![CDATA[Routing]]></term>
##       <term><![CDATA[System-on-a-chip]]></term>
##     </thesaurusterms>
##     <pubtitle><![CDATA[Computers, IEEE Transactions on]]></pubtitle>
##     <punumber><![CDATA[12]]></punumber>
##     <pubtype><![CDATA[Journals & Magazines]]></pubtype>
##     <publisher><![CDATA[IEEE]]></publisher>
##     <volume><![CDATA[63]]></volume>
##     <issue><![CDATA[6]]></issue>
##     <py><![CDATA[2014]]></py>
##     <spage><![CDATA[1487]]></spage>
##     <epage><![CDATA[1500]]></epage>
##     <abstract><![CDATA[Many-core processors will have many processing cores with a network-on-chip (NoC) that provides access to shared resources such as main memory and on-chip caches. However, locally-fair arbitration in multi-stage NoC can lead to globally unfair access to shared resources and impact system-level performance depending on where each task is physically placed. In this work, we propose an arbitration to provide equality-of-service (EoS) in the network and provide support for location-oblivious task placement. We propose using probabilistic arbitration combined with distance-based weights to achieve EoS and overcome the limitation of round-robin arbiter. However, the complexity of probabilistic arbitration results in high area and long latency which negatively impacts performance. In order to reduce the hardware complexity, we propose an hybrid arbiter that switches between a simple arbiter at low load and a complex arbiter at high load. The hybrid arbiter is enabled by the observation that arbitration only impacts the overall performance and global fairness at a high load. We evaluate our arbitration scheme with synthetic traffic patterns and GPGPU benchmarks. Our results shows that hybrid arbiter that combines round-robin arbiter with probabilistic distance-based arbitration reduces performance variation as task placement is varied and also improves average IPC.]]></abstract>
##     <issn><![CDATA[0018-9340]]></issn>
##     <htmlFlag><![CDATA[1]]></htmlFlag>
##     <arnumber><![CDATA[6319294]]></arnumber>
##     <doi><![CDATA[10.1109/TC.2012.241]]></doi>
##     <publicationId><![CDATA[6319294]]></publicationId>
##     <mdurl><![CDATA[http://ieeexplore.ieee.org/xpl/articleDetails.jsp?tp=&arnumber=6319294&contentType=Journals+%26+Magazines]]></mdurl>
##     <pdf><![CDATA[http://ieeexplore.ieee.org/stamp/stamp.jsp?arnumber=6319294]]></pdf>
##   </document>
##   <document>
##     <rank>1192</rank>
##     <title><![CDATA[A Fourier-Based Approach to the Angiographic Assessment of Flow Diverter Efficacy in the Treatment of Cerebral Aneurysms]]></title>
##     <authors><![CDATA[Benz, T.;  Kowarschik, M.;  Endres, J.;  Redel, T.;  Demirci, S.;  Navab, N.]]></authors>
##     <affiliations><![CDATA[Dept. of Comput. Sci., Tech. Univ. Munchen, Munich, Germany]]></affiliations>
##     <controlledterms>
##       <term><![CDATA[Fourier transforms]]></term>
##       <term><![CDATA[biomedical equipment]]></term>
##       <term><![CDATA[blood vessels]]></term>
##       <term><![CDATA[brain]]></term>
##       <term><![CDATA[diagnostic radiography]]></term>
##       <term><![CDATA[diseases]]></term>
##       <term><![CDATA[feature extraction]]></term>
##       <term><![CDATA[frequency-domain analysis]]></term>
##       <term><![CDATA[haemodynamics]]></term>
##       <term><![CDATA[medical disorders]]></term>
##       <term><![CDATA[medical image processing]]></term>
##       <term><![CDATA[neurophysiology]]></term>
##       <term><![CDATA[patient treatment]]></term>
##       <term><![CDATA[spectral analysis]]></term>
##     </controlledterms>
##     <thesaurusterms>
##       <term><![CDATA[Aneurysm]]></term>
##       <term><![CDATA[Arteries]]></term>
##       <term><![CDATA[Biomedical image processing]]></term>
##       <term><![CDATA[Blood]]></term>
##       <term><![CDATA[Brain models]]></term>
##       <term><![CDATA[Embolization]]></term>
##       <term><![CDATA[Mathematical model]]></term>
##     </thesaurusterms>
##     <pubtitle><![CDATA[Medical Imaging, IEEE Transactions on]]></pubtitle>
##     <punumber><![CDATA[42]]></punumber>
##     <pubtype><![CDATA[Journals & Magazines]]></pubtype>
##     <publisher><![CDATA[IEEE]]></publisher>
##     <volume><![CDATA[33]]></volume>
##     <issue><![CDATA[9]]></issue>
##     <py><![CDATA[2014]]></py>
##     <spage><![CDATA[1788]]></spage>
##     <epage><![CDATA[1802]]></epage>
##     <abstract><![CDATA[Flow diversion is an emerging endovascular treatment option for cerebral aneurysms. Quantitative assessment of hemodynamic changes induced by flow diversion can aid clinical decision making in the treatment of cerebral aneurysms. In this article, besides summarizing past key research efforts, we propose a novel metric for the angiographic assessment of flow diverter deployments in the treatment of cerebral aneurysms. By analyzing the frequency spectra of signals derived from digital subtraction angiography (DSA) series, the metric aims to quantify the prevalence of frequency components that correspond to the patient-specific heart rate. Indicating the decoupling of aneurysms from healthy blood circulation, our proposed metric could advance clinical guidelines for treatment success prediction. The very promising results of a retrospective feasibility study on 26 DSA series warrant future efforts to study the validity of the proposed metric within a clinical setting.]]></abstract>
##     <issn><![CDATA[0278-0062]]></issn>
##     <htmlFlag><![CDATA[1]]></htmlFlag>
##     <arnumber><![CDATA[6807795]]></arnumber>
##     <doi><![CDATA[10.1109/TMI.2014.2320602]]></doi>
##     <publicationId><![CDATA[6807795]]></publicationId>
##     <mdurl><![CDATA[http://ieeexplore.ieee.org/xpl/articleDetails.jsp?tp=&arnumber=6807795&contentType=Journals+%26+Magazines]]></mdurl>
##     <pdf><![CDATA[http://ieeexplore.ieee.org/stamp/stamp.jsp?arnumber=6807795]]></pdf>
##   </document>
##   <document>
##     <rank>1193</rank>
##     <title><![CDATA[Multiband Carrierless Amplitude Phase Modulation for High Capacity Optical Data Links]]></title>
##     <authors><![CDATA[Olmedo, M.I.;  Tianjian Zuo;  Jensen, J.B.;  Qiwen Zhong;  Xiaogeng Xu;  Popov, S.;  Monroy, I.T.]]></authors>
##     <affiliations><![CDATA[Dept. of Photonics Eng., Tech. Univ. of Denmark (DTU), Lyngby, Denmark]]></affiliations>
##     <controlledterms>
##       <term><![CDATA[amplitude modulation]]></term>
##       <term><![CDATA[forward error correction]]></term>
##       <term><![CDATA[optical links]]></term>
##       <term><![CDATA[optical transceivers]]></term>
##     </controlledterms>
##     <thesaurusterms>
##       <term><![CDATA[Bandwidth]]></term>
##       <term><![CDATA[Bit error rate]]></term>
##       <term><![CDATA[Optical fiber communication]]></term>
##       <term><![CDATA[Optical filters]]></term>
##       <term><![CDATA[Optical modulation]]></term>
##       <term><![CDATA[Optical transmitters]]></term>
##       <term><![CDATA[Quadrature amplitude modulation]]></term>
##     </thesaurusterms>
##     <pubtitle><![CDATA[Lightwave Technology, Journal of]]></pubtitle>
##     <punumber><![CDATA[50]]></punumber>
##     <pubtype><![CDATA[Journals & Magazines]]></pubtype>
##     <publisher><![CDATA[IEEE]]></publisher>
##     <volume><![CDATA[32]]></volume>
##     <issue><![CDATA[4]]></issue>
##     <py><![CDATA[2014]]></py>
##     <spage><![CDATA[798]]></spage>
##     <epage><![CDATA[804]]></epage>
##     <abstract><![CDATA[Short range optical data links are experiencing bandwidth limitations making it very challenging to cope with the growing data transmission capacity demands. Parallel optics appears as a valid short-term solution. It is, however, not a viable solution in the long-term because of its complex optical packaging. Therefore, increasing effort is now put into the possibility of exploiting higher order modulation formats with increased spectral efficiency and reduced optical transceiver complexity. As these type of links are based on intensity modulation and direct detection, modulation formats relying on optical coherent detection can not be straight forwardly employed. As an alternative and more viable solution, this paper proposes the use of carrierless amplitude phase (CAP) in a novel multiband approach (MultiCAP) that achieves record spectral efficiency, increases tolerance towards dispersion and bandwidth limitations, and reduces the complexity of the transceiver. We report on numerical simulations and experimental demonstrations with capacity beyond 100 Gb/s transmission using a single externally modulated laser. In addition, an extensive comparison with conventional CAP is also provided. The reported experiment uses MultiCAP to achieve 102.4 Gb/s transmission, corresponding to a data payload of 95.2 Gb/s error free transmission by using a 7% forward error correction code. The signal is successfully recovered after 15 km of standard single mode fiber in a system limited by a 3 dB bandwidth of 14 GHz.]]></abstract>
##     <issn><![CDATA[0733-8724]]></issn>
##     <htmlFlag><![CDATA[1]]></htmlFlag>
##     <arnumber><![CDATA[6626568]]></arnumber>
##     <doi><![CDATA[10.1109/JLT.2013.2284926]]></doi>
##     <publicationId><![CDATA[6626568]]></publicationId>
##     <mdurl><![CDATA[http://ieeexplore.ieee.org/xpl/articleDetails.jsp?tp=&arnumber=6626568&contentType=Journals+%26+Magazines]]></mdurl>
##     <pdf><![CDATA[http://ieeexplore.ieee.org/stamp/stamp.jsp?arnumber=6626568]]></pdf>
##   </document>
##   <document>
##     <rank>1194</rank>
##     <title><![CDATA[Blue Laser Diode Wavelength Selection With a Variable Reflectivity Resonant Mirror]]></title>
##     <authors><![CDATA[Byrd, M.J.;  Woodward, R.H.;  Pung, A.J.;  Johnson, E.G.;  Lee, K.J.;  Magnusson, R.;  Binun, P.;  McCormick, K.]]></authors>
##     <affiliations><![CDATA[Dept. of Electr. & Comput. Eng., Clemson Univ., Clemson, SC, USA]]></affiliations>
##     <controlledterms>
##       <term><![CDATA[III-V semiconductors]]></term>
##       <term><![CDATA[gallium compounds]]></term>
##       <term><![CDATA[laser beams]]></term>
##       <term><![CDATA[laser cavity resonators]]></term>
##       <term><![CDATA[laser mirrors]]></term>
##       <term><![CDATA[laser mode locking]]></term>
##       <term><![CDATA[laser tuning]]></term>
##       <term><![CDATA[optical filters]]></term>
##       <term><![CDATA[reflectivity]]></term>
##       <term><![CDATA[semiconductor lasers]]></term>
##     </controlledterms>
##     <thesaurusterms>
##       <term><![CDATA[Cavity resonators]]></term>
##       <term><![CDATA[Diode lasers]]></term>
##       <term><![CDATA[Gratings]]></term>
##       <term><![CDATA[Mirrors]]></term>
##       <term><![CDATA[Power generation]]></term>
##       <term><![CDATA[Reflectivity]]></term>
##       <term><![CDATA[Semiconductor lasers]]></term>
##     </thesaurusterms>
##     <pubtitle><![CDATA[Photonics Technology Letters, IEEE]]></pubtitle>
##     <punumber><![CDATA[68]]></punumber>
##     <pubtype><![CDATA[Journals & Magazines]]></pubtype>
##     <publisher><![CDATA[IEEE]]></publisher>
##     <volume><![CDATA[26]]></volume>
##     <issue><![CDATA[23]]></issue>
##     <py><![CDATA[2014]]></py>
##     <spage><![CDATA[2311]]></spage>
##     <epage><![CDATA[2314]]></epage>
##     <abstract><![CDATA[A guided-mode resonance filter is designed and fabricated for externally locking a GaN blue laser diode. The external mirror design is polarization selective which allows the reflectivity to be tuned, in order to optimize the output coupling of the cavity. The resonance demonstrates a line-width &lt;;0.5 nm, centered at 445.6 nm with an output power in excess of 0.5 W of CW power without temperature control.]]></abstract>
##     <issn><![CDATA[1041-1135]]></issn>
##     <htmlFlag><![CDATA[1]]></htmlFlag>
##     <arnumber><![CDATA[6882125]]></arnumber>
##     <doi><![CDATA[10.1109/LPT.2014.2351351]]></doi>
##     <publicationId><![CDATA[6882125]]></publicationId>
##     <mdurl><![CDATA[http://ieeexplore.ieee.org/xpl/articleDetails.jsp?tp=&arnumber=6882125&contentType=Journals+%26+Magazines]]></mdurl>
##     <pdf><![CDATA[http://ieeexplore.ieee.org/stamp/stamp.jsp?arnumber=6882125]]></pdf>
##   </document>
##   <document>
##     <rank>1195</rank>
##     <title><![CDATA[Behavioral Modeling and Linearization of Crosstalk and Memory Effects in RF MIMO Transmitters]]></title>
##     <authors><![CDATA[Amin, S.;  Landin, P.N.;  Handel, P.;  Ro&#x0308; nnow, D.]]></authors>
##     <affiliations><![CDATA[Dept. of Electron. Math. & Natural Sci., Univ. of Gavle, Ga&#x0308;vle, Sweden]]></affiliations>
##     <controlledterms>
##       <term><![CDATA[MIMO communication]]></term>
##       <term><![CDATA[crosstalk]]></term>
##       <term><![CDATA[distortion]]></term>
##       <term><![CDATA[radiofrequency power amplifiers]]></term>
##       <term><![CDATA[transmitters]]></term>
##     </controlledterms>
##     <thesaurusterms>
##       <term><![CDATA[Crosstalk]]></term>
##       <term><![CDATA[Kernel]]></term>
##       <term><![CDATA[MIMO]]></term>
##       <term><![CDATA[Mathematical model]]></term>
##       <term><![CDATA[Polynomials]]></term>
##       <term><![CDATA[Radio frequency]]></term>
##       <term><![CDATA[Transmitters]]></term>
##     </thesaurusterms>
##     <pubtitle><![CDATA[Microwave Theory and Techniques, IEEE Transactions on]]></pubtitle>
##     <punumber><![CDATA[22]]></punumber>
##     <pubtype><![CDATA[Journals & Magazines]]></pubtype>
##     <publisher><![CDATA[IEEE]]></publisher>
##     <volume><![CDATA[62]]></volume>
##     <issue><![CDATA[4]]></issue>
##     <part><![CDATA[1]]></part>
##     <py><![CDATA[2014]]></py>
##     <spage><![CDATA[810]]></spage>
##     <epage><![CDATA[823]]></epage>
##     <abstract><![CDATA[This paper proposes three novel models for behavioral modeling and digital pre-distortion (DPD) of nonlinear 2 &#x00D7; 2 multiple-input multiple-output (MIMO) transmitters in the presence of crosstalk. The proposed models are extensions of the single-input single-output generalized memory polynomial model. Three types of crosstalk effects were studied and characterized as linear, nonlinear, and nonlinear &amp; linear crosstalk. A comparative study was performed with previously published models for the linearization of crosstalk in a nonlinear 2 &#x00D7; 2 MIMO transmitter. The experiments indicate that, depending on the type of crosstalk, the selection of the correct model in the transmitter is necessary for behavioral modeling and sufficient DPD performance. The effects of coherent and partially noncoherent signal generation on the performance of DPD were also studied. For crosstalk levels of -30 dB, the difference in the normalized mean square error and adjacent channel power ratio was found to be 3-4 dB between coherent and partially noncoherent signal generation.]]></abstract>
##     <issn><![CDATA[0018-9480]]></issn>
##     <htmlFlag><![CDATA[1]]></htmlFlag>
##     <arnumber><![CDATA[6774462]]></arnumber>
##     <doi><![CDATA[10.1109/TMTT.2014.2309932]]></doi>
##     <publicationId><![CDATA[6774462]]></publicationId>
##     <mdurl><![CDATA[http://ieeexplore.ieee.org/xpl/articleDetails.jsp?tp=&arnumber=6774462&contentType=Journals+%26+Magazines]]></mdurl>
##     <pdf><![CDATA[http://ieeexplore.ieee.org/stamp/stamp.jsp?arnumber=6774462]]></pdf>
##   </document>
## </root>
\end{verbatim}
\end{kframe}
\end{knitrout}

The following figure represents the top 5 keywords based on academic journals published in 2014.
The data is retrieved by IEEE API, and the search query limits academic journals that are open-access, and published in 2014.
The top five keywords are: 33

\begin{knitrout}
\definecolor{shadecolor}{rgb}{0.969, 0.969, 0.969}\color{fgcolor}\begin{kframe}
\begin{alltt}
\hlstd{df} \hlkwb{<-} \hlkwd{data.frame}\hlstd{(sortedTable)}
\hlkwd{colnames}\hlstd{(df)} \hlkwb{<-} \hlkwd{c}\hlstd{(}\hlstr{"Frequencies"}\hlstd{)}

\hlstd{c1}\hlkwb{<-}\hlkwd{c}\hlstd{(}\hlstr{"1."}\hlstd{,}\hlstr{"2."}\hlstd{,}\hlstr{"3."}\hlstd{,}\hlstr{"4."}\hlstd{,}\hlstr{"5."}\hlstd{)}

\hlstd{cols}\hlkwb{<-}\hlkwd{paste}\hlstd{(c1,} \hlkwd{row.names}\hlstd{(df),} \hlkwc{sep}\hlstd{=}\hlstr{" "}\hlstd{)}

\hlstd{g} \hlkwb{<-} \hlkwd{ggplot}\hlstd{(df,} \hlkwd{aes}\hlstd{(}\hlkwc{x}\hlstd{=cols,} \hlkwc{y}\hlstd{=Frequencies))}
\hlstd{g} \hlopt{+} \hlkwd{xlab}\hlstd{(}\hlstr{"Top 5 Keywords"}\hlstd{)}  \hlopt{+} \hlkwd{geom_bar}\hlstd{(}\hlkwc{stat}\hlstd{=}\hlstr{"identity"}\hlstd{)}
\end{alltt}
\end{kframe}
\includegraphics[width=\maxwidth]{figure/showGraph-1} 

\end{knitrout}




\end{document}
